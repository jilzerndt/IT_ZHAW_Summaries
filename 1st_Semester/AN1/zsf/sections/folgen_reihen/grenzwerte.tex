

\begin{definition}{$\varepsilon$-Definition}
    \\Folge $\sequence$ heisst \emph{konvergent}, falls es $l \in \R$ gibt, sodass \vspace{1mm}
    
    $\forall \varepsilon > 0$ die Menge $\{n \in \N^* : a_n \notin~] l - \varepsilon, l + \varepsilon[\,\}$ endlich ist. ($\N^*$: $\N$/$0$.) \vspace{1mm} \\ 
    Einfach gesagt: das heisst, dass $|a_n - a| < \epsilon$ ab einem gewissen n für alle $\epsilon$ gilt.
   
\end{definition}
Bem: $l$ bezeichnet den Grenzwert $\lim_{n \to \infty} a_n$
\begin{definition}{Formelle Grenzwert Definition}
    Folgende Aussagen sind äquivalent:
    \begin{enumerate}
        \item $\sequence$ konvergiert gegen $l = \lim_{n \to \infty} a_n$
        \item $\forall \varepsilon > 0~\exists N \geq 1$, sodass $|a_n -l | < \varepsilon \quad \forall n \geq N$.
    \end{enumerate}
\end{definition}
\begin{lemma}{Einzigartigkeit Grenzwert}
     Es gibt max. ein $l \in \R$ für $a_n$ mit dieser Eigenschaft (max. 1 Grenzwert)
\end{lemma}
\begin{theorem}{Rechenregeln mit Folgen}
    \\Sei $\sequence,\sequence[b]$ konvergente Folgen mit$a = \lim_{n \to \infty} a_n$, $b = \lim_{n \to \infty} b_n$
    \begin{enumerate}
        \item $(a_n \pm b_n)_{n \geq 1}$ konvergent, $\lim_{n \to \infty} (a_n \pm b_n) = a \pm b$.
        \item $(a_n \cdot b_n)_{n \geq 1}$ konvergent, $\lim_{n \to \infty} (a_n \cdot b_n) = a \cdot b$.
        \item $(a_n \div  b_n)_{n \geq 1}$ konvergent, $\lim_{n \to \infty} (a_n \div b_n) = a \div b$.
        \\(solange $b_n \neq 0 ~ \forall n \geq 1$ und $b \neq 0$)
        \item Falls $\exists K \geq 1$ mit $a_n \leq b_n ~ \forall n \geq K$ folgt $a \leq b$.
    \end{enumerate}
\end{theorem}





