
\begin{definition}{Supremum und Infimum}
    $A \subseteq \R, ~A\neq \emptyset$
    \begin{enumerateroman}
        \item $A$ nach oben beschränkt. Dann gibt es eine kleinste obere Schranke von $A$: $c \coloneqq \sup A$. Das \emph{Supremum} von $A$.
        \item $A$ nach unten beschränkt. Dann gibt es eine kleinste untere Schranke von $A$: $c \coloneqq \inf A$. Das \emph{Infimum} von $A$.
    \end{enumerateroman}
    \tcblower 
    Vereinfacht formuliert: Für ein abgeschlossenes, halboffenes oder offenes Intervall $[a,b], [a,b), (a,b]$ oder $(a,b)$ gilt $inf = a$, $sup = b$ (solange $a, b \neq \infty$)
\end{definition}

\begin{concept}{Satz von Weierstrass}
    \begin{itemize}
        \item $\sequence$ monoton wachsend und nach oben beschränkt. $\Rightarrow$ $\sequence$ konvergiert mit $\lim_{n \to \infty} a_n = \sup \{a_n : n \geq 1\}$
        \item $\sequence$ monoton fallend und nach unten beschränkt. $\Rightarrow$ $\sequence$ konvergiert mit $\lim_{n \to \infty} a_n = \inf \{a_n : n \geq 1\}$
    \end{itemize}
\end{concept}

\begin{theorem}{Sandwich-Satz}
    Sei $\lim a_n = \alpha$ und $\lim c_n = \alpha$ und $a_n \leq b_n \leq c_n, \forall n \geq k$ dann gilt $\lim b_n = \alpha$\\
    Bmk: k steht hier für eine beliebige natürliche Zahl, ab der die Bedingung immer gilt. Also wie bei der Grenzwert-Definition mit dem <<Gürtel>> um den Grenzwert - das gilt ja auch erst ab einem gewissen Wert n.\\
    Bmk 2: Einfach gesagt heisst das, dass wenn wir den Grenzwert von zwei Folgen bereits kennen und dieser für beide gleich ist, und wir eine dritte Folge haben die <<zwischen>> die zwei bekannten Folgen passt (daher Sandwich-Satz), wissen wir dass auch die dritte Folge den gleichen Grenzwert wie die anderen zwei hat.
\end{theorem}    



