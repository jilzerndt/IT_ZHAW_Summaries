\noindent Der Grenzwert einer Reihe kann auch mit Partialbruchzerlegung berechnet werden.
\begin{example}
    Was ist der Grenzwert von $\sum_{n=1}^\infty \frac{2}{(n+1)(n+3)}$?
    \tcblower
    \begin{equation*}
        \frac{2}{(n+1)(n+3)} = \frac{a}{n+1} + \frac{b}{n+3} \Rightarrow \begin{cases}
            a + b = 0\\
            3a - b = 2
        \end{cases}
        \quad
        \begin{matrix}
            b = -a = -\frac{1}{2}\\
            a = \frac{1}{2}
        \end{matrix}
    \end{equation*}
    \begin{align*}
        \Rightarrow \sum_{n=1}^\infty \frac{2}{(n+1)(n+3)} &= \frac{1}{2}\sum_{n=1}^\infty\left(\frac{1}{n+1} - \frac{1}{n+3}\right)\\
        & = \frac{1}{2}\left(\frac{1}{2} - \cancel{\frac{1}{4}} + \frac{1}{3} - \cancel{\frac{1}{5}} + \cancel{\frac{1}{4}} - \cancel{\frac{1}{6}} + \ldots\right)\\
        &= \frac{1}{2}\left(\frac{1}{2} + \frac{1}{3}\right) = \frac{5}{2 \cdot 6} = \frac{5}{12}
    \end{align*}
\end{example}