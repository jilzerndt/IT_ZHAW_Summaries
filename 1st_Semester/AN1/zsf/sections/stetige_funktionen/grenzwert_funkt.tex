\noindent Für mehr wichtige/spezielle Grenzwerte, siehe Grenzwerte von Folgen.
\begin{highlight}{Wichtige Grenzwerte}
    \begin{center}
        \begin{minipage}{0.4\linewidth}
                \tcbsubtitle{Harmonische Folge:}
                $$\lim _{n \rightarrow \infty} \frac{1}{n}=0$$
                \tcbsubtitle{Geometrische Folge:}
                $$\lim _{n \rightarrow \infty} q^n=0 \quad(q<1)$$
        \end{minipage}
        \hfill\vline\hfill
        \begin{minipage}{0.5\linewidth}
            \tcbsubtitle{n-te Wurzel:}
            $$\lim _{n \rightarrow \infty} \sqrt[n]{a}=1$$
            \tcbsubtitle{Eulerzahl:}
            $$\lim _{n \rightarrow \infty}\left(1+\frac{1}{n}\right)^n=e$$
        \end{minipage}
    \end{center}
\end{highlight}

\begin{definition}{Konvergenz einer Funktion}\\
    Die Funktion $y = f(x)$ hat an der Stelle $x_0$ den Grenzwert $y_0$ falls:\\
    für jede Folge $\left(x_{n}\right)$ mit $\lim _{n \rightarrow \infty} x_{n}=x_{0}$ gilt $\lim _{n \rightarrow \infty} f\left(x_{n}\right)=y_{0}$\\
    Bmk: Die Stelle $x_{0}$ muss nicht im Definitionsbereich $D$ sein.
\end{definition}

\begin{definition}{Konvergenz/Divergenz}
    \begin{itemize}
  \item Konvergenz:
  Funktion mit Grenzwert $x \rightarrow \infty$
  \item Divergenz:
  Funktion ohne Grenzwert $x \rightarrow \infty$
  \item Bestimmte Divergenz:
  Funktion mit $\lim _{x \rightarrow \infty} f(x)= \pm \infty$
\end{itemize}
\end{definition}

\begin{definition}{Links- und Rechtsseitige Grenzwerte}\\
    Sei $f : D \to \R$ und $x_0 \in \R$. Wir nehmen an, $x_0$ ist ein Häufungspunkt.\\ Was vereinfacht heisst, dass die Funktion an dieser Stelle evtl. einen Sprung macht, da sich z.B. die Definition ändert. Beispiel:
    \begin{equation*}
            f(x) = \begin{cases}
                0 & x < 0\\
                1 & x \geq 0
            \end{cases}
        \end{equation*}
    Setze in diesem Beispiel $x_0 = 0$, und prüfe ob sich die Funktion von rechts und links dem selben Wert nähert bei $x_0$. (NEIN in diesem Bsp.)

    \emph{Formell:} Eine Funktion ist dann gleichmässig konvergent wenn für alle Werte der Funktion gilt

    $$\lim_{x \to x_0^+} f(x) = \lim_{x \to x_0^-} f(x)$$

    (Linksseitiger Grenzwert = Rechtsseitiger Grenzwert)
\end{definition}








