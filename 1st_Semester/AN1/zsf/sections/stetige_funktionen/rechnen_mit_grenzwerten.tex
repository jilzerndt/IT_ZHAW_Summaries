\begin{corollary}{Rechnen mit Grenzwerten von Funktionen}
    \begin{itemize}
        \item $\lim_{x \to x_0} (f + g)(x) = \lim_{x \to x_0} f(x) + \lim_{x \to x_0} g(x)$
        \item $\lim_{x \to x_0} (f \cdot g)(x) = \lim_{x \to x_0} f(x) \cdot \lim_{x \to x_0} g(x)$
        \item Sei $f \leq g$, so ist $\lim_{x \to x_0} f(x) \leq \lim_{x \to x_0} g(x)$
        \item Falls $g_1 \leq f \leq g_2$ und $\lim_{x \to x_0} g_1(x) = \lim_{x \to x_0} g_2(x)$, so existiert $\lim_{x \to x_0} f(x) = \lim_{x \to x_0} g_1(x)$
    \end{itemize}
\end{corollary}

\begin{concept}{l'Hospital Kettenregel Trick}
    Seien $f,g : ]a,b[ \to \R$ differenzierbar mit \\ $g'(x) \neq 0 \quad \forall x \in ]a,b[$. Falls $\lim_{x \to b^-} f(x) = 0$, $\lim_{x \to b^-} g(x) = 0$ und $\lambda \coloneqq \lim_{x \to b^-} \frac{f'(x)}{g'(x)}$ existiert, folgt
    \begin{iequation}
        \lim_{x \to b^-} \frac{f(x)}{g(x)} = \lim_{x \to b^-}\frac{f'(x)}{g'(x)}
    \end{iequation}
    \tcblower
    \emph{Nur für $\lim_{x \to 0} \frac{f(x)}{g(x)} = \frac{\mlqq 0 \mrqq}{\mlqq 0 \mrqq}$ oder $\frac{\mlqq \infty \mrqq}{\mlqq \infty \mrqq}$ erlaubt.}
\end{concept}

\raggedcolumns
\columnbreak

\subsubsection{Strategien und Rechentricks}
\begin{KR}{Erweitern mit}
    $\left(\frac{1}{n^k}\right)$\\
    $k=$ höchste Potenz
\end{KR}
\begin{example}
Beispiel:
    $$
    \begin{aligned}
    \lim _{n \rightarrow \infty} \frac{3 n^2+2 n+1}{5 n^2+4 n+2} \Longrightarrow & \textcolor{pink}{\frac{\frac{1}{n^2}}{\frac{1}{n^2}}} \cdot \frac{3 n^2+2 n+1}{5 n^2+4 n+2} =\frac{\frac{3 n^2}{n^2}+\frac{2 n}{n^2}+\frac{1}{n^2}}{\frac{5 n^2}{n^2}+\frac{4 n}{n^2}+\frac{2}{n^2}} \\
    &= \frac{3+\frac{2}{n}+\frac{1}{n^2}}{5+\frac{4}{n}+\frac{2}{n^2}}=\frac{3+0+0}{5+0+0}
    \end{aligned}
    $$
\end{example}

\begin{KR}{Erweitern mit}
    $\left(\frac{1}{a^k}\right)$\\
$k=$ höchste Potenz \\
$a=$ grösste Basis
\end{KR}
\begin{example}
Beispiel:
    $$
    \begin{aligned}
    \lim _{n \rightarrow \infty} \frac{3^{n+1}+2^n}{3^n+2} \Longrightarrow &  \textcolor{pink}{\frac{\frac{1}{3^n}}{\frac{1}{3^n}}} \cdot \frac{3 \cdot 3^n+2^n}{3^n+2}=\frac{\frac{3 \cdot 3^n}{3^n}+\frac{2^n}{3^n}}{\frac{3^n}{3^n}+\frac{2}{3^n}} \\
    & = \frac{3+\frac{2^n}{3^n}}{1+\frac{2}{3^n}}=\frac{3+0}{1+0}
    \end{aligned}
    $$
\end{example}

\begin{KR}{Erweitern mit}
    $\sqrt{a(n)}+\sqrt{b(n)}$
\end{KR}
\begin{example}
Beispiel:
    $$
    \begin{aligned}
    &\lim _{n \rightarrow \infty} \sqrt{n^{2}+n}-\sqrt{n^{2}-2 n}\\
    &\Longrightarrow \textcolor{pink}{\frac{\sqrt{n^{2}+n}+\sqrt{n^{2}-2 n}}{\sqrt{n^{2}+n}+\sqrt{n^{2}-2 n}}} \cdot \frac{\sqrt{n^{2}+n}-\sqrt{n^{2}-2 n}}{1} \\
    &=\frac{\left(\sqrt{n^{2}+n}\right)^{2}-\left(\sqrt{n^{2}-2 n}\right)^{2}}{\sqrt{n^{2}+n}-\sqrt{n^{2}-2 n}}=\frac{3 n}{\sqrt{n^{2}+n}-\sqrt{n^{2}-2 n}}\\
    &=\frac{\frac{1}{n}}{\frac{1}{n}} \cdot \frac{3 n}{\sqrt{n^{2}+n}-\sqrt{n^{2}-2 n}} \frac{\frac{3 n}{n}}{\sqrt{\frac{1}{n^{2}} \cdot\left(n^{2}+n\right)}-\sqrt{\frac{1}{n^{2}} \cdot\left(n^{2}-2 n\right)}}\\
    &=\frac{3}{\sqrt{\frac{n^{2}+n}{n^{2}}}-\sqrt{\frac{n^{2}-n}{n^{2}}}}=\frac{3}{\sqrt{1+\frac{1}{n}}-\sqrt{1+\frac{1}{n}}}=\frac{3}{\sqrt{1}+\sqrt{1}}=\frac{3}{2}
    \end{aligned}
    $$
\end{example}

\begin{KR}{Erweitern zu}
    $\lim _{n \rightarrow \infty}\left(1+\frac{1}{5 n}\right)^{n}$
    $$
      \lim _{x \rightarrow \infty}\left(\left(\left(1+\frac{1}{x}\right)^{x}\right)^{a}\right)=e^{a}
    $$
\end{KR}
\begin{example}
Beispiel:
    $$
    \begin{aligned}
    &\lim _{x \rightarrow \infty}\left(1+\frac{2}{3 n}\right)^{4 n}
    =\left(\left(1+\frac{1}{\frac{3 n}{2}}\right)^{\frac{3 n}{2}}\right)^{a}=e^{a}=e^{\frac{8}{3}}\\
    &\text{Wir rechnen also:}\\
    & 4 n=\frac{3 n}{2} \cdot a \text{ und } a=\frac{4 n}{\frac{3 n}{2}}=\frac{8}{3}
    \end{aligned}
    $$
\end{example}



   
    


