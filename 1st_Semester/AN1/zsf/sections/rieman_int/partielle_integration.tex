
\begin{concept}{Partielle Integration}\\
     Seien $a < b$ reelle Zahlen und\\$f,g:[a,b] \to \R$ stetig differenzierbar. Dann gilt
    \begin{equation*}
        \int_a^b f(x) g'(x) \dif x = f(b) g(b) - f(a) g(a) - \int_a^b f'(x)g(x) \dif x
    \end{equation*}
    bzw. für unbestimmte Integrale

    $$
    \int\left(f(x) \cdot g^{\prime}(x)\right) d x=f(x) \cdot g(x)-\int\left(f^{\prime}(x) \cdot g(x)\right) d x
    $$
\end{concept}
\begin{remark}
    $\uparrow 1$ falls arc- oder log-Funktion vorkommt, $x^{n}, \frac{1}{1-x^{2}}, \frac{1}{1+x^{2}}$\\

    $\downarrow x^{n}, \arcsin (x), \arccos (x), \arctan (x)$,
\end{remark}
\begin{KR}{Prioritäten}\\
    Für die partielle Integration $f(x)$ nach folgender Priorität auswählen:\\
    \begin{array}{lll}
        1. \log_e, \log_a  &  3. x^2, 5x^3 & 5. e^x, 5a^x\\
        2. \arcsin, \arccos & 4. \sin, \cos, \tan 
    \end{array}
\end{KR}

