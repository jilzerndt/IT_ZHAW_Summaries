
\begin{corollary}{Nützliche Regeln für Partialbruchzerlegung}\\
	Sei $I \subseteq \R$ ein Intervall und $f: I \to \R$ stetig.
	\begin{enumerate}
		\item Seien $a,b,c \in \R$, sodass das abgeschlossene Intervall mit den Endpunkten $a+c$, $b+c$ in $I$ enthalten ist.
			Dann gilt
			\begin{equation*}
				\int_{a+c}^{b+c} f(x) \dif x = \int_{a}^{b} f(t+c) \dif t
			\end{equation*}
		\item Seien $a,b,c \in \R$ mit $c \neq 0$, sodass das abgeschlossene Intervall mit Endpunkten $ac$, $bc$ in $I$ enthalten ist.
			Dann gilt
			\begin{equation*}
				\int_{a}^{b} f(ct) \dif t = \frac{1}{c} \int_{ac}^{bc} f(x) \dif x
			\end{equation*}
	\end{enumerate}
\end{corollary}
\emph{Symmetrie ungerader Funktionen beachten}.
% Falls Funktion ungerade ist und symmetrische Grenzen hat
$\int_{-\frac{\pi}{2}}^{\frac{\pi}{2}} \underbrace{(\sin x)^7 \cos x}_{\text{ungerade}} \dif x = 0$


