
\begin{KR}{Strategie}
    \\\emph{Bruchform:}
    \begin{enumerate}
        \item Vereinfache, so dass ein einfacher Nenner entsteht
        \item Partialbruchzerlegung
        \item $\frac{u'}{2\sqrt{u}}$ oder $\frac{u'}{u}$ erkennen $\Rightarrow \sqrt{u}$ oder $log|u|$
    \end{enumerate}
    \emph{Produktform:}
    \begin{enumerate}
        \item Partielle Integration anwenden (evtl. mehrmals)
        \item Kettenregel verwenden
    \end{enumerate}
    \emph{Potenzen:}\\
        $\int_{a}^{b} f(x)^{c} d x$ umformen in $\int_{a}^{b}\left(f(x)^{c} \cdot 1\right) d x$ oder $\int_{a}^{b}\left(f(x)^{c-1} \cdot f(x)\right) d x$ um dann partielle Integration anzuwenden\\
    \emph{Exponentenform:}\\
        $e / \log$ Trick verwenden, wenn Variabel im Exponenten ist.\\
    \emph{Produkt mit $e, \sin , \cos$}\\
        Mehrmals partielle Integration anwenden, wobei sin, cos immer $g^{\prime}$ und immer $f$ ist.\\
    \emph{Summe im Integral:}\\
        Summe aus dem Integral herausziehen (dafür muss die Reihe gleichmässig konvergieren)
\end{KR}
