\begin{comment}
    Sei $R(x) = \frac{P(x)}{Q(x)}$ eine rationale Funktion. Dann lässt sich $\int R(x) \dif x$ als elementare Funktion darstellen.\\
Dh. als Funktion bestehend aus Polynomen, rationalen, exponentiellen, logarithmischen, trigonometrischen und/oder inversen trig. Funktionen.
Die Rechnung besteht aus 3 Schritten:
\begin{enumerate}
	\item Reduktion auf den Fall indem $\grad(P) < \grad(Q)$.
	\item Zerlegung von $Q$ in lineare und quadratische Faktoren, sowie Partialbruchzerlegung von $R$.
	\item Integration der Partialbrüche.
\end{enumerate}

\paragraph{Zu Schritt 1}
    Sei $R(x) = \frac{P(x)}{Q(x)}$. Falls $\grad(P) \geq \grad(Q)$ wenden wir den euklidischen Alg. an: $P(x) = S(x)Q(x) + \hat{P}(x)$, wobei $\grad \left( \hat{P} \right) < \grad{Q}$.
    Dann ist $\frac{P(x)}{Q(x)}  = S(x) + \frac{\hat{P}(x)}{Q(x)}$.
\paragraph{Zu Schritt 2}
	Seine $P,Q$ Polynome mit $\grad(P) < \grad(Q)$ und $Q$ mit Produktzerlegung $Q(x) = \prod_{j=1}^l \left( (x - \alpha_j)^2 + \beta_j^2 \right)^{m_j} \prod_{i=1}^k (x- \gamma_i)^{n_i}$.
    \\Dann gibt es $A_{ij}, B_{ij}, C_{ij}$ reelle Zahlen mit
\begin{equation*}
	\frac{P(x)}{Q(x)} = \sum^{l}_{i=1} \sum^{m_i}_{j=1} \frac{(A_{ij} + B_{ij}x)}{\left( (x-\alpha_i)^2 + \beta_i^2 \right)^j} + \sum^{k}_{i=1} \sum^{n_i}_{j=1} \frac{C_{ij}}{(x-\gamma_i)^j} 
\end{equation*}
\end{comment}

\begin{KR}{Stammfunktionen rationaler Funktionen}\\
    Die Stammfunktion einer Funktion $R(x)=\frac{P(x)}{Q(x)}$ bestehend aus rationalen Funktionen lässt sich als eine Funktion von Polynomen rationalen, exponentiallen, logarithmischen, trigonometrischen und inver trigonometrischen Funktionen darstellen.

Zuerst wollen wir das $\operatorname{grad}(P)<\operatorname{grad}(Q)$ gilt, falls die nicht der Fall ist uhren wir Polynomdivision aus. Danach bestimmen wir alle reellen und komenter Nun gilt:

$$
R(x)=\sum_{k=1}^{N} R_{k}(x)+\sum_{k=1}^{M} Z_{k}(x)
$$

Hier ist $N$ die Anzahl reeller Nullstellen und $M$ die Anzahl der komplexen Nullstellen. Es gilt:

$$
\begin{gathered}
R_{k}(x)=\frac{a_{k_{1}}}{\left(x-x_{k}\right)}+\frac{a_{k_{2}}}{\left(x-x_{k}\right)^{2}}+\ldots+\frac{a_{k_{n_{k}}}}{\left(x-x_{k}\right)^{n_{k}}} \\
Q_{k}(x)=\frac{a_{k_{1}}+b_{k_{1}} x}{\left(\left(x-\alpha_{k}\right)^{2}+\beta_{k}^{2}\right)}+\ldots+\frac{a_{k_{m_{k}}}+b_{k_{m_{k}}} x}{\left(\left(x-\alpha_{k}\right)^{2}+\beta_{k}^{2}\right)^{m_{k}}}
\end{gathered}
$$

Somit können wir die Funktion mit Partialbruchzerlegung in einzelnen Brüchen darstellen, welche dann leichter zu integrieren sind. Wenn wi eine reelle Nullstelle haben so gil:

$$
\int \frac{1}{\left(x-\gamma_{i}\right)^{n}}=\left\{\begin{array}{l}
\ln \left(x-\gamma_{i}\right) \quad \text { für } n=1 \\
\frac{-1}{(n-1)\left(x \gamma_{i}\right)^{n-1}}
\end{array}\right.
$$

Für komplexe Nullstellen gilt:

$$
\begin{gathered}
\frac{A+B x}{\left((x-\alpha)^{2}+\beta^{2}\right)^{j}}=\frac{B(x-\alpha)}{\left((x-\alpha)^{2}+\beta^{2}\right)^{j}}+\frac{A+B \alpha}{\left((x-\alpha)^{2}+\beta^{2}\right)^{j}} \\
\int \frac{B(x-\alpha)}{\left((x-\alpha)^{2}+\beta^{2}\right)^{j}}=\left\{\begin{array}{l}
\frac{B}{2} \ln \left((x-\alpha)^{2}+\beta^{2}\right)^{j} \quad \text { für } j=1 \\
\frac{B}{(2(1-j))\left((x-\alpha)^{2}+\beta^{2}\right)^{j-1}}
\end{array}\right.
\end{gathered}
$$

Für den letzen Term brauchen wir die Substitution $(x-\alpha)=\beta t$

$$
\int \frac{A+B \alpha}{\left((x-\alpha)^{2}+\beta^{2}\right)^{j}}=\frac{A+B \alpha}{\beta^{2 j-1}} \cdot \int \frac{1}{\left(t^{2}+1\right)^{j}} d t
$$
\end{KR}

\begin{example}
    Bsp.

$$
\int \frac{x^{2}-x+2}{x^{3}-x^{2}+x-1}
$$

Wir finden die erste Nullstelle $(x-1)$ durch ausprobieren. Danach führen wir Polynomdivision (durch $x-1$ ) aus und erhalten damit die weite Nullstelle $\left(x^{2}+1\right)$, Da $x^{2}+1$ eine komplexe Nullstelle ist, nehme wir dafür $A+B x$

$$
\frac{A+B x}{x^{2}+1}+\frac{C}{x-1}=\frac{x^{2}-x+2}{\left(x^{2}+1\right)(x-1)}
$$

$\Longrightarrow x^{2}-x+2=(A+C) \cdot x^{2}+(B-A) x+(C-B) \cdot 1$

$\Longrightarrow B=0, A=-1, C=1$

$\Longrightarrow \int \frac{1}{x-1}+\frac{-1}{x^{2}+1}=\ln (x-1)-\arctan (x)+C$
\end{example}



