
\begin{KR}{Polynomdivision}\\
    $\frac{P(x)}{q(x)} = S(x) + \frac{r(x)}{q(x)}$ \qquad $P,q,S,r$ Polynome\\
\emph{!} Vorzeichen von Nullstellen umdrehen.
\tcblower
$$
\begin{array}{cc}
\left(x^3-2 x^2-5 x-6\right):(x-1)=x^2-x-6 & \mid x^3: x=x^2 \\
-\left(x^3-x^2\right) & \mid-x^2: x=-x \\
-x^2-5 x & \\
-\left(x^2-x\right) & \mid-6 x: x=-6 \\
\hline-(-6 x+6) &
\end{array}
$$

Eine Polynomfunktion vom Grad $n$ hat höchstens $n$ reelle Nullstellen.
$$
f(x)=a_n \cdot\left(x-x_1\right) \cdot\left(x-x_2\right) \cdot \ldots \cdot\left(x-x_n\right)
$$
\end{KR}


