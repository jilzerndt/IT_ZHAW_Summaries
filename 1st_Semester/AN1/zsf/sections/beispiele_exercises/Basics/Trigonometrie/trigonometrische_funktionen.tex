\begin{example}
	Sei $x$ und $\theta$ reelle Zahlen mit $x + \frac{1}{x} = 2 \cos(\theta)$.
	Zu zeigen: Für jede natürliche Zahl $n \geq 1$ gilt $x^n + \frac{1}{x^n} = 2 \cos (n \theta)$
	\tcblower
	\emph{Induktionsanfang}\\
    $n=2:~x^2 + \frac{1}{x^2} = (x + \frac{1}{x})^2 -2 = 4 \cos^2 (\theta) -2\\
					 = 2 ( 2 \cos^2 (\theta) -1) = 2 ( \cos^2 (\theta) - ( 1 - \cos^2 (\theta)))\\
					 = 2 ( \cos^2 (\theta) - \sin^2 (\theta)) = 2 \cos (2 \theta)$\\
	\emph{Induktionshypothese:}
	gilt für $n=k$ und $n=k+1$\\
	\emph{Induktionsschritt}\\
        $x^{k+2} + \frac{1}{x^{k+2}} = ( x^{k+1} + \frac{1}{x^{k+1}})( x + \frac{1}{x}) -( x^{k} + \frac{1}{x^{k}})\\
		= 4 \cos( (n+1) \theta) \cos(\theta) - 2 \cos (n \theta)\\ 
        (*)= 4 \cos( (n+1) \theta) \cos(\theta) - 2 ( \cos( (n+1) \theta) \cos(-\theta) - \sin( (n+1) \theta) \sin(-\theta))\\
		= 2 \cos( (n+1) \theta) \cos(\theta) - 2 \sin( (n+1) \theta) \sin (\theta)\\
		= 2 \cos( (n+2) \theta)$\\
	Der Schritt (*) lässt sich folgendermassen berechnen:\\
    $\cos(n \theta) = \cos (n\theta + \theta -\theta) = \cos(n \theta + \theta) \cos(-\theta) - \sin(n \theta + \theta ) \sin ( -\theta)$
\end{example}
