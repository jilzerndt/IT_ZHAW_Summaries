\begin{example}
    Bestimme die Menge aller $x\in \R$, sodass $\sum_{n=1}^\infty \frac{4^n}{n+4}x^n$ konvergiert.
    \tcblower
    Der Konvergenzradius ist
    \begin{equation*}
        \rho = \frac{1}{\limsup\limits_{k \to \infty} \sqrt[k]{|c_k|} = 0} = \lim_{n \to \infty} \frac{\sqrt[n]{n+4}}{4} = \frac{1}{4}
    \end{equation*}
    Also Konvergenz in $\left] -\frac{1}{4}, \frac{1}{4}\right[$, Divergenz: $|x| > \frac{1}{4}$\\
    Randpunkte: $x \pm \frac{1}{4}$ einzeln untersuchen:\\
    $x = \frac{1}{4} \Rightarrow \sum_n a_n x^n = \sum_n \frac{1}{n+4} \Rightarrow$ divergent, da harmonische Reihe.
    $x = - \frac{1}{4} \Rightarrow \sum_n a_n x^n = \sum_n (-1)^n \frac{1}{n+4} \Rightarrow$ konvergent, da alternierende harmonische Reihe.
\end{example}