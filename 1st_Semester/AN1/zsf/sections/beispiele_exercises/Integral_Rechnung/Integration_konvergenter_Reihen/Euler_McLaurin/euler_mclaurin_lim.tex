\begin{example}
	Zeige mithilfe der Euler-McLaurin Summenformel, dass $\lim_{n \to \infty}\left[ \left( 1 + \frac{1}{2} + \ldots + \frac{1}{n} \right) - \ln(n) \right]$ existiert.\\
	Definiere $f(x) = \frac{1}{1+x}$ und wende an:\\
    $\frac{1}{2} + \frac{1}{3} + \ldots + \frac{1}{n}  = \sum^{n}_{i=1} f(i)\\
		= \int_{0}^{n} f(x) \dif x + \frac{1}{2} \left( f(n) - f(0) \right) + \int_{0}^{n} \tilde{B}_1(x) \cdot f'(x) \dif x \\
		= \ln(1+x) + \frac{1}{2} \left( \frac{1}{n+1} -1 \right) + \underbrace{\int_{0}^{n} \tilde{B}_1 (x) \left( - \frac{1}{(1+x)^2}  \right) \dif x}_{a_n \coloneqq}$\\
	$a_n$ ist eine Cauchy-Folge in $\R$ und mit S2.20 ergibt sich:\\
    $\left( 1 + \frac{1}{2} + \ldots + \frac{1}{n} \right) - \ln(n) = 1 + \ln(n+1) - \ln(n) + \frac{1}{2}  \left( \frac{1}{n+1} -1 \right)\\ 
									        \qquad + a_n\\
									    = \ln \left( 1 + \frac{1}{n} \right) + \frac{1}{2(n+1)} + \frac{1}{2} + a_n$
	Die rechte (und somit die linke) Seite konvergiert gegen $\ln (1) + 0 + a + \frac{1}{2} = a + \frac{1}{2}$
\end{example}