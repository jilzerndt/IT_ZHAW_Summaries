\begin{example}
    Zeige, dass die rekursiv definierte Folge $(a_n)_{n \in N}$ mit $a_1 = 1$, $a_{n+1} = \sqrt{1 + a_n}$, $n\geq 1$, konvergent ist. Bestimme den Grenzwert.
    \tcblower
    \textbf{Behaupt. 1:} $a_n \leq 2$, $\forall n \in \N$\\
    Verankerung: $n = 1: a_1 = 1 \leq 2$\\
    IS: $n \to n+1$ nach IA:\\
    $a_n \leq 2 \Rightarrow a_n + 1 = \sqrt{1 + a_n} \leq \sqrt{1 + 2} < 2$\\
    \textbf{Behaupt. 2:} $(a_n)$ ist monoton wachsend.\\
    Verankerung: $a_1 = 1 \leq \sqrt{1 + 1} = a_2$\\
    IS: $n \to n+1$ nach IA:\\
    $a_{n+1} = \sqrt{1 + a_n} \leq \sqrt{1 + a_{n+1}} = a_{n+2}$\\
    Da jede nach oben beschränkte, monoton wachsende Folge, konvergent ist, konvergiert $a_n \to a \in \R$.\\
    Für $a$ muss Rek.Vorschrift $a = \sqrt{1 + a}$ erhalten bleiben. Somit:\\
    $a^2 = 1 + 1 \Leftrightarrow a = \frac{1 \pm \sqrt{5}}{2}$\\
    Da $a_n \geq 0 ~\forall n \in \N$ muss $a \geq 0$ $\Rightarrow$ $a = \frac{1 + \sqrt{5}}{2}$
\end{example}