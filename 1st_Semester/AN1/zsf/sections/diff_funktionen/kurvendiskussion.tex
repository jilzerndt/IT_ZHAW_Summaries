\begin{KR}{Vorgehen Kurvendiskussion allgemein}
Kurvendiskussion für eine Funktion $y=f(x)$

\begin{itemize}
  \item Definitionsbereich
  \item Symmetrie
\end{itemize}

Gerade / Ungerade

\begin{itemize}
  \item Schnittpunkte mit den Koordinatenachsen
\end{itemize}

Nullstellen: Schnittpunkte mit Y-Achse

\begin{itemize}
  \item Verhalten für $x \rightarrow \infty$
  \item Relative Extrema, inkl. Typbestimmung
  \item Wendepunkte, insbesondere Sattelpunkte
\end{itemize}
\end{KR}

\begin{concept}{Graphen skizzieren}
$y=0.5 \cdot\left(x^{2}-5 x+4\right)$

Nullstellen

\begin{itemize}
  \item $0=0.5 \cdot(x-1) \cdot(x-4) \quad \rightarrow x_{1}=1, x_{2}=4$
\end{itemize}

Y-Achsenabschnitt

\begin{itemize}
  \item $y=0.5 \cdot x^{2}-2.5 \cdot x+2 \quad \rightarrow y_{0}=2$
\end{itemize}

\begin{center}
\includegraphics[width=\linewidth]{images/2024_01_20_7bfda6c084929ccc01ffg-09}
\end{center}

Relative Extrema $n=\operatorname{gerade}(n>1)$

\begin{itemize}
  \item Positiv $y=0.5 \cdot(x-3)^{n} \cdot(\ldots)+3 \rightarrow D=$ Hochpunkt
  \item Negativ $y=-0.5 \cdot(x-3)^{n} \cdot(\ldots)+1 \rightarrow E=$ Tiefpunkt
\end{itemize}

Wechselpunkte $n=$ ungerade $(n>2)$

\begin{itemize}
  \item $y=0.5 \cdot(x-3)^{n} \cdot(\ldots)+c \quad \rightarrow F=$ Wechselpunkt
\includegraphics[width=\linewidth]{images/2024_01_20_7bfda6c084929ccc01ffg-09(2)}
\end{itemize}


    
\end{concept}

Beispiel - Funktionen
\includegraphics[width=\linewidth]{images/2024_01_20_7bfda6c084929ccc01ffg-09(1)}