
\begin{definition}{n-fache Differenzierbarkeit}
	\begin{enumerate}
		\item Für $n \geq 2$ ist $f$ \emph{$n$-mal differenzierbar} in $D$ falls $f^{n-1}$ in $D$ differenzierbar ist. Dann ist $f^{(n)} \coloneqq \left(f^{(n-1)}\right)'$ und nennnt sich die $n$-te Ableitung von $f$.
		\item Die Funktion $f$ ist \emph{$n$-mal stetig differenzierbar} in $D$, falls sie $n$-mal differenzierbar ist und falls $f^{(n)}$ in $D$ stetig ist.
		\item Die Funktion $f$ ist in $D$ \emph{glatt}, falls sie $\forall n \geq 1$, $n$-mal differenzierbar ist. 
	\end{enumerate}
\end{definition}


\begin{itemize}
	\item $\exp$, $\sin$, $\cos$, $\sinh$, $\cosh$, $\tanh$ sind glatt auf $\R$
	\item Alle Polynome sind auf ganz $\R$ glatt
	\item $\ln : ]0, + \infty[ \to \R$ ist glatt
\end{itemize}

\begin{theorem}{Rechnen mit höheren Ableitungen}\\
	Sei $D \subseteq \R$, $n \geq 1$ und $f,g \to \R$ $n$-mal differenzierbar in $D$.
	\begin{enumerate}
		\item $f+g$ ist $n$-mal differenzierbar und $(f + g)^{(n)} = f^{(n)} + g^{(n)}$
		\item $f \cdot g$ ist $n$-mal differenzierbar und\\ $(f \cdot g)^{(n)} = \sum_{k=0}^n \binom{n}{k} f^{(k)} g^{(n-k)}$.
        \item $\frac{f}{g}$ ist n-mal differenzierbar falls $g(x) \neq 0, \forall x \in D$
        \item $(g \circ f)$ ist n-mal differenzierbar
	\end{enumerate}
\end{theorem}



