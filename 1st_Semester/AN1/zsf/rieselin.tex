\begin{KR}{Integrieren von Flächen}
Nullstellen bestimmen: $=>$ Fläche oberhalb $\mathrm{x}$ Achse, + Fläche evtl unterhalb x Achse...
\end{KR}

\begin{KR}{Transformation: Funktionen zeichnen}
    \begin{itemize}
    \item $y=f(x) B y=f(a x)$ : Streckung des Graphen um Faktor $\frac{1}{a}$ in x Richtung
  \item $y=f(x)$ ß $y=f(x+b)$ : Verschiebung des Graphen um $|b|$ in $\mathbf{x}$ Richtung
  \item $y=f(x)$ ß $y=c f(x)$ : Streckung des Graphen um Faktor $c$ in y Richtung
  \item $y=f(x)$ ß $y=f(x)+d$ : Verschiebung des Graphen um $d$ Einheiten in y Richtung
\end{itemize}
\end{KR}

\begin{concept}{Graphen von Polynomen}
    \begin{itemize}
  \item Hat so viele Nullstellen wie Grad des Polynoms (Fundamentalssatz der Algebra)
  \item bei Faktorisiertem angeben, noch Vorfaktor beachten $=>$ Falls Punkt gegeben kann dieser berechnet werden
  \item für jede Nullstelle:
  \item Einfach: Schneidend
  \item Doppelt: Berührend
  \item Dreifach: Wie $x^{3}$
  \item für Asymptoten: anlehnend
\end{itemize}
\end{concept}

\includegraphics[scale=0.4]{Analysis1/pics/2024_01_22_0c4ec55175f5e3a7f87dg-2(1).jpg}

\begin{concept}{Horner-Schema}
    Um nach Nullstellen aufzulösen: Nullstelle einsetzen und das Resultat (ohne Null am Ende) mit Grad -1 auffüllen = Lösung der Polynom Division...\\
    Im Beispiel: Rest würde als $+0 /$ ( $x-5$ ) angefügt werden
\end{concept}

\includegraphics[scale=0.3]{Analysis1/pics/2024_01_22_0c4ec55175f5e3a7f87dg-2(2).jpg}

\begin{KR}{Monotonie zeigen}
    \begin{itemize}
  \item $a_{n+1}-a_{n} \geq 0 \Rightarrow$ monoton wachsend (bzw. umgekehrt fallend)
  \item $\frac{a_{n+1}}{a_{n}} \geq 1$ und $a_{n} \geq 0$ dann monoton wachsend
\end{itemize}
\end{KR}

\begin{KR}{Grenzwert Berechnen Tricks}
\begin{itemize}
  \item " $\frac{\infty}{\infty} "$ Trick: Erweitern mit $\frac{1}{n^{k}}$ (k: grösster Exponent)
\end{itemize}

$\lim _{n \rightarrow \infty} \frac{2 n^{6}-n^{3}}{7 n^{6}+n^{5}-3} \cdot \frac{\frac{1}{n^{6}}}{\frac{1}{n^{6}}}=\frac{2}{7}$

\begin{itemize}
  \item " $\frac{\infty}{\infty} "$ Trick: Erweitern mit $\frac{1}{a^{k}}$ (a: grösste Basis, k: kleinster Exponent) $\lim _{n \rightarrow \infty} \frac{7^{n-1}+2^{n+1}}{7^{n}+5} \cdot \frac{\frac{1}{7^{n-1}}}{\frac{1}{7^{n-1}}}=\frac{1}{7}$
  \item " $\infty-\infty$ " Trick: Erweitern mit $\sqrt{\cdots}+\sqrt{\cdots}$
\end{itemize}

$\lim _{n \rightarrow \infty}\left(\sqrt{n^{2}+n}-\sqrt{n^{2}+1}\right)=1 / 2$

\begin{itemize}
  \item Kettenfunktionen: Trick Stetigkeit von $\mathrm{f}(\mathrm{x})$ ausnützen $\Rightarrow$ zuerst $\lim _{n \rightarrow \infty} a_{n}$ berechnen, danach $\mathrm{f}(\mathrm{x})$ (ohne nochmals lim) anwenden.
  \item e-like...: Trick: umformen zu $\left(\left(1+\frac{1}{x}\right)^{x}\right)^{a} \Rightarrow e^{a}$
\end{itemize}
\end{KR}

\begin{KR}{Kurvendiskussion}
    \begin{itemize}
  \item Monotonieverhalten (monoton wachsend / fallende Bereiche bestimmen)
  \item Krümmungsverhalten / Minima / Maxima etc überprüfen
  \item Kritische Punkte: hinreichende Bedingung nicht erfüllt
\end{itemize}

\begin{enumerate}
  \item Keine Aussage!

  \item Leite $\mathrm{f}(\mathrm{x})$ so lange ab bis die $n$-te Ableitung an Stelle $\neq 0$ ist. Dann gilt:

  \item Wenn $\mathrm{n}$ gerade ist, hat $\mathrm{f}(\mathrm{x})$ an der Stelle $\mathrm{x} 0$ ein relatives Extremum, und zwar ein relatives Maximum im Fall $f^{(n)}\left(x_{0}\right)<0$ und ein relatives Minimum im Fall $f^{(n)}\left(x_{0}\right)>0$

  \item Wenn $n$ ungerade ist, hat $f(x)$ an der Stelle $x_{0}$ einen Wendepunkt (und damit einen Sattelpunkt).

\end{enumerate}
\end{KR}

\begin{KR}{Differentialrechnung Tricks}
    \begin{itemize}
  \item Überall differenzierbar: Einheitliche Tangente (Ableitung 0 setzen) und dh: Grenzwerte müssen denselben Wert ergeben
  \item Zwei Funktionskurven berühren sich (aww): bedeutet dass sie an einer Stelle x0 den gleichen Funktionswert und die gleiche Ableitung haben
  \item Tangente bestimmen (Linearisierung): $f\left(x_{0}\right)+f^{\prime}\left(x_{0}\right)\left(x-x_{0}\right)$
\end{itemize}
\end{KR}

\begin{KR}{Extremwertaufgaben}
    \begin{enumerate}
  \item Gleichung aufstellen

  \item Ableiten

  \item Kandidaten für Min/Max finden

  \item Begründen durch 2. Ableitung / Graph

\end{enumerate}
\end{KR}

\begin{KR}{Ungleichungen}

No Flipping needed
    \begin{itemize}
  \item $\log _{a}()$ [mit a $>1$ 1] (oder andere monoton wachsende Funktion)
  \item Addition / Subtraktion
  \item Mult. mit Positiver Zahl
  \item Divi. durch Positive Zahl
\end{itemize}

Flipping needed

\begin{itemize}
  \item $\log _{a}()[$ mit $0<\mathrm{a}<1]$ (oder andere monoton fallende Funktion)
  \item Mult. mit Negativer Zahl
  \item Divi. durch Negative Zahl
\end{itemize}
\end{KR}