
Wird verwendet, um Summen (wie $\sum_{i=1}^n \frac{1}{i}$, $\sum_{i=1}^n \ln{i}$ oder $\sum_{i=1}^n i^\ell$) abzuschätzen.
\begin{theorem}[important]{Euler-McLaurin Summenformel}
	Sei $f : [0,n] \to \R$ $k$-mal stetig differenzierbar, $k \geq 1$.
	Dann gilt:
	\begin{enumerate}
		\item Für $k=1$:\\ $\sum^{n}_{i=1} f(i) = \int_{0}^{n} f(x) \dif x + \frac{1}{2} ( f(n) - f(0) ) + \int_{0}^{n} \tilde{B}_1 (x) f'(x) \dif x$
		\item Für $k \geq 2$:\\ $\sum^{n}_{i=1} f(i)\\ = \int_{0}^{n} f(x) \dif x + \frac{1}{2} ( f(n) - f(0) )\\
						     + \sum^{k}_{j=2} \frac{(-1)^j B_j}{j!} ( f^{(j-1)} (n) - f^{(j-1)} (0) ) + \tilde{R}_k$\\
			wobei $\tilde{R}_k = \frac{(-1)^{k-1}}{k!} \int_{0}^{n} \tilde{B}_k (x) f^{(k)} (x) \dif x$
	\end{enumerate}
\end{theorem}



\begin{highlight}{\impemph{!}}
	Für Potenzsummen der Form $1^l + 2^l + \ldots + n^l$ wobei $l \geq 1, l \in \N$ kann die Summenformel mit $f(x) = x^l$ und $k=l+1$ angewendet werden.
	Daraus folgt für alle $l \geq 1$:\\
    $1^l + 2^l + \ldots + n^l = \frac{1}{l+1} \sum^{l}_{j=0} (-1)^j \begin{pmatrix}l+1\\j\end{pmatrix} B_j n^{l+1-j}$

\end{highlight}






