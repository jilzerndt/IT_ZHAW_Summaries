
\begin{concept}{Substitution}\\
    Die Substitution ist die Umkehrung der Kettenregel. D.h. wir wollen Substitution vorallem verwenden, wenn wir innere Funktionen haben.
    \begin{equation*}
        \int_{g(b)}^{g(a)} f(x) \dif x = \int_{a}^b f(g(t)) g'(t) \dif t
    \end{equation*}
    bzw. für unbestimmte Integrale

    $$
    \int f(g(t)) \cdot g^{\prime}(t) d t=\left.\int f(x) d x\right|_{x=g(t)}
    $$
\end{concept}



\begin{formula}{Nützliche Substitutionen}
    \begin{itemize}
        \item $e^{x}, \sinh (x), \cosh (x)$, subst: $t=e^{a x}, d x=\frac{d t}{a t}$ Dann $\sinh =\cosh =\frac{t^{2}-1}{2 t}$
        \item $\log (x)$ subst: $t=\log (x), x=e^{t}, d x=e^{t} d t$
        \item für gerade $n: \cos ^{n}(x), \sin ^{n}(x), \tan (x)$ Sub: $t=\tan (x)$, $d y=\frac{1}{1+t^{2}} d t, \sin ^{2}(x)=\frac{t^{2}}{1+t^{2}}, \cos ^{2}(x)=\frac{1}{1+t^{2}}$
        \item für ungerade $n: \cos ^{n}(x), \sin ^{n}(x)$, Sub: $t=\tan (x / 2)$, $d y=\frac{2}{1+t^{2}} d t, \sin (x)=\frac{2 t}{1+t^{2}}, \cos (x)=\frac{1-t^{2}}{1+t^{2}}$
        \item $\int \sqrt{1-x^{2}} d x$ sub: $x=\sin (x)$ oder $\cos (x)$
        \item $\int \sqrt{1+x^{2}} d x$ sub: $x=\sinh (x)$
    \end{itemize}
\end{formula}

\begin{example}
    Bsp. $\int \frac{x}{\sqrt{9-x^{2}}} d x$ substitution mit $t=\sqrt{9-x^{2}}$.

    $$
    \Rightarrow x=\sqrt{9-t^{2}} \Rightarrow x^{\prime}=\frac{-2 t}{2 \sqrt{9-t^{2}}} \Rightarrow d x=\frac{-t \cdot d t}{\sqrt{9-t^{2}}}
    $$
    
    $\int-d t=-t$ rücksubstitution $\Rightarrow-\sqrt{9-x^{2}}$
\end{example}
