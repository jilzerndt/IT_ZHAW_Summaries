\begin{theorem}{Eigenschaften Integrierbarkeit}
    \begin{itemize}
        \item Sei $f : [a,b] \to \R$ \textit{stetig}. Dann ist $f$ integrierbar.
        \item Sei $f: [a,b] \to \R$ \textit{monoton}. Dann ist $f$ integrierbar.
        \item Seien $f,g : [a,b] \to \R$ beschränkt, integrierbar und $\lambda \in \R$.
                Dann sind $f+g$, $\lambda \cdot f$, $f \cdot g$, $|f|$, $\max(f,g)$ und $\frac{f}{g}$ \\
                (falls $|g(x)| \geq \beta > 0 \quad \forall x \in [a,b]$) integrierbar.
        \item Seien $P,Q$ Polynome und $ [a,b]$ ein Intervall, in dem $Q$ keine Nullstelle besitzt. Dann ist $ [a,b] \to \R$, $x \mapsto \frac{P(x)}{Q(x)}$ integrierbar.
        \item Sind $f, g$ in einer endlichen Menge an Punkten verschieden, sind entweder beide oder keine der Beiden integrierbar
        \item Sei $f: [a,b] \to \R$ stetig in dem kompakten Intervall $ [a,b]$. Dann ist $f$ in $ [a,b]$ gleichmässig stetig.
    \end{itemize}
\end{theorem}



