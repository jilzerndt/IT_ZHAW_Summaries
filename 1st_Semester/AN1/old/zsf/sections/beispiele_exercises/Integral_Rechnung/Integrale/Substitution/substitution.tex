\begin{example}
    Berechne: $\int_0^1 \frac{x^2 - 6x + 8}{x+1} \dif x$ mit Substitution.\\ \\
    Substituierte mit $u$: $ u = x + 1 \Longrightarrow x = u-1,~ \dif x = \tikz[remember picture, baseline=(substanchone.base)] \node[outer sep = 1pt, inner sep = 0pt] (substanchone) {$1$}; \cdot \dif u$\\
    Passe Grenzen an: 
    \begin{gather*}
        \int_0^1 \Longrightarrow \int_{u_1 = x_1 + 1 = 0 + 1}^{u_2 = x_2 + 1 = 1 + 1} \Longrightarrow \int_1^2
    \end{gather*}
    Berechne:
    \begin{align*}
        \int_0^1 \frac{x^2 - 6x + 8}{x+1} \dif x &\overset{\text{S}}{=} \int_1^2 \frac{(u-1)^2 - 6(u-1) + 15}{u} \cdot \tikz[remember picture, baseline=(substanchtwo.base)] \node[outer sep = 1pt, inner sep = 0pt] (substanchtwo) {$1$}; \cdot \dif u\\
        = \int_1^2 \frac{u^2 - 8u + 15}{u} \dif u
        \overset{\text{Lin.}}&{=} \frac{1}{2}\left[u^2\right]_1^2 - 8 [u]_1^2 + 15\left[\ln(|u|)\right]_1^2\\
        &=\frac{3}{2} - 8 + 15\ln(2) = \frac{30\ln(2) -13}{2}
    \end{align*}
\end{example}
\begin{tikzpicture}[remember picture, overlay]
    \draw[->, thick, red, opacity = 0.4] (substanchone) to[bend left] (substanchtwo);
\end{tikzpicture}