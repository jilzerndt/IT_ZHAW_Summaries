\begin{KR}{Trick Gerade/Ungerade}
    Für ungerade Funktionen gilt $\int_{-a}^{+a} f(x) \dif x = 0$.
    \begin{itemize}
        \item Summe/Komposition: ungerade und ungerade
        \item Produkt/Quotient: ungerade und gerade
        \item Ableitung: gerade $\longrightarrow$ ungerade
    \end{itemize}
    Bsp ungerade: $f(x)$ = $-x$, $x$, $sin(x)$, $tan(x)$, Polynomfunktionen mit ungeradem Exponent\\
    gerade: $1$, $x^2$, $cos(x)$, $sec(x)$, Polynomf. mit geradem Exponent\\
    beides: $f(x) = 0$
\end{KR}

\begin{example}
	Berechne $\int_{-\infty}^{+\infty} \sin(x)\exp\left(-x^2\right) \dif x$\\ \\
	Wir wissen für die Funktionen:\\
        $\sin(x)$ $\to$ ungerade und $\exp(-x^2)$ $\to$ gerade\\ 
	Das Produkt einer ungeraden und geraden Funktion ist eine ungerade Funktion.\\
	Für ungerade Funktionen gilt $\int_{-a}^{+a} f(x) \dif x = 0$. Daraus folgt: 
	$\int_{-\infty}^{+\infty} \sin(x)\exp\left(-x^2\right) \dif x = 0$
\end{example}