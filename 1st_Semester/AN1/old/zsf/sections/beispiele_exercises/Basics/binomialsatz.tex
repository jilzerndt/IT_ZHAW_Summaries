\begin{example}
	Seien $f(x) = e^{ax}, g(x) = e^{bx}$ mit $a,b \in \R$. Zeige mittels Satz 4.34.2 den Binomialsatz.
	\tcblower
	Nach mehrfacher Anwendung der Kettenregel gilt:\\ $(f \cdot g)'(0) = (x+y)^n \cdot e^{(x+y) \cdot 0} = (x +
	y)^n$
	%\begin{equation*}
	%	(f \cdot g)'(0) = (x+y)^n \cdot e^{(x+y) \cdot 0} = (x + y)^n
	%\end{equation*}
	und $f^{(k)} (0) = x^k \cdot e^{x \cdot 0} = x^k, \quad g^{(n-k)} (0) = y^{n-k} \cdot e^{y \cdot 0} = n^{n-k}$
%	\begin{equation*}
%		f^{(k)} (0) = x^k \cdot e^{x \cdot 0} = x^k, \quad g^{(n-k)} (0) ? y^{n-k} \cdot e^{y \cdot 0} = n^{n-k}
%	\end{equation*}
	Aus Satz 4.34.2 folgt:
	\begin{equation*}
		(x+y)^n = (f \cdot g)^{(n)} = \sum^{n}_{k=0} \begin{pmatrix}n\\k\end{pmatrix} f^{(k)} (0) \cdot g^{(n-k)}
		(0) = \sum^{n}_{k=0} \begin{pmatrix}n\\k\end{pmatrix} x^k \cdot y^{n-k}
	\end{equation*}
\end{example}