\subsubsection{Tipps Für Multiple-Choice}%
\label{sub:tipps_fur_multiple_choice}

\begin{itemize}
	\item \impemph{Richtig lesen!!!}
	\item (Gegen-)Beispiele suchen.
	\hrule
	\item $f: [0,1] \to \R$ monoton $\Longrightarrow$ $f$ beschränkt.
	\item Monotonieverhalten bei verketteten/miteinander verechneten Funktionen bleibt nicht umbedingt erhalten.
	\item Verkettete Funktionen:
		\begin{itemize}
			\item Falls äussere beschränkt $\Rightarrow$ Verkettung beschränkt.
			\item Verkettung stetiger Funktionen ist stetig.
			\item Verkettung (streng) konvexer Funktionen, muss nicht konvex sein.
		\end{itemize}
	\item Schauen ob Funktionsfolge $f_n$ \underline{gleichmässig} und nicht nur punktweise gegen $f$ konvergiert.
		$ \Rightarrow$ Gewisse Eigenschaft nur dann gültig.
	\item Sattelpunkt ist \underline{kein} Extrema.
	\item Nur weil $f$ stetig ist, muss $f'$ nicht stetig sein.
	\item Sei $f : \R \to \R$ eine ungerade Funktion. $\Rightarrow$ $f^{(i)} (0) = 0$ für $i$ gerade.
	\item $f$ differenzierbar $\Longrightarrow$ $f$ stetig $\Longrightarrow$ $f$ integrierbar. (K4.5, S5.15).
	\item $X,Y,Z \subset \R$, $f: X \to Y$ und $g: Y \to Z$, sodass $g \circ f : X \to Z$ bijektiv. 
		$\Rightarrow$ $f$ injektiv, $g$ surjektiv
	\item Sei $a,b \in \R$ mit $a < b$ mit $f: ]a,b[ \to \R$. Falls $f^2$ und $f^3$ in $]a,b[$
		differenzierbar und $f(x) \neq 0~\forall x \in ]a,b[$ $ \Rightarrow$ $f$ in $]a,b[$ differenzierbar.
	\item I.A. muss Grenzfunktion nicht differenzierbar sein, und wenn sie es ist muss ihre Ableitung nicht
		gleich dem Grenzwert der Ableitung sein.
	\item Für Verhalten von Funktionen auch Werte innerhalb des Intervalls betrachten.
	\item Absolute Konvergenz von $\sum^{\infty}_{k=1} a_k$ impliziert $\lim_{k \to \infty} a_k = 0$
	\item $(f_k)_{k \geq 1}$ Folge von differenzierbaren Funktionen auf $[0,1]$. $f$ Funktion auf $[0,1]$. $f_k$
		konvergiert gleichmässig zu $f$ für $k \to \infty$ $\longrightarrow$ Grenzfunktion stetig. $\longrightarrow$ $f$ integrierbar.
\end{itemize}
