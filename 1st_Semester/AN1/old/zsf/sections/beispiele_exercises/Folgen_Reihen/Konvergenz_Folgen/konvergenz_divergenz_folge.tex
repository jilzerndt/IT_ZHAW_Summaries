\begin{example}
	Sei $\sequence$ die rekursiv definierte Folge $a_1 = -\ln 2$ und\\$a_{n+1} = \exp (a_k) - 1, ~ n \geq 1$\\
	Überprüfen Sie, ob $\sequence$ konvergiert oder nicht. Wenn $\sequence$ konvergiert, zeigen Sie das und
	berechnen Sie einen Grenzwert. Wenn die Folge nicht konvergiert, begründen Sie die Divergenz.
	\tcblower
	Sei $f(x) = \exp(x) -x -1,~ x \in \R$, dann ist $f'(x) = \exp(x) -1$ und $f'(x) > 0$ falls $x > 0$ und $f'(x) <
	0$ falls $x < 0$. Da $f(0) = 0,~ f(x) > 0$ für $x \neq 0$. Als Konsequenz: $\exp(x) - (x+1) \geq 0$ für
	beliebige $x \in \R$.\\
	Zeige $a_n \leq a_{n+1} \leq 0$ für $n \in \N$ mit Induktion:\\
	Nehme an: $a_{n-1} \leq a_n \leq 0$, dann ist $a_{n+1} = \exp(a_n) - 1 \leq 0$ und $a_{n+1} = \exp (a_n) - 1
	\geq a_n$, weil $\exp(x) - (x+1) \geq 0$ für $x \in \R$\\
	Durch Weierstrass ist $\sequence$ konvergent, da $\sequence$ monoton wachsend und nach oben begrenzt.\\
	Sei $\alpha$ der Grenzwert von $\sequence$. Somit ist $\alpha = \exp(\alpha) - 1$ (Stetigkeit von $\exp$). Durch
	die Analyse von $f$ wissen wir, dass $f(0) = 0$ und $f(x) > 0$ für $x \neq 0$. Somit $\alpha = \exp(\alpha) - 1
	\Rightarrow \alpha = 0$.
\end{example}