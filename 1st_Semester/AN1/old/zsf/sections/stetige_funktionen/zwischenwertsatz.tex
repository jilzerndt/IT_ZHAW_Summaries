
\begin{lemma} {Zwischenwertsatz}
    Sei $I \subseteq \R$ ein Intervall, $f: I \to \R$ eine stetige Funktion und $a,b \in I$. Für jedes $c$ zwischen $f(a)$ und $f(b)$ gibt es ein $z$ zwischen $a$ und $b$ mit $f(z) = c$.
\end{lemma}
\noindent Der Zwischenwertsatz wird oftmals verwendet um zu zeigen, dass eine Funktion einen gewissen Wert annimmt.
\begin{KR}{Anwendung: Nullstellen}
    \begin{enumerate}
        \item Sei $f:[a,b] \to \R$ stetig. Falls $f(a) \cdot f(b) < 0$ dann $\exists c \in ]a,b[$ mit $f(c) = 0$ (Also eine Nullstelle)
        \item  Sei $P(x) = a_n x^n + a_{n-1}x^{n-1} + \ldots + a_0$ ein Polynom mit $a_n \neq 0$ und $n$ ungerade. Dann besitzt $P$ mindestens eine Nullstelle in $\R$
    \end{enumerate}
\end{KR}
Für $c > 0$ besitzt $Q(x) = x^2 + c$ keine Nullstelle in $\R$.
\begin{center}
    \begin{minipage}{0.44\linewidth}
        Falls Funktion wie abgebildet springt, lässt sich der Zwischenwertsatz nicht anwenden.
    \end{minipage}
    \hfill
    \begin{minipage}{0.25\linewidth}
        \begin{equation*}
            f(x) = \begin{cases}
                0 & x < 0\\
                1 & x \geq 0
            \end{cases}
        \end{equation*}
    \end{minipage}
    \hfill
    \begin{minipage}{0.25\linewidth}
        \begin{center}
            \begin{tikzpicture}
                \begin{axis}[
                    axis x line = middle,
                    axis y line = none,
                    ytick style={draw=none},
                    yticklabels={,,},
                    xtick style={draw=none},
                    xticklabels={,,},
                    width = 30mm,
                    ticklabel style = {font=\tiny},
                    ymin = -0.2,
                    ymax = 0.9,
                    xmin = -2,
                    xmax = 2
                ]
                    \filldraw[darkblue] (axis cs: 0,0.75) circle (0.5pt);
                    \addplot[thick, darkblue, ->] coordinates {(-2,0) (0,0)};
                    \addplot[thick, darkblue, ->] coordinates {(0,0.75) (2,0.75)};
                    \begin{pgfonlayer}{bg}
                        \fill[green!30, rounded corners = 1pt] (axis cs: -0.2, 0.85) rectangle (axis cs: 0.2, -0.1);
                    \end{pgfonlayer}
                \end{axis}
            \end{tikzpicture}
        \end{center}
    \end{minipage}
\end{center}