\begin{definition}{Kompaktes Intervall}
    Ein Intervall $I \subseteq \R$ ist kompakt, falls es von der Form $I=[a,b], a \leq b$ ist.
    
\end{definition}
Bem: Für $x_0$ in einem kompakten Intervall gibt es immer min. eine Folge $(a_n)_n≥1, an \in \I$, so dass $lim_{n \to \infty} a_n = x_0$.
\begin{lemma} {Min-Max Satz}
    Sei $\sequence[x]$ eine konvergente Folge in $\R$ mit Grenzwert $\lim_{n \to \infty} x_n \in \R$. Sei $a \leq b$.  Falls $\{x_n : n \geq 1\} \subseteq [a,b]$ folgt $\lim_{n \to \infty} x_n \in [a,b]$
\end{lemma}
\begin{theorem}{}
    Sei $f:I= [a,b] \to \R$ stetig auf einem kompakten Intervall $I$. Dann gibt es $u \in I$ und $v \in I$ mit 
    \begin{equation*}
        \tikz[baseline=(minmaxfu.base)] \node[pin={[font=\footnotesize]180:Mind 1 Min}, outer sep = 0pt, inner sep= 0pt](minmaxfu){$f(u)$}; \leq f(x) \leq \tikz[remember picture, baseline=(minmaxfv.base)] \node[outer sep = 0pt, inner sep= 0pt](minmaxfv){$f(v)$}; \qquad \forall x \in I
    \end{equation*}
    Insbesondere ist $f$ beschränkt. \tikz[remember picture, overlay] \draw[gray, very thin] (minmaxfv) --++ (320:0.5) node[font = \footnotesize, black, anchor = north] {Mind 1 Max};
\end{theorem}
\noindent Zusammen mit Zwischenwertsatz folgt: $f([a,b]) = [f(a), f(b)]$

\vspace{1mm}