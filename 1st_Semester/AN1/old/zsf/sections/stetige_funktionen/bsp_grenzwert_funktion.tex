\begin{example2}{Grenzwert einer Funktion}

    $f(x)=\frac{x^{2}-1}{x-1}$ an der Stelle $x_{0}=1$

    \begin{center}
    \begin{tabular}{|c|c|c|}
    \hline
    $\boldsymbol{n}$ & $\boldsymbol{x}_{\boldsymbol{n}}=\mathbf{1}-\frac{\mathbf{1}}{\boldsymbol{n}}$ & $\boldsymbol{f}\left(\boldsymbol{x}_{\boldsymbol{n}}\right)$ \\
    \hline
    $\mathbf{1}$ & 0 & 1 \\
    \hline
    $\mathbf{2}$ & 0.5 & 1.5 \\
    \hline
    $\mathbf{3}$ & 0.66 & 1.66 \\
    \hline
    $\mathbf{4}$ & 0.75 & 1.75 \\
    \hline
    $\mathbf{5}$ & 0.8 & 1.8 \\
    \hline
    $\mathbf{1 0}$ & 0.9 & 1.9 \\
    \hline
    $\mathbf{1 0 0}$ & 0.99 & 1.99 \\
    \hline
    $\mathbf{1 0 0 0}$ & 0.999 & 1.999 \\
    \hline
    $\boldsymbol{n} \rightarrow \infty$ & $x_{0}=1$ & 2 \\
    \hline
    \end{tabular}
    \end{center}
    
    $$
    \lim _{x \rightarrow 1} f(x)=2
    $$
\end{example2}