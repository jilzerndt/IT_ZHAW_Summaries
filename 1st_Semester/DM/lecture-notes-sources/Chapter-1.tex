1.2\chapter{Grundbegriffe und elementare Logik}


Am Anfang aller Logik steht\dots
\begin{quote}
Wenn ``Doken'' stets ``derig'' sind und wenn es ``Raken'' gibt die auch ``Doken'' sind, dann gibt es
derige Raken und alle underigen Raken sind keine Doken.
\end{quote}
 \dots die Erkenntnis, dass
gewisse Argumente unabhängig von deren Inhalt aber aufgrund ihrer Struktur als eindeutig
schlüssig/korrekt identifizierbar sind. Dieses Kapitel gibt Ihnen eine informelle Einführung in die Prädikatenlogik.

\section*{Beispiele für Anwendungen in der Informatik}
\begin{itemize}
\item Grundlage für die Entwicklung einer soliden ``Theorie der Informatik''.
\item Künstliche Intelligenz, Wissensrepräsentation, Expertensysteme.
\item Allgegenwärtig in der Programmierung (z.B. ``if \dots then \dots else\dots''-Befehle).
\item Formale Verifikation der Korrektheit von Programmen.
\end{itemize}



\section*{Lernziele}
Sie kennen die Konzepte von
\begin{itemize}
\item Aussagen und Prädikaten.
\item universeller und existenzieller Quantifikation.
\end{itemize}
Sie verstehen wie
\begin{itemize}
\item durch Implikation, Äquivalenz, Negation, Konjunktion und Disjunktion neue Aussagen und Prädikate aus bereits bestehenden gewonnen werden.
\item durch Quantifikation von Prädikaten neue Aussagen und Prädikate gewonnen werden.
\end{itemize}
Sie sind in der Lage
\begin{itemize}
\item natürlichsprachliche (mathematische) Aussagen in der Sprache der Prädikatenlogik zu formalisieren.
\item mittels Fallunterscheidung, Widerspruchsargumenten und Kontraposition einfache mathematische Tatsachen zu beweisen.
\end{itemize}
Sie bewerten
\begin{itemize}
\item einfache Beweise und Argumente bezüglich ihrer Korrektheit und Stringenz.
\end{itemize}

\section*{Literatur und Links}
Ergänzende Literatur:
\begin{itemize}
\item  \cite{diskreteStrukturen} Kapitel 1.2 bis 1.4.
\item \cite{hartmann} Kapitel 2.
\item \cite{haggarty} Kapitel 2.
\end{itemize}
Weiterführende Literatur:
\begin{itemize}
\item \cite{flum} ganzes Buch.
\end{itemize}
Nützliche Links:
\begin{itemize}
\item \url{http://de.wikipedia.org/wiki/Pr%C3%A4dikatenlogik_erster_Stufe}
\item \url{https://openlogicproject.org/}
\end{itemize}

\section{Aussagen, Prädikate und Quantoren}

Wir werden im folgenden Abschnitt auf pragmatische Art und Weise die grundlegenden Konzepte der
Logik und Mathematik
kennenlernen. Um nicht nur langweilige Beispiele machen zu können, werden wir in diesem Kapitel auf
gewisse mathematische Begriffe wie z.b. ``natürliche Zahlen'' $(0,1,2,\dots)$ oder
``Primzahlen'' $(2,3,5,7,11\dots)$  zurückgreifen, ohne diese vorher sauber eingeführt zu haben. Die Anschauung,
welche Sie von der Schule mitbringen, sollte aber ausreichen um die Beispiele zu
verstehen.

\begin{df}
 Unter einer \textit{Aussage} wollen wir ein ``sprachliches Gebilde'' oder Ausdruck verstehen, welchem ein Wahrheitswert ``wahr'' oder ``falsch'' zugeordnet werden kann.
\end{df}

\begin{rk}
  Obwohl nach Definition jede Aussage einen eindeutigen Wahrheitswert besitzt, bedeutet dies nicht, dass dieser bekannt sein muss. Der Satz ``es gibt unendlich viele Primzahlen'' war beispielsweise bereits eine Aussage, bevor man wusste, dass er wahr ist.
\end{rk}

\begin{bsp}
 Einige Beispiele für Aussagen mit ihren Wahrheitswerten:
\begin{enumerate}
  \item ``$3+4=106$'' (falsch)
  \item ``Jede natürliche Zahl ist entweder durch $2$ oder durch $3$ teilbar.'' (falsch)
  \item ``Es gibt unendlich viele natürliche Zahlen.'' (wahr)
\end{enumerate}
\end{bsp}

\begin{rk}
Wir sagen, dass eine Variable $x$ frei in einem Ausdruck $A$ vorkommt, falls $x$ weder für einen noch für eine Menge von konkreten Werten steht, sondern einen reinen ``Platzhalter'' darstellt. Beispiele in denen die Variable $x$ frei vorkommt sind: ``$x<3$'' oder ``$x$ ist ein Tisch''. Im Gegensatz dazu kommt $x$ in ``alle $x$, die durch $4$ teilbar sind, sind gerade'' nicht frei vor, weil in dieser Aussage die Gesamtheit (Menge) aller möglichen Belegungen von $x$ betrachtet wird. In einem Ausdruck können beliebig viele Variablen frei vorkommen und wir schreiben $A(x,y,z,\dots)$, um anzuzeigen, dass in einem Ausdruck $A$ die Variablen $x,y,z,\dots$ frei vorkommen.
\end{rk}

\begin{df}
Es sei $n$ eine natürliche Zahl. Ein Ausdruck, in dem $n$ viele Variablen frei vorkommen und der bei Belegung aller freien Variablen in eine Aussage übergeht, nennen wir ein \textit{$n$-stelliges Prädikat}.
\end{df}

\begin{rk}
  Aussagen sind $0$-stellige Prädikate.
\end{rk}

\begin{rk}
  Ist $A(x)$ ein Prädikat und ist $y$ ein mathematisches Objekt (z.B. $y=17$) so, dass $A(y)$ eine wahre Aussage ist, dann sagen wir, dass das Prädikat (manchmal auch die Eigenschaft) $A$ auf $y$ zutrifft. Das Prädikat $x>100$ trifft zum Beispiel auf die Zahl $232$ zu, weil $232>100$ eine wahre Aussage ist.
\end{rk}

\begin{bsp}
  Einige Beispiele\footnote{Die Zeichenfolge ``$:=$'' steht für ``ist definiert als'' oder ``ist per Definition gleich''.} für Prädikate:
  \begin{enumerate}
  \item $P(p):= $\,``$p$ ist eine Primzahl.''
  \item $T(x):= $\,``$x$ ist eine durch $21$ teilbare ganze Zahl.''
  \item $G(r):= $\,``$r>0$''
  \item $Q(x,y):= $\,``$x^2+14x-15=y$''
  \end{enumerate}
  Die Aussagen
  \begin{align*}
  &T(42)& 	    &P(7)&		&Q(2,17)&\\
  &T(357)&  &P(2)& 	&Q(1,0)&
  \end{align*}
  sind alle wahr. Deshalb können wir, entsprechend der vorhergehenden Bemerkung, z.B. ``$T$ trifft auf $42$'' zu oder auch ``$7$ hat die Eigenschaft $P$'' sagen.
\end{bsp}

\begin{bsp}
  Weitere Beispiele für Prädikate:
  \begin{itemize}
    \item $A(x,y) := x + y < 10$ ist ein zweistelliges Pädikat mit den freien Variablen $x$ und $y$.
    \item $B(x,y,z) := x + y < z$ ist ein dreistelliges Prädikat.
    \item Wenn wir die Variable $z$ in $B$ mit dem Wert $10$ belegen, dann erhalten wir das zweistellige Prädikat $B(x,y,10)$, welches gleichbedeutend mit dem Prädikat $A$ ist.
  \end{itemize}
\end{bsp}

\subsection*{Junktoren}
Aus gegebenen Aussagen lassen sich durch Verknüpfung neue komplexere Aussagen
gewinnen. Betrachten wir zum Beispiel die Aussagen
\[
A:=\text{``}78\text{ ist keine Primzahl}
\]
und
\[
 B:=``15\text{ ist keine Primzahl}'',
\]
so können wir eine neue Aussage, nennen wir sie $C$, betrachten. $C$ soll ausdrücken, dass sowohl $A$ als auch $B$ wahr ist, d.h.
\[
 C:=\text{``}78\text{ ist keine Primzahl \textbf{und} }15\text{ ist keine Primzahl}\text{''}
\]
oder etwas anders formuliert (aber mit gleichem Wahrheitswert)
\[
 C:=\text{``weder die }15\text{ noch die }78\text{ ist eine Primzahl}\text{''}.
\]

Wir werden nun einige abkürzende Schreibweisen einführen um bequem über solche zusammengesetzten Aussagen sprechen zu können.

\begin{df}
Es seien $A$ und $B$ beliebige Prädikate. Wir führen folgende abkürzende Schreibweisen ein:
\begin{itemize}
\item $\neg A$ (gesprochen: Nicht $A$) ist das Prädikat, welches (für jede Belegung) genau dann wahr ist, wenn $A$ falsch ist.
 \item $A\wedge B$ (gesprochen: $A$ und $B$)  ist das Prädikat, welches (für jede Belegung) genau dann wahr ist, wenn sowohl $A$ als auch $B$ wahr sind.
\item $A\vee B$ (gesprochen: $A$ oder $B$)  ist das Prädikat, welches (für jede Belegung) genau dann wahr ist, wenn $A$ wahr ist oder $B$ wahr ist (oder beide wahr sind).
\item $A\Rightarrow B$ (gesprochen: $A$ impliziert $B$) ist das Prädikat, welches (für jede Belegung) genau dann wahr ist, wenn $\neg A\vee B$ wahr ist.
\item $A\Leftrightarrow B$ (gesprochen: $A$ äquivalent $B$)  ist das Prädikat, welches (für jede Belegung) genau dann wahr ist, wenn $A\Rightarrow B$ und $B\Rightarrow A$ wahr sind.
\end{itemize}
Die Zeichen $\neg,\Rightarrow,\wedge$ und $\vee$ nennen wir \textit{Junktoren}.
\end{df}

\begin{rk}
Das Prädikat $A\Rightarrow B$ besagt, dass in jedem Fall in dem $A$ wahr ist auch $B$ wahr sein muss.
Die Äquivalenz zweier Prädikate besagt also, dass diese stets denselben Wahrheitswert haben. Umgangssprachlich wird oft vorausgesetzt, dass zwischen den Prädikaten $A$ und $B$ ein ``inhaltlicher Zusammenhang'' bestehen muss, damit $A\Rightarrow B$ gelten kann. Dies ist in der mathematischen Logik nicht der Fall. Die Aussagen
\begin{align*}
\textit{Es gibt Einhörner}\Rightarrow 8\textit{ ist eine Primzahl}
\end{align*}
und
\begin{align*}
\textit{Spinat ist grün}\Rightarrow 2\textit{ ist eine Primzahl}
\end{align*}
sind beispielsweise beide (mathematisch gesehen) wahr.
\end{rk}


\begin{bsp}
Gegeben sind die Aussagen $A$ und $B$:
\begin{enumerate}
\item[] $A$ :=``Alle Hasen haben lange Ohren.''
\item[] $B$ := ``Es gibt Hasen mit kurzen Beinen.''
\end{enumerate}
Es gilt:
\begin{enumerate}
 \item $\neg A$ entspricht ``Es gibt mindestens einen Hasen, der keine langen Ohren hat.''
\item $A\wedge \neg B$ entspricht ``Alle Hasen haben lange Ohren und keine kurzen Beine.''
\item $A\Rightarrow B$ entspricht ``Wenn alle Hasen lange Ohren haben, dann gibt es Hasen mit kurzen Beinen.''
\end{enumerate}
\end{bsp}

\begin{ueb}
Negieren Sie umgangssprachlich folgende Aussagen (so präzise wie möglich).
\begin{enumerate}
\item Alle Autos haben vier Räder.
\item Zwillinge haben stets die identische Haarfarbe.
\item Es gibt flugunfähige Vögel.
\item Alle Dinosaurier sind ausgestorben.
\end{enumerate}
\end{ueb}
\begin{lsg}~
\ifthenelse{\boolean{ml}}{
	\begin{enumerate}
	\item Es gibt ein Auto, das nicht vier Räder hat.
  \\ \textbf{Anmerkung:} Es kann auch mehrere solche Autos geben.
	\item Es gibt ein Zwillingspaar mit verschiedenen Haarfarben.
	\item Alle Vögel sind flugfähig.
	\item Mindestens ein Dinosaurier lebt noch.
	\end{enumerate}
}{
\answerspace{5cm}
}
\end{lsg}

Wir werden nun einige Umformungsregeln betrachten, die unterschiedlich zusammengesetzte
Aussagen, rein aufgrund ihrer logischen Struktur, als äquivalent deklarieren. Wir werden diese Regeln
als evident betrachten und sie ohne Beweis übernehmen. Diese Regeln werden es uns erlauben  mit Aussagen und Prädikaten zu ``rechnen''.

\begin{rk}[Junktorenregeln]
 Seien $A,B$ und $C$ beliebige Aussagen. Es gelten folgende Äquivalenzen
\begin{itemize}
 \item Regel der doppelten Negation:
\[
 \neg\neg A\Leftrightarrow A
\]
\item Kommutativität:
\begin{align*}
  A\wedge B\Leftrightarrow B\wedge A\\
  A\vee B\Leftrightarrow B\vee A
\end{align*}
\item Assoziativität:
\begin{align*}
 (A\wedge B)\wedge C\Leftrightarrow A\wedge (B\wedge C)\\
 (A\vee B)\vee C\Leftrightarrow A\vee (B\vee C)
\end{align*}

\item Distributivität:
\begin{align*}
 A\wedge (B\vee C)\Leftrightarrow (A\wedge B)\vee (A\wedge C)\\
 A\vee (B\wedge C)\Leftrightarrow (A\vee B)\wedge (A\vee C)
\end{align*}

\item Regeln von De Morgan:
\begin{align*}
 \neg(A\wedge B)\Leftrightarrow\neg A\vee\neg B\\
\neg(A\vee B)\Leftrightarrow \neg A\wedge\neg B
\end{align*}
\end{itemize}
\end{rk}


\begin{bsp}[Kontraposition]
Wir können die eben aufgestellten Rechenregeln dazu verwenden um wiederum neue Tatsachen abzuleiten. Unter anderem folgt daraus das sogenannte Prinzip der \textit{Kontraposition}. Dieses Prinzip besagt, dass $A\Rightarrow B$ äquivalent ist zu $\neg B\Rightarrow\neg A$. Wollen wir dies nun mit unseren Rechenregeln nachvollziehen, so beginnen wir mit $A\Rightarrow B$ und wenden nacheinander verschiedene Regeln an um schlussendlich $\neg B\Rightarrow \neg A$ zu erhalten:

\begin{align*}
                     &A\Rightarrow B\\
   \Leftrightarrow\, &\neg A\lor B                  &(\text{Definition von }A\Rightarrow B)\\
   \Leftrightarrow\, &B\lor \neg A                  &(\text{Kommutativität})\\
   \Leftrightarrow\, &\neg\neg B\lor\neg A          &(\text{Doppelte Negation})\\
   \Leftrightarrow\, &\neg B\Rightarrow \neg A      &(\text{Definition von }\neg B\Rightarrow\neg A)
\end{align*}
\end{bsp}

\subsection*{Quantoren}
Quantoren sind Symbole  anhand derer wir aus Prädikaten neue Prädikate oder Aussagen gewinnen
können. Wir betrachten das Beispiel des Prädikates
\[
 A(x):=\text{``}x\text{ ist eine Primzahl und } x\text{ ist ein Teiler von }24\text{''}
\]
und die Aussage
\[
 B:= ``\text{es gibt eine Primzahl welche ein Teiler von }24\text{ ist''}
\]
mit anderen Worten,
\[
 B:=``\text{es \textbf{existiert} ein }x\text{ mit }A(x)\text{''}.
\]
Wir sagen, dass $B$ aus $A(x)$ durch existenzielle Quantifizierung über $x$ entsteht.

Andererseits können wir aus dem Prädikat $A(x)$ aber auch die (offensichtlich falsche) Aussage
\[
 C:=\text{``alle Zahlen sind Primzahlen und ein Teiler von }24\text{''}
\]
konstruieren. Diese ist gleichbedeutend mit
\[
 C:=\text{''\textbf{alle} Zahlen }x\text{ erfüllen }A(x).
\]
Wir sagen, dass $C$ aus $A(x)$ durch universelle Quantifizierung entsteht\footnote{Obwohl in den Aussagen $B$ und
$C$ formal die Variable $x$ vorkommt, steht sie nicht als Platzhalter für ein einzusetzendes Objekt, sondern
``läuft'' über die Gesamtheit aller möglichen Objekte. Wir sagen, dass die Variable nicht frei sondern durch
einen Quantor gebunden ist.}.

\begin{df}
Es sei $M$ eine Menge. Ist $A(x)$ ein Prädikat, dann können wie folgt neue Prädikate geformt werden:
\begin{itemize}
\item $\forall x\,A(x)$ (gesprochen: Für alle $x$ gilt $A(x)$) trifft genau dann zu, wenn $A$ auf jedes (mathematische) Objekt zutrifft.
\item $\forall x\in M\,A(x)$ (gesprochen: Für alle $x$ aus $M$ gilt $A(x)$) trifft genau dann zu, wenn $A$ auf jedes Element aus $M$ zutrifft.
\item $\exists x\,A(x)$ (gesprochen: Es gibt ein $x$ mit $A(x)$) trifft genau dann zu, wenn es (mindestens) ein  Objekt gibt, auf welches $A$ zutrifft.
\item $\exists x\in M\,A(x)$ (gesprochen: Es gibt ein $x$ aus $M$ mit $A(x)$) trifft genau dann zu, wenn es (mindestens) ein Element aus $M$ gibt, auf welches $A$ zutrifft.
\end{itemize}
Die Symbole $\forall$ und $\exists$ heissen \textit{Allquantor} und \textit{Existenzquantor}.
\end{df}

\begin{rk}
In mathematischen Texten werden Prädikate von der Form $\exists x\,A(x)$ oft als ``es gibt \textbf{mindestens} ein $x$ mit $A(x)$'' ausgedrückt. Diese Ausdrucksform ist inhaltliche gleichbedeutend mit ``es gibt ein $x$ mit $A(x)$''. Auch in diesem Text werden wir beide sprechweisen synonym verwenden.
\end{rk}

\begin{rk}
  Einige geläufige Notationen und Abkürzungen im Zusammenhang mit Quantoren sind:
  \begin{itemize}
  \item $\forall x,y\,(\dots)$ als Abkürzung von $\forall x\,\forall y\, (\dots)$ und $\exists x,y\,(\dots)$ als Abkürzung für $\exists x\,\exists y\,(\dots)$. Entsprechende Abkürzungen gelten auch für drei oder mehr quantifizierte Variablen.
  \item $\exists ! x\,(\dots)$ für ``es gibt \textbf{genau} ein $x$ mit $\dots$''.
  \item $\nexists x\,(\dots)$ für $\neg\exists x\,(\dots)$.
  \item $\forall x<y\,(\dots)$ für $\forall x\,(x<y\Rightarrow \dots)$. Diese Notation wird auch für andere ähnliche Relationen wie $ >,\leq, \geq, \subseteq, $ usw. verwendet.
\end{itemize}
\end{rk}

\begin{rk}
Ein $n$-stelliges Prädikat wird durch Quantifizierung (einer freien Variable) zu einem neuen $n-1$ stelligen Prädikat.
\end{rk}


\begin{bsp}
 Einige quantifizierte Aussagen mit ihren Wahrheitswerten:
\begin{enumerate}
 \item Es sei $S$ die Menge aller Schweine und $R(x)$ das Prädikat ``$x$ ist rosa''. Es gilt
\[
 \text{``}\exists x\in S\; R(x)\text{''}\Leftrightarrow\text{``es gibt rosa Schweine''}.
\]
Diese Aussage ist offensichtlich wahr. Wenn wir nun die Allquantifizierung betrachten, so erhalten wir
\[
 \text{``}\forall x\in S\; R(x)\text{''}\Leftrightarrow\text{``alle Schweine sind rosa''}.
\]
Dies ist eine falsche Aussage, da etwa Wildschweine einerseits Elemente von $S$ sind aber andererseits $R$ nicht erfüllen, da sie nicht rosa sind.
\item Wir wollen nun die Aussage
\[
A:=``\,\text{alle Informatiker können programmieren''}
\]
mit Quantoren ausdrucken.
Wir definieren dazu zuerst das Prädikat
\[
B(x):=``x\text{ kann programmieren''}.
\]
Wir haben nun zwei mögliche Vorgehensweisen. Einerseits können wir die Menge $I$ aller Informatiker betrachten und kommen dann mittels der Aussage
\[
 \forall x\in I\;B(x)
\]
zum Ziel. Andererseits können wir auch $A$ umformulieren als ``alles was ein Informatiker ist kann programmieren'' und erhalten die gewünschte Aussage mit einem uneingeschränkten Quantor
\[
 \forall x(x\in I\Rightarrow B(x)).
\]
Diesen Zusammenhang zwischen eingeschränkten und uneingeschränkten Quantoren werden wir in der nächsten Bemerkung zu ``Rechenregeln für Quantoren'' allgemein formulieren.
\end{enumerate}
\end{bsp}

\begin{rk}[Quantorenregeln]
 Ist $A(x)$ ein Prädikat und $K$ eine Menge, so gelten folgende Äquivalenzen:
\begin{enumerate}
 \item Vertauschungsregel für unbeschränkte Quantoren
\[
 \forall x\, A(x)\Leftrightarrow \neg\exists x\,\neg A(x)
\]
\item Vertauschungsregel für beschränkte Quantoren
\[
 \forall x\in K \;A(x)\Leftrightarrow \neg\exists x\in K\;\neg A(x)
\]
\item Beschränkter und unbeschränkter Allquantor
\[
 \forall x\in K\;A(x)\Leftrightarrow \forall x(x\in K\Rightarrow A(x))
\]
\item Beschränkter und unbeschränkter Existenzquantor
\[
\exists x\in K\; A(x)\Leftrightarrow \exists x(x\in K\wedge A(x))
\]
\end{enumerate}
\end{rk}

\begin{bsp}
 Mit den Rechenregeln für Quantoren und den Rechenregeln für Junktoren können wir wieder neue Tatsachen (=Wahrheitswerte neuer Aussagen) herleiten. Als Beispiel betrachten wir das Duale zur Vertauschungsregel für unbeschränkte Quantoren, nämlich:
\[
 \exists xA(x)\Leftrightarrow \neg \forall x\neg A(x)
\]
Wir beginnen also mit $\exists xA(x)$ und erhalten durch Anwenden der Rechenregeln $\neg \forall x\neg A(x)$.
\begin{align*}
 &\exists xA(x)\\
\Leftrightarrow&\neg\neg\exists xA(x)&(\text{Doppelte Negation}) \\
\Leftrightarrow&\neg(\neg\exists x A(x))\\
\Leftrightarrow&\neg(\neg\exists x \neg(\neg A(x)))&(\text{Doppelte Negation})\\
\Leftrightarrow&\neg(\forall x\neg A(x))&(\text{Vertauschungsregel})
\end{align*}
\end{bsp}

\begin{wrn}
 Wir haben keine Distributionsregel mit Quantoren und Junktoren. Die Äquivalenzen
\[
\forall x\, A(x)\,\lor\,\forall x B(x)\,\Leftrightarrow\, \forall x\,(A(x)\,\lor\, B(x))
\]
und
\[
 \exists x A(x)\wedge\exists x B(x)\Leftrightarrow \exists x (A(x)\wedge B(x))
\]
gelten im Allgemeinen \textbf{nicht}. Wir betrachten dazu als Gegenbeispiel die Aussagen
\[
A(x):=``x \text{ ist eine gerade natürliche Zahl''}
\]
und
\[
B(x):=``x\text{ ist eine ungerade natürliche Zahl''}.
\]
Die Aussage
\[
 \exists x A(x)\wedge\exists x B(x)
\]
besagt also in diesem Fall, dass es mindestens eine gerade natürliche Zahl gibt und dass es ebenfalls mindestens eine ungerade natürliche Zahl gibt. Diese Aussage ist offensichtlich wahr. Die Aussage
\[
  \exists x (A(x)\wedge B(x))
\]
besagt nun aber, dass es eine natürliche Zahl gibt, welche ``gleichzeitig'' gerade und ungerade ist, was offensichtlich falsch ist. Die beiden Aussagen sind also nicht äquivalent.
\end{wrn}

\begin{ueb}
  Es seien $P(x)$ ein einstelliges und $Q(y,z)$ ein zweistelliges Prädikat. Formalisieren Sie:
  \begin{enumerate}
      \item Es gibt genau ein $x$ mit $P(x)$.
      \item Es gibt mindestens zwei Dinge mit der Eigenschaft $P$.
      \item Es gibt höchstens ein $x$ mit $P(x)$.
      \item Wenn $P(x)$ und $P(y)$ gilt, dann gilt stets auch $Q(x,y)$.
      \item Für kein $x$ gilt $Q(x,x)$.
  \end{enumerate}
\end{ueb}

\begin{lsg}
  \ifthenelse{\boolean{ml}}{
    Mögliche Lösungen:
  \begin{enumerate}
    \item $\exists x\,(P(x))\,\land\, \forall y,z\, (P(y)\land P(z)\,\Rightarrow\, y=z)$
    \item $\exists x,y\,(P(x)\land P(y)\land x\neq y)$
    \item $\neg \exists x,y\,(P(x)\land P(y)\land x\neq y)$
    \item $\forall x,y\,(P(x)\land P(y)\,\Rightarrow Q(x,y))$
    \item $\forall x\,\neg Q(x,x)$
  \end{enumerate}
  }{~
  \answerspace{6cm}}
\end{lsg}

\begin{ueb}
Geben Sie Prädikate $P(x)$ und $Q(x)$ an, so dass $\forall x\,P(x)\,\lor\,\forall x\,Q(x)$ falsch, aber $\forall x\,(Q(x)\,\lor\, P(x))$ wahr ist.
\end{ueb}
\begin{lsg}
\ifthenelse{\boolean{ml}}{
Zum Beispiel:
\[
P(x):= x>10
\]
und
\[
Q(x):= x<11.
\]
Die Aussage $\forall x\, P(x)\lor \forall x\,Q(x)$ bedeutet, dass jede Zahl grösser als Zehn ist oder, dass jede Zahl kleiner als $11$ ist, diese Aussage ist falsch. Die Aussage $\forall x\,(P(x)\lor Q(x))$ besagt hingegen wahrheitsgemäss, dass jede Zahl (für sich selbst betrachtet) entweder kleiner als $11$ oder grösser als $10$ ist.
}{~
\answerspace{5cm}}
\end{lsg}




\begin{ueb}
Gruppieren Sie folgende Aussagen so, dass in jeder Gruppe alle Aussagen äquivalent sind und keine äquivalenten Aussagen in verschiedenen Gruppen sind.
\begin{itemize}
\item[1.] $\forall x\,(P(x)\Rightarrow Q(x))$
\item[2.] $\exists x\,(P(x)\Leftrightarrow Q(x))$
\item[3.] $\forall x\,(Q(x)\Rightarrow P(x))$
\item[4.] $\forall x\,(\neg P(x)\Rightarrow \neg Q(x))$
\item[5.] $\forall x\,(\neg Q(x)\Rightarrow \neg P(x))$
\item[6.] $\neg\exists x\,(\neg \neg Q(x)\land\neg P(x) )$
\item[7.] $\neg\exists x\,(P(x)\land\neg Q(x))$
\item[8.] $\exists x\, (P(x)\land Q(x))\lor \exists x (\neg P(x)\land\neg Q(x))$
\item[9.] $\forall x\,\exists y\,(P(x)\land P(y))$
\end{itemize}
\end{ueb}
\begin{lsg}
\ifthenelse{\boolean{ml}}{~
Die Aussagen $1.,\,5.,\,7.$ und $3.,\,4.,\,6.$ und $2.,8.$ sind jeweils untereinander äquivalent. Es gibt keine weiteren Äquivalenzen.
}{~
\answerspace{2cm}
}
\end{lsg}



\section{Grundlegende Beweistechniken}

Wir wollen im Folgenden einige der elementarsten Standardbeweistechniken besprechen. Natürlich sollen diese Techniken in etwas komplexeren Beweisen auch beliebig kombiniert werden dürfen. Wir könnten beispielsweise zum Beweis einer Äquivalenz die eine Richtung durch Kontraposition und die andere Richtung direkt oder durch Widerspruch beweisen.

\subsection*{Direkter Beweis einer Implikation}
\begin{itemize}
 \item[] \textbf{Problemstellung:} Es gilt eine Aussage $A\,\Rightarrow \, B$ zu beweisen.
\item[] \textbf{Lösungsstrategie:} Wir geben, basierend auf der Annahme, dass $A$ wahr ist, \textit{zwingende} Argumente für die Richtigkeit von $B$.
\item[]\textbf{Beispiel:} Wir zeigen, wenn $x$ und $y$ gerade (natürliche) Zahlen sind, dann ist auch $x\cdot y$ gerade.
\begin{proof}
Wir nehmen an $x,y$ seien (irgendwelche) gerade natürliche Zahlen (Voraussetzung). Da $x,y$ gerade sind, gibt es natürliche Zahlen $n_x$ und $n_y$ so, dass
\begin{align*}
x=2\cdot n_x&&y=2\cdot n_y
\end{align*}
gilt. Für das Produkt $x\cdot y$ gilt folglich
\[
x\cdot y \,=\, (2\cdot n_x)\cdot(2\cdot n_y)=2\cdot(n_x\cdot 2\cdot n_y)
\]
und ist somit dass $x\cdot y$ ein vielfaches von $2$ also gerade ist.
\end{proof}
\end{itemize}

\subsection*{Beweis durch Widerspruch}
\begin{itemize}
 \item[] \textbf{Problemstellung:} Es gilt eine Aussage $A$ zu beweisen.
\item[] \textbf{Lösungsstrategie:} Nehmen Sie an, die Aussage $A$ wäre falsch und benützen Sie diese Annahme um einen Widerspruch herzuleiten. Leiten Sie also unter der Annahme der Falschheit von $A$ eine Aussage her von der bereits bekannt ist, dass sie falsch ist oder im Widerspruch zur Annahme steht.
\item[]\textbf{Beispiel:} $A$:=``Es gibt keine grösste natürliche Zahl''
\begin{proof}
 Wir nehmen an, dass es eine grösste natürliche Zahl gibt, wir nennen sie $m$. Wir wissen, dass
für jede natürliche Zahl $n$ gilt, dass einerseits $n+1$ ebenfalls eine natürliche Zahl ist und dass
andererseits $n<n+1$ erfüllt ist. Wir wenden dies auf die natürliche Zahl $m$ an und erhalten
damit eine grössere natürliche Zahl (nämlich $m+1$). Dies steht jedoch im
Widerspruch zu unserer ursprünglichen Annahme, dass $m$ die grösste natürliche Zahl sei.
\end{proof}
\end{itemize}

\subsection*{Beweis durch (Gegen-) Beispiel}
\begin{itemize}
 \item[] \textbf{Problemstellung:} Es gilt zu zeigen, dass eine bestimmte Eigenschaft nicht auf alle Objekte (aus einem Kontext) zutrifft.
\item[] \textbf{Lösungsstrategie:} Geben Sie konkret ein Objekt an, welches die erwähnte Eigenschaft nicht besitzt.
\item[]\textbf{Beispiel:} ``Nicht jede natürliche Zahl ist eine Quadratzahl\footnote{Von der Form $x^2$ für eine geeignete natürliche Zahl $x$.}.''
\begin{proof}
Weil die Funktion $f(x)=x^2$ monoton ist (später mehr dazu) und weil $1\cdot1<2<2\cdot2$ gilt, kann die Zahl $2$ nicht als Quadrat von einer natürlichen Zahl geschrieben werden. Somit ist $2$ das (oder ein) gesuchte Gegenbeispiel.
\end{proof}
\end{itemize}

\subsection*{Beweis durch Kontraposition}
\begin{itemize}
 \item[] \textbf{Problemstellung:} Es gilt eine Aussage von der Form $A\Rightarrow B$ zu beweisen.
\item[] \textbf{Lösungsstrategie:} Beweisen Sie die Kontraposition $\neg B\Rightarrow\neg A$.
\item[]\textbf{Beispiel:} ``Für jede natürliche Zahl $n$ gilt: $(n^2+1=1)\Rightarrow (n=0)$''
\begin{proof}
Ist $n\neq 0$ so folgt, dass auch $n^2\neq 0$ gilt. Dies impliziert, dass für jede weitere natürliche Zahl $m$ die Ungleichung $n^2+m\neq m$ erfüllt ist. Insbesondere gilt daher, dass (der Fall $m=1$) $n^2+1\neq 1$ gilt.
\end{proof}
\end{itemize}

\subsection*{Beweis einer Äquivalenz}
\begin{itemize}
 \item[] \textbf{Problemstellung:} Es gilt eine Aussage von der Form $A\Leftrightarrow B$ zu beweisen.
\item[] \textbf{Lösungsstrategie:} Beweisen Sie $B\Rightarrow A$ sowie $A\Rightarrow B$.
\item[]\textbf{Beispiel 1:} ``Für jede natürliche Zahl $n$ gilt: $(n^2+1=1)\Leftrightarrow (n=0)$''
\begin{proof}
Wir haben in den vorhergehenden Beispielen bereits $A\Rightarrow B$ bewiesen, wir müssen also nur noch $B\Rightarrow A$ beweisen. Wir nehmen also $B$ an, es gelte also $n=0$. Draus folgt $n^2=n\cdot n=0\cdot 0=0$ und somit $n^2+1=0+1=1$.
\end{proof}
\item[]\textbf{Beispiel 2:} ``Für jede natürliche Zahl $n$ gilt: $(n\text{ ist gerade})\Leftrightarrow (n^2\text{ ist gerade}).$''
\begin{proof}
Wir beweisen zuerst $(n\text{ ist gerade})\Rightarrow (n^2\text{ ist gerade})$. Wir nehmen also an, dass $n$ eine gerade natürliche Zahl ist. Daraus folgt, dass es eine weitere natürliche Zahl $k$ mit $n=2\cdot k$ gibt. Es folgt, dass
\[
n^2=n\cdot n=2\cdot k\cdot 2\cdot k=2\cdot (k\cdot 2\cdot k)
\]
offenbar gerade ist.\\
Nun wollen wir noch die ``Rückrichtung'' $(n\text{ ist gerade})\Leftarrow (n^2\text{ ist gerade})$ beweisen. Wir wollen diese Richtung durch Kontraposition beweisen und nehmen also an, dass $n$ ungerade sei. Es folgt, dass es eine natürliche Zahl $k$ mit $2k+1=n$ gibt. Folglich gilt:
\[
  n^2=(2k+1)(2k+1)=4k^2+4k+1=\underbrace{4(k^2+k)}_{\text{gerade}}+1.
\]
Also ist $n^2$ ungerade.
\end{proof}
\end{itemize}


\begin{ueb}
Beweisen Sie: Jeder (ganzzahlige) Geldbetrag von mindestens $4$ Cents lässt sich allein mit Zwei- und Fünfcentstücken bezahlen.\\
\textit{Hinweis:} Machen Sie eine Fallunterscheidung ob der zu bezahlende Betrag gerade oder ungerade ist.
\end{ueb}
\begin{lsg}
\ifthenelse{\boolean{ml}}{
Zuerst bemerken wir, dass jeder gerade Betrag mit Zweicentstücken bezahlt werden kann. Ist der gegebene Betrag, sagen wir $x$ cent, ungerade, so muss er mindestens fünf Cent entsprechen, es gilt also $x\geq 5$. Weil $x$ ungerade ist, ist $x-5$ gerade. Wie wir bereits festgestellt haben können wir diesen geraden Betrag mit lauter Zweicentstücken bezahlen. Der gesamte Betrag kann also mit einem Fünfcentstück und Zweicentstücken bezahlt werden.
}{~
\answerspace{8cm}
}
\end{lsg}

\begin{ueb}
Beweisen Sie, dass man $\sqrt{2}$ nicht als Bruch schreiben kann.\\
\textit{Hinweis:} Wenden Sie ein Widerspruchsargument an.
\end{ueb}
\begin{lsg}
\ifthenelse{\boolean{ml}}{
Wir nehmen an, dass $\sqrt{2}$ als gekürzten Bruch dargestellt werden kann und leiten daraus einen Widerspruch her. Es seien $a,b$ ganze, teilerfremde Zahlen mit
\begin{align*}
\sqrt{2}=\frac{a}{b}.
\end{align*}
Quadrieren auf beiden Seiten ergibt
\begin{align*}
  2=\frac{a^2}{b^2}
\end{align*}
und somit
\begin{align*}
  a^2=2b^2.
\end{align*}
Die Zahl $a^2$ muss also gerade sein. Daraus folgt, dass auch $a$ selbst gerade ist. Weil $a$ gerade ist, gibt es eine ganze Zahl $c$ mit der Eigenschaft
\[
a=2c.
\]
Daraus folgt
\[
2b^2=a^2=(2c)^2=4c^2
\]
und damit
\[
b^2=2c^2.
\]
Anhand der letzten Gleichung sehen wir, dass $b^2$ und somit auch $b$ gerade sein muss, dies widerspricht aber der Annahme, dass die Zahlen $a$ und $b$ teilerfremd seien.
}{~
\answerspace{8cm}}
\end{lsg}

