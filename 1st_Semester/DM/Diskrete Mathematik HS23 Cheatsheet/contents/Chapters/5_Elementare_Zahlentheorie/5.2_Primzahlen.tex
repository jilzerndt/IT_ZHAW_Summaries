\subsection{Primzahlen}

\begin{definition}{Primzahl}
Eine natürliche Zahl $p\in\N$ ist eine \textit{Primzahl}, wenn $|T(p)|=2$ gilt. Die Menge aller Primzahlen bezeichnen wir mit $\mathbb{P}$.
\end{definition}

\begin{remark}
Ist $p$ eine Primzahl, dann gilt $T(p)=\{1,p\}$.
\end{remark}
\begin{proof}{Primzahl}
Für jede Zahl $n\in\N$ gilt offensichtlich $n\in T(n)$ und $1\in T(n)$. Bei Primzahlen kommt dazu, dass (wegen $|T(n)|=2$) keine weiteren Teiler existieren.
\end{proof}

\begin{lemma}{Lemma von Euklid}
Folgende Aussagen sind für $p\in\N$ mit $p\neq 1$ äquivalent:
\begin{enumerate}
\item[1.] $\forall n,m\in\N\,\big(p|nm\Rightarrow p|n\vee p|m\big)$
\item[2.] $p\in\mathbb{P}$
\end{enumerate}
\end{lemma}


\begin{lemma}{Primteiler}
 Jede ganze Zahl $z$ mit $z\notin\{-1,1\}$ besitzt einen \textit{Primfaktor} (einen Teiler, der eine Primzahl ist). Formal können wir dies als
\[
\forall z\in\Z\,\big(z\notin\{-1,1\}\Rightarrow T(z)\cap\P\neq\emptyset\big)
\]
ausdrücken.
\end{lemma}


\begin{theorem}{}
 Es gibt unendlich viele Primzahlen.
\end{theorem}
\begin{proof}{Unendlich viele Primzahlen}
Wir machen einen Widerspruchsbeweis. Wir nehmen an, dass es nur endlich viele Primzahlen $\P=\{p_1,..,p_n\}$ gibt. Nach Satz \ref{Primteiler} gibt es eine Primzahl $p_i$ so, dass
 \[
  p_i\,|\,(\prod_{j=1}^np_j)+1.
 \]
Es gibt also eine natürliche Zahl $k$ so, dass
\[
 p_i\cdot k=(\prod_{j=1}^np_j)+1
\]
gilt. Daraus folgt
\begin{align*}
 1=p_i\cdot k-(\prod_{j=i}^np_j)&=p_i\cdot k-(p_1\cdot..\cdot p_i\cdot..\cdot p_n)\\
&=p_i\cdot k-p_i(\underbrace{p_1\cdot..\cdot p_{i-1}\cdot p_{i+1}\cdot..\cdot p_n}_{:=p})\\
&=p_i(k-p).
\end{align*}
Es folgt also, dass $p_i$ ein Teiler von $1$ ist, das steht aber im Widerspruch zu $p_i\in\P$.
\end{proof}

%\begin{corollary}
%Es gibt eine eindeutig bestimmte Folge $(p_i)_{i\in\N}$ in $\P$, so dass
%\begin{align*}
% &\P=\{p_i\mid i\in\N\}\\
%&\forall i,j\in\N\,\big(i<j\Rightarrow p_i<p_j\big)
%\end{align*}
%gilt. Wir nennen das $i$-te Glied $p_i$ dieser Folge die $i$-te Primzahl.
%\end{corollary}
%\begin{proof}
% Wir definieren $(p_i)_{i\in\N}$ rekursiv wie folgt:
%\begin{align*}
% p_1&=2\\
%P_{n+1}&=\min\{p\in\P\mid p>p_n\}
%\end{align*}
%
%\end{proof}
%



\begin{theorem}{Primfaktoren}
 Jede natürliche Zahl grösser als $1$ ist das Produkt von endlich vielen Primzahlen.
\end{theorem}


\begin{theorem}{Primfaktorzerlegung}
Es sei $p_i$ jeweils die $i$-te Primzahl. Für jede natürliche Zahl $n>1$ gibt es eine eindeutig bestimmte, endliche Folge $a_1,..,a_k$ von natürlichen Zahlen mit $a_k\neq 0$, so dass
\[
 n=\prod_{i=1}^k p_i^{a_i}
\]
gilt.
\end{theorem}
