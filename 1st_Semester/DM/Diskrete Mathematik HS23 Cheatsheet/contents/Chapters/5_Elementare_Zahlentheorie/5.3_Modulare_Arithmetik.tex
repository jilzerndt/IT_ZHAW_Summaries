\subsection{Modulare Arithmetik}

\begin{definition}{Modulo}
Es sei $n\in\N$ beliebig. Wir definieren eine Relation $\equiv_n$ auf $\Z$ wie folgt:
\[
 r\equiv_n s:\Leftrightarrow n|(r-s).
\]
Gilt für $r,s\in Z$ die Relation $r\equiv_ns$, dann sagen wir, dass $r$ gleich $s$ modulo $n$ ist und schreiben $r=s \:mod\, n$.
\end{definition}


\begin{remark}
 Die Relation $\equiv_n$ ist für jede natürliche Zahl $n$ eine Äquivalenzrelation auf $\Z$.
\end{remark}

\begin{remark}
Es sei $n\in\N$ beliebig. Für je zwei ganze Zahlen $x$ und $y$ gilt $x\modn y$ genau dann, wenn $x$ und $y$ denselben Rest bei Division durch $n$ lassen.
\end{remark}


\begin{corollary}{}
 Es sei $n\in\N$ beliebig. Jede ganze Zahl $z$ steht mit genau einer natürlichen Zahl aus $\{0,..n-1\}$ in der Relation $\equiv_n$.
\end{corollary}

\begin{definition}{Restklasse}
Es sei $n\in\N$ beliebig. Für jede ganze Zahl $z$ bezeichnen wir mit
\[
 [z]_n:=\{x\in\Z\mid x\modn z\}
\]
die Äquivalenzklasse von $z$ bezüglich der Relation $\modn$ und nennen diese auch die \textit{Restklasse} von $z$. Abkürzend bezeichnen wir $[z]_n$ auch mit $\bar k$, wenn $k\in\{0,..,n-1\}$ und $z\modn k$ gilt.
\end{definition}

\begin{corollary}{}
Es sei $n\in\N$ beliebig. Es gilt
\[
 [z]_n=\{z+yn\mid y\in\Z\}=\{....z-3n,z-2n,z-n,z,z+n,z+2n,z+3n,..\}.
\]
\end{corollary}


\begin{remark}
Es sei $n\in\N$ beliebig. Für ganze Zahlen $x,x'$ und $y,y'$ gelten\footnote{Wenn die natürliche Zahl $n$ aus dem Kontext klar ersichtlich ist, so lassen wir diese in der Notation $[x]_n$ auch manchmal weg und schreiben bloss $[x]$.}:
\begin{enumerate}
 \item $[x]=[x']\land [y]=[y']\Rightarrow [x+y]=[x'+y']$
 \item $[x]=[x']\land [y]=[y']\Rightarrow [xy]=[x'y']$
\end{enumerate}
\end{remark}

\begin{definition}{Menge aller Restklassen}
 Es sei $n\in\N$ beliebig. Die Menge aller Restklassen von $\Z$ modulo $n$ bezeichnen wir mit
\[
\Z/n=\{[z]_n\mid z\in\Z\}=\{\bar k\mid 0\leq k<n-1\wedge z\modn k\}=\{\bar 0,\bar1,\bar2,..,\overline{n-1}\}.
\]
Wir definieren zwei Verknüpfungen $\cdot:(\Z/n)^2\rightarrow \Z/n$ und $+:(\Z/n)^2\rightarrow \Z/n$ durch die Zuordnungen
\[
 [x]_n+[y]_n:=[x+y]_n
\]
und
\[
 [x]_n\cdot[y]_n:=[xy]_n.
\]
\end{definition}

\begin{theorem}{Primzahlen und Restklassen}
Es sei $n\in\N\backslash\{1\}$ beliebig. Folgende Aussagen sind äquivalent:
\begin{enumerate}
\item[1.] $n$ ist eine Primzahl.
\item[2.] Für jedes $\bar k\in\Z/n$ mit $\bar k\neq\bar 0$ gibt es genau ein $r\in\{0,..,n-1\}$ mit $\bar k\cdot\bar r=\bar 1$.
\end{enumerate}
Die zweite Aussage besagt, dass man in $\Z/n$ Gleichungen von der Form $ax=b$ stets nach $x$ auflösen kann. Sind $\bar k,\bar r\in\Z/n$ mit $\bar k\cdot\bar r=\bar 1$, so sagen wir $\bar r$ sei invers (bezüglich der Multiplikation) zu $\bar k$ und schreiben auch $(\bar{k})^{-1}$ für $\bar r$.
\end{theorem}

\subsection{Chinesischer Restsatz}

\begin{howto}{Lösen simultaner Kongruenzen}
Wir wollen ein System simultaner Kongruenzen mit zwei Gleichungen lösen, etwa
\begin{align*}
x&\equiv_{n_1} y_1\\
x&\equiv_{n_2} y_2
\end{align*}
mit $n_1$ und $n_2$ teilerfremd. Wir gehen schrittweise wie folgt vor:
 \begin{enumerate}
  \item Durch sukzessives Teilen mit Rest (wie im Beweis von Satz \ref{hauptideal}) erhalten wir ganze Zahlen $a,b$ mit $an_1+bn_2=1$.
\item Wir setzen $x:=y_1bn_2+y_2an_1$.
 \end{enumerate}
\end{howto}

\begin{lemma}{Fermat}
Ist $a\in\Z/n$ mit $n>0$ invertierbar, dann ist die Funktion
\begin{align*}
f&:\Z/p\to\Z/p\\
f&(x)=\bar a\cdot x
\end{align*}
surjektiv.
\end{lemma}

\begin{lemma}{Kleiner Fermat}
Ist $p\in\P$ und $a$ kein Vielfaches von $p$, dann gilt
\[
a^{p-1}\equiv_p1.
\]
\end{lemma}