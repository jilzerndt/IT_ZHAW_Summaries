\subsubsection{Mengen}

\begin{example}~
\begin{itemize}
\item Die Menge $\{2,34,77\}$ enthält die drei Elemente $2$, $34$ und $77$.
\item Die Menge $\{\,\}$ heisst \textit{leere Menge}. Die leere Menge ist die einzige Menge, die gar keine Elemente besitzt, sie wird mit $\varnothing $ bezeichnet.
\end{itemize}
\end{example}

\begin{remark}
Wenn keine Missverständnisse zu befürchten sind, so beschreibt man Mengen auch durch ``angedeutete'' Aufzählung ihrer Elemente. Die Menge $\mathbb{N}$ der \textit{natürlichen Zahlen} wird beispielsweise durch
\[
\N:=\{0,1,2,\dots\}
\]
beschrieben. 
Die Menge der \textit{ganzen Zahlen} wird durch
\[
\Z:=\{\dots, -2,-1,0,1,2\dots\}
\]
beschrieben.
\end{remark}


\begin{remark}
Die Tatsache, dass Mengen durch ihre Elemente eindeutig beschrieben werden hat zur Folge, dass Mengen sehr ``unstrukturierte Datentypen'' sind, d.h. Mengen haben keine ``innere Ordnung''. Es gelten unter anderem:
\begin{itemize}
\item Für beliebige $z,x_1,\dots,x_n$
\[
z\in\{x_1,\dots x_n\}\,\Leftrightarrow\, z=x_1\lor\dots\lor z=x_n
\]
\item Für alle $x$
\[
\{x\}=\{x,x\}=\{x,x,x\}=\dots
\]
\item Für alle $x,y$
\[
\{x,y\}=\{y,x\}.
\]
\end{itemize}
\end{remark}

\begin{example}~
\begin{itemize}
\item Die Menge aller Hühner ist eine (echte) Teilmenge der Menge aller Vögel, weil alle Hühner Vögel sind (und weil es Vögel gibt die keine Hühner sind).
\item Die Menge aller Primzahlen ist eine (echte) Teilmenge von $\N$.
\item Die Menge aller Primzahlen ist \textit{keine} Teilmenge aller ungeraden Zahlen, weil die Zahl $2$ eine Primzahl aber keine ungerade Zahl ist.
\end{itemize}
\end{example}

\begin{figure*}[h]
\begin{example}
Wenn man aus der Menge aller Tische die Dinge mit der Eigenschaft ``drei Beine zu haben'' aussondert (und zusammenfasst), dann erhält man die Menge aller dreibeinigen Tische.
\begin{center}
\begin{framed}
\def\firstcircle{(0,0) circle (2cm)}
\def\secondcircle{(0.5,-1.8) circle (1.5cm)}

\colorlet{circle edge}{blue!50}
\colorlet{circle area}{blue!20}

\tikzset{filled/.style={fill=circle area, draw=circle edge, thick},
    outline/.style={draw=circle edge, thick}}

\setlength{\parskip}{5mm}
% Set A and B
\begin{tikzpicture}
\begin{scope}
\clip \firstcircle;
\fill[filled] \secondcircle;
\end{scope}
\draw[outline] \firstcircle;
\node[anchor=south] at (0,0.2) {Alle Tische ($T$)};
\node[anchor=south] at (0.4,-1.4) {$3$-beinig ($B$)};
\node[anchor=south] at (0.0,-2.8) {$B=\{x\in T\mid x\text{ hat $3$ Beine} \}$};
\end{tikzpicture}

\caption*{Veranschaulichung der Mengenbildung durch prädikative Schreibweise.}
\end{framed}
\end{center}
\end{example}
\end{figure*}

\begin{example}
Die Menge aller geraden natürlichen Zahlen erhält man auch durch die prädikative Schreibweise,
\begin{itemize}
\item $\{n\in\N\mid n\text{ ist gerade}\}$
\item $\{n\in\N\mid \exists z\in\N\,(n=2\cdot z)\}$
\end{itemize}
\end{example}

\begin{example}
Die Menge der geraden natürlichen Zahlen lässt sich nun mithilfe der Funktion $F(x)=2\cdot x$ als
\[
\{F(x)\mid x\in\N\} = \{2x\mid x\in\N\}
\]
schreiben.
\end{example}

\begin{figure*}[h]
\begin{example}
Die Schnittmenge der Menge der Luxusgüter mit der Menge aller Uhren beinhaltet genau die Luxusuhren.
\begin{center}
\begin{framed}
\def\firstcircle{(0,0) circle (1.8cm)}
\def\secondcircle{(2.5,0) circle (1.8cm)}

\colorlet{circle edge}{blue!50}
\colorlet{circle area}{blue!20}

\tikzset{filled/.style={fill=circle area, draw=circle edge, thick},
    outline/.style={draw=circle edge, thick}}

\setlength{\parskip}{5mm}
% Set A and B
\begin{tikzpicture}
\begin{scope}
\clip \firstcircle;
\fill[filled] \secondcircle;
\end{scope}
\draw[outline] \firstcircle;
\node at (-0.5,0) {Uhren};
%
\draw[outline] \secondcircle;
\node at (3,0.0) {Luxusgüter};
%
\node[anchor=south] at (current bounding box.north) {Luxusuhren};
\end{tikzpicture}

\caption*{Veranschaulichung der Mengenbildung durch Schnitt von zwei Mengen.}
\end{framed}
\end{center}
\end{example}
\end{figure*}

\begin{example}\label{bsp:vereinSchnitt}~
\begin{enumerate}
\item $\N=\{n\in \N\mid n\text{ ist gerade}\}\cup \{n\in \N\mid n\text{ ist ungerade}\}$
\item $\varnothing=\{n\in \N\mid n\text{ ist gerade}\}\cap \{n\in \N\mid n\text{ ist ungerade}\}$
\item Sind $X_a$ und $X_b$ beliebige Mengen, dann gilt:
\[
X_a\cup X_b=\bigcup_{i\in\{a,b\}}X_i.
\]
\item Ist für jede natürliche Zahl $n$ die Menge $X_n$ als $\{0,\dots, n\}$ gegeben, dann gilt
\[
\bigcup_{n\in \N}X_n\,=\,\N
\]
und
\[
\bigcap_{n\in \N}X_n\,=\,\{0\}.
\]
\end{enumerate}
\end{example}


\begin{example}
    Beschreiben Sie folgende Mengen:
    \begin{enumerate}
        \item $\{0,2,4,\dots \}\cap \{p\in\N\mid p\text{ ist eine Primzahl} \}$
        \item $\N\cap\{\N\}$
        \item $\N\cup\{\N\}$
        \item $\{3x\mid x\in\N \} \cap \{5x\mid x\in\N \}$
    \end{enumerate}
    \begin{example}
        \ifthenelse{\boolean{ml}}
        {~
            \begin{enumerate}
                \item $\{2\}$
                \item $\varnothing$
                \item $\{\N,0,1,2,\dots \}$
                \item $\{15x\mid x\in\N \}$
            \end{enumerate}
        }{~
            \answerspace{3cm}}
    \end{example}
\end{example}

\begin{figure*}[h]
   \begin{remark}\label{rk:disjunkt-paarweisedisjunkt}
       Für Mengen $\{X_i\mid i \in I \}$, hat die Annahme
       \[
           \bigcap_{i\in I}X_i=\varnothing
       \]
       nicht notwendigerweise zur Folge, dass die $X_i$'s paarweise disjunkt sind.
        \begin{center}
            \begin{framed}
                \def\firstcircle{(-4.5,2) circle (0.5cm)}
                \def\secondcircle{(-5.4,2) circle (0.5cm)}
                \def\thirdcircle{(-7,2) circle (0.5cm)}
                %
                \def\firstcircleA{(0,2) circle (0.5cm)}
                \def\secondcircleA{(1.1,2) circle (0.5cm)}
                \def\thirdcircleA{(2.2,2) circle (0.5cm)}
                %
                \colorlet{circle edge}{blue!50}
                \colorlet{circle area}{red!50}

                \tikzset{filled/.style={fill=circle area, draw=circle edge, thick},
                    outline/.style={draw=circle edge, thick}}

                \setlength{\parskip}{5mm}
                % Set A and B
                \begin{tikzpicture}
                \begin{scope}
                \clip \firstcircle;
                \fill[filled] \secondcircle;
                \end{scope}
                \draw[outline] \firstcircle node {$C$};
                \draw[outline] \secondcircle node {$B$};
                \draw[outline] \thirdcircle node {$A$};
                \node[anchor=south] at (1.2,0.8) {Paarweise disjunkt};
                %
                \draw[outline] \firstcircleA node {$A$};
                \draw[outline] \secondcircleA node {$B$};
                \draw[outline] \thirdcircleA node {$C$};
                %
                \node[anchor=south] at (-5.1,0.8) {Disjunkt ($A \cap B\cap C=\varnothing$),};
                \node[anchor=south] at (-5.3,0.3) {\textit{nicht} paarweise disjunkt};
                \end{tikzpicture}
            \end{framed}
        \end{center}
    \end{remark}
\end{figure*}

\begin{example}
Die Menge der ungeraden Zahlen können wir als
\[
\N\setminus\{2x\mid x\in\N\}
\]
schreiben.
\end{example}

\begin{example}
Beschreiben Sie folgende Mengen.
\begin{enumerate}
\item $\N\setminus\{x\in\N\mid x \text{ ist gerade} \}$
\item $\{x\in\N\mid x\text{ ist gerade} \}\setminus \{3x\mid x\in \N \}$
\item $\N\setminus (\N\setminus \Z)$
\end{enumerate}
\end{example}
\begin{example}
\ifthenelse{\boolean{ml}}
{~
\begin{enumerate}
\item $\{x\in \N\mid x\text{ ist ungerade} \}$
\item Die Menge aller geraden und nicht durch $3$ teilbaren natürlichen Zahlen.
\item $\N$
\end{enumerate}
}
{~
\answerspace{4cm}}
\end{example}

\begin{example}
Zeigen Sie für beliebige Mengen $A$ und $B$:
\begin{enumerate}
\item $A\setminus(A\setminus B)=A\cap B$
\item $(A\setminus B)\setminus B=A\setminus B$
\end{enumerate}
\end{example}
\begin{example}
\ifthenelse{\boolean{ml}}
{~
	\begin{enumerate}
	\item Es sei $x$ beliebig, es gilt:
	\begin{align*}
	x\in A\setminus (A\setminus B)&\Leftrightarrow x\in A\land (x\notin A\setminus B)\\
	&\Leftrightarrow x\in A\land \neg(x\in A\land x\notin B)\\
	&\Leftrightarrow x\in A\land (x\notin A\lor x\in B)\\
	&\Leftrightarrow (x\in A\land x\notin A)\lor (x\in A\land x\in B)\\
	&\Leftrightarrow x\in A\land x\in B\\
	&\Leftrightarrow x\in A\cap B
	\end{align*}
	\item Es sei $x$ beliebig, es gilt:
	\begin{align*}
	x\in (A\setminus B)\setminus B&\Leftrightarrow x\notin B\land (x\in A\setminus B)\\
	&\Leftrightarrow x\notin B\land (x\in A\land x\notin B)\\
	&\Leftrightarrow x\in A\land x\notin B\\
	&\Leftrightarrow x\in A\setminus B
	\end{align*}
	\end{enumerate}
}
{~
	\answerspace{8cm}
}
\end{example}

\begin{example}
\begin{enumerate}
\item $\mathcal{P}(\varnothing)=\{\varnothing\}\neq \varnothing$
\item $\mathcal{P}(\{0,1\})=\big\{ \varnothing,\{0\},\{1\},\{0,1\} \big\}$
\end{enumerate}
\end{example}

\begin{example}~
Beschreiben Sie in aufzählender Form:
\begin{enumerate}
\item $\mathcal{P}(\{3,4\})$
\item $\mathcal{P}(\{a,\{c\}\})$
\item $\mathcal{P}(\{\{\{x\}\}\})$
\end{enumerate}
\end{example}
\begin{example}
\ifthenelse{\boolean{ml}}
{~
	\begin{enumerate}
	\item $ \{\varnothing, \{3\}, \{4\}, \{3, 4\}\} $
	\item $ \{\varnothing, \{a\}, \{\{c\}\}, \{a, \{c\}\}\} $
	\item $ \{\varnothing, \{\{\{x\}\}\}\} $
	\end{enumerate}
}
{~
\answerspace{4cm}
}
\end{example}

\begin{example}
Geben Sie Mengen $A$ und $B$ an, mit
\[
\mathcal{P}(A)\cup\mathcal{P}(B)\neq\mathcal{P}(A\cup B).
\]
\end{example}
\begin{example}
\ifthenelse{\boolean{ml}}{~
	z.B. $A=\{1,2\}$ und $B=\{2,3\}$
	}{~
	\answerspace{2.5cm}
	}
\end{example}

\begin{example}~
\begin{itemize}
    \item Die Menge aller geraden natürlichen Zahlen und die Menge aller ungeraden natürlichen Zahlen bilden zusammen eine Partition der natürlichen Zahlen. Genauer, falls $G$ die Menge der geraden natürlichen Zahlen und $U$ die Menge der ungeraden natürlichen Zahlen ist, dann ist die Menge $\{G,U\}$ eine Partition von $\N$ mit zwei Blöcken.
    \item Für
    \begin{align*}
    A_i=\{i,-i\}
    \end{align*}
    ist die Menge $P=\{A_i\mid i\in\N\}$ eine Partition der Menge $\Z$ (in unendlich viele Blöcke).
\end{itemize}
\end{example}

\begin{example}~
\begin{enumerate}
\item Geben Sie eine Partition von $\N$ in unendlich viele Blöcke an.
\item Geben Sie eine Partition von $\N$ an, deren Blöcke alle unendlich gross sind.
\item Geben Sie eine Partition der rationalen Zahlen in unendlich viele, unendlich grosse
Blöcke an.
\end{enumerate}
\end{example}
\begin{example}
\ifthenelse{\boolean{ml}}{~
\begin{enumerate}
\item Z.B. $\{\{0\},\{1\},\dots \}$ also $\{P_i\mid i\in\N\}$ mit $P_i=\{i\}$.
\item Z.B. $\{P_1,P_2\}$ mit $P_1=\{x\in \N\mid x\text{ ist gerade} \}$ und $P_2=\{x\in \N\mid x\text{ ist ungerade} \}$.
\item Z.B. $ \{P_i\mid i\in \Z\}$ wobei $P_i=\{ x\in\Q\mid i<x\leq i+1 \}$.
\end{enumerate}}{~
\answerspace{6cm}}
\end{example}