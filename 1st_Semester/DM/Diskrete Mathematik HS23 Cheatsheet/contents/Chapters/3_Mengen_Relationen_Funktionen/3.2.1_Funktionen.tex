\subsection{Funktionen}

\begin{definition}{Funktion}
    Es seien $A$ und $B$ beliebige Mengen. Eine Relation $f\subseteq A\times B$ ist eine \textit{Funktion} von $A$ nach $B$, falls:
    \begin{align*}
    \forall x\in A\exists!y\in B((x,y)\in f)
    \end{align*}
    gilt. In diesem Fall schreiben wir $f:A\to B.$
\end{definition}

\begin{concept}{Notation Funktionen}
    Im Kontext einer Funktion $f:A\to B$ verwenden wir folgende Schreibweisen und Konventionen:
    \begin{itemize}
        \item Da zu jedem $x\in A$ ein eindeutig bestimmtes Element $y\in B$ mit $(x,y)\in f$ existiert, kann dieses $y$ mit $f(x)$ bezeichnet und \textit{Funktionswert von $f$ bei $x$} genannt werden.
        \item Die Menge aller Funktionswerte $Im(f) := \{f(x)\mid x\in A \}$ wird als \textit{Bild(menge)} von $f$ bezeichnet.
        \item Die Menge $A$ nennen wir den Definitionsbereich von $f$ und schreiben dafür auch $Dom(f)$.
        \item Der Definitionsbereich ist eindeutig durch die Funktion gegeben:
        \begin{align*}
            A=Dom(f)=\{x\mid \exists y ((x,y)\in f) \}=\{x\mid \exists y (f(x)=y )\}
        \end{align*}
        \item Die Menge $B$ ist durch die Voraussetzung $f:A\to B$ nicht eindeutig bestimmt, tatsächlich gilt $f:A\to B$ für jede Menge $B$ mit $Im(f)\subseteq B$.
    \end{itemize}
\end{concept}

\begin{definition}{Injektiv}
    Eine Funktion $f$ ist genau dann \textit{injektiv}, wenn die Relation
    \begin{align*}
        f^{-1}=\{(y,x)\mid (x,y)\in f\}
    \end{align*}
    eine Funktion ist.\\
    Umgangssprachlich: Jeder Output kann nur mittels einem einzigen Inputelement erreicht werden.
\end{definition}

\begin{definition}{Umkehrfunktion}\\
    Ist $f:A\to B$ eine injektive Funktion, dann nennt man $f^{-1}:Im(f)\to A$ die \textit{Umkehrfunktion} oder \textit{inverse Funktion} von $f$.
\end{definition}


\begin{lemma}{Äquivalenzen zur Injektivität}\\
    Für $f:A \to B$ sind folgende Aussagen äquivalent.
    \begin{enumerate}
        \item Die Funktion $f$ ist injektiv
        \item Für alle $x,y\in A$ gilt: Aus $x\neq y$ folgt $f(x)\neq f(y)$
        \item Für alle $x,y\in A$ gilt: Aus $f(x)=f(y)$ folgt $x=y$
    \end{enumerate}
\end{lemma}

\begin{definition}{Surjektiv}
    Eine Funktion $f:A\to B$ heisst \textit{surjektiv} auf $B$, wenn $B=Im(f)$.\\
    Umgangssprachlich: Realisiert jedes Element einer gegebenen Zielmenge als Funktionswert.
\end{definition}

\begin{definition}{Bijektiv}
    Ist die Funktion $f$ injektiv und surjektiv, so sagen wir $f:A\to B$ sei \textit{bijektiv}.
\end{definition}



%Funktionen lassen sich bei Bedarf auf gewünschte Definitionsbereiche ``einschränken''.
%
%\begin{definition}
%    Ist $f:A\to B$ eine Funktion und $X$ eine beliebige Menge, dann ist die \textit{Einschränkung} von $f$ auf $X$ wie %folgt gegeben:
%    \begin{align*}
%        &f\upharpoonright X:A\cap X\to B\\
%        &f\upharpoonright X(x)=f(x).
%    \end{align*}
%    Umgekehrt ist eine Funktion $g$ eine \textit{Erweiterung} von $f$, wenn $g\upharpoonright A= f$ gilt.
%\end{definition}
%

\begin{definition}{Komposition}
    Sind $f:A\to B$ und $g:B\to C$ Funktionen, dann ist die Komposition $g$ nach $f$ durch
    \begin{align*}
        &g\circ f:A\to C\\
        (&g\circ f)(x)=g(f(x))
    \end{align*}
    gegeben.
\end{definition}

\begin{lemma}{Regeln der Komposition}\\
    Für beliebige Funktionen $f:X\to Y$ und $g:Y\to Z$ gelten folgende Aussagen:
    \begin{enumerate}
        \item Falls $f:X\to Y$ und $g:Y\to Z$ injektiv sind, dann ist auch $g\circ f:X\to Z$ injektiv.
        \item Falls $f:X\to Y$ und $g:Y\to Z$ surjektiv sind, dann ist auch $g\circ f:X\to Z$ surjektiv.
    \end{enumerate}
\end{lemma}

\begin{proof}{Injektivität zeigen}
    Die Aussagen in b) und c) sind offensichtlich äquivalent (Kontraposition). Für die Äquivalenz von $a)$ und $c)$ sei $f$ injektiv. Die Relation $f^{-1}=\{(y,x)\mid (x,y)\in f\}$ sei also eine Funktion. Daraus folgt, dass zu jedem $y$ höchstens ein $x$ mit $(y,x)\in f^{-1}$ existiert. Formal heisst das:
    \begin{align*}
        (y,x)\in f^{-1}\land (y,x')\in f^{-1}\Rightarrow x=x'
    \end{align*}
    Dies ist gleichbedeutend mit
    \begin{align*}
        (x,y)\in f\land (x',y)\in f\Rightarrow x=x'
    \end{align*}
    und somit
    \begin{align*}
        f(x)=y\land f(x')=y\Rightarrow x=x'
    \end{align*}
    was genau der Aussage in c) entspricht.
\end{proof}

\begin{proof}{Injektivität und Surjektivität der Komposition zeigen}
    \begin{enumerate}
        \item Wir nehmen an, dass $f:X\to Y$ und $g:Y\to Z$ injektiv sind und zeigen, dass $g\circ f:X\to Z$ injektiv ist. Es seien $a,b\in X$ verschiedene Elemente. Weil $f$ injektiv ist, folgt $f(a)\neq f(b)$ und folglich aus der Injektivität von $g$, wie gewünscht
        \begin{align*}
            g\circ f(a) = g(f(a))\neq g(f(b))=g\circ f(b).
        \end{align*}
        \item Für die zweite Behauptung müssen wir zeigen, dass zu jedem $z\in Z$ ein $x\in X$ existiert mit $g(f(x))= z$. Es sei also $z\in Z$ beliebig. Weil $g:Y\to Z$ surjektiv ist, gibt es ein $y\in Y$ mit $g(y)=z$. Weil $f:X\to Y$ ebenfalls surjektiv ist, gibt es weiter ein $x\in X$ mit $f(x) = y$. Insgesamt haben wir wie gewünscht
        \begin{align*}
            g(f(x))=g(y)=z.
        \end{align*}
    \end{enumerate}
\end{proof}

