\subsection{Syntax}


\begin{definition}{Alphabet der Aussagenlogik}
Das \textit{Alphabet der Aussagenlogik} (auch Zeichenvorrat genannt) besteht aus:
\begin{itemize}
\item Konstanten $\top$ und $\bot$.
\item Variablen $p,q,r,s,\dots,p_0,p_1,p_2,\dots$
\item Klammern $(,)$
\item Junktoren $\neg,\land,\lor,\to$
\end{itemize}
Die Menge der Variablen bezeichnen wir mit $\mathbb{V}$.
\end{definition}

\begin{comment}
Nachdem wir nun die Zeichen festgelegt haben, aus welchen die ``Wörter der
Aussagenlogik'' zusammengesetzt sind, werden
wir in der nächsten Definition, die für uns interessanten Wörter festlegen. Wir
definieren, also im Sinn vom einführenden Beispiel, die
``zulässigen Wörter'' (genannt Formeln) der Aussagenlogik.
\end{comment}

\begin{definition}{Atomare Formel}
Jede Variable und jede Konstante ist eine \textit{atomare Formel}. Wir bezeichnen die
Menge aller atomaren Formeln mit $\mathbb{A}:=\{\bot,\top,p,q,r,s,\dots,p_0,p_1,p_2,\dots
\}$.
Die \textit{Formeln} der Aussagenlogik sind dann wie folgt gegeben:
\begin{itemize}
\item Alle atomaren Formeln sind Formeln.
\item Sind $P$ und $Q$ schon Formeln, dann auch: $(P\land Q)$, $(P\lor Q)$, $(P\to Q)$ und $\neg P$.
\end{itemize}
Wir schreiben $\mathbb{F}$ für die Menge aller aussagenlogischen Formeln.
\end{definition}

\begin{comment}
\begin{remark} Ist eine Formel von einem Klammernpaar umgeben, dann lassen wir die äussersten
    Klammern
zugunsten einer besseren Lesbarkeit weg; wir schreiben beispielsweise $(p_0\lor p_1)\land
p_3$ anstelle von $((p_0\lor p_1)\land p_3)$. Des Weiteren setzen wir folgende Operatorrangfolge fest: Die Negation bindet stärker als die Konjunktion und die Disjunktion, die wiederum stärker binden als die Implikation.
\end{remark}
\end{comment}

