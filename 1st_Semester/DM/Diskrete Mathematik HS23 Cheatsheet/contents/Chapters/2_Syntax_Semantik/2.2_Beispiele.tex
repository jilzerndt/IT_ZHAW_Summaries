\subsubsection{Semantik}

\begin{remark}
    Die Junktoren können wir auch als boolesche Funktionen (Funktionen, die
    Wahrheitswerte verarbeiten) anschauen:
\begin{align*}
\for(x,y) &= \begin{cases}
\true&\text{falls }x=\true\text{ oder }y=\true\\
\false&\text{sonst}
\end{cases}\\
\fand(x,y) &= \begin{cases}
\true&\text{falls }x=\true\text{ und }y=\true\\
\false&\text{sonst}
\end{cases}\\
\fnot(x) &=\begin{cases}
\true&\text{falls }x=\false\\
\false&\text{sonst}
\end{cases}
\end{align*}
Durch diese Interpretation können wir die obige Definition etwas knapper formulieren:
\begin{itemize}
\item $\widehat{B}(F\land G)=\fand(\widehat{B}(F),\widehat{B}(G))$
\item $\widehat{B}(F\lor G)=\for(\widehat{B}(F),\widehat{B}(G))$
\item $\widehat{B}(\neg F)=\fnot(\widehat{B}(F))$
\end{itemize}
Mithilfe dieser Darstellung können wir, wenn eine Belegung $B$ gegeben ist, den Wahrheitswert einer beliebigen aussagenlogischen Formel unter der Belegung $B$ ``berechnen''.
\end{remark}

\begin{example}
Es sei eine Belegung $B$ gegeben, die $B(p_n)=\true$ genau dann erfüllt, wenn $n$ eine
gerade Zahl ist. Wir berechnen den Wahrheitswert von $(p_4\to(p_5\to p_6))\lor p_{13}.$
\tcblower
\begin{align*}
\widehat{B}((p_4\to(p_5\to p_6))\lor p_{13})&=\for(\widehat{B}(p_4\to(p_5\to p_6)),
\underbrace{\widehat{B}(p_{13})}_{\false}\}\\
&=\widehat{B}(p_4\to(p_5\to p_6))\\
&=\widehat{B}(\neg p_4\,\lor\,(p_5\to p_6))\\
&=\for(\widehat{B}(\neg p_4),\widehat{B}(p_5\to p_6))\\
&=\for( \fnot(\underbrace{\widehat{B}( p_4)}_{\true}),\widehat{B}(\neg p_5\,\lor\,
p_6))\\
&=\for( \false,\widehat{B}(\neg p_5\,\lor\,
p_6))\\
&=\widehat{B}(\neg p_5\,\lor\, p_6)\\
&=\for(\widehat{B}(\neg p_5),\underbrace{\widehat{B}(p_6)}_{\true})\\
&=\true
\end{align*}
\end{example}

\begin{example}
Von einer Belegung $B:\mathbb{V}\to \{\false,\true\}$ seien folgende Werte bekannt:
\begin{align*}
B(p) &= B(q) = B(r) = B(s) =\true\\
B(u) &= B(v) = \false
\end{align*}
Bestimmen Sie $\hat B$ von folgenden Formeln:
\begin{enumerate}
\item $p\to s$
\item $(u\to r)\land s$
\item $v\lor((r\to s)\land u)$
\end{enumerate}
\tcblower
\begin{enumerate}
\item
\[
\hat B(p\to s)=\hat B(\neg p\lor s)=\for(\hat B(\neg p),\underbrace{\hat
B(s)}_{\true})=\true
\]
\item
\begin{align*}
\hat B((u\to r)\land s)&=\fand (\hat B((u\to r)), \underbrace{\hat
B(s)}_{\true})=\hat B(u\to r)\\
&=\hat B(\neg u\lor r)=\for(\hat B(\neg u),\underbrace{\hat B(r)}_{\true})=\true
\end{align*}
\item
\begin{align*}
\hat B(v\lor((r\to s)\land u))
&=\for(\underbrace{\hat B(v)}_{\false},\hat B((r\to s)\land u))\\
&=\hat B((r\to s)\land u)\\
&=\fand(\hat B(r\to s),\underbrace{\hat B(u)}_{\false})=\false
\end{align*}
\end{enumerate}
\end{example}

\subsubsection{Wahrheitstabellen}

\begin{example}
    Die Teilformeln von der Formel $p_0\to (q\lor p_1)$ sind: $p_0,p_1,q,(q\lor p_1)$ und $p_0\to (q\lor p_1)$. Eine vollständige Wahrheitstabelle von $p_0\to (q\lor p_1)$ ist:
    \begin{center}
        \begin{tabular} {| c | c | c || c | c |}
            \hline
            $p_0$ & $q$ & $p_1$ & $q\lor p_1$ & $p_0\to (q\lor p_1)$ \\ \hline
            0 & 0 & 0 & 0 & 1 \\
            0 & 0 & 1 & 1 & 1\\
            0 & 1 & 0 & 1 & 1\\
            0 & 1 & 1 & 1& 1\\
            1 & 0 & 0 & 0 & 0\\
            1 & 0 & 1 & 1 & 1\\
            1 & 1 & 0 & 1 & 1\\
            1 & 1 & 1 & 1 & 1\\ \hline
        \end{tabular}
        %
    \end{center}
    \smallskip
\end{example}

\subsubsection{Semantische Eigenschaften}

\begin{example}
    Einige allgemeingültige Formeln:
    \begin{align*}
    &p\lor\neg p& &p\to (q\to p)& &F\to F&.
    \end{align*}
    Einige erfüllbare  nicht allgemeingültige Formeln:
    \begin{align*}
    &p_1\lor (p_2\lor p_3)& &p_3& &p\to q&
    \end{align*}
    Einige unerfüllbare Formeln:
    \begin{align*}
    &(p_1\to\neg p_1)\land(\neg p_1\to p_1)& &\neg p_3\land p_3& &\neg (F\to F)&
    \end{align*}
    Welche der Formeln
    \begin{align*}
    &(p_1\to (p_2\lor p_1))\lor(\neg p_1\lor (p_2\land p_1))& &\neg p_3\land p_3& &\neg (F\to \neg F)&
    \end{align*}
    sind allgemeingültig, welche erfüllbar und welche unerfüllbar?
\end{example}

\begin{example}
    Zeigen Sie: Eine aussagenlogische Formel $F$ ist genau dann allgemeingültig, wenn $\neg F$ unerfüllbar ist.
    \tcblower
    Es sei $F$ eine beliebige aussagenlogische Formel. Wir müssen folgende Behauptungen beweisen:
        \begin{itemize}
            \item Ist $F$ allgemeingültig, dann ist $\neg F$ nicht erfüllbar.
            \item Ist $\neg F$ nicht erfüllbar, dann ist $F$ allgemeingültig.
        \end{itemize}
        Für die erste Behauptung nehmen wir an, dass $F$ allgemeingültig sei. Aus der Allgemeingültigkeit von $F$ folgt $\hat B(F)=\true$ für jede Belegung $B$. Somit gilt $\hat B(\neg F)=\false$ für jede Belegung $B$, also ist die Formel $\neg F$ unerfüllbar.

        Für die zweite Behauptung nehmen wir nun an, dass die Formel $\neg F$ nicht erfüllbar sei. Es gilt somit $\hat B(\neg F)=\false$ für jede Belegung $B$. Daraus folgt $\hat B(F)=\true$ für jede Belegung $B$ und somit, dass $F$ allgemeingültig ist.
\end{example}


\begin{example}
    Ist die Behauptung korrekt, dass jede Formel genau dann erfüllbar ist, wenn ihre Negation nicht erfüllbar ist? Begründen Sie Ihre Antwort.
    \tcblower
    Die Behauptung ist falsch. Ein Gegenbeispiel zu der Aussage ist die (atomare) Formel $p$, sie ist erfüllbar und die Negation $\neg p$ ist ebenfalls erfüllbar.
\end{example}


\begin{example}
    Geben Sie zwei erfüllbare Formeln $F$ und $G$ an, so dass die Formel $F\land G$ nicht erfüllbar ist.
    \tcblower
    Die Formeln $F:=p$ und $G:=\neg p$ sind beide erfüllbar, die Formel $p\land \neg p$ ist jedoch unerfüllbar.
\end{example}

\begin{example}
    Zeigen Sie mit der Methode der Wahrheitstabellen, dass die Formeln $p\to q$ und $q\to p$ nicht äquivalent sind.
    \tcblower
    \begin{center}
            %
            \begin{tabular} {|c|c||c|c|}
                \hline
                $p$ & $q$ & $p\to q$ & $q \to p$ \\
                \hline
                0 & 0 & 1 & 1\\
                0 & 1 & \colorbox{red}{1} & \colorbox{red}{0}\\
                1 & 0 & \colorbox{red}{0} & \colorbox{red}{1}\\
                1 & 1 & 1 & 1\\
                \hline
            \end{tabular}
        \end{center}
\end{example}

\subsubsection{Normalformen}

\begin{example}
Beispiele für Literale: $p$, $\neg q$, $\neg p_{34}$.
\end{example}

\begin{example}
Die Formel
\[
\neg(p\lor q)
\]
ist in keiner der oben eingeführten Normalformen. Die Formel
\[
(\neg p\lor q)\land ((p\land p_1)\lor(p_2\land p_3))
\]
ist in $NNF$ aber weder in $DNF$ noch in $KNF$. Die Formel
\[
p\lor q
\]
ist in $NNF$, $KNF$ und $DNF$.
\end{example}

\begin{example}
Wir bringen die Formel
\[
(\neg p\to q)\to ((p\land p_1)\lor(p_2\land p_3))
\]
in $DNF$. 
\tcblower
Wir eliminieren zuerst alle Implikationen und doppelten Negationen:
\begin{align*}
(\neg p\to q)\textcolor{red}{\to} ((p\land p_1)\lor(p_2\land p_3))\,&\equiv\, \neg(\neg p\textcolor{red}{\to} q)\lor ((p\land p_1)\lor(p_2\land p_3))\\
&\equiv\, \neg(\textcolor{red}{\neg\neg} p \lor q)\lor ((p\land p_1)\lor(p_2\land p_3))\\
&\equiv \neg(p \lor q)\lor ((p\land p_1)\lor(p_2\land p_3)).
\end{align*}
Als Nächstes eliminieren wir alle Negationen, die nicht in Literalen vorkommen:
\begin{align*}
\textcolor{red}{\neg(p \lor q)}\lor ((p\land p_1)\lor(p_2\land p_3))\,&\equiv\,(\neg p\land\neg q)\lor ((p\land p_1)\lor(p_2\land p_3)).
\end{align*}
Die Formel, die wir erhalten haben, ist sowohl in $NNF$ als auch in $DNF$. Wir konstruieren nun noch eine zur Formel
\[
(p\land p_1)\lor(p_2\land p_3)
\]
äquivalente Formel in $KNF$. Wir wenden sukzessive die Distributivgesetze an:
\begin{align*}
(p\land p_1)\textcolor{red}{\lor}(p_2\land p_3)&\,\equiv\,\big( (p\land p_1)\lor p_2)\land (\textcolor{red}{(p\land p_1)\lor p_3})\\
&\equiv\, (\textcolor{red}{(p\land p_1)\lor p_2})\land ((p\lor p_3)\land(p_1\lor p_3))\\
&\equiv\, ((p\lor p_2)\land(p_1\lor p_2))\land ((p\lor p_3)\land(p_1\lor p_3)).
\end{align*}
\end{example}

\begin{example}
Bringen Sie die Formel
\[
(p_1\to p_3)\lor(p_1\land p_2)
\]
in $KNF$ und in $DNF$.
\tcblower
\begin{align*}
(p_1\to p_3)\lor(p_1\land p_2)&\equiv \underbrace{(\neg p_1\lor p_3)\lor (p_1\land p_2)}_{DNF}\\
&\equiv \underbrace{((\neg p_1\lor p_3)\lor p_1)\land ((\neg p_1\lor p_3)\lor p_2)}_{KNF}
\end{align*}
\end{example}

\begin{example}
    Beispielhaft für dieses Vorgehens, bringen wir die Formel
    \begin{align*}
        p_0\to (q\land p_1)
    \end{align*}
    in $DNF$ und $KNF$. 
    \tcblower
    Zuerst erstellen wir eine Wahrheitstabelle von $p_0\to (q\land p_1)$ und markieren zu jeder relevanten Zeile das gewonnene Disjunktionsglied.
    \begin{center}
        \begin{tabular} {| c | c | c || c | c | c |}
            \hline
            $p_0$ & $q$ & $p_1$ & $q\land p_1$ & $p_0\to (q\land p_1)$ & \\
            \hline
            0 & 0 & 0 & 0 & 1 & \textcolor{red}{$\neg p_0 \land \neg q\land \neg p_1$}\\
            0 & 0 & 1 & 0 & 1 & \textcolor{red}{$\neg p_0 \land \neg q\land      p_1$}\\
            0 & 1 & 0 & 0 & 1 & \textcolor{red}{$\neg p_0 \land      q\land \neg p_1$}\\
            0 & 1 & 1 & 1 & 1 & \textcolor{red}{$\neg p_0 \land      q\land      p_1$}\\
            1 & 0 & 0 & 0 & 0 & \textcolor{red}{$-$}\\
            1 & 0 & 1 & 0 & 0 & \textcolor{red}{$-$}\\
            1 & 1 & 0 & 0 & 0 & \textcolor{red}{$-$}\\
            1 & 1 & 1 & 1 & 1 & \textcolor{red}{$     p_0 \land      q\land      p_1$}\\
            \hline
        \end{tabular}
        %
    \end{center}
    \smallskip
    Zusammengefasst ergibt sich die folgende Formel in $DNF$:
    \begin{align*}
       (\neg p_0 \land \neg q\land \neg p_1)
       \lor (\neg p_0 \land \neg q\land      p_1)
       &\lor (\neg p_0 \land      q\land \neg p_1)
       \lor (\neg p_0 \land      q\land      p_1)\\
       &\lor      (p_0 \land      q\land      p_1).
    \end{align*}
    Zum Erstellen einer Formel in KNF, betrachten wir die Wahrheitstabelle der negierten Formel:
    \begin{center}
        \begin{tabular} {| c | c | c || c | c | c | c |}
            \hline
            $p_0$ & $q$ & $p_1$ & $q\land p_1$ & $p_0\to (q\land p_1)$ & $\neg(p_0\to (q\land p_1))$ & \\
            \hline
            0 & 0 & 0 & 0 & 1 & 0 & \textcolor{red}{$-$}\\
            0 & 0 & 1 & 0 & 1 & 0 & \textcolor{red}{$-$}\\
            0 & 1 & 0 & 0 & 1 & 0 & \textcolor{red}{$-$}\\
            0 & 1 & 1 & 1 & 1 & 0 & \textcolor{red}{$-$}\\
            1 & 0 & 0 & 0 & 0 & 1 & \textcolor{red}{$p_0 \land \neg q\land \neg p_1$}\\
            1 & 0 & 1 & 0 & 0 & 1 & \textcolor{red}{$p_0 \land \neg q\land      p_1$}\\
            1 & 1 & 0 & 0 & 0 & 1 & \textcolor{red}{$p_0 \land q\land \neg p_1$}\\
            1 & 1 & 1 & 1 & 1 & 0 & \textcolor{red}{$-$}\\
            \hline
        \end{tabular}
        %
    \end{center}
    \smallskip
    Durch Anwendung der DeMorgan Regeln erhalten wir daraus eine passende Formel in $KNF$:
    \begin{align*}
        \neg&((p_0 \land \neg q\land \neg p_1)\lor
              (p_0 \land \neg q\land p_1)\lor
              (p_0 \land q\land \neg p_1)
            )\\
            &\equiv
              \neg(p_0 \land \neg q\land \neg p_1)\land
              \neg(p_0 \land \neg q\land p_1)\land
              \neg(p_0 \land q\land \neg p_1)\\
            &\equiv
              (\neg p_0 \lor q\lor p_1)\land
              (\neg p_0 \lor q\lor \neg p_1)\land
              (\neg p_0 \lor \neg q\lor p_1)
    \end{align*}
\end{example}