\subsection{Grundlegende Beweistechniken}

\begin{comment}
    Wir wollen im Folgenden einige der elementarsten Standardbeweistechniken besprechen. Natürlich sollen diese Techniken in etwas komplexeren Beweisen auch beliebig kombiniert werden dürfen. Wir könnten beispielsweise zum Beweis einer Äquivalenz die eine Richtung durch Kontraposition und die andere Richtung direkt oder durch Widerspruch beweisen.    
\end{comment}

\begin{howto}{Direkter Beweis einer Implikation}\\
\textbf{Problemstellung:} Es gilt eine Aussage $A\,\Rightarrow \, B$ zu beweisen.\\
\textbf{Lösungsstrategie:} Wir geben, basierend auf der Annahme, dass $A$ wahr ist, \textit{zwingende} Argumente für die Richtigkeit von $B$.
\tcblower
\textbf{Beispiel:} Wir zeigen, wenn $x$ und $y$ gerade (natürliche) Zahlen sind, dann ist auch $x\cdot y$ gerade.

Wir nehmen an $x,y$ seien (irgendwelche) gerade natürliche Zahlen (Voraussetzung). Da $x,y$ gerade sind, gibt es natürliche Zahlen $n_x$ und $n_y$ so, dass
\begin{align*}
x=2\cdot n_x&&y=2\cdot n_y
\end{align*}
gilt. Für das Produkt $x\cdot y$ gilt folglich
\[
x\cdot y \,=\, (2\cdot n_x)\cdot(2\cdot n_y)=2\cdot(n_x\cdot 2\cdot n_y)
\]
und ist somit dass $x\cdot y$ ein vielfaches von $2$ also gerade ist.
\end{howto}

\begin{howto}{Beweis durch Kontraposition}\\
    \textbf{Problemstellung:} Es gilt eine Aussage von der Form $A\Rightarrow B$ zu beweisen.\\
    \textbf{Lösungsstrategie:} Beweisen Sie die Kontraposition $\neg B\Rightarrow\neg A$.
    \tcblower
    \textbf{Beispiel:} ``Für jede natürliche Zahl $n$ gilt: $(n^2+1=1)\Rightarrow (n=0)$''
Ist $n\neq 0$ so folgt, dass auch $n^2\neq 0$ gilt. Dies impliziert, dass für jede weitere natürliche Zahl $m$ die Ungleichung $n^2+m\neq m$ erfüllt ist. Insbesondere gilt daher, dass (der Fall $m=1$) $n^2+1\neq 1$ gilt.
\end{howto}

\begin{howto}{Beweis einer Äquivalenz}\\
 \textbf{Problemstellung:} Es gilt eine Aussage von der Form $A\Leftrightarrow B$ zu beweisen.\\
   \textbf{Lösungsstrategie:} Beweisen Sie $B\Rightarrow A$ sowie $A\Rightarrow B$.
   \tcblower
  \textbf{Beispiel 1:} ``Für jede natürliche Zahl $n$ gilt: $(n^2+1=1)\Leftrightarrow (n=0)$''
Wir haben in den vorhergehenden Beispielen bereits $A\Rightarrow B$ bewiesen, wir müssen also nur noch $B\Rightarrow A$ beweisen. Wir nehmen also $B$ an, es gelte also $n=0$. Draus folgt $n^2=n\cdot n=0\cdot 0=0$ und somit $n^2+1=0+1=1$.
\begin{comment}
    \textbf{Beispiel 2:} ``Für jede natürliche Zahl $n$ gilt: $(n\text{ ist gerade})\Leftrightarrow (n^2\text{ ist gerade}).$''
Wir beweisen zuerst $(n\text{ ist gerade})\Rightarrow (n^2\text{ ist gerade})$. Wir nehmen also an, dass $n$ eine gerade natürliche Zahl ist. Daraus folgt, dass es eine weitere natürliche Zahl $k$ mit $n=2\cdot k$ gibt. Es folgt, dass
\[
n^2=n\cdot n=2\cdot k\cdot 2\cdot k=2\cdot (k\cdot 2\cdot k)
\]
offenbar gerade ist.\\
Nun wollen wir noch die ``Rückrichtung'' $(n\text{ ist gerade})\Leftarrow (n^2\text{ ist gerade})$ beweisen. Wir wollen diese Richtung durch Kontraposition beweisen und nehmen also an, dass $n$ ungerade sei. Es folgt, dass es eine natürliche Zahl $k$ mit $2k+1=n$ gibt. Folglich gilt:
\[
  n^2=(2k+1)(2k+1)=4k^2+4k+1=\underbrace{4(k^2+k)}_{\text{gerade}}+1.
\]
Also ist $n^2$ ungerade.
\end{comment}
\end{howto}



\begin{howto}{Beweis durch Widerspruch}\\
    \textbf{Problemstellung:} Es gilt eine Aussage $A$ zu beweisen.\\
    \textbf{Lösungsstrategie:} Nehmen Sie an, die Aussage $A$ wäre falsch und benützen Sie diese Annahme um einen Widerspruch herzuleiten. Leiten Sie also unter der Annahme der Falschheit von $A$ eine Aussage her von der bereits bekannt ist, dass sie falsch ist oder im Widerspruch zur Annahme steht.
    \tcblower
    \textbf{Beispiel:} $A$:=``Es gibt keine grösste natürliche Zahl''
     Wir nehmen an, dass es eine grösste natürliche Zahl gibt, wir nennen sie $m$. Wir wissen, dass
    für jede natürliche Zahl $n$ gilt, dass einerseits $n+1$ ebenfalls eine natürliche Zahl ist und dass
    andererseits $n<n+1$ erfüllt ist. Wir wenden dies auf die natürliche Zahl $m$ an und erhalten
    damit eine grössere natürliche Zahl (nämlich $m+1$). Dies steht jedoch im
    Widerspruch zu unserer ursprünglichen Annahme, dass $m$ die grösste natürliche Zahl sei.
\end{howto}

\begin{howto}{Beweis durch (Gegen-) Beispiel}\\
 \textbf{Problemstellung:} Es gilt zu zeigen, dass eine bestimmte Eigenschaft nicht auf alle Objekte (aus einem Kontext) zutrifft.\\
   \textbf{Lösungsstrategie:} Geben Sie konkret ein Objekt an, welches die erwähnte Eigenschaft nicht besitzt.
   \tcblower
  \textbf{Beispiel:} ``Nicht jede natürliche Zahl ist eine Quadratzahl\footnote{Von der Form $x^2$ für eine geeignete natürliche Zahl $x$.}.''

Weil die Funktion $f(x)=x^2$ monoton ist (später mehr dazu) und weil $1\cdot1<2<2\cdot2$ gilt, kann die Zahl $2$ nicht als Quadrat von einer natürlichen Zahl geschrieben werden. Somit ist $2$ das (oder ein) gesuchte Gegenbeispiel.
\end{howto}



