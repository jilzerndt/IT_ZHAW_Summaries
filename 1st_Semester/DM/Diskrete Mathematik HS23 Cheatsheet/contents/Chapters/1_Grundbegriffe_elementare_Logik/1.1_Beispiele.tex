\subsubsection{Prädikate}

\begin{example}
 Beispiel für eine Aussage mit Wahrheitswert: ``$3+4=106$'' (falsch)
\end{example}

\begin{example}
  Beispiele für Prädikate:
  \begin{enumerate}
  \item $P(p):= $\,``$p$ ist eine Primzahl.''
  \item $T(x):= $\,``$x$ ist eine durch $21$ teilbare ganze Zahl.''
  \item $G(r):= $\,``$r>0$''
  \item $Q(x,y):= $\,``$x^2+14x-15=y$''
  \end{enumerate}
  Die Aussagen
  \begin{align*}
  &T(42)& 	    &P(7)&		&Q(2,17)&\\
  &T(357)&  &P(2)& 	&Q(1,0)&
  \end{align*}
  sind alle wahr. Deshalb können wir, entsprechend der vorhergehenden Bemerkung, z.B. ``$T$ trifft auf $42$'' zu oder auch ``$7$ hat die Eigenschaft $P$'' sagen.
\end{example}

\begin{example}
  Weitere Beispiele für Prädikate:
  \begin{itemize}
    \item $A(x,y) := x + y < 10$ ist ein zweistelliges Pädikat mit den freien Variablen $x$ und $y$.
    \item $B(x,y,z) := x + y < z$ ist ein dreistelliges Prädikat.
    \item Wenn wir die Variable $z$ in $B$ mit dem Wert $10$ belegen, dann erhalten wir das zweistellige Prädikat $B(x,y,10)$, welches gleichbedeutend mit dem Prädikat $A$ ist.
  \end{itemize}
\end{example}

\subsubsection{Junktoren}

\begin{example2}{Kontraposition}
Wir können die eben aufgestellten Rechenregeln dazu verwenden um wiederum neue Tatsachen abzuleiten. Unter anderem folgt daraus das sogenannte Prinzip der \textit{Kontraposition}. Dieses Prinzip besagt, dass $A\Rightarrow B$ äquivalent ist zu $\neg B\Rightarrow\neg A$. Wollen wir dies nun mit unseren Rechenregeln nachvollziehen, so beginnen wir mit $A\Rightarrow B$ und wenden nacheinander verschiedene Regeln an um schlussendlich $\neg B\Rightarrow \neg A$ zu erhalten:
\begin{align*}
                     &A\Rightarrow B\\
   \Leftrightarrow\, &\neg A\lor B                  &(\text{Definition von }A\Rightarrow B)\\
   \Leftrightarrow\, &B\lor \neg A                  &(\text{Kommutativität})\\
   \Leftrightarrow\, &\neg\neg B\lor\neg A          &(\text{Doppelte Negation})\\
   \Leftrightarrow\, &\neg B\Rightarrow \neg A      &(\text{Definition von }\neg B\Rightarrow\neg A)
\end{align*}
\end{example2}

\subsubsection{Quantoren}

\begin{example}
 Einige quantifizierte Aussagen mit ihren Wahrheitswerten:
\begin{enumerate}
 \item Es sei $S$ die Menge aller Schweine und $R(x)$ das Prädikat ``$x$ ist rosa''. Es gilt
\[
 \text{``}\exists x\in S\; R(x)\text{''}\Leftrightarrow\text{``es gibt rosa Schweine''}.
\]
Diese Aussage ist offensichtlich wahr. Wenn wir nun die Allquantifizierung betrachten, so erhalten wir
\[
 \text{``}\forall x\in S\; R(x)\text{''}\Leftrightarrow\text{``alle Schweine sind rosa''}.
\]
Dies ist eine falsche Aussage, da etwa Wildschweine einerseits Elemente von $S$ sind aber andererseits $R$ nicht erfüllen, da sie nicht rosa sind.
\item Wir wollen nun die Aussage
\[
A:=``\,\text{alle Informatiker können programmieren''}
\]
mit Quantoren ausdrucken.
Wir definieren dazu zuerst das Prädikat
\[
B(x):=``x\text{ kann programmieren''}.
\]
Wir haben nun zwei mögliche Vorgehensweisen. Einerseits können wir die Menge $I$ aller Informatiker betrachten und kommen dann mittels der Aussage
\[
 \forall x\in I\;B(x)
\]
zum Ziel. Andererseits können wir auch $A$ umformulieren als ``alles was ein Informatiker ist kann programmieren'' und erhalten die gewünschte Aussage mit einem uneingeschränkten Quantor
\[
 \forall x(x\in I\Rightarrow B(x)).
\]
Diesen Zusammenhang zwischen eingeschränkten und uneingeschränkten Quantoren werden wir in der nächsten Bemerkung zu ``Rechenregeln für Quantoren'' allgemein formulieren.
\end{enumerate}
\end{example}

\begin{example}
  Einige geläufige Notationen und Abkürzungen im Zusammenhang mit Quantoren sind:
  \begin{itemize}
  \item $\forall x,y\,(\dots)$ als Abkürzung von $\forall x\,\forall y\, (\dots)$ und $\exists x,y\,(\dots)$ als Abkürzung für $\exists x\,\exists y\,(\dots)$. Entsprechende Abkürzungen gelten auch für drei oder mehr quantifizierte Variablen.
  \item $\exists ! x\,(\dots)$ für ``es gibt \textbf{genau} ein $x$ mit $\dots$''.
  \item $\nexists x\,(\dots)$ für $\neg\exists x\,(\dots)$.
  \item $\forall x<y\,(\dots)$ für $\forall x\,(x<y\Rightarrow \dots)$. Diese Notation wird auch für andere ähnliche Relationen wie $ >,\leq, \geq, \subseteq, $ usw. verwendet.
\end{itemize}
\end{example}

\begin{example}
 Mit den Rechenregeln für Quantoren und den Rechenregeln für Junktoren können wir wieder neue Tatsachen (=Wahrheitswerte neuer Aussagen) herleiten. Als Beispiel betrachten wir das Duale zur Vertauschungsregel für unbeschränkte Quantoren, nämlich:
\[
 \exists xA(x)\Leftrightarrow \neg \forall x\neg A(x)
\]
Wir beginnen also mit $\exists xA(x)$ und erhalten durch Anwenden der Rechenregeln $\neg \forall x\neg A(x)$.
\begin{align*}
 &\exists xA(x)\\
\Leftrightarrow&\neg\neg\exists xA(x)&(\text{Doppelte Negation}) \\
\Leftrightarrow&\neg(\neg\exists x A(x))\\
\Leftrightarrow&\neg(\neg\exists x \neg(\neg A(x)))&(\text{Doppelte Negation})\\
\Leftrightarrow&\neg(\forall x\neg A(x))&(\text{Vertauschungsregel})
\end{align*}
\end{example}

\begin{example}
Gegeben sind die Aussagen $A$ und $B$:
\begin{enumerate}
\item[] $A$ :=``Alle Hasen haben lange Ohren.''
\item[] $B$ := ``Es gibt Hasen mit kurzen Beinen.''
\end{enumerate}
\tcblower
Es gilt:
\begin{enumerate}
 \item $\neg A$ entspricht ``Es gibt mindestens einen Hasen, der keine langen Ohren hat.''
\item $A\wedge \neg B$ entspricht ``Alle Hasen haben lange Ohren und keine kurzen Beine.''
\item $A\Rightarrow B$ entspricht ``Wenn alle Hasen lange Ohren haben, dann gibt es Hasen mit kurzen Beinen.''
\end{enumerate}
\end{example}

\begin{example}
Negieren Sie umgangssprachlich folgende Aussagen (so präzise wie möglich).
\begin{enumerate}
\item Alle Autos haben vier Räder.
\item Zwillinge haben stets die identische Haarfarbe.
\item Es gibt flugunfähige Vögel.
\item Alle Dinosaurier sind ausgestorben.
\end{enumerate}
\tcblower
	\begin{enumerate}
	\item Es gibt ein Auto, das nicht vier Räder hat.
  \\ \textbf{Anmerkung:} Es kann auch mehrere solche Autos geben.
	\item Es gibt ein Zwillingspaar mit verschiedenen Haarfarben.
	\item Alle Vögel sind flugfähig.
	\item Mindestens ein Dinosaurier lebt noch.
	\end{enumerate}
\end{example}

\begin{example}
  Es seien $P(x)$ ein einstelliges und $Q(y,z)$ ein zweistelliges Prädikat. Formalisieren Sie:
  \begin{enumerate}
      \item Es gibt genau ein $x$ mit $P(x)$.
      \item Es gibt mindestens zwei Dinge mit der Eigenschaft $P$.
      \item Es gibt höchstens ein $x$ mit $P(x)$.
      \item Wenn $P(x)$ und $P(y)$ gilt, dann gilt stets auch $Q(x,y)$.
      \item Für kein $x$ gilt $Q(x,x)$.
    \end{enumerate}
      \tcblower
      Mögliche Lösungen:
      \begin{enumerate}
        \item $\exists x\,(P(x))\,\land\, \forall y,z\, (P(y)\land P(z)\,\Rightarrow\, y=z)$
        \item $\exists x,y\,(P(x)\land P(y)\land x\neq y)$
        \item $\neg \exists x,y\,(P(x)\land P(y)\land x\neq y)$
        \item $\forall x,y\,(P(x)\land P(y)\,\Rightarrow Q(x,y))$
        \item $\forall x\,\neg Q(x,x)$
      \end{enumerate}
  \end{example}

\begin{example}
Geben Sie Prädikate $P(x)$ und $Q(x)$ an, so dass $\forall x\,P(x)\,\lor\,\forall x\,Q(x)$ falsch, aber $\forall x\,(Q(x)\,\lor\, P(x))$ wahr ist.
\tcblower
Zum Beispiel:
\[
P(x):= x>10
\]
und
\[
Q(x):= x<11.
\]
Die Aussage $\forall x\, P(x)\lor \forall x\,Q(x)$ bedeutet, dass jede Zahl grösser als Zehn ist oder, dass jede Zahl kleiner als $11$ ist, diese Aussage ist falsch. Die Aussage $\forall x\,(P(x)\lor Q(x))$ besagt hingegen wahrheitsgemäss, dass jede Zahl (für sich selbst betrachtet) entweder kleiner als $11$ oder grösser als $10$ ist.
\end{example}

\begin{example}
Gruppieren Sie folgende Aussagen so, dass in jeder Gruppe alle Aussagen äquivalent sind und keine äquivalenten Aussagen in verschiedenen Gruppen sind.
\begin{itemize}
\item[1.] $\forall x\,(P(x)\Rightarrow Q(x))$
\item[2.] $\exists x\,(P(x)\Leftrightarrow Q(x))$
\item[3.] $\forall x\,(Q(x)\Rightarrow P(x))$
\item[4.] $\forall x\,(\neg P(x)\Rightarrow \neg Q(x))$
\item[5.] $\forall x\,(\neg Q(x)\Rightarrow \neg P(x))$
\item[6.] $\neg\exists x\,(\neg \neg Q(x)\land\neg P(x) )$
\item[7.] $\neg\exists x\,(P(x)\land\neg Q(x))$
\item[8.] $\exists x\, (P(x)\land Q(x))\lor \exists x (\neg P(x)\land\neg Q(x))$
\item[9.] $\forall x\,\exists y\,(P(x)\land P(y))$
\end{itemize}
\tcblower
Die Aussagen $1.,\,5.,\,7.$ und $3.,\,4.,\,6.$ und $2.,8.$ sind jeweils untereinander äquivalent. Es gibt keine weiteren Äquivalenzen.
\end{example}