\subsection{Aussagen, Prädikate und Quantoren}
\begin{definition}{Aussage}
    `sprachliches Gebilde'' oder Ausdruck verstehen, welchem ein Wahrheitswert ``wahr'' oder ``falsch'' zugeordnet werden kann
\end{definition}
\begin{remark}
            Die Zeichenfolge ``$:=$'' steht für ``ist definiert als'' oder ``ist per Definition gleich''.
\end{remark}
\begin{comment}
\begin{remark}
  Obwohl nach Definition jede Aussage einen eindeutigen Wahrheitswert besitzt, bedeutet dies nicht, dass dieser bekannt sein muss. Der Satz ``es gibt unendlich viele Primzahlen'' war beispielsweise bereits eine Aussage, bevor man wusste, dass er wahr ist.
\end{remark}
\end{comment}



\begin{definition}{Freie Variable}
Wir sagen, dass $x$ als eine freie Variable in einem Ausdruck $A$ vorkommt, falls $x$ einen reinen Platzhalter darstellt (steht weder für einen noch für eine Menge von konkreten Werten). Beispiele: ``$x<3$'' oder ``$x$ ist ein Tisch'' 
\\ Im Gegensatz dazu kommt $x$ in ``alle $x$, die durch $4$ teilbar sind, sind gerade'' nicht frei vor, weil in dieser Aussage die Gesamtheit (Menge) aller möglichen Belegungen von $x$ betrachtet wird. 
\end{definition}

%\begin{comment}
\begin{remark}
In einem Ausdruck können beliebig viele Variablen frei vorkommen und wir schreiben $A(x,y,z,\dots)$, um anzuzeigen, dass in einem Ausdruck $A$ die Variablen $x,y,z,\dots$ frei vorkommen.
\end{remark}
%\end{comment}

\begin{definition}{$n$-stelliges Prädikat}
Es sei $n$ eine natürliche Zahl. Ein Ausdruck, in dem $n$ viele Variablen frei vorkommen und der bei Belegung aller freien Variablen in eine Aussage übergeht, nennen wir ein $n$-stelliges Prädikat. Aussagen sind $0$-stellige Prädikate.
\end{definition}

\begin{comment}
\begin{remark}
  Ist $A(x)$ ein Prädikat und ist $y$ ein mathematisches Objekt (z.B. $y=17$) so, dass $A(y)$ eine wahre Aussage ist, dann sagen wir, dass das Prädikat (manchmal auch die Eigenschaft) $A$ auf $y$ zutrifft. Das Prädikat $x>100$ trifft zum Beispiel auf die Zahl $232$ zu, weil $232>100$ eine wahre Aussage ist.
\end{remark}
\end{comment}

\subsubsection{Junktoren}

\begin{concept}{Junktoren}
Es seien $A$ und $B$ beliebige Prädikate. Wir führen folgende abkürzende Schreibweisen ein:
\begin{itemize}
\item $\neg A$ (gesprochen: Nicht $A$) ist das Prädikat, welches (für jede Belegung) genau dann wahr ist, wenn $A$ falsch ist.
 \item $A\wedge B$ (gesprochen: $A$ und $B$)  ist das Prädikat, welches (für jede Belegung) genau dann wahr ist, wenn sowohl $A$ als auch $B$ wahr sind.
\item $A\vee B$ (gesprochen: $A$ oder $B$)  ist das Prädikat, welches (für jede Belegung) genau dann wahr ist, wenn $A$ wahr ist oder $B$ wahr ist (oder beide wahr sind).
\item $A\Rightarrow B$ (gesprochen: $A$ impliziert $B$) ist das Prädikat, welches (für jede Belegung) genau dann wahr ist, wenn $\neg A\vee B$ wahr ist.
\item $A\Leftrightarrow B$ (gesprochen: $A$ äquivalent $B$)  ist das Prädikat, welches (für jede Belegung) genau dann wahr ist, wenn $A\Rightarrow B$ und $B\Rightarrow A$ wahr sind.
\end{itemize}
Die Zeichen $\neg,\Rightarrow,\wedge$ und $\vee$ nennen wir \textit{Junktoren}.
\end{concept}

\begin{comment}
\begin{remark}
Das Prädikat $A\Rightarrow B$ besagt, dass in jedem Fall in dem $A$ wahr ist auch $B$ wahr sein muss.
Die Äquivalenz zweier Prädikate besagt also, dass diese stets denselben Wahrheitswert haben. Umgangssprachlich wird oft vorausgesetzt, dass zwischen den Prädikaten $A$ und $B$ ein ``inhaltlicher Zusammenhang'' bestehen muss, damit $A\Rightarrow B$ gelten kann. Dies ist in der mathematischen Logik nicht der Fall. Die Aussagen
\begin{align*}
\textit{Es gibt Einhörner}\Rightarrow 8\textit{ ist eine Primzahl}
\end{align*}
und
\begin{align*}
\textit{Spinat ist grün}\Rightarrow 2\textit{ ist eine Primzahl}
\end{align*}
sind beispielsweise beide (mathematisch gesehen) wahr.
\end{remark}
\end{comment}




\begin{lemma}{Junktorenregeln}
 Seien $A,B$ und $C$ beliebige Aussagen. Es gelten folgende Äquivalenzen:\\ \\
 \begin{tabular}{lc}
    Doppelte Negation:  & $\neg\neg A\Leftrightarrow A$\\
    Kommutativität:     & $ A\wedge B\Leftrightarrow B\wedge A$\\
                        & $A\vee B\Leftrightarrow B\vee A$\\
    Assoziativität:     & $(A\wedge B)\wedge C\Leftrightarrow A\wedge (B\wedge C)$\\
                        & $(A\vee B)\vee C\Leftrightarrow A\vee (B\vee C)$\\
    Distributivität:    & $A\wedge (B\vee C)\Leftrightarrow (A\wedge B)\vee (A\wedge C)$\\
                        & $A\vee (B\wedge C)\Leftrightarrow (A\vee B)\wedge (A\vee C)$\\
    De Morgan:          & $\neg(A\wedge B)\Leftrightarrow\neg A\vee\neg B$\\
                        & $\neg(A\vee B)\Leftrightarrow \neg A\wedge\neg B$\\
    Kontraposition:     & $A\Rightarrow B\Leftrightarrow \neg B\Rightarrow \neg A$
 \end{tabular}
\end{lemma}


\subsubsection{Quantoren}
\begin{comment}
Quantoren sind Symbole  anhand derer wir aus Prädikaten neue Prädikate oder Aussagen gewinnen
können. Wir betrachten das Beispiel des Prädikates
\[
 A(x):=\text{``}x\text{ ist eine Primzahl und } x\text{ ist ein Teiler von }24\text{''}
\]
und die Aussage
\[
 B:= ``\text{es gibt eine Primzahl welche ein Teiler von }24\text{ ist''}
\]
mit anderen Worten,
\[
 B:=``\text{es \textbf{existiert} ein }x\text{ mit }A(x)\text{''}.
\]
Wir sagen, dass $B$ aus $A(x)$ durch existenzielle Quantifizierung über $x$ entsteht.

Andererseits können wir aus dem Prädikat $A(x)$ aber auch die (offensichtlich falsche) Aussage
\[
 C:=\text{``alle Zahlen sind Primzahlen und ein Teiler von }24\text{''}
\]
konstruieren. Diese ist gleichbedeutend mit
\[
 C:=\text{''\textbf{alle} Zahlen }x\text{ erfüllen }A(x).
\]
Wir sagen, dass $C$ aus $A(x)$ durch universelle Quantifizierung entsteht\footnote{Obwohl in den Aussagen $B$ und
$C$ formal die Variable $x$ vorkommt, steht sie nicht als Platzhalter für ein einzusetzendes Objekt, sondern
``läuft'' über die Gesamtheit aller möglichen Objekte. Wir sagen, dass die Variable nicht frei sondern durch
einen Quantor gebunden ist.}.
\end{comment}

\begin{concept}{Quantoren}
Es sei $M$ eine Menge. Ist $A(x)$ ein Prädikat, dann können wie folgt neue Prädikate geformt werden:
\begin{itemize}
\item $\forall x\,A(x)$ (gesprochen: Für alle $x$ gilt $A(x)$) trifft genau dann zu, wenn $A$ auf jedes (mathematische) Objekt zutrifft.
%\item $\forall x\in M\,A(x)$ (gesprochen: Für alle $x$ aus $M$ gilt $A(x)$) trifft genau dann zu, wenn $A$ auf jedes Element aus $M$ zutrifft.
\item $\exists x\,A(x)$ (gesprochen: Es gibt ein $x$ mit $A(x)$) trifft genau dann zu, wenn es (mindestens) ein  Objekt gibt, auf welches $A$ zutrifft.
%\item $\exists x\in M\,A(x)$ (gesprochen: Es gibt ein $x$ aus $M$ mit $A(x)$) trifft genau dann zu, wenn es (mindestens) ein Element aus $M$ gibt, auf welches $A$ zutrifft.
\end{itemize}
Die Symbole $\forall$ und $\exists$ heissen \textit{Allquantor} und \textit{Existenzquantor}.
\end{concept}

\begin{comment}
\begin{remark}
In mathematischen Texten werden Prädikate von der Form $\exists x\,A(x)$ oft als ``es gibt \textbf{mindestens} ein $x$ mit $A(x)$'' ausgedrückt. Diese Ausdrucksform ist inhaltliche gleichbedeutend mit ``es gibt ein $x$ mit $A(x)$''. Auch in diesem Text werden wir beide sprechweisen synonym verwenden.
\end{remark}
\end{comment}

\begin{remark}
Ein $n$-stelliges Prädikat wird durch Quantifizierung (einer freien Variable) zu einem neuen $n-1$ stelligen Prädikat.
\end{remark}

\begin{lemma}{Quantorenregeln}
 Ist $A(x)$ ein Prädikat und $K$ eine Menge, so gelten folgende Äquivalenzen:
\begin{enumerate}
 \item Vertauschungsregel für unbeschränkte Quantoren
\[
 \forall x\, A(x)\Leftrightarrow \neg\exists x\,\neg A(x)
\]
\item Vertauschungsregel für beschränkte Quantoren
\[
 \forall x\in K \;A(x)\Leftrightarrow \neg\exists x\in K\;\neg A(x)
\]
\item Beschränkter und unbeschränkter Allquantor
\[
 \forall x\in K\;A(x)\Leftrightarrow \forall x(x\in K\Rightarrow A(x))
\]
\item Beschränkter und unbeschränkter Existenzquantor
\[
\exists x\in K\; A(x)\Leftrightarrow \exists x(x\in K\wedge A(x))
\]
\end{enumerate}
\end{lemma}

\begin{comment}
\begin{remark}
 Wir haben keine Distributionsregel mit Quantoren und Junktoren. Die Äquivalenzen
\[
\forall x\, A(x)\,\lor\,\forall x B(x)\,\Leftrightarrow\, \forall x\,(A(x)\,\lor\, B(x))
\]
und
\[
 \exists x A(x)\wedge\exists x B(x)\Leftrightarrow \exists x (A(x)\wedge B(x))
\]
gelten im Allgemeinen \textbf{nicht}. 

Wir betrachten dazu als Gegenbeispiel die Aussagen
\[
A(x):=``x \text{ ist eine gerade natürliche Zahl''}
\]
und
\[
B(x):=``x\text{ ist eine ungerade natürliche Zahl''}.
\]
Die Aussage
\[
 \exists x A(x)\wedge\exists x B(x)
\]
besagt also in diesem Fall, dass es mindestens eine gerade natürliche Zahl gibt und dass es ebenfalls mindestens eine ungerade natürliche Zahl gibt. Diese Aussage ist offensichtlich wahr. Die Aussage
\[
  \exists x (A(x)\wedge B(x))
\]
besagt nun aber, dass es eine natürliche Zahl gibt, welche ``gleichzeitig'' gerade und ungerade ist, was offensichtlich falsch ist. Die beiden Aussagen sind also nicht äquivalent.

\end{remark}
\end{comment}



