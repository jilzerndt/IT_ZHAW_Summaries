\section{Elementare Zahlentheorie}

\begin{lemma}{Rechenregeln auf $\Z$}
    Für alle $r,s,z\in\Z$ gelten folgende Gleichungen.
    \begin{align*}
        -1\cdot z&=-z\\
        -(-z)&=z\\
        -z+z&=0 &\text{ Inverse Elemente bezüglich }+\\
        0\cdot z&=0 &\text{ Absorbtion}\\
        1\cdot z&=z &\text{ Neutrales Element bezüglich }\cdot\\
        0+z&=z &\text{ Neutrales Element bezüglich }+\\
        r(sz)&=(rs)z &\text{ Assoziativität von } \cdot\\
        r+(s+z)&=(r+s)+z &\text{ Assoziativität von }+\\
        rs&=sr &\text{ Kommutativität von }\cdot\\
        r+s&=s+r &\text{ Kommutativität von }+\\
        r(s+z)&=rs+rz &\text{ Distributivität}\\
        rx=ry&\Rightarrow x=y\lor r=0&\text{Kürzbarkeit}
    \end{align*}
\end{lemma}

\subsection{Teilbarkeit und Euklidischer Algorithmus}

\begin{definition}{Teilbarkeit}
    Sind $x,y\in\mathbb{Z}$ ganze Zahlen, so sagen wir, dass $x$ \textit{ein Teiler von} $y$ ist, falls es ein $k\in\mathbb{Z}$ gibt mit $xk=y$. Wir schreiben in diesem Fall $x|y$. Es gilt also
    \[
        x|y:\Leftrightarrow \exists k\in\Z(y=xk).
    \]
    Mit $T(y)$ bezeichnen wir die Menge aller natürlichen Zahlen, welche Teiler von $y$ sind, also $T(y)=\{x\in\N\mid x|y\}$.
\end{definition}

\begin{example}
    \begin{enumerate}
        \item Die Zahl $1$ ist ein Teiler jeder ganzen Zahl $z$, da $1\cdot z=z$.
        \item $T(0)=\N$.
    \end{enumerate}
\end{example}

\begin{remark}
    Die Teilbarkeitsrelation ist reflexiv und transitiv auf der Menge $\Z$, auf der Menge $\N$ ist die Teilbarkeitsrelation sogar eine Halbordnung.
\end{remark}

\begin{proof}{Eigenschaften Teilbarkeitsrelation}
    Wir zeigen, dass die Teilbarkeitsrelation reflexiv, transitiv und für natürliche Zahlen auch antisymmetrisch ist.
    \begin{itemize}
        \item Reflexivität: Dies gilt, da jede ganze Zahl sich selbst teilt.
        \item Transitivität: Seien $x,y,z$ ganze Zahlen. Aus $x|y$ und $y|z$ folgt, dass es ganze Zahlen $k_1,k_2$ gibt mit $x\cdot k_1=y$ und $y\cdot k_2=z$. Es folgt
            \[
                x\cdot(k_1\cdot k_2)=(x\cdot k_1)\cdot k_2=y\cdot k_2=z.
            \]
            Somit existiert eine ganze Zahl $k$ (nämlich $k=k_1\cdot k_2$) mit $k\cdot x=z$, also gilt $x|z$ wie gewünscht.\qedhere
        \item Antisymmetrie auf $\N$: Wir müssen zeigen, dass für natürliche Zahlen $x$ und $y$ aus $x|y$ und $y|x$ folgt, dass $x=y$ gilt. Es gelte also $xk=y$ und $x=yr$ für ganze Zahlen $k,r$. Es folgt
            \[
                x=yr=(xk)r=x(kr)
            \]
            und $kr=1$. Daraus ergeben sich zwei mögliche Fälle; $k=r=1$ oder $k=r=-1$. Im Fall $k=r=-1$ folgt $x=-y$, was im Widerspruch dazu steht, dass $x$ und $y$ natürliche Zahlen sind. Es bleibt also nur der Fall $k=r=1$ und somit, wie gewünscht, $x=y$.
    \end{itemize}
\end{proof}

\begin{remark}
    Sind $x,y\in\Z$ und gilt $x\cdot y=1$ so gilt $|x|=|y|=1$.
\end{remark}

\begin{lemma}{Teilen mit Rest}
    Sind $n,m\in\N\backslash\{0\}$, dann gibt es eindeutig bestimmte Zahlen $k,r\in\N$, so dass Folgendes gilt:
    \begin{enumerate}
        \item $m=kn+r$
        \item $r<n$
    \end{enumerate}
    Wir sagen in diesem Zusammenhang, dass die Zahl $r$ den \textit{Rest} von der (ganzzahligen) Division von $m$ durch $n$ ist.
\end{lemma}

\begin{definition}{Kleistes gemeinsames Vielfaches}
    Seien $n,m\in\Z$. Wir definieren das \textit{kleinste gemeinsame Vielfache von $n$ und $m$} als
    \[
        kgV(n,m):=\min\{k\in\N\mid n|k\wedge m|k\}.
    \]
    Ist $n\neq0$ oder $ m\neq 0$, dann definieren wir den \textit{grössten gemeinsamen Teiler} von $n$ und $m$ als
    \[
        ggT(n,m):=\max\{k\in\N\mid k|n\wedge k|m\}.
    \]
\end{definition}

\begin{lemma}{Grösster gemeinsamer Teiler}
    Sind $x,y,z\in\Z$, dann sind folgende Aussagen äquivalent:
    \begin{enumerate}
        \item[1.] $ x|y\wedge x|z$
        \item[2.] $x|y\wedge x|(y-z) $
    \end{enumerate}
\end{lemma}

\begin{proof}{ggT}
    $1.\Rightarrow 2.$: Wenn $x|y\wedge x|z$, dann gibt es ganze Zahlen $k,k'\in\Z$, so dass $y=kx$ und $z=k'x$. Es gilt also $y-z=kx-k'x=(k-k')x$.

    $2.\Rightarrow 1.$: Es seien $k,k'\in\Z$, so dass $y=kx$ und $y-z=k'x$. Durch Einsetzen erhält man $ kx-z=k'x $ und somit $z=kx-k'x=x(k-k')$.
\end{proof}

\begin{lemma}{Euklidischer Algorithmus}
    Für $n,m\in\N$ mit $0<n< m$ gilt
    \[
        ggT(n,m)=ggT(n,m-n)=ggT(m,m-n).
    \]
\end{lemma}

\begin{definition}{Teilerfremd}
    Zwei ganze Zahlen $x,y$ heissen \textit{teilerfremd}, wenn $ggT(x,y)=1$ gilt.
\end{definition}

\begin{theorem}{Lemma von Bézout}
    Sind $x,y\in\Z$ mit $x,y\neq 0$, dann gibt es ganze Zahlen $a,b$ so dass
    \[
        ggT(x,y)=ax+by
    \]
    gilt. Die Zahlen $a$ und $b$ werden Bézout Koeffizienten genannt.
\end{theorem}

\subsection{Primzahlen}

\begin{definition}{Primzahl}
    Eine natürliche Zahl $p\in\N$ ist eine \textit{Primzahl}, wenn $|T(p)|=2$ gilt. Die Menge aller Primzahlen bezeichnen wir mit $\mathbb{P}$.
\end{definition}

\begin{remark}
    Ist $p$ eine Primzahl, dann gilt $T(p)=\{1,p\}$.
\end{remark}

\begin{proof}{Primzahl}
    Für jede Zahl $n\in\N$ gilt offensichtlich $n\in T(n)$ und $1\in T(n)$. Bei Primzahlen kommt dazu, dass (wegen $|T(n)|=2$) keine weiteren Teiler existieren.
\end{proof}

\begin{lemma}{Lemma von Euklid}
    Folgende Aussagen sind für $p\in\N$ mit $p\neq 1$ äquivalent:
    \begin{enumerate}
        \item[1.] $\forall n,m\in\N\,\big(p|nm\Rightarrow p|n\vee p|m\big)$
        \item[2.] $p\in\mathbb{P}$
    \end{enumerate}
\end{lemma}

\begin{lemma}{Primteiler}
    Jede ganze Zahl $z$ mit $z\notin\{-1,1\}$ besitzt einen \textit{Primfaktor} (einen Teiler, der eine Primzahl ist). Formal können wir dies als
    \[
        \forall z\in\Z\,\big(z\notin\{-1,1\}\Rightarrow T(z)\cap\P\neq\emptyset\big)
    \]
    ausdrücken.
\end{lemma}

\begin{theorem}{Unendlich viele Primzahlen}
    Es gibt unendlich viele Primzahlen.
\end{theorem}

\begin{proof}{Unendlich viele Primzahlen}
    Wir machen einen Widerspruchsbeweis. Wir nehmen an, dass es nur endlich viele Primzahlen $\P=\{p_1,..,p_n\}$ gibt. Nach dem Primteiler-Lemma gibt es eine Primzahl $p_i$ so, dass
    \[
        p_i\,|\,(\prod_{j=1}^np_j)+1.
    \]
    Es gibt also eine natürliche Zahl $k$ so, dass
    \[
        p_i\cdot k=(\prod_{j=1}^np_j)+1
    \]
    gilt. Daraus folgt
    \begin{align*}
        1=p_i\cdot k-(\prod_{j=i}^np_j)&=p_i\cdot k-(p_1\cdot..\cdot p_i\cdot..\cdot p_n)\\
        &=p_i\cdot k-p_i(\underbrace{p_1\cdot..\cdot p_{i-1}\cdot p_{i+1}\cdot..\cdot p_n}_{:=p})\\
        &=p_i(k-p).
    \end{align*}
    Es folgt also, dass $p_i$ ein Teiler von $1$ ist, das steht aber im Widerspruch zu $p_i\in\P$.
\end{proof}

\begin{theorem}{Primfaktoren}
    Jede natürliche Zahl grösser als $1$ ist das Produkt von endlich vielen Primzahlen.
\end{theorem}

\begin{theorem}{Primfaktorzerlegung}
    Es sei $p_i$ jeweils die $i$-te Primzahl. Für jede natürliche Zahl $n>1$ gibt es eine eindeutig bestimmte, endliche Folge $a_1,..,a_k$ von natürlichen Zahlen mit $a_k\neq 0$, so dass
    \[
        n=\prod_{i=1}^k p_i^{a_i}
    \]
    gilt.
\end{theorem}

\subsection{Modulare Arithmetik}

\begin{definition}{Modulo}
    Es sei $n\in\N$ beliebig. Wir definieren eine Relation $\equiv_n$ auf $\Z$ wie folgt:
    \[
        r\equiv_n s:\Leftrightarrow n|(r-s).
    \]
    Gilt für $r,s\in Z$ die Relation $r\equiv_ns$, dann sagen wir, dass $r$ gleich $s$ modulo $n$ ist und schreiben $r=s \:mod\, n$.
\end{definition}

\begin{remark}
    Die Relation $\equiv_n$ ist für jede natürliche Zahl $n$ eine Äquivalenzrelation auf $\Z$.
\end{remark}

\begin{remark}
    Es sei $n\in\N$ beliebig. Für je zwei ganze Zahlen $x$ und $y$ gilt $x\modn y$ genau dann, wenn $x$ und $y$ denselben Rest bei Division durch $n$ lassen.
\end{remark}

\begin{corollary}{Restklassen}
    Es sei $n\in\N$ beliebig. Jede ganze Zahl $z$ steht mit genau einer natürlichen Zahl aus $\{0,..n-1\}$ in der Relation $\equiv_n$.
\end{corollary}

\begin{definition}{Restklasse}
    Es sei $n\in\N$ beliebig. Für jede ganze Zahl $z$ bezeichnen wir mit
    \[
        [z]_n:=\{x\in\Z\mid x\modn z\}
    \]
    die Äquivalenzklasse von $z$ bezüglich der Relation $\modn$ und nennen diese auch die \textit{Restklasse} von $z$. Abkürzend bezeichnen wir $[z]_n$ auch mit $\bar k$, wenn $k\in\{0,..,n-1\}$ und $z\modn k$ gilt.
\end{definition}

\begin{corollary}{Darstellung Restklassen}
    Es sei $n\in\N$ beliebig. Es gilt
    \[
        [z]_n=\{z+yn\mid y\in\Z\}=\{....z-3n,z-2n,z-n,z,z+n,z+2n,z+3n,..\}.
    \]
\end{corollary}

\begin{remark}
    Es sei $n\in\N$ beliebig. Für ganze Zahlen $x,x'$ und $y,y'$ gelten:
    \begin{enumerate}
        \item $[x]=[x']\land [y]=[y']\Rightarrow [x+y]=[x'+y']$
        \item $[x]=[x']\land [y]=[y']\Rightarrow [xy]=[x'y']$
    \end{enumerate}
\end{remark}

\begin{definition}{Menge aller Restklassen}
    Es sei $n\in\N$ beliebig. Die Menge aller Restklassen von $\Z$ modulo $n$ bezeichnen wir mit
    \[
        \Z/n=\{[z]_n\mid z\in\Z\}=\{\bar k\mid 0\leq k<n-1\wedge z\modn k\}=\{\bar 0,\bar1,\bar2,..,\overline{n-1}\}.
    \]
    Wir definieren zwei Verknüpfungen $\cdot:(\Z/n)^2\rightarrow \Z/n$ und $+:(\Z/n)^2\rightarrow \Z/n$ durch die Zuordnungen
    \[
        [x]_n+[y]_n:=[x+y]_n
    \]
    und
    \[
        [x]_n\cdot[y]_n:=[xy]_n.
    \]
\end{definition}

\begin{theorem}{Primzahlen und Restklassen}
    Es sei $n\in\N\backslash\{1\}$ beliebig. Folgende Aussagen sind äquivalent:
    \begin{enumerate}
        \item[1.] $n$ ist eine Primzahl.
        \item[2.] Für jedes $\bar k\in\Z/n$ mit $\bar k\neq\bar 0$ gibt es genau ein $r\in\{0,..,n-1\}$ mit $\bar k\cdot\bar r=\bar 1$.
    \end{enumerate}
    Die zweite Aussage besagt, dass man in $\Z/n$ Gleichungen von der Form $ax=b$ stets nach $x$ auflösen kann. Sind $\bar k,\bar r\in\Z/n$ mit $\bar k\cdot\bar r=\bar 1$, so sagen wir $\bar r$ sei invers (bezüglich der Multiplikation) zu $\bar k$ und schreiben auch $(\bar{k})^{-1}$ für $\bar r$.
\end{theorem}

\subsubsection{Chinesischer Restsatz}

\begin{howto}{Lösen simultaner Kongruenzen}
    Wir wollen ein System simultaner Kongruenzen mit zwei Gleichungen lösen, etwa
    \begin{align*}
        x&\equiv_{n_1} y_1\\
        x&\equiv_{n_2} y_2
    \end{align*}
    mit $n_1$ und $n_2$ teilerfremd. Wir gehen schrittweise wie folgt vor:
    \begin{enumerate}
        \item Durch sukzessives Teilen mit Rest erhalten wir ganze Zahlen $a,b$ mit $an_1+bn_2=1$.
        \item Wir setzen $x:=y_1bn_2+y_2an_1$.
    \end{enumerate}
\end{howto}

\begin{lemma}{Fermat}
    Ist $a\in\Z/n$ mit $n>0$ invertierbar, dann ist die Funktion
    \begin{align*}
        f&:\Z/p\to\Z/p\\
        f&(x)=\bar a\cdot x
    \end{align*}
    surjektiv.
\end{lemma}

\begin{lemma}{Kleiner Fermat}
    Ist $p\in\P$ und $a$ kein Vielfaches von $p$, dann gilt
    \[
        a^{p-1}\equiv_p1.
    \]
\end{lemma}
