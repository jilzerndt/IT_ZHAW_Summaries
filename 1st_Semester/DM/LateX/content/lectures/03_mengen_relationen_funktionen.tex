\section{Mengen, Relationen und Funktionen}

\subsection{Der Mengenbegriff und grundlegende Definitionen}

\begin{concept}{Notation Mengen}
    Ist $X$ eine Menge und $y$ ein \textit{Element} von $X$, dann schreiben wir $y\in X$. Ist $y$ kein Element von $X$, dann schreiben wir $y\notin X$.
\end{concept}

\begin{remark}
    Die erste \textit{definierende Eigenschaft} von Mengen ist die Tatsache, dass jede Menge durch ihre Elemente vollständig beschrieben ist.
\end{remark}

\begin{definition}{Definierende Eigenschaft}
    Zwei Mengen sind genau dann gleich, wenn sie dieselben Elemente enthalten: Es gilt für alle Mengen $X$ und $Y$ die Äquivalenz
    \[
        X=Y\,\Leftrightarrow\,\forall z\, (z\in X\Leftrightarrow z\in Y).
    \]
\end{definition}

\begin{concept}{Explizite Schreibweise}
    Sind mathematische Objekte $x_1,\dots,x_n$ gegeben, dann schreiben wir
    \[
        \{x_1,\dots,x_n\}
    \]
    für die Menge die als Elemente genau $x_1,\dots,x_n$ hat.
\end{concept}

\begin{example}
    \begin{itemize}
        \item Die Menge $\{2,34,77\}$ enthält die drei Elemente $2$, $34$ und $77$.
        \item Die Menge $\{\,\}$ heisst \textit{leere Menge}. Die leere Menge ist die einzige Menge, die gar keine Elemente besitzt, sie wird mit $\varnothing $ bezeichnet.
    \end{itemize}
\end{example}

\begin{remark}
    Wenn keine Missverständnisse zu befürchten sind, so beschreibt man Mengen auch durch ``angedeutete'' Aufzählung ihrer Elemente. Die Menge $\mathbb{N}$ der \textit{natürlichen Zahlen} wird beispielsweise durch
    \[
        \N:=\{0,1,2,\dots\}
    \]
    beschrieben.
    Die Menge der \textit{ganzen Zahlen} wird durch
    \[
        \Z:=\{\dots, -2,-1,0,1,2\dots\}
    \]
    beschrieben.
\end{remark}

\begin{remark}
    Die Tatsache, dass Mengen durch ihre Elemente eindeutig beschrieben werden hat zur Folge, dass Mengen sehr ``unstrukturierte Datentypen'' sind, d.h. Mengen haben keine ``innere Ordnung''. Es gelten unter anderem:
    \begin{itemize}
        \item Für beliebige $z,x_1,\dots,x_n$
            \[
                z\in\{x_1,\dots x_n\}\,\Leftrightarrow\, z=x_1\lor\dots\lor z=x_n
            \]
        \item Für alle $x$
            \[
                \{x\}=\{x,x\}=\{x,x,x\}=\dots
            \]
        \item Für alle $x,y$
            \[
                \{x,y\}=\{y,x\}.
            \]
    \end{itemize}
\end{remark}

\begin{concept}{Prädikative Schreibweise}
    Ist $X$ eine Menge und ist $\mathsf{E}$ eine Eigenschaft (Prädikat), dann bezeichnen wir mit
    \[
        \big\{z\in X\mid \mathsf{E}(z)\big\}
    \]
    oder mit
    \[
        \big\{z\mid z\in X\land\mathsf{E}(z)\big\}
    \]
    die Menge aller Elemente $z$ von $X$ mit der Eigenschaft $\mathsf{E}(z)$.
\end{concept}

\begin{example}
    Die Menge aller geraden natürlichen Zahlen erhält man auch durch die prädikative Schreibweise,
    \begin{itemize}
        \item $\{n\in\N\mid n\text{ ist gerade}\}$
        \item $\{n\in\N\mid \exists z\in\N\,(n=2\cdot z)\}$
    \end{itemize}
\end{example}

\begin{concept}{Ersetzungsschreibweise}
    Ist $F$ eine Funktion und ist $X$ eine Menge, dann beinhaltet die Menge
    \[
        \big\{F(x)\mid x\in X \big\}
    \]
    alle Funktionswerte $F(x)$, die man dadurch erhalten kann, dass man ein Element $x\in X$ in $F$ einsetzt:
    \[
        \big\{F(x)\mid x\in X\big\}:=\{y\mid \exists x\in X\,(y=F(x))\}.
    \]
\end{concept}

\begin{remark}
    Ist eine Funktion $f$ und eine Menge von der Form
    \begin{align*}
        X=\{x_1,x_2,x_3,\dots\}
    \end{align*}
    gegeben, dann entspricht die Menge $\{f(x)\mid x\in X\}$ der Menge
    \begin{align*}
        \{f(x_1),f(x_2),f(x_3),\dots\}.
    \end{align*}
\end{remark}

\begin{example}
    Die Menge der geraden natürlichen Zahlen lässt sich nun mithilfe der Funktion $F(x)=2\cdot x$ als
    \[
        \{F(x)\mid x\in\N\} = \{2x\mid x\in\N\}
    \]
    schreiben.
\end{example}

\begin{definition}{Teilmengen}
    Wir schreiben $X\subseteq Y$ und sagen $X$ ist eine \textit{Teilmenge} von $Y$, wenn jedes Element von $X$ auch ein Element von $Y$ ist:
    \[
        X\subseteq Y:\,\Leftrightarrow\,\forall x\,(x\in X\Rightarrow x\in Y).
    \]
    Wir schreiben $X\subsetneq Y$ und sagen $X$ ist eine \textit{echte Teilmenge} von $Y$, falls $X$ eine von $Y$ verschiedene Teilmenge von $Y$ ist:
    \[
        X\subsetneq Y\,:\Leftrightarrow\, X\subseteq Y\land X\neq Y.
    \]
\end{definition}

\begin{example}
    \begin{itemize}
        \item Die Menge aller Hühner ist eine (echte) Teilmenge der Menge aller Vögel, weil alle Hühner Vögel sind (und weil es Vögel gibt die keine Hühner sind).
        \item Die Menge aller Primzahlen ist eine (echte) Teilmenge von $\N$.
        \item Die Menge aller Primzahlen ist \textit{keine} Teilmenge aller ungeraden Zahlen, weil die Zahl $2$ eine Primzahl aber keine ungerade Zahl ist.
    \end{itemize}
\end{example}

\begin{lemma}{Äquivalenz}\\
    Zwei Mengen $X$ und $Y$ sind gleich, wenn $X\subseteq Y$ und $Y\subseteq X$ gilt.
\end{lemma}

\begin{definition}{Potenzmenge}
    Ist $A$ eine beliebige Menge, dann bezeichnen wir mit
    \[
        \mathcal{P}(A):=\{x\mid x\subseteq A\}
    \]
    die \textit{Potenzmenge} von $A$, die genau die Teilmengen von $A$ als Elemente enthält.
\end{definition}

\begin{definition}{Schnitt- und Vereinigungsmenge}\\
    Sind $X$ und $Y$ Mengen, dann ist
    \[
        X\cup Y:=\{x\mid x\in X\lor x\in Y \}
    \]
    die \textit{Vereinigung} von $X$ mit $Y$. Die \textit{Schnittmenge} von $X$ und $Y$ ist durch
    \[
        X\cap Y:=\{x\in X\mid x\in Y \}=\{x\in Y\mid x\in X\}=\{x\mid x\in X\land x\in Y\}
    \]
    gegeben. Ist $I$ eine Menge so, dass für alle Elemente $i\in I$ auch $A_i$ eine Menge ist, und $I\neq\varnothing$, dann wird die Vereinigung resp. Schnittmenge von $\{A_i\mid i\in I\}$ genannt.
    \begin{align*}
        \bigcup_{i\in I}A_i:=\{x\mid\exists i\in I\,(x\in A_i) \} & & \bigcap_{i\in I}A_i:=\{x\mid\forall i\in I\,(x\in A_i) \}
    \end{align*}
\end{definition}

\begin{example}
    \begin{enumerate}
        \item $\N=\{n\in \N\mid n\text{ ist gerade}\}\cup \{n\in \N\mid n\text{ ist ungerade}\}$
        \item $\varnothing=\{n\in \N\mid n\text{ ist gerade}\}\cap \{n\in \N\mid n\text{ ist ungerade}\}$
        \item Sind $X_a$ und $X_b$ beliebige Mengen, dann gilt:
            \[
                X_a\cup X_b=\bigcup_{i\in\{a,b\}}X_i.
            \]
        \item Ist für jede natürliche Zahl $n$ die Menge $X_n$ als $\{0,\dots, n\}$ gegeben, dann gilt
            \[
                \bigcup_{n\in \N}X_n\,=\,\N
            \]
            und
            \[
                \bigcap_{n\in \N}X_n\,=\,\{0\}.
            \]
    \end{enumerate}
\end{example}

\begin{definition}{Komplementärmenge}\\
    Sind $X$ und $Y$ beliebige Mengen, so definieren wir als
    \[
        X\setminus Y:=\{x\in X\mid x\notin Y\}
    \]
    die Menge aller Elemente von $X$, die nicht zu $Y$ gehören. Die Menge $X\setminus Y$ nennt man ``$X$ ohne $Y$''. Ist eine ``Grundmenge'' $A$ (implizit oder explizit) vorgegeben, so bezeichnet man die Menge $A\setminus Y$ auch als ``Komplement'' oder ``Komplementärmenge'' von $X$ (relativ zu $A$).
\end{definition}

\begin{lemma}{Rechenregeln}
    Es gelten für beliebige Mengen $A,B$ und $C$ folgende Identitäten: (gleich wie Junktorenregeln)
    \begin{center}
        $A\cap A=A\text{ und }A\cup A=A$\\
        $A\cup B=B\cup A\text{ und }A\cap B=B\cap A.$\\
        $A\cap(B\cap C)=(A\cap B)\cap C\text{ und }A\cup(B\cup C)=(A\cup B)\cup C$\\
        $A\cap(B\cup C)=(A\cap B)\cup (A\cap C)\text{ und }A\cup(B\cap C)=(A\cup B)\cap (A\cup C)$\\
        $(C\backslash A)\cap (C\backslash B)=C\backslash (A\cup B)\text{ und }(C\backslash A)\cup (C\backslash B)=C\backslash (A\cap B)$
    \end{center}
    Charakterisierung der Teilmengenbeziehung:
    \[
        A\subseteq B\Leftrightarrow A\cap B= A\Leftrightarrow A\cup B=B
    \]
\end{lemma}

\begin{definition}{Disjunkte Mengen}
    Zwei Mengen $X$ und $Y$ heissen \textit{disjunkt}, falls sie keine gemeinsamen Elemente haben, d.h. falls $X\cap Y=\varnothing$ gilt. Wir sagen eine Menge $\{X_i\mid i\in I \}$ von Mengen bestehe aus \textit{paarweise disjunkten} Mengen, wenn folgendes gilt:
    \[
        \forall i,j\in I\,(i\neq j\Rightarrow X_i\cap X_j=\varnothing).
    \]
\end{definition}

\begin{definition}{Partitionen}
    Eine \textit{Partition} $P=\{P_i\mid i\in I \}$ einer Menge $A$, ist eine Menge von Teilmengen von $A$, die folgende beiden Voraussetzungen erfüllt:
    \begin{itemize}
        \item Die Elemente von $P$ sind nichtleer und paarweise disjunkt.
        \item $\bigcup_{i\in I}P_i=A$
    \end{itemize}
    Die Elemente einer Partition werden \textit{Blöcke} der Partition genannt.
\end{definition}

\subsection{Funktionen}

\begin{definition}{Tupel}
    Es sei $n>0$ eine natürliche Zahl. Ein $n$\textit{-Tupel} ist ein Term von der Form
    \[
        (x_1,\dots,x_n).
    \]
    Für beliebige Tupel gilt:
    \[
        (x_1,\dots,x_n)=(y_1,\dots y_k):\Leftrightarrow n=k\land x_1=y_1\land\dots\land y_n=x_n.
    \]
    $2$-Tupel nennen wir \textit{Paare} und $3$-Tupel \textit{Tripel}.\\
    \textit{Tupel} haben im Gegensatz zu Mengen mehr innere Struktur - die Reihenfolge und Wiederholung von Elementen sind wesentlich, sie sind gewissermassen die mathematische Entsprechung zu Listen und Arrays in der Informatik.
\end{definition}

\begin{definition}{Kartesisches Produkt}
    Die Gesamtheit aller Tupel mit Elementen aus einer oder mehreren gegebenen Mengen nennt man kartesisches Produkt.\\
    Es seien $A_1,\dots, A_n$ Mengen und $n\in\N$ mit $n>0$.
    Das \textit{kartesische Produkt} von $A_1,\dots, A_n$, ist die Menge aller $n$-Tupel mit Einträgen aus den Mengen $A_1,\dots ,A_n$:
    \begin{align*}
        \prod_{i=1}^{n}A_i=\big\{(a_1,\dots,a_n)\mid a_1\in A_1\land\dots\land a_n\in A_n \big\}.
    \end{align*}
\end{definition}

\begin{remark}
    Wir schreiben auch $A_1\times A_2\times \dots\times A_n$ für $\prod_{i=1}^nA_i$.\\ Insbesondere schreiben wir $X\times Y$ für das kartesische Produkt von zwei Mengen $X$ und $Y$, konkret heisst das:
    \[
        X\times Y:=\{(x,y)\mid x\in X\land y\in Y \}.
    \]
    Für das $n$-fache kartesisches Produkt $A\times A\times\dots\times A$ einer Menge $A$ mit sich selbst schreiben wir auch $A^n$.
\end{remark}

\begin{example}
    Die Menge der rationalen Zahlen
    \[
        \Q:=\left\{\frac{x}{y}\mid x\in \Z\land y\in\N\setminus\{0\}\right\}
    \]
    kann man als das kartesische Produkt
    \[
        \Z\times(\N\setminus\{0\})
    \]
    auffassen.
\end{example}

\begin{definition}{Relation}
    Eine $n$-stellige \textit{Relation} $R$ auf den Mengen $A_1,\dots A_n$ ist eine Menge von $n$-Tupeln aus $A_1\times\dots \times A_n$. Mit anderen Worten, die Relationen auf $A_1,\dots,A_n$ sind genau die Teilmengen
    \begin{align*}
        R\subseteq A_1\times\dots \times A_n.
    \end{align*}
    Ist $R$ eine $n$-stellige Relation und gilt $(x_1,\dots,x_n)\in R$, dann sagen wir, dass die Elemente $x_1,\dots,x_n$ zueinander in Relation $R$ stehen.
    \tcblower
    Eine $2$-stellige Relation $R\subseteq X\times Y$ heisst auch eine \textit{binäre Relation} auf den Mengen $X$ und $Y$. Ist $R$ eine binäre Relation, so schreiben wir auch $xRy$ für $(x,y)\in R$.
\end{definition}

\begin{definition}{Funktion}
    Es seien $A$ und $B$ beliebige Mengen. Eine Relation $f\subseteq A\times B$ ist eine \textit{Funktion} von $A$ nach $B$, falls:
    \begin{align*}
        \forall x\in A\exists!y\in B((x,y)\in f)
    \end{align*}
    gilt. In diesem Fall schreiben wir $f:A\to B.$
\end{definition}

\begin{concept}{Notation Funktionen}
    Im Kontext einer Funktion $f:A\to B$ verwenden wir folgende Schreibweisen und Konventionen:
    \begin{itemize}
        \item Da zu jedem $x\in A$ ein eindeutig bestimmtes Element $y\in B$ mit $(x,y)\in f$ existiert, kann dieses $y$ mit $f(x)$ bezeichnet und \textit{Funktionswert von $f$ bei $x$} genannt werden.
        \item Die Menge aller Funktionswerte $Im(f) := \{f(x)\mid x\in A \}$ wird als \textit{Bild(menge)} von $f$ bezeichnet.
        \item Die Menge $A$ nennen wir den Definitionsbereich von $f$ und schreiben dafür auch $Dom(f)$.
        \item Der Definitionsbereich ist eindeutig durch die Funktion gegeben:
            \begin{align*}
                A=Dom(f)=\{x\mid \exists y ((x,y)\in f) \}=\{x\mid \exists y (f(x)=y )\}
            \end{align*}
        \item Die Menge $B$ ist durch die Voraussetzung $f:A\to B$ nicht eindeutig bestimmt, tatsächlich gilt $f:A\to B$ für jede Menge $B$ mit $Im(f)\subseteq B$.
    \end{itemize}
\end{concept}

\begin{definition}{Injektiv}
    Eine Funktion $f$ ist genau dann \textit{injektiv}, wenn die Relation
    \begin{align*}
        f^{-1}=\{(y,x)\mid (x,y)\in f\}
    \end{align*}
    eine Funktion ist.\\
    Umgangssprachlich: Jeder Output kann nur mittels einem einzigen Inputelement erreicht werden.
\end{definition}

\begin{definition}{Umkehrfunktion}\\
    Ist $f:A\to B$ eine injektive Funktion, dann nennt man $f^{-1}:Im(f)\to A$ die \textit{Umkehrfunktion} oder \textit{inverse Funktion} von $f$.
\end{definition}

\begin{lemma}{Äquivalenzen zur Injektivität}\\
    Für $f:A \to B$ sind folgende Aussagen äquivalent.
    \begin{enumerate}
        \item Die Funktion $f$ ist injektiv
        \item Für alle $x,y\in A$ gilt: Aus $x\neq y$ folgt $f(x)\neq f(y)$
        \item Für alle $x,y\in A$ gilt: Aus $f(x)=f(y)$ folgt $x=y$
    \end{enumerate}
\end{lemma}

\begin{proof}{Injektivität zeigen}
    Die Aussagen in b) und c) sind offensichtlich äquivalent (Kontraposition). Für die Äquivalenz von $a)$ und $c)$ sei $f$ injektiv. Die Relation $f^{-1}=\{(y,x)\mid (x,y)\in f\}$ sei also eine Funktion. Daraus folgt, dass zu jedem $y$ höchstens ein $x$ mit $(y,x)\in f^{-1}$ existiert. Formal heisst das:
    \begin{align*}
        (y,x)\in f^{-1}\land (y,x')\in f^{-1}\Rightarrow x=x'
    \end{align*}
    Dies ist gleichbedeutend mit
    \begin{align*}
        (x,y)\in f\land (x',y)\in f\Rightarrow x=x'
    \end{align*}
    und somit
    \begin{align*}
        f(x)=y\land f(x')=y\Rightarrow x=x'
    \end{align*}
    was genau der Aussage in c) entspricht.
\end{proof}

\begin{definition}{Surjektiv}
    Eine Funktion $f:A\to B$ heisst \textit{surjektiv} auf $B$, wenn $B=Im(f)$.\\
    Umgangssprachlich: Realisiert jedes Element einer gegebenen Zielmenge als Funktionswert.
\end{definition}

\begin{definition}{Bijektiv}
    Ist die Funktion $f$ injektiv und surjektiv, so sagen wir $f:A\to B$ sei \textit{bijektiv}.
\end{definition}

\begin{definition}{Komposition}
    Sind $f:A\to B$ und $g:B\to C$ Funktionen, dann ist die Komposition $g$ nach $f$ durch
    \begin{align*}
        &g\circ f:A\to C\\
        (&g\circ f)(x)=g(f(x))
    \end{align*}
    gegeben.
\end{definition}

\begin{lemma}{Regeln der Komposition}\\
    Für beliebige Funktionen $f:X\to Y$ und $g:Y\to Z$ gelten folgende Aussagen:
    \begin{enumerate}
        \item Falls $f:X\to Y$ und $g:Y\to Z$ injektiv sind, dann ist auch $g\circ f:X\to Z$ injektiv.
        \item Falls $f:X\to Y$ und $g:Y\to Z$ surjektiv sind, dann ist auch $g\circ f:X\to Z$ surjektiv.
    \end{enumerate}
\end{lemma}

\begin{proof}{Injektivität und Surjektivität der Komposition zeigen}
    \begin{enumerate}
        \item Wir nehmen an, dass $f:X\to Y$ und $g:Y\to Z$ injektiv sind und zeigen, dass $g\circ f:X\to Z$ injektiv ist. Es seien $a,b\in X$ verschiedene Elemente. Weil $f$ injektiv ist, folgt $f(a)\neq f(b)$ und folglich aus der Injektivität von $g$, wie gewünscht
            \begin{align*}
                g\circ f(a) = g(f(a))\neq g(f(b))=g\circ f(b).
            \end{align*}
        \item Für die zweite Behauptung müssen wir zeigen, dass zu jedem $z\in Z$ ein $x\in X$ existiert mit $g(f(x))= z$. Es sei also $z\in Z$ beliebig. Weil $g:Y\to Z$ surjektiv ist, gibt es ein $y\in Y$ mit $g(y)=z$. Weil $f:X\to Y$ ebenfalls surjektiv ist, gibt es weiter ein $x\in X$ mit $f(x) = y$. Insgesamt haben wir wie gewünscht
            \begin{align*}
                g(f(x))=g(y)=z.
            \end{align*}
    \end{enumerate}
\end{proof}

\subsection{Grössenvergleiche von unendlichen Mengen}

\begin{definition}{Endlichkeit und Abzählbarkeit}
    \begin{itemize}
        \item Eine Menge $X$ heisst \textit{endlich}, falls $X=\varnothing$ oder eine natürliche Zahl $n\geq 1$ und eine bijektive Funktion $f:X \to \{1,\dots,n\}$ existieren.
            Ist $X\neq\varnothing$ eine endliche Menge, dann existiert eine Darstellung der Form $X=\{x_1,x_2,\dots,x_n\}$ wobei die Elemente $x_i$ paarweise verschieden sind (d.h. es gilt $i\neq j\Rightarrow x_i\neq x_j$). In diesem Fall hat die Menge $X$ genau $n$ viele Elemente und wir schreiben $|X|=n$. Weiter schreiben wir $|\varnothing| = 0$.
        \item Nicht endliche Mengen nennen wir \textit{unendlich}.
        \item Eine Menge $X$ heisst \textit{abzählbar}, wenn eine surjektive Funktion $F:\N\to X$ existiert oder wenn $X=\varnothing$ gilt.
        \item Die Menge $X$ heisst \textit{abzählbar unendlich}, wenn $X$ abzählbar und unendlich ist.
        \item Eine \textit{überabzählbare} Menge ist eine Menge, die nicht abzählbar ist.
    \end{itemize}
\end{definition}

\begin{lemma}{Schubfachprinzip}
    Wenn $n$ Objekte auf $m$ Behälter verteilt werden und $n>m$ gilt, dann gibt es mindestens einen Behälter, der mehr als ein Objekt enthält. Formal, sind $n>m$ natürliche Zahlen und gelte $|X|= n$ sowie $|Y|=m$, dann gibt es keine injektive Funktion
    \begin{align*}
        F: X\to Y.
    \end{align*}
\end{lemma}

\begin{lemma}{Injektive Abbildung der natürlichen Zahlen}\\
    Gibt es eine injektive Funktion $F:\N\to A$, dann ist die Menge $A$ unendlich.
\end{lemma}

\begin{proof}{Unendlichkeit zeigen}
    Es sei eine Menge $A$ und eine injektive Funktion $F:\N\to A$ gegeben. Wäre die Menge $A$ endlich, dann gäbe es eine natürliche Zahl $n$ mit $|A|=n$. Die Funktion
    \begin{align*}
        G&:\{0,\dots,n\}\to A\\
        G&(x) = F(x)
    \end{align*}
    wäre injektiv und würde, wegen $|\{0,\dots,n\}|=n+1$, dem Schubfachprinzip widersprechen.
\end{proof}

\begin{lemma}{Abzählbare Mengen}
    Folgende Aussagen sind für unendliche Mengen $A$ äquivalent:
    \begin{enumerate}
        \item Die Menge $A$ ist abzählbar.
        \item Es gibt eine surjektive Funktion $F_{\N,A}:\N\to A$.
        \item Es gibt eine injektive Funktion $F_{A,\N}:A\to\N$.
        \item Es gibt eine bijektive Funktion $B_{\N,A}:\N\to A$.
        \item Es gibt eine bijektive Funktion $B_{A,\N}:A\to\N$.
    \end{enumerate}
\end{lemma}

\begin{lemma}{Endliche Mengen}
    Jede endliche Menge ist abzählbar.
\end{lemma}

\begin{lemma}{Teilmengen}
    Jede Teilmenge einer abzählbaren Menge ist abzählbar.
\end{lemma}

\begin{lemma}{Transitivität}
    Ist $X$ eine abzählbare Menge und gibt es eine surjektive Funktion $F:X\to Y$, dann ist auch $Y$ abzählbar.
\end{lemma}

\begin{lemma}{Erstes Diagonalargument Cantor}
    Die Menge $\N\times\N$, bestehend aus allen Paaren von natürlichen Zahlen, ist abzählbar.
\end{lemma}

\begin{lemma}{Vereinigung}
    Jede Vereinigung von abzählbar vielen abzählbaren Mengen ist abzählbar. Konkret, jede Vereinigung von der Form
    \[
        \bigcup_{i\in\N}A_i
    \]
    ist abzählbar, wenn alle $A_i$'s abzählbar sind.
\end{lemma}

\begin{corollary}{Kartesisches Produkt}
    Die Menge $\Z\times \Z$ ist abzählbar.
\end{corollary}

\begin{corollary}{Rationale Zahlen}
    Die Menge $\Q=\big\{\frac{x}{y}\mid x,y\in \Z\big\}$ der rationalen Zahlen (Brüche) ist abzählbar.
\end{corollary}

\begin{theorem}{Zweites Diagonalargument Cantor}
    Die Menge aller unendlichen Binärsequenzen (Sequenzen aus Nullen und Einsen) ist überabzählbar.
\end{theorem}

\begin{corollary}{Intervall}
    Das Intervall $(0,1)=\{r\in\R\mid 0<r<1 \}$ ist überabzählbar. Insbesondere ist die Menge $\R$ der reellen Zahlen überabzählbar.
\end{corollary}

\begin{corollary}{Potenzmenge}
    Die Potenzmenge von $\N$ ist überabzählbar.
\end{corollary}

\begin{corollary}{Menge aller Funktionen}
    Die Menge aller Funktionen $F:\N\to\N$ ist überabzählbar.
\end{corollary}

\begin{corollary}{Unberechenbare Funktionen}
    Es gibt Funktionen $F:\N\to\N$, die von keinem Java, C, C++, Fortran\dots Programm berechenbar sind. Solche Funktionen heissen \textit{unberechenbar}.
\end{corollary}

\subsection{Relationen}

\begin{definition}{Gerichteter Graph}
    Ein \textit{(gerichteter) Graph} ist ein Paar $G=(V,E)$ bestehend aus einer Menge $V$ (Knotenmenge)
    und einer binären Relation $E\subseteq V\times V$ (Kantenmenge).
\end{definition}

\begin{definition}{Eigenschaften von Relationen}
    Eine binäre Relation $R$ auf einer Menge $X$ heisst:
    \begin{itemize}
        \item \textit{Reflexiv}, wenn für alle $x\in X$
            \[
                xRx
            \]

        \item \textit{Symmetrisch}, wenn für alle $x,y\in X$
            \[
                xRy\,\Rightarrow\, yRx
            \]

        \item \textit{Antisymmetrisch}, wenn für alle $x,y\in X$
            \[
                xRy\land yRx\,\Rightarrow x=y
            \]

        \item \textit{Transitiv}, wenn für alle $x,y,z\in X$
            \[
                xRy\land yRz\,\Rightarrow \, xRz
            \]

    \end{itemize}
\end{definition}

\subsubsection{Äquivalenzrelationen}

Äquivalenzrelationen sind in einem gewissen Sinn verallgemeinerte Gleichheitsrelationen. Sie werden dazu verwendet, (im Sinn der Relation) ähnliche Objekte miteinander zu identifizieren und als ``gleich'' zu behandeln.

\begin{definition}{Äquivalenzrelation}
    \textit{Äquivalenzrelationen} sind reflexive, symmetrische und transitive Relationen.
\end{definition}

\begin{definition}{Äquivalenzklasse}
    Es sei $R$ eine Äquivalenzrelation auf einer Menge $X$ und $x\in X$. Die \textit{Äquivalenzklasse} $[x]_R$ von $x$ bezüglich $R$ ist die Menge aller Elemente von $X$, die zu $x$ in Relation $R$ stehen:
    \[
        [x]_R:=\{y\in X\mid xRy \}
    \]
    Jedes Element einer Äquivalenzklasse nennen wir einen \textit{Repräsentanten} der entsprechenden Äquivalenzklasse. Die \textit{Faktormenge} $\faktor{X}{R}$ \textit{von} $X$ \textit{modulo} $R$ ist die Menge aller Äquivalenzklassen:
    \[
        \faktor{X}{R}:=\big\{ [x]_R\mid x\in X \big\}
    \]
\end{definition}

\begin{lemma}{Äquivalente Elemente}
    Ist $\sim $ eine Äquivalenzrelation auf einer Menge $X$ und gilt $x,y\in X$ mit $x\sim y$, dann gilt $[x]_\sim=[y]_\sim$. Mit anderen Worten, äquivalente Elemente repräsentieren stets dieselbe Äquivalenzklasse.
\end{lemma}

\begin{corollary}{Elemente als Repräsentanten}
    Ist $\sim $ eine Äquivalenzrelation auf $X$ und sind $x,y\in X$ mit $x\in[y]_\sim$, dann gilt $[x]_\sim=[y]_\sim$. Mit anderen Worten, jedes Element einer Äquivalenzklasse ist auch ein Repräsentant dieser Äquivalenzklasse.
\end{corollary}

\begin{lemma}{Disjunktheit Equivalenzklassen}
    Ist $\sim $ eine Äquivalenzrelation auf $X$ und sind $x,y\in X$ mit $[x]_\sim\neq[y]_\sim$, dann gilt $[x]_\sim\cap[y]_\sim=\varnothing$.
    Mit anderen Worten, verschiedene Äquivalenzklassen sind immer disjunkt.
\end{lemma}

\begin{lemma}{Equivalenzklassen Partition}
    Ist $\sim$ eine Äquivalenzrelation auf einer Menge $X$, dann ist die Faktormenge $\faktor{X}{\sim}$ eine Partition von $X$.
\end{lemma}

\begin{lemma}{Partition induziert Äquivalenzrelation}
    Ist $P=\{A_i\mid i\in I\}$ eine Partition von der Menge $X$, dann ist die Relation $\sim$, gegeben durch
    \[
        x\sim y:\Leftrightarrow \exists i\in I\,(x\in A_i\land y\in A_i),
    \]
    eine Äquivalenzrelation auf $X$. Zusätzlich gilt
    \[
        \faktor{X}{\sim}=P.
    \]
\end{lemma}

\begin{lemma}{Äquivalenzrelationen: Verallgemeinerte Gleichheit}
    Für jede Relation $\sim$ auf einer Menge $X$ sind folgende beiden Aussagen äquivalent.
    \begin{enumerate}
        \item[1.] Die Relation $\sim$ ist eine Äquivalenzrelation.
        \item[2.] Es gibt eine Menge $Y$ und ein Funktion $F:X\to Y$ so, dass für alle $x,y\in X$
            \[
                x\sim y\Leftrightarrow F(x)=F(y)
            \]
            gilt.
    \end{enumerate}
\end{lemma}

\subsubsection{Ordnungsrelationen}

\begin{definition}{Minimale Elemente}
    Es sei $R$ eine binäre Relation auf der Menge $M$.
    \begin{itemize}
        \item Zwei Elemente $x,y\in M$ heissen $R$-\textit{unvergleichbar}, falls weder $xRy$ noch $yRx$ gilt.
        \item Ein Element $x\in X$ einer Teilmenge $X\subseteq M$ von $M$ heisst $R$-\textit{minimal in $X$}, falls es kein anderes Element $y\in X$ mit $yRx$ gibt.
        \item  Ein Element $x\in X$ einer Teilmenge $X\subseteq M$ von $M$ heisst $R$-\textit{maximal in $X$}, falls es kein anderes Element $y\in X$ mit $xRy$ gibt.
    \end{itemize}
    Wenn keine Missverständnisse zu befürchten sind, dann schreiben wir anstelle von $R$-minimal, $R$-maximal und $R$-unvergleichbar auch einfach minimal, maximal und unvergleichbar.
\end{definition}

\begin{definition}{Ordnungsrelation}
    Es sei $R$ eine binäre Relation auf der Menge $M$.
    \begin{itemize}
        \item $R$ ist eine \textit{Präordnung} auf $M$, wenn $R$ reflexiv und transitiv ist.
        \item $R$ ist eine \textit{Halbordnung} auf $M$, wenn $R$ reflexiv, antisymmetrisch und transitiv ist.
        \item $R$ ist eine \textit{totale oder lineare Ordnung} auf $M$, wenn $R$ eine Halbordnung ist und keine $R$-unvergleichbaren Elemente existieren.
        \item $R$ ist eine \textit{Wohlordnung} auf $M$, wenn $R$ eine totale Ordnung auf $M$ ist so, dass jede Teilmenge $X\neq\varnothing$ von $M$ (mindestens) ein $R$-minimales Element enthält.
    \end{itemize}
\end{definition}

\begin{example}
    \begin{itemize}
        \item Die Relation $\leq$ auf der Menge $\R$ ist eine totale Ordnung, die aber keine Wohlordnung ist (die Menge $\{x\in\R\mid 0<x<1\}$ hat kein kleinstes Element). Auf der Menge $\N$ ist $\leq$ eine Wohlordnung. Auf der Menge $\Z$ ist die Relation $\leq$ keine Wohlordnung. Wieso?
        \item Ist $A$ eine Menge von Mengen, dann ist die Teilmengenrelation $\subseteq$ eine Halbordnung.
        \item Die Teilerrelation $T$ auf der Menge $\Z$ ist eine Halbordnung aber keine totale Ordnung. Die Elemente $7$ und $5$ sind $T$-unvergleichlich.
    \end{itemize}
\end{example}

\begin{definition}{Transitiver Abschluss}
    Es sei $R$ eine (bin\"are) Relation.
    \begin{itemize}
        \item Als \textit{transitiven Abschluss} von $R$ bezeichnet man die kleinste
            (bezüglich $\subseteq$) transitive Relation, die $R$ als Teilmenge enthält,
            sie wird mit $R^+$ notiert.
        \item Die kleinste Relation, die $R^+$ enthält und reflexiv ist, nennt man den
            \textit{reflexiv-transitiven Abschluss} von $R$, sie wird mit $R^*$ bezeichnet.
    \end{itemize}
\end{definition}

\begin{remark}
    F\"ur eine beliebige (bin\"are) Relation $R$ gilt genau dann $xR^*y$, wenn es
    eine endliche Folge $x=k_1,\dots,k_n=y$ gibt, so dass $k_iRk_{i+1}$ f\"ur alle
    Indices $i=1,\dots,n-1$ gilt. Es gilt also genau dann $xR^*y$, wenn es eine Folge von
    Elementen gibt, die mit $x$ beginnt, mit $y$ endet und deren Elemente alle der Reihe
    nach in Relation $R$ zueinander stehen. Ist $G=(V,E)$ ein Graph, dann bedeutet
    $xE^*y$, dass in $G$ ein Pfad von $x$ nach $y$ existiert.
\end{remark}

\begin{definition} {Pfad und Zyklus}
    Ein \textit{Weg} oder \textit{Pfad} in einem Graph $G=(V,E)$ ist eine endliche Folge
    $k_1,\dots,k_n\in V$ von Knoten, so dass $k_iEk_{i+1}$ f\"ur alle Indices
    $i=1,\dots,n-1$ gilt. Die Knoten $k_1$ und $k_n$ bezeichnet man als \textit{Anfangs-}
    und \textit{Endpunkt} des Pfades. Gilt zusätzlich $k_1=k_n$, dann spricht man von einem \textit{Zyklus}.
\end{definition}

\begin{definition}{Topologische Sortierung}
    Es sei $M$ eine endliche Menge und $G=(M,E)$ ein DAG. Eine lineare Ordnung $\preceq\subseteq M\times M$ ist eine \textit{topologische Sortierung} von $G$, wenn für alle $a,b\in M$
    \begin{align*}
        a E^* b  \Rightarrow a\preceq b
    \end{align*}
    gilt.
\end{definition}

\begin{lemma}{Topologische Sortierung DAG}
    Jeder endliche DAG besitzt (mindestens) eine topologische Sortierung.
\end{lemma}

\begin{lemma}{Halbordnung DAG}
    Folgende Aussagen sind äquivalent:
    \begin{enumerate}
        \item $(V,E\setminus \Delta_V)$ ist ein DAG.
        \item $E^*$ ist eine Halbordnung auf $V$.
    \end{enumerate}
\end{lemma}

\begin{corollary}{Halbordnung - lineare Ordnung}
    Jede endliche Halbordnung kann zu einer linearen Ordnung erweitert werden. Formal, zu jeder Halbordnung $\preceq$ auf einer Menge $M$ gibt es eine lineare Ordnung $\ll$ auf $M$, so dass
    \begin{align*}
        a\preceq b \Rightarrow a\ll b
    \end{align*}
    gilt.
\end{corollary}

\begin{lemma}{Wohlordnung - kleinstes Element}
    Ist $\preceq$ eine Wohlordnung auf einer Menge $M$, dann gibt es keine unendlich absteigende Folge
    \[
        a_0\succeq a_1\succeq\dots\succeq a_n\succeq a_{n+1}\succeq\dots
    \]
    von verschiedenen Elementen aus $M$.
\end{lemma}

\begin{definition}{Hasse-Diagramm}
    Es sei $\preceq$ eine Halbordnung auf einer Menge $M$. Das \textit{Hasse-Diagramm} von $R$ ist eine vereinfachte Darstellung des Graphen $(M,\preceq)$.
    \begin{itemize}
        \item Die Richtung eines Pfeiles $a\to b$ für Elemente $a,b\in M$ wird dadurch zum Ausdruck gebracht, dass sich der Knoten $b$ oberhalb von $a$ befindet.
        \item Pfeile zwischen zwei Punkten $a,b$ werden gelöscht, wenn es einen weiteren Punkt $c$ mit $a\preceq c\preceq b$ gibt.
        \item Pfeile, die von einem Punkt auf denselben Punkt zeigen (Schleifen), werden weggelassen.
    \end{itemize}
\end{definition}
