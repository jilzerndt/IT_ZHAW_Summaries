\section{Rekursive Strukturen}

\subsection{Die grundlegende Struktur der natürlichen Zahlen}

\begin{concept}{Peano-Axiome}
    Von dieser Anschauung geleitet, listen wir nun einige Grundtatsachen über die Struktur $\N$ auf. Diese Grundannahmen entsprechen den sogenannten \textit{Peano-Axiomen}.

    \begin{itemize}
        \item Die Zahl $0$ ist eine natürliche Zahl. Jede natürliche Zahl $k$ hat genau einen Nachfolger $k+1$. Der Nachfolger jeder natürlichen Zahl ist wiederum eine natürliche Zahl.
        \item Die Zahl $0$ ist die einzige natürliche Zahl, die kein Nachfolger ist:
            \[
                \forall n\in\N\,(\underbrace{\forall k\in\N\,(n\neq k+1)}_{n\text{ ist kein Nachfolger}}\Leftrightarrow n=0 ).
            \]
        \item Jede natürliche Zahl ist Nachfolger von höchstens einer natürlichen Zahl:
            \[
                \forall n,m\in\N\,(n+1=m+1\Rightarrow n=m).
            \]
        \item \textit{Das Prinzip der (vollständigen) Induktion}: Es sei $A(n)$ eine Eigenschaft (ein Prädikat) von natürlichen Zahlen. Aus den beiden Voraussetzungen
            \begin{itemize}
                \item[] \textbf{Induktionsverankerung (I.V.):} $A(0)$
                \item[] \textbf{Induktionsschritt (I.S.):} $\forall n\in \N\,(A(n)\Rightarrow A(n+1))$,
            \end{itemize}
            folgt die Gültigkeit von $\forall n\in\N\,(A(n))$.
    \end{itemize}
\end{concept}

\begin{lemma}{Induktionsschritt}
    Der Induktionsschritt ist stets von der Form
    \[
        \forall n\in\N\,\big( \underbrace{A(n)}_{\text{Induktionsannahme}}\,\Rightarrow A(n+1)\,\big)
    \]
    für ein Prädikat $A$. Der Teil $A(n)$ wird dabei \textit{Induktionsannahme} genannt, weil er beim Nachweis von $A(n+1)$ als Annahme verwendet werden darf.
\end{lemma}

\begin{howto}{Vollständige Induktion}
    Das Prinzip der vollständigen Induktion ist ein mächtiges Mittel um viele verschiedene Behauptungen über natürliche Zahlen beweisen zu können. Will man eine Aussage von der Form
    \[
        \text{Jede natürliche Zahl }n\text{ erfüllt }E(n)
    \]
    für ein Prädikat $E$ beweisen, dann muss man, wenn man die Eigenschaft $E$ nicht für alle natürlichen Zahlen \textit{simultan} beweisen kann, im Prinzip unendlich viele Schritte bewältigen:
    \begin{enumerate}
        \item[1.] Schritt: Zeige $E(0)$.
        \item[2.] Schritt: Zeige $E(1)$.
        \item[3.] Schritt: Zeige $E(2)$.
        \item[$\vdots$]
    \end{enumerate}
    Die Stärke des Induktionsargumentes liegt nun  darin, all diese unendlich vielen Schritte auf zwei Schritte zu reduzieren:
    \begin{enumerate}
        \item[1.] Schritt (I.V.): Zeige $E(0)$.
        \item[2.] Schritt (I.S.): Zeige, dass die Eigenschaft $E$ unter Nachfolgern erhalten bleibt. Intuitiv könnte man sagen, dass die Eigenschaft $E$ von jeder natürlichen Zahl auf die nächste ``vererbt'' wird.
    \end{enumerate}
\end{howto}

\begin{example}
    Wir benützen ein Induktionsargument um zu beweisen, dass alle natürlichen Zahlen $n>1$ für beliebige reelle Zahlen $r>-1, r\neq 0$ die folgende Eigenschaft haben:
    \[
        (1+r)^n>1+nr.
    \]
    \tcblower
    \begin{itemize}
        \item \textbf{Verankerung $(n=2)$:} Die Verankerung gilt, wegen
            \[
                (1+r)^2=1+2r+r^2>1+2r.
            \]
        \item \textbf{Schritt $(n\to n+1)$:} Wir nehmen nun an, dass die Aussage für $n$ gilt (I.A.) und zeigen sie für $n+1$:
            \begin{align*}
                (1+r)^{n+1}&=(1+r)^n(1+r)\\
                &\stackrel{I.A.}{>}(1+nr)(1+r)\\
                &=1+nr+r+\underbrace{nr^2}_{\text{positiv}}\\
                &>1+(n+1)r.\qedhere
            \end{align*}
    \end{itemize}
\end{example}

\begin{example}
    Für jede endliche Menge $X$ gilt
    \[
        |\mathcal{P}(X)|=2^{|X|}.
    \]
    \tcblower
    Wir führen den Beweis durch Induktion nach der Anzahl Elemente der Menge $X$.
    \begin{itemize}
        \item \textbf{Verankerung ($|X|=0$):} Die einzige Menge mit $0$ Elementen ist die leere Menge, es gilt also wie gewünscht
            \[
                |\mathcal{P}(X)|=|\mathcal{P}(\varnothing)|=|\{\varnothing\}|=1=2^0=2^{|X|}.
            \]
        \item \textbf{Schritt:} Es sei nun $X$ eine $n+1$ elementige Menge. Aufgrund der Induktionsannahme können wir davon ausgehen, dass für alle Mengen $Y$ mit $n$ Elementen die Gleichung
            \[
                |\mathcal{P}(Y)|=2^{|Y|}
            \]
            erfüllt ist. Da $X\neq\varnothing$ gilt, können wir ein $x\in X$ auswählen. Wir unterteilen die Potenzmenge von $X$ in zwei disjunkte, gleich grosse Teile $A$ und $B$:
            \begin{align*}
                A=\{Y\subseteq X\mid x\notin Y \}\\
                B=\{Y\subseteq X\mid x\in Y \}.
            \end{align*}
            Es gilt:
            \begin{align*}
                |\mathcal{P}(X)|&=|A\cup B|=|A|+|B|\\
                &=|A|+|A|=2|A|=2|\mathcal{P}(X\setminus\{x\})|\\&\stackrel{I.A.}{=}2\cdot2^n=2^{n+1}.\qedhere
            \end{align*}
    \end{itemize}
\end{example}

\begin{lemma}{Vollständige Induktion mit Mengen}
    Für jede Menge $X$ von natürlichen Zahlen gilt: Wenn $X$ die Bedingungen
    \begin{itemize}
        \item Induktionsverankerung: $0\in X$
        \item Induktionsschritt: $\forall n\,(n\in X\Rightarrow n+1\in X)$
    \end{itemize}
    erfüllt, dann ist bereits $X=\N$.
\end{lemma}

\begin{proof}{Induktion Prädikate}
    Ist $E(n)$ das Prädikat $n\in X$, dann folgt mit vollständiger Induktion sofort $\forall n\, (E(n))$ und somit $\N=X$.
\end{proof}

\begin{definition}{Grösser/Kleiner als Ordnung}
    Die Ordnung $\leq$ auf den natürlichen Zahlen ist durch
    \[
        x\leq y:\Leftrightarrow \exists k\in\N\,(x+k=y)
    \]
    gegeben. Wir schreiben weiter
    \[
        x<y:\Leftrightarrow x\leq y\land x\neq y.
    \]
\end{definition}

\begin{lemma}{Minimumprinzip}
    Jede nichtleere Menge von natürlichen Zahlen hat ein minimales Element.
\end{lemma}

\begin{lemma}{Descending Chains}
    Es gibt keine unendlich absteigende Folge
    \[
        a_0>a_1>\dots >a_n>a_{n+1}>\dots
    \]
    von natürlichen Zahlen.
\end{lemma}

\begin{howto}{Der kleinste Verbrecher}
    Die Beweismethode des ``kleinsten Verbrechers'' geht wie folgt: Will man zeigen, dass alle natürlichen Zahlen eine Eigenschaft $E$ haben, dann geht man davon aus, dass wenn dies nicht der Fall wäre, es eine kleinste natürliche Zahl $n_0$ (den kleinsten Verbrecher) gäbe, die \textit{nicht} die Eigenschaft $E$ hat. Führt man diese Annahme zu einem Widerspruch, so hat man die ursprüngliche Behauptung bewiesen. Obwohl die Methode des ``kleinsten Verbrechers'' also nichts anderes als die Kombination eines Widerspruchsargumentes mit dem Minimumprinzip ist, handelt es sich doch um eine sehr ``anwenderfreundliche'' und einprägsame Beschreibung dieser Argumentationsfolge.
\end{howto}

\begin{example}
    Beweisen Sie mit der Methode des ``kleinsten Verbrechers''.
    Jede natürliche Zahl von der Form ($n^2+n$) ist gerade.
    \tcblower
    Wir nehmen an, dass es ungerade natürliche Zahlen von der Form $n^2+n$ gibt. Die Zahl $n^2+n$ sei die kleinste solche Zahl (der kleinste Verbrecher). Weil $n$ nicht Null sein kann (sonst wäre $n^2+n$ gerade), muss es ein $k\in \N $ mit $n=k+1$ geben. Weil $k<n$ gilt, muss $k^2+k$ aber gerade sein. Daraus folgt
    \begin{align*}
        n^2+n &= (k+1)^2+(k+1)= k^2 + 2k + 1 + k + 1\\
        &= \underbrace{k^2+k}_{gerade}+\underbrace{2k+2}_{gerade}
    \end{align*}
    und somit, dass $n^2+n$ gerade ist (im Widerspruch zur Annahme).
\end{example}

\begin{lemma}{Rechenregeln für Partialsummen}
    Sind $(a_i)_{i\in\N}$ und $(a_i)_{i\in\N}$ beliebige Folgen und ist $c\in\N$, dann gilt für jedes $n\in\N$:
    \[
        \sum_{i=1}^n(ca_i+cb_i)=c\big(\sum_{i=1}^na_i+\sum_{i=1}^nb_i\big)
    \]
\end{lemma}

\subsection{Vom Induktionsbeweis zum rekursiven Algorithmus}

\begin{example}[Türme von Hanoi]
    ``Die Türme von Hanoi'' ist ein Geduldspiel. Das Spiel besteht aus drei gleich grossen Stäben A, B und C, auf die mehrere gelochte Scheiben gelegt werden, alle verschieden gross. Zu Beginn liegen alle Scheiben auf Stab A, der Grösse nach geordnet, mit der grössten Scheibe unten und der kleinsten oben. Ziel des Spiels ist es, den kompletten Scheiben-Stapel von A nach C zu versetzen.
    Bei jedem Zug darf die oberste Scheibe eines beliebigen Stabes auf einen der beiden anderen Stäbe gelegt werden, vorausgesetzt, dort liegt nicht schon eine kleinere Scheibe. Folglich sind zu jedem Zeitpunkt des Spieles die Scheiben auf jedem Feld der Grösse nach geordnet.

    Wir wollen beweisen, dass ``die Türme von Hanoi'' mit beliebig vielen Scheiben erfolgreich gespielt werden können.
    \begin{proof} Wir benutzen ein Induktionsargument ($n$ sei die Anzahl Scheiben):
        \begin{itemize}
            \item \textbf{Verankerung $n=0$:} Dieser Fall ist trivial, da es keine Scheiben zu bewegen gibt.
            \item \textbf{Induktionsschritt $n\to n+1$:} Wir betrachten das Spiel mit $n+1$ Scheiben. Nach Induktionsvoraussetzung gibt es eine Lösungsstrategie für das Spiel mit nur $n$ Scheiben. Diese Strategie können wir offensichtlich dazu verwenden, um alle bis auf die grösste Scheibe auf den Stab B zu verschieben. Nun können wir die grösste Scheibe auf den Stab $C$ verschieben, um anschliessend nochmal die Strategie für das Spiel mit $n$ Scheiben anzuwenden und alle kleineren Scheiben auf den Stab C zu bewegen. Das Spiel ist somit auch für $n+1$ Scheiben lösbar. \qedhere
        \end{itemize}
    \end{proof}
\end{example}

\subsection{Rekursive Definitionen}

Rekursive Definitionen bezeichnen die mathematisch einwandfreie Art, ein Objekt durch Bezugnahme (Selbstreferenz) auf das zu definierende Objekt selbst zu definieren.

\begin{example}
    Ein Palindrom ist ein Wort, das rückwärts und vorwärts gelesen gleich lautet. Beispiele von Palindromen sind $yxy,acaca,arbbra,b,a,\dots$. Obwohl es uns anschaulich klar ist, welche Wörter Palindrome sind und welche nicht, ist unsere Beschreibung keine mathematisch präzise Definition. Wie können wir also Palindrome definieren (eindeutig beschreiben), ohne auf unsere Vorstellung von rückwärts und vorwärts lesen angewiesen zu sein? Durch Rekursion:\\
    Ein Wort $w$ ist ein Palindrom, wenn mindestens eine der beiden folgenden Bedingungen erfüllt ist:
    \begin{itemize}
        \item Das Wort $w$ besteht aus einem oder gar keinem Buchstaben (Länge von $w$ $<2$).
        \item Es gibt einen Buchstaben (Zeichen, Char) $x$ und ein $\underbrace{Palindrom}_{Selbstreferenz}$ $u$ so, dass $w=xux$ gilt.
    \end{itemize}
\end{example}

\begin{theorem}[Rekursive Definitionen]
    Ist $M$ eine Menge und $G:M\times\N\rightarrow M$ sowie $c\in M$, dann gibt es eine eindeutig bestimmte Funktion $F:\N\rightarrow M$, welche die Gleichungen (Rekursionsgleichungen)
    \begin{align*}
        F(0)&=c\\
        F(k+1)&=G(\underbrace{F(k)}_{Selbstbezug},k)
    \end{align*}
    erfüllt.
\end{theorem}

\begin{proof}[Beweisidee]
    Die Behauptung besteht aus einer Eindeutigkeitsaussage und einer Existenzaussage:
    \begin{itemize}
        \item Die Funktion $F:\N\to M$ ist durch die Rekursionsgleichungen eindeutig bestimmt. Das heisst, dass es keine andere  Funktion gibt, die den Rekursionsgleichungen von $F$ genügt.
        \item Es gibt überhaupt eine Funktion, die den Rekursionsgleichungen genügt.
    \end{itemize}
    Wir beweisen zuerst die Eindeutigkeitsbedingung. Wir nehmen an, dass $F$ und $H$ zwei Funktionen sind, die beide die oben genannten Rekursionsgleichungen erfüllen und zeigen, dass daraus $F=H$ folgt. Es genügt mit Induktion zu zeigen, dass für jede natürliche Zahl $n\in \N$ die Gleichung $F(n)=H(n)$ gilt (weil dann $H=F$ gilt).
    \begin{itemize}
        \item Verankerung ($n=0$): Aufgrund von
            \[
                F(0)=c=H(0)
            \]
            ist die Induktionsverankerung erfüllt.
        \item Schritt ($n\to n+1$): Wir nehmen an, dass $F(n)=H(n)$ gilt und müssen $F(n+1)=H(n+1)$ beweisen. Dies folgt sofort aus
            \[
                F(n+1)=G(F(n),n)\stackrel{IA}{=}G(H(n),n)=H(n+1).
            \]
    \end{itemize}
\end{proof}

\begin{example}
    Die üblichen arithmetischen Grundoperationen können alle relativ kompakt als rekursive Definitionen geschrieben werden:
    \begin{itemize}
        \item Die Addition von natürlichen Zahlen:
            \begin{align*}
                x+0&=x\\
                x+(n+1)&=(x+n)+1
            \end{align*}
        \item Die Multiplikation von natürlichen Zahlen:
            \begin{align*}
                x\cdot 0 &= 0\\
                x\cdot(n+1)&=(x\cdot n)+x
            \end{align*}
        \item Die Exponentiation von natürlichen Zahlen:
            \begin{align*}
                x^0&=1\\
                x^{n+1}&=x\cdot x^{n}
            \end{align*}
        \item Die Fakultätsfunktion:
            \begin{align*}
                0!&=1\\
                (n+1)!&=n!\cdot(n+1)
            \end{align*}
        \item Endliche Summen:
            \begin{align*}
                \sum_{i=1}^{0}a_i&=0\\
                \sum_{i=1}^{n+1}a_{i}&=(\sum_{i=0}^na_i)+a_{n+1}
            \end{align*}
        \item Endliche Produkte:
            \begin{align*}
                \prod_{i=1}^{0}a_i&=1\\
                \prod_{i=1}^{n+1}a_i&= (\prod_{i=1}^na_i)\cdot a_{n+1}
            \end{align*}
    \end{itemize}
\end{example}

Die üblichen Rechenregeln für natürliche Zahlen lassen sich aufgrund dieser rekursiven Definitionen mit Induktion (und genügend Geduld) beweisen.

\begin{lemma}{Rechenregeln für Addition}
    Für alle natürlichen Zahlen $n,m,k$ gelten folgende Rechenregeln für deren Addition:
    \begin{enumerate}
        \item Neutrales Element: $0+n=n$
        \item Kommutativität: $n+m=m+n$
        \item Assoziativität: $(n+m)+k=n+(m+k)$
        \item Kürzbarkeit: $n+k=m+k\Rightarrow n=m$
    \end{enumerate}
\end{lemma}

\begin{remark}
    Wegen der Assoziativität der Addition, können wir Klammern in endlichen Summen von natürlichen Zahlen weglassen.
\end{remark}

\begin{lemma}[Rechenregeln für die Multiplikation]
    Für alle $n,m,k\in\N$ gelten folgende Identitäten:
    \begin{enumerate}
        \item \textit{Absorbtion:} $0\cdot n=0$
        \item \textit{Neutrales Element:} $1\cdot n=n$
        \item \textit{Kommutativität:} $n\cdot m=m\cdot n$
        \item \textit{Assoziativität:} $n\cdot(m\cdot k)=(n\cdot m)\cdot k$
        \item \textit{Distributivität:} $n\cdot(m+k)=nm+nk$
    \end{enumerate}
\end{lemma}
