\chapter{Elementare Zahlentheorie}

\begin{lemma}{Rechenregeln auf $\Z$}
Für alle $r,s,z\in\Z$ gelten folgende Gleichungen.
\begin{align*}
-1\cdot z&=-z\\
-(-z)&=z\\
-z+z&=0 &\text{ Inverse Elemente bezüglich }+\\
0\cdot z&=0 &\text{ Absorbtion}\\
1\cdot z&=z &\text{ Neutrales Element bezüglich }\cdot\\
0+z&=z &\text{ Neutrales Element bezüglich }+\\
r(sz)&=(rs)z &\text{ Assoziativität von } \cdot\\
r+(s+z)&=(r+s)+z &\text{ Assoziativität von }+\\
rs&=sr &\text{ Kommutativität von }\cdot\\
r+s&=s+r &\text{ Kommutativität von }+\\
r(s+z)&=rs+rz &\text{ Distributivität}\\
rx=ry&\Rightarrow x=y\lor r=0&\text{Kürzbarkeit}
\end{align*}
\end{lemma}


\subsection{Teilbarkeit und Euklidischer Algorithmus}
\begin{definition}
 Sind $x,y\in\mathbb{Z}$ ganze Zahlen, so sagen wir, dass $x$ \textit{ein Teiler von} $y$ ist, falls es ein $k\in\mathbb{Z}$ gibt mit $xk=y$. Wir schreiben in diesem Fall $x|y$. Es gilt also
\[
 x|y:\Leftrightarrow \exists k\in\Z(y=xk).
\]
Mit $T(y)$ bezeichnen wir die Menge aller natürlichen Zahlen, welche Teiler von $y$ sind, also $T(y)=\{x\in\N\mid x|y\}$.
\end{definition}
\begin{example}~
 \begin{enumerate}
  \item Die Zahl $1$ ist ein Teiler jeder ganzen Zahl $z$, da $1\cdot z=z$.
\item $T(0)=\N$.
 \end{enumerate}

\end{example}
\begin{remark}
 Die Teilbarkeitsrelation ist reflexiv und transitiv auf der Menge $\Z$, auf der Menge $\N$ ist die Teilbarkeitsrelation sogar eine Halbordnung.
\end{remark}
\begin{proof}{Eigenschaften Teilbarkeitsrelation}
Wir zeigen, dass die Teilbarkeitsrelation reflexiv, transitiv und für natürliche Zahlen auch antisymmetrisch ist.
\begin{itemize}
\item Reflexivität: Dies gilt, da jede ganze Zahl sich selbst teilt.
\item Transitivität: Seien $x,y,z$ ganze Zahlen. Aus $x|y$ und $y|z$ folgt, dass es ganze Zahlen $k_1,k_2$ gibt mit $x\cdot k_1=y$ und $y\cdot k_2=z$. Es folgt
\[
x\cdot(k_1\cdot k_2)=(x\cdot k_1)\cdot k_2=y\cdot k_2=z.
\]
 Somit existiert eine ganze Zahl $k$ (nämlich $k=k_1\cdot k_2$) mit $k\cdot x=z$, also gilt $x|z$ wie gewünscht.\qedhere
\item Antisymmetrie auf $\N$: Wir müssen zeigen, dass für natürliche Zahlen $x$ und $y$ aus $x|y$ und $y|x$ folgt, dass $x=y$ gilt. Es gelte also $xk=y$ und $x=yr$ für ganze Zahlen $k,r$. Es folgt
\[
x=yr=(xk)r=x(kr)
\]
und $kr=1$. Daraus ergeben sich zwei mögliche Fälle; $k=r=1$ oder $k=r=-1$. Im Fall $k=r=-1$ folgt $x=-y$, was im Widerspruch dazu steht, dass $x$ und $y$ natürliche Zahlen sind. Es bleibt also nur der Fall $k=r=1$ und somit, wie gewünscht, $x=y$.
\end{itemize}
\end{proof}


\begin{remark}\label{Zeinheiten}
Sind $x,y\in\Z$ und gilt $x\cdot y=1$ so gilt $|x|=|y|=1$.
\end{remark}


\begin{lemma}{Teilen mit Rest}
 Sind $n,m\in\N\backslash\{0\}$, dann gibt es eindeutig bestimmte Zahlen $k,r\in\N$, so dass Folgendes gilt:
\begin{enumerate}
\item $m=kn+r$
 \item $r<n$
\end{enumerate}
Wir sagen in diesem Zusammenhang, dass die Zahl $r$ den \textit{Rest} von der (ganzzahligen) Division von $m$ durch $n$ ist.
\end{lemma}

\begin{definition}{Kleistes gemeinsames Vielfaches}
Seien $n,m\in\Z$. Wir definieren das \textit{kleinste gemeinsame Vielfache von $n$ und $m$} als
\[
 kgV(n,m):=\min\{k\in\N\mid n|k\wedge m|k\}.
\]
Ist $n\neq0$ oder $ m\neq 0$, dann definieren wir den \textit{grössten gemeinsamen Teiler} von $n$ und $m$ als
\[
 ggT(n,m):=\max\{k\in\N\mid k|n\wedge k|m\}.
\]
\end{definition}

\begin{lemma}{Grösster gemeinsamer Teiler}
 Sind $x,y,z\in\Z$, dann sind folgende Aussagen äquivalent:
\begin{enumerate}
\item[1.] $ x|y\wedge x|z$
\item[2.] $x|y\wedge x|(y-z) $
\end{enumerate}
\end{lemma}
\begin{proof}{ggT}
 $1.\Rightarrow 2.$: Wenn $x|y\wedge x|z$, dann gibt es ganze Zahlen $k,k'\in\Z$, so dass $y=kx$ und $z=k'x$. Es gilt also $y-z=kx-k'x=(k-k')x$.

$2.\Rightarrow 1.$: Es seien $k,k'\in\Z$, so dass $y=kx$ und $y-z=k'x$. Durch Einsetzen erhält man $ kx-z=k'x $ und somit $z=kx-k'x=x(k-k')$.
\end{proof}

\begin{lemma}{Euklidischer Algorithmus}
 Für $n,m\in\N$ mit $0<n< m$ gilt
\[
 ggT(n,m)=ggT(n,m-n)=ggT(m,m-n).
\]
\end{lemma}


\begin{definition}{Teilerfremd}
 Zwei ganze Zahlen $x,y$ heissen \textit{teilerfremd}, wenn $ggT(x,y)=1$ gilt.
\end{definition}

%\begin{ern}
%In den übungen haben Sie bewiesen, dass man das Prinzip der vollständigen Induktion dahingehend verallgemeinern kann, dass man in der Induktionsannahme auf alle ``Vorgängerfälle'' Bezug nehmen darf. Formal haben Sie gezeigt, für jede Menge $X\subseteq\N$ mit der Eigenschaft
%\[
%\forall n\in\N\big(\forall m<n\,(m\in X\Rightarrow n\in X)\big),
%\]
%$X=\N$ gilt. Dieses Verfahren werden wir im folgenden Beweis anwenden.
%\end{ern}

\begin{theorem}{Lemma von Bézout}
Sind $x,y\in\Z$ mit $x,y\neq 0$, dann gibt es ganze Zahlen $a,b$ so dass
\[
 ggT(x,y)=ax+by
\]
gilt. Die Zahlen $a$ und $b$ werden Bézout Koeffizienten genannt.
\end{theorem}


