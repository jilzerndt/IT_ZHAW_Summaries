

\subsection{Die grundlegende Struktur der natürlichen Zahlen}

\begin{concept}{Peano-Axiome}
    Von dieser Anschauung geleitet, listen wir nun einige Grundtatsachen über die Struktur $\N$ auf. Diese Grundannahmen entsprechen den sogenannten \textit{Peano-Axiomen}.

\begin{itemize}
 \item Die Zahl $0$ ist eine natürliche Zahl. Jede natürliche Zahl $k$ hat genau einen Nachfolger $k+1$. Der Nachfolger jeder natürlichen Zahl ist wiederum eine natürliche Zahl.
 \item Die Zahl $0$ ist die einzige natürliche Zahl, die kein Nachfolger ist:
 \[
 \forall n\in\N\,(\underbrace{\forall k\in\N\,(n\neq k+1)}_{n\text{ ist kein Nachfolger}}\Leftrightarrow n=0 ).
 \]
 \item Jede natürliche Zahl ist Nachfolger von höchstens einer natürlichen Zahl:
 \[
 \forall n,m\in\N\,(n+1=m+1\Rightarrow n=m).
 \]
\item \textit{Das Prinzip der (vollständigen) Induktion}: Es sei $A(n)$ eine Eigenschaft (ein Prädikat) von natürlichen Zahlen. Aus den beiden Voraussetzungen
\begin{itemize}
\item[] \textbf{Induktionsverankerung (I.V.):} $A(0)$
\item[] \textbf{Induktionsschritt (I.S.):} $\forall n\in \N\,(A(n)\Rightarrow A(n+1))$,
\end{itemize}
folgt die Gültigkeit von $\forall n\in\N\,(A(n))$.
\end{itemize}
\end{concept}
 

\begin{lemma}{Induktionsschritt}
Der Induktionsschritt ist stets von der Form
\[
\forall n\in\N\,\big( \underbrace{A(n)}_{\text{Induktionsannahme}}\,\Rightarrow A(n+1)\,\big)
\]
für ein Prädikat $A$. Der Teil $A(n)$ wird dabei \textit{Induktionsannahme} genannt, weil er beim Nachweis von $A(n+1)$ als Annahme verwendet werden darf.
\end{lemma}

\begin{howto}{Vollständige Induktion}
Das Prinzip der vollständigen Induktion ist ein mächtiges Mittel um viele verschiedene Behauptungen über natürliche Zahlen beweisen zu können. Will man eine Aussage von der Form
\[
\text{Jede natürliche Zahl }n\text{ erfüllt }E(n)
\]
für ein Prädikat $E$ beweisen, dann muss man, wenn man die Eigenschaft $E$ nicht für alle natürlichen Zahlen \textit{simultan} beweisen kann, im Prinzip unendlich viele Schritte bewältigen:
\begin{enumerate}
\item[1.] Schritt: Zeige $E(0)$.
\item[2.] Schritt: Zeige $E(1)$.
\item[3.] Schritt: Zeige $E(2)$.
\item[$\vdots$]
\end{enumerate}
Die Stärke des Induktionsargumentes liegt nun  darin, all diese unendlich vielen Schritte auf zwei Schritte zu reduzieren:
\begin{enumerate}
\item[1.] Schritt (I.V.): Zeige $E(0)$.
\item[2.] Schritt (I.S.): Zeige, dass die Eigenschaft $E$ unter Nachfolgern erhalten bleibt. Intuitiv könnte man sagen, dass die Eigenschaft $E$ von jeder natürlichen Zahl auf die nächste ``vererbt'' wird.
\end{enumerate}
\end{howto}


\begin{example}
Wir benützen ein Induktionsargument um zu beweisen, dass alle natürlichen Zahlen $n>1$ für beliebige reelle Zahlen $r>-1, r\neq 0$ die folgende Eigenschaft haben:
\[
 (1+r)^n>1+nr.
\]
\tcblower
\begin{itemize}
\item \textbf{Verankerung $(n=2)$:} Die Verankerung gilt, wegen
\[
(1+r)^2=1+2r+r^2>1+2r.
\]
\item \textbf{Schritt $(n\to n+1)$:} Wir nehmen nun an, dass die Aussage für $n$ gilt (I.A.) und zeigen sie für $n+1$:
\begin{align*}
(1+r)^{n+1}&=(1+r)^n(1+r)\\
&\stackrel{I.A.}{>}(1+nr)(1+r)\\
&=1+nr+r+\underbrace{nr^2}_{\text{positiv}}\\
&>1+(n+1)r.\qedhere
\end{align*}
\end{itemize}
\end{example}

\begin{example}
Für jede endliche Menge $X$ gilt
\[
|\mathcal{P}(X)|=2^{|X|}.
\]
\tcblower
Wir führen den Beweis durch Induktion nach der Anzahl Elemente der Menge $X$.
\begin{itemize}
\item \textbf{Verankerung ($|X|=0$):} Die einzige Menge mit $0$ Elementen ist die leere Menge, es gilt also wie gewünscht
\[
|\mathcal{P}(X)|=|\mathcal{P}(\varnothing)|=|\{\varnothing\}|=1=2^0=2^{|X|}.
\]
\item \textbf{Schritt:} Es sei nun $X$ eine $n+1$ elementige Menge. Aufgrund der Induktionsannahme können wir davon ausgehen, dass für alle Mengen $Y$ mit $n$ Elementen die Gleichung
\[
|\mathcal{P}(Y)|=2^{|Y|}
\]
 erfüllt ist. Da $X\neq\varnothing$ gilt, können wir ein $x\in X$ auswählen. Wir unterteilen die Potenzmenge von $X$ in zwei disjunkte, gleich grosse Teile $A$ und $B$:
 \begin{align*}
 A=\{Y\subseteq X\mid x\notin Y \}\\
 B=\{Y\subseteq X\mid x\in Y \}.
 \end{align*}
Es gilt:
\begin{align*}
|\mathcal{P}(X)|&=|A\cup B|=|A|+|B|\\
&=|A|+|A|=2|A|=2|\mathcal{P}(X\setminus\{x\})|\\&\stackrel{I.A.}{=}2\cdot2^n=2^{n+1}.\qedhere
\end{align*}
\end{itemize}
\end{example}

\begin{lemma}{Vollständige Induktion mit Mengen}
Für jede Menge $X$ von natürlichen Zahlen gilt: Wenn $X$ die Bedingungen
\begin{itemize}
\item Induktionsverankerung: $0\in X$
\item Induktionsschritt: $\forall n\,(n\in X\Rightarrow n+1\in X)$
\end{itemize}
erfüllt, dann ist bereits $X=\N$.
\end{lemma}
\begin{proof}{Induktion Prädikate}
Ist $E(n)$ das Prädikat $n\in X$, dann folgt mit vollständiger Induktion sofort $\forall n\, (E(n))$ und somit $\N=X$.
\end{proof}

\begin{definition}{Grösser/Kleiner als Ordnung}
Die Ordnung $\leq$ auf den natürlichen Zahlen ist durch
\[
x\leq y:\Leftrightarrow \exists k\in\N\,(x+k=y)
\]
gegeben. Wir schreiben weiter
\[
x<y:\Leftrightarrow x\leq y\land x\neq y.
\]
\end{definition}


\begin{lemma}{Minimumprinzip}
Jede nichtleere Menge von natürlichen Zahlen hat ein minimales Element.
\end{lemma}
\begin{proof}{Minimumprinzip}
Wir zeigen, dass jede Menge von natürlichen Zahlen, die kein minimales Element enthält, leer ist. Dazu wählen wir eine beliebige Menge $X\subseteq\N$ ohne minimales Element. Um zu zeigen, dass die Menge $X$ leer ist, genügt es zu zeigen, dass die Menge
\[
Y=\{n\in\N\mid \forall x\in X\,(n<x) \}
\]
aller natürlichen Zahlen, die ``unterhalb'' von $X$ liegen, bereits alle natürlichen Zahlen enthält. Wir zeigen $Y=\N$ mithilfe von Satz~\ref{satz:mengeninduktion}.
\begin{itemize}
\item \textbf{Verankerung:} Es gilt $0\in Y$, weil sonst $0$ das minimale Element von $X$ wäre, was unserer Wahl von $X$ widerspricht.
\item \textbf{Induktionsschritt:} Ist $n\in Y$, dann gilt für alle Elemente $x$ von $X$ die Ungleichung $n<x$. Es gilt daher $n+1\leq x$ für alle Elemente $x$ von $X$. Da $n+1$ kein minimales Element von $X$ sein kann, gilt daher $n+1\in Y$.
\end{itemize}
Aus dem Satz der Mengeninduktion folgt nun, dass $Y=\N$ und somit wie gewünscht $X=\varnothing$ ist.
\end{proof}

\begin{lemma}{Descending Chains}
Es gibt keine unendlich absteigende Folge
\[
a_0>a_1>\dots >a_n>a_{n+1}>\dots
\]
von natürlichen Zahlen.
\end{lemma}

\begin{howto}{Der kleinste Verbrecher}
Die Beweismethode des ``kleinsten Verbrechers'' geht wie folgt: Will man zeigen, dass alle natürlichen Zahlen eine Eigenschaft $E$ haben, dann geht man davon aus, dass wenn dies nicht der Fall wäre, es eine kleinste natürliche Zahl $n_0$ (den kleinsten Verbrecher) gäbe, die \textit{nicht} die Eigenschaft $E$ hat. Führt man diese Annahme zu einem Widerspruch, so hat man die ursprüngliche Behauptung bewiesen. Obwohl die Methode des ``kleinsten Verbrechers'' also nichts anderes als die Kombination eines Widerspruchsargumentes mit Satz~\ref{satz:minimumprinzip} ist, handelt es sich doch um eine sehr ``anwenderfreundliche'' und einprägsame Beschreibung dieser Argumentationsfolge.
\end{howto}

\begin{example}
	Beweisen Sie mit der Methode des ``kleinsten Verbrechers''.
	Jede natürliche Zahl von der Form ($n^2+n$) ist gerade.
 \tcblower
 Wir nehmen an, dass es ungerade natürliche Zahlen von der Form $n^2+n$ gibt. Die Zahl $n^2+n$ sei die kleinste solche Zahl (der kleinste Verbrecher). Weil $n$ nicht Null sein kann (sonst wäre $n^2+n$ gerade), muss es ein $k\in \N $ mit $n=k+1$ geben. Weil $k<n$ gilt, muss $k^2+k$ aber gerade sein. Daraus folgt
		\begin{align*}
			n^2+n &= (k+1)^2+(k+1)= k^2 + 2k + 1 + k + 1\\
			&= \underbrace{k^2+k}_{gerade}+\underbrace{2k+2}_{gerade}
		\end{align*}
		und somit, dass $n^2+n$ gerade ist (im Widerspruch zur Annahme).
\end{example}

\begin{lemma}{Rechenregeln für Partialsummen}
 Sind $(a_i)_{i\in\N}$ und $(a_i)_{i\in\N}$ beliebige Folgen und ist $c\in\N$, dann gilt für jedes $n\in\N$:
\[
 \sum_{i=1}^n(ca_i+cb_i)=c\big(\sum_{i=1}^na_i+\sum_{i=1}^nb_i\big)
\]
\end{lemma}

