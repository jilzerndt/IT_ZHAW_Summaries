\subsection{Der Mengenbegriff und grundlegende Definitionen}

\begin{concept}{Notation Mengen}
Ist $X$ eine Menge und $y$ ein \textit{Element} von $X$, dann schreiben wir $y\in X$. Ist $y$ kein Element von $X$, dann schreiben wir $y\notin X$.
\end{concept}

\begin{remark}
    Die erste \textit{definierende Eigenschaft} von Mengen ist die Tatsache, dass jede Menge durch ihre Elemente vollständig beschrieben ist.
\end{remark}


\begin{definition}{Definierende Eigenschaft}
Zwei Mengen sind genau dann gleich, wenn sie dieselben Elemente enthalten: Es gilt für alle Mengen $X$ und $Y$ die Äquivalenz
\[
X=Y\,\Leftrightarrow\,\forall z\, (z\in X\Leftrightarrow z\in Y).
\]
\end{definition}

%Da Mengen bereits durch Angabe ihrer Elemente bestimmt werden, können wir jede (endliche) Menge durch Auflisten ihrer Elemente festlegen.

\begin{concept}{Explizite Schreibweise}
Sind mathematische Objekte $x_1,\dots,x_n$ gegeben, dann schreiben wir
\[
\{x_1,\dots,x_n\}
\]
für die Menge die als Elemente genau $x_1,\dots,x_n$ hat.
\end{concept}

\begin{comment}
Wir führen im Folgenden einige Operationen und Schreibweisen ein, mithilfe derer wir neue Mengen (aus bereits vorhandenen) generieren können.
Wir erhalten beispielsweise die Menge aller Primzahlen aus der Menge der natürlichen Zahlen, indem wir
\[
\{p\in\N\mid p\text{ hat genau $2$ Teiler}\}
\]
schreiben.
\end{comment}

\begin{concept}{Prädikative Schreibweise}
Ist $X$ eine Menge und ist $\mathsf{E}$ eine Eigenschaft (Prädikat), dann bezeichnen wir mit
\[
\big\{z\in X\mid \mathsf{E}(z)\big\}
\]
oder mit
\[
\big\{z\mid z\in X\land\mathsf{E}(z)\big\}
\]
die Menge aller Elemente $z$ von $X$ mit der Eigenschaft $\mathsf{E}(z)$.
\end{concept}


\begin{concept}{Ersetzungsschreibweise}
Ist $F$ eine Funktion und ist $X$ eine Menge, dann beinhaltet die Menge
\[
\big\{F(x)\mid x\in X \big\}
\]
alle Funktionswerte $F(x)$, die man dadurch erhalten kann, dass man ein Element $x\in X$ in $F$ einsetzt:
\[
\big\{F(x)\mid x\in X\big\}:=\{y\mid \exists x\in X\,(y=F(x))\}.
\]
\end{concept}

\begin{remark}
    Ist eine Funktion $f$ und eine Menge von der Form
    \begin{align*}
        X=\{x_1,x_2,x_3,\dots\}
    \end{align*}
    gegeben, dann entspricht die Menge $\{f(x)\mid x\in X\}$ der Menge
    \begin{align*}
        \{f(x_1),f(x_2),f(x_3),\dots\}.
    \end{align*}
\end{remark}

\begin{comment}
\begin{remark}
    Das Prinzip der Ersetzungsschreibweise findet sich als Funktion zum Manipulieren von Datensätzen in vielen Programmiersprachen wieder:
    \begin{itemize}
        \item Haskell: \texttt{map, fmap}
        \item Java: \texttt{map()}
        \item Python: \texttt{map}
        \item C\#: \texttt{.select}
    \end{itemize}
\end{remark}
\end{comment}


\begin{definition}{Teilmengen}
 Wir schreiben $X\subseteq Y$ und sagen $X$ ist eine \textit{Teilmenge} von $Y$, wenn jedes Element von $X$ auch ein Element von $Y$ ist:
\[
X\subseteq Y:\,\Leftrightarrow\,\forall x\,(x\in X\Rightarrow x\in Y).
\]
Wir schreiben $X\subsetneq Y$ und sagen $X$ ist eine \textit{echte Teilmenge} von $Y$, falls $X$ eine von $Y$ verschiedene Teilmenge von $Y$ ist:
\[
X\subsetneq Y\,:\Leftrightarrow\, X\subseteq Y\land X\neq Y.
\]
\end{definition}

\begin{lemma}{Äquivalenz}\\
Zwei Mengen $X$ und $Y$ sind gleich, wenn $X\subseteq Y$ und $Y\subseteq X$ gilt.
\end{lemma}

\begin{definition}{Potenzmenge}
Ist $A$ eine beliebige Menge, dann bezeichnen wir mit
\[
 \mathcal{P}(A):=\{x\mid x\subseteq A\}
\]
die \textit{Potenzmenge} von $A$, die genau die Teilmengen von $A$ als Elemente enthält.
\end{definition}

\begin{definition}{Schnitt- und Vereinigungsmenge}\\
Sind $X$ und $Y$ Mengen, dann ist
\[
X\cup Y:=\{x\mid x\in X\lor x\in Y \}
\]
die \textit{Vereinigung} von $X$ mit $Y$. Die \textit{Schnittmenge} von $X$ und $Y$ ist durch
\[
X\cap Y:=\{x\in X\mid x\in Y \}=\{x\in Y\mid x\in X\}=\{x\mid x\in X\land x\in Y\}
\]
gegeben. Ist $I$ eine Menge so, dass für alle Elemente $i\in I$ auch $A_i$ eine Menge ist, und $I\neq\varnothing$, dann wird die Vereinigung resp. Schnittmenge von $\{A_i\mid i\in I\}$ genannt.
\begin{align*}
    \bigcup_{i\in I}A_i:=\{x\mid\exists i\in I\,(x\in A_i) \} & & \bigcap_{i\in I}A_i:=\{x\mid\forall i\in I\,(x\in A_i) \}
\end{align*}
\end{definition}

\begin{definition}{Komplementärmenge}\\
 Sind $X$ und $Y$ beliebige Mengen, so definieren wir als
 \[
 X\setminus Y:=\{x\in X\mid x\notin Y\}
 \]
die Menge aller Elemente von $X$, die nicht zu $Y$ gehören. Die Menge $X\setminus Y$ nennt man ``$X$ ohne $Y$''. Ist eine ``Grundmenge'' $A$ (implizit oder explizit) vorgegeben, so bezeichnet man die Menge $A\setminus Y$ auch als ``Komplement'' oder ``Komplementärmenge'' von $X$ (relativ zu $A$).
\end{definition}

\begin{lemma}{Rechenregeln}
 Es gelten für beliebige Mengen $A,B$ und $C$ folgende Identitäten: (gleich wie Junktorenregeln)
\begin{center}
    $A\cap A=A\text{ und }A\cup A=A$\\
    $A\cup B=B\cup A\text{ und }A\cap B=B\cap A.$\\
    $A\cap(B\cap C)=(A\cap B)\cap C\text{ und }A\cup(B\cup C)=(A\cup B)\cup C$\\
    $A\cap(B\cup C)=(A\cap B)\cup (A\cap C)\text{ und }A\cup(B\cap C)=(A\cup B)\cap (A\cup C)$\\
    $(C\backslash A)\cap (C\backslash B)=C\backslash (A\cup B)\text{ und }(C\backslash A)\cup (C\backslash B)=C\backslash (A\cap B)$
\end{center}
Charakterisierung der Teilmengenbeziehung:
\[
A\subseteq B\Leftrightarrow A\cap B= A\Leftrightarrow A\cup B=B
\]
\end{lemma}



\begin{definition}{Disjunkte Mengen}
 Zwei Mengen $X$ und $Y$ heissen \textit{disjunkt}, falls sie keine gemeinsamen Elemente haben, d.h. falls $X\cap Y=\varnothing$ gilt. Wir sagen eine Menge $\{X_i\mid i\in I \}$ von Mengen bestehe aus \textit{paarweise disjunkten} Mengen, wenn folgendes gilt:
 \[
 \forall i,j\in I\,(i\neq j\Rightarrow X_i\cap X_j=\varnothing).
 \]
\end{definition}

\begin{definition}{Partitionen}
Eine \textit{Partition} $P=\{P_i\mid i\in I \}$ einer Menge $A$, ist eine Menge von Teilmengen von $A$, die folgende beiden Voraussetzungen erfüllt:
\begin{itemize}
\item Die Elemente von $P$ sind nichtleer und paarweise disjunkt.
\item $\bigcup_{i\in I}P_i=A$
\end{itemize}
Die Elemente einer Partition werden \textit{Blöcke} der Partition genannt.
\end{definition}


