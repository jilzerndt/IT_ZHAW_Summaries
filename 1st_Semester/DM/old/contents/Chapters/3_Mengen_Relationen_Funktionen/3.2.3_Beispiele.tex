
\begin{figure*}[h]
    %\begin{example}
    Das kartesische Produkt der Menge $\N$ mit sich selbst enthält alle möglichen Paare von natürlichen Zahlen.
    \begin{center}
    %\begin{framed}
    %
    \begin{tikzpicture}[dot/.style={circle,fill=black,minimum size=4pt,inner sep=0pt, outer sep=-1pt}]

    % horizontal axis
    \draw[->] (0,0) -- (5,0) node[anchor=north] {$\N$};
    % labels
    \draw	(3.5,0) node[anchor=north] {$14$};
    \draw	(-0.5,1.3) node[anchor=north] {$5$};
    \draw	(3.5,1) node[dot]{};
    \draw	(3.7,1.7) node[anchor=north]{$(14,5)$};

    % vertical axis
    \draw[->] (0,0) -- (0,3) node[anchor=east] {$\N$};
    % nominal speed
    \draw[dotted] (3.5,0) -- (3.5,1);
    \draw[dotted] (0,1) -- (3.5,1);
    \end{tikzpicture}
    %
    \caption*{Das Paar $(14,5)$ ist ein Element von $\N\times\N$.}
    %\end{framed}
    \end{center}
    %\end{example}
    \end{figure*}