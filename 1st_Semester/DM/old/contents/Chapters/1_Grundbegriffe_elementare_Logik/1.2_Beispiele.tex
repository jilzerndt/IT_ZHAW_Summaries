\begin{example}
Beweisen Sie: Jeder (ganzzahlige) Geldbetrag von mindestens $4$ Cents lässt sich allein mit Zwei- und Fünfcentstücken bezahlen.\\
\textit{Hinweis:} Machen Sie eine Fallunterscheidung ob der zu bezahlende Betrag gerade oder ungerade ist.
\tcblower
Zuerst bemerken wir, dass jeder gerade Betrag mit Zweicentstücken bezahlt werden kann. Ist der gegebene Betrag, sagen wir $x$ cent, ungerade, so muss er mindestens fünf Cent entsprechen, es gilt also $x\geq 5$. Weil $x$ ungerade ist, ist $x-5$ gerade. Wie wir bereits festgestellt haben können wir diesen geraden Betrag mit lauter Zweicentstücken bezahlen. Der gesamte Betrag kann also mit einem Fünfcentstück und Zweicentstücken bezahlt werden.
\end{example}

\begin{example}
Beweisen Sie, dass man $\sqrt{2}$ nicht als Bruch schreiben kann.\\
\textit{Hinweis:} Wenden Sie ein Widerspruchsargument an.
\tcblower
Wir nehmen an, dass $\sqrt{2}$ als gekürzten Bruch dargestellt werden kann und leiten daraus einen Widerspruch her. Es seien $a,b$ ganze, teilerfremde Zahlen mit
\begin{align*}
\sqrt{2}=\frac{a}{b}.
\end{align*}
Quadrieren auf beiden Seiten ergibt
\begin{align*}
  2=\frac{a^2}{b^2}
\end{align*}
und somit
\begin{align*}
  a^2=2b^2.
\end{align*}
Die Zahl $a^2$ muss also gerade sein. Daraus folgt, dass auch $a$ selbst gerade ist. Weil $a$ gerade ist, gibt es eine ganze Zahl $c$ mit der Eigenschaft
\[
a=2c.
\]
Daraus folgt
\[
2b^2=a^2=(2c)^2=4c^2
\]
und damit
\[
b^2=2c^2.
\]
Anhand der letzten Gleichung sehen wir, dass $b^2$ und somit auch $b$ gerade sein muss, dies widerspricht aber der Annahme, dass die Zahlen $a$ und $b$ teilerfremd seien.
\end{example}