\subsection{Kräfte}

Die Wirkung von Kräften auf Objekte verändert diese:

\begin{itemize}
  \item Wenn eine Kraft auf ein hartes, aber ansonsten frei bewegliches Teilchen wirkt, verändert das Teilchen seine Geschwindigkeit.
  \item Je nachdem, wie die Kraft ansetzt, beginnt das Teilchen auch noch zu rotieren.
  \item Weiche Teilchen werden durch die Wirkung von Kräften deformiert.
  \item Die Wirkung von sogenannten Kernkräften kann die Natur eines Teilchens fundamental ändern (Verwandlung von Neutronen in Protonen, usw.)
  \item In der klassischen Physik sind Kräfte durch Vektoren gegeben, deren Grösse aus Massen, Ladungen, Richtungsvektoren usw. berechnet werden.
  \item In der modernen Physik sind Kräfte das Resultat des Austausches von Teilchen (z.B. Elektrodynamik: elektromagnetische Abstossung kann durch Austausch von Photonen zwischen Elektronen erklärt werden).
\end{itemize}

\subsubsection*{Zusammenhang zwischen Kraft und Beschleunigung}
Wenn wir es mit harten Teilchen zu tun haben, gilt: Eine Kraft verändert die Geschwindigkeit eines Objektes, d.h. das Objekt wird beschleunigt (oder gebremst). Formal gibt es einen Zusammenhang zwischen Kraft und Beschleunigung, das dritte Newton'sche Gesetz:

$$
\vec{F}=m \vec{a}
$$

Dabei bezeichnet $\vec{F}$ eine Kraft und $\vec{a}$ eine Beschleunigung (Einheit der Kraft: Newton $[\mathrm{N}]$ bzw. $\left[\mathrm{kg} \mathrm{m} \mathrm{s}^{-2}\right]$, Einheit der Beschleunigung $\left.\left[\mathrm{m} \mathrm{s}^{-2}\right]\right)$.

Der Pfeil oberhalb der Kraft und der Beschleunigung besagt, dass Grössen Vektoren sind. Sie werden erst im vierten Semester mit Vektoren konfrontiert. In dieser Vorlesung tun wir so, als ob Kräfte immer nur in einer Dimension wirken, und dann kann man auch sagen:

$$
F=m a
$$

wobei Beschleunigung und Kräfte dann als Zahlen behandelt werden. Dies ist aus vielerlei Gründen schlecht. Für Sie als InformatikerInnen: Das ist wie Spaghetti-Code. In speziellen Fällen ist es auf den ersten Blick ein bisschen einfacher, aber die Resultate sind nicht verallgemeinerbar. Speziell gilt: Kräfte haben eine Richtung, das heisst, wenn Sie mit Kräften als Zahlen arbeiten, müssen Sie auf die Vorzeichen achten. Noch etwas zum Wort Beschleunigung: Jede Geschwindigkeitsänderung ist eine Beschleunigung. In der Physik nennen wir auch einen Bremsvorgang «Beschleunigung».

Ein wichtiger Spezialfall ist die Situation konstanter Beschleunigung. Viele technische Beschleunigungsvorgänge (speziell das Bremsen auf der Strasse) können durch die Annahme konstanter Beschleunigungen gut angenähert werden. Auch der freie Fall also beispielsweise wenn ein Blumentopf zu Boden fällt - ist durch eine konstante Beschleunigung gut angenähert. Diese Beschleunigung wird «Erdbeschleunigung» $g$ genannt und beträgt $g=9.81 \mathrm{~ms}^{-2} \approx 10 \mathrm{~ms}^{-2}$.

\subsubsection*{Konstante Beschleunigung}
Wir geben Ihnen hier einige Formeln an, deren Herleitung wir nicht zeigen. Sobald Sie integrieren können (im Sinne der Integralrechnung), können Sie versuchen, diese Formeln herzuleiten. Sie müssen die Formeln aber benützen können, auch wenn Sie die Herleitung nicht kennen. Wir rechnen im Weiteren immer in einer Dimension, d.h. ohne Vektoren. Der Preis für diese Ungenauigkeit ist, dass wir uns über Vorzeichen Gedanken machen müssen.

Für den Fall der konstanten Beschleunigung

$$
a=\text { konstant }
$$

bekommt man für die Geschwindigkeit

$$
v(t)=v_{0}+a\left(t-t_{0}\right)
$$

Dabei bezeichnet $t_{0}$ den Zeitpunkt, an welchem wir unsere Berechnung starten - meist setzen wir $t_{0}=0$. Die Startgeschwindigkeit $v\left(t_{0}\right)=v_{0}$ ist die Geschwindigkeit zum Zeitpunkt des Starts unserer Berechnung. Für den Ort bekommt man:

$$
s(t)=s_{0}+v_{0}\left(t-t_{0}\right)+\frac{a}{2}\left(t-t_{0}\right)^{2}
$$

Dabei ist $s_{0}=s\left(t_{0}\right)$ die Koordinate des Ortes, an dem wir unsere Rechnung starten.

\paragraph*{Case study 1: Keine Beschleunigung}
Beschleunigung $a=0$, Startzeit $t_{0}=0$ bedeutet konstante Geschwindigkeit: Das System behält seine Anfangsgeschwindigkeit $v_{0}$, da ja die Abwesenheit von Beschleunigung bedeutet, dass sich die Geschwindigkeit nicht ändert. Der Ort ändert sich linear mit der Zeit:

$$
v(t)=v_{0}, s(t)=v_{0} t
$$

\paragraph*{Case Study 2: Bremsweg}
Bei $t_{0}=0$ rennt eine Katze auf die Strasse. Sie fahren mit einer Geschwindigkeit $v_{0}$ und Ihre Bremsen können eine Bremsbeschleunigung $a_{B}$ (näherungsweise konstant, negativ, Auto: $-8 \mathrm{~m} / \mathrm{s}^{2}$ bis $-10 \mathrm{~m} / \mathrm{s}^{2}$, Velo: $-5.5 \mathrm{~m} / \mathrm{s}^{2}$ (normal) bis $-7 \mathrm{~m} / \mathrm{s}^{2}$ (Profi) aufbringen. Wie gross ist Ihr Anhalteweg $s_{A}$ ?

Der totale Anhalteweg setzt sich zusammen aus einer Reaktionsphase, der Dauer $t_{R}$ und der Länge $s_{R}$, die Sie brauchen, um die Situation zu realisieren und auf die Bremse zu stehen, und dem Weg $s_{B}$, den Ihre Bremse braucht, das Auto auf null abzubremsen. $t_{R}$ ist physiologisch vorgegeben (etwa $1 \mathrm{~s}$ ).

Sie erhalten für $s_{R}$

$$
s_{R}=v_{0} t_{R}
$$

Wir suchen nun die Zeit $t_{B}$, die es braucht bis $a_{B}$ die Geschwindigkeit auf null reduziert hat (Startzeit $t_{0}=0$, wir beginnen mit der Zeitmessung nach der Reaktionsphase):

$$
v\left(t_{B}\right)=0=v_{0}+a\left(t_{B}-t_{0}\right) \Rightarrow-\frac{v_{0}}{a_{B}}=t_{B}
$$

Damit ergibt sich für die Länge des Weges:

$$
s_{B}=v_{0} t_{B}+\frac{a_{B}}{2} t_{B}^{2}=\frac{v_{0}^{2}}{2 a_{B}}
$$

Der gesamte Anhalteweg ist die Summe der beiden berechneten Wege:

$$
s_{A}=s_{R}+s_{B}=v_{0} t_{R}+\frac{v_{0}^{2}}{2 a_{B}}
$$

\paragraph*{Case Study 3: Die BMS - Formeln der Bewegung}
Aus der BMS kennen Sie vielleicht die folgenden Formeln für $t_{0}=, s_{0}=0, v_{0}=0$ und $a \neq 0$. Diese Formeln gestatten es Innen, Bewegungen mit konstanter Beschleunigung, welche aus einer Ruhelage beginnen, rechnerisch zu erfassen. Die Tabelle liest sich wie folgt: Die Spaltentitel geben die gesuchte Grösse an. Die Zeilentitel geben die Variable an, welche in der Formel nicht vorkommt.

\renewcommand{\arraystretch}{1.8}
\begin{center}
\begin{tabular}{c|c|c|c|c}
- & $t$ & $s$ & $v$ & $a$  \\
\hline
$t$ & - & $s=\frac{v^{2}}{2 a}$ & $v=\sqrt{2 a s}$ & $a=\frac{v^{2}}{2 s}$ \\
\hline
$s$ & $t=\frac{a}{v}$ & - & $v=a t$ & $a=\frac{v}{t}$  \\
\hline
$v$ & $t=\sqrt{\frac{2 s}{a}}$ & $s=\frac{1}{2} a t^{2}$ & - & $a=\frac{2 s}{t^{2}}$  \\
\hline
$a$ & $t=\frac{2 s}{v}$ & $s=\frac{v}{2}$ & $v=\frac{2s}{t}$ & - \\
\hline
\end{tabular}
\end{center}
\renewcommand{\arraystretch}{1}

Tabelle 1: Diese Beziehungen gelten nur und ausschliesslich bei konstanter Beschleunigung, $s(0)=v(0)=0$ und eindimensionaler Bewegung.

\paragraph*{Case Study 4: Freier Fall und schiefer Wurf}
Wenn Sie nahe dem Erdboden (< 10 km über Meeresspiegel) einen Körper loslassen, erfährt dieser unabhängig von seiner Masse eine Beschleunigung von:

$$
\vec{a}=\left(\begin{array}{c}
0 \\
0 \\
-g
\end{array}\right)
$$

Folgende Bemerkungen sind relevant:

\begin{itemize}
  \item Wir arbeiten vektoriell.

  \item Die $z$-Achse bezeichnet die Vertikale. Die Konstante $g$ ist gleich $9.81 \mathrm{~m} / \mathrm{s}^{2}$.

  \item Wir vernachlässigen hier alle Reibungseffekte (Luftwiderstand!)

  \item Die Beschleunigung in $x$ - und in $y$-Richtung ist 0 !

\end{itemize}

Da die Beschleunigung in $x$ - und in $y$-Richtung 0 ist, sind die Geschwindigkeiten in diese Richtungen konstant. Da sie sich nicht verändern, entsprechen sie den Anfangsgeschwindigkeiten $\left(v_{x, 0}, v_{y, 0}\right)$. Da die Beschleunigung in $z$-Richtung konstant ist, stehen Zeit und die Geschwindigkeit in diese Richtung in einem linearen Zusammenhang. Diese Sachverhalte in einem Vektor zusammengefasst ergeben folglich:

$$
\vec{v}(t)=\left(\begin{array}{c}
v_{x, 0} \\
v_{y, 0} \\
v_{z, 0}-g t
\end{array}\right)
$$

Dabei gilt für die Anfangswerte (die sogenannten «Anfangsbedingungen»):

$$
\vec{v}(0)=\vec{v}_{0}=\left(\begin{array}{l}
v_{x, 0} \\
v_{y, 0} \\
v_{z, 0}
\end{array}\right), \vec{s}(0)=\vec{s}_{0}=\left(\begin{array}{l}
s_{x, 0} \\
s_{y, 0} \\
s_{z, 0}
\end{array}\right)
$$

Wichtiges Detail: Die obigen Formeln gelten nur, falls unsere Rechnungen bei einer Anfangszeit $t_{0}=0$ beginnen!

Für die Position $\vec{s}(t)$ des Körpers ergibt sich:

$$
\vec{s}(t)=\left(\begin{array}{c}
s_{x, 0}+v_{x, 0} \cdot t \\
s_{y, 0}+v_{y, 0} \cdot t \\
s_{z, 0}+v_{z, 0} \cdot t-g \frac{t^{2}}{2}
\end{array}\right)
$$

Beachten Sie, dass die Position manchmal mit $\vec{r}(t)$ statt $\vec{s}(t)$ bezeichnet wird. An den grundlegenden Sachverhalten ändert dies aber nichts - man benutzt bloss einen anderen Buchstaben. Die beiden letzten Gleichungen stellen die allgemeine Lösung für den schiefen Wurf dar. Beachten Sie, dass diese Formeln nur unter den Anfangsbedingungen und der Annahme $t_{0}=0$ gelten. Beachten Sie eine Konsequenz aus GI. (14): Die Bewegung in vertikaler Richtung hängt überhaupt nicht davon ab, ob eine Bewegung in horizontaler Richtung stattfindet! Die Anfangsgeschwindigkeiten in $x$ und $y$-Richtung haben keinen Einfluss auf die Bewegung in $z$-Richtung und umgekehrt.

\section*{Physikalische Kräfte}
Wir unterscheiden zwei Arten von Kräften:

\begin{itemize}
  \item Fundamentalkräfte:
  \begin{itemize}
      \item Wirken zwischen Elementarteilchen\\
      $\rightarrow$ Aus innen ergeben sich alle anderen Kräfte.
      \item Sind die Grundlage unseres Verständnisses der Dynamik.
      \item Können sehr exakt berechnet werden.
      \item Fluktuieren nicht im klassischen Sinne (Es gibt aber Quantenfluktuationen)
  \end{itemize}
  \item Nicht-fundamentale Kräfte:
  \begin{itemize}
      \item Treten als Mittelwerte aus dem Zusammenspiel der Fundamentalkräfte auf.
      \item Werden oft phänomenologisch ${ }^{1}$ beschrieben.
      \item Da sie Mittelwerte sind, fluktuieren sie.
  \end{itemize}
\end{itemize}

Der heutige Wissensstand, d.h. der Stand des aktuellen Irrtums, geht von vier Fundamentalkräften aus. Dies ist eine Konsequenz des Standardmodells der Teilchenphysik. Zwei davon erklären praktisch alle beobachtbaren Phänomene des Alltags ausser dem Sonnenlicht und dem Licht anderer Sterne:

\begin{itemize}
  \item Gravitation
  \item Elektromagnetismus
\end{itemize}

Die Physik kennt ferner zwei weitere Fundamentalkräfte:
\begin{itemize}
    \item Die starke Kernkraft, welche die Atomkerne zusammenhält und die Verwandlung von Kernteilchen beschreibt.
    \footnotetext{${ }^{1}$ Sich auf Grössen beziehend, welche nicht unbedingt fundamental sein müssen. Beispiel: Der Druck eines Gases ist keine fundamentale Eigenschaft der Gasatome, aber eine Konsequenz der Interaktion vieler Gasatome mit einer Gefässwand.}
    \item Die schwache Kernkraft, die für radioaktive Zerfälle mitverantwortlich ist und ebenfalls Kernteilchen verwandelt.
\end{itemize}

Das Standardmodell ist sehr erfolgreich; allerdings erklärt es einige Phänomene nicht und ist damit sicher noch nicht die abschliessende Antwort.

Nicht-fundamentale Kräfte ergeben sich aus dem Zusammenspiel vieler einzelner Teilchen. Zu den nicht fundamentalen Kräften gehören:

\begin{itemize}
  \item Reibungskräfte
  \item Luftwiderstand
  \item Kapillarkräfte
  \item elastische Kräfte in deformierbaren Medien
  \item $\ldots$
\end{itemize}

Nicht-fundamentale Kräfte ergeben sich aus der Interaktion sehr vieler Teilchen, welche durch fundamentale Kräfte interagieren. Beispiel: Reibungskräfte ergeben sich aus der elektromagnetischen Interaktion zwischen Atomen ${ }^{2}$. In der Praxis ist es natürlich unmöglich, die Wechselwirkung zwischen allen Atomen in einem Turnschuh und dem Boden auf der Rennbahn zu berechnen. Es zeigt sich aber, dass sich zum Beispiel Reibungskräfte relativ gut (d.h. in sehr gut Näherung) als Funktion von einfach zu messenden Grössen und einigen wenigen Materialkonstanten beschreiben lassen.
\begin{center}
\begin{equation*}
\begin{array}{|c|c|c|c|}
\hline \text { Kraft } & \text { Vektorform } & \text { «Zahlform» } & \text { Grössen } \\
\hline \begin{array}{l}
\text { Gravitations- } \\
\text { kraft zwi- } \\
\text { schen zwei } \\
\text { Massen }
\end{array} & \vec{F}_{12}=-\gamma \frac{m_1 m_2}{\left|\vec{r}_1-\vec{r}_2\right|^2} \underbrace{\frac{\vec{r}_1-\vec{r}_2}{\left|\vec{r}_1-\vec{r}_2\right|}}_{\vec{n}_{12}} & F_{12}=-\gamma \frac{m_1 m_2}{r^2} & \begin{array}{l}
m_i: \text { Masse } i[\mathrm{~kg}] \\
\vec{F}_{12}: \text { Kraft auf Masse } m_1, \text { verursacht durch } m_2 \\
\vec{r}_i \text { Position von Masse } i[\mathrm{~m}] \\
\vec{n}_{12}=\frac{\vec{r}_{12}}{\left|\vec{r}_{12}\right|}: \text { Einheitsvektor von } m_2 \text { zu } m_1 \\
\gamma=6.67 \cdot 10^{-11} \frac{\mathrm{Nm}^2}{\mathrm{~kg}^2}: \text { Gravitationskonstante }
\end{array} \\
\hline \begin{array}{l}
\text { Spezialfall: } \\
\text { Schwerkraft } \\
\text { auf der Erde }
\end{array} & \vec{F}_G=\left(\begin{array}{c}
0 \\
0 \\
-m g
\end{array}\right) & F_G=m g & \begin{array}{l}
m: \text { Masse }[\mathrm{kg}] \\
g=9.81 \frac{\mathrm{m}}{\mathrm{s}^2} \text { : Erdbeschleunigung }
\end{array} \\
\hline \begin{array}{l}
\text { Kraft zwi- } \\
\text { schen zwei } \\
\text { Ladungen } \\
\text { Coulomb- } \\
\text { kraft }
\end{array} & \vec{F}_{12}=\frac{1}{4 \pi \varepsilon_0} \frac{q_1 q_2}{\left|\vec{r}_1-\vec{r}_2\right|^2} \underbrace{\mid \vec{r}_1-\vec{r}_2}_{\vec{r}_{12}} & F_{12}=\frac{1}{4 \pi \varepsilon_0} \frac{q_1 q_2}{r^2} & \begin{array}{l}
q_i: \text { Ladung } i[\mathrm{C}] \\
\vec{F}_{12}: \text { Kraft auf Ladung } q_1, \text { verursacht durch } q_2 \\
\vec{r}_i \text { Position von Ladung } i[\mathrm{~m}] \\
\vec{n}_{12}=\frac{\vec{r}_{12}}{\mid \vec{r}_{12}}: \text { : Einheitsvektor von } q_2 \text { zu } q_1 \\
\varepsilon_0=8.859 \cdot 10^{-12} \frac{\mathrm{C}^2}{\mathrm{Jm}}: \text { «Elektrische Feldkonstante» }
\end{array} \\
\hline \text { Federkraft } & \vec{F}_{\mathrm{s}}=-k(|\vec{x}|-L) \frac{\vec{x}}{|\vec{x}|} & F_{\mathrm{S}}=-k(x-L) & \begin{array}{l}
F_s: \text { Federkraft }[\mathrm{N}] \\
k: \text { Federkonstante }\left[\mathrm{kg} / \mathrm{s}^2\right] \\
L: \text { Ruhelänge der Feder }[\mathrm{m}] \\
\vec{x}: \text { Länge der Feder }[\mathrm{m}]
\end{array} \\
\hline
\end{array}
\end{equation*}
\end{center}

Tabelle 2: Einige physikalische Kräfte.
\footnotetext{2 Wobei hier auch noch quantenmechanische Effekte berücksichtigt werden müssen. Auf den ersten
} Blick einfache Kräfte sind erstaunlich kompliziert, wenn man die phänomenologische Ebene verlässt!

Einige Bemerkungen:

\begin{itemize}
  \item Gravitation:
  \begin{itemize}
      \item Zwischen zwei Massen wirkt eine Anziehungskraft.
      \item Die Kraft ist umgekehrt proportional zum Abstand der Massen.
      \item Die Kraft ist proportional zum Produkt der beiden Massen.
      \item Die Gravitation kann nicht abgeschirmt werden.
  \end{itemize}
  \item Die Schwerkraft auf der Erde ergibt sich aus der Gravitationskraft, wenn man die Masse der Erde und den Radius der Erde in die Formel der Gravitationskraft einsetzt.
  \item Coulomb-Kraft:
  \begin{itemize}
      \item Die Grundursache elektrischer Phänomene ist die Existenz elektrischer Ladungen.
      \item Es gibt zwei Arten von Ladungen: Positive und negative Ladung.
      \item Zwischen Ladungen wirken Kräfte: Gleiche Ladungen stossen sich ab, ungleiche ziehen sich an.
      \item In allen physikalischen Prozessen bleibt die Ladung exakt erhalten.
      \item Warum zwei und nicht sieben Ladungen? Heute haben wir Erklärungen für die Dualität der Ladungen, ebenso für deren Erhaltung; diese Erklärungen stützen sich auf tiefliegende mathematische Einsichten.
      \item Einheit der Ladung: Coulomb [C]. Ein Elektron trägt eine Ladung von - $e$, wobei $e$ die Elementarladung $e=1.602189 \cdot 10^{-19} \mathrm{C}$ beträgt.
  \end{itemize}
  \item In der Tabelle zeigen wir Ihnen eine lineare Feder, d.h. eine Feder, deren Kraft proportional zur Auslenkung der Feder ist. Dies ist, wie immer, eine Annäherung an die realen Verhältnisse.
\end{itemize}
