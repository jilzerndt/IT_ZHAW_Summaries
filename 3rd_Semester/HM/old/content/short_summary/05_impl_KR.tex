\begin{KR}{Nullstellenverfahren - Praktisches Vorgehen}
\begin{enumerate}
    \item Voraussetzungen prüfen:
    \begin{itemize}
        \item Existiert Nullstelle? (z.B. Vorzeichenwechsel)
        \item Sind Startwerte geeignet?
        \item Ist Funktion ausreichend glatt? (für Newton)
    \end{itemize}
    
    \item Verfahren wählen:
    \begin{itemize}
        \item Newton: Wenn Ableitung verfügbar und Startwert nahe Lösung
        \item Sekanten: Wenn keine Ableitung aber zwei Startwerte nahe Lösung
        \item Fixpunkt: Wenn Funktion kontraktiv
    \end{itemize}
    
    \item Implementierung:
    \begin{itemize}
        \item Konvergenzkriterien definieren
        \item Maximale Iterationszahl festlegen
        \item Fehlerabschätzung einbauen
        \item Divergenzschutz implementieren
    \end{itemize}
    
    \item Auswertung:
    \begin{itemize}
        \item Konvergenzverhalten prüfen
        \item Fehler abschätzen
        \item Ergebnis validieren
    \end{itemize}
\end{enumerate}
\end{KR}

\begin{KR}{Implementation iterativer Verfahren}
\begin{enumerate}
    \item Wählen Sie Startvektor $x^{(0)}$
    \item Wählen Sie Abbruchkriterien:
        \begin{itemize}
            \item Maximale Iterationszahl $k_{max}$
            \item Toleranz $\epsilon$ für Änderung $\|x^{(k+1)} - x^{(k)}\|$
            \item Toleranz für Residuum $\|Ax^{(k)} - b\|$
        \end{itemize}
    \item Führen Sie Iteration durch bis Kriterien erfüllt
\end{enumerate}
\end{KR}

\begin{KR}{Implementation von Nullstellenverfahren}
\begin{enumerate}
    \item Grundstruktur:
    \begin{itemize}
        \item Funktion $f(x)$ definieren
        \item Ableitung $f'(x)$ falls nötig
        \item Abbruchkriterien festlegen
        \item Iterations-Schleife implementieren
    \end{itemize}
    
    \item Abbruchkriterien:
    \begin{lstlisting}[language=Python, style=base]
def newton(f, df, x0, eps=1e-6, maxiter=100):
    x = x0
    for i in range(maxiter):
        fx = f(x)
        if abs(fx) < eps:  # Funktionswert klein genug
            return x
        dfx = df(x)
        if abs(dfx) < eps:  # Division durch Null vermeiden
            raise ValueError("Ableitung nahe Null")
        x_new = x - fx/dfx
        if abs(x_new - x) < eps:  # Aenderung klein genug
            return x_new
        x = x_new
    raise RuntimeError("Keine Konvergenz")
    \end{lstlisting}
    
    \item Fehlerbehandlung beachten:
    \begin{itemize}
        \item Division durch Null bei $f'(x) = 0$
        \item Keine Konvergenz
        \item Über-/Unterlauf bei großen Zahlen
    \end{itemize}
\end{enumerate}
\end{KR}