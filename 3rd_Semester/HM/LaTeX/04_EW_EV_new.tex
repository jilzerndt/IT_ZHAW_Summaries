\section{Eigenwerte und Eigenvektoren}

\subsection{Komplexe Zahlen}

\begin{concept}{Komplexe Zahlen}\\
Die Menge der komplexen Zahlen $\mathbb{C}$ erweitert die reellen Zahlen $\mathbb{R}$ durch Einführung der imaginären Einheit $i$ mit der Eigenschaft:
$$i^2 = -1$$

Eine komplexe Zahl $z$ ist ein geordnetes Paar $(x,y)$ mit $x,y \in \mathbb{R}$:
$$z = x + iy$$

Die Menge aller komplexen Zahlen ist definiert als:
$$\mathbb{C} = \{z \mid z = x + iy \text{ mit } x,y \in \mathbb{R}\}$$
\end{concept}

\begin{definition}{Bestandteile komplexer Zahlen}
\vspace{1mm}\\
\begin{minipage}[t]{0.45\textwidth}
    \textbf{Realteil:} $\operatorname{Re}(z) = x$
\end{minipage}
\begin{minipage}[t]{0.5\textwidth}
    \textbf{Konjugation:} $\overline{z} = x - iy$
\end{minipage}
\vspace{2mm}\\
\begin{minipage}[t]{0.45\textwidth}
    \textbf{Imaginärteil:} $\operatorname{Im}(z) = y$
\end{minipage}
\begin{minipage}[t]{0.53\textwidth}
    \textbf{Betrag:} $|z| = \sqrt{x^2 + y^2} = \sqrt{z \cdot z^*}$
\end{minipage}
\end{definition}

\begin{concept}{Darstellungsformen}
\begin{itemize}
    \item Normalform: $z = x + iy$
    \item Trigonometrische Form: $z = r(\cos\varphi + i\sin\varphi)$
    \item Exponentialform: $z = re^{i\varphi}$
\end{itemize}
$x = r\cos\varphi, \quad y = r\sin\varphi, \quad r = \sqrt{x^2 + y^2}\\
    \varphi = \arcsin\left(\frac{y}{r}\right) = \arccos\left(\frac{x}{r}\right)\\
    e^{i\varphi} = \cos\varphi + i\sin\varphi \text{ (Euler-Formel)}$

\end{concept}

\begin{KR}{Umrechnung zwischen Darstellungsformen komplexer Zahlen}
\paragraph{Von Normalform in trigonometrische Form/Exponentialform}
\begin{enumerate}
   \item Berechne Betrag $r = \sqrt{x^2 + y^2}$
   \item Berechne Winkel mit einer der Formeln:
   \begin{itemize}
       \item $\varphi = \arctan(\frac{y}{x})$ falls $x > 0$
       \item $\varphi = \arctan(\frac{y}{x}) + \pi$ falls $x < 0$
       \item $\varphi = \frac{\pi}{2}$ falls $x = 0, y > 0$
       \item $\varphi = -\frac{\pi}{2}$ falls $x = 0, y < 0$
       \item $\varphi$ unbestimmt falls $x = y = 0$
   \end{itemize}
   \item Trigonometrische Form: $z = r(\cos\varphi + i\sin\varphi)$
   \item Exponentialform: $z = re^{i\varphi}$
\end{enumerate}

\paragraph{Von trigonometrischer Form in Normalform}
\begin{enumerate}
   \item Realteil: $x = r\cos\varphi$
   \item Imaginärteil: $y = r\sin\varphi$
   \item Normalform: $z = x + iy$
\end{enumerate}

\paragraph{Von Exponentialform in Normalform/trigonometrische Form}
\begin{enumerate}
   \item Trigonometrische Form durch Euler-Formel:
   $$re^{i\varphi} = r(\cos\varphi + i\sin\varphi)$$
   \item Dann wie oben in Normalform umrechnen
\end{enumerate}

\paragraph{Wichtige Hinweise:}
\begin{itemize}
   \item Achten Sie auf das korrekte Quadranten beim Winkel
   \item Winkelfunktionen im Bogenmaß verwenden
   \item Bei Umrechnung in Normalform Euler-Formel nutzen
   \item Vorzeichen bei Exponentialform beachten
\end{itemize}

\end{KR}

\begin{example2}{Darstellungsformen}
Gegeben: $z = 3 - 11i$ in Normalform
$$r = \sqrt{3^2 + 11^2} = \sqrt{130}, \quad \varphi = \arcsin\left(\frac{11}{\sqrt{130}}\right) = 1.3 \text{rad} = 74.74^{\circ}$$
\textbf{Trigonometrische Form:} $z = \sqrt{130}(\cos(1.3) + i\sin(1.3))$
\vspace{2mm}\\
\textbf{Exponentialform:} $z = \sqrt{130}e^{i\cdot 1.3}$
\end{example2}

\begin{theorem}{Rechenoperationen mit komplexen Zahlen}\\
Für $z_1 = x_1 + iy_1$ und $z_2 = x_2 + iy_2$ gilt:
\vspace{1mm}\\
\begin{minipage}[t]{0.45\textwidth}
    \textbf{Addition:}\\
    $z_1 + z_2 = (x_1 + x_2) + i(y_1 + y_2)$
\end{minipage}
\hspace{3mm}
\begin{minipage}[t]{0.45\textwidth}
    \textbf{Subtraktion:}\\
    $z_1 - z_2 = (x_1 - x_2) + i(y_1 - y_2)$
\end{minipage}

\vspace{2mm}
\textbf{Multiplikation:}
\begin{align*}
    z_1 \cdot z_2 &= (x_1x_2 - y_1y_2) + i(x_1y_2 + x_2y_1)\\
    &= r_1r_2e^{i(\varphi_1 + \varphi_2)} \text{ (in Exponentialform)}
\end{align*}

\textbf{Division:}
\begin{align*}
    \frac{z_1}{z_2} &= \frac{z_1 \cdot z_2^*}{z_2 \cdot z_2^*} = \frac{(x_1x_2 + y_1y_2) + i(y_1x_2 - x_1y_2)}{x_2^2 + y_2^2}\\
    &= \frac{r_1}{r_2}e^{i(\varphi_1 - \varphi_2)} \text{ (in Exponentialform)}
\end{align*}
\end{theorem}

\begin{theorem}{Potenzen und Wurzeln}\\
Für eine komplexe Zahl in Exponentialform $z = re^{i\varphi}$ gilt:
\begin{itemize}
    \item n-te Potenz: $z^n = r^ne^{in\varphi} = r^n(\cos(n\varphi) + i\sin(n\varphi))$
    \item n-te Wurzel: $z_k = \sqrt[n]{r}e^{i\frac{\varphi + 2\pi k}{n}}$, $k = 0,1,\ldots,n-1$
\end{itemize}
\end{theorem}

\begin{lemma}{Fundamentalsatz der Algebra}\\
Eine algebraische Gleichung n-ten Grades mit komplexen Koeffizienten:
$$a_nz^n + a_{n-1}z^{n-1} + \cdots + a_1z + a_0 = 0$$
besitzt in $\mathbb{C}$ genau n Lösungen (mit Vielfachheiten gezählt).
\end{lemma}

\columnbreak

\subsection{Eigenwerte und Eigenvektoren}

\begin{definition}{Eigenwerte und Eigenvektoren}\\
Für eine Matrix $A \in \mathbb{R}^{n\times n}$ heißt $\lambda \in \mathbb{C}$ Eigenwert von $A$, wenn es einen Vektor $x \in \mathbb{C}^n \backslash \{0\}$ gibt mit:
$$Ax = \lambda x$$
Der Vektor $x$ heißt dann Eigenvektor zum Eigenwert $\lambda$.
\end{definition}

\begin{concept}{Bestimmung von Eigenwerten}\\
Ein Skalar $\lambda$ ist genau dann Eigenwert von $A$, wenn gilt:
$$\det(A - \lambda I_n) = 0$$
Diese Gleichung heißt charakteristische Gleichung. Das zugehörige Polynom
$$p(\lambda) = \det(A - \lambda I_n)$$
ist das charakteristische Polynom von $A$.
\end{concept}

\begin{theorem}{Eigenschaften von Eigenwerten}\\
Für eine Matrix $A \in \mathbb{R}^{n\times n}$ gilt:
\begin{itemize}
    \item $\det(A) = \prod_{i=1}^n \lambda_i$ (Produkt der Eigenwerte)
    \item $\operatorname{tr}(A) = \sum_{i=1}^n \lambda_i$ (Summe der Eigenwerte)
    \item Bei einer Dreiecksmatrix sind die Diagonalelemente die Eigenwerte
    \item Ist $\lambda$ Eigenwert von $A$, so ist $\frac{1}{\lambda}$ Eigenwert von $A^{-1}$
\end{itemize}
\end{theorem}

\begin{concept}{Vielfachheiten}\\
Für einen Eigenwert $\lambda$ unterscheidet man:
\begin{itemize}
    \item Algebraische Vielfachheit: Vielfachheit als Nullstelle des charakteristischen Polynoms
    \item Geometrische Vielfachheit: Dimension des Eigenraums $= n - \operatorname{rg}(A-\lambda I_n)$
\end{itemize}
Die geometrische Vielfachheit ist stets kleiner oder gleich der algebraischen Vielfachheit.
\end{concept}

\begin{KR}{Bestimmung von Eigenwerten und Eigenvektoren}
\begin{enumerate}
    \item Charakteristisches Polynom aufstellen: $p(\lambda) = \det(A-\lambda I_n)$
    \item Eigenwerte durch Lösen von $p(\lambda) = 0$ bestimmen
    \item Für jeden Eigenwert $\lambda_i$:
        \begin{itemize}
            \item System $(A-\lambda_i I_n)x = 0$ aufstellen
            \item Lösungsraum = Eigenraum bestimmen
            \item Basis des Eigenraums = linear unabhängige Eigenvektoren
        \end{itemize}
\end{enumerate}
\end{KR}

\begin{example2}{Eigenwertberechnung}
$$A = \begin{pmatrix} 1 & 0 & 0\\ 2 & 3 & 0\\ 0 & 1 & 2\end{pmatrix}$$
\begin{enumerate}
    \item Da $A$ eine Dreiecksmatrix ist, sind die Diagonalelemente die Eigenwerte:
    $$\lambda_1 = 1, \lambda_2 = 3, \lambda_3 = 2$$
    \item $\det(A) = \lambda_1\cdot\lambda_2\cdot\lambda_3 = 6$
    \item $\operatorname{tr}(A) = \lambda_1 + \lambda_2 + \lambda_3 = 6$
    \item Spektrum: $\sigma(A) = \{1,2,3\}$
\end{enumerate}
\end{example2}

\subsubsection{Numerische Berechnung von Eigenwerten}

\begin{concept}{Ähnliche Matrizen}\\
Zwei Matrizen $A,B \in \mathbb{R}^{n\times n}$ heißen ähnlich, wenn es eine reguläre Matrix $T$ gibt mit:
$$B = T^{-1}AT$$

Eine Matrix $A$ heißt diagonalisierbar, wenn sie ähnlich zu einer Diagonalmatrix $D$ ist:
$$D = T^{-1}AT$$
\end{concept}

\begin{theorem}{Eigenschaften ähnlicher Matrizen}\\
Für ähnliche Matrizen $A$ und $B = T^{-1}AT$ gilt:
\begin{enumerate}
    \item $A$ und $B$ haben dieselben Eigenwerte mit gleichen algebraischen Vielfachheiten
    \item Ist $x$ Eigenvektor von $B$ zum Eigenwert $\lambda$, so ist $Tx$ Eigenvektor von $A$ zum Eigenwert $\lambda$
    \item Bei Diagonalisierbarkeit:
    \begin{itemize}
        \item Die Diagonalelemente von $D$ sind die Eigenwerte von $A$
        \item Die Spalten von $T$ sind die Eigenvektoren von $A$
    \end{itemize}
\end{enumerate}
\end{theorem}

\begin{definition}{Spektralradius}
Der Spektralradius einer Matrix $A$ ist definiert als:
$$\rho(A) = \max\{|\lambda| \mid \lambda \text{ ist Eigenwert von } A\}$$
Er gibt den Betrag des betragsmäßig größten Eigenwerts an.
\end{definition}

\subsubsection{Iterative Verfahren}

\begin{concept}{Von-Mises-Iteration (Vektoriteration)}\\
Für eine diagonalisierbare Matrix $A$ mit Eigenwerten $|\lambda_1| > |\lambda_2| \geq \cdots \geq |\lambda_n|$ konvergiert die Folge:
$$v^{(k+1)} = \frac{Av^{(k)}}{\|Av^{(k)}\|_2}, \quad
\lambda^{(k+1)} = \frac{(v^{(k)})^TAv^{(k)}}{(v^{(k)})^Tv^{(k)}}$$
gegen einen Eigenvektor $v$ zum betragsmäßig größten Eigenwert $\lambda_1$.
\end{concept}

\begin{KR}{Von-Mises-Iteration durchführen}
\begin{enumerate}
    \item Wähle Startvektor $v^{(0)}$ mit $\|v^{(0)}\|_2 = 1$
    \item Für $k = 0,1,2,\ldots$:
    \begin{itemize}
        \item Berechne $w^{(k)} = Av^{(k)}$
        \item Normiere: $v^{(k+1)} = \frac{w^{(k)}}{\|w^{(k)}\|_2}$
        \item Berechne Rayleigh-Quotienten $\lambda^{(k+1)}$
        \item Prüfe Konvergenz
    \end{itemize}
\end{enumerate}
\end{KR}

\begin{example2}{Von-Mises-Iteration}{Berechnung des größten Eigenwerts}
\begin{lstlisting}[language=Python, style=basesmol]
import numpy as np
def power_iteration(A, tol=1e-10, max_iter=100):
    n = A.shape[0]
    v = np.random.rand(n)
    v = v / np.linalg.norm(v)
    for i in range(max_iter):
        w = A @ v
        v_new = w / np.linalg.norm(w)
        # Rayleigh-Quotient
        lambda_k = v_new.T @ A @ v_new
        if np.linalg.norm(v_new - v) < tol:
            return lambda_k, v_new
        v = v_new
    return lambda_k, v_new
\end{lstlisting}
\end{example2}

\begin{concept}{QR-Verfahren}\\
Das QR-Verfahren transformiert die Matrix $A$ iterativ in eine obere Dreiecksmatrix, deren Diagonalelemente die Eigenwerte sind:
\begin{enumerate}
    \item Initialisierung: $A_0 := A$, $P_0 := I_n$
    \item Für $i = 0,1,2,\ldots$:
    \begin{itemize}
        \item QR-Zerlegung: $A_i = Q_iR_i$
        \item Neue Matrix: $A_{i+1} = R_iQ_i$
        \item Update: $P_{i+1} = P_iQ_i$
    \end{itemize}
\end{enumerate}
\end{concept}

\begin{KR}{QR-Verfahren anwenden}
\begin{enumerate}
    \item Matrix $A_0 = A$ vorbereiten
    \item In jedem Schritt $i$:
    \begin{itemize}
        \item QR-Zerlegung mit Householder oder Givens
        \item Neue Matrix durch Multiplikation $R_iQ_i$
        \item Konvergenz prüfen: Subdiagonalelemente $\approx$ 0?
    \end{itemize}
    \item Eigenwerte: Diagonalelemente der Endmatrix
    \item Eigenvektoren: Spalten von $P = P_1P_2\cdots P_k$
\end{enumerate}
\end{KR}

\begin{example2}{QR-Verfahren}{Implementation in Python}
\begin{lstlisting}[language=Python, style=basesmol]
def qr_algorithm(A, tol=1e-10, max_iter=100):
    n = A.shape[0]
    Q_prod = np.eye(n)
    A_k = A.copy()
    
    for k in range(max_iter):
        # QR decomposition
        Q, R = np.linalg.qr(A_k)
        # New iteration
        A_k = R @ Q
        # Update transformation matrix
        Q_prod = Q_prod @ Q
        
        # Check convergence
        if np.abs(np.tril(A_k, -1)).max() < tol:
            break
            
    return np.diag(A_k), Q_prod
\end{lstlisting}
\end{example2}

\begin{remark2}{Numerische Stabilität}
\begin{itemize}
    \item QR-Verfahren ist numerisch stabiler als Vektoriteration
    \item Findet alle Eigenwerte, nicht nur den größten
    \item Benötigt mehr Rechenaufwand
    \item Konvergiert linear für reelle, quadratisch für komplexe Eigenwerte
\end{itemize}
\end{remark2}