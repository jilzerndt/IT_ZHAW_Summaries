\section{Rechnerarithmetik}

\subsection{Zahlendarstellung und Maschinenzahlen}

Maschinendarstellbare Zahlen $M$ zur Basis $B$ :

$$
M=\left\{x \in \mathbb{R} \mid x= \pm 0 . m_{1} m_{2} m_{3} \ldots m_{n} \cdot B^{ \pm e_{1} e_{2} \ldots e_{l}}\right\} \cup\{0\}
$$

Dabei gilt $m_{1} \neq 0, m_{i}, e_{i} \in\{0,1, \ldots, B-1\}$ für $i \neq 0$ und $B \in \mathbb{N}(B>1)$

\section*{Der Wert $\widehat{\boldsymbol{\omega}}$ einer solchen Zahl ist definiert als}
$$
\widehat{\omega}=\sum_{i=1}^{n} m_{i} B^{\hat{\mathrm{e}}-i}, \quad \hat{\mathrm{e}}=\sum_{l=1}^{l} e_{i} B^{l-i}
$$

$x$ wird als n -stellige Gleitpunktzahl zur Basis $B$ bezeichnet.\\
Beispiel: $\underbrace{0.3211}_{n=4} \cdot \underbrace{4^{12}}_{l=2}$

\begin{enumerate}
  \item $\hat{e}=1 \cdot 4^{1}+2 \cdot 4^{0}=6$
  \item $\widehat{\omega}=3 \cdot 4^{5}+2 \cdot 4^{4}+1 \cdot 4^{3}+1 \cdot 4^{2}=3664$
\end{enumerate}

\section*{Gleitpunktzahlen}
\begin{itemize}
  \item Single Precision (32 Bit) $\quad V=1$ Bit $\quad E=8$ Bit $\quad M=23$ Bit
  \item Double Precision (64 Bit) $V=1$ Bit $\quad E=11$ Bit $\quad M=52$ Bit
\end{itemize}

Bei allgemeiner Basis $B$ gilt (Maschinengenauigkeit $=e p s$ )

$$
\text { eps }:=\frac{B}{2} \cdot B^{-n}, \quad e p s_{10}:=5 \cdot 10^{-n}
$$

Sie bezeichnet den maximalen relativen Fehler, der durch Rundungen entstehen kann.

$$
\left|\frac{r d(x)-x}{x}\right| \leq 5 \cdot 10^{-n} \quad\left(\text { da } x \geq 10^{e-1}\right)
$$

\subsection{Approximations- und Rundungsfehler}

Die Maschinenzahlen sind nicht gleichmässig verteilt. Bei jedem Rechner gibt es eine grösste $\left(x_{\max }\right)$ und kleinste $\left(x_{\min }\right)$ positive Maschinenzahl.

\begin{itemize}
  \item $x_{\text {max }}=B^{e_{\max }}-B^{e_{\max }-n}=\left(1-B^{-n}\right) \cdot B^{e_{\max }}$
  \item $x_{\text {min }}=B^{e_{\text {min }}-1}$
\end{itemize}

\section*{Definition}
Gegeben sei eine Näherung $\tilde{x}$ zu einem exakten Wert $x$

\begin{itemize}
  \item Absoluter Fehler\\
$|\tilde{x}-x|$
  \item Relativer Fehler\\
$\left|\frac{\tilde{x}-x}{x}\right| b z w \cdot \frac{|\tilde{x}-x|}{|x|}$
\end{itemize}

\section*{Fehlerfortpflanzung bei Funktionsauswertungen / Konditionierung}
Näherung für den absoluten und relativen Fehler bei Funktionsauswertungen

$$
\begin{aligned}
\underbrace{|f(\tilde{x})-f(x)|}_{\text {absoluter Fehler von } f(x)} & \approx\left|f^{\prime}(x)\right| \cdot \underbrace{|\tilde{x}-x|}_{\text {absoluter Fehler von } x} \\
\underbrace{\frac{|f(\tilde{x})-f(x)|}{|f(x)|}}_{\text {relativer Fehler von } f(x)} & \approx \underbrace{\frac{\left|f^{\prime}(x)\right| \cdot|x|}{|f(x)|}}_{\text {Konditionszahl } K} . \quad \underbrace{|\tilde{x}-x|}_{\begin{array}{c}
|x| \\
\text { relativer Fehler von } x
\end{array}}
\end{aligned}
$$

Den Faktor $K$ nennt man Konditionszahl.

$$
K:=\frac{\left|f^{\prime}(x)\right| \cdot|x|}{|f(x)|}
$$

Relative Fehlervergrösserung von $x$, bei einer Funktionsauswertung von $f(x)$.

\begin{center}
\begin{tabular}{ll}
- Gut konditionierte Probleme & Konditionszahl ist klein $(\leq 1)$ \\
- Schlecht konditionierte Probleme & Konditionszahl ist gross $(>1)$ \\
\end{tabular}
\end{center}