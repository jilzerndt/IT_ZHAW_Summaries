\section{Additional Examples}

\subsection{Rechnerarithmetik}

\begin{example2}{Werteberechnung ausführlich} 
Gegeben sei die Maschinenzahl zur Basis $B=2$:
$$x = \underbrace{0.1101}_{\text{n=4}} \cdot \underbrace{2^{101}_2}_{\text{l=3}}$$

\textbf{1. Normalisierung prüfen:}
\begin{itemize}
    \item $m_1 = 1 \neq 0$ $\rightarrow$ normalisiert
\end{itemize}

\textbf{2. Exponent berechnen:}
\begin{align*}
\hat{e} &= 1 \cdot 2^2 + 0 \cdot 2^1 + 1 \cdot 2^0 \\
&= 4 + 0 + 1 = 5
\end{align*}

\textbf{3. Wert berechnen:}
\begin{align*}
\hat{\omega} &= 1 \cdot 2^{5-1} + 1 \cdot 2^{5-2} + 0 \cdot 2^{5-3} + 1 \cdot 2^{5-4} \\
&= 1 \cdot 2^4 + 1 \cdot 2^3 + 0 \cdot 2^2 + 1 \cdot 2^1 \\
&= 16 + 8 + 0 + 2 \\
&= 26
\end{align*}

Also ist $x = 26$
\end{example2}

\begin{example2}{Weitere Beispiele}
\begin{enumerate}
    \item Basis 10: $0.3141 \cdot 10^2$
    \begin{itemize}
        \item Normalisiert, da $m_1 = 3 \neq 0$
        \item $\hat{e} = 2$
        \item $\hat{\omega} = 3 \cdot 10^1 + 1 \cdot 10^0 + 4 \cdot 10^{-1} + 1 \cdot 10^{-2} = 31.41$
    \end{itemize}
    
    \item Basis 16 (hex): $0.A5F \cdot 16^3$
    \begin{itemize}
        \item Normalisiert, da $m_1 = A = 10 \neq 0$
        \item $\hat{e} = 3$
        \item $\hat{\omega} = 10 \cdot 16^2 + 5 \cdot 16^1 + 15 \cdot 16^0 = 2655$
    \end{itemize}
\end{enumerate}
\end{example2}

\begin{example2}{Werteberechnung} Berechnung einer Zahl zur Basis B=2:
\begin{minipage}{0.45\textwidth}
    $$\underbrace{0.1011}_{\text{n=4}} \cdot \underbrace{2^{3}}_{\text{l=1}}$$
\end{minipage}
\begin{minipage}[t]{0.5\textwidth}
    1. Exponent: $\hat{e} = 3$ \\ 
    2. Wert: $\hat{\omega} = 1\cdot2^2 + 0\cdot2^1 + 1\cdot2^0 + 1\cdot2^{-1}$ \\
    $= 4 + 0 + 1 + 0.5 = 5.5$
\end{minipage}
\end{example2}

\subsection{Numerische Lösung von Nullstellenproblemen}

\begin{example2}{Fixpunktiteration} Nullstellen von $p(x)=x^3-x+0.3$\\
    %TODO: check if this is correct and/or relevant - either correct or replace with better example
Fixpunktgleichung: $x_{n+1} = F(x_n) = x_n^3 + 0.3$
\begin{enumerate}
    \item $F'(x) = 3x^2$ steigt monoton
    \item Für $I=[0,0.5]$: $F(0)=0.3 > 0$, $F(0.5)=0.425 < 0.5$
    \item $\alpha = \max_{x \in [0,0.5]} |3x^2| = 0.75 < 1$
    \item Konvergenz für Startwerte in $[0,0.5]$ gesichert
\end{enumerate}
\end{example2}



\begin{example2}{Newton-Verfahren} Berechnung von $\sqrt[3]{2}$
Nullstellenproblem: $f(x)=x^3-2$
\vspace{1mm}\\
\begin{minipage}[t]{0.65\textwidth}
    \vspace{-3mm}
    Ableitung: $f'(x)=3x^2$, Startwert $x_0=1$
    \begin{enumerate}
        \item $x_1 = 1 - \frac{1^3-2}{3 \cdot 1^2} = 1.333333$
        \item $x_2 = 1.333333 - \frac{1.333333^3-2}{3 \cdot 1.333333^2} = 1.259921$
        \item $x_3 = 1.259921 - \frac{1.259921^3-2}{3 \cdot 1.259921^2} = 1.259921$
    \end{enumerate}
\end{minipage}
\begin{minipage}[t]{0.3\textwidth}
    Quadratische Konvergenz sichtbar durch schnelle Annäherung an $\sqrt[3]{2} \approx 1.259921$
\end{minipage}
\end{example2}

\subsection{Numerische Lösung von LGS}

\begin{example2}{Pivotisierung in der Praxis}
Betrachten Sie das System:
$$\begin{psmallmatrix}
0.001 & 1\\
1 & 1
\end{psmallmatrix}
\begin{psmallmatrix}
x_1\\
x_2
\end{psmallmatrix} = 
\begin{psmallmatrix}
1\\
2
\end{psmallmatrix}$$

\paragraph{Ohne Pivotisierung:}
Division durch 0.001 führt zu großen Rundungsfehlern:
$$x_1 \approx 1000 \cdot (1 - x_2)$$

\paragraph{Mit Pivotisierung:}
Nach Zeilenvertauschung:
$$\begin{psmallmatrix}
1 & 1\\
0.001 & 1
\end{psmallmatrix}
\begin{psmallmatrix}
x_1\\
x_2
\end{psmallmatrix} = 
\begin{psmallmatrix}
2\\
1
\end{psmallmatrix}$$
Liefert stabile Lösung: $x_1 = 1$, $x_2 = 1$
\end{example2}

\begin{example2}{LR-Zerlegung mit Pivotisierung}
Gegeben sei das System:
$$A = \begin{psmallmatrix}
1 & 2 & 1\\
3 & 8 & 1\\
0 & 4 & 1
\end{psmallmatrix}, \quad b = \begin{psmallmatrix}
2\\
3\\
5
\end{psmallmatrix}$$

\paragraph{1. Erste Spalte}
Max Element in 1. Spalte: $|a_{21}| = 3$, tausche Z1 und Z2:
$$P_1 = \begin{psmallmatrix}
0 & 1 & 0\\
1 & 0 & 0\\
0 & 0 & 1
\end{psmallmatrix}, \quad 
A^{(1)} = \begin{psmallmatrix}
3 & 8 & 1\\
1 & 2 & 1\\
0 & 4 & 1
\end{psmallmatrix}$$

Eliminationsfaktoren: $l_{21} = \frac{1}{3}$, $l_{31} = 0$\\
Nach Elimination:
$$A^{(2)} = \begin{psmallmatrix}
3 & 8 & 1\\
0 & -\frac{2}{3} & \frac{2}{3}\\
0 & 4 & 1
\end{psmallmatrix}$$

\paragraph{2. Zweite Spalte}
Max Element: $|a_{32}| = 4$, tausche Z2 und Z3:
$$P_2 = \begin{psmallmatrix}
1 & 0 & 0\\
0 & 0 & 1\\
0 & 1 & 0
\end{psmallmatrix}$$

Eliminationsfaktor: $l_{32} = -\frac{1}{6}$\\
Nach Elimination:
$$R = \begin{psmallmatrix}
3 & 8 & 1\\
0 & 4 & 1\\
0 & 0 & \frac{5}{6}
\end{psmallmatrix}$$

\paragraph{Endergebnis}
$$P = P_2P_1 = \begin{psmallmatrix}
0 & 1 & 0\\
0 & 0 & 1\\
1 & 0 & 0
\end{psmallmatrix}, \quad
L = \begin{psmallmatrix}
1 & 0 & 0\\
\frac{1}{3} & 1 & 0\\
0 & -\frac{1}{6} & 1
\end{psmallmatrix}$$

\paragraph{Lösung des Systems}
\begin{enumerate}
    \item $Pb = \begin{psmallmatrix} 3\\ 5\\ 2 \end{psmallmatrix}$
    \item $Ly = Pb$: $y = \begin{psmallmatrix} 3\\ 4\\ 1 \end{psmallmatrix}$
    \item $Rx = y$: $x = \begin{psmallmatrix} 1\\ 0\\ \frac{6}{5} \end{psmallmatrix}$
\end{enumerate}
\end{example2}

\begin{example2}{QR-Zerlegung}
Gegeben sei die Matrix:
$$A = \begin{psmallmatrix}
1 & 1\\
1 & 0\\
0 & 1
\end{psmallmatrix}$$

\paragraph{1. Erste Spalte}
$v_1 = \begin{psmallmatrix} 1\\ 1\\ 0 \end{psmallmatrix}$, 
$\|v_1\| = \sqrt{2}$

Householder-Vektor:
$w_1 = v_1 + \sqrt{2}\begin{psmallmatrix} 1\\ 0\\ 0 \end{psmallmatrix} = 
\begin{psmallmatrix} 1+\sqrt{2}\\ 1\\ 0 \end{psmallmatrix}$

Normierung:
$u_1 = \frac{1}{\sqrt{4+2\sqrt{2}}}
\begin{psmallmatrix} 1+\sqrt{2}\\ 1\\ 0 \end{psmallmatrix}$

Erste Householder-Matrix:
$$H_1 = I - 2u_1u_1^T = 
\begin{psmallmatrix}
-\frac{1}{\sqrt{2}} & -\frac{1}{\sqrt{2}} & 0\\
-\frac{1}{\sqrt{2}} & \frac{1}{\sqrt{2}} & 0\\
0 & 0 & 1
\end{psmallmatrix}$$

\paragraph{2. Zweite Spalte}
Nach Anwendung von $H_1$:
$$H_1A = \begin{psmallmatrix}
-\sqrt{2} & -\frac{1}{\sqrt{2}}\\
0 & \frac{1}{\sqrt{2}}\\
0 & 1
\end{psmallmatrix}$$

Untervektor für zweite Transformation:
$v_2 = \begin{psmallmatrix} \frac{1}{\sqrt{2}}\\ 1 \end{psmallmatrix}$

Analog zur ersten Transformation erhält man:
$$H_2 = \begin{psmallmatrix}
1 & 0 & 0\\
0 & -\frac{1}{\sqrt{5}} & -\frac{2}{\sqrt{5}}\\
0 & -\frac{2}{\sqrt{5}} & \frac{1}{\sqrt{5}}
\end{psmallmatrix}$$

\paragraph{Endergebnis}
$$Q = H_1^TH_2^T = \begin{psmallmatrix}
\frac{1}{\sqrt{2}} & \frac{1}{\sqrt{2}} & 0\\
\frac{1}{\sqrt{2}} & -\frac{1}{\sqrt{2}} & 0\\
0 & 0 & 1
\end{psmallmatrix}$$

$$R = H_2H_1A = \begin{psmallmatrix}
\sqrt{2} & 1\\
0 & \sqrt{2}\\
0 & 0
\end{psmallmatrix}$$

\paragraph{Verifikation}
\begin{itemize}
    \item $Q^TQ = QQ^T = I$ (Orthogonalität)
    \item $QR = A$ (bis auf Rundungsfehler)
    \item R ist obere Dreiecksmatrix
\end{itemize}
\end{example2}

\begin{example2}{Iterative Verfahren}{Vergleich Jacobi und Gauss-Seidel}
    %TODO: check if this is correct and/or relevant - either correct or replace with better example
System:
$$\begin{psmallmatrix}
4 & -1 & 0\\
-1 & 4 & -1\\
0 & -1 & 4
\end{psmallmatrix}x = \begin{psmallmatrix}
1\\
5\\
0
\end{psmallmatrix}$$

\begin{center}
\begin{tabular}{c|cc|cc}
k & \multicolumn{2}{c|}{Jacobi} & \multicolumn{2}{c}{Gauss-Seidel}\\
\hline
0 & $(0,0,0)^T$ & & $(0,0,0)^T$ &\\
1 & $(0.25,1.25,0)^T$ & 1.25 & $(0.25,1.31,0.08)^T$ & 1.31\\
2 & $(0.31,1.31,0.31)^T$ & 0.31 & $(0.33,1.33,0.33)^T$ & 0.02\\
3 & $(0.33,1.33,0.33)^T$ & 0.02 & $(0.33,1.33,0.33)^T$ & 0.00
\end{tabular}
\end{center}
\end{example2}

\begin{example2}{Konvergenzverhalten}
Betrachten Sie das System:
$$\begin{psmallmatrix}
4 & 1 & 0\\
1 & 4 & 1\\
0 & 1 & 4
\end{psmallmatrix}
\begin{psmallmatrix}
x_1\\
x_2\\
x_3
\end{psmallmatrix} =
\begin{psmallmatrix}
1\\
2\\
3
\end{psmallmatrix}$$

Die Matrix ist diagonaldominant:
$|a_{ii}| = 4 > 1 = \sum_{j\neq i} |a_{ij}|$

\begin{center}
\begin{tabular}{c|cc|cc}
k & \multicolumn{2}{c|}{Residuum} & \multicolumn{2}{c}{Rel. Fehler}\\
& Jacobi & G-S & Jacobi & G-S\\
\hline
0 & 3.74 & 3.74 & - & -\\
1 & 0.94 & 0.47 & 0.935 & 0.468\\
2 & 0.23 & 0.06 & 0.246 & 0.125\\
3 & 0.06 & 0.01 & 0.065 & 0.017\\
4 & 0.01 & 0.001 & 0.016 & 0.002
\end{tabular}
\end{center}

\textbf{Beobachtungen:}
\begin{itemize}
    \item Gauss-Seidel konvergiert etwa doppelt so schnell wie Jacobi
    \item Das Residuum fällt linear (geometrische Folge)
    \item Die Konvergenz ist gleichmäßig (keine Oszillationen)
\end{itemize}
\end{example2}

\columnbreak

\subsection{Eigenvektoren und Eigenwerte}

\begin{example2}{Darstellungsformen}
Gegeben: $z = 3 - 11i$ in Normalform
$$r = \sqrt{3^2 + 11^2} = \sqrt{130}, \quad \varphi = \arcsin(\frac{11}{\sqrt{130}}) = 1.3 \text{rad} = 74.74^{\circ}$$
\textbf{Trigonometrische Form:} $z = \sqrt{130}(\cos(1.3) + i\sin(1.3))$
\vspace{2mm}\\
\textbf{Exponentialform:} $z = \sqrt{130}e^{i\cdot 1.3}$
\end{example2}

\begin{example2}{Eigenwertberechnung}
%TODO: check if this is correct and/or relevant - either correct or replace with better example
$A = \begin{psmallmatrix} 1 & 0 & 0\\ 2 & 3 & 0\\ 0 & 1 & 2\end{psmallmatrix}$
\begin{enumerate}
    \item Da $A$ eine Dreiecksmatrix ist, sind die Diagonalelemente die \\
    Eigenwerte:
    $\lambda_1 = 1, \lambda_2 = 3, \lambda_3 = 2$
    \item $\det(A) = \lambda_1\cdot\lambda_2\cdot\lambda_3 = 6$
    \item $\operatorname{tr}(A) = \lambda_1 + \lambda_2 + \lambda_3 = 6$
    \item Spektrum: $\sigma(A) = \{1,2,3\}$
\end{enumerate}
\end{example2}

\begin{example2}{Von-Mises-Iteration}
Berechne größten Eigenwert der Matrix:
\vspace{2mm}\\
$A = \begin{psmallmatrix}
4 & -1 & 1\\
-1 & 3 & -2\\
1 & -2 & 3
\end{psmallmatrix}$, $\quad$
Startvektor: $v^{(0)} = \begin{psmallmatrix}1\\ 0\\ 0\end{psmallmatrix}$

\begin{center}
\begin{tabular}{c|c|c}
k & $v^{(k)}$ & $\lambda^{(k)}$ \\\hline
0 & $(1, 0, 0)^T$ & -\\
1 & $(0.970, -0.213, 0.119)^T$ & 4.000\\
2 & $(0.957, -0.239, 0.164)^T$ & 4.827\\
3 & $(0.953, -0.244, 0.178)^T$ & 4.953\\
4 & $(0.952, -0.245, 0.182)^T$ & 4.989
\end{tabular}
\end{center}

Konvergenz gegen $\lambda_1 \approx 5$ \\ Eigenvektor $v \approx (0.952, -0.245, 0.182)^T$
\end{example2}

\begin{example2}{QR-Verfahren}
Matrix:
$$A = \begin{psmallmatrix}
2 & -1 & 1\\
-1 & 3 & 0\\
1 & 0 & 1
\end{psmallmatrix}$$

\paragraph{QR-Iteration:}
\begin{enumerate}
    \item $A_0 = A$
    \item Nach erster Iteration:
    $$A_1 = \begin{psmallmatrix}
    3.21 & -0.83 & 0.62\\
    -0.83 & 2.13 & 0.41\\
    0.62 & 0.41 & 0.66
    \end{psmallmatrix}$$
    \item Nach 5 Iterationen:
    $$A_5 \approx \begin{psmallmatrix}
    4 & 0 & 0\\
    0 & 1 & 0\\
    0 & 0 & 1
    \end{psmallmatrix}$$
\end{enumerate}

Die Diagonalelemente von $A_5$ sind die Eigenwerte: $\lambda_1 = 4, \lambda_2 = 1, \lambda_3 = 1$
\end{example2}