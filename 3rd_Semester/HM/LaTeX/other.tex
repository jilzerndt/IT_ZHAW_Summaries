\begin{KR}{Fehlerabschätzung für Nullstellen}
\begin{enumerate}
    \item Voraussetzungen prüfen:
    \begin{itemize}
        \item Funktion $f$ muss stetig sein
        \item Nullstelle muss von ungerader Ordnung sein (Vorzeichenwechsel)
    \end{itemize}
    \item Fehlertoleranz $\epsilon$ festlegen
    \item Nullstelleneinschluss prüfen:
    \begin{itemize}
        \item Berechne $f(x_n-\epsilon)$ und $f(x_n+\epsilon)$
        \item Prüfe Vorzeichenwechsel: $f(x_n-\epsilon) \cdot f(x_n+\epsilon) < 0$
    \end{itemize}
    \item Fehler auswerten:
    \begin{itemize}
        \item Falls Vorzeichenwechsel: $|x_n-\xi| < \epsilon$
        \item Falls kein Vorzeichenwechsel: $\epsilon$ vergrößern und wiederholen
    \end{itemize}
    \item Zusätzliche Konvergenzprüfungen:
    \begin{itemize}
        \item Relative Änderung: $\frac{|x_n-x_{n-1}|}{|x_n|} < \epsilon_r$
        \item Residuum: $|f(x_n)| < \epsilon_f$
    \end{itemize}
\end{enumerate}
\end{KR}

\begin{KR}{Fehlerabschätzung in der Praxis}
\begin{enumerate}
    \item Numerische Fehlerabschätzung
    \begin{itemize}
        \item Absolute Änderung: $|x_n - x_{n-1}| < \epsilon_1$
        \item Funktionswert: $|f(x_n)| < \epsilon_2$
        \item Vorzeichenwechsel prüfen: $f(x_n-\epsilon) \cdot f(x_n+\epsilon) < 0$
    \end{itemize}
    
    \item Theoretische Fehlerabschätzung
    \begin{itemize}
        \item Fixpunktiteration: $|x_n-\bar{x}| \leq \frac{\alpha^n}{1-\alpha}|x_1-x_0|$
        \item Newton-Verfahren: $|x_{n+1}-\bar{x}| \leq c|x_n-\bar{x}|^2$
        \item Sekantenverfahren: $|x_{n+1}-\bar{x}| \leq c|x_n-\bar{x}|^{1.618}$
    \end{itemize}
    
    \item Zusätzliche Sicherheitsaspekte
    \begin{itemize}
        \item Divergenzcheck durchführen
        \item Überlauf/Unterlauf prüfen
        \item Division durch Null vermeiden
    \end{itemize}
\end{enumerate}
\end{KR}