\section{Prüfungstipps}

\begin{concept}{Allgemeine Hinweise}
\begin{itemize}
    \item Prüfungszeit: 120 Minuten für 6 Aufgaben $\rightarrow$ ca. 20 min pro Aufgabe
    \item Alle Aufgaben gleich gewichtet (10 Punkte)
    \item Lösungsweg muss vollständig und nachvollziehbar sein
    \item Zwischenschritte sind wichtig - auch bei falschen Endergebnissen gibt es Punkte
    \item Python-Code muss lauffähig sein und kommentiert werden
\end{itemize}
\end{concept}

\begin{KR}{Was ist immer dabei?}
\begin{enumerate}
    \item Rechnerarithmetik/Konditionierung:
    \begin{itemize}
        \item Maschinengenauigkeit berechnen
        \item Darstellungsbereich bestimmen
        \item Konditionszahl analysieren
        \item Fehlerfortpflanzung abschätzen
    \end{itemize}
    
    \item Nullstellenverfahren:
    \begin{itemize}
        \item Newton oder Fixpunktiteration
        \item Konvergenznachweis (Banach)
        \item A-priori/a-posteriori Abschätzungen
        \item 2-3 Iterationsschritte von Hand
    \end{itemize}
    
    \item Lineare Gleichungssysteme:
    \begin{itemize}
        \item Direkte Verfahren (Gauss, LR)
        \item Iterative Verfahren (Jacobi, Gauss-Seidel)
        \item Konvergenzbetrachtungen
        \item Praktische Anwendungen
    \end{itemize}
    
    \item Eigenwerte:
    \begin{itemize}
        \item Charakteristisches Polynom
        \item Eigenvektoren berechnen
        \item QR-Verfahren
        \item Von-Mises Iteration
    \end{itemize}
\end{enumerate}
\end{KR}

\begin{concept}{Typische Fallstricke}
\begin{itemize}
    \item Bei Konditionierung:
    \begin{itemize}
        \item Vorzeichen bei Fehlerabschätzungen beachten
        \item Grenzwertbetrachtungen durchführen
        \item Auf Sonderfälle achten (z.B. $x \to 0$)
    \end{itemize}
    
    \item Bei Nullstellenproblemen:
    \begin{itemize}
        \item Konvergenzradius beachten
        \item Startwert sinnvoll wählen
        \item Abbruchkriterien definieren
    \end{itemize}
    
    \item Bei LGS:
    \begin{itemize}
        \item Pivotisierung nicht vergessen
        \item Zeilenvertauschungen dokumentieren
        \item Diagonaldominanz prüfen
    \end{itemize}
    
    \item Bei Eigenwerten:
    \begin{itemize}
        \item Vielfachheiten unterscheiden
        \item Auf komplexe Eigenwerte achten
        \item QR-Schritte sauber durchführen
    \end{itemize}
\end{itemize}
\end{concept}

\begin{KR}{Effiziente Prüfungsstrategie}
\begin{enumerate}
    \item Erste Durchsicht:
    \begin{itemize}
        \item Alle Aufgaben überfliegen
        \item Schwierigkeitsgrad einschätzen
        \item Zeitplan erstellen
    \end{itemize}
    
    \item Bei jeder Aufgabe:
    \begin{itemize}
        \item Methode identifizieren
        \item Zwischenschritte planen
        \item Ergebnisse verifizieren
    \end{itemize}
    
    \item Zeit einteilen:
    \begin{itemize}
        \item Einfache Aufgaben zuerst
        \item Zeit für Kontrolle einplanen
        \item Nicht zu lange an einer Aufgabe festbeissen
    \end{itemize}
    
    \item Python-Code:
    \begin{itemize}
        \item Grundgerüst schnell erstellen
        \item Gut kommentieren
        \item Ausgabe klar kennzeichnen
    \end{itemize}
\end{enumerate}
\end{KR}

\begin{example2}{Musterlösung strukturieren}
Für eine typische Aufgabe:
\begin{enumerate}
    \item Aufgabenstellung analysieren:
    \begin{itemize}
        \item Welche Methode ist gefragt?
        \item Was sind die gegebenen Grössen?
        \item Was ist das Ziel?
    \end{itemize}
    
    \item Lösungsweg skizzieren:
    \begin{itemize}
        \item Formeln aufschreiben
        \item Zwischenschritte planen
        \item Benötigte Berechnungen identifizieren
    \end{itemize}
    
    \item Berechnung durchführen:
    \begin{itemize}
        \item Schrittweise vorgehen
        \item Zwischenergebnisse notieren
        \item Einheiten mitführen
    \end{itemize}
    
    \item Ergebnis überprüfen:
    \begin{itemize}
        \item Plausibilitätskontrolle
        \item Dimensionskontrolle
        \item Eventuell Probe durchführen
    \end{itemize}
\end{enumerate}
\end{example2}