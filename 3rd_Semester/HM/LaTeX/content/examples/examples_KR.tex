\section{Kochrezepte und Beispiele}

\subsection{Rechnerarithmetik}

\begin{KR}{Maschinenzahlen analysieren}
\begin{enumerate}
    \item Anzahl Maschinenzahlen bestimmen:
    \begin{itemize}
        \item Basis $B$ identifizieren 
        \item Mantissenlänge $n$ bestimmen
        \item Exponentenlänge $l$ bestimmen
        \item Berechnen: $2 \cdot (B-1) \cdot B^{n-1} \cdot (B^l-1)$
    \end{itemize}
    \item Darstellungsbereich bestimmen:
    \begin{itemize}
        \item Größte Zahl: $x_{max} = (1-B^{-n})B^{e_{max}}$
        \item Kleinste positive Zahl: $x_{min} = B^{e_{min}-1}$
    \end{itemize}
    \item Maschinengenauigkeit berechnen:
    \begin{itemize}
        \item Allgemein: $eps = \frac{B}{2}B^{-n}$
        \item Dezimal: $eps = 5 \cdot 10^{-n}$
    \end{itemize}
\end{enumerate}
\end{KR}

\begin{example2}{Maschinenzahlen analysieren}
Gegeben: 15-stellige Gleitpunktzahlen mit 5-stelligem Exponenten im Dualsystem.

\paragraph{Lösung:}
\begin{enumerate}
    \item Basis $B=2$, $n=15$, $l=5$
    \item Anzahl verschiedener Zahlen:
    \begin{itemize}
        \item Pro Stelle: $B-1=1$ mögliche Ziffern
        \item Mantisse: $(B-1)B^{n-1} = 2^{14}$ Kombinationen
        \item Exponent: $B^l = 2^5 = 32$ Kombinationen
        \item Mit Vorzeichen: $2 \cdot 2^{14} \cdot 31 = 1\,015\,808$ Zahlen
    \end{itemize}
    \item Maschinengenauigkeit:
    $eps = \frac{2}{2}2^{-15} = 2^{-15} \approx 3.052 \cdot 10^{-5}$
\end{enumerate}
\end{example2}

\begin{KR}{Konditionierung analysieren}
\begin{enumerate}
    \item Ableitung $f'(x)$ bestimmen
    \item Konditionszahl berechnen:
    $$K = \frac{|f'(x)| \cdot |x|}{|f(x)|}$$
    \item Fehler abschätzen:
    \begin{itemize}
        \item Absolut: $|f(\tilde{x})-f(x)| \approx |f'(x)| \cdot |\tilde{x}-x|$
        \item Relativ: $\frac{|f(\tilde{x})-f(x)|}{|f(x)|} \approx K \cdot \frac{|\tilde{x}-x|}{|x|}$
    \end{itemize}
    \item Konditionierung beurteilen:
    \begin{itemize}
        \item $K \leq 1$: gut konditioniert
        \item $K > 1$: schlecht konditioniert
        \item $K \gg 1$: sehr schlecht konditioniert
    \end{itemize}
\end{enumerate}
\end{KR}

\begin{example2}{Konditionierung berechnen}
Für $f(x) = \sqrt{1+x^2}$ und $x_0 = 10^{-8}$:

\paragraph{Lösung:}
\begin{enumerate}
    \item $f'(x) = \frac{x}{\sqrt{1+x^2}}$
    \item $K = \frac{|x \cdot x|}{|\sqrt{1+x^2} \cdot (1+x^2)|} = \frac{x^2}{(1+x^2)^{3/2}}$
    \item Für $x_0 = 10^{-8}$:
    \begin{itemize}
        \item $K(10^{-8}) \approx 10^{-16}$ (gut konditioniert)
        \item Relativer Fehler wird um Faktor $10^{-16}$ verkleinert
    \end{itemize}
\end{enumerate}
\end{example2}

\begin{KR}{Auslöschung erkennen und vermeiden}
\begin{enumerate}
    \item Auslöschung identifizieren:
    \begin{itemize}
        \item Subtraktion ähnlich großer Zahlen
        \item Viele signifikante Stellen verschwinden
    \end{itemize}
    \item Alternative Berechnungen:
    \begin{itemize}
        \item Algebraische Umformungen
        \item Andere mathematische Identitäten
        \item Taylorentwicklung für kleine Werte
    \end{itemize}
    \item Beispiele für bessere Formeln:
    \begin{itemize}
        \item $\sqrt{1+x^2}-1 \rightarrow \frac{x^2}{\sqrt{1+x^2}+1}$
        \item $1-\cos(x) \rightarrow 2\sin^2(x/2)$
        \item $\ln(1+x) \rightarrow x-\frac{x^2}{2}$ für kleine $x$
    \end{itemize}
\end{enumerate}
\end{KR}

\begin{example2}{Auslöschung vermeiden}
Berechnen Sie $\sqrt{1+10^{-16}}-1$ mit 16 Dezimalstellen.

\paragraph{Lösung:}
\begin{enumerate}
    \item Direkte Berechnung:
    \begin{itemize}
        \item $\sqrt{1+10^{-16}} \approx 1.0000000000000000$
        \item $\sqrt{1+10^{-16}}-1 \approx 0.0000000000000000$
        \item Alle signifikanten Stellen gehen verloren
    \end{itemize}
    \item Alternative Formel:
    \begin{itemize}
        \item $\frac{10^{-16}}{\sqrt{1+10^{-16}}+1} \approx 5 \cdot 10^{-17}$
        \item Korrekte signifikante Stellen bleiben erhalten
    \end{itemize}
\end{enumerate}
\end{example2}

\raggedcolumns

\subsection{Nullstellenprobleme}

\begin{KR}{Nullstellenproblem systematisch lösen}
\begin{enumerate}
    \item Existenz prüfen:
    \begin{itemize}
        \item Intervall $[a,b]$ identifizieren
        \item Vorzeichenwechsel prüfen: $f(a) \cdot f(b) < 0$ 
        \item Stetigkeit von $f$ sicherstellen
    \end{itemize}
    
    \item Verfahren auswählen:
    \begin{itemize}
        \item Fixpunktiteration: wenn einfache Umformung $x = F(x)$ möglich
        \item Newton: wenn $f'(x)$ leicht berechenbar
        \item Sekantenverfahren: wenn $f'(x)$ schwer berechenbar
    \end{itemize}
    
    \item Konvergenz sicherstellen:
    \begin{itemize}
        \item Fixpunktiteration: $|F'(x)| < 1$ im relevanten Bereich
        \item Newton: $|\frac{f(x)f''(x)}{[f'(x)]^2}| < 1$ im relevanten Bereich
        \item Geeigneten Startwert wählen
    \end{itemize}
    
    \item Abbruchkriterien festlegen:
    \begin{itemize}
        \item Funktionswert: $|f(x_n)| < \epsilon_1$
        \item Iterationsschritte: $|x_{n+1}-x_n| < \epsilon_2$
        \item Maximale Iterationszahl
    \end{itemize}
\end{enumerate}
\end{KR}

\begin{example2}{Verfahrensauswahl}
Finden Sie die positive Nullstelle von $f(x) = x^3 - 2x - 5$

\paragraph{Vorgehen:}
\begin{enumerate}
    \item Existenz:
    \begin{itemize}
        \item $f(2) = -1 < 0$ und $f(3) = 16 > 0$
        \item $\Rightarrow$ Nullstelle in $[2,3]$
    \end{itemize}
    
    \item Verfahrenswahl:
    \begin{itemize}
        \item $f'(x) = 3x^2 - 2$ leicht berechenbar
        \item $\Rightarrow$ Newton-Verfahren geeignet
    \end{itemize}
    
    \item Konvergenzcheck:
    \begin{itemize}
        \item $f'(x) > 0$ für $x > 0.82$
        \item $f''(x) = 6x$ monoton
        \item $\Rightarrow$ Newton-Verfahren konvergiert
    \end{itemize}
\end{enumerate}
\end{example2}

\begin{KR}{Fixpunktiteration anwenden}
\begin{enumerate}
    \item Umformung vorbereiten:
    \begin{itemize}
        \item $f(x) = 0$ in $x = F(x)$ umformen
        \item Verschiedene Umformungen testen
        \item Form mit kleinstem $|F'(x)|$ wählen
    \end{itemize}
    
    \item Konvergenznachweis:
    \begin{itemize}
        \item Intervall $[a,b]$ bestimmen mit $F([a,b]) \subseteq [a,b]$
        \item $\alpha = \max_{x \in [a,b]} |F'(x)|$ berechnen
        \item Prüfen ob $\alpha < 1$
    \end{itemize}
    
    \item Fehlerabschätzung:
    \begin{itemize}
        \item A-priori: $|x_n-\bar{x}| \leq \frac{\alpha^n}{1-\alpha}|x_1-x_0|$
        \item A-posteriori: $|x_n-\bar{x}| \leq \frac{\alpha}{1-\alpha}|x_n-x_{n-1}|$
    \end{itemize}
    
    \item Iterationszahl bestimmen:
    $$n \geq \frac{\ln(\frac{\text{tol}(1-\alpha)}{|x_1-x_0|})}{\ln \alpha}$$
\end{enumerate}
\end{KR}

\begin{example2}{Fixpunktiteration}
Bestimmen Sie $\sqrt{3}$ mittels Fixpunktiteration.

\paragraph{Lösung:}
\begin{enumerate}
    \item Umformung: $x^2 = 3 \Rightarrow x = \frac{x^2+3}{2x} =: F(x)$
    
    \item Konvergenznachweis für $[1,2]$:
    \begin{itemize}
        \item $F'(x) = \frac{x^2-3}{2x^2}$
        \item $|F'(x)| \leq \alpha = 0.25 < 1$ für $x \in [1,2]$
        \item $F([1,2]) \subseteq [1,2]$
    \end{itemize}
    
    \item Für Genauigkeit $10^{-6}$:
    \begin{itemize}
        \item $|x_1-x_0| = |1.5-2| = 0.5$
        \item $n \geq \frac{\ln(10^{-6} \cdot 0.75/0.5)}{\ln 0.25} \approx 12$
    \end{itemize}
\end{enumerate}
\end{example2}

\begin{KR}{Newton-Verfahren anwenden}
\begin{enumerate}
    \item Vorbereitung:
    \begin{itemize}
        \item $f'(x)$ bestimmen
        \item Startwert $x_0$ nahe der Nullstelle wählen
        \item $f'(x_0) \neq 0$ prüfen
    \end{itemize}
    
    \item Iteration durchführen:
    $$x_{n+1} = x_n - \frac{f(x_n)}{f'(x_n)}$$
    
    \item Konvergenz prüfen:
    \begin{itemize}
        \item $|\frac{f(x)f''(x)}{[f'(x)]^2}| < 1$ im relevanten Bereich
        \item Quadratische Konvergenz erwarten
    \end{itemize}
    
    \item Fehlerabschätzung:
    \begin{itemize}
        \item $|x_n-\bar{x}| < \epsilon$ falls
        \item $f(x_n-\epsilon) \cdot f(x_n+\epsilon) < 0$
    \end{itemize}
\end{enumerate}
\end{KR}

\begin{example2}{Newton vs Sekanten}
Bestimmen Sie $\sqrt{2}$ mit beiden Verfahren.

\paragraph{Newton-Verfahren:} $f(x) = x^2-2$
\begin{itemize}
    \item $f'(x) = 2x$
    \item $x_0 = 1.5$
    \item $x_1 = 1.5 - \frac{1.5^2-2}{2\cdot1.5} = 1.4167$
    \item $x_2 = 1.4167 - \frac{1.4167^2-2}{2\cdot1.4167} = 1.4142$
\end{itemize}

\paragraph{Sekantenverfahren:}
\begin{itemize}
    \item $x_0 = 1$, $x_1 = 2$
    \item $x_2 = x_1 - \frac{x_1-x_0}{f(x_1)-f(x_0)}f(x_1) = 1.5$
    \item $x_3 = 1.5 - \frac{1.5-2}{1.5^2-2}1.5 = 1.4545$
    \item $x_4 = 1.4545 - \frac{1.4545-1.5}{1.4545^2-2}1.4545 = 1.4143$
\end{itemize}

\paragraph{Vergleich:}
\begin{itemize}
    \item Newton: Schnellere Konvergenz (quadratisch)
    \item Sekanten: Keine Ableitungsberechnung nötig
    \item Beide erreichen $10^{-4}$ Genauigkeit in 4-5 Schritten
\end{itemize}
\end{example2}

\begin{KR}{Typische Prüfungsaufgaben lösen}
\begin{enumerate}
    \item Theorieaufgaben:
    \begin{itemize}
        \item Konvergenzordnungen vergleichen
        \item Konvergenzbeweise durchführen
        \item Fehlerabschätzungen herleiten
    \end{itemize}
    
    \item Praktische Aufgaben:
    \begin{itemize}
        \item Existenz von Nullstellen nachweisen
        \item Geeignetes Verfahren auswählen
        \item 2-3 Iterationsschritte durchführen
        \item Konvergenzgeschwindigkeit vergleichen
    \end{itemize}
    
    \item Implementierungsaufgaben:
    \begin{itemize}
        \item Verfahren in Python implementieren
        \item Abbruchkriterien einbauen
        \item Konvergenzverhalten visualisieren
    \end{itemize}
\end{enumerate}
\end{KR}

\subsection{Lineare Gleichungssysteme}

\begin{KR}{Systematisches Vorgehen bei LGS}
\begin{enumerate}
    \item Eigenschaften der Matrix analysieren:
    \begin{itemize}
        \item Diagonaldominanz prüfen
        \item Konditionszahl berechnen oder abschätzen
        \item Symmetrie erkennen
    \end{itemize}
    
    \item Verfahren auswählen:
    \begin{itemize}
        \item Direkte Verfahren: für kleinere Systeme
        \item Iterative Verfahren: für große, dünnbesetzte Systeme
        \item Spezialverfahren: für symmetrische/bandförmige Matrizen
    \end{itemize}
    
    \item Implementation planen:
    \begin{itemize}
        \item Pivotisierung bei Gauss berücksichtigen
        \item Speicherbedarf beachten
        \item Abbruchkriterien festlegen
    \end{itemize}
\end{enumerate}
\end{KR}

\begin{example2}{Verfahrensauswahl}
Gegeben ist das LGS $Ax = b$ mit:
$$A = \begin{pmatrix} 
4 & -1 & 0 \\
-1 & 4 & -1 \\
0 & -1 & 4
\end{pmatrix}, \quad b = \begin{pmatrix} 1 \\ 0 \\ 1 \end{pmatrix}$$

\paragraph{Analyse:}
\begin{itemize}
    \item Matrix ist symmetrisch
    \item Streng diagonaldominant
    \item Dünnbesetzt (tridiagonal)
\end{itemize}

\paragraph{Verfahrenswahl:}
\begin{itemize}
    \item Gauss: möglich wegen kleiner Dimension
    \item Gauss-Seidel: konvergiert wegen Diagonaldominanz
    \item LR-Zerlegung: effizient wegen Bandstruktur
\end{itemize}
\end{example2}

\begin{KR}{Gauss-Algorithmus mit Pivotisierung}
\begin{enumerate}
    \item Vorbereitung:
    \begin{itemize}
        \item Erweiterte Matrix $(A|b)$ aufstellen
        \item Pivotisierungsstrategie wählen
    \end{itemize}
    
    \item Elimination:
    \begin{itemize}
        \item Für jede Spalte $i$:
        \item Pivotelement in Spalte $i$ bestimmen
        \item Zeilenvertauschung falls nötig
        \item Nullen unterhalb erzeugen
    \end{itemize}
    
    \item Rückwärtseinsetzen:
    $$x_i = \frac{b_i - \sum_{j=i+1}^n a_{ij}x_j}{a_{ii}}, \quad i=n,n-1,\ldots,1$$
    
    \item Kontrolle:
    \begin{itemize}
        \item Residuum $\|Ax-b\|$ berechnen
        \item Pivotisierungsschritte protokollieren
    \end{itemize}
\end{enumerate}
\end{KR}

\begin{example2}{Gauss mit Pivotisierung}
Lösen Sie $Ax = b$ mit:
$$A = \begin{pmatrix} 
1 & 2 & 1 \\
2 & 4 & -1 \\
4 & -2 & 1
\end{pmatrix}, \quad b = \begin{pmatrix} 1 \\ 2 \\ 0 \end{pmatrix}$$

\paragraph{Lösung:}
\begin{enumerate}
    \item Erste Spalte: Pivot $a_{31} = 4$ $\rightarrow$ Z1 $\leftrightarrow $ Z3
    $$\begin{pmatrix} 
    4 & -2 & 1 & | & 0 \\
    2 & 4 & -1 & | & 2 \\
    1 & 2 & 1 & | & 1
    \end{pmatrix}$$
    
    \item Eliminationsschritte:
    $$\begin{pmatrix} 
    4 & -2 & 1 & | & 0 \\
    0 & 5 & -1.5 & | & 2 \\
    0 & 2.5 & 0.75 & | & 1
    \end{pmatrix}$$
    $$\begin{pmatrix} 
    4 & -2 & 1 & | & 0 \\
    0 & 5 & -1.5 & | & 2 \\
    0 & 0 & 1.5 & | & 0.2
    \end{pmatrix}$$
    
    \item Rückwärtseinsetzen:
    \begin{align*}
        x_3 &= 0.2/1.5 = \frac{2}{15} \\
        x_2 &= (2 + 1.5 \cdot \frac{2}{15})/5 = 0.5 \\
        x_1 &= (0 + 2 \cdot 0.5 - 1 \cdot \frac{2}{15})/4 = 0.2
    \end{align*}
\end{enumerate}
\end{example2}

\begin{KR}{LR-Zerlegung durchführen}
\begin{enumerate}
    \item Zerlegung bestimmen:
    \begin{itemize}
        \item Gauss-Schritte durchführen
        \item Eliminationsfaktoren in $L$ speichern
        \item Resultierende Dreiecksmatrix ist $R$
    \end{itemize}
    
    \item System lösen:
    \begin{itemize}
        \item Vorwärtseinsetzen: $Ly = b$
        \item Rückwärtseinsetzen: $Rx = y$
    \end{itemize}
    
    \item Bei Pivotisierung:
    \begin{itemize}
        \item Permutationsmatrix $P$ erstellen
        \item $PA = LR$ speichern
        \item $Ly = Pb$ lösen
    \end{itemize}
\end{enumerate}
\end{KR}

\begin{example2}{LR-Zerlegung}
Bestimmen Sie die LR-Zerlegung von:
$$A = \begin{pmatrix}
2 & 1 & 1 \\
4 & -2 & -1 \\
-2 & 3 & -1
\end{pmatrix}$$

\paragraph{Lösung:}
\begin{enumerate}
    \item Eliminationsfaktoren:
    \begin{itemize}
        \item $l_{21} = 4/2 = 2$
        \item $l_{31} = -2/2 = -1$
        \item $l_{32} = 1/(-4) = -0.25$
    \end{itemize}
    
    \item Zerlegung:
    $$L = \begin{pmatrix}
    1 & 0 & 0 \\
    2 & 1 & 0 \\
    -1 & -0.25 & 1
    \end{pmatrix}$$
    $$R = \begin{pmatrix}
    2 & 1 & 1 \\
    0 & -4 & -3 \\
    0 & 0 & -0.25
    \end{pmatrix}$$
\end{enumerate}
\end{example2}

\begin{KR}{Iterative Verfahren implementieren}
\begin{enumerate}
    \item Matrix zerlegen:
    \begin{itemize}
        \item $A = L + D + R$ für Jacobi
        \item $A = (L+D) + R$ für Gauss-Seidel
    \end{itemize}
    
    \item Konvergenz prüfen:
    \begin{itemize}
        \item Diagonaldominanz
        \item Spektralradius der Iterationsmatrix
    \end{itemize}
    
    \item Iteration durchführen:
    \begin{itemize}
        \item Jacobi: $x^{(k+1)} = -D^{-1}(L+R)x^{(k)} + D^{-1}b$
        \item Gauss-Seidel: $x^{(k+1)} = -(D+L)^{-1}Rx^{(k)} + (D+L)^{-1}b$
    \end{itemize}
    
    \item Abbruchkriterien:
    \begin{itemize}
        \item Residuum: $\|Ax^{(k)}-b\| < \epsilon$
        \item Änderung: $\|x^{(k+1)}-x^{(k)}\| < \epsilon$
        \item Maximale Iterationen
    \end{itemize}
\end{enumerate}
\end{KR}

\begin{example2}{Jacobi vs Gauss-Seidel}
Gegeben sei das System:
$$\begin{pmatrix}
4 & -1 & 0 \\
-1 & 4 & -1 \\
0 & -1 & 4
\end{pmatrix} x = \begin{pmatrix}
1 \\
0 \\
1
\end{pmatrix}$$

\paragraph{Vergleich nach 4 Iterationen:}
\begin{center}
\begin{tabular}{c|cc}
k & Jacobi & Gauss-Seidel \\\hline
1 & 0.250 & 0.250 \\
2 & 0.281 & 0.297 \\
3 & 0.295 & 0.299 \\
4 & 0.298 & 0.300
\end{tabular}
\end{center}

\paragraph{Beobachtungen:}
\begin{itemize}
    \item Gauss-Seidel konvergiert schneller
    \item Beide Verfahren konvergieren monoton
    \item Konvergenz gegen $x_1 = 0.3$
\end{itemize}
\end{example2}

\subsection{Komplexe Zahlen und Eigenwertprobleme}

\begin{KR}{Komplexe Zahlen umrechnen}
\begin{enumerate}
    \item Normalform $\leftrightarrow$ Polarform:
    \begin{itemize}
        \item Betrag: $r = \sqrt{x^2 + y^2}$
        \item Winkel: $\varphi = \arctan(\frac{y}{x})$ (Quadrant beachten!)
        \item Normalform: $z = x + iy$
        \item Polarform: $z = r(\cos\varphi + i\sin\varphi) = re^{i\varphi}$
    \end{itemize}
    
    \item Rechenoperationen:
    \begin{itemize}
        \item Addition: $(x_1 + iy_1) + (x_2 + iy_2) = (x_1+x_2) + i(y_1+y_2)$
        \item Multiplikation: $r_1r_2e^{i(\varphi_1 + \varphi_2)}$
        \item Division: $\frac{r_1}{r_2}e^{i(\varphi_1 - \varphi_2)}$
        \item n-te Potenz: $r^ne^{in\varphi}$
    \end{itemize}
\end{enumerate}
\end{KR}

\begin{example2}{Komplexe Operationen}
Gegeben $z_1 = 1+i$ und $z_2 = 2-i$:

\paragraph{Umrechnung in Polarform:}
\begin{itemize}
    \item $z_1: r_1 = \sqrt{2}$, $\varphi_1 = \frac{\pi}{4}$
    \item $z_2: r_2 = \sqrt{5}$, $\varphi_2 = -\arctan(\frac{1}{2})$
\end{itemize}

\paragraph{Berechnungen:}
\begin{itemize}
    \item $z_1 \cdot z_2 = (2-i)(1+i) = (2+1) + i(2-1) = 3+i$
    \item $z_1^3 = (\sqrt{2})^3(\cos(\frac{3\pi}{4}) + i\sin(\frac{3\pi}{4}))$
\end{itemize}
\end{example2}

\begin{KR}{Eigenwerte bestimmen}
\begin{enumerate}
    \item Charakteristisches Polynom aufstellen:
    \begin{itemize}
        \item $p(\lambda) = \det(A-\lambda I)$ berechnen
        \item Auf Standardform bringen
    \end{itemize}
    
    \item Nullstellen bestimmen:
    \begin{itemize}
        \item Quadratische Formel für $n=2$
        \item Cardano-Formel für $n=3$
        \item Numerische Verfahren für $n>3$
    \end{itemize}
    
    \item Vielfachheiten bestimmen:
    \begin{itemize}
        \item Algebraische Vielfachheit: Nullstellenordnung
        \item Geometrische Vielfachheit: $n-\operatorname{rang}(A-\lambda I)$
    \end{itemize}
\end{enumerate}
\end{KR}

\begin{example2}{Charakteristisches Polynom}
Bestimmen Sie die Eigenwerte von:
$$A = \begin{pmatrix}
2 & -1 & 0 \\
-1 & 2 & -1 \\
0 & -1 & 2
\end{pmatrix}$$

\paragraph{Lösung:}
\begin{enumerate}
    \item $p(\lambda) = \det(A-\lambda I)$:
    $$\begin{vmatrix}
    2-\lambda & -1 & 0 \\
    -1 & 2-\lambda & -1 \\
    0 & -1 & 2-\lambda
    \end{vmatrix}$$
    
    \item Determinante entwickeln:
    $$p(\lambda) = (2-\lambda)^3 - 2(2-\lambda) = -\lambda^3 + 6\lambda^2 - 11\lambda + 6$$
    
    \item Nullstellen:
    $$\lambda_1 = 1, \lambda_2 = 2, \lambda_3 = 3$$
\end{enumerate}
\end{example2}

\begin{KR}{Eigenvektoren bestimmen}
\begin{enumerate}
    \item Für jeden Eigenwert $\lambda$:
    \begin{itemize}
        \item $(A-\lambda I)x = 0$ aufstellen
        \item Homogenes LGS lösen
        \item Lösungsvektor normieren
    \end{itemize}
    
    \item Bei mehrfachen Eigenwerten:
    \begin{itemize}
        \item Geometrische Vielfachheit bestimmen
        \item Basis des Eigenraums finden
    \end{itemize}
    
    \item Kontrolle:
    \begin{itemize}
        \item $Ax = \lambda x$ überprüfen
        \item Orthogonalität bei symmetrischen Matrizen
        \item Linear unabhängig?
    \end{itemize}
\end{enumerate}
\end{KR}

\begin{example2}{Eigenvektoren}
Bestimmen Sie die Eigenvektoren zum Eigenwert $\lambda=2$ der Matrix:
$$A = \begin{pmatrix}
2 & 1 & 0 \\
0 & 2 & 0 \\
0 & -1 & 2
\end{pmatrix}$$

\paragraph{Lösung:}
\begin{enumerate}
    \item $(A-2I)x = 0$:
    $$\begin{pmatrix}
    0 & 1 & 0 \\
    0 & 0 & 0 \\
    0 & -1 & 0
    \end{pmatrix} \begin{pmatrix}
    x_1 \\ x_2 \\ x_3
    \end{pmatrix} = \begin{pmatrix}
    0 \\ 0 \\ 0
    \end{pmatrix}$$
    
    \item Homogenes System lösen:
    \begin{itemize}
        \item $x_2 = 0$ (aus 1. Zeile)
        \item $x_1, x_3$ frei wählbar
    \end{itemize}
    
    \item Basis des Eigenraums:
    $$v_1 = \begin{pmatrix} 1 \\ 0 \\ 0 \end{pmatrix}, 
    v_2 = \begin{pmatrix} 0 \\ 0 \\ 1 \end{pmatrix}$$
\end{enumerate}
\end{example2}

\begin{KR}{QR-Algorithmus anwenden}
\begin{enumerate}
    \item Initialisierung:
    \begin{itemize}
        \item $A_0 := A$
        \item $Q_0 := I_n$
    \end{itemize}
    
    \item Iteration:
    \begin{itemize}
        \item QR-Zerlegung: $A_k = Q_kR_k$
        \item Neue Matrix: $A_{k+1} = R_kQ_k$
        \item Update: $P_{k+1} = P_kQ_k$
    \end{itemize}
    
    \item Abbruch wenn:
    \begin{itemize}
        \item Subdiagonalelemente klein
        \item Diagonalelemente konvergieren
        \item Maximale Iterationen erreicht
    \end{itemize}
\end{enumerate}
\end{KR}

\begin{example2}{QR-Iteration}
Führen Sie eine QR-Iteration durch für:
$$A = \begin{pmatrix}
1 & 1 \\
1 & 0
\end{pmatrix}$$

\paragraph{Lösung:}
\begin{enumerate}
    \item QR-Zerlegung von $A$:
    $$Q_1 = \frac{1}{\sqrt{2}}\begin{pmatrix}
    1 & -1 \\
    1 & 1
    \end{pmatrix}, 
    R_1 = \frac{1}{\sqrt{2}}\begin{pmatrix}
    \sqrt{2} & 1 \\
    0 & -1
    \end{pmatrix}$$
    
    \item Neue Matrix:
    $$A_1 = R_1Q_1 = \begin{pmatrix}
    1.5 & 0.5 \\
    0.5 & -0.5
    \end{pmatrix}$$
    
    \item Konvergenz nach mehreren Iterationen gegen:
    $$A_\infty \approx \begin{pmatrix}
    \phi & 0 \\
    0 & -\phi^{-1}
    \end{pmatrix}$$
    mit $\phi = \frac{1+\sqrt{5}}{2}$
\end{enumerate}
\end{example2}

\begin{KR}{Vektoriteration durchführen}
\begin{enumerate}
    \item Voraussetzungen prüfen:
    \begin{itemize}
        \item Matrix diagonalisierbar
        \item $|\lambda_1| > |\lambda_2|$
    \end{itemize}
    
    \item Iteration:
    \begin{itemize}
        \item $w^{(k)} = Av^{(k)}$
        \item $v^{(k+1)} = \frac{w^{(k)}}{\|w^{(k)}\|}$
        \item $\lambda^{(k+1)} = \frac{(v^{(k)})^TAv^{(k)}}{(v^{(k)})^Tv^{(k)}}$
    \end{itemize}
    
    \item Konvergenz:
    \begin{itemize}
        \item $v^{(k)} \to$ Eigenvektor zu $|\lambda_1|$
        \item $\lambda^{(k)} \to |\lambda_1|$
    \end{itemize}
\end{enumerate}
\end{KR}

\begin{example2}{Von-Mises-Iteration}
Bestimmen Sie den betragsmäßig größten Eigenwert von:
$$A = \begin{pmatrix}
3 & 1 \\
1 & 3
\end{pmatrix}$$

\paragraph{Lösung:}
\begin{enumerate}
    \item Start mit $v^{(0)} = \frac{1}{\sqrt{2}}\begin{pmatrix} 1 \\ 1 \end{pmatrix}$
    
    \item Erste Iteration:
    \begin{itemize}
        \item $w^{(0)} = \begin{pmatrix} 4 \\ 4 \end{pmatrix}$
        \item $v^{(1)} = \frac{1}{\sqrt{2}}\begin{pmatrix} 1 \\ 1 \end{pmatrix}$
        \item $\lambda^{(1)} = 4$
    \end{itemize}
    
    \item Ergebnis:
    \begin{itemize}
        \item Eigenvektor bereits gefunden
        \item Eigenwert $\lambda = 4$ ist korrekt
    \end{itemize}
\end{enumerate}
\end{example2}