\section{Aufgaben aus alten SEP}

\begin{example2}{Rechnerarithmetik}
Sie arbeiten auf einem Rechner mit 5-stelliger Gleitpunktarithmetik im Dualsystem. Für den Exponenten haben Sie zusätzlich zum Vorzeichen 3 Stellen zur Verfügung.

\paragraph{Aufgaben:}
\begin{enumerate}
    \item Was ist der grösste Exponent, den Sie speichern können?
    \item Welches ist die kleinste darstellbare positive Zahl, welche die grösste? Approximieren Sie diese durch normierte 4-stellige Maschinenzahlen im Dezimalsystem.
    \item Vergleichen Sie Ihren Rechner mit dem Ihres Kollegen, der mit einem 2-stelligen Hexadezimalsystem arbeitet. Wer rechnet genauer?
\end{enumerate}
\end{example2}

\begin{example2}{Konditionierung}
Gegeben ist die Funktion
$$f(x) = x^2 \cdot \sin(x)$$
mit ihrer Ableitung
$$f'(x) = 2x\sin(x) + x^2\cos(x)$$
und $x \in \mathbb{R}$ im Bogenmass.

\paragraph{Aufgaben:}
\begin{enumerate}
    \item Bestimmen Sie die Konditionszahl von $f(x)$ in Abhängigkeit von $x$.
    \item Berechnen Sie näherungsweise, mit welchem absoluten Fehler $x_0 = \pi/3$ höchstens behaftet sein darf, damit der relative Fehler von $f(x_0)$ höchstens 10\% beträgt.
    \item Bestimmen Sie numerisch das Verhalten der Konditionszahl von $f(x)$ für $x \to 0$ und geben Sie an, was das Ergebnis über die Konditionierung von $f(x)$ für $x = 0$ aussagt.
    \item Plotten Sie die Konditionszahl von $f(x)$ halblogarithmisch im Bereich $x \in [-2\pi, 3\pi]$ und geben Sie an, was das Ergebnis über die Konditionierung von $f(x)$ für $x \in \mathbb{R}$ aussagt.
\end{enumerate}
\end{example2}

\begin{example2}{Nullstellenprobleme}
Gesucht ist der Schnittpunkt der Funktionen $g(x) = \exp(x)$ und $h(x) = \sqrt{x+2}$.

\paragraph{Aufgaben:}
\begin{enumerate}
    \item Bestimmen Sie mit dem Newton-Verfahren und dem Startwert $x_0 = 0.5$ den Schnittpunkt bis auf einen absoluten Fehler von höchstens $10^{-7}$ genau.
    
    \item Zeigen Sie, dass der Fixpunkt der Iteration
    $$x_{k+1} = \ln(\sqrt{x_k + 2})$$
    gerade dem Schnittpunkt von $g(x)$ und $h(x)$ entspricht, und dass die Iteration für jeden Startwert $x_0$ im Intervall $[0.5, 1.5]$ konvergiert. Prüfen Sie dazu die Voraussetzungen des Banachschen Fixpunktsatzes.
    
    Der Startwert sei $x_0 = 0.5$. Bestimmen Sie mit Hilfe der a priori Abschätzung die Anzahl der benötigten Schritte, wenn der absolute Fehler der Näherung kleiner als $10^{-7}$ sein soll.
\end{enumerate}
\end{example2}

\begin{example2}{Lineare Gleichungssysteme}
Betrachten Sie das lineare Gleichungssystem $Ax = b$ mit
$$A = \begin{pmatrix}
1 & 1 & 1\\
2 & 2+\varepsilon & 5\\
4 & 6 & 8
\end{pmatrix}
\quad \text{und} \quad
b = \begin{pmatrix}
1\\
0\\
0
\end{pmatrix}$$
wobei $\varepsilon$ eine reelle Zahl ist.

\paragraph{Aufgaben:}
\begin{enumerate}
    \item Berechnen Sie manuell die LR-Zerlegung ohne Zeilenvertauschung der Matrix $A$ für ein allgemeines $\varepsilon \neq 0$.
    
    \item Bestimmen Sie mit Hilfe der Matrizen $L$ und $R$ aus (a) die Lösung von $A \cdot x = b$ für $\varepsilon = 2^{-52}$ (Maschinengenauigkeit). Schreiben Sie dazu ein Python-Skript und verwenden Sie numpy.linalg.solve().
    
    \item Lösen Sie nun das lin. Gleichungssystem $A \cdot x = b$ für $\varepsilon = 2^{-52}$ direkt mit numpy.linalg.solve(). Weshalb erhalten Sie nicht das gleiche Resultat wie bei b)? Begründen Sie!
\end{enumerate}
\end{example2}

\begin{example2}{Iterative Verfahren}
Mit Hilfe des Jacobiverfahrens soll die Lösung eines linearen Gleichungssystems mit einer n-dimensionalen tridiagonalen Systemmatrix $A$ bestimmt werden. Die Matrix und die rechte Seite sind definiert mit $c > 0$:
$$A = \begin{pmatrix}
c & -1 & & & \\
-1 & c & -1 & & \\
& -1 & c & -1 & \\
& & \ddots & \ddots & \ddots \\
& & & -1 & c
\end{pmatrix}
\quad \text{und} \quad
b = \begin{pmatrix}
c\\
0\\
\vdots\\
0\\
c
\end{pmatrix}$$

\paragraph{Aufgaben:}
\begin{enumerate}
    \item Bestimmen Sie die Matrix $B$ für $n=6$, $c=4.0$ und daraus $\|B\|_\infty$. Für welche $c \in \mathbb{R}$ gilt $\|B\|_\infty < 1$?
    
    \item Welche Anzahl Iterationen benötigt man maximal, um eine numerische Lösung mit einer Genauigkeit von $10^{-3}$ in der $\infty$-Norm zu erhalten? (a priori Abschätzung)
    
    \item Bestimmen Sie die Lösung mit Hilfe einer Implementierung des Jacobiverfahrens. Führen Sie die vorher bestimmte Anzahl von Iterationen aus!
    
    \item Bestimmen Sie die Genauigkeit der errechneten Lösung mit Hilfe der a posteriori Abschätzung?
\end{enumerate}
\end{example2}

\begin{example2}{Eigenwerte}
Gegeben ist die Matrix:
$$A = \begin{pmatrix}
13 & -4\\
30 & -9
\end{pmatrix}$$

\paragraph{Aufgaben:}
\begin{enumerate}
    \item Bestimmen Sie mit Hilfe des charakteristischen Polynoms die Eigenwerte von $A$, und bestimmen Sie die zugehörigen Eigenräume.
    
    \item Bestimmen Sie eine zu $A$ ähnliche Diagonalmatrix $D$ sowie eine zugehörige Basiswechselmatrix $T$, deren Spalten bezüglich der 2-Norm auf die Länge 1 normiert sein sollen. Es soll dann also gelten: $D = T^{-1}AT$.
    
    \item Bestimmen Sie mit Hilfe der von Mises Iteration numerisch den betragsmässig grössten Eigenwert von $A$, sowie einen zugehörigen Eigenvektor. Iterieren Sie dabei 40 Mal, ausgehend vom Startvektor $v = (1,0)^T$. Stellen Sie den absoluten Fehler der numerischen Näherung des Eigenwertes in Abhängigkeit der Iterationszahl halblogarithmisch dar.
\end{enumerate}
\end{example2}

\begin{example2}{Rechnerarithmetik und Gleitpunktdarstellung}
Gegeben seien zwei verschiedene Rechenmaschinen. Die erste davon arbeite mit einer 46-stelligen Binärarithmetik und die zweite einer 14-stelligen Dezimalarithmetik.

\paragraph{Aufgaben:}
\begin{enumerate}
    \item Welche Maschine rechnet genauer? (Mit Begründung!)
    \item Stellen Sie die Zahl $x = \sqrt{3}$ korrekt gerundet als Maschinenzahl $\tilde{x}$ in einer Fliesskomma-Arithmetik mit 5 Binärstellen dar, und geben Sie den relativen Fehler von $\tilde{x}$ im Dezimalformat an.
\end{enumerate}
\end{example2}

\begin{example2}{Konditionierung}
Gegeben ist die Funktion
$$f: (0,\infty) \rightarrow \mathbb{R}; x \mapsto y = f(x) = x^2 \cdot e^{-x}$$

\paragraph{Aufgabe:}
Das Argument $x$ sei mit einem betragsmässigen relativen Fehler von bis zu 5\% behaftet. Bestimmen Sie mit Hilfe der Kondition alle $x$, für welche unter dieser Voraussetzung der Betrag des relativen Fehlers des Funktionswertes $y = f(x)$ ebenfalls höchstens 5\% wird.
\end{example2}

\begin{example2}{LGS und Kondition}
Gegeben ist ein lineares Gleichungssystem $Ax = b$ mit
$$A = \begin{pmatrix}
10^{-5} & 10^{-5}\\
2 & 3
\end{pmatrix}, \quad
b = \begin{pmatrix}
0\\
1
\end{pmatrix}$$
und der exakten Lösung $x_e = \begin{pmatrix}-1\\1\end{pmatrix}$

\paragraph{Aufgaben:}
\begin{enumerate}
    \item Bestimmen Sie die Kondition cond$(A)$ der Matrix $A$ in der 1-Norm.
    
    \item Gegeben ist nun die fehlerbehaftete rechte Seite $\tilde{b} = \begin{pmatrix}10^{-5}\\1\end{pmatrix}$. Berechnen Sie die entsprechende Lösung $\tilde{x}$.
    
    \item Bestimmen Sie für die Lösung aus b) den relativen Fehler $\frac{\|\tilde{x}-x\|_1}{\|x\|_1}$, und vergleichen Sie diesen mit der Abschätzung aufgrund der Kondition. Was stellen Sie fest?
\end{enumerate}
\end{example2}

\begin{example2}{Fixpunktiteration}
Die Gleichung $2^x = 2x$ hat eine Lösung im Intervall $I = [0.5, 1.5]$ für die zugehörige Fixpunktiteration
$$x_{k+1} = \frac{1}{2} \cdot 2^{x_k}, \quad x_0 = 1.5$$

\paragraph{Aufgaben:}
\begin{enumerate}
    \item Überprüfen Sie mit Hilfe des Fixpunktsatzes von Banach und mit obigem Intervall, dass die angegebene Fixpunktiteration tatsächlich konvergiert.
    
    \item Bestimmen Sie mit Hilfe der a priori Fehlerabschätzung, wieviele Schritte es höchstens braucht, um einen absoluten Fehler von maximal $10^{-8}$ garantieren zu können.
\end{enumerate}
\end{example2}

\begin{example2}{Iterative LGS-Lösung}
Gegeben ist das lineare Gleichungssystem $Ax = b$ mit
$$A = \begin{pmatrix}
30 & 10 & 5\\
10 & a & 20\\
5 & 20 & 50
\end{pmatrix}
\quad \text{und} \quad
b = \begin{pmatrix}
5a\\
a\\
5a
\end{pmatrix}$$
Dabei ist $a \in \mathbb{N}$ ein ganzzahliger Parameter.

\paragraph{Aufgaben:}
\begin{enumerate}
    \item Welche Bedingung muss $a$ erfüllen, damit $A$ diagonal dominant ist und also das Jacobi-Verfahren konvergiert?
    
    \item Berechnen Sie den ersten Iterationsschritt des Jacobi-Verfahrens für den Startvektor $x^{(0)} = (a,0,a)^T$.
    
    \item Bestimmen Sie für $a \geq 60$ mittels der a priori Abschätzung und bezüglich der $\infty$-Norm die Anzahl Iterationsschritte $n = n(a)$ als Funktion von $a$, um eine vorgegebene Fehlerschranke $\epsilon$ zu erreichen.
\end{enumerate}
\end{example2}

\begin{example2}{Bakterienpopulation}
Die folgende Abbildung zeigt den gemessenen Verlauf einer Bakterienpopulation $g(t)$ (Einheit: Mio. Bakterien) als Funktion der Zeit $t$ (Einheit: Stunden).

Es wird vermutet, dass sich $g(t)$ darstellen lässt als Funktion mit den drei (vorerst unbekannten) Parametern $a, b, c \in \mathbb{R}$ gemäss
$$g(t) = a + b \cdot e^{c\cdot t}$$

\paragraph{Aufgaben:}
\begin{enumerate}
    \item Bestimmen Sie eine Näherung für die drei Parameter $a, b, c \in \mathbb{R}$, indem Sie 3 Messpunkte $(t_i, g(t_i))$ (für $i = 1, 2, 3$) aus der Abbildung herauslesen, das zugehörige Gleichungssystem aufstellen und für das Newton-Verfahren für Gleichungssysteme die erste Iteration angeben (inkl. Jacobi-Matrix und $\delta^{(0)}$). Verwenden Sie als Startvektor $(1, 2, 3)^T$.

    \item Bestimmen Sie mit Ihrer Näherung aus a) den Zeitpunkt $t$, an dem die Population auf 1600 [Mio. Bakterien] angewachsen ist. Verwenden Sie dafür das Newton-Verfahren mit einem sinnvollen Startwert $t_0$ und einer Genauigkeit von $|t_n - t_{n-1}| < 10^{-4}$. Geben Sie die verwendete Iterationsgleichung explizit an.

    Falls Sie Aufgabe a) nicht lösen konnten, so verwenden Sie $g(t) = 5 + 3 \cdot e^{4t}$.
\end{enumerate}
\end{example2}

\begin{example2}{Chilli-Festival}
Für ein Chilli-Festival werden BBQ-Saucen in den drei verschiedenen Schärfegraden sehr scharf (SS), scharf (S) und mild (M) benötigt, zudem sollen drei verschiedene Hersteller dem Publikum zur Blindverkostung vorgelegt werden können. Die entsprechenden Marketing Pakete der drei Hersteller reichen für die folgende Anzahl Portionen:

\begin{center}
\begin{tabular}{l|ccc}
Hersteller & SS & S & M \\\hline
Blair's Extreme & 240 & 60 & 60\\
Mad Dog & 120 & 180 & 90\\
Dave's Gourmet & 80 & 170 & 500
\end{tabular}
\end{center}

Aus früheren Jahren weiss man, dass das Publikum die verschiedenen Schärfegrade ungefähr im Verhältnis 3/4/5 konsumiert und ca. 12'000 Portionen erwartet werden. Der Einfachheit halber rechnet das Komitee deshalb mit folgender Portionierung: SS 3080, S 4070, M 5030.

\paragraph{Aufgaben:}
\begin{enumerate}
    \item Stellen Sie das entsprechende Gleichungssystem auf.
    \item Wie viele Marketing Pakete von den einzelnen Herstellern werden benötigt um den voraussichtlichen Bedarf abzudecken, ohne Überschuss zu generieren? Berechnen Sie die Lösung mit dem Gauss-Algorithmus.
    \item Wie hoch ist der absolute und relative Fehler, wenn die Besucherzahlen die Erwartung um 5\% übersteigt?
    \item Was können Sie über die Konditionierung des Problems sagen?
\end{enumerate}
\end{example2}

\begin{example2}{Nichtlineares Gleichungssystem}
Das nichtlineare Gleichungssystem
$$1 - x^2 = y^2$$
$$\frac{(x-2)^2}{a} + \frac{(y-1)^2}{b} = 1$$
soll mit dem Newton-Verfahren mit $x^{(0)} = (x^{(0)}, y^{(0)})^T = (2,-1)^T$ gelöst werden.

\paragraph{Aufgaben:}
\begin{enumerate}
    \item Führen Sie den ersten Iterationsschritt manuell aus und bestimmen Sie $x^{(1)}$.
    
    \item Bestimmen Sie die Näherungslösung mit MATLAB für $a = 2$ und $b = 4$ auf $\|f(x^{(k)})\|_\infty < 10^{-8}$ genau.
    
    \item Wie viele Lösungen besitzt das Gleichungssystem aus b) insgesamt? Begründen Sie Ihre Antwort, z.B. mit einem geeigneten Plot.
\end{enumerate}
\end{example2}

\begin{example2}{Elektrischer Widerstand}
Sie messen den Gesamtwiderstand von 3 verschiedenen, in Serie geschalteten Widerstände. Daraus ergibt sich folgendes Gleichungssystem zur Bestimmung der jeweiligen Widerstände $R_i$:
$$67 = 1R_1 + 3R_2 + 7R_3$$
$$21 = 1R_3 + 15R_1$$
$$44 = 1R_2 + 6R_3$$

\paragraph{Aufgaben:}
\begin{enumerate}
    \item Konvertieren Sie die Gleichung in die Form: $y = Ax$ und zerlegen Sie die Matrix $A$ zur Lösung mit dem Gauss-Seidel-Verfahren in die Form: $A = L + R + D$. Schreiben Sie alle Matrizen ($A$, $L$, $R$, $D$) auf!
    
    \item Implementieren Sie das Gauss-Seidel-Verfahren in MATLAB und berechnen Sie die ersten 5 Iterationsschritte mit dem Startvektor $\vec{x}_0 = \vec{0}$. Geben Sie die Iterationsschritte entweder im MATLAB aus oder schreiben Sie diese auf das Blatt.
    
    \item Was ist der Vorteil eines iterativen Lösungsverfahrens wie Gauss-Seidel gegenüber direkten Lösungsverfahren?
\end{enumerate}
\end{example2}

\begin{example2}{Gleitpunktarithmetik}
\paragraph{Aufgaben:}
\begin{enumerate}
    \item Bestimmen Sie die Anzahl verschiedener Maschinenzahlen auf einem Rechner, der 15-stellige Gleitpunktzahlen mit 5-stelligen Exponenten sowie dazugehörige Vorzeichen im Dualsystem verwendet.
    
    \item Geben Sie die Maschinengenauigkeit einer Rechenmaschine an, die mit 16-stelliger Dezimalarithmetik arbeitet.
\end{enumerate}
\end{example2}

\begin{example2}{Nichtlineare Gleichung mit zwei Methoden}
Für die Gleichung
$$x^2\cos(x) = 1$$
sind mit $x_0 = 1$ der erste Schritt des Newton-Verfahrens sowie des Sekantenverfahrens durchzuführen.

\paragraph{Aufgabe:}
\begin{enumerate}
    \item Berechnen Sie den ersten Iterationsschritt für beide Verfahren. Für das Sekantenverfahren ist zusätzlich $x_{-1} = 0$ gegeben.
    
    \item Vergleichen Sie die beiden Ergebnisse und diskutieren Sie mögliche Vor- und Nachteile der beiden Verfahren.
\end{enumerate}
\end{example2}

\begin{example2}{Matrix-Eigenwerte}
Für die Matrix
$$A = \begin{pmatrix}
3 & -2 & 0\\
-2 & 3 & -2\\
0 & -2 & 3
\end{pmatrix}$$

\paragraph{Aufgaben:}
\begin{enumerate}
    \item Bestimmen Sie das charakteristische Polynom.
    
    \item Zeigen Sie, dass $\lambda = 1$ ein Eigenwert ist und bestimmen Sie einen zugehörigen Eigenvektor.
    
    \item Zeigen Sie, dass die Matrix $A$ drei reelle Eigenwerte besitzt. Geben Sie diese an.
    
    \item Bestimmen Sie für jeden Eigenwert die algebraische und geometrische Vielfachheit.
\end{enumerate}
\end{example2}

\begin{example2}{QR-Zerlegung und Eigenwerte}
Gegeben ist die Matrix
$$A = \begin{pmatrix}
1 & 2 & 2\\
2 & 1 & 2\\
2 & 2 & 1
\end{pmatrix}$$

\paragraph{Aufgaben:}
\begin{enumerate}
    \item Führen Sie drei Schritte des QR-Verfahrens durch.
    
    \item Was beobachten Sie bezüglich der Konvergenz? 
    
    \item Vergleichen Sie die Diagonalelemente der resultierenden Matrix mit den tatsächlichen Eigenwerten von $A$.
\end{enumerate}
\end{example2}

