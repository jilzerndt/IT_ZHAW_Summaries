\begin{KR}{Nullstellenverfahren - Praktisches Vorgehen}
\begin{enumerate}
    \item Voraussetzungen prüfen:
    \begin{itemize}
        \item Existiert Nullstelle? (z.B. Vorzeichenwechsel)
        \item Sind Startwerte geeignet?
        \item Ist Funktion ausreichend glatt? (für Newton)
    \end{itemize}
    
    \item Verfahren wählen:
    \begin{itemize}
        \item Newton: Wenn Ableitung verfügbar und Startwert nahe Lösung
        \item Sekanten: Wenn keine Ableitung aber zwei Startwerte nahe Lösung
        \item Fixpunkt: Wenn Funktion kontraktiv
    \end{itemize}
    
    \item Implementierung:
    \begin{itemize}
        \item Konvergenzkriterien definieren
        \item Maximale Iterationszahl festlegen
        \item Fehlerabschätzung einbauen
        \item Divergenzschutz implementieren
    \end{itemize}
    
    \item Auswertung:
    \begin{itemize}
        \item Konvergenzverhalten prüfen
        \item Fehler abschätzen
        \item Ergebnis validieren
    \end{itemize}
\end{enumerate}
\end{KR}

\begin{KR}{Implementation iterativer Verfahren}
\begin{enumerate}
    \item Wählen Sie Startvektor $x^{(0)}$
    \item Wählen Sie Abbruchkriterien:
        \begin{itemize}
            \item Maximale Iterationszahl $k_{max}$
            \item Toleranz $\epsilon$ für Änderung $\|x^{(k+1)} - x^{(k)}\|$
            \item Toleranz für Residuum $\|Ax^{(k)} - b\|$
        \end{itemize}
    \item Führen Sie Iteration durch bis Kriterien erfüllt
\end{enumerate}
\end{KR}