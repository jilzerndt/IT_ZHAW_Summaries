\section{Python Implementations}

\subsection{Basics}

\begin{definition}{math}
    \texttt{math} is a Python library that provides support for mathematical functions and constants. It is part of the Python Standard Library.
    
    use \texttt{import math} to import the library.
\end{definition}

\begin{formula}{Important math functions}
    \begin{itemize}
        \item \texttt{math.sqrt()} - square root
        \item \texttt{math.exp()} - exponential function
        \item \texttt{math.log()} - natural logarithm
        \item \texttt{math.sin()} - sine function
        \item \texttt{math.cos()} - cosine function
        \item \texttt{math.tan()} - tangent function
        \item \texttt{math.degrees()} - convert radians to degrees
        \item \texttt{math.radians()} - convert degrees to radians
        \item \texttt{math.pi} - mathematical constant $\pi$
        \item \texttt{math.e} - mathematical constant $e$
    \end{itemize}
\end{formula}

\begin{definition}{numpy}
    \texttt{numpy} is a Python library that provides support for large, multi-dimensional arrays and matrices, along with a collection of mathematical functions to operate on these arrays.
    
    use \texttt{import numpy as np} to import the library.

    \textcolor{pink}{\textbf{IMPORTANT}}: import is left out in the following examples - so make sure not to forget it in your code!
\end{definition}

\begin{formula}{Important numpy functions}
    \begin{itemize}
        \item \texttt{np.array()} - create an array
        \item \texttt{np.dot()} - dot product of two arrays
        \item \texttt{np.sum()} - sum of array elements
        \item \texttt{np.min()} - minimum value of array elements
        \item \texttt{np.max()} - maximum value of array elements
        \item \texttt{np.abs()} - absolute value of array elements
    \end{itemize}

    Mathematical Operations (element-wise on np.arrays):
    \begin{itemize}
        \item \texttt{np.sin()} - sine 
        \item \texttt{np.cos()} - cosine 
        \\ (etc. for other trigonometric functions)
        \item \texttt{np.degrees()} - convert radians to degrees
        \item \texttt{np.radians()} - convert degrees to radians
        \item \texttt{np.sqrt()} - square root 
        \item \texttt{np.exp()} - exponential of array elements
        \item \texttt{np.log()} - natural logarithm of array elements
        \item \texttt{np.add(), np.subtract(), np.multiply(), np.divide()} - element-wise operations
        \item \texttt{np.round(), np.floor(), np.ceil()} - rounding functions
        \item \texttt{np.pi, np.e} - mathematical constants
    \end{itemize}
\end{formula}

\begin{concept}{Matrix Operations with numpy}
    \begin{itemize}
        \item Tiefe Kopie einer Matrix: \texttt{B = np.copy(A)}
        \item Matrix addition: \texttt{A + B}
        \item Matrix subtraction: \texttt{A - B}
        \item Matrix multiplication: \texttt{mp.matmul(A, B)}
        \item Matrix-Vector multiplication: \texttt{np.matvec(A, x)}
        \item Vector-Matrix multiplication: \texttt{np.vecmat(x, A)}
        \item Scalar multiplication: \texttt{A * B}
        \item Element-wise division: \texttt{A / B}
        \item Matrix transpose: \texttt{np.transpose(A)}
        \item Matrix inverse: \texttt{np.linalg.inv(A)}
        \item Euklidische Vektornorm: \texttt{np.linalg.norm(x)}
        \item Vektor auf Länge 1 normieren: \texttt{np.linalg.norm(x, ord=1)}
        \item Determinant of a matrix: \texttt{np.linalg.det(A)}
        \item Eigenvalues and eigenvectors: \texttt{np.linalg.eig(A)}
    \end{itemize}
\end{concept}

\begin{theorem}{Analysis and Linear Algebra with numpy}
    \begin{itemize}
        \item Evaluate a function at a point: \texttt{f(x)}
        \item Evaluate a function on an array: \texttt{f(x\_array)}
        \item Derivative of a function: \texttt{np.gradient(f, h)}
        \item Integral of a function: \texttt{np.trapz(f, x)}
        \item Solve a system of linear equations: \texttt{np.linalg.solve(A, b)}
        \item Find the roots of a polynomial: \texttt{np.roots(p)}
        \item Interpolate data points: \texttt{np.interp(x, x\_data, y\_data)}
    \end{itemize}
\end{theorem}

\begin{example2}{Matrix Operations with numpy}
\begin{lstlisting}[language=Python, style=basesmol]
import numpy as np
# Create matrices
A_2 = np.array([[1, 2], [3, 4]]) # 2x2 matrix
A_3 = np.array([[1, 2, 3], [4, 5, 6], [7, 8, 9]]) # 3x3 matrix

# Matrix addition
C = A_2 + A_2
print(C) # [[2 4] [6 8]]
\end{lstlisting}
\end{example2}

\begin{definition}{matplotlib}
    \texttt{matplotlib} is a Python library that provides support for creating static, animated, and interactive visualizations in Python.
    
    use \texttt{import matplotlib.pyplot as plt} to import the library.
\end{definition}

\begin{formula}{Important matplotlib functions}
    \begin{itemize}
        \item \texttt{plt.plot()} - plot lines and/or markers
        \item \texttt{plt.scatter()} - create a scatter plot
        \item \texttt{plt.bar()} - create a bar plot
        \item \texttt{plt.hist()} - create a histogram
        \item \texttt{plt.pie()} - create a pie chart
        \item \texttt{plt.xlabel(), plt.ylabel()} - set the labels of the x and y axes
        \item \texttt{plt.title()} - set the title of the plot
        \item \texttt{plt.legend()} - add a legend to the plot
        \item \texttt{plt.show()} - display the plot
    \end{itemize}
\end{formula}

\begin{examplecode}{Funktion, Ableitung, Stammfunktion und Plot}
\begin{lstlisting}[language=Python, style=basesmol]
import numpy as np
import matplotlib.pyplot as plt

x = np.linspace(-10, 10, 400)
y = (x**5) - (5*x**4) - (30*x**3) + (110*x**2) + 29*x - 105
dy = (5*x**4) - (20*x**3) - (90*x**2) + (220*x) + 29
F = (1/6)*x**6 - (1/5)*x**5 - (15/4)*x**4 + (110/3)*x**3 + (29/2)*x**2 - 105*x

plt.figure(figsize=(10, 6))
plt.plot(x, y, label='$f(x) = x^5 - 5x^4 - 30x^3 + 110x^2 + 29x - 105$')
plt.plot(x, dy, label="Ableitung $f'(x)$", linestyle='--')
plt.plot(x, F, label="Stammfunktion $F(x)$", linestyle=':')

plt.title("Funktion, ihre Ableitung und Stammfunktion")
plt.xlabel("x-Achse")
plt.ylabel("y-Achse")
plt.xlim(-10, 10)
plt.ylim(-1000, 1000)
plt.grid(True)
plt.legend()
plt.show()
\end{lstlisting}    
\end{examplecode}

\begin{definition}{Decimal} is a Python library that provides support for arbitrary-precision arithmetic. It allows for precise calculations with a specified number of decimal places.
    
    use \texttt{from decimal import Decimal, getcontext} to import the library.
\end{definition}

\begin{example2}{Bibliotheksverwendung} Beispiel: Präzise Berechnung mit Decimal
\begin{lstlisting}[language=Python, style=basesmol]
from decimal import Decimal, getcontext

# Set precision
getcontext().prec = 40

# Precise calculation
x = Decimal('1.0') / Decimal('7.0')
print(x)  # 0.1428571428571428571428571428571428571428
\end{lstlisting}
\end{example2}

\begin{definition}{scipy}
    \texttt{scipy} is a Python library that provides support for scientific and technical computing. It builds on the capabilities of \texttt{numpy} and provides additional modules for optimization, linear algebra, integration, interpolation, and more.
    
    use \texttt{import scipy as sp} to import the library.
\end{definition}

\begin{formula}{Important scipy functions}
\begin{itemize}
    \item \texttt{sp.integrate.quad()} - numerical integration
    \item \texttt{sp.optimize.minimize()} - function minimization
    \item \texttt{sp.linalg.solve()} - solve linear system of equations
    \item \texttt{sp.fft.fft()} - fast Fourier transform
    \item \texttt{sp.interpolate.interp1d()} - 1-dimensional interpolation
    \item \texttt{sp.stats.norm()} - normal distribution functions
\end{itemize}
\end{formula}

\begin{definition}{sympy}
    \texttt{sympy} is a Python library for symbolic mathematics. It provides support for symbolic computation, algebraic manipulation, calculus, equation solving, and more.
    
    use \texttt{import sympy as sp} to import the library.    
\end{definition}

\begin{formula}{Important sympy functions}
\begin{itemize}
    \item \texttt{sp.symbols()} - define symbolic variables
    \item \texttt{sp.diff()} - differentiate expressions
    \item \texttt{sp.integrate()} - integrate expressions
    \item \texttt{sp.solve()} - solve equations
    \item \texttt{sp.limit()} - calculate limits
    \item \texttt{sp.series()} - compute series expansions
    \item \texttt{sp.Matrix()} - create matrices
    \item \texttt{sp.Eq()} - create equations
    \item \texttt{expression.evalf()} - evaluate numerical expressions accurately
\end{itemize}
\end{formula}

\subsubsection{Hilfsfunktionen}

\begin{examplecode}{is\_diagonally\_dominant} Diagonaldominanz prüfen
\begin{lstlisting}[language=Python, style=basesmol]
def is_diagonally_dominant(A):
    # Get the diagonal elements of A
    diag = np.diag(np.abs(A))
    # Sum of the absolute values of the off-diagonal elements
    off_diag = np.sum(np.abs(A), axis=1) - diag
    # Check if the diagonal elements are greater than the sum of the absolute values of the off-diagonal elements
    return np.all(diag >= off_diag)
\end{lstlisting}
\end{examplecode}


\begin{examplecode}{convergence\_check} Konvergenzkriterien prüfen
\begin{lstlisting}[language=Python, style=basesmol]
def convergence_check(x_new, x_old, f_new, f_old, tolerance):
    # Absoluter Fehler im Funktionswert
    if np.abs(f_new) < tolerance:
        return True, "Function value < tolerance"
    # Relative Aenderung der x-Werte
    if np.abs(x_new - x_old) < tolerance * (1 + np.abs(x_new)):
        return True, "Relative change < tolerance"
    # Relative Aenderung der Funktionswerte
    if np.abs(f_new - f_old) < tolerance * (1 + np.abs(f_new)):
        return True, "Function change < tolerance"
    # Divergenzcheck
    if np.abs(f_new) > 2 * np.abs(f_old):
        return False, "Divergence detected"

    return False, "Not yet converged"
\end{lstlisting}
\end{examplecode}

