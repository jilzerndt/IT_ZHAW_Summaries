
\section{Python Implementationen}

\subsubsection{Hilfsfunktionen}

\subsection{Numerische Lösung von Nullstellenproblemen}

\begin{examplecode}{Fehlerabschätzung durch Vorzeichenwechsel}
\begin{lstlisting}[language=Python, style=basesmol]
def error_estimate(f, x, eps=1e-5):
    fx_left = f(x - eps)
    fx_right = f(x + eps)
    
    if fx_left * fx_right < 0:
        return eps  # Nullstelle liegt in (x-eps, x+eps)
    return None

    #Returns: Fehlerschranke eps wenn Vorzeichenwechsel, sonst None
\end{lstlisting}
\end{examplecode}

\begin{examplecode}{Verschiedene Abbruchkriterien prüfen} Konvergenzkriterien
\begin{lstlisting}[language=Python, style=basesmol]
def convergence_criteria(x_new, x_old, f_new, f_old, tol=1e-6):
    # Absoluter Fehler im Funktionswert
    if abs(f_new) < tol:
        return True, "Funktionswert < tol"
        
    # Relative Aenderung der x-Werte
    if abs(x_new - x_old) < tol * (1 + abs(x_new)):
        return True, "Relative Aenderung < tol"
        
    # Relative Aenderung der Funktionswerte
    if abs(f_new - f_old) < tol * (1 + abs(f_new)):
        return True, "Funktionsaenderung < tol"
        
    # Divergenzcheck
    if abs(f_new) > 2 * abs(f_old):
        return False, "Divergenz detektiert"
        
    return False, "Noch nicht konvergiert"

    #Returns: (konvergiert?, grund)
\end{lstlisting}
\end{examplecode}

\begin{examplecode}{Fixpunktiteration}
\begin{lstlisting}[language=Python, style=basesmol]
def fixed_point_it(f, x0, tol=1e-6, max_it=100):
    x = x0
    for i in range(max_it):
        x_new = f(x)
        if abs(x_new - x) < tol:
            return x_new, i+1
        x = x_new
    raise ValueError("Keine Konvergenz")

# Optimierte Version mit Fehlerschaetzung
def fixed_point_it_opt(f, x0, tol=1e-6, max_it=100):
    x = x0
    alpha = None  # Schaetzung fuer Lipschitz-Konstante
    for i in range(max_iter):
        x_new = f(x)
        dx = abs(x_new - x)
        # Lipschitz-Konstante schaetzen
        if i > 0 and dx > 0:
            alpha_new = dx / dx_old
            if alpha is None or alpha_new > alpha:
                alpha = alpha_new
        # A-posteriori Fehlerabschaetzung
        if alpha is not None and alpha < 1:
            error = alpha * dx / (1 - alpha)
            if error < tol:
                return x_new, i+1
        x = x_new
        dx_old = dx
    raise ValueError("Keine Konvergenz")
\end{lstlisting}
\end{examplecode}

\begin{examplecode}{Newton-Verfahren}
\begin{lstlisting}[language=Python, style=basesmol]
def newton(f, df, x0, tol=1e-6, max_iter=100):
    x = x0
    for i in range(max_iter):
        fx = f(x)
        if abs(fx) < tol:
            return x, i+1
        dfx = df(x)
        if dfx == 0:
            raise ValueError("Ableitung Null")
        x = x - fx/dfx
    raise ValueError("Keine Konvergenz")

# Optimierte Version mit Fehlerkontrolle
def newton_safe(f, df, x0, tol=1e-6, max_it=100):
    x = x0
    fx = f(x)
    for i in range(max_it):
        dfx = df(x)
        if dfx == 0:
            raise ValueError("Ableitung Null")
        dx = fx/dfx
        x_new = x - dx
        fx_new = f(x_new)
        # Verschiedene Konvergenzkriterien
        if abs(fx_new) < tol:  # Funktionswert
            return x_new, i+1
        if abs(dx) < tol * (1 + abs(x)):  # Relative Aenderung
            return x_new, i+1
        if abs(fx_new) >= abs(fx):  # Divergenzcheck
            raise ValueError("Divergenz detektiert")
        x, fx = x_new, fx_new
    raise ValueError("Keine Konvergenz")
\end{lstlisting}
\end{examplecode}



\begin{examplecode}{Sekantenverfahren}
\begin{lstlisting}[language=Python, style=basesmol]
# Einfache Version
def secant(f, x0, x1, tol=1e-6, max_iter=100):
    fx0 = f(x0)
    fx1 = f(x1)
    
    for i in range(max_iter):
        if abs(fx1) < tol:
            return x1, i+1
            
        if fx1 == fx0:
            raise ValueError("Division durch Null")
            
        x2 = x1 - fx1 * (x1 - x0)/(fx1 - fx0)
        x0, x1 = x1, x2
        fx0, fx1 = fx1, f(x2)
        
    raise ValueError("Keine Konvergenz")

# Optimierte Version mit Fehlerkontrolle
def secant_safe(f, x0, x1, tol=1e-6, max_iter=100):
    fx0 = f(x0)
    fx1 = f(x1)
    
    if abs(fx0) < abs(fx1):  # Stelle mit kleinerem f-Wert als x1
        x0, x1 = x1, x0
        fx0, fx1 = fx1, fx0
    
    for i in range(max_iter):
        if abs(fx1) < tol:
            return x1, i+1
            
        if fx1 == fx0:
            raise ValueError("Division durch Null")
            
        # Sekanten-Schritt
        d = fx1 * (x1 - x0)/(fx1 - fx0)
        x2 = x1 - d
        
        # Konvergenzpruefungen
        if abs(d) < tol * (1 + abs(x1)):  # Relative Aenderung
            return x2, i+1
            
        fx2 = f(x2)
        if abs(fx2) >= abs(fx1):  # Divergenzcheck
            if i == 0:
                raise ValueError("Schlechte Startwerte")
            return x1, i+1
            
        x0, x1 = x1, x2
        fx0, fx1 = fx1, fx2
        
    raise ValueError("Keine Konvergenz")
\end{lstlisting}
\end{examplecode}




\begin{examplecode}{Nullstellensuche mit Fehlerabschätzung} 
\begin{lstlisting}[language=Python, style=basesmol]
def root_finder_with_error(f, x0, tol=1e-6, max_iter=100):
    x_old = x0
    f_old = f(x_old)
    
    for i in range(max_iter):
        # Iterationsschritt (hier Newton als Beispiel)
        x_new = x_old - f_old/derivative(f, x_old)
        f_new = f(x_new)
        
        # Pruefe Konvergenzkriterien
        converged, reason = convergence_criteria(
            x_new, x_old, f_new, f_old, tol)
            
        if converged:
            # Schaetze finalen Fehler
            error = error_estimate(f, x_new, tol)
            return {
                'root': x_new,
                'iterations': i+1,
                'error_bound': error,
                'convergence_reason': reason
            }
            
        x_old, f_old = x_new, f_new
        
    raise ValueError(f"Keine Konvergenz nach {max_iter} Iterationen")

    # Returns: Dictionary mit Ergebnissen

# Beispielnutzung
def example_function(x):
    return x**2 - 2

result = root_finder_with_error(example_function, 1.0)
print(f"Nullstelle: {result['root']:.10f}")
print(f"Iterationen: {result['iterations']}")
print(f"Fehlerschranke: {result['error_bound']:.10f}")
print(f"Konvergenzgrund: {result['convergence_reason']}")

# Ausgabe etwa:
# Nullstelle: 1.4142135624
# Iterationen: 5
# Fehlerschranke: 1e-06
# Konvergenzgrund: Funktionswert < tol

\end{lstlisting}
\end{examplecode}

\subsection{Numerische Lösung von Gleichungssystemen}

\begin{examplecode}{Gauss-Elimination mit Pivotisierung}
\begin{lstlisting}[language=Python, style=basesmol]
def gauss_elimination(A, b):
    n = len(A)
    # Erweiterte Matrix erstellen
    M = [[A[i][j] for j in range(n)] + [b[i]] for i in range(n)]

    # Vorwaertselimination
    for i in range(n):
        pivot = M[i][i]
        if abs(pivot) < 1e-10:
            continue
        for j in range(i+1, n):
            factor = M[j][i] / pivot
            for k in range(i, n+1):
                M[j][k] -= factor * M[i][k]

    # Rueckwaertssubstitution
    x = [0] * n
    for i in range(n-1, -1, -1):
        if abs(M[i][i]) < 1e-10:
            x[i] = 1  # Freie Variable
            continue
        sum_val = sum(M[i][j] * x[j] for j in range(i+1, n))
        x[i] = (M[i][n] - sum_val) / M[i][i]
    
    return x
\end{lstlisting}    
\end{examplecode}





\begin{examplecode}{LR-Zerlegung Implementation}
\begin{lstlisting}[language=Python, style=basesmol]
def lr_decomposition(A):
    n = len(A)
    # Kopiere A um Original nicht zu veraendern
    R = [[A[i][j] for j in range(n)] for i in range(n)]
    L = [[1.0 if i == j else 0.0 for j in range(n)] for i in range(n)]
    P = [[1.0 if i == j else 0.0 for j in range(n)] for i in range(n)]
    
    for k in range(n-1):
        # Pivotisierung
        pivot = k
        for i in range(k+1, n):
            if abs(R[i][k]) > abs(R[pivot][k]):
                pivot = i
        if abs(R[pivot][k]) < 1e-10:  # Numerische Null
            raise ValueError("Matrix ist (fast) singulaer")
        # Zeilenvertauschung falls noetig
        if pivot != k:
            R[k], R[pivot] = R[pivot], R[k]
            # L und P anpassen fuer Zeilen < k
            for j in range(k):
                L[k][j], L[pivot][j] = L[pivot][j], L[k][j]
            P[k], P[pivot] = P[pivot], P[k]

        # Elimination
        for i in range(k+1, n):
            factor = R[i][k] / R[k][k]
            L[i][k] = factor
            for j in range(k, n):
                R[i][j] -= factor * R[k][j]
               
    return P, L, R
\end{lstlisting}
\end{examplecode}

\begin{examplecode}{QR-Zerlegung Implementation}
\begin{lstlisting}[language=Python, style=basesmol]
def qr_decomposition(A):
    m = len(A)
    n = len(A[0])
    # Kopiere A nach R (deep copy)
    R = [[A[i][j] for j in range(n)] for i in range(m)]
    # Initialisiere Q als Einheitsmatrix
    Q = [[1.0 if i == j else 0.0 for j in range(m)] 
         for i in range(m)]
    
    def vector_norm(v): # Norm eines Vektors
        return (sum(x*x for x in v)) ** 0.5
    
    def matrix_mult(A, B): # Matrixmultiplikation
        m, n = len(A), len(B[0])
        p = len(B)
        C = [[0.0] * n for _ in range(m)]

        for i in range(m):
            for j in range(n):
                C[i][j] = sum(A[i][k] * B[k][j] 
                             for k in range(p))

        return C
    
    def householder_reflection(x):
        n = len(x)
        v = [xi for xi in x]  # Kopiere x

        # Berechne Norm des Teilvektors
        sigma = sum(v[i]*v[i] for i in range(1, n))
        if sigma == 0 and x[0] >= 0:
            beta = 0
        elif sigma == 0 and x[0] < 0:
            beta = -2
        else:
            mu = (x[0]*x[0] + sigma)**0.5
            if x[0] <= 0:
                v[0] = x[0] - mu
            else:
                v[0] = -sigma/(x[0] + mu)

            beta = 2*v[0]*v[0]/(sigma + v[0]*v[0])

            # Normiere v
            temp = v[0]
            for i in range(n):
                v[i] /= temp

        return v, beta
\end{lstlisting}
\end{examplecode}

\begin{KR}{QR-Zerlegung - Praktisches Vorgehen}
\begin{enumerate}
    \item Vorbereitungen
    \begin{itemize}
        \item Matrix A kopieren für R
        \item Q als Einheitsmatrix initialisieren
        \item Householder-Vektoren speichern
    \end{itemize}

    \item Für jede Spalte k = 1,...,n-1:
    \begin{itemize}
        \item Untervektor $v_k$ aus k-ter Spalte extrahieren
        \item Householder-Vektor berechnen:
            \begin{itemize}
                \item $w_k = v_k + \text{sign}(v_{k1})\|v_k\|e_1$
                \item $u_k = \frac{w_k}{\|w_k\|}$
            \end{itemize}
        \item Householder-Matrix auf Untermatrix anwenden:
            \begin{itemize}
                \item $H_k = I - 2u_ku_k^T$
                \item $R_{k:n,k:n} = H_k \cdot R_{k:n,k:n}$
            \end{itemize}
        \item Q aktualisieren: $Q = Q \cdot H_k^T$
    \end{itemize}

    \item System lösen durch
    \begin{itemize}
        \item $y = Q^Tb$ berechnen
        \item Rückwärtseinsetzen: $Rx = y$
    \end{itemize}
\end{enumerate}
\end{KR}

\begin{examplecode}{QR-Zerlegung Implementation (Fortsetzung)}
\begin{lstlisting}[language=Python, style=basesmol]
    # Hauptschleife der QR-Zerlegung
    for k in range(n):
        # Extrahiere k-te Spalte ab k-ter Zeile
        x = [R[i][k] for i in range(k, m)]
        if len(x) > 1:  # wenn noch Untermatrix existiert
            # Berechne Householder-Transformation
            v, beta = householder_reflection(x)
            # Wende Householder auf R an
            for j in range(k, n):
                # Berechne w = beta * (v^T * R_j)
                w = beta * sum(v[i-k]*R[i][j] 
                             for i in range(k, m))
                # Update R
                for i in range(k, m):
                    R[i][j] -= v[i-k] * w
            # Update Q
            for j in range(m):
                w = beta * sum(v[i-k]*Q[j][i+k] 
                             for i in range(len(v)))
                for i in range(len(v)):
                    Q[j][k+i] -= v[i] * w
    # Transponiere Q am Ende
    Q = [[Q[j][i] for j in range(m)] for i in range(m)]
    
    return Q, R # Q (orthogonal) und R (obere Dreiecksmatrix)

# Beispiel fuer Verwendung
def solve_qr(A, b): # Loest Ax = b mittels QR-Zerlegung
    Q, R = qr_decomposition(A)
    # Berechne Q^T * b
    y = [sum(Q[i][j] * b[j] 
             for j in range(len(b))) 
         for i in range(len(b))]
    # Rueckwaertseinsetzen
    n = len(R)
    x = [0] * n
    for i in range(n-1, -1, -1):
        s = sum(R[i][j] * x[j] for j in range(i+1, n))
        if abs(R[i][i]) < 1e-10:
            raise ValueError("Matrix (fast) singulaer")
        x[i] = (y[i] - s) / R[i][i]
    return x
\end{lstlisting}
\end{examplecode}

\begin{examplecode}{Analyse der Genauigkeit einer Näherungslösung}
\begin{lstlisting}[language=Python, style=basesmol]
def error_analysis(A, x, b, x_approx):
    n = len(A)
    # Residuum berechnen
    r = [b[i] - sum(A[i][j] * x_approx[j] 
                   for j in range(n)) 
         for i in range(n)]
    res_norm = max(abs(ri) for ri in r)
    # Relativer Fehler (falls exakte Loesung bekannt)
    if x is not None:
        rel_error = max(abs(x[i] - x_approx[i]) 
                       for i in range(n)) / \
                   max(abs(x[i]) for i in range(n))
    else:
        rel_error = None
    # Absoluter Fehler (falls exakte Loesung bekannt)
    if x is not None:
        abs_error = max(abs(x[i] - x_approx[i]) 
                       for i in range(n))
    else:
        abs_error = None
    
    return {
        'residual_norm': res_norm,
        'relative_error': rel_error,
        'residual': r
    }
\end{lstlisting}
\end{examplecode}

\begin{examplecode}{Jacobi-Verfahren Implementation}
\begin{lstlisting}[language=Python, style=basesmol]
def jacobi_method(A, b, x0, tol=1e-6, max_iter=100):

    n = len(A)
    x = x0.copy()
    x_new = [0.0] * n
    
    for iter in range(max_iter):

        # Jacobi-Iteration
        for i in range(n):
            sum1 = sum(A[i][j] * x[j] for j in range(i))
            sum2 = sum(A[i][j] * x[j] for j in range(i+1, n))
            x_new[i] = (b[i] - sum1 - sum2) / A[i][i]
            
        # Konvergenzpruefung
        diff = max(abs(x_new[i] - x[i]) for i in range(n))

        if diff < tol:
            return x_new, iter + 1

        x = x_new.copy()
        
    raise ValueError(f"Keine Konvergenz nach {max_iter} Iterationen")
\end{lstlisting}
\end{examplecode}



\begin{examplecode}{Gauss-Seidel-Verfahren Implementation}
\begin{lstlisting}[language=Python, style=basesmol]
def gauss_seidel_method(A, b, x0, tol=1e-6, max_iter=100):

    n = len(A)
    x = x0.copy()
    
    for iter in range(max_iter):
        x_old = x.copy()

        # Gauss-Seidel-Iteration
        for i in range(n):
            sum1 = sum(A[i][j] * x[j] for j in range(i))
            sum2 = sum(A[i][j] * x_old[j] for j in range(i+1, n))
            x[i] = (b[i] - sum1 - sum2) / A[i][i]
            
        # Konvergenzpruefung
        diff = max(abs(x[i] - x_old[i]) for i in range(n))

        if diff < tol:
            return x, iter + 1
            
    raise ValueError(f"Keine Konvergenz nach {max_iter} Iterationen")
\end{lstlisting}
\end{examplecode}

\begin{examplecode}{Analyse von LGS auf numerische Stabilität}
\begin{lstlisting}[language=Python, style=basesmol]
def analyze_matrix(A, b):
    n = len(A)
    # 1. Grundlegende Eigenschaften
    diag_dom = is_diagonally_dominant(A)
    scaling = max(abs(A[i][j]) for i in range(n) 
                 for j in range(n))
    # 2. Konditionszahl schaetzen 
    def matrix_norm_inf(M):
        return max(sum(abs(M[i][j]) for j in range(len(M))) 
                  for i in range(len(M)))

    def inverse_power_iteration(M, max_iter=100):
        x = [1.0] * n
        for _ in range(max_iter):
            y = solve_triangular(M, x)
            norm = max(abs(yi) for yi in y)
            x = [yi/norm for yi in y]
        return 1.0/norm

    norm_A = matrix_norm_inf(A)
    try:
        norm_Ainv = inverse_power_iteration(A)
        cond = norm_A * norm_Ainv
    except:
        cond = float('inf')
    # 3. Analyse der Diagonalelemente
    min_diag = min(abs(A[i][i]) for i in range(n))
    max_offdiag = max(abs(A[i][j]) for i in range(n) 
                     for j in range(n) if i != j)
    # 4. Empfehlungen generieren
    recommendations = []
    if not diag_dom:
        recommendations.append(
            "Matrix nicht diagonaldominant - "
            "Iterative Verfahren koennten divergieren")
    if cond > 1e4:
        recommendations.append(
            f"Hohe Konditionszahl ({cond:.1e}) - "
            "Ergebnisse koennten ungenau sein")
    if min_diag < max_offdiag/100:
        recommendations.append(
            "Kleine Diagonalelemente - "
            "Pivotisierung empfohlen")
    if scaling > 1e8:
        recommendations.append(
            "Grosse Zahlenunterschiede - "
            "Skalierung empfohlen")
    return {
        "recommandations": recommendations, 
        "results": cond, diag_dom, scaling, min_diag, max_offdiag
    }
\end{lstlisting}
\end{examplecode}

\begin{KR}{Implementation iterativer Verfahren}
\begin{enumerate}
    \item Wählen Sie Startvektor $x^{(0)}$
    \item Wählen Sie Abbruchkriterien:
        \begin{itemize}
            \item Maximale Iterationszahl $k_{max}$
            \item Toleranz $\epsilon$ für Änderung $\|x^{(k+1)} - x^{(k)}\|$
            \item Toleranz für Residuum $\|Ax^{(k)} - b\|$
        \end{itemize}
    \item Führen Sie Iteration durch bis Kriterien erfüllt
\end{enumerate}
\end{KR}

\begin{examplecode}{Hilfsfunktion für Optimiere iterative Verfahren}
\begin{lstlisting}[language=Python, style=basesmol]
def is_diagonally_dominant(A): # Matrix diagonaldominant?
    n = len(A)
    for i in range(n):
        if abs(A[i][i]) <= sum(abs(A[i][j]) for j in range(n) if j != i):
            return False
    return True
\end{lstlisting}
\end{examplecode}

\begin{examplecode}{Optimierte iterative Verfahren Implementation}
    \begin{itemize}
        \item Optimierte Version mit Konvergenzanalyse
        \item Löst $Ax = b$ mit verschiedenen iterativen Verfahren
        \item \texttt{method}: 'jacobi' oder 'gauss-seidel'
        \item \texttt{omega}: Relaxationsparameter (1.0 = Standard)
    \end{itemize}
\begin{lstlisting}[language=Python, style=basesmol]
def iterative_solver(A, b, method='gauss_seidel', tol=1e-6, max_iter=100, omega=1.0):

    n = len(A)
    x = [0.0] * n  # Startvektor
    D = [[A[i][j] if i == j else 0 for j in range(n)] 
         for i in range(n)]  # Diagonalmatrix
    L = [[A[i][j] if i > j else 0 for j in range(n)] 
         for i in range(n)]  # Untere Dreiecksmatrix
    U = [[A[i][j] if i < j else 0 for j in range(n)] 
         for i in range(n)]  # Obere Dreiecksmatrix
    
    # Konvergenzcheck
    if not is_diagonally_dominant(A):
        print("Warnung: Matrix nicht diagonaldominant")
    
    iterations = []
    residuals = []
    
    for iter in range(max_iter):
        x_old = x.copy()
        if method == 'jacobi':
            for i in range(n):
                sum_term = sum(A[i][j] * x_old[j] 
                             for j in range(n) if j != i)
                x[i] = (1-omega) * x_old[i] + \
                       (omega/A[i][i]) * (b[i] - sum_term)
        else:  # gauss_seidel
            for i in range(n):
                sum1 = sum(A[i][j] * x[j] for j in range(i))
                sum2 = sum(A[i][j] * x_old[j] 
                          for j in range(i+1, n))
                x[i] = (1-omega) * x_old[i] + \ (omega/A[i][i]) * (b[i] - sum1 - sum2)

        # Berechne Residuum und relative Aenderung
        res = max(abs(sum(A[i][j] * x[j] for j in range(n)) - b[i]) for i in range(n))
        diff = max(abs(x[i] - x_old[i]) for i in range(n))
        
        iterations.append(x.copy())
        residuals.append(res)
        
        if diff < tol:
            return {
                'solution': x,
                'iterations': iterations,
                'residuals': residuals,
                'iteration_count': iter + 1
            }
            
    raise ValueError(f"Keine Konvergenz nach {max_iter} " + 
                    f"Iterationen\nLetztes Residuum: {res}")
\end{lstlisting}
\end{examplecode}

\subsection{Eigenwerte und Eigenvektoren}



\begin{examplecode}{Komplexe Zahlen in Python}
\begin{lstlisting}[language=Python, style=basesmol]
def complex_operations(z1, z2):
    """Grundlegende Operationen mit komplexen Zahlen."""
    # Basisfunktionen
    def to_polar(z):
        r = (z.real**2 + z.imag**2)**0.5
        phi = math.atan2(z.imag, z.real)
        return r, phi
    
    def from_polar(r, phi):
        return r * (math.cos(phi) + 1j*math.sin(phi))
    
    try:
        # Addition und Subtraktion
        z_add = z1 + z2
        z_sub = z1 - z2
        
        # Multiplikation und Division
        z_mul = z1 * z2
        z_div = z1 / z2 if z2 != 0 else None
        
        # Polarform
        r1, phi1 = to_polar(z1)
        r2, phi2 = to_polar(z2)
        
        # Exponentialform
        z1_exp = from_polar(r1, phi1)
        z2_exp = from_polar(r2, phi2)
        
        return {
            'addition': z_add,
            'subtraktion': z_sub,
            'multiplikation': z_mul,
            'division': z_div,
            'polar_z1': (r1, phi1),
            'polar_z2': (r2, phi2)
        }
        
    except Exception as e:
        print(f"Fehler bei Berechnung: {e}")
        return None

# Beispiel
z1 = 3 - 11j
z2 = 2 + 5j
results = complex_operations(z1, z2)
\end{lstlisting}
\end{examplecode}

\begin{examplecode}{Determinante}
\begin{lstlisting}[language=Python, style=basesmol]
def det_2x2(matrix):
    return matrix[0][0]*matrix[1][1] - matrix[0][1]*matrix[1][0]

def det_3x3(matrix):
    det = 0
    # Entwicklung nach erster Zeile
    for i in range(3):
        minor = []
        for j in range(1,3):
            row = []
            for k in range(3):
                if k != i:
                    row.append(matrix[j][k])
            minor.append(row)
        det += ((-1)**i) * matrix[0][i] * det_2x2(minor)
    return det
\end{lstlisting}
\end{examplecode}

\begin{examplecode}{Charakteristisches Polynom Koeffizienten berechnen}
\begin{lstlisting}[language=Python, style=basesmol]
def char_poly_coeff(matrix):
    n = len(matrix)
    coeffs = [0] * (n+1)
    # Lambda^n Koeffizient
    coeffs[n] = (-1)**n
    # Spur (Lambda^(n-1) Koeffizient)
    trace = sum(matrix[i][i] for i in range(n))
    coeffs[n-1] = (-1)**(n-1) * trace
    # Determinante (konstanter Term)
    coeffs[0] = det_3x3(matrix)
    
    return coeffs

# Beispielmatrix
A = [[2, 1, 0],
     [1, 2, 1],
     [0, 1, 2]]

# Charakteristisches Polynom berechnen
poly = char_poly_coeff(A)
print("Charakteristisches Polynom:")
print(f"p(lambda) = lambda^3 {poly[2]:+}lambda^2 {poly[1]:+}lambda {poly[0]:+}")
\end{lstlisting}
\end{examplecode}

\begin{examplecode}{Eigenvektoren berechnen} \textcolor{pink}{\texttt{use Gauss from previous example}} 
\begin{lstlisting}[language=Python, style=basesmol]
def find_eigenvector(A, eigenvalue):
    n = len(A)
    # A - lambda*I
    A_lambda = [[A[i][j] - (eigenvalue if i==j else 0) 
                for j in range(n)] for i in range(n)]
    
    # Loese (A - lambda*I)x = 0
    b = [0] * n
    return gauss_elimination(A_lambda, b)

# Eigenvektoren fuer lambda = 1,2,3 berechnen
eigenvalues = [1, 2, 3]
for ev in eigenvalues:
    vec = find_eigenvector(A, ev)
    print(f"Eigenvektor fuer lambda={ev}:", vec)
\end{lstlisting}
\end{examplecode}

\begin{examplecode}{Von-Mises-Iteration}
\begin{lstlisting}[language=Python, style=basesmol]
def matrix_vector_mult(A, v):
    n = len(A)
    result = [0] * n
    for i in range(n):
        for j in range(n):
            result[i] += A[i][j] * v[j]
    return result

def vector_norm(v):
    return sum(x*x for x in v) ** 0.5

def normalize_vector(v):
    norm = vector_norm(v)
    return [x/norm for x in v]

def power_iteration(A, tol=1e-10, max_iter=100):
    n = len(A)
    # Startvektor
    v = normalize_vector([1] + [0]*(n-1))
    
    for _ in range(max_iter):
        # Matrix-Vektor-Multiplikation
        w = matrix_vector_mult(A, v)
        # Normierung
        v_new = normalize_vector(w)
        
        # Rayleigh-Quotient
        lambda_k = sum(v_new[i] * A[i][j] * v_new[j] 
                      for i in range(n) 
                      for j in range(n))
        
        # Konvergenzpruefung
        if vector_norm([v_new[i]-v[i] for i in range(n)]) < tol:
            return lambda_k, v_new
            
        v = v_new
    return lambda_k, v
\end{lstlisting}
\end{examplecode}

\begin{examplecode}{QR-Verfahren}
\begin{lstlisting}[language=Python, style=basesmol]
def matmul(A, B):
    n = len(A)
    C = [[0.0] * n for _ in range(n)]
    for i in range(n):
        for j in range(n):
            C[i][j] = sum(A[i][k] * B[k][j] 
                     for k in range(n))
    return C

def transpose(A):
    n = len(A)
    return [[A[j][i] for j in range(n)] 
            for i in range(n)]

def householder(x):
    n = len(x)
    # Norm berechnen
    s = sum(x[i]**2 for i in range(1, n))
    v = [0.0] * n
    if s == 0:
        return v
    
    v[0] = x[0]
    norm_x = (x[0]**2 + s)**0.5
    if x[0] <= 0:
        v[0] = x[0] - norm_x
    else:
        v[0] = -s/(x[0] + norm_x)
    
    for i in range(1, n):
        v[i] = x[i]
    
    return normalize_vector(v)

def qr_algorithm(A, tol=1e-10, max_iter=100):
    n = len(A)
    A_k = [row[:] for row in A]  # Kopiere A
    
    for k in range(max_iter):
        # QR-Zerlegung mit Householder
        Q = [[1.0 if i==j else 0.0 
              for j in range(n)] 
              for i in range(n)]
        R = [row[:] for row in A_k]
        
        for j in range(n-1):
            v = householder([R[i][j] 
                 for i in range(j, n)])
            # Householder-Transformation anwenden
            
        # Neue Iteration A_(k+1) = RQ
        A_k = matmul(R, Q)
        
        # Konvergenztest
        if max(abs(A_k[i][j]) 
               for i in range(1, n) 
               for j in range(i)) < tol:
            break
    
    return [A_k[i][i] for i in range(n)]
\end{lstlisting}
\end{examplecode}
