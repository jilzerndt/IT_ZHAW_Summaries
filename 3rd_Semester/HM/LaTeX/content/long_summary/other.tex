\begin{KR}{Fehlerabschätzung für Nullstellen}
\begin{enumerate}
    \item Voraussetzungen prüfen:
    \begin{itemize}
        \item Funktion $f$ muss stetig sein
        \item Nullstelle muss von ungerader Ordnung sein (Vorzeichenwechsel)
    \end{itemize}
    \item Fehlertoleranz $\epsilon$ festlegen
    \item Nullstelleneinschluss prüfen:
    \begin{itemize}
        \item Berechne $f(x_n-\epsilon)$ und $f(x_n+\epsilon)$
        \item Prüfe Vorzeichenwechsel: $f(x_n-\epsilon) \cdot f(x_n+\epsilon) < 0$
    \end{itemize}
    \item Fehler auswerten:
    \begin{itemize}
        \item Falls Vorzeichenwechsel: $|x_n-\xi| < \epsilon$
        \item Falls kein Vorzeichenwechsel: $\epsilon$ vergrößern und wiederholen
    \end{itemize}
    \item Zusätzliche Konvergenzprüfungen:
    \begin{itemize}
        \item Relative Änderung: $\frac{|x_n-x_{n-1}|}{|x_n|} < \epsilon_r$
        \item Residuum: $|f(x_n)| < \epsilon_f$
    \end{itemize}
\end{enumerate}
\end{KR}

\begin{KR}{Fehlerabschätzung in der Praxis}
\begin{enumerate}
    \item Numerische Fehlerabschätzung
    \begin{itemize}
        \item Absolute Änderung: $|x_n - x_{n-1}| < \epsilon_1$
        \item Funktionswert: $|f(x_n)| < \epsilon_2$
        \item Vorzeichenwechsel prüfen: $f(x_n-\epsilon) \cdot f(x_n+\epsilon) < 0$
    \end{itemize}
    
    \item Theoretische Fehlerabschätzung
    \begin{itemize}
        \item Fixpunktiteration: $|x_n-\bar{x}| \leq \frac{\alpha^n}{1-\alpha}|x_1-x_0|$
        \item Newton-Verfahren: $|x_{n+1}-\bar{x}| \leq c|x_n-\bar{x}|^2$
        \item Sekantenverfahren: $|x_{n+1}-\bar{x}| \leq c|x_n-\bar{x}|^{1.618}$
    \end{itemize}
    
    \item Zusätzliche Sicherheitsaspekte
    \begin{itemize}
        \item Divergenzcheck durchführen
        \item Überlauf/Unterlauf prüfen
        \item Division durch Null vermeiden
    \end{itemize}
\end{enumerate}
\end{KR}

\begin{KR}{Auswahl des Lösungsverfahrens}
\begin{enumerate}
    \item Analyse der Matrix:
    \begin{itemize}
        \item Dimensionen (n × n)
        \item Struktur (dicht/dünn besetzt)
        \item Konditionszahl (falls berechenbar)
    \end{itemize}
    
    \item Direkte Verfahren wenn:
    \begin{itemize}
        \item Matrix dicht besetzt und n < 1000
        \item Hohe Genauigkeit gefordert
        \item Mehrere rechte Seiten zu lösen
    \end{itemize}
    
    \item Iterative Verfahren wenn:
    \begin{itemize}
        \item Matrix dünn besetzt
        \item Matrix sehr groß (n > 1000)
        \item Moderate Genauigkeit ausreichend
        \item Matrix diagonaldominant
    \end{itemize}
    
    \item Empfohlene Methoden:
    \begin{itemize}
        \item Standard: LR mit Pivotisierung
        \item Symmetrisch positiv definit: QR
        \item Große dünn besetzte Systeme: Gauss-Seidel
        \item Schlecht konditioniert: QR
    \end{itemize}
\end{enumerate}
\end{KR}

\begin{KR}{LR-Zerlegung - Praktisches Vorgehen}
\begin{enumerate}
    \item Voraussetzungen prüfen
    \begin{itemize}
        \item Matrix regulär?
        \item Diagonalelemente ungleich Null?
        \item Pivotisierung nötig?
    \end{itemize}

    \item Zerlegung durchführen
    \begin{itemize}
        \item Matrix kopieren für R
        \item L als Einheitsmatrix initialisieren
        \item P als Einheitsmatrix initialisieren (falls Pivotisierung)
    \end{itemize}

    \item Für jede Spalte k = 1,...,n-1:
    \begin{itemize}
        \item Falls Pivotisierung: Größtes Element in Spalte k finden
        \item Zeilenvertauschung in R und P dokumentieren
        \item Eliminationsfaktoren $l_{ik} = \frac{r_{ik}}{r_{kk}}$ berechnen
        \item Zeile i von Zeile k subtrahieren: $r_{ij} := r_{ij} - l_{ik}r_{kj}$
        \item Eliminationsfaktoren in L speichern
    \end{itemize}

    \item System lösen durch
    \begin{itemize}
        \item Vorwärtseinsetzen: $Ly = Pb$
        \item Rückwärtseinsetzen: $Rx = y$
    \end{itemize}
\end{enumerate}
\end{KR}

\begin{KR}{Fehleranalyse in der Praxis}
\begin{enumerate}
    \item Analyse der Eingangsdaten
    \begin{itemize}
        \item Konditionszahl der Matrix bestimmen
        \item Struktur der Matrix untersuchen
        \item Größenordnung der Einträge prüfen
    \end{itemize}
    
    \item Fehlerquellen identifizieren
    \begin{itemize}
        \item Rundungsfehler bei der Eingabe
        \item Fehler bei der Elimination
        \item Akkumulation von Rundungsfehlern
        \item Instabilitäten durch schlechte Kondition
    \end{itemize}
    
    \item Fehlerabschätzungen berechnen
    \begin{itemize}
        \item A-priori: $\frac{\|\Delta x\|}{\|x\|} \leq \text{cond}(A) \cdot \frac{\|\Delta b\|}{\|b\|}$
        \item A-posteriori: $\|Ax-b\|$ (Residuum)
        \item Iterative Verfahren: Konvergenzrate
    \end{itemize}
    
    \item Maßnahmen zur Fehlerreduktion
    \begin{itemize}
        \item Pivotisierung verwenden
        \item Skalierung der Matrix
        \item Höhere Genauigkeit verwenden
        \item Alternatives Lösungsverfahren wählen
    \end{itemize}
\end{enumerate}
\end{KR}

\begin{example2}{PageRank-Algorithmus}
Google-Matrix für ein kleines Webnetzwerk mit 3 Seiten:
$$A = \begin{psmallmatrix}
0 & 1/2 & 1/3\\
1 & 0 & 1/3\\
0 & 1/2 & 1/3
\end{psmallmatrix}$$

Der PageRank entspricht dem Eigenvektor zum Eigenwert 1:
\begin{itemize}
    \item $\lambda = 1$ ist größter Eigenwert
    \item PageRank: $v \approx (0.39, 0.41, 0.20)^T$
    \item Seite 2 hat höchste Relevanz, Seite 3 niedrigste
\end{itemize}
\end{example2}