\section{Numerische Lösung linearer Gleichungssysteme}

\subsection{Grundlagen}

\begin{concept}{Lineares Gleichungssystem}\\
Ein lineares Gleichungssystem der Form $Ax = b$ besteht aus:
\vspace{2mm}\\
\resizebox{\linewidth}{!}{
$A=\left[\begin{array}{ccc}
a_{11} & \cdots & a_{1n} \\
\vdots & \ddots & \vdots \\
a_{n1} & \cdots & a_{nn}
\end{array}\right] \in \mathbb{R}^{n \times n}, \quad 
x=\left(\begin{array}{c}
x_1 \\
\vdots \\
x_n
\end{array}\right) \in \mathbb{R}^n, \quad 
b=\left(\begin{array}{c}
b_1 \\
\vdots \\
b_n
\end{array}\right) \in \mathbb{R}^n$}
\end{concept}

\subsection{Der Gauss-Algorithmus}

\begin{concept}{Grundidee Gauss-Elimination}\\
Transformation des Gleichungssystems $Ax=b$ in ein äquivalentes System $\tilde{A}x=\tilde{b}$, wobei $\tilde{A}$ eine obere Dreiecksmatrix ist.

Erlaubte Operationen:
\begin{itemize}
    \item $z_j := z_j - \lambda z_i$ für $i<j$ und $\lambda \in \mathbb{R}$
    \item $z_i \rightarrow z_j$ (Vertauschen von Zeilen)
\end{itemize}
\end{concept}

\begin{KR}{Gauss-Algorithmus}
\paragraph{1. Elimination (Vorwärts)}
\begin{enumerate}
    \item Für $i=1,\ldots,n-1$:
    \item \quad Für $j=i+1,\ldots,n$:
    \item \quad\quad Berechne $\lambda_{ji} = a_{ji}/a_{ii}$
    \item \quad\quad $z_j := z_j - \lambda_{ji} z_i$
\end{enumerate}

\paragraph{2. Rückwärtseinsetzen}
$$x_i = \frac{b_i - \sum_{j=i+1}^n a_{ij}x_j}{a_{ii}}, \quad i=n,n-1,\ldots,1$$
\end{KR}

\begin{example2}{Gauss-Elimination}
\vspace{-3mm}\\
Gegebenes System:
$A = \begin{pmatrix}
2 & 1 & -1\\
4 & -1 & 3\\
-2 & 2 & 1
\end{pmatrix}, \quad b = \begin{pmatrix}
3\\
1\\
4
\end{pmatrix}$
\vspace{2mm}\\
\textbf{  Eliminationsschritte: }
\vspace{2mm}\\
$\begin{pmatrix}
    2 & 1 & -1 & | & 3\\
    0 & -3 & 5 & | & -5\\
    0 & 3 & -1 & | & 7
    \end{pmatrix}
    \Rightarrow
    \begin{pmatrix}
    2 & 1 & -1 & | & 3\\
    0 & -3 & 5 & | & -5\\
    0 & 0 & 4 & | & 2
    \end{pmatrix}$

    \raggedright
$
\begin{array}{lrl}
    \textbf{Rückwärtseinsetzen: }& x_3 &= \frac{2}{4} = \frac{1}{2}\\
    &x_2 &= \frac{-5 - 5(\frac{1}{2})}{-3} = -2\\
    &x_1 &= \frac{3 - 1(-2) - (-1)(\frac{1}{2})}{2} = 1
\end{array}
$



\begin{enumerate}
    \item Elimination der ersten Spalte:
    $$\begin{pmatrix}
    2 & 1 & -1 & | & 3\\
    0 & -3 & 5 & | & -5\\
    0 & 3 & -1 & | & 7
    \end{pmatrix}$$
    
    \item Elimination der zweiten Spalte:
    $$\begin{pmatrix}
    2 & 1 & -1 & | & 3\\
    0 & -3 & 5 & | & -5\\
    0 & 0 & 4 & | & 2
    \end{pmatrix}$$
    
    \item Rückwärtseinsetzen:
    \begin{align*}
        x_3 &= \frac{2}{4} = \frac{1}{2}\\
        x_2 &= \frac{-5 - 5(\frac{1}{2})}{-3} = -2\\
        x_1 &= \frac{3 - 1(-2) - (-1)(\frac{1}{2})}{2} = 1
    \end{align*}
\end{enumerate}
\end{example2}

\subsection{Pivotisierung}

\begin{concept}{Spaltenpivotisierung}\\
Strategie zur numerischen Stabilisierung des Gauss-Algorithmus durch Auswahl des betragsmäßig größten Elements als Pivotelement.

Vor jedem Eliminationsschritt in Spalte $i$:
\begin{itemize}
    \item Suche $k$ mit $|a_{ki}| = \max\{|a_{ji}| \mid j = i,\ldots,n\}$
    \item Falls $a_{ki} \neq 0$: Vertausche Zeilen $i$ und $k$
    \item Falls $a_{ki} = 0$: Matrix ist singulär
\end{itemize}
\end{concept}

\begin{KR}{Gauss mit Pivotisierung}\\
Erweiterter Gauss-Algorithmus mit Spaltenpivotisierung:
\begin{enumerate}
    \item Für $i=1,\ldots,n-1$:
    \item \quad Finde $k \geq i$ mit $|a_{ki}| = \max\{|a_{ji}| \mid j = i,\ldots,n\}$
    \item \quad Falls $a_{ki} = 0$: Stop (Matrix singulär)
    \item \quad Vertausche Zeilen $i$ und $k$
    \item \quad Für $j=i+1,\ldots,n$:
    \item \quad\quad $z_j := z_j - \frac{a_{ji}}{a_{ii}}z_i$
\end{enumerate}
\end{KR}

\begin{example2}{Pivotisierung}{Gauss-Elimination mit Pivotisierung}
System:
$$\begin{pmatrix}
1 & 2 & -1\\
4 & 8 & -3\\
9 & 18 & -8
\end{pmatrix}x = \begin{pmatrix}
1\\
4\\
9
\end{pmatrix}$$

\begin{enumerate}
    \item Erste Spalte: Pivot $|9|$ → Tausche Zeilen 1 und 3
    \item Nach Elimination der ersten Spalte:
    $$\begin{pmatrix}
    9 & 18 & -8 & | & 9\\
    0 & 0 & 0.89 & | & 0\\
    0 & 0 & 0.89 & | & 0
    \end{pmatrix}$$
    \item System ist schlecht konditioniert (identische Zeilen)
\end{enumerate}
\end{example2}

\subsection{Matrix-Zerlegungen}

\begin{definition}{Dreieckszerlegung}\\
Eine Matrix $A \in \mathbb{R}^{n\times n}$ kann zerlegt werden in:
\vspace{1mm}\\
\begin{minipage}[t]{0.45\textwidth}
    \textbf{Untere Dreiecksmatrix L:}\\
    $l_{ij} = 0$ für $j > i$\\
    Diagonale meist normiert ($l_{ii}=1$)
\end{minipage}
\hspace{3mm}
\begin{minipage}[t]{0.45\textwidth}
    \textbf{Obere Dreiecksmatrix R:}\\
    $r_{ij} = 0$ für $i > j$\\
    Diagonalelemente $\neq 0$
\end{minipage}
\end{definition}

\begin{theorem}{LR-Zerlegung}\\
Jede reguläre Matrix $A$, für die der Gauss-Algorithmus ohne Zeilenvertauschungen durchführbar ist, lässt sich zerlegen in:
$$A = LR$$
wobei $L$ eine normierte untere und $R$ eine obere Dreiecksmatrix ist.
\end{theorem}

\begin{KR}{Berechnung der LR-Zerlegung}\\
So berechnen Sie die LR-Zerlegung:
\begin{enumerate}
    \item Führen Sie Gauss-Elimination durch
    \item $R$ ist die resultierende obere Dreiecksmatrix
    \item Die Eliminationsfaktoren $-\frac{a_{ji}}{a_{ii}}$ bilden $L$
    \item Lösen Sie dann nacheinander:
        \begin{itemize}
            \item $Ly = b$ (Vorwärtseinsetzen)
            \item $Rx = y$ (Rückwärtseinsetzen)
        \end{itemize}
\end{enumerate}
\end{KR}

\begin{example2}{LR-Zerlegung}{Berechnung einer LR-Zerlegung}
Matrix:
$$A = \begin{pmatrix}
2 & 1 & 1\\
4 & -1 & 0\\
-2 & 3 & 1
\end{pmatrix}$$

Schrittweise Elimination führt zu:
$$L = \begin{pmatrix}
1 & 0 & 0\\
2 & 1 & 0\\
-1 & 2 & 1
\end{pmatrix}, \quad
R = \begin{pmatrix}
2 & 1 & 1\\
0 & -3 & -2\\
0 & 0 & -2
\end{pmatrix}$$
\end{example2}

\begin{concept}{QR-Zerlegung}\\
Eine orthogonale Matrix $Q \in \mathbb{R}^{n\times n}$ erfüllt: $Q^T Q = QQ^T = I_n$

Die QR-Zerlegung einer Matrix $A$ ist:
$$A = QR$$
wobei $Q$ orthogonal und $R$ eine obere Dreiecksmatrix ist.
\end{concept}

\begin{definition}{Householder-Transformation}\\
Eine Householder-Matrix hat die Form:
$$H = I_n - 2uu^T$$
mit $u \in \mathbb{R}^n$, $\|u\| = 1$. Es gilt:
\begin{itemize}
    \item $H$ ist orthogonal ($H^T = H^{-1}$)
    \item $H$ ist symmetrisch ($H^T = H$)
    \item $H^2 = I_n$
\end{itemize}
\end{definition}

\begin{KR}{QR-Zerlegung mit Householder}\\
So berechnen Sie die QR-Zerlegung:
\begin{enumerate}
    \item Initialisierung: $R := A$, $Q := I_n$
    \item Für $i = 1,\ldots,n-1$:
        \begin{itemize}
            \item Bilde Vektor $v_i$ aus i-ter Spalte von $R$ ab Position $i$
            \item $w_i := v_i + \text{sign}(v_{i1})\|v_i\|e_1$
            \item $u_i := w_i/\|w_i\|$
            \item $H_i := I_{n-i+1} - 2u_iu_i^T$
            \item Erweitere $H_i$ zu $Q_i$ durch $I_{i-1}$ links oben
            \item $R := Q_iR$
            \item $Q := QQ_i^T$
        \end{itemize}
\end{enumerate}
\end{KR}

\begin{example2}{QR-Zerlegung}{Berechnung einer QR-Zerlegung}
Matrix:
$$A = \begin{pmatrix}
1 & 1\\
1 & 0\\
0 & 1
\end{pmatrix}$$

Erste Householder-Transformation:
\begin{enumerate}
    \item $v_1 = (1,1,0)^T$
    \item $w_1 = (1,1,0)^T + \sqrt{2}(1,0,0)^T$
    \item $u_1 = \frac{1}{\sqrt{6}}(1+\sqrt{2},1,0)^T$
    \item $H_1 = I_3 - 2u_1u_1^T$
\end{enumerate}

Nach allen Transformationen:
$$Q = \begin{pmatrix}
-0.7071 & -0.5774 & -0.4082\\
-0.7071 & 0.5774 & 0.4082\\
0 & -0.5774 & 0.8165
\end{pmatrix}, \quad
R = \begin{pmatrix}
-1.4142 & -0.7071\\
0 & -1.2247\\
0 & 0
\end{pmatrix}$$
\end{example2}

\subsection{Fehleranalyse}

\begin{definition}{Matrix- und Vektornormen}\\
Eine Vektornorm $\|\cdot\|$ erfüllt für alle $x,y \in \mathbb{R}^n, \lambda \in \mathbb{R}$:
\begin{itemize}
    \item $\|x\| \geq 0$ und $\|x\| = 0 \Leftrightarrow x = 0$
    \item $\|\lambda x\| = |\lambda| \cdot \|x\|$
    \item $\|x + y\| \leq \|x\| + \|y\|$ (Dreiecksungleichung)
\end{itemize}
\end{definition}

\begin{concept}{Wichtige Normen}\\
\begin{minipage}[t]{0.3\textwidth}
    \textbf{1-Norm:}\\
    $\|x\|_1 = \sum_{i=1}^n |x_i|$\\
    $\|A\|_1 = \max_j \sum_{i=1}^n |a_{ij}|$
\end{minipage}
\hspace{2mm}
\begin{minipage}[t]{0.3\textwidth}
    \textbf{2-Norm:}\\
    $\|x\|_2 = \sqrt{\sum_{i=1}^n x_i^2}$\\
    $\|A\|_2 = \sqrt{\rho(A^TA)}$
\end{minipage}
\hspace{2mm}
\begin{minipage}[t]{0.3\textwidth}
    $\infty$\textbf{-Norm:}\\
    $\|x\|_\infty = \max_i |x_i|$\\
    $\|A\|_\infty = \max_i \sum_{j=1}^n |a_{ij}|$
\end{minipage}
\end{concept}

\begin{theorem}{Fehlerabschätzung für LGS}\\
Sei $\|\cdot\|$ eine Norm, $A \in \mathbb{R}^{n\times n}$ regulär und $Ax = b$, $A\tilde{x} = \tilde{b}$. Dann gilt:
\vspace{1mm}\\
\begin{minipage}[t]{0.47\textwidth}
    \textbf{Absoluter Fehler:}\\
    $\|x - \tilde{x}\| \leq \|A^{-1}\| \cdot \|b - \tilde{b}\|$
\end{minipage}
\hspace{2mm}
\begin{minipage}[t]{0.47\textwidth}
    \textbf{Relativer Fehler:}\\
    $\frac{\|x - \tilde{x}\|}{\|x\|} \leq \text{cond}(A) \cdot \frac{\|b - \tilde{b}\|}{\|b\|}$
\end{minipage}

Mit der Konditionszahl $\text{cond}(A) = \|A\| \cdot \|A^{-1}\|$
\end{theorem}

\begin{concept}{Konditionierung}\\
Die Konditionszahl beschreibt die numerische Stabilität eines LGS:
\begin{itemize}
    \item $\text{cond}(A) \approx 1$: gut konditioniert
    \item $\text{cond}(A) \gg 1$: schlecht konditioniert
    \item $\text{cond}(A) \to \infty$: singulär
\end{itemize}
\end{concept}

\subsection{Iterative Verfahren}

\begin{definition}{Zerlegung der Systemmatrix}\\
Für iterative Verfahren wird $A$ zerlegt in:
$$A = L + D + R$$
wobei:
\begin{itemize}
    \item $L$: streng untere Dreiecksmatrix
    \item $D$: Diagonalmatrix
    \item $R$: streng obere Dreiecksmatrix
\end{itemize}
\end{definition}

\begin{concept}{Jacobi-Verfahren}\\
Gesamtschrittverfahren mit der Iteration:
$$x^{(k+1)} = -D^{-1}(L + R)x^{(k)} + D^{-1}b$$

Komponentenweise:
$$x_i^{(k+1)} = \frac{1}{a_{ii}}\left(b_i - \sum_{j=1,j\neq i}^n a_{ij}x_j^{(k)}\right)$$
\end{concept}

\begin{concept}{Gauss-Seidel-Verfahren}\\
Einzelschrittverfahren mit der Iteration:
$$x^{(k+1)} = -(D+L)^{-1}Rx^{(k)} + (D+L)^{-1}b$$

Komponentenweise:
$$x_i^{(k+1)} = \frac{1}{a_{ii}}\left(b_i - \sum_{j=1}^{i-1} a_{ij}x_j^{(k+1)} - \sum_{j=i+1}^n a_{ij}x_j^{(k)}\right)$$
\end{concept}

\begin{theorem}{Konvergenzkriterien}\\
Ein iteratives Verfahren konvergiert, wenn:
\begin{enumerate}
    \item Die Matrix $A$ diagonaldominant ist:\\
    $|a_{ii}| > \sum_{j\neq i} |a_{ij}|$ für alle $i$
    \item Der Spektralradius der Iterationsmatrix kleiner 1 ist:\\
    $\rho(B) < 1$ mit $B$ als jeweilige Iterationsmatrix
\end{enumerate}
\end{theorem}

\begin{KR}{Implementation iterativer Verfahren}\\
So implementieren Sie iterative Verfahren:
\begin{enumerate}
    \item Wählen Sie Startvektor $x^{(0)}$
    \item Wählen Sie Abbruchkriterien:
        \begin{itemize}
            \item Maximale Iterationszahl $k_{max}$
            \item Toleranz $\epsilon$ für Änderung $\|x^{(k+1)} - x^{(k)}\|$
            \item Toleranz für Residuum $\|Ax^{(k)} - b\|$
        \end{itemize}
    \item Führen Sie Iteration durch bis Kriterien erfüllt
\end{enumerate}
\end{KR}

\begin{example2}{Iterative Verfahren}{Vergleich Jacobi und Gauss-Seidel}
System:
$$\begin{pmatrix}
4 & -1 & 0\\
-1 & 4 & -1\\
0 & -1 & 4
\end{pmatrix}x = \begin{pmatrix}
1\\
5\\
0
\end{pmatrix}$$

\begin{center}
\begin{tabular}{c|cc|cc}
k & \multicolumn{2}{c|}{Jacobi} & \multicolumn{2}{c}{Gauss-Seidel}\\
\hline
0 & $(0,0,0)^T$ & & $(0,0,0)^T$ &\\
1 & $(0.25,1.25,0)^T$ & 1.25 & $(0.25,1.31,0.08)^T$ & 1.31\\
2 & $(0.31,1.31,0.31)^T$ & 0.31 & $(0.33,1.33,0.33)^T$ & 0.02\\
3 & $(0.33,1.33,0.33)^T$ & 0.02 & $(0.33,1.33,0.33)^T$ & 0.00
\end{tabular}
\end{center}
\end{example2}

