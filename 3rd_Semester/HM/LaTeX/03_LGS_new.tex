\section{Numerische Lösung linearer Gleichungssysteme}

\subsubsection{Pivotisierung}

\begin{concept}{Spaltenpivotisierung}\\
Strategie zur numerischen Stabilisierung des Gauss-Algorithmus durch Auswahl des betragsmäßig größten Elements als Pivotelement.

Vor jedem Eliminationsschritt in Spalte $i$:
\begin{itemize}
    \item Suche $k$ mit $|a_{ki}| = \max\{|a_{ji}| \mid j = i,\ldots,n\}$
    \item Falls $a_{ki} \neq 0$: Vertausche Zeilen $i$ und $k$
    \item Falls $a_{ki} = 0$: Matrix ist singulär
\end{itemize}
\end{concept}

\begin{KR}{Gauss-Algorithmus mit Pivotisierung}
\paragraph{1. Elimination (Vorwärts)}
\begin{enumerate}
    \item Für $i=1,\ldots,n-1$:
    \item \quad Finde $k \geq i$ mit $|a_{ki}| = \max\{|a_{ji}| \mid j = i,\ldots,n\}$
    \item \quad Falls $a_{ki} = 0$: Stop (Matrix singulär)
    \item \quad Vertausche Zeilen $i$ und $k$
    \item \quad Für $j=i+1,\ldots,n$:
    \item \quad\quad $z_j := z_j - \frac{a_{ji}}{a_{ii}}z_i$
\end{enumerate}

\paragraph{2. Rückwärtseinsetzen}
$$x_i = \frac{b_i - \sum_{j=i+1}^n a_{ij}x_j}{a_{ii}}, \quad i=n,n-1,\ldots,1$$
\end{KR}


\begin{example2}{Gauss mit Pivotisierung}
\vspace{-3mm}\\
Gegebenes System:
$A = \begin{pmatrix}
0 & 1 & 1\\
2 & 4 & -2\\
0 & 3 & 15
\end{pmatrix}, \quad b = \begin{pmatrix}
4\\
2\\
36
\end{pmatrix}$
\vspace{2mm}\\
\textbf{Eliminationsschritte: }
\vspace{2mm}\\
$\begin{pmatrix}
    2 & 4 & -2 & | & 2\\
    0 & 3 & 15 & | & 36\\
    0 & 1 & 1 & | & 4
    \end{pmatrix}
    \Rightarrow
    \begin{pmatrix}
    2 & 4 & -2 & | & 2\\
    0 & 3 & 15 & | & 36\\
    0 & 0 & -2 & | & -8
    \end{pmatrix}$

    \vspace{2mm}
$
\begin{array}{lrl}
    \textbf{Rückwärtseinsetzen: }& x_3 &= \frac{-8}{-2} = 4\\
    &x_2 &= \frac{36 - 15(4)}{3} = 1\\
    &x_1 &= \frac{2 - 4(4) + 2}{2} = -6
\end{array}
$
    
\end{example2}



\begin{concept}{Permutationsmatrizen}\\
Eine Permutationsmatrix $P$ ist eine Matrix, die aus der Einheitsmatrix durch Zeilenvertauschungen entsteht. Für die Vertauschung der $i$-ten und $j$-ten Zeile hat $P_k$ die Form:
\begin{itemize}
    \item $p_{ii} = p_{jj} = 0$ 
    \item $p_{ij} = p_{ji} = 1$
    \item Alle anderen Elemente wie in $I_n$
\end{itemize}
Wichtige Eigenschaften:
\begin{itemize}
    \item $P^{-1} = P^T = P$
    \item Mehrere Vertauschungen: $P = P_l \cdot ... \cdot P_1$
\end{itemize}
\end{concept}

\begin{example2}{Zeilenvertauschung} für Matrix A mit Permutationsmatrix $P_1$:\\

$\underbrace{\begin{pmatrix}
1 & 2 & 3\\
4 & 5 & 6\\
7 & 8 & 9
\end{pmatrix}}_{A} \cdot 
\underbrace{\begin{pmatrix}
0 & 0 & 1\\
0 & 1 & 0\\
1 & 0 & 0
\end{pmatrix}}_{P_1} =
\begin{pmatrix}
7 & 8 & 9\\
4 & 5 & 6\\
1 & 2 & 3
\end{pmatrix}$
\vspace{2mm}\\
$\Rightarrow A \cdot P_1$ bewirkt die Vertauschung von Zeile 1 und 3
\end{example2}

\subsection{Matrix-Zerlegungen}

\begin{definition}{Dreieckszerlegung}
Eine Matrix $A \in \mathbb{R}^{n\times n}$ kann zerlegt werden in:
\vspace{1mm}\\
\begin{minipage}[t]{0.5\textwidth}
    \textbf{Untere Dreiecksmatrix L:}\\
    $l_{ij} = 0$ für $j > i$\\
    Diagonale normiert ($l_{ii}=1$)
\end{minipage}
\hspace{3mm}
\begin{minipage}[t]{0.45\textwidth}
    \textbf{Obere Dreiecksmatrix R:}\\
    $r_{ij} = 0$ für $i > j$\\
    Diagonalelemente $\neq 0$
\end{minipage}
\end{definition}

\subsubsection{LR-Zerlegung}

\begin{theorem}{LR-Zerlegung}\\
Jede reguläre Matrix $A$, für die der Gauss-Algorithmus ohne Zeilenvertauschungen durchführbar ist, lässt sich zerlegen in:
$A = LR$
wobei $L$ eine normierte untere und $R$ eine obere Dreiecksmatrix ist.
\end{theorem}

\begin{KR}{Berechnung der LR-Zerlegung}\\
So berechnen Sie die LR-Zerlegung:
\begin{enumerate}
    \item Führen Sie Gauss-Elimination durch
    \item $R$ ist die resultierende obere Dreiecksmatrix
    \item Die Eliminationsfaktoren $-\frac{a_{ji}}{a_{ii}}$ bilden $L$
    \item Lösen Sie dann nacheinander:
        \begin{itemize}
            \item $Ly = b$ (Vorwärtseinsetzen)
            \item $Rx = y$ (Rückwärtseinsetzen)
        \end{itemize}
\end{enumerate}
\end{KR}

\begin{example2}[breakable]{LR-Zerlegung}
$A = \begin{pmatrix}
-1 & 1 & 1\\
1 & -3 & -2\\
5 & 1 & 4
\end{pmatrix}, \quad b = \begin{pmatrix}
0\\
5\\
3
\end{pmatrix}$

\paragraph{Schritt 1: Erste Spalte}
Max. Element in 1. Spalte: $|a_{31}| = 5$, also Z1 und Z3 tauschen:
$$P_1 = \begin{pmatrix}
0 & 0 & 1\\
0 & 1 & 0\\
1 & 0 & 0
\end{pmatrix}, \quad A^{(1)} = \begin{pmatrix}
5 & 1 & 4\\
1 & -3 & -2\\
-1 & 1 & 1
\end{pmatrix}$$
Berechne Eliminationsfaktoren:
$l_{21} = \frac{1}{5}, \quad l_{31} = -\frac{1}{5}$
\vspace{2mm}\\
Nach Elimination:
$A^{(2)} = \begin{pmatrix}
5 & 1 & 4\\
0 & -3.2 & -2.8\\
0 & 1.2 & 1.8
\end{pmatrix}$

\paragraph{Schritt 2: Zweite Spalte}
Max. Element in 2. Spalte unter Diagonale: $|-3.2| > |1.2|$, \\ keine Vertauschung nötig.
\vspace{2mm}\\
Berechne Eliminationsfaktor:
$l_{32} = -\frac{1.2}{-3.2} = \frac{3}{8}$
\vspace{2mm}\\
Nach Elimination:
$R = \begin{pmatrix}
5 & 1 & 4\\
0 & -3.2 & -2.8\\
0 & 0 & 2.85
\end{pmatrix}$

\paragraph{Endergebnis}
Die LR-Zerlegung mit $PA = LR$ ist:
\vspace{2mm}\\
$$P = P_1 = \begin{pmatrix}
0 & 0 & 1\\
0 & 1 & 0\\
1 & 0 & 0
\end{pmatrix}$$, 
$$L = \begin{pmatrix}
1 & 0 & 0\\
\frac{1}{5} & 1 & 0\\
-\frac{1}{5} & \frac{3}{8} & 1
\end{pmatrix}, 
R = \begin{pmatrix}
5 & 1 & 4\\
0 & -3.2 & -2.8\\
0 & 0 & 2.85
\end{pmatrix}$$

\paragraph{Lösung des Systems}
\begin{enumerate}
    \item $Pb = \begin{psmallmatrix} 3\\ 5\\ 0 \end{psmallmatrix}$
    \item Löse $Ly = Pb$ durch Vorwärtseinsetzen:
    $y = \begin{pmatrix} 3\\ 4.4\\ 2.85 \end{pmatrix}$
    \item Löse $Rx = y$ durch Rückwärtseinsetzen:
    $x = \begin{pmatrix} 1\\ -1\\ 1 \end{pmatrix}$
\end{enumerate}

\paragraph{Probe}
$$Ax = \begin{pmatrix}
-1 & 1 & 1\\
1 & -3 & -2\\
5 & 1 & 4
\end{pmatrix} \begin{pmatrix} 1\\ -1\\ 1 \end{pmatrix} = \begin{pmatrix} 0\\ 5\\ 3 \end{pmatrix} = b$$
\end{example2}

\begin{KR}{Zeilenvertauschungen verfolgen}
\begin{enumerate}
    \item Initialisiere $P = I_n$
    \item Für jede Vertauschung von Zeile $i$ und $j$:
    \begin{itemize}
        \item Erstelle $P_k$ durch Vertauschen von Zeilen $i,j$ in $I_n$
        \item Aktualisiere $P = P_k \cdot P$
        \item Wende Vertauschung auf Matrix an: $A := P_kA$
    \end{itemize}
    \item Bei der LR-Zerlegung mit Pivotisierung:
    \begin{itemize}
        \item $PA = LR$ 
        \item Löse $Ly = Pb$ und $Rx = y$
    \end{itemize}
\end{enumerate}
\end{KR}

\begin{examplecode}{LR-Zerlegung mit Pivotisierung}
\begin{lstlisting}[language=Python, style=basesmol]
def lr_decomposition_with_pivoting(A):
    n = len(A)
    P = np.eye(n)    # Permutationsmatrix
    L = np.eye(n)    # Untere Dreiecksmatrix
    R = A.copy()     # Wird zur oberen Dreiecksmatrix
    
    for k in range(n-1):
        # Finde Pivotelement
        pivot = np.argmax(abs(R[k:,k])) + k
        
        if pivot != k:
            # Erzeuge Permutationsmatrix
            P_k = np.eye(n)
            P_k[[k,pivot]] = P_k[[pivot,k]]
            
            # Aktualisiere Matrizen
            P = P_k @ P
            R[[k,pivot]] = R[[pivot,k]]
            if k > 0:
                L[[k,pivot], :k] = L[[pivot,k], :k]
                
        # Elimination durchfuehren
        for i in range(k+1, n):
            factor = R[i,k] / R[k,k]
            L[i,k] = factor
            R[i,k:] -= factor * R[k,k:]
            
    return P, L, R
\end{lstlisting}
\end{examplecode}



\begin{concept}{Vorteile der Permutationsmatrix}
    \begin{itemize}
        \item Exakte Nachverfolgung aller Zeilenvertauschungen
        \item Einfache Rückführung auf ursprüngliche Reihenfolge durch $P^{-1}$
        \item Kompakte Darstellung mehrerer Vertauschungen
        \item Numerisch stabile Implementierung der Pivotisierung
    \end{itemize}
\end{concept}

\columnbreak

\subsubsection{QR-Zerlegung}

\begin{concept}{QR-Zerlegung}\\
Eine orthogonale Matrix $Q \in \mathbb{R}^{n\times n}$ erfüllt: $Q^T Q = QQ^T = I_n$
\vspace{1mm}\\
Die QR-Zerlegung einer Matrix $A$ ist: $A = QR$
\vspace{1mm}\\
wobei $Q$ orthogonal und $R$ eine obere Dreiecksmatrix ist.
\end{concept}

\begin{definition}{Householder-Transformation}\\
Eine Householder-Matrix hat die Form:
$H = I_n - 2uu^T$

mit $u \in \mathbb{R}^n$, $\|u\| = 1$. Es gilt:
\begin{itemize}
    \item $H$ ist orthogonal ($H^T = H^{-1}$)
    \item $H$ ist symmetrisch ($H^T = H$)
    \item $H^2 = I_n$
\end{itemize}
\end{definition}

\begin{KR}{QR-Zerlegung mit Householder}
\begin{enumerate}
    \item Initialisierung: $R := A$, $Q := I_n$
    \item Für $i = 1,\ldots,n-1$:
        \begin{itemize}
            \item Bilde Vektor $v_i$ aus i-ter Spalte von $R$ ab Position $i$
            \item $w_i := v_i + \text{sign}(v_{i1})\|v_i\|e_1$
            \item $u_i := w_i/\|w_i\|$
            \item $H_i := I_{n-i+1} - 2u_iu_i^T$
            \item Erweitere $H_i$ zu $Q_i$ durch $I_{i-1}$ links oben
            \item $R := Q_iR$
            \item $Q := QQ_i^T$
        \end{itemize}
\end{enumerate}
\end{KR}

\begin{example2}[breakable]{QR-Zerlegung}
$A = \begin{pmatrix}
1 & -2 & 0\\
4 & -2 & 6\\
3 & 1 & 0
\end{pmatrix}$

\paragraph{Schritt 1: Erste Spalte}
Erste Spalte $a_1$ und Einheitsvektor $e_1$:
$a_1 = \begin{psmallmatrix} 1\\ 4\\ 3 \end{psmallmatrix}, \quad e_1 = \begin{psmallmatrix} 1\\ 0\\ 0 \end{psmallmatrix}$

\textbf{Householder-Vektor berechnen:}
\vspace{1mm}
\begin{enumerate}
    \item $|a_1| = \sqrt{1^2 + 4^2 + 3^2} = \sqrt{26} \approx 5.10$
    \item $v_1 = a_1 + \sqrt{26}e_1 = \begin{psmallmatrix} 6.10\\ 4\\ 3 \end{psmallmatrix}$,
    $|v_1| = \sqrt{6.10^2 + 4^2 + 3^2} = 7.89$
    \item $u_1 = \frac{v_1}{|v_1|} = \begin{psmallmatrix} 0.77\\ 0.51\\ 0.38 \end{psmallmatrix}$
\end{enumerate}

Householder-Matrix $H_1 = I - 2u_1u_1^T = \begin{pmatrix}
-0.20 & -0.78 & -0.59\\
-0.78 & 0.49 & -0.39\\
-0.59 & -0.39 & 0.71
\end{pmatrix}$

Nach Anwendung auf $A$:
\vspace{-3mm}
$$A^{(1)} = H_1A = \begin{pmatrix}
-5.10 & 0.59 & -4.51\\
0 & -2.93 & 3.70\\
0 & 0.31 & -1.73
\end{pmatrix}$$

\paragraph{Schritt 2: Zweite Spalte}
Untermatrix $A_2$ (ohne erste Zeile/Spalte):
$A_2 = \begin{pmatrix} -2.93 & 3.70\\ 0.31 & -1.73 \end{pmatrix}$

\textbf{Householder-Vektor berechnen:}
\vspace{1mm}
\begin{enumerate}
    \item $|a_2| = \sqrt{(-2.93)^2 + 0.31^2} = 2.94$
    \item $v_2 = \begin{pmatrix} -5.87\\ 0.31 \end{pmatrix}$
    \item $|v_2| = 5.88$
    \item $u_2 = \begin{pmatrix} -1.00\\ 0.05 \end{pmatrix}$
\end{enumerate}

Erweiterte Householder-Matrix $Q_2 = \begin{pmatrix}
1 & 0 & 0\\
0 & -0.99 & 0.10\\
0 & 0.10 & 0.99
\end{pmatrix}$

Nach Anwendung auf $A^{(1)}$:
$R = Q_2A^{(1)} = \begin{pmatrix}
-5.10 & 0.59 & -4.51\\
0 & 2.94 & -3.86\\
0 & 0 & -1.33
\end{pmatrix}$

\paragraph{Endergebnis}
Die QR-Zerlegung $A = QR$ ist:
$$Q = H_1^TQ_2^T = \begin{pmatrix}
-0.20 & 0.72 & -0.67\\
-0.78 & -0.52 & -0.33\\
-0.59 & 0.46 & 0.67
\end{pmatrix}$$

$$R = \begin{pmatrix}
-5.10 & 0.59 & -4.51\\
0 & 2.94 & -3.86\\
0 & 0 & -1.33
\end{pmatrix}$$

\paragraph{Probe}
\vspace{-2mm}
$$QR = \begin{pmatrix}
1 & -2 & 0\\
4 & -2 & 6\\
3 & 1 & 0
\end{pmatrix} = A$$
\end{example2}

\begin{examplecode}{QR-Zerlegung Implementation}
\begin{lstlisting}[language=Python, style=basesmol]
def householder_vector(x):
    # Berechne Householder-Vektor fuer Spalte x
    alpha = np.linalg.norm(x)
    v = x.copy()
    v[0] += np.sign(x[0]) * alpha
    v = v / np.linalg.norm(v)
    return v

def householder_reflection(A, k):
    m, n = A.shape
    v = householder_vector(A[k:, k])

    # Householder-Matrix anwenden
    H = np.eye(m-k)
    H -= 2 * np.outer(v, v)

    # Auf Untermatrix anwenden
    A[k:, k:] = H @ A[k:, k:]
    return A

def qr_householder(A):
    m, n = A.shape
    R = A.copy()
    Q = np.eye(m)

    for k in range(n):
        v = householder_vector(R[k:, k])
        H = np.eye(m)
        H[k:, k:] -= 2 * np.outer(v, v)
        R = H @ R
        Q = Q @ H.T
    
    return Q, R
\end{lstlisting}

\paragraph{Numerische Vorteile}
\begin{itemize}
    \item Numerisch stabil
    \item Keine Wurzeloperationen während der Elimination
    \item Orthogonalität der Transformation bleibt erhalten
    \item Gute Eignung für Eigenwertberechnung
\end{itemize}
\end{examplecode}

\subsection{Fehleranalyse}

\begin{definition}{Matrix- und Vektornormen}\\
Eine Vektornorm $\|\cdot\|$ erfüllt für alle $x,y \in \mathbb{R}^n, \lambda \in \mathbb{R}$:
\begin{itemize}
    \item $\|x\| \geq 0$ und $\|x\| = 0 \Leftrightarrow x = 0$
    \item $\|\lambda x\| = |\lambda| \cdot \|x\|$
    \item $\|x + y\| \leq \|x\| + \|y\|$ (Dreiecksungleichung)
\end{itemize}
\end{definition}

\begin{concept}{Wichtige Normen}

\textbf{1-Norm:}
        $$\|x\|_1 = \sum_{i=1}^n |x_i|,
        \|A\|_1 = \max_j \sum_{i=1}^n |a_{ij}|$$
\textbf{2-Norm:}
        $$\|x\|_2 = \sqrt{\sum_{i=1}^n x_i^2}, 
        \|A\|_2 = \sqrt{\rho(A^TA)}$$
$\infty$\textbf{-Norm:}
        $$\|x\|_\infty = \max_i |x_i|, 
        \|A\|_\infty = \max_i \sum_{j=1}^n |a_{ij}|$$
\end{concept}

\begin{theorem}{Fehlerabschätzung für LGS}\\
Sei $\|\cdot\|$ eine Norm, $A \in \mathbb{R}^{n\times n}$ regulär und $Ax = b$, $A\tilde{x} = \tilde{b}$
\vspace{1mm}\\
\begin{minipage}[t]{0.47\textwidth}
    \textbf{Absoluter Fehler:}\\
    $$\|x - \tilde{x}\| \leq \|A^{-1}\| \cdot \|b - \tilde{b}\|$$
\end{minipage}
\hspace{2mm}
\begin{minipage}[t]{0.47\textwidth}
    \textbf{Relativer Fehler:}\\
    $$\frac{\|x - \tilde{x}\|}{\|x\|} \leq \text{cond}(A) \cdot \frac{\|b - \tilde{b}\|}{\|b\|}$$
\end{minipage}
\vspace{1mm}\\
Mit der Konditionszahl $\text{cond}(A) = \|A\| \cdot \|A^{-1}\|$
\end{theorem}

\begin{concept}{Konditionierung}\\
Die Konditionszahl beschreibt die numerische Stabilität eines LGS:
\begin{itemize}
    \item $\text{cond}(A) \approx 1$: gut konditioniert
    \item $\text{cond}(A) \gg 1$: schlecht konditioniert
    \item $\text{cond}(A) \to \infty$: singulär
\end{itemize}
\end{concept}

\begin{example2}{Konditionierung}
$$A = \begin{pmatrix}
1 & 1\\
1 & 1.01
\end{pmatrix}, \quad b = \begin{pmatrix}
2\\
2.01
\end{pmatrix}$$
\vspace{2mm}\\
Konditionszahl:
$\text{cond}(A) = \|A\| \cdot \|A^{-1}\| \approx 400$
\vspace{2mm}\\
\paragraph{Fehlerabschätzung}
Absoluter Fehler:
$$\|x - \tilde{x}\| \leq 400 \cdot 0.01 = 4$$
Relativer Fehler:
$$\frac{\|x - \tilde{x}\|}{\|x\|} \leq 400 \cdot \frac{0.01}{2} = 2$$
\end{example2}

\subsection{Iterative Verfahren}

\begin{definition}{Zerlegung der Systemmatrix}\\
Für iterative Verfahren wird $A$ zerlegt in: $A = L + D + R$
\begin{itemize}
    \item $L$: streng untere Dreiecksmatrix
    \item $D$: Diagonalmatrix
    \item $R$: streng obere Dreiecksmatrix
\end{itemize}
\end{definition}

\begin{concept}{Jacobi-Verfahren}\\
Gesamtschrittverfahren mit der Iteration:
$$x^{(k+1)} = -D^{-1}(L + R)x^{(k)} + D^{-1}b$$

Komponentenweise:
$$x_i^{(k+1)} = \frac{1}{a_{ii}}\left(b_i - \sum_{j=1,j\neq i}^n a_{ij}x_j^{(k)}\right)$$
\end{concept}

\begin{concept}{Gauss-Seidel-Verfahren}\\
Einzelschrittverfahren mit der Iteration:
$$x^{(k+1)} = -(D+L)^{-1}Rx^{(k)} + (D+L)^{-1}b$$

Komponentenweise:
$$x_i^{(k+1)} = \frac{1}{a_{ii}}\left(b_i - \sum_{j=1}^{i-1} a_{ij}x_j^{(k+1)} - \sum_{j=i+1}^n a_{ij}x_j^{(k)}\right)$$
\end{concept}

\begin{theorem}{Konvergenzkriterien}\\
Ein iteratives Verfahren konvergiert, wenn:
\begin{enumerate}
    \item Die Matrix $A$ diagonaldominant ist:\\
    $|a_{ii}| > \sum_{j\neq i} |a_{ij}|$ für alle $i$
    \item Der Spektralradius der Iterationsmatrix kleiner 1 ist:\\
    $\rho(B) < 1$ mit $B$ als jeweilige Iterationsmatrix
\end{enumerate}
\end{theorem}

\begin{KR}{Implementation iterativer Verfahren}\\
So implementieren Sie iterative Verfahren:
\begin{enumerate}
    \item Wählen Sie Startvektor $x^{(0)}$
    \item Wählen Sie Abbruchkriterien:
        \begin{itemize}
            \item Maximale Iterationszahl $k_{max}$
            \item Toleranz $\epsilon$ für Änderung $\|x^{(k+1)} - x^{(k)}\|$
            \item Toleranz für Residuum $\|Ax^{(k)} - b\|$
        \end{itemize}
    \item Führen Sie Iteration durch bis Kriterien erfüllt
\end{enumerate}
\end{KR}

\begin{example2}{Iterative Verfahren}{Vergleich Jacobi und Gauss-Seidel}
System:
$$\begin{pmatrix}
4 & -1 & 0\\
-1 & 4 & -1\\
0 & -1 & 4
\end{pmatrix}x = \begin{pmatrix}
1\\
5\\
0
\end{pmatrix}$$

\begin{center}
\begin{tabular}{c|cc|cc}
k & \multicolumn{2}{c|}{Jacobi} & \multicolumn{2}{c}{Gauss-Seidel}\\
\hline
0 & $(0,0,0)^T$ & & $(0,0,0)^T$ &\\
1 & $(0.25,1.25,0)^T$ & 1.25 & $(0.25,1.31,0.08)^T$ & 1.31\\
2 & $(0.31,1.31,0.31)^T$ & 0.31 & $(0.33,1.33,0.33)^T$ & 0.02\\
3 & $(0.33,1.33,0.33)^T$ & 0.02 & $(0.33,1.33,0.33)^T$ & 0.00
\end{tabular}
\end{center}
\end{example2}

