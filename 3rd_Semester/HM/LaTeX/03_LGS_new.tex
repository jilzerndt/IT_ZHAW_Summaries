\section{Numerische Lösung linearer Gleichungssysteme}

\subsection{Grundlagen}

\begin{concept}{Lineares Gleichungssystem}\\
Ein lineares Gleichungssystem der Form $Ax = b$ besteht aus:
$$A=\left[\begin{array}{ccc}
a_{11} & \cdots & a_{1n} \\
\vdots & \ddots & \vdots \\
a_{n1} & \cdots & a_{nn}
\end{array}\right] \in \mathbb{R}^{n \times n}, \quad 
x=\left(\begin{array}{c}
x_1 \\
\vdots \\
x_n
\end{array}\right) \in \mathbb{R}^n, \quad 
b=\left(\begin{array}{c}
b_1 \\
\vdots \\
b_n
\end{array}\right) \in \mathbb{R}^n$$
\end{concept}

\subsection{Der Gauss-Algorithmus}

\begin{concept}{Grundidee Gauss-Elimination}\\
Transformation des Gleichungssystems $Ax=b$ in ein äquivalentes System $\tilde{A}x=\tilde{b}$, wobei $\tilde{A}$ eine obere Dreiecksmatrix ist.

Erlaubte Operationen:
\begin{itemize}
    \item $z_j := z_j - \lambda z_i$ für $i<j$ und $\lambda \in \mathbb{R}$
    \item $z_i \rightarrow z_j$ (Vertauschen von Zeilen)
\end{itemize}
\end{concept}

\begin{KR}{Gauss-Algorithmus}\\
So lösen Sie ein lineares Gleichungssystem mit dem Gauss-Algorithmus:

\paragraph{1. Elimination (Vorwärts)}
\begin{enumerate}
    \item Für $i=1,\ldots,n-1$:
    \item \quad Für $j=i+1,\ldots,n$:
    \item \quad\quad Berechne $\lambda_{ji} = a_{ji}/a_{ii}$
    \item \quad\quad $z_j := z_j - \lambda_{ji} z_i$
\end{enumerate}

\paragraph{2. Rückwärtseinsetzen}
$$x_i = \frac{b_i - \sum_{j=i+1}^n a_{ij}x_j}{a_{ii}}, \quad i=n,n-1,\ldots,1$$
\end{KR}

\begin{example2}{Gauss-Elimination}{Lösung eines 3×3 Systems}
Gegebenes System:
$$A = \begin{pmatrix}
2 & 1 & -1\\
4 & -1 & 3\\
-2 & 2 & 1
\end{pmatrix}, \quad b = \begin{pmatrix}
3\\
1\\
4
\end{pmatrix}$$

\begin{enumerate}
    \item Elimination der ersten Spalte:
    $$\begin{pmatrix}
    2 & 1 & -1 & | & 3\\
    0 & -3 & 5 & | & -5\\
    0 & 3 & -1 & | & 7
    \end{pmatrix}$$
    
    \item Elimination der zweiten Spalte:
    $$\begin{pmatrix}
    2 & 1 & -1 & | & 3\\
    0 & -3 & 5 & | & -5\\
    0 & 0 & 4 & | & 2
    \end{pmatrix}$$
    
    \item Rückwärtseinsetzen:
    \begin{align*}
        x_3 &= \frac{2}{4} = \frac{1}{2}\\
        x_2 &= \frac{-5 - 5(\frac{1}{2})}{-3} = -2\\
        x_1 &= \frac{3 - 1(-2) - (-1)(\frac{1}{2})}{2} = 1
    \end{align*}
\end{enumerate}
\end{example2}

\subsection{Pivotisierung}

\begin{concept}{Spaltenpivotisierung}\\
Strategie zur numerischen Stabilisierung des Gauss-Algorithmus durch Auswahl des betragsmäßig größten Elements als Pivotelement.

Vor jedem Eliminationsschritt in Spalte $i$:
\begin{itemize}
    \item Suche $k$ mit $|a_{ki}| = \max\{|a_{ji}| \mid j = i,\ldots,n\}$
    \item Falls $a_{ki} \neq 0$: Vertausche Zeilen $i$ und $k$
    \item Falls $a_{ki} = 0$: Matrix ist singulär
\end{itemize}
\end{concept}

\begin{KR}{Gauss mit Pivotisierung}\\
Erweiterter Gauss-Algorithmus mit Spaltenpivotisierung:
\begin{enumerate}
    \item Für $i=1,\ldots,n-1$:
    \item \quad Finde $k \geq i$ mit $|a_{ki}| = \max\{|a_{ji}| \mid j = i,\ldots,n\}$
    \item \quad Falls $a_{ki} = 0$: Stop (Matrix singulär)
    \item \quad Vertausche Zeilen $i$ und $k$
    \item \quad Für $j=i+1,\ldots,n$:
    \item \quad\quad $z_j := z_j - \frac{a_{ji}}{a_{ii}}z_i$
\end{enumerate}
\end{KR}

\begin{example2}{Pivotisierung}{Gauss-Elimination mit Pivotisierung}
System:
$$\begin{pmatrix}
1 & 2 & -1\\
4 & 8 & -3\\
9 & 18 & -8
\end{pmatrix}x = \begin{pmatrix}
1\\
4\\
9
\end{pmatrix}$$

\begin{enumerate}
    \item Erste Spalte: Pivot $|9|$ → Tausche Zeilen 1 und 3
    \item Nach Elimination der ersten Spalte:
    $$\begin{pmatrix}
    9 & 18 & -8 & | & 9\\
    0 & 0 & 0.89 & | & 0\\
    0 & 0 & 0.89 & | & 0
    \end{pmatrix}$$
    \item System ist schlecht konditioniert (identische Zeilen)
\end{enumerate}
\end{example2}

\subsection{Matrix-Zerlegungen}

\begin{definition}{Dreieckszerlegung}\\
Eine Matrix $A \in \mathbb{R}^{n\times n}$ kann zerlegt werden in:
\vspace{1mm}\\
\begin{minipage}[t]{0.45\textwidth}
    \textbf{Untere Dreiecksmatrix L:}\\
    $l_{ij} = 0$ für $j > i$\\
    Diagonale meist normiert ($l_{ii}=1$)
\end{minipage}
\hspace{3mm}
\begin{minipage}[t]{0.45\textwidth}
    \textbf{Obere Dreiecksmatrix R:}\\
    $r_{ij} = 0$ für $i > j$\\
    Diagonalelemente $\neq 0$
\end{minipage}
\end{definition}

\begin{theorem}{LR-Zerlegung}\\
Jede reguläre Matrix $A$, für die der Gauss-Algorithmus ohne Zeilenvertauschungen durchführbar ist, lässt sich zerlegen in:
$$A = LR$$
wobei $L$ eine normierte untere und $R$ eine obere Dreiecksmatrix ist.
\end{theorem}

\begin{KR}{Berechnung der LR-Zerlegung}\\
So berechnen Sie die LR-Zerlegung:
\begin{enumerate}
    \item Führen Sie Gauss-Elimination durch
    \item $R$ ist die resultierende obere Dreiecksmatrix
    \item Die Eliminationsfaktoren $-\frac{a_{ji}}{a_{ii}}$ bilden $L$
    \item Lösen Sie dann nacheinander:
        \begin{itemize}
            \item $Ly = b$ (Vorwärtseinsetzen)
            \item $Rx = y$ (Rückwärtseinsetzen)
        \end{itemize}
\end{enumerate}
\end{KR}

\begin{example2}{LR-Zerlegung}{Berechnung einer LR-Zerlegung}
Matrix:
$$A = \begin{pmatrix}
2 & 1 & 1\\
4 & -1 & 0\\
-2 & 3 & 1
\end{pmatrix}$$

Schrittweise Elimination führt zu:
$$L = \begin{pmatrix}
1 & 0 & 0\\
2 & 1 & 0\\
-1 & 2 & 1
\end{pmatrix}, \quad
R = \begin{pmatrix}
2 & 1 & 1\\
0 & -3 & -2\\
0 & 0 & -2
\end{pmatrix}$$
\end{example2}

