
\section{Python}

\begin{remark2}{Numerische Bibliotheken} Verwendung spezialisierter Bibliotheken\\
Für kritische numerische Berechnungen:
\begin{itemize}
    \item NumPy: Optimierte Array-Operationen
    \item SciPy: Wissenschaftliches Rechnen
    \item Mpmath: Beliebige Präzision
    \item Decimal: Dezimalarithmetik
\end{itemize}
\end{remark2}

\begin{example2}{Bibliotheksverwendung} Beispiel: Präzise Berechnung mit Decimal
\begin{lstlisting}[language=Python, style=basesmol]
from decimal import Decimal, getcontext

# Set precision
getcontext().prec = 40

# Precise calculation
x = Decimal('1.0') / Decimal('7.0')
print(x)  # 0.1428571428571428571428571428571428571428
\end{lstlisting}
\end{example2}

\subsection{NumPy}

\begin{remark2}{NumPy} NumPy: Numerische Python-Bibliothek
\begin{itemize}
    \item Effiziente Implementierung von Arrays
    \item Vektorisierte Operationen
    \item Lineare Algebra, Fourier-Transformation, Zufallszahlen
\end{itemize} 
ACHTUNG: darf an der Prüfung höchstwahrscheinlich nicht verwendet werden!
aber falls doch, hier die einfachen Implementationen von allem :D
\end{remark2}

\begin{examplecode}{Eigenwerte und Eigenvektoren}
\begin{lstlisting}[language=Python, style=basesmol]
import numpy as np
A = np.array([[1, 0, 0], [2, 3, 0], [0, 1, 2]])
# Eigenwerte
eigenvalues = np.linalg.eigvals(A)
# Eigenwerte und Eigenvektoren
eigenvalues, eigenvectors = np.linalg.eig(A)
\end{lstlisting}
\end{examplecode}