
\section{Python}

\begin{remark2}{Numerische Bibliotheken} Verwendung spezialisierter Bibliotheken\\
Für kritische numerische Berechnungen:
\begin{itemize}
    \item NumPy: Optimierte Array-Operationen
    \item SciPy: Wissenschaftliches Rechnen
    \item Mpmath: Beliebige Präzision
    \item Decimal: Dezimalarithmetik
\end{itemize}
\end{remark2}

\begin{example2}{Bibliotheksverwendung} Beispiel: Präzise Berechnung mit Decimal
\begin{lstlisting}[language=Python, style=basesmol]
from decimal import Decimal, getcontext

# Set precision
getcontext().prec = 40

# Precise calculation
x = Decimal('1.0') / Decimal('7.0')
print(x)  # 0.1428571428571428571428571428571428571428
\end{lstlisting}
\end{example2}

\subsection{NumPy}

\begin{remark2}{NumPy} NumPy: Numerische Python-Bibliothek
\begin{itemize}
    \item Effiziente Implementierung von Arrays
    \item Vektorisierte Operationen
    \item Lineare Algebra, Fourier-Transformation, Zufallszahlen
\end{itemize} 
ACHTUNG: darf an der Prüfung höchstwahrscheinlich nicht verwendet werden!
aber falls doch, hier die einfachen Implementationen von allem :D
\end{remark2}

\begin{examplecode}{Gauss-Algorithmus mit Pivotisierung}
\begin{lstlisting}[language=Python, style=basesmol]
def gauss_elimination(A, b):
    n = len(b)
    for i in range(n-1):

        # Pivotisierung
        k = np.argmax(abs(A[i:, i])) + i
        if A[k, i] == 0:
            raise ValueError("Matrix ist singulaer")
        A[[i, k]] = A[[k, i]]
        b[[i, k]] = b[[k, i]]
        # Elimination
        for j in range(i+1, n):
            factor = A[j, i] / A[i, i]
            A[j, i:] -= factor * A[i, i:]
            b[j] -= factor * b[i]

    # Rueckwaertseinsetzen
    x = np.zeros(n)
    for i in range(n-1, -1, -1):
        x[i] = (b[i] - np.dot(A[i, i+1:], x[i+1:])) / A[i, i]

    return x
\end{lstlisting}
\end{examplecode}

\begin{examplecode}{LR-Zerlegung Implementation}
\begin{lstlisting}[language=Python, style=basesmol]
# Optimierte Version mit NumPy
def lr_decomposition_numpy(A):
    n = len(A)
    R = np.array(A, dtype=float)
    L = np.eye(n)
    P = np.eye(n)
    
    for k in range(n-1):
        # Pivotisierung
        pivot = np.argmax(abs(R[k:,k])) + k
        
        if abs(R[pivot,k]) < 1e-10:
            raise ValueError("Matrix ist (fast) singulaer")
            
        if pivot != k:
            # Zeilenvertauschung
            R[[k,pivot]] = R[[pivot,k]]
            L[[k,pivot], :k] = L[[pivot,k], :k]
            P[[k,pivot]] = P[[pivot,k]]
            
        # Elimination
        L[k+1:,k] = R[k+1:,k] / R[k,k]
        R[k+1:] -= np.outer(L[k+1:,k], R[k])
        
    return P, L, R
\end{lstlisting}
\end{examplecode}

\begin{examplecode}{Eigenwerte und Eigenvektoren}
\begin{lstlisting}[language=Python, style=basesmol]
import numpy as np
A = np.array([[1, 0, 0], [2, 3, 0], [0, 1, 2]])
# Eigenwerte
eigenvalues = np.linalg.eigvals(A)
# Eigenwerte und Eigenvektoren
eigenvalues, eigenvectors = np.linalg.eig(A)
\end{lstlisting}
\end{examplecode}

\begin{examplecode}{Von-Mises-Iteration}
\begin{lstlisting}[language=Python, style=basesmol]
import numpy as np
def power_iteration(A, tol=1e-10, max_iter=100):
    n = A.shape[0]
    v = np.random.rand(n)
    v = v / np.linalg.norm(v)
    for i in range(max_iter):
        w = A @ v
        v_new = w / np.linalg.norm(w)
        # Rayleigh-Quotient
        lambda_k = v_new.T @ A @ v_new
        if np.linalg.norm(v_new - v) < tol:
            return lambda_k, v_new
        v = v_new
    return lambda_k, v_new
\end{lstlisting}
\end{examplecode}

\begin{examplecode}{Von-Mises-Iteration} Berechnung des größten Eigenwerts
    %TODO: check if this is correct
\begin{lstlisting}[language=Python, style=basesmol]
def power_iteration(A, tol=1e-10, max_iter=100):
    n = len(A)
    v = [random.random() for _ in range(n)]
    v = [x / np.linalg.norm(v) for x in v]
    for i in range(max_iter):
        w = [sum(A[i][j] * v[j] for j in range(n)) for i in range(n)]
        v_new = [x / np.linalg.norm(w) for x in w]
        # Rayleigh-Quotient
        lambda_k = sum(v_new[i] * A[i][j] * v_new[j] for i in range(n) for j in range(n))
        if np.linalg.norm([v_new[i] - v[i] for i in range(n)]) < tol:
            return lambda_k, v_new
        v = v_new
    return lambda_k, v_new
\end{lstlisting}
\end{examplecode}

\begin{examplecode}{Fehlerabschätzung}
\begin{lstlisting}[language=Python, style=basesmol]
def error_estimate(A, b, x, b_tilde):
    # Absoluter Fehler
    abs_error = np.linalg.norm(x - np.linalg.solve(A, b_tilde))
    # Relativer Fehler
    rel_error = abs_error / np.linalg.norm(x)
    return abs_error, rel_error
\end{lstlisting}
\end{examplecode}

\begin{examplecode}{Jacobi- und Gauss-Seidel-Verfahren}
    %TODO: check if this is correct and/or relevant - either correct or replace with better example
\begin{lstlisting}[language=Python, style=basesmol]
def jacobi_iteration(A, b, x):
    D = np.diag(np.diag(A))
    L = np.tril(A, -1)
    R = np.triu(A, 1)
    x_new = np.linalg.solve(D, b - (L + R) @ x)
    return x_new

def gauss_seidel_iteration(A, b, x):
    D = np.diag(np.diag(A))
    L = np.tril(A, -1)
    R = np.triu(A, 1)
    x_new = np.linalg.solve(D + L, b - R @ x)
    return x_new
\end{lstlisting}
\end{examplecode}

\begin{examplecode}{Komplexe Zahlen in Python}
%TODO: check if this is correct and/or relevant - either correct or replace with better example
\begin{lstlisting}[language=Python, style=basesmol]
import numpy as np
z1 = 3 - 11j
z2 = 2 + 5j
# Addition
z_add = z1 + z2
# Subtraktion
z_sub = z1 - z2
# Multiplikation
z_mul = z1 * z2
# Division
z_div = z1 / z2
# Betrag
r = np.abs(z1)
# Winkel
phi = np.angle(z1)
# Exponentialform
z_exp = r * np.exp(1j * phi)
# Potenz
z_pow = z1 ** 2
# Wurzel
z_sqrt = np.sqrt(z1)

# Darstellungsformen
z_trig = r * (np.cos(phi) + 1j * np.sin(phi))
z_norm = z_trig.real + 1j * z_trig.imag
\end{lstlisting}
\end{examplecode}

\begin{examplecode}{EW und EV über Charakteristisches Polynom}
    %TODO: check if this is correct and/or relevant - either correct or replace with better example
\begin{lstlisting}[language=Python, style=basesmol]
A = np.array([[1, 0, 0], [2, 3, 0], [0, 1, 2]])
# Charakteristisches Polynom
p = np.poly(A)
# Eigenwerte
eigenvalues = np.roots(p)
# Eigenvektoren
eigenvectors = []
for l in eigenvalues:
    eigenvectors.append(np.linalg.solve(A - l * np.eye(A.shape[0]), np.zeros(A.shape[0])))
\end{lstlisting}
\end{examplecode}

\begin{examplecode}{QR-Verfahren}
    %TODO: check if this is correct and/or relevant - either correct or replace with better example
    %needs to be implemented without using specialized libraries (no numpy, etc)
\begin{lstlisting}[language=Python, style=basesmol]
def qr_algorithm(A, tol=1e-10, max_iter=100):
    n = A.shape[0]
    Q_prod = np.eye(n)
    A_k = A.copy()
    
    for k in range(max_iter):
        # QR decomposition
        Q, R = np.linalg.qr(A_k)
        # New iteration
        A_k = R @ Q
        # Update transformation matrix
        Q_prod = Q_prod @ Q
        
        # Check convergence
        if np.abs(np.tril(A_k, -1)).max() < tol:
            break
            
    return np.diag(A_k), Q_prod
\end{lstlisting}
\end{examplecode}