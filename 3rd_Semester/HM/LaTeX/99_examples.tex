\section{Examples}

\subsection{Rechnerarithmetik}

\begin{example2}{Werteberechnung ausführlich} 
Gegeben sei die Maschinenzahl zur Basis $B=2$:
$$x = \underbrace{0.1101}_{\text{n=4}} \cdot \underbrace{2^{101}_2}_{\text{l=3}}$$

\textbf{1. Normalisierung prüfen:}
\begin{itemize}
    \item $m_1 = 1 \neq 0$ $\rightarrow$ normalisiert
\end{itemize}

\textbf{2. Exponent berechnen:}
\begin{align*}
\hat{e} &= 1 \cdot 2^2 + 0 \cdot 2^1 + 1 \cdot 2^0 \\
&= 4 + 0 + 1 = 5
\end{align*}

\textbf{3. Wert berechnen:}
\begin{align*}
\hat{\omega} &= 1 \cdot 2^{5-1} + 1 \cdot 2^{5-2} + 0 \cdot 2^{5-3} + 1 \cdot 2^{5-4} \\
&= 1 \cdot 2^4 + 1 \cdot 2^3 + 0 \cdot 2^2 + 1 \cdot 2^1 \\
&= 16 + 8 + 0 + 2 \\
&= 26
\end{align*}

Also ist $x = 26$
\end{example2}

\begin{example2}{Weitere Beispiele}
\begin{enumerate}
    \item Basis 10: $0.3141 \cdot 10^2$
    \begin{itemize}
        \item Normalisiert, da $m_1 = 3 \neq 0$
        \item $\hat{e} = 2$
        \item $\hat{\omega} = 3 \cdot 10^1 + 1 \cdot 10^0 + 4 \cdot 10^{-1} + 1 \cdot 10^{-2} = 31.41$
    \end{itemize}
    
    \item Basis 16 (hex): $0.A5F \cdot 16^3$
    \begin{itemize}
        \item Normalisiert, da $m_1 = A = 10 \neq 0$
        \item $\hat{e} = 3$
        \item $\hat{\omega} = 10 \cdot 16^2 + 5 \cdot 16^1 + 15 \cdot 16^0 = 2655$
    \end{itemize}
\end{enumerate}
\end{example2}

\begin{example2}{Werteberechnung} Berechnung einer Zahl zur Basis B=2:
\begin{minipage}{0.45\textwidth}
    $$\underbrace{0.1011}_{\text{n=4}} \cdot \underbrace{2^{3}}_{\text{l=1}}$$
\end{minipage}
\begin{minipage}[t]{0.5\textwidth}
    1. Exponent: $\hat{e} = 3$ \\ 
    2. Wert: $\hat{\omega} = 1\cdot2^2 + 0\cdot2^1 + 1\cdot2^0 + 1\cdot2^{-1}$ \\
    $= 4 + 0 + 1 + 0.5 = 5.5$
\end{minipage}
\end{example2}

\subsection{Numerische Lösung von Nullstellenproblemen}

\begin{example2}{Fixpunktiteration} Nullstellen von $p(x)=x^3-x+0.3$\\
    %TODO: check if this is correct and/or relevant - either correct or replace with better example
Fixpunktgleichung: $x_{n+1} = F(x_n) = x_n^3 + 0.3$
\begin{enumerate}
    \item $F'(x) = 3x^2$ steigt monoton
    \item Für $I=[0,0.5]$: $F(0)=0.3 > 0$, $F(0.5)=0.425 < 0.5$
    \item $\alpha = \max_{x \in [0,0.5]} |3x^2| = 0.75 < 1$
    \item Konvergenz für Startwerte in $[0,0.5]$ gesichert
\end{enumerate}
\end{example2}

\begin{examplecode}{Fixpunktiteration}
    \begin{lstlisting}[language=Python, style=basesmol]
def fixed_point_iteration(f, x0, tol=1e-6, max_iter=100):
    for n in range(max_iter):
        x1 = f(x0)
        if abs(x1 - x0) < tol:
            return x1
        x0 = x1
    raise ValueError("No convergence")
    \end{lstlisting}
\end{examplecode}



\begin{example2}{Newton-Verfahren} Berechnung von $\sqrt[3]{2}$
Nullstellenproblem: $f(x)=x^3-2$
\vspace{1mm}\\
\begin{minipage}[t]{0.65\textwidth}
    \vspace{-3mm}
    Ableitung: $f'(x)=3x^2$, Startwert $x_0=1$
    \begin{enumerate}
        \item $x_1 = 1 - \frac{1^3-2}{3 \cdot 1^2} = 1.333333$
        \item $x_2 = 1.333333 - \frac{1.333333^3-2}{3 \cdot 1.333333^2} = 1.259921$
        \item $x_3 = 1.259921 - \frac{1.259921^3-2}{3 \cdot 1.259921^2} = 1.259921$
    \end{enumerate}
\end{minipage}
\begin{minipage}[t]{0.3\textwidth}
    Quadratische Konvergenz sichtbar durch schnelle Annäherung an $\sqrt[3]{2} \approx 1.259921$
\end{minipage}
\end{example2}

