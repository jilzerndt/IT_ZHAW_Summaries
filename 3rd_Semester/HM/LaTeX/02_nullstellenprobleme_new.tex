\section{Numerische Lösung von Nullstellenproblemen}

\begin{definition}{Fixpunktgleichung}\\
Eine Gleichung der Form $F(x)=x$ heisst Fixpunktgleichung. Die Lösungen $\bar{x}$, für die $F(\bar{x})=\bar{x}$ erfüllt ist, heissen Fixpunkte.
\end{definition}

\subsection{Fixpunktiteration}

\begin{concept}{Grundprinzip der Fixpunktiteration}\\
Gegeben sei $F:[a,b] \rightarrow \mathbb{R}$ mit $x_0 \in [a,b]$. Die rekursive Folge
$$x_{n+1} \equiv F(x_n), \quad n=0,1,2,\ldots$$
heisst Fixpunktiteration von $F$ zum Startwert $x_0$.
\end{concept}

\begin{theorem}{Konvergenzverhalten}\\
Sei $F:[a,b] \rightarrow \mathbb{R}$ mit stetiger Ableitung $F'$ und $\bar{x} \in [a,b]$ ein Fixpunkt von $F$. Dann gilt für die Fixpunktiteration $x_{n+1}=F(x_n)$:
\vspace{1mm}\\
\begin{minipage}[t]{0.45\textwidth}
    \textbf{Anziehender Fixpunkt:}\\
    $|F'(\bar{x})| < 1$\\
    $x_n$ konvergiert gegen $\bar{x}$,\\
    falls $x_0$ nahe genug bei $\bar{x}$
\end{minipage}
\hspace{3mm}
\begin{minipage}[t]{0.45\textwidth}
    \textbf{Abstossender Fixpunkt:}\\
    $|F'(\bar{x})| > 1$\\
    $x_n$ konvergiert für keinen\\
    Startwert $x_0 \neq \bar{x}$
\end{minipage}
\end{theorem}

\begin{lemma}{Banachscher Fixpunktsatz}\\
Sei $F:[a,b] \rightarrow [a,b]$ und es existiere eine Konstante $\alpha$ mit:
\begin{itemize}
    \item $0 < \alpha < 1$ (Lipschitz-Konstante)
    \item $|F(x)-F(y)| \leq \alpha|x-y|$ für alle $x,y \in [a,b]$
\end{itemize}

Dann gilt:
\begin{itemize}
    \item $F$ hat genau einen Fixpunkt $\bar{x}$ in $[a,b]$
    \item Die Fixpunktiteration konvergiert gegen $\bar{x}$ für alle $x_0 \in [a,b]$
    \item Fehlerabschätzungen:
    \begin{itemize}
        \item a-priori: $|x_n-\bar{x}| \leq \frac{\alpha^n}{1-\alpha} \cdot |x_1-x_0|$
        \item a-posteriori: $|x_n-\bar{x}| \leq \frac{\alpha}{1-\alpha} \cdot |x_n-x_{n-1}|$
    \end{itemize}
\end{itemize}
\end{lemma}

\begin{KR}{Konvergenznachweis für Fixpunktiteration}\\
So überprüfen Sie, ob eine Fixpunktiteration konvergiert:
\begin{enumerate}
    \item Prüfen Sie, ob $F:[a,b] \rightarrow [a,b]$ gilt:\\
           $F(a) > a$ und $F(b) < b$
    \item Bestimmen Sie $\alpha = \max_{x \in [a,b]} |F'(x)|$
    \item Prüfen Sie, ob $\alpha < 1$
    \item Berechnen Sie die nötigen Iterationen für Toleranz tol:\\
           $n \geq \frac{\ln(\frac{tol \cdot (1-\alpha)}{|x_1-x_0|})}{\ln \alpha}$
\end{enumerate}
\end{KR}

\begin{example2}{Fixpunktiteration} Nullstellen von $p(x)=x^3-x+0.3$\\
Fixpunktgleichung: $x_{n+1} = F(x_n) = x_n^3 + 0.3$
\begin{enumerate}
    \item $F'(x) = 3x^2$ steigt monoton
    \item Für $I=[0,0.5]$: $F(0)=0.3 > 0$, $F(0.5)=0.425 < 0.5$
    \item $\alpha = \max_{x \in [0,0.5]} |3x^2| = 0.75 < 1$
    \item Konvergenz für Startwerte in $[0,0.5]$ gesichert
\end{enumerate}
\end{example2}

\subsection{Newton-Verfahren}

\begin{concept}{Grundprinzip Newton-Verfahren}\\
Approximation der Nullstelle durch sukzessive Tangentenberechnung:
$$x_{n+1} = x_n - \frac{f(x_n)}{f'(x_n)}$$

Konvergiert, wenn für alle $x$ im relevanten Intervall gilt:
$$\left|\frac{f(x) \cdot f''(x)}{[f'(x)]^2}\right| < 1$$
\end{concept}

\begin{KR}{Newton-Verfahren anwenden}\\
So finden Sie eine Nullstelle mit dem Newton-Verfahren:
\begin{enumerate}
    \item Funktion $f(x)$ und Ableitung $f'(x)$ aufstellen
    \item Geeigneten Startwert $x_0$ nahe der Nullstelle wählen
    \item Iterieren bis zur gewünschten Genauigkeit:
    $$x_{n+1} = x_n - \frac{f(x_n)}{f'(x_n)}$$
    \item Konvergenz prüfen durch Vergleich aufeinanderfolgender Werte
\end{enumerate}
\end{KR}

\begin{theorem}{Vereinfachtes Newton-Verfahren}\\
Alternative Variante mit konstanter Ableitung:
$$x_{n+1} = x_n - \frac{f(x_n)}{f'(x_0)}$$
Konvergiert langsamer, aber benötigt weniger Rechenaufwand.
\end{theorem}

\begin{concept}{Sekantenverfahren}\\
Alternative zum Newton-Verfahren ohne Ableitungsberechnung. Verwendet zwei Punkte $(x_{n-1}, f(x_{n-1}))$ und $(x_n, f(x_n))$:
$$x_{n+1} = x_n - \frac{x_n-x_{n-1}}{f(x_n)-f(x_{n-1})} \cdot f(x_n)$$
Benötigt zwei Startwerte $x_0$ und $x_1$.
\end{concept}

\subsection{Konvergenzverhalten}

\begin{definition}{Konvergenzordnung}\\
Sei $(x_n)$ eine gegen $\bar{x}$ konvergierende Folge. Die Konvergenzordnung $q \geq 1$ ist definiert durch:
$$|x_{n+1}-\bar{x}| \leq c \cdot |x_n-\bar{x}|^q$$
wobei $c > 0$ eine Konstante ist. Für $q = 1$ muss zusätzlich $c < 1$ gelten.
\end{definition}

\begin{theorem}{Konvergenzordnungen der Verfahren}\\
Die verschiedenen Verfahren zeigen folgende Konvergenzgeschwindigkeiten:
\vspace{1mm}\\
\begin{minipage}[t]{0.3\textwidth}
    \textbf{Newton-Verfahren:}\\
    Quadratische Konvergenz\\
    $q = 2$
\end{minipage}
\hspace{2mm}
\begin{minipage}[t]{0.3\textwidth}
    \textbf{Vereinfachtes Newton:}\\
    Lineare Konvergenz\\
    $q = 1$
\end{minipage}
\hspace{2mm}
\begin{minipage}[t]{0.3\textwidth}
    \textbf{Sekantenverfahren:}\\
    Superlineare Konvergenz\\
    $q = \frac{1+\sqrt{5}}{2} \approx 1.618$
\end{minipage}
\end{theorem}

\begin{example2}{Konvergenzgeschwindigkeit}{Vergleich der Verfahren}
Startwert $x_0 = 1$, Funktion $f(x) = x^2 - 2$, Ziel: $\sqrt{2}$
\begin{center}
\begin{tabular}{l|c|c|c}
n & Newton & Vereinfacht & Sekanten \\\hline
1 & 1.5000000 & 1.5000000 & 1.5000000\\
2 & 1.4166667 & 1.4500000 & 1.4545455\\
3 & 1.4142157 & 1.4250000 & 1.4142857\\
4 & 1.4142136 & 1.4125000 & 1.4142136
\end{tabular}
\end{center}
\end{example2}

\subsection{Fehlerabschätzung}

\begin{lemma}{Nullstellensatz von Bolzano}\\
Sei $f:[a,b] \rightarrow \mathbb{R}$ stetig. Falls $f(a) \cdot f(b) < 0$, dann existiert mindestens eine Nullstelle $\xi \in (a,b)$.
\end{lemma}

\begin{KR}{Fehlerabschätzung für Nullstellen}\\
So schätzen Sie den Fehler einer Näherungslösung ab:
\begin{enumerate}
    \item Sei $x_n$ der aktuelle Näherungswert
    \item Wähle Toleranz $\epsilon > 0$
    \item Prüfe Vorzeichenwechsel: $f(x_n-\epsilon) \cdot f(x_n+\epsilon) < 0$
    \item Falls ja: Nullstelle liegt in $(x_n-\epsilon, x_n+\epsilon)$
    \item Damit gilt: $|x_n-\xi| < \epsilon$
\end{enumerate}
\end{KR}

\begin{example2}{Praktische Fehlerabschätzung}{Fehlerbestimmung bei $f(x)=x^2-2$}
\begin{enumerate}
    \item Näherungswert: $x_3 = 1.4142157$
    \item Mit $\epsilon = 10^{-5}$:
    \item $f(x_3-\epsilon) = 1.4142057^2 - 2 < 0$
    \item $f(x_3+\epsilon) = 1.4142257^2 - 2 > 0$
    \item Also: $|x_3-\sqrt{2}| < 10^{-5}$
\end{enumerate}
\end{example2}

\begin{remark2}{Abbruchkriterien}{Praktische Implementierung}\\
In der Praxis verwendet man meist mehrere Abbruchkriterien:
\begin{itemize}
    \item Absolute Änderung: $|x_n - x_{n-1}| < \epsilon_1$
    \item Funktionswert: $|f(x_n)| < \epsilon_2$
    \item Maximale Iterationszahl: $n < n_{max}$
    \item Kombination dieser Kriterien
\end{itemize}
\end{remark2}