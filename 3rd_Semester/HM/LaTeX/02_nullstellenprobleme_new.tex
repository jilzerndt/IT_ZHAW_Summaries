\section{Numerische Lösung von Nullstellenproblemen}

\begin{remark}
    NSP: Nullstellenproblem, NS: Nullstelle
\end{remark}

\begin{definition}{Fixpunktgleichung}
ist eine Gleichung der Form: $F(x)=x$\\
Die Lösungen $\bar{x}$, für die $F(\bar{x})=\bar{x}$ erfüllt ist, heissen Fixpunkte.
\end{definition}

\subsubsection{Fixpunktiteration}

\begin{concept}{Grundprinzip der Fixpunktiteration}
sei $F:[a,b] \rightarrow \mathbb{R}$ mit $x_0 \in [a,b]$ 
\vspace{-6mm}\\
$$\text{Die rekursive Folge }x_{n+1} \equiv F(x_n), \quad n=0,1,2,\ldots$$
\vspace{-4mm}\\
heisst Fixpunktiteration von $F$ zum Startwert $x_0$.
\end{concept}

\begin{theorem}{Konvergenzverhalten}\\
Sei $F:[a,b] \rightarrow \mathbb{R}$ mit stetiger Ableitung $F'$ und $\bar{x} \in [a,b]$ ein Fixpunkt von $F$. Dann gilt für die Fixpunktiteration $x_{n+1}=F(x_n)$:
\vspace{1mm}\\
\begin{minipage}[t]{0.45\textwidth}
    \textbf{Anziehender Fixpunkt:}
    \vspace{-3mm}\\
    $$|F'(\bar{x})| < 1$$
    \vspace{-4mm}\\
    $x_n$ konvergiert gegen $\bar{x}$,\\
    falls $x_0$ nahe genug bei $\bar{x}$
\end{minipage}
\hspace{3mm}
\begin{minipage}[t]{0.45\textwidth}
    \textbf{Abstossender Fixpunkt:}
    \vspace{-3mm}\\
    $$|F'(\bar{x})| > 1$$
    \vspace{-4mm}\\
    $x_n$ konvergiert für keinen\\
    Startwert $x_0 \neq \bar{x}$
\end{minipage}
\end{theorem}

\begin{lemma}{Banachscher Fixpunktsatz}
$F:[a,b] \rightarrow [a,b]$ und $\exists$ Konstante $\alpha$:
\begin{itemize}
    \item $0 < \alpha < 1$ (Lipschitz-Konstante)
    \item $|F(x)-F(y)| \leq \alpha|x-y|$ für alle $x,y \in [a,b]$
\end{itemize}
\vspace{2mm}


\begin{minipage}[t]{0.35\textwidth}
    Dann gilt:
\begin{itemize}
    \item $F$ hat genau einen Fixpunkt $\bar{x}$ in $[a,b]$
    \item Die Fixpunktiteration konvergiert gegen $\bar{x}$ für alle $x_0 \in [a,b]$
\end{itemize}
\end{minipage}
\hspace{2mm}
\begin{minipage}[t]{0.55\textwidth}
    \textbf{Fehlerabschätzungen}:
    \vspace{-2mm}\\
    $$\textbf{a-priori: } |x_n-\bar{x}| \leq \frac{\alpha^n}{1-\alpha} \cdot |x_1-x_0|$$
    $$\textbf{a-posteriori: } |x_n-\bar{x}| \leq \frac{\alpha}{1-\alpha} \cdot |x_n-x_{n-1}|$$
\end{minipage}
\end{lemma}

\begin{KR}{Konvergenznachweis für Fixpunktiteration}\\
So überprüfen Sie, ob eine Fixpunktiteration konvergiert:
\begin{enumerate}
    \item Prüfen Sie, ob $F:[a,b] \rightarrow [a,b]$ gilt:\\
           $F(a) > a$ und $F(b) < b$
    \item Bestimmen Sie $\alpha = \max_{x \in [a,b]} |F'(x)|$
    \item Prüfen Sie, ob $\alpha < 1$
\end{enumerate}

\begin{minipage}[t]{0.45\textwidth}
    4. Berechnen Sie die nötigen Iterationen für Toleranz tol:
\end{minipage}
\begin{minipage}[t]{0.45\textwidth}
    \vspace{-3mm}
$$n \geq \frac{\ln(\frac{tol \cdot (1-\alpha)}{|x_1-x_0|})}{\ln \alpha}$$
\end{minipage}
\end{KR}

\begin{example2}{Fixpunktiteration} Nullstellen von $p(x)=x^3-x+0.3$\\
Fixpunktgleichung: $x_{n+1} = F(x_n) = x_n^3 + 0.3$
\begin{enumerate}
    \item $F'(x) = 3x^2$ steigt monoton
    \item Für $I=[0,0.5]$: $F(0)=0.3 > 0$, $F(0.5)=0.425 < 0.5$
    \item $\alpha = \max_{x \in [0,0.5]} |3x^2| = 0.75 < 1$
    \item Konvergenz für Startwerte in $[0,0.5]$ gesichert
\end{enumerate}
\end{example2}

\begin{examplecode}{Fixpunktiteration}
    \begin{lstlisting}[language=Python, style=basesmol]
def fixed_point_iteration(f, x0, tol=1e-6, max_iter=100):
    for n in range(max_iter):
        x1 = f(x0)
        if abs(x1 - x0) < tol:
            return x1
        x0 = x1
    raise ValueError("No convergence")
    \end{lstlisting}
\end{examplecode}

\subsection{Newton-Verfahren}

\begin{concept}{Grundprinzip Newton-Verfahren}
    \vspace{-2mm}\\
\begin{minipage}[t]{0.6\textwidth}
    Approximation der NS durch \\ sukzessive Tangentenberechnung:
\end{minipage}
\begin{minipage}{0.3\textwidth}
    \vspace{-3mm}
    $$x_{n+1} = x_n - \frac{f(x_n)}{f'(x_n)}$$
\end{minipage}
\vspace{-2mm}\\
\begin{minipage}[t]{0.6\textwidth}
    Konvergiert, wenn für alle $x$ im \\ relevanten Intervall gilt:
\end{minipage}
\begin{minipage}{0.3\textwidth}
    \vspace{-3mm}
    $$\left|\frac{f(x) \cdot f''(x)}{[f'(x)]^2}\right| < 1$$
\end{minipage}
\end{concept}

\begin{KR}{Newton-Verfahren anwenden}
\begin{enumerate}
    \item Funktion $f(x)$ und Ableitung $f'(x)$ aufstellen
    \item Geeigneten Startwert $x_0$ nahe der Nullstelle wählen
    \item Iterieren bis zur gewünschten Genauigkeit:
    $x_{n+1} = x_n - \frac{f(x_n)}{f'(x_n)}$
    \item Konvergenz prüfen durch Vergleich aufeinanderfolgender Werte
\end{enumerate}
\end{KR}

\begin{example2}{Newton-Verfahren} Nullstellen von $f(x)=x^2-2$\\
Ableitung: $f'(x) = 2x$, Startwert $x_0 = 1$
\vspace{1mm}\\
\begin{minipage}[t]{0.65\textwidth}
    \vspace{-3mm}
    \begin{enumerate}
        \item $x_1 = 1 - \frac{1^2-2}{2 \cdot 1} = 1.5$
        \item $x_2 = 1.5 - \frac{1.5^2-2}{2 \cdot 1.5} = 1.4167$
        \item $x_3 = 1.4167 - \frac{1.4167^2-2}{2 \cdot 1.4167} = 1.4142$
    \end{enumerate}
\end{minipage}
\begin{minipage}[t]{0.3\textwidth}
    $\rightarrow$ Konvergenz \\ gegen $\sqrt{2}$ nach \\ wenigen Schritten
\end{minipage}
\end{example2}

\begin{examplecode}{Newton-Verfahren}
    \begin{lstlisting}[language=Python, style=basesmol]
def newton(f, df, x0, tol=1e-6, max_iter=100):
    for n in range(max_iter):
        x1 = x0 - f(x0) / df(x0)
        if abs(x1 - x0) < tol:
            return x1
        x0 = x1
    raise ValueError("No convergence")
    \end{lstlisting}
\end{examplecode}

\begin{theorem}{Vereinfachtes Newton-Verfahren}\\
    \begin{minipage}{0.5\textwidth}
        Alternative Variante mit \\ konstanter Ableitung:
    \end{minipage}
    \begin{minipage}{0.25\textwidth}
        \vspace{-5mm}
        $$x_{n+1} = x_n - \frac{f(x_n)}{f'(x_0)}$$
    \end{minipage}
    \vspace{1mm}\\
    Konvergiert langsamer, aber benötigt weniger Rechenaufwand.
\end{theorem}

\begin{concept}{Sekantenverfahren}\\
    Alternative zum Newton-Verfahren ohne Ableitungsberechnung. Verwendet zwei Punkte $(x_{n-1}, f(x_{n-1}))$ und $(x_n, f(x_n))$:
    \vspace{-2mm}\\
    $$x_{n+1} = x_n - \frac{x_n-x_{n-1}}{f(x_n)-f(x_{n-1})} \cdot f(x_n)$$
    \vspace{-3mm}\\
    Benötigt zwei Startwerte $x_0$ und $x_1$.
\end{concept}


\begin{example2}{Sekantenverfahren} Nullstellen von $f(x)=x^2-2$\\
Startwerte $x_0 = 1$ und $x_1 = 2$
\vspace{1mm}\\
\begin{minipage}[t]{0.65\textwidth}
    \vspace{-3mm}
    \begin{enumerate}
        \item $x_2 = 1 - \frac{1-2}{1^2-2} \cdot 1 = 1.5$
        \item $x_3 = 1.5 - \frac{1.5-1}{1.5^2-2} \cdot 1.5 = 1.4545$
        \item $x_4 = 1.4545 - \frac{1.4545-1.5}{1.4545^2-2} \cdot 1.4545 = 1.4143$
    \end{enumerate}
\end{minipage}
\hspace{2mm}
\begin{minipage}[t]{0.28\textwidth}
    $\rightarrow$ Konvergenz\\ gegen $\sqrt{2}$ nach \\wenigen Schritten
\end{minipage}
\end{example2}

\begin{examplecode}{Sekantenverfahren}
    \begin{lstlisting}[language=Python, style=basesmol]
def secant(f, x0, x1, tol=1e-6, max_iter=100):
    for n in range(max_iter):
        x2 = x1 - (x1 - x0) / (f(x1) - f(x0)) * f(x1)
        if abs(x2 - x1) < tol:
            return x2
        x0, x1 = x1, x2
    raise ValueError("No convergence")
    \end{lstlisting}
\end{examplecode}

\subsubsection{Konvergenzverhalten}

\begin{definition}{Konvergenzordnung}
    Sei $(x_n)$ eine gegen $\bar{x}$ konvergierende Folge. \\
    Die Konvergenzordnung $q \geq 1$ ist definiert durch:
    \vspace{-2mm}\\
    $$|x_{n+1}-\bar{x}| \leq c \cdot |x_n-\bar{x}|^q$$
    wo $c > 0$ eine Konstante. Für $q = 1$ muss zusätzl. $c < 1$ gelten.
\end{definition}

\begin{theorem}{Konvergenzordnungen der Verfahren} Konvergenzgeschwindigkeiten
    \vspace{-2mm}\\
    \textbf{Newton-Verfahren:} Quadratische Konvergenz: $q = 2$
    \vspace{1mm}\\
    \textbf{Vereinfachtes Newton:} Lineare Konvergenz: $q = 1$
    \vspace{1mm}\\
    \textbf{Sekantenverfahren:} Superlineare Konvergenz: $q = \frac{1+\sqrt{5}}{2} \approx 1.618$
\end{theorem}

\begin{example2}{Konvergenzgeschwindigkeit} Vergleich der Verfahren:
    \vspace{1mm}\\
    Startwert $x_0 = 1$, Funktion $f(x) = x^2 - 2$, Ziel: $\sqrt{2}$
    \begin{center}
    \begin{tabular}{l|c|c|c}
    n & Newton & Vereinfacht & Sekanten \\\hline
    1 & 1.5000000 & 1.5000000 & 1.5000000\\
    2 & 1.4166667 & 1.4500000 & 1.4545455\\
    3 & 1.4142157 & 1.4250000 & 1.4142857\\
    4 & 1.4142136 & 1.4125000 & 1.4142136
    \end{tabular}
    \end{center}
\end{example2}
\subsection{Fehlerabschätzung}

\begin{lemma}{Nullstellensatz von Bolzano}
Sei $f:[a,b] \rightarrow \mathbb{R}$ stetig. Falls 
\vspace{-1mm}\\
$$f(a) \cdot f(b) < 0$$ 
\vspace{-3mm}\\
dann existiert mindestens eine Nullstelle $\xi \in (a,b)$.
\end{lemma}

\begin{KR}{Fehlerabschätzung für Nullstellen}\\
So schätzen Sie den Fehler einer Näherungslösung ab:
\begin{enumerate}
    \item Sei $x_n$ der aktuelle Näherungswert
    \item Wähle Toleranz $\epsilon > 0$
    \item Prüfe Vorzeichenwechsel: $f(x_n-\epsilon) \cdot f(x_n+\epsilon) < 0$
    \item Falls ja: Nullstelle liegt in $(x_n-\epsilon, x_n+\epsilon)$
    \item Damit gilt: $|x_n-\xi| < \epsilon$
\end{enumerate}
\end{KR}

\begin{example2}{Praktische Fehlerabschätzung} Fehlerbestimmung bei $f(x)=x^2-2$
    \vspace{-1mm}\\
    \begin{minipage}[t]{0.6\textwidth}
        \vspace{-3mm}
        \begin{enumerate}
            \item Näherungswert: $x_3 = 1.4142157$
            \item Mit $\epsilon = 10^{-5}$:
            \item $f(x_3-\epsilon) = 1.4142057^2 - 2 < 0$
            \item $f(x_3+\epsilon) = 1.4142257^2 - 2 > 0$
        \end{enumerate}
    \end{minipage}
    \begin{minipage}[t]{0.35\textwidth}
        \textbf{Also}: $|x_3-\sqrt{2}| < 10^{-5}$
        \vspace{-1mm}\\
        $\rightarrow$ Nullstelle liegt in $(1.4142057, 1.4142257)$
    \end{minipage}
\end{example2}

\begin{remark2}{Abbruchkriterien}{Praktische Implementierung}\\
In der Praxis verwendet man meist mehrere Abbruchkriterien:
\begin{itemize}
    \item Absolute Änderung: $|x_n - x_{n-1}| < \epsilon_1$
    \item Funktionswert: $|f(x_n)| < \epsilon_2$
    \item Maximale Iterationszahl: $n < n_{max}$
    \item Kombination dieser Kriterien
\end{itemize}
\end{remark2}

\begin{examplecode}{Fehlerabschätzung}
    \begin{lstlisting}[language=Python, style=basesmol]
def error_estimate(f, x, eps=1e-5):
    if f(x - eps) * f(x + eps) < 0:
        return eps
    return None
    \end{lstlisting}
\end{examplecode}
