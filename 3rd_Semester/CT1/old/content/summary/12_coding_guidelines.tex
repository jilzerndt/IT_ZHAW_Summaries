\section{Coding Guidelines}

\begin{concept}{Guidelines Purpose}\\
Coding guidelines are essential for:
\begin{itemize}
  \item Reducing the number of bugs through:
    \begin{itemize}
      \item Improved robustness
      \item Better correctness
      \item Enhanced maintainability
    \end{itemize}
  \item Facilitating code reading within a team
  \item Improving portability for code reuse
\end{itemize}
\end{concept}

\begin{KR}{Code Structure and Organization}\\
Key guidelines for code organization:

1. Code Appearance:
\begin{itemize}
  \item Indentation: 4 spaces, no tabs
  \item Maximum of 80 characters per line
  \item No more than one statement per line
  \item Use parentheses to aid clarity (don't rely on operator precedence)
\end{itemize}

2. Braces Usage:
\begin{itemize}
  \item Non-function statement blocks (if, else, switch, for, while, do):
    \begin{itemize}
      \item Opening brace last on line
      \item Closing brace first on line
    \end{itemize}
  \item Functions:
    \begin{itemize}
      \item Opening brace beginning of next line
      \item Closing brace first on line
    \end{itemize}
  \item Always use braces for single statements and empty statements
\end{itemize}
\end{KR}

\begin{definition}{Function Requirements}\\
Guidelines for function implementation:
\begin{itemize}
  \item Keep functions short and focused (max ~50 lines)
  \item Do just one thing
  \item No more than 5-10 local variables
  \item No more than 3 parameters
  \item Function prototypes must include parameter names with data types
  \item Maximum of 3 levels of indentation
  \item Single exit point at bottom of function
  \item Use const for call-by-reference parameters that shouldn't be modified
\end{itemize}
\end{definition}

\begin{concept}{Return Value Conventions}\\
Guidelines for function return values:
\begin{itemize}
  \item Return values must be checked by the caller
  \item For functions named as actions/commands:
    \begin{itemize}
      \item Return error-code integer (0 success, -Exxx failure)
      \item Use Posix error codes where possible
      \item Document custom error codes in header files
    \end{itemize}
  \item For predicate functions:
    \begin{itemize}
      \item Return boolean success value
      \item Example: pci\_dev\_present() returns 1 for success, 0 for failure
    \end{itemize}
  \item Computation functions return actual results
    \begin{itemize}
      \item Indicate failure through out-of-range values
      \item Use NULL or ERR\_PTR for pointer returns
    \end{itemize}
\end{itemize}
\end{concept}

\begin{concept}{Naming Conventions}\\
Rules for naming different code elements:
\begin{itemize}
  \item Macro names (\#define): All uppercase letters
  \item Function/variable names: No uppercase letters
  \item Use descriptive names for functions, globals, and important locals
  \item Use underscores to separate words (e.g., count\_active\_users())
  \item Short names acceptable for auxiliary locals (e.g., i for loop counters)
  \item Don't encode types in names - let compiler do type checking
\end{itemize}
\end{concept}

\begin{concept}{Comments and Documentation}\\
Guidelines for code documentation:
\begin{itemize}
  \item All comments must be in English
  \item C99 comments (//) are allowed
  \item Explain WHAT code does, not HOW
  \item Don't repeat what the statement says in comments
  \item Comments shall never be nested
  \item Document all assumptions in comments
  \item Interface documentation in header files only
  \item Comment function prototypes in header, not in source file
\end{itemize}
\end{concept}

\begin{concept}{Type Usage}\\
Rules for data type usage:
\begin{itemize}
  \item Use fixed-width C99 types from stdint.h
    \begin{itemize}
      \item uint8\_t/int32\_t instead of unsigned char/int
    \end{itemize}
  \item Restrict char type to string operations
  \item No bit-fields within signed integer types
  \item Don't use bitwise operators on signed integers
  \item Don't mix signed/unsigned in comparisons
  \item Use 'U' suffix for unsigned decimal constants
  \item One data declaration per line for clarity
\end{itemize}
\end{concept}

\begin{concept}{Header File Organization}\\
Requirements for header files:
\begin{itemize}
  \item One header file per module
  \item Include preprocessor guards against multiple inclusion
  \item Only add immediately needed \#includes
  \item No variable definitions/declarations
  \item Keep interface minimal - only expose necessary functions
  \item Document all public functions in header
  \item Private functions must be declared static
  \item Function prototypes in module interface
\end{itemize}
\end{concept}

\begin{KR}{Module Design}\\
Example of proper module organization:

1. Header File (module.h):
\begin{lstlisting}[language=C, style=basesmol]
#ifndef _MODULE_H
#define _MODULE_H

typedef enum {
    STATE_A = 0x00,
    STATE_B = 0x01
} module_state_t;

void module_init(void);
void module_set_state(module_state_t state);
module_state_t module_get_state(void);

#endif /* _MODULE_H */
\end{lstlisting}

2. Implementation File (module.c):
\begin{lstlisting}[language=C, style=basesmol]
#include "module.h"

static module_state_t current_state;
static void helper_function(void);

void module_init(void) {
    current_state = STATE_A;
}

void module_set_state(module_state_t state) {
    current_state = state;
    helper_function();
}

module_state_t module_get_state(void) {
    return current_state;
}

static void helper_function(void) {
    // Implementation
}
\end{lstlisting}
\end{KR}

\begin{remark}
Important considerations:
\begin{itemize}
  \item Coding guidelines are subjective
  \item Different organizations have different standards
  \item Follow existing project conventions
  \item Guidelines help maintain consistency
  \item Use automated checks and peer reviews
  \item Document any deviations from standards
\end{itemize}
\end{remark}