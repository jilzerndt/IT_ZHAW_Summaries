\subsection{Examples}



\begin{example2}{Logical Instructions}\\
Bit manipulation examples:

1. Inverting register contents (one's complement):
\begin{lstlisting}[language=armasm, style=basesmol]
MVNS    R1, R1          ; Invert all bits in R1
\end{lstlisting}

2. Selective bit manipulation:
\begin{lstlisting}[language=armasm, style=basesmol]
; Set bits 3..0 to 1, bits 7..4 to 0
; Invert bits 17-16, keep others unchanged
MOVS    R0, #0xF     ; Pattern for bits 3..0
ORRS    R1, R0       ; Set bits 3..0 to 1
MOVS    R0, #0xF0    ; Pattern for bits 7..4
BICS    R1, R0       ; Clear bits 7..4
MOVS    R0, #0x30       
LSLS    R0, #12      ; 0x30000 shift 4 nibbles = 12bit
EORS    R1, R0       ; Invert bits 17-16
\end{lstlisting}
\end{example2}

\raggedcolumns



\begin{example2}{Multiplication Using Shifts}\\
Multiply R7 by 43 using two different methods:

1. Using multiplication instruction:
\begin{lstlisting}[language=armasm, style=basesmol]
MOVS    R0, #43         ; Load constant
MULS    R7, R0, R7      ; Multiply
\end{lstlisting}

2. Using shifts and additions:
\begin{lstlisting}[language=armasm, style=basesmol]
MOVS    R7, R0          ; Copy original value
LSLS    R0, R0, #1      ; *2
ADDS    R7, R7, R0
LSLS    R0, R0, #2      ; *8=4*2
ADDS    R7, R7, R0
LSLS    R0, R0, #2      ; *32=8*4
ADDS    R7, R7, R0      ; Total = 1 + 2 + 8 + 32 = 43
\end{lstlisting}
\end{example2}

\begin{example2}{Reverse Engineering}\\
Convert assembly to C code:

Given C variables:
\begin{lstlisting}[language=C, style=basesmol]
uint32_t ux, uy, uz;    // Variables in memory
\end{lstlisting}

Example 1:
\begin{lstlisting}[language=armasm, style=basesmol]
; Assembly code
LDR     R0, =ux         ; Load address
LDR     R1, [R0]        ; Load value
LSLS    R1, R1, #1      ; Shift left by 1
LDR     R2, =uy         ; Load uy address
LDR     R3, [R2]        ; Load uy value
ADDS    R3, R1          ; Add shifted ux
LSLS    R3, R3, #3      ; Shift left by 3
STR     R3, [R2]        ; Store back in uy
\end{lstlisting}

Equivalent C code:
\begin{lstlisting}[language=C, style=basesmol]
uy = ((ux << 1) + uy) << 3;
// or: 
uy = 8 * (2 * ux + uy);
\end{lstlisting}
\end{example2}

\begin{example2}{Reverse Engineering 2}\\
Convert assembly to C code:

Die Speicherstellen im Assembler werden den ursprünglichen Variablen im C-Programm wie folgt zugeordnet:
\begin{lstlisting}[language=armasm, style=basesmol]
ux: DCD ? ; uint32_t ux
uy: DCD ? ; uint32_t uy
uz: DCD ? ; uint32_t uz
\end{lstlisting}

Example 2:
\begin{lstlisting}[language=armasm, style=basesmol]
; Assembly code
LDR     R0, =ux         ; Load address
LDR     R1, [R0]        ; Load value
LDR     R2, =uy         ; Load address
LDR     R3, [R2]        ; Load value
LDR     R4, =uz         ; Load address
LDR     R5, [R4]        ; Load value
LSRS    R1, R1, #3      ; Shift right by 3
LSLS    R3, R3, #4      ; Shift left by 4
ORRS    R1, R3          ; OR to combine
MVNS    R1, R1          ; Invert bits
ANDS    R1, R5          ; AND with uz
STR     R1, [R4]        ; Store back in uz
\end{lstlisting}

Equivalent C code:
\begin{lstlisting}[language=C, style=basesmol]
uz = ~( (ux >> 3) | (uy << 4)) & uz;
// or:
uz = ~( (ux / 8) | (uy * 16)) & uz;
\end{lstlisting}
\end{example2}

\begin{example2}{Type Casting}
Examples of explicit casting in C:
\vspace{1mm}\\
1. Unsigned to signed casting:
\begin{lstlisting}[language=C, style=basesmol]
uint8_t ux = 100;           // 0x64 or 0110 0100
int8_t sx = (int8_t)ux;     // Same bit pattern
                            // Interpreted as 100
\end{lstlisting}

Als welche Dezimalzahl wird der Inhalt der Variable sx nach dem Cast interpretiert?
\vspace{1mm}\\
100d --> ux hat Speicherinhalt 0x64 oder Binär 0110'0100
\vspace{1mm}\\
sx hat denselben Speicherinhalt, wird aber als signed
interpretiert. Da das höchstwertigste Bit nicht gesetzt ist, hat die
Interpretation aber in diesem Fall keinen Einfluss.
\vspace{1mm}\\
sx wird ebenfalls als 100d interpretiert
\vspace{2mm}\\
2. Signed to unsigned casting:
\begin{lstlisting}[language=C, style=basesmol]
int8_t sx = -10;           // 0xF6 or 1111 0110
uint8_t ux = (uint8_t)sx;  // Same bits but unsigned
                           // 246 (128+64+32+16+4+2)
\end{lstlisting}

Als welche Dezimalzahl wird der Inhalt der Variable ux nach dem Cast interpretiert?
\vspace{1mm}\\
-10d --> sx hat Speicherinhalt 0xF6 oder Binär 1111'0110
(-128+64+32+16+4 + 2)
\vspace{1mm}\\
ux hat denselben Speicherinhalt, wird aber als unsigned
interpretiert. So ergibt sich 128 + 64 + 32 + 16 + 4 + 2 = 246d
\end{example2}







