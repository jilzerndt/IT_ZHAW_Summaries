\section{Browser APIs und DOM}

\subsection{Vordefinierte Objekte}

\begin{definition}{Browser Objekte}
    Im Browser stehen spezielle globale Objekte zur Verfügung:
    \begin{itemize}
        \item \texttt{window}: Browserfenster und globaler Scope
        \item \texttt{document}: Das aktuelle HTML-Dokument
        \item \texttt{navigator}: Browser-Informationen
        \item \texttt{location}: URL-Informationen
        \item \texttt{history}: Browser-Verlauf
    \end{itemize}
\end{definition}

\begin{concept}{Document Object Model (DOM)}
    Das DOM ist eine Baumstruktur, die das HTML-Dokument repräsentiert:
    \begin{itemize}
        \item Jeder HTML-Tag wird zu einem Element-Knoten
        \item Text innerhalb von Tags wird zu Text-Knoten
        \item Attribute werden zu Attribut-Knoten
        \item Kommentare werden zu Kommentar-Knoten
    \end{itemize}
\end{concept}

\begin{KR}{DOM Navigation}
Zugriff auf DOM-Elemente:
\begin{lstlisting}[language=JavaScript, style=basesmol]
// Element ueber ID finden
const elem = document.getElementById('myId');

// Elemente ueber CSS-Selektor finden
const elem1 = document.querySelector('.myClass');
const elems = document.querySelectorAll('div.myClass');

// Navigation im DOM-Baum
elem.parentNode          // Elternknoten
elem.childNodes         // Alle Kindknoten
elem.children           // Nur Element-Kindknoten
elem.firstChild         // Erster Kindknoten
elem.lastChild          // Letzter Kindknoten
elem.nextSibling        // Naechster Geschwisterknoten
elem.previousSibling    // Vorheriger Geschwisterknoten
\end{lstlisting}
\end{KR}

\begin{KR}{DOM Manipulation}
Elemente erstellen und manipulieren:
\begin{lstlisting}[language=JavaScript, style=basesmol]
// Neues Element erstellen
const newDiv = document.createElement('div');
const newText = document.createTextNode('Hello');
newDiv.appendChild(newText);

// Element einfuegen
parentElem.appendChild(newDiv);
parentElem.insertBefore(newDiv, referenceElem);

// Element entfernen
elem.remove();
parentElem.removeChild(elem);

// Attribute manipulieren
elem.setAttribute('class', 'myClass');
elem.getAttribute('class');
elem.removeAttribute('class');

// HTML/Text Inhalt
elem.innerHTML = '<span>Text</span>';
elem.textContent = 'Nur Text';
\end{lstlisting}
\end{KR}

\subsection{Events}

\begin{concept}{Event Handling}
    Events sind Ereignisse, die im Browser auftreten:
    \begin{itemize}
        \item Benutzerinteraktionen (Klicks, Tastatureingaben)
        \item DOM-Änderungen
        \item Ressourcen laden
        \item Timer
    \end{itemize}
\end{concept}

\begin{KR}{Event Listener}
Event Listener registrieren und entfernen:
\begin{lstlisting}[language=JavaScript, style=basesmol]
// Event Listener hinzufuegen
element.addEventListener('click', function(event) {
    console.log('Clicked!', event);
});

// Mit Arrow Function
element.addEventListener('click', (event) => {
    console.log('Clicked!', event);
});

// Event Listener entfernen
const handler = (event) => {
    console.log('Clicked!', event);
};
element.addEventListener('click', handler);
element.removeEventListener('click', handler);
\end{lstlisting}
\end{KR}

\begin{formula}{Wichtige Event-Typen}
    \begin{itemize}
        \item Maus: \texttt{click}, \texttt{dblclick}, \texttt{mousedown}, \texttt{mouseup}, \texttt{mousemove}
        \item Tastatur: \texttt{keydown}, \texttt{keyup}, \texttt{keypress}
        \item Formular: \texttt{submit}, \texttt{change}, \texttt{input}, \texttt{focus}, \texttt{blur}
        \item Dokument: \texttt{DOMContentLoaded}, \texttt{load}, \texttt{unload}
        \item Fenster: \texttt{resize}, \texttt{scroll}
    \end{itemize}
\end{formula}

\begin{code}{Event Bubbling und Capturing}
\begin{lstlisting}[language=JavaScript, style=basesmol]
// Bubbling (default)
element.addEventListener('click', handler);

// Capturing
element.addEventListener('click', handler, true);

// Event-Ausbreitung stoppen
element.addEventListener('click', (event) => {
    event.stopPropagation();
});

// Default-Verhalten verhindern
element.addEventListener('click', (event) => {
    event.preventDefault();
});
\end{lstlisting}
\end{code}

\subsection{Browser Storage}

\begin{definition}{Storage APIs}
    Browser bieten verschiedene Möglichkeiten zur Datenspeicherung:
    \begin{itemize}
        \item \texttt{localStorage}: Permanente Speicherung
        \item \texttt{sessionStorage}: Temporäre Speicherung (nur für aktuelle Session)
        \item \texttt{cookies}: Kleine Datenpakete, die auch zum Server gesendet werden
        \item \texttt{indexedDB}: NoSQL-Datenbank im Browser
    \end{itemize}
\end{definition}

\begin{KR}{LocalStorage Verwendung}
\begin{lstlisting}[language=JavaScript, style=basesmol]
// Daten speichern
localStorage.setItem('key', 'value');
localStorage.setItem('user', JSON.stringify({
    name: 'John',
    age: 30
}));

// Daten abrufen
const value = localStorage.getItem('key');
const user = JSON.parse(localStorage.getItem('user'));

// Daten loeschen
localStorage.removeItem('key');
localStorage.clear();  // Alles loeschen
\end{lstlisting}
\end{KR}

\subsection{Canvas und SVG}

\begin{concept}{Grafik im Browser}
    Zwei Haupttechnologien für Grafiken:
    \begin{itemize}
        \item Canvas: Pixel-basierte Grafik
        \item SVG: Vektor-basierte Grafik
    \end{itemize}
\end{concept}

\begin{KR}{Canvas Grundlagen}
\begin{lstlisting}[language=JavaScript, style=basesmol]
const canvas = document.querySelector('canvas');
const ctx = canvas.getContext('2d');

// Rechteck zeichnen
ctx.fillStyle = 'red';
ctx.fillRect(10, 10, 100, 50);

// Pfad zeichnen
ctx.beginPath();
ctx.moveTo(10, 10);
ctx.lineTo(50, 50);
ctx.stroke();

// Text zeichnen
ctx.font = '20px Arial';
ctx.fillText('Hello', 10, 50);
\end{lstlisting}
\end{KR}

\begin{KR}{SVG Manipulation}
\begin{lstlisting}[language=JavaScript, style=basesmol]
// SVG-Element erstellen
const svg = document.createElementNS(
    "http://www.w3.org/2000/svg", 
    "svg"
);
svg.setAttribute("width", "100");
svg.setAttribute("height", "100");

// Kreis hinzufuegen
const circle = document.createElementNS(
    "http://www.w3.org/2000/svg", 
    "circle"
);
circle.setAttribute("cx", "50");
circle.setAttribute("cy", "50");
circle.setAttribute("r", "40");
circle.setAttribute("fill", "red");

svg.appendChild(circle);
\end{lstlisting}
\end{KR}
