\section{Client-Server Interaktion}

\subsection{Formulare}

\begin{definition}{HTML-Formulare}
    Formulare ermöglichen Benutzereingaben und Datenübertragung:
    \begin{itemize}
        \item \texttt{<form>} Element mit \texttt{action} und \texttt{method}
        \item \texttt{method="GET"}: Daten in URL (sichtbar)
        \item \texttt{method="POST"}: Daten im Request-Body (unsichtbar)
        \item Verschiedene Input-Typen: text, password, checkbox, radio, etc.
    \end{itemize}
\end{definition}

\begin{KR}{Formular Handling}
\begin{lstlisting}[language=HTML, style=basesmol]
<!-- HTML Form -->
<form action="/submit" method="POST">
    <label for="username">Username:</label>
    <input type="text" id="username" name="username">
    
    <label for="password">Password:</label>
    <input type="password" id="password" name="password">
    
    <button type="submit">Login</button>
</form>

<!-- JavaScript Handler -->
form.addEventListener('submit', (event) => {
    event.preventDefault(); // Verhindert Standard-Submit
    
    const formData = new FormData(form);
    // Zugriff auf Formular-Daten
    const username = formData.get('username');
    const password = formData.get('password');
});
\end{lstlisting}
\end{KR}

\begin{formula}{Formular Events}
    Wichtige Events bei Formularen:
    \begin{itemize}
        \item \texttt{submit}: Formular wird abgeschickt
        \item \texttt{reset}: Formular wird zurückgesetzt
        \item \texttt{change}: Wert eines Elements wurde geändert
        \item \texttt{input}: Wert wird gerade geändert
        \item \texttt{focus}: Element erhält Fokus
        \item \texttt{blur}: Element verliert Fokus
    \end{itemize}
\end{formula}

\subsection{AJAX und Fetch API}

\begin{concept}{AJAX}
    Asynchronous JavaScript And XML:
    \begin{itemize}
        \item Asynchrone Kommunikation mit dem Server
        \item Kein vollständiges Neuladen der Seite nötig
        \item Moderne Alternative: Fetch API
        \item Datenformate: JSON, XML, Plain Text
    \end{itemize}
\end{concept}

\begin{KR}{Fetch API Grundlagen}
\begin{lstlisting}[language=JavaScript, style=basesmol]
// GET Request
fetch('https://api.example.com/data')
    .then(response => response.json())
    .then(data => console.log(data))
    .catch(error => console.error('Error:', error));

// POST Request
fetch('https://api.example.com/data', {
    method: 'POST',
    headers: {
        'Content-Type': 'application/json',
    },
    body: JSON.stringify({
        key: 'value'
    })
})
    .then(response => response.json())
    .then(data => console.log(data));

// Mit async/await
async function fetchData() {
    try {
        const response = await fetch('https://api.example.com/data');
        const data = await response.json();
        console.log(data);
    } catch (error) {
        console.error('Error:', error);
    }
}
\end{lstlisting}
\end{KR}

\subsection{Cookies und Sessions}

\begin{definition}{Cookies}
    HTTP-Cookies sind kleine Datenpakete:
    \begin{itemize}
        \item Werden vom Server gesetzt
        \item Im Browser gespeichert
        \item Bei jedem Request mitgesendet
        \item Haben Name, Wert, Ablaufdatum und Domain
    \end{itemize}
\end{definition}

\begin{KR}{Cookie Handling}
\begin{lstlisting}[language=JavaScript, style=basesmol]
// Cookie setzen
document.cookie = "username=John Doe; expires=Thu, 18 Dec 2024 12:00:00 UTC; path=/";

// Cookie lesen
const cookies = document.cookie.split(';').reduce((acc, cookie) => {
    const [name, value] = cookie.trim().split('=');
    acc[name] = value;
    return acc;
}, {});

// Cookie loeschen
document.cookie = "username=; expires=Thu, 01 Jan 1970 00:00:00 UTC; path=/;";
\end{lstlisting}
\end{KR}

\begin{concept}{Sessions}
    Server-seitige Speicherung von Benutzerdaten:
    \begin{itemize}
        \item Session-ID wird in Cookie gespeichert
        \item Daten bleiben auf dem Server
        \item Sicherer als Cookies für sensible Daten
        \item Temporär (bis Browser geschlossen wird)
    \end{itemize}
\end{concept}

\subsection{REST APIs}

\begin{definition}{REST Prinzipien}
    Representational State Transfer:
    \begin{itemize}
        \item Zustandslos (Stateless)
        \item Ressourcen-orientiert
        \item Einheitliche Schnittstelle
        \item Standard HTTP-Methoden
    \end{itemize}
\end{definition}

\begin{theorem}{HTTP-Methoden}
    \begin{center}
    \begin{tabular}{|l|l|}
    \hline
    Methode & Verwendung \\
    \hline
    GET & Daten abrufen \\
    \hline
    POST & Neue Daten erstellen \\
    \hline
    PUT & Daten aktualisieren (komplett) \\
    \hline
    PATCH & Daten aktualisieren (teilweise) \\
    \hline
    DELETE & Daten löschen \\
    \hline
    \end{tabular}
    \end{center}
\end{theorem}

\begin{KR}{REST API Implementierung mit Express}
\begin{lstlisting}[language=JavaScript, style=basesmol]
const express = require('express');
const app = express();
app.use(express.json());

// GET - Alle Benutzer abrufen
app.get('/api/users', (req, res) => {
    res.json(users);
});

// GET - Einzelnen Benutzer abrufen
app.get('/api/users/:id', (req, res) => {
    const user = users.find(u => u.id === parseInt(req.params.id));
    if (!user) return res.status(404).send('User not found');
    res.json(user);
});

// POST - Neuen Benutzer erstellen
app.post('/api/users', (req, res) => {
    const user = {
        id: users.length + 1,
        name: req.body.name
    };
    users.push(user);
    res.status(201).json(user);
});

// PUT - Benutzer aktualisieren
app.put('/api/users/:id', (req, res) => {
    const user = users.find(u => u.id === parseInt(req.params.id));
    if (!user) return res.status(404).send('User not found');
    
    user.name = req.body.name;
    res.json(user);
});

// DELETE - Benutzer loeschen
app.delete('/api/users/:id', (req, res) => {
    const user = users.find(u => u.id === parseInt(req.params.id));
    if (!user) return res.status(404).send('User not found');
    
    const index = users.indexOf(user);
    users.splice(index, 1);
    res.json(user);
});
\end{lstlisting}
\end{KR}

\begin{corollary}{HTTP Status Codes}
    \begin{center}
    \begin{tabular}{|l|l|}
    \hline
    Code & Bedeutung \\
    \hline
    200 & OK - Erfolgreich \\
    \hline
    201 & Created - Ressource erstellt \\
    \hline
    400 & Bad Request - Fehlerhafte Anfrage \\
    \hline
    401 & Unauthorized - Nicht authentifiziert \\
    \hline
    403 & Forbidden - Keine Berechtigung \\
    \hline
    404 & Not Found - Ressource nicht gefunden \\
    \hline
    500 & Internal Server Error - Serverfehler \\
    \hline
    \end{tabular}
    \end{center}
\end{corollary}
