\subsection{Event Handling}

\begin{KR}{Event Handler}
Grundlegende Event Handling Schritte:

1. Event Listener registrieren:
\begin{lstlisting}[language=JavaScript, style=basesmol]
element.addEventListener("event", handler)
element.removeEventListener("event", handler)
\end{lstlisting}

2. Event Handler mit Event-Objekt:
\begin{lstlisting}[language=JavaScript, style=basesmol]
element.addEventListener("click", (event) => {
  console.log(event.type)    // Art des Events
  console.log(event.target)  // Ausloesendes Element
  event.preventDefault()     // Default verhindern
  event.stopPropagation()   // Bubbling stoppen
})
\end{lstlisting}

Wichtige Event-Typen:
\begin{itemize}
  \item Mouse: click, mousedown, mouseup, mousemove
  \item Keyboard: keydown, keyup, keypress
  \item Form: submit, change, input
  \item Document: DOMContentLoaded, load
  \item Window: resize, scroll
\end{itemize}
\end{KR}

\subsection{Formulare}

\begin{KR}{Formular Handling}
1. Formular erstellen:
\begin{lstlisting}[language=HTML, style=basesmol]
<form action="/api/submit" method="post">
  <input type="text" name="username">
  <input type="password" name="password">
  <button type="submit">Login</button>
</form>
\end{lstlisting}

2. Formular Events abfangen:
\begin{lstlisting}[language=JavaScript, style=basesmol]
form.addEventListener("submit", (e) => {
  e.preventDefault() 
  // Eigene Verarbeitung
})
\end{lstlisting}

3. Formulardaten verarbeiten:
\begin{lstlisting}[language=JavaScript, style=basesmol]
const formData = new FormData(form)
fetch("/api/submit", {
  method: "POST",
  body: formData
})
\end{lstlisting}
\end{KR}

\subsection{Fetch API}

\begin{KR}{HTTP Requests mit Fetch}
1. GET Request:
\begin{lstlisting}[language=JavaScript, style=basesmol]
fetch("/api/data")
  .then(response => response.json())
  .then(data => console.log(data))
  .catch(error => console.error(error))
\end{lstlisting}

2. POST Request:
\begin{lstlisting}[language=JavaScript, style=basesmol]
fetch("/api/create", {
  method: "POST",
  headers: {
    "Content-Type": "application/json"
  },
  body: JSON.stringify(data)
})
\end{lstlisting}

3. Mit async/await:
\begin{lstlisting}[language=JavaScript, style=basesmol]
async function getData() {
  try {
    const response = await fetch("/api/data")
    const data = await response.json()
    return data
  } catch (error) {
    console.error(error)
  }
}
\end{lstlisting}
\end{KR}