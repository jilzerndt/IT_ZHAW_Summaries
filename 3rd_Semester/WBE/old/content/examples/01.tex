\subsection{JavaScript Exam Preparation}

\begin{concept}{Key Concepts for Exam}
    Essential topics to focus on:
    \begin{itemize}
        \item Data types and type coercion
        \item Variable scoping (var, let, const)
        \item Functions (declaration vs expression)
        \item Objects and prototypes
        \item Asynchronous programming
        \item Promise chains and async/await
        \item Event loop understanding
        \item Array methods and manipulation
    \end{itemize}
\end{concept}

\begin{formula}{Common Pitfalls}
    Watch out for these tricky areas:
    \begin{itemize}
        \item typeof null returns 'object'
        \item Hoisting behavior differences between var and let
        \item this keyword behavior in different contexts
        \item Promise resolution order
        \item Event loop execution order
        \item Closure scope understanding
        \item Array method return values
    \end{itemize}
\end{formula}

\begin{KR}{Sample Multiple Choice Questions}

\paragraph{Data Types and Coercion}
What is the output of the following code?
\begin{lstlisting}[language=JavaScript, style=basesmol]
console.log(typeof typeof 42);
\end{lstlisting}
\begin{itemize}
    \item a) "number"
    \item b) "string" -correct
    \item c) "undefined"
    \item d) "object"
\end{itemize}
Explanation: typeof 42 returns "number", and typeof "number" returns "string"

\paragraph{Variable Scope}
What is logged to the console?
\begin{lstlisting}[language=JavaScript, style=basesmol]
for(var i = 0; i < 3; i++) {
    setTimeout(() => console.log(i), 1);
}
\end{lstlisting}
\begin{itemize}
    \item a) 0, 1, 2
    \item b) 3, 3, 3 - correct
    \item c) undefined, undefined, undefined
    \item d) Error
\end{itemize}
Explanation: var creates one shared binding, loop finishes before timeouts execute

\paragraph{Promises}
What is the output sequence?
\begin{lstlisting}[language=JavaScript, style=basesmol]
console.log('1');
Promise.resolve().then(() => console.log('2'));
setTimeout(() => console.log('3'), 0);
console.log('4');
\end{lstlisting}
\begin{itemize}
    \item a) 1, 2, 3, 4
    \item b) 1, 4, 2, 3 - correct
    \item c) 1, 4, 3, 2
    \item d) 1, 2, 4, 3
\end{itemize}
Explanation: Microtasks (Promises) execute before macrotasks (setTimeout)
\end{KR}

\begin{KR}{MC exercises}
\paragraph{Objects and Prototypes}
What is logged?
\begin{lstlisting}[language=JavaScript, style=basesmol]
const proto = { value: 42 };
const obj = Object.create(proto);
obj.value = undefined;
console.log('value' in obj);
\end{lstlisting}
\begin{itemize}
    \item a) false
    \item b) true - correct
    \item c) undefined
    \item d) 42
\end{itemize}
Explanation: 'in' operator checks own and inherited properties

\paragraph{Array Methods}
What is returned?
\begin{lstlisting}[language=JavaScript, style=basesmol]
[1, 2, 3].map(x => x * 2).filter(x => x > 4);
\end{lstlisting}
\begin{itemize}
    \item a) [6]  - correct
    \item b) [4, 6]
    \item c) [2, 4, 6]
    \item d) []
\end{itemize}
Explanation: map creates [2,4,6], filter keeps only values > 4
\end{KR}

\begin{concept}{Common Exam Patterns}
Look for questions about:
\begin{itemize}
    \item Type coercion in comparisons (== vs ===)
    \item Scope and closure behavior
    \item Promise and async execution order
    \item Object property inheritance
    \item Array method chaining results
    \item Event loop and task queue order
    \item this context in different situations
\end{itemize}
\end{concept}

\begin{examplecode}{Critical Code Patterns}
\begin{lstlisting}[language=JavaScript, style=basesmol]
// Closure Example
function counter() {
    let count = 0;
    return () => ++count;
}

// Promise Chain
Promise.resolve(1)
    .then(x => x + 1)
    .then(x => Promise.resolve(x + 1))
    .then(console.log); // 3

// Event Loop Order
console.log(1);
setTimeout(() => console.log(2), 0);
Promise.resolve().then(() => console.log(3));
console.log(4);
// Outputs: 1, 4, 3, 2

// Prototype Chain
function Animal(name) {
    this.name = name;
}
Animal.prototype.speak = function() {
    return `${this.name} makes a sound.`;
};
\end{lstlisting}
\end{examplecode}

\begin{theorem}{Key Exam Strategies}
    \begin{itemize}
        \item Read code examples carefully for edge cases
        \item Consider asynchronous execution order
        \item Check for scope and closure effects
        \item Remember type coercion rules
        \item Understand prototype chain inheritance
        \item Know common array method behaviors
        \item Consider the event loop for timing questions
    \end{itemize}
\end{theorem}

\subsection{more examples}

\begin{KR}{Typische JavaScript-Aufgaben}
\begin{lstlisting}[language=JavaScript, style=basesmol]
// 1. Type Checking und Konvertierung
typeof(someVariable)                // Typ pruefen
Number("123")                       // String zu Number
String(123)                         // Number zu String
Boolean(expression)                 // zu Boolean
Array.isArray(arr)                 // Array pruefen

// 2. Array Manipulation
array.push(item)                   // am Ende hinzufuegen
array.pop()                        // letztes Element entfernen
array.unshift(item)                // am Anfang hinzufuegen
array.shift()                      // erstes Element entfernen
array.splice(start, deleteCount)   // Elemente entfernen
array.slice(start, end)           // Teil-Array erstellen

// 3. Object Handling
Object.keys(obj)                   // Array von Keys
Object.values(obj)                 // Array von Values
Object.entries(obj)                // Array von [key, value] Paaren
Object.assign({}, obj1, obj2)      // Objekte zusammenfuehren
{...obj1, ...obj2}                // Spread Operator

// 4. String Manipulation
str.split(delimiter)              // String zu Array
arr.join(delimiter)               // Array zu String
str.substring(start, end)         // Teilstring
str.replace(search, replace)      // Ersetzen
str.trim()                        // Whitespace entfernen
\end{lstlisting}
\end{KR}

\begin{examplecode}{Typische Prüfungsaufgaben}
\begin{lstlisting}[language=JavaScript, style=basesmol]
// Was ist die Ausgabe?
console.log(typeof null)           // "object"
console.log(typeof undefined)      // "undefined"
console.log(typeof [])             // "object"
console.log(typeof (() => {}))     // "function"

// Vergleiche
console.log(null == undefined)     // true
console.log(null === undefined)    // false
console.log([1,2] == [1,2])       // false
console.log("5" == 5)             // true
console.log("5" === 5)            // false

// Array-Methoden
let arr = [1, 2, 3];
arr.push(4);                      // [1,2,3,4]
arr.pop();                        // [1,2,3]
arr.unshift(0);                   // [0,1,2,3]
arr.shift();                      // [1,2,3]
\end{lstlisting}
\end{examplecode}

\begin{KR}{Funktions-Patterns}
\begin{lstlisting}[language=JavaScript, style=basesmol]
// 1. Default Parameter
function greet(name = 'User') {
    return `Hello ${name}!`;
}

// 2. Rest Parameter
function sum(...numbers) {
    return numbers.reduce((a, b) => a + b, 0);
}

// 3. Closure
function counter() {
    let count = 0;
    return {
        increment: () => ++count,
        getCount: () => count
    };
}

// 4. IIFE (Immediately Invoked Function Expression)
const counter = (() => {
    let count = 0;
    return {
        increment: () => ++count,
        getCount: () => count
    };
})();

// 5. Callback Pattern
function fetchData(callback) {
    setTimeout(() => {
        callback('Data');
    }, 1000);
}
\end{lstlisting}
\end{KR}

\begin{code}{ASYNC/AWAIT}
\begin{lstlisting}[language=JavaScript, style=basesmol]
/* Bekanntes Beispiel */
const readHosts =() => {
    readFilePromise('/etc/hosts')
        .then(console.log)
        . catch(() => {
            console.log("Error reading file")
        })
}
/* Mit async/await */
const readHosts = async () => {
    try {
        console.log(await readFilePromise('/etc/hosts'))
    }
    catch (err) {
        console.log("Error reading file")
    }
}
\end{lstlisting}
Beispiel 2:
\begin{lstlisting}[language=JavaScript, style=basesmol]
function resolveAfter2Seconds (x) {
    return new Promise(resolve => {
        setTimeout(() => {
            resolve(x)
        }, 2000)
    })
}
async function add1(x) {
    var a = resolveAfter2Seconds(20)
    var b = resolveAfter2Seconds(30)
    return x + await a + await b
}
add1(10).then(console.log)
\end{lstlisting}
\end{code}