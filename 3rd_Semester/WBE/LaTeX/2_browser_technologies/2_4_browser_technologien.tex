\subsection{Web Graphics}

\begin{KR}{SVG Grafiken}
1. SVG erstellen:
\begin{lstlisting}[language=HTML, style=basesmol]
<svg width="200" height="200">
  <circle cx="100" cy="100" r="50" fill="red"/>
  <rect x="20" y="20" width="50" height="50" fill="blue"/>
</svg>
\end{lstlisting}

2. SVG mit JavaScript manipulieren:
\begin{lstlisting}[language=JavaScript, style=basesmol]
const circle = document.querySelector('circle')
circle.setAttribute('fill', 'green')
circle.setAttribute('r', '60')

// Event Listener fuer SVG-Elemente
circle.addEventListener('click', () => {
  circle.setAttribute('fill', 'yellow')
})
\end{lstlisting}

Vorteile SVG:
\begin{itemize}
  \item Skalierbar ohne Qualitätsverlust
  \item Teil des DOM (manipulierbar)
  \item Gute Browser-Unterstützung
  \item Event-Handler möglich
\end{itemize}
\end{KR}

\begin{KR}{Canvas API}
1. Canvas erstellen:
\begin{lstlisting}[language=HTML, style=basesmol]
<canvas id="myCanvas" width="200" height="200"></canvas>
\end{lstlisting}

2. Context holen und zeichnen:
\begin{lstlisting}[language=JavaScript, style=basesmol]
const canvas = document.getElementById('myCanvas')
const ctx = canvas.getContext('2d')

// Rechteck zeichnen
ctx.fillStyle = 'red'
ctx.fillRect(10, 10, 100, 100)

// Pfad zeichnen
ctx.beginPath()
ctx.moveTo(10, 10)
ctx.lineTo(100, 100)
ctx.stroke()

// Text zeichnen
ctx.font = '20px Arial'
ctx.fillText('Hello', 50, 50)

// Bild zeichnen
const img = new Image()
img.onload = () => ctx.drawImage(img, 0, 0)
img.src = 'image.png'
\end{lstlisting}

3. Transformationen:
\begin{lstlisting}[language=JavaScript, style=basesmol]
// Speichern des aktuellen Zustands
ctx.save()

// Transformationen
ctx.translate(100, 100)  // Verschieben
ctx.rotate(Math.PI / 4)  // Rotieren
ctx.scale(2, 2)         // Skalieren

// Zeichnen...

// Wiederherstellen des gespeicherten Zustands
ctx.restore()
\end{lstlisting}

Wichtige Canvas-Methoden:
\begin{itemize}
  \item clearRect(): Bereich löschen
  \item save()/restore(): Kontext speichern/wiederherstellen
  \item translate()/rotate()/scale(): Transformationen
  \item drawImage(): Bilder zeichnen
  \item getImageData()/putImageData(): Pixel-Manipulation
\end{itemize}
\end{KR}