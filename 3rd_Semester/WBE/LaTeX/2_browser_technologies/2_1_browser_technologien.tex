\section{Browser-Technologien}

\subsection{Vordefinierte Objekte}
\begin{concept}{Browser-Objekte}
Browser-Objekte existieren auf der Browser-Plattform und referenzieren verschiedene Aspekte:
\begin{itemize}
  \item \textbf{document:} Repräsentiert die aktuelle Webseite, Zugriff auf DOM
  \item \textbf{window:} Repräsentiert das Browserfenster, globale Funktionen/Methoden  
  \item \textbf{navigator:} Browser- und Geräteinformationen
  \item \textbf{location:} URL-Manipulation und Navigation
\end{itemize}
\end{concept}

\begin{KR}{document-Objekt}
Wichtige Methoden des document-Objekts:
\begin{lstlisting}[language=JavaScript, style=basesmol]
// Element finden
document.getElementById("id")  
document.querySelector("selector") 
document.querySelectorAll("selector")

// DOM manipulieren
document.createElement("tag")    
document.createTextNode("text") 
document.createAttribute("attr")

// Event Handler
document.addEventListener("event", handler)
\end{lstlisting}
\end{KR}

\begin{KR}{window-Objekt}
Das window-Objekt als globaler Namespace:
\begin{lstlisting}[language=JavaScript, style=basesmol]
// Globale Methoden
window.alert("message")
window.setTimeout(callback, delay)
window.requestAnimationFrame(callback)

// Eigenschaften
window.innerHeight  // Viewport Hoehe
window.pageYOffset  // Scroll Position
window.location    // URL Infos
\end{lstlisting}
\end{KR}

\subsection{DOM (Document Object Model)}

\begin{KR}{DOM Manipulation}
Grundlegende Schritte zur DOM Manipulation:

1. Element(e) finden:
\begin{lstlisting}[language=JavaScript, style=basesmol]
let element = document.getElementById("id")
let elements = document.querySelectorAll(".class")
\end{lstlisting}

2. Elemente erstellen:
\begin{lstlisting}[language=JavaScript, style=basesmol]
let newElem = document.createElement("div")
let text = document.createTextNode("content")
newElem.appendChild(text)
\end{lstlisting}

3. DOM modifizieren:
\begin{lstlisting}[language=JavaScript, style=basesmol]
// Hinzufuegen
parent.appendChild(newElem)
parent.insertBefore(newElem, referenceNode)

// Entfernen
element.remove()
parent.removeChild(element)

// Ersetzen
parent.replaceChild(newElem, oldElem)
\end{lstlisting}

4. Attribute/Style setzen:
\begin{lstlisting}[language=JavaScript, style=basesmol]
element.setAttribute("class", "highlight")
element.style.backgroundColor = "red"
\end{lstlisting}
\end{KR}