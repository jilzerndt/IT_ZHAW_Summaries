\section{UI-Bibliotheken und Komponenten}

\subsection{Frameworks und Bibliotheken}

\begin{definition}{Framework vs. Bibliothek}
    \begin{itemize}
        \item \textbf{Bibliothek}: 
            \begin{itemize}
                \item Kontrolle beim eigenen Programm
                \item Funktionen werden nach Bedarf verwendet
                \item Beispiel: jQuery
            \end{itemize}
        \item \textbf{Framework}: 
            \begin{itemize}
                \item Rahmen für die Anwendung
                \item Kontrolle liegt beim Framework
                \item "Hollywood-Prinzip": don't call us, we'll call you
            \end{itemize}
    \end{itemize}
\end{definition}

\begin{concept}{Komponentenbasierte Entwicklung}
    Grundprinzipien:
    \begin{itemize}
        \item UI in wiederverwendbare Komponenten aufteilen
        \item Klarer Datenfluss (Props down, Events up)
        \item Deklarativer Ansatz
        \item Komponenten können verschachtelt werden
        \item Zustandsverwaltung in Komponenten
        \item Container vs. Präsentations-Komponenten
    \end{itemize}
\end{concept}

\subsection{JSX und SJDON}

\begin{definition}{JSX}
    \begin{itemize}
        \item XML-Syntax in JavaScript
        \item Muss zu JavaScript transpiliert werden
        \item HTML-Tags in Kleinbuchstaben
        \item Eigene Komponenten mit Großbuchstaben
        \item JavaScript-Ausdrücke in \{...\}
    \end{itemize}
\begin{lstlisting}[language=JavaScript, style=basesmol]
// JSX Komponente
const Welcome = ({name}) => (
    <div className="welcome">
        <h1>Hello, {name}</h1>
        <p>Welcome to our site!</p>
    </div>
);

// Verwendung
const element = <Welcome name="Alice" />;
\end{lstlisting}
\end{definition}

\begin{definition}{SJDON}
    Simple JavaScript DOM Notation:
    \begin{itemize}
        \item Alternative zu JSX
        \item Verwendet pure JavaScript Arrays und Objekte
        \item Kein Kompilierungsschritt nötig
        \item Array-basierte Notation
    \end{itemize}
\begin{lstlisting}[language=JavaScript, style=basesmol]
// SJDON Komponente
const Welcome = ({name}) => [
    "div", {className: "welcome"},
    ["h1", `Hello, ${name}`],
    ["p", "Welcome to our site!"]
];

// Verwendung
const element = [Welcome, {name: "Alice"}];
\end{lstlisting}
\end{definition}

\begin{KR}{Vergleich JSX und SJDON}
\begin{lstlisting}[language=JavaScript, style=basesmol]
// JSX
const element = (
    <div style={{background: 'salmon'}}>
        <h1>Hello World</h1>
        <h2 style={{textAlign: 'right'}}>
            from Web Framework
        </h2>
    </div>
);

// SJDON
const element = [
    "div", {style: "background:salmon"},
    ["h1", "Hello World"],
    ["h2", {style: "text-align:right"}, 
        "from Web Framework"]
];
\end{lstlisting}
\end{KR}

\subsection{SuiWeb Framework}

\begin{concept}{SuiWeb Grundkonzepte}
    Simple User Interface Toolkit for Web Exercises:
    \begin{itemize}
        \item Komponentenbasiert wie React
        \item Unterstützt JSX und SJDON
        \item Datengesteuert mit Props und State
        \item Vereinfachte Implementation für Lernzwecke
        \item Props sind read-only
        \item State für veränderliche Daten
    \end{itemize}
\end{concept}

\subsection{State Management}

\begin{concept}{State Hook}
    \begin{itemize}
        \item Zustandsverwaltung in Funktionskomponenten
        \item Initialisierung mit useState Hook
        \item State Updates lösen Re-Rendering aus
        \item Asynchrone Updates werden gequeued
    \end{itemize}
\end{concept}

\begin{KR}{State Verwaltung}
\begin{lstlisting}[language=JavaScript, style=basesmol]
const Counter = () => {
    // State initialisieren
    const [count, setCount] = useState(0);
    
    // Event Handler
    const increment = () => setCount(count + 1);
    const decrement = () => setCount(count - 1);
    
    return [
        "div",
        ["button", {onclick: decrement}, "-"],
        ["span", count],
        ["button", {onclick: increment}, "+"]
    ];
};

// Komplexere State Objekte
const Form = () => {
    const [state, setState] = useState({
        username: '',
        email: '',
        isValid: false
    });
    
    const updateField = (field, value) => {
        setState({
            ...state,
            [field]: value
        });
    };
};
\end{lstlisting}
\end{KR}

\begin{KR}{Kontrollierte Eingabefelder}
\begin{lstlisting}[language=JavaScript, style=basesmol]
const InputForm = () => {
    const [text, setText] = useState("");
    
    return [
        "form",
        ["input", {
            type: "text",
            value: text,
            oninput: e => setText(e.target.value)
        }],
        ["p", "Eingabe: ", text]
    ];
};
\end{lstlisting}
\end{KR}

\subsection{Komponenten-Design}

\begin{concept}{Container Components}
    \begin{itemize}
        \item Trennung von Daten und Darstellung
        \item Container kümmern sich um:
            \begin{itemize}
                \item Datenbeschaffung
                \item Zustandsverwaltung
                \item Event Handling
            \end{itemize}
        \item Präsentationskomponenten sind zustandslos
    \end{itemize}
\end{concept}

\begin{KR}{Container Komponente}
\begin{lstlisting}[language=JavaScript, style=basesmol]
const TodoContainer = () => {
    const [todos, setTodos] = useState([]);
    
    // Daten laden
    if (todos.length === 0) {
        fetchTodos().then(data => setTodos(data));
    }
    
    // Event Handler
    const addTodo = (text) => {
        setTodos([...todos, {
            id: Date.now(),
            text,
            completed: false
        }]);
    };
    
    const toggleTodo = (id) => {
        setTodos(todos.map(todo =>
            todo.id === id
                ? {...todo, completed: !todo.completed}
                : todo
        ));
    };
    
    // Render Präsentationskomponente
    return [TodoList, {
        todos,
        onToggle: toggleTodo,
        onAdd: addTodo
    }];
};

// Präsentationskomponente
const TodoList = ({todos, onToggle, onAdd}) => [
    "div",
    [TodoForm, {onAdd}],
    ["ul",
        ...todos.map(todo => [
            TodoItem, {
                key: todo.id,
                todo,
                onToggle
            }
        ])
    ]
];
\end{lstlisting}
\end{KR}

\begin{formula}{Component Design Principles}
    \begin{itemize}
        \item Single Responsibility Principle
        \item DRY (Don't Repeat Yourself)
        \item KISS (Keep It Simple, Stupid)
        \item Lifting State Up
        \item Props down, Events up
        \item Komposition über Vererbung
    \end{itemize}
\end{formula}

\subsection{Styling in SuiWeb}

\begin{KR}{Style Optionen}
\begin{lstlisting}[language=JavaScript, style=basesmol]
// String Style
["div", {style: "color: blue; font-size: 16px"}]

// Style Objekt
const styles = {
    container: {
        backgroundColor: "lightgray",
        padding: "10px"
    },
    text: {
        color: "darkblue",
        fontSize: "14px"
    }
};

// Kombinierte Styles
["div", {
    style: [
        styles.container,
        {borderRadius: "5px"}
    ]
}]
\end{lstlisting}
\end{KR}

\begin{formula}{Styling Best Practices}
    \begin{itemize}
        \item Konsistente Styling-Methode verwenden
        \item Styles in separaten Objekten/Modulen
        \item Wiederverwendbare Style-Definitionen
        \item Responsive Design beachten
        \item CSS-Klassen für komplexe Styles
    \end{itemize}
\end{formula}

\subsection{Performance Optimierung}

\begin{concept}{Rendering Optimierung}
    \begin{itemize}
        \item Virtuelles DOM für effizientes Re-Rendering
        \item Batching von State Updates
        \item Memoization von Komponenten
        \item Lazy Loading
        \item Key Prop für Listen-Elemente
    \end{itemize}
\end{concept}

\begin{KR}{Performance Best Practices}
\begin{lstlisting}[language=JavaScript, style=basesmol]
// Effiziente Listen-Rendering
const List = ({items}) => [
    "ul",
    ...items.map(item => [
        "li",
        {key: item.id},  // Wichtig für Performance
        item.text
    ])
];

// Lazy Loading
const LazyComponent = async () => {
    const module = await import('./Component.js');
    return module.default;
};
\end{lstlisting}
\end{KR}