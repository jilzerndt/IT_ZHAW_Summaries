\section{Web Development Introduction}

\begin{definition}{Web Architecture}
    \textbf{Client-Server Model:}
    \begin{itemize}
        \item Browser (Client) sendet Anfragen an Server
        \item Server verarbeitet Anfragen und sendet Antworten
        \item Kommunikation über HTTP/HTTPS (Port 80/443)
    \end{itemize}
\end{definition}

\begin{concept}{Core Technologies}
    \textbf{Client-Side} $\rightarrow$ Front-end Development
    \begin{itemize}
        \item HTML: Structure and content
        \item CSS: Styling and layout
        \item JavaScript: Behavior and interactivity
        \item Browser APIs and Web Standards
    \end{itemize}

    \textbf{Server-Side} $\rightarrow$ Back-end Development
    \begin{itemize}
        \item Choice of platform and programming language
        \item Generates browser-compatible output
        \item Examples: Node.js, Express, REST APIs
    \end{itemize}
\end{concept}

\begin{theorem}{Internet vs. WWW}
    \textbf{Internet:}
    \begin{itemize}
        \item Global network of interconnected computer networks
        \item Various services: Email, FTP, WWW, etc.
        \item Core protocols: TCP/IP
        \item Originally ARPANET (1969)
    \end{itemize}
    
    \textbf{World Wide Web:}
    \begin{itemize}
        \item Service built on top of the Internet
        \item Developed by Tim Berners-Lee at CERN (1990s)
        \item Based on: HTTP, HTML, URLs
        \item Browser as client application
    \end{itemize}
\end{theorem}

\begin{concept}{Web Standards}
    \textbf{Standards Organizations:}
    \begin{itemize}
        \item \textbf{W3C} (World Wide Web Consortium)
            \begin{itemize}
                \item Founded 1994 at MIT
                \item Led by Tim Berners-Lee
                \item Standardizes web technologies
            \end{itemize}
        \item \textbf{WHATWG} (Web Hypertext Application Technology Working Group)
            \begin{itemize}
                \item Founded by Apple, Mozilla, Opera
                \item Later joined by Microsoft, Google
                \item Maintains HTML Living Standard
            \end{itemize}
        \item \textbf{Browser Vendors}
            \begin{itemize}
                \item Implement and influence standards
                \item Chrome, Firefox, Safari, Edge
                \item Growing influence on web development
            \end{itemize}
    \end{itemize}
\end{concept}

\begin{corollary}{Common Ports}
    \begin{center}
    \begin{tabular}{|l|l|}
        \hline
        \textbf{Port} & \textbf{Service} \\
        \hline
        20 & FTP (Data) \\
        21 & FTP (Control) \\
        22 & SSH \\
        23 & Telnet \\
        25 & SMTP \\
        53 & DNS \\
        80 & HTTP \\
        443 & HTTPS \\
        \hline
    \end{tabular}
    \end{center}
\end{corollary}

\begin{concept}{Web Development Approaches}
    Historical evolution:
    \begin{enumerate}
        \item Static web pages
        \item Server-generated content (CGI, Perl)
        \item Server-side scripting (PHP)
        \item Client-side scripting (JavaScript)
        \item Single Page Applications (SPAs)
        \item Modern web frameworks
    \end{enumerate}
    
    Current trends:
    \begin{itemize}
        \item Component-based development
        \item Client-side rendering
        \item RESTful APIs
        \item Progressive Web Apps (PWAs)
        \item Responsive design
    \end{itemize}
\end{concept}