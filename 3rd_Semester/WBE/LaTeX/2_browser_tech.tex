\section{Browser Technologies}

\subsection{Document Object Model (DOM)}

\begin{concept}{DOM Structure}
    \begin{itemize}
        \item Tree representation of HTML document
        \item Each HTML element becomes a node
        \item Nodes can be elements, text, or attributes
        \item Provides API for dynamic manipulation
        \item Foundation for interactive web applications
    \end{itemize}
\end{concept}

\begin{KR}{DOM Manipulation}
\begin{lstlisting}[language=JavaScript, style=basesmol]
// Selecting elements
const element = document.getElementById('myId');
const elements = document.getElementsByClassName('myClass');
const element = document.querySelector('.myClass');
const elements = document.querySelectorAll('div.myClass');

// Creating elements
const div = document.createElement('div');
const text = document.createTextNode('Hello');
div.appendChild(text);

// Modifying elements
element.innerHTML = '<span>New content</span>';
element.textContent = 'New text';
element.setAttribute('class', 'newClass');
element.classList.add('newClass');
element.style.backgroundColor = 'red';

// Tree navigation
element.parentNode
element.childNodes
element.children
element.firstChild
element.nextSibling
\end{lstlisting}
\end{KR}

\subsection{Events}

\begin{definition}{Event Handling}
    Events represent interactions or state changes:
    \begin{itemize}
        \item User interactions (clicks, keyboard input)
        \item Document loading stages
        \item Network status changes
        \item Timer completions
    \end{itemize}
\end{definition}

\begin{KR}{Event Listeners}
\begin{lstlisting}[language=JavaScript, style=basesmol]
// Adding event listeners
element.addEventListener('click', (event) => {
    console.log('Clicked!', event);
    event.preventDefault();  // Prevent default behavior
    event.stopPropagation(); // Stop event bubbling
});

// Removing event listeners
const handler = (event) => {
    console.log('Handler');
};
element.addEventListener('click', handler);
element.removeEventListener('click', handler);

// Event delegation
document.addEventListener('click', (event) => {
    if (event.target.matches('.button')) {
        // Handle button clicks
    }
});
\end{lstlisting}
\end{KR}

\begin{formula}{Common Events}
    \begin{itemize}
        \item Mouse: \texttt{click}, \texttt{dblclick}, \texttt{mouseover}, \texttt{mouseout}
        \item Keyboard: \texttt{keydown}, \texttt{keyup}, \texttt{keypress}
        \item Form: \texttt{submit}, \texttt{change}, \texttt{input}, \texttt{focus}, \texttt{blur}
        \item Document: \texttt{DOMContentLoaded}, \texttt{load}
        \item Window: \texttt{resize}, \texttt{scroll}
    \end{itemize}
\end{formula}

\subsection{Browser APIs}

\begin{concept}{Web APIs}
    Modern browsers provide numerous APIs:
    \begin{itemize}
        \item Storage (localStorage, sessionStorage)
        \item Fetch (network requests)
        \item Canvas and WebGL (graphics)
        \item Web Workers (parallel processing)
        \item Geolocation
        \item WebSockets (real-time communication)
    \end{itemize}
\end{concept}

\begin{KR}{Web Storage}
\begin{lstlisting}[language=JavaScript, style=basesmol]
// localStorage (persists between sessions)
localStorage.setItem('key', 'value');
const value = localStorage.getItem('key');
localStorage.removeItem('key');
localStorage.clear();

// sessionStorage (cleared when session ends)
sessionStorage.setItem('key', 'value');
const value = sessionStorage.getItem('key');

// Storing objects
const user = { name: 'John', age: 30 };
localStorage.setItem('user', JSON.stringify(user));
const storedUser = JSON.parse(localStorage.getItem('user'));
\end{lstlisting}
\end{KR}

\begin{KR}{Fetch API}
\begin{lstlisting}[language=JavaScript, style=basesmol]
// GET request
fetch('https://api.example.com/data')
    .then(response => response.json())
    .then(data => console.log(data))
    .catch(error => console.error('Error:', error));

// POST request
fetch('https://api.example.com/data', {
    method: 'POST',
    headers: {
        'Content-Type': 'application/json',
    },
    body: JSON.stringify({
        name: 'John',
        age: 30
    })
})
.then(response => response.json())
.then(data => console.log(data));

// With async/await
async function fetchData() {
    try {
        const response = await fetch('https://api.example.com/data');
        const data = await response.json();
        console.log(data);
    } catch (error) {
        console.error('Error:', error);
    }
}
\end{lstlisting}
\end{KR}

\subsection{Forms and HTTP}

\begin{definition}{HTML Forms}
    Forms enable user input and data submission:
    \begin{itemize}
        \item \texttt{<form>} element with action and method
        \item Various input types (text, password, checkbox, etc.)
        \item Form validation (HTML5 and JavaScript)
        \item Data submission via GET or POST
    \end{itemize}
\end{definition}

\begin{KR}{Form Handling}
\begin{lstlisting}[language=JavaScript, style=basesmol]
// Form submission
const form = document.querySelector('form');
form.addEventListener('submit', async (event) => {
    event.preventDefault();
    
    const formData = new FormData(form);
    try {
        const response = await fetch('/submit', {
            method: 'POST',
            body: formData
        });
        const result = await response.json();
        console.log(result);
    } catch (error) {
        console.error('Error:', error);
    }
});

// Form validation
const input = document.querySelector('input');
input.addEventListener('input', (event) => {
    if (input.validity.typeMismatch) {
        input.setCustomValidity('Please enter a valid email');
    } else {
        input.setCustomValidity('');
    }
});
\end{lstlisting}
\end{KR}

\begin{formula}{HTTP Methods}
    \begin{center}
    \begin{tabular}{|l|l|}
    \hline
    \textbf{Method} & \textbf{Purpose} \\
    \hline
    GET & Retrieve data \\
    POST & Create new resource \\
    PUT & Update entire resource \\
    PATCH & Partial update \\
    DELETE & Remove resource \\
    \hline
    \end{tabular}
    \end{center}
\end{formula}

\subsection{Express.js}

\begin{concept}{Express Framework}
    Minimal web application framework for Node.js:
    \begin{itemize}
        \item Routing system
        \item Middleware support
        \item Static file serving
        \item Template engine integration
        \item Error handling
    \end{itemize}
\end{concept}

\begin{KR}{Express Basic Server}
\begin{lstlisting}[language=JavaScript, style=basesmol]
const express = require('express');
const app = express();

// Middleware
app.use(express.json());
app.use(express.urlencoded({ extended: true }));
app.use(express.static('public'));

// Routes
app.get('/', (req, res) => {
    res.send('Hello World');
});

app.post('/api/data', (req, res) => {
    const data = req.body;
    // Process data
    res.json({ success: true, data });
});

// Error handling
app.use((err, req, res, next) => {
    console.error(err.stack);
    res.status(500).send('Something broke!');
});

// Start server
app.listen(3000, () => {
    console.log('Server running on port 3000');
});
\end{lstlisting}
\end{KR}

\subsection{Security Considerations}

\begin{concept}{Web Security}
    Common security concerns:
    \begin{itemize}
        \item Cross-Site Scripting (XSS)
        \item Cross-Site Request Forgery (CSRF)
        \item SQL Injection
        \item Session Hijacking
        \item Man-in-the-Middle Attacks
    \end{itemize}
\end{concept}

\begin{KR}{Security Best Practices}
\begin{lstlisting}[language=JavaScript, style=basesmol]
// Input sanitization
const sanitizeHTML = require('sanitize-html');
const cleanHTML = sanitizeHTML(dirtyHTML);

// CSRF Protection
app.use(csrf());
<form>
    <input type="hidden" name="_csrf" value="<%= csrfToken %>">
</form>

// Secure cookies
app.use(session({
    secret: 'secret-key',
    cookie: {
        secure: true,
        httpOnly: true,
        sameSite: 'strict'
    }
}));

// CORS
app.use(cors({
    origin: 'https://trusted-domain.com',
    methods: ['GET', 'POST']
}));
\end{lstlisting}
\end{KR}