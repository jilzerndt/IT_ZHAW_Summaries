\subsection{Browser Technologies Exam Preparation}

\begin{concept}{Key Browser API Concepts}
    Essential topics to focus on:
    \begin{itemize}
        \item DOM structure and manipulation
        \item Event handling and bubbling
        \item Browser storage mechanisms (localStorage, cookies, sessions)
        \item Window and document objects
        \item Browser events lifecycle
        \item Form handling and validation
        \item Canvas vs SVG
        \item AJAX and Fetch API
    \end{itemize}
\end{concept}

\begin{formula}{Common Pitfalls}
    Watch out for these tricky areas:
    \begin{itemize}
        \item Event propagation order (capturing vs bubbling)
        \item Storage API limitations and restrictions
        \item DOM traversal methods and returns
        \item Event delegation vs direct binding
        \item Canvas context state management
        \item Async nature of fetch requests
        \item CORS and same-origin policy
    \end{itemize}
\end{formula}

\begin{KR}{Sample Multiple Choice Questions}

\paragraph{DOM Manipulation}
What is returned by this code?
\begin{lstlisting}[language=JavaScript, style=basesmol]
document.querySelectorAll('div').length;
document.getElementsByTagName('div').length;
// When a new div is added dynamically
\end{lstlisting}
\begin{itemize}
    \item a) Both stay the same
    \item b) QuerySelectorAll stays same, getElementsByTagName updates - correct
    \item c) Both update
    \item d) Both throw error
\end{itemize}
Explanation: getElementsByTagName returns a live NodeList, querySelectorAll returns a static NodeList

\paragraph{Event Handling}
In what order are the events logged?
\begin{lstlisting}[language=JavaScript, style=basesmol]
<div onclick="console.log('1')">
    <button onclick="console.log('2')">Click</button>
</div>
\end{lstlisting}
\begin{itemize}
    \item a) 1, 2
    \item b) 2, 1 - correct
    \item c) Just 2
    \item d) Just 1
\end{itemize}
Explanation: Events bubble up from target to ancestors by default

\paragraph{Local Storage}
What happens here?
\begin{lstlisting}[language=JavaScript, style=basesmol]
localStorage.setItem('number', 42);
console.log(typeof localStorage.getItem('number'));
\end{lstlisting}
\begin{itemize}
    \item a) "number"
    \item b) "string" - correct
    \item c) "object"
    \item d) "undefined"
\end{itemize}
Explanation: localStorage always stores values as strings
\end{KR}

\begin{KR}{MC}
\paragraph{Canvas vs SVG}
Which statement is true?
\begin{itemize}
    \item a) SVG elements can have event listeners - correct
    \item b) Canvas elements maintain separate elements for shapes
    \item c) SVG is pixel-based
    \item d) Canvas is better for complex animations
\end{itemize}
Explanation: SVG elements become part of the DOM and can have event listeners attached

\paragraph{Fetch API}
What is logged?
\begin{lstlisting}[language=JavaScript, style=basesmol]
fetch('/api/data')
    .then(response => response.json())
    .then(data => console.log(data))
    .catch(err => console.log('Error'));
// When server returns 404
\end{lstlisting}
\begin{itemize}
    \item a) Error
    \item b) undefined
    \item c) null
    \item d) The response body - correct
\end{itemize}
Explanation: fetch() only rejects on network failure, not HTTP errors
\end{KR}

\begin{concept}{Common Exam Patterns}
Look for questions about:
\begin{itemize}
    \item DOM method return types (NodeList vs HTMLCollection)
    \item Event propagation and prevention
    \item Storage API value types and limitations
    \item Differences between Canvas and SVG
    \item Asynchronous operations with fetch
    \item CORS and security restrictions
    \item Form validation and submission
\end{itemize}
\end{concept}

\begin{examplecode}{Critical Code Patterns}
\begin{lstlisting}[language=JavaScript, style=basesmol]
// Event Delegation
document.getElementById('parent').addEventListener('click', (e) => {
    if (e.target.matches('.child')) {
        console.log('Child clicked');
    }
});

// Storage with Objects
const user = { name: 'John', age: 30 };
localStorage.setItem('user', JSON.stringify(user));
const savedUser = JSON.parse(localStorage.getItem('user'));

// Async Form Submission
form.addEventListener('submit', async (e) => {
    e.preventDefault();
    const formData = new FormData(form);
    try {
        const response = await fetch('/api/submit', {
            method: 'POST',
            body: formData
        });
        if (!response.ok) throw new Error('Network response was not ok');
        const result = await response.json();
    } catch (error) {
        console.error('Error:', error);
    }
});
\end{lstlisting}
\end{examplecode}

\begin{theorem}{Key Exam Strategies}
    \begin{itemize}
        \item Understand DOM live vs static collections
        \item Know event propagation phases
        \item Remember storage API limitations
        \item Understand async operation order
        \item Know security restrictions (CORS, same-origin)
        \item Recognize Canvas vs SVG use cases
        \item Consider browser compatibility issues
    \end{itemize}
\end{theorem}

\subsection{more examples}


\begin{code}{Event abonnieren/entfernen}
    Mit der Methode addEventListener() wird ein Event abonniert. Mit removeEventListener kann das Event entfernt werden.
  \begin{lstlisting}[language=JavaScript, style=basesmol]
  <button>Act-once button</button>
  <script>
    let button = document.querySelector("button")
    function once () {
      console.log("Done.")
      button.removeEventListener("click", once)
    }
    button.addEventListener("click", once)
  </script>
  \end{lstlisting}
  \end{code}

  \begin{KR}{Class und Prototype Patterns}
\begin{lstlisting}[language=JavaScript, style=basesmol]
// 1. Class Definition
class Animal {
    constructor(name) {
        this.name = name;
    }
    
    speak() {
        return `${this.name} makes a sound`;
    }
}

// 2. Inheritance
class Dog extends Animal {
    constructor(name, breed) {
        super(name);
        this.breed = breed;
    }
    
    speak() {
        return `${this.name} barks`;
    }
}

// 3. Getter/Setter
class Circle {
    constructor(radius) {
        this._radius = radius;
    }
    
    get radius() {
        return this._radius;
    }
    
    set radius(value) {
        if (value <= 0) throw new Error('Invalid radius');
        this._radius = value;
    }
    
    get area() {
        return Math.PI * this._radius ** 2;
    }
}

// 4. Static Methods
class MathUtils {
    static add(x, y) {
        return x + y;
    }
    
    static multiply(x, y) {
        return x * y;
    }
}
\end{lstlisting}
\end{KR}

\begin{KR}{Error Handling Patterns}
\begin{lstlisting}[language=JavaScript, style=basesmol]
// 1. Try-Catch mit spezifischen Errors
try {
    JSON.parse(invalidJson);
} catch (error) {
    if (error instanceof SyntaxError) {
        console.error('Invalid JSON');
    } else {
        console.error('Unknown error:', error);
    }
}

// 2. Custom Error Classes
class ValidationError extends Error {
    constructor(message) {
        super(message);
        this.name = 'ValidationError';
    }
}

// 3. Async Error Handling
async function fetchData() {
    try {
        const response = await fetch(url);
        if (!response.ok) {
            throw new Error(`HTTP error: ${response.status}`);
        }
        return await response.json();
    } catch (error) {
        console.error('Fetch failed:', error);
        throw error; // Re-throw fuer weitere Verarbeitung
    }
}

// 4. Promise Error Chain
fetchData()
    .then(processData)
    .then(saveData)
    .catch(error => {
        if (error instanceof NetworkError) {
            retryOperation();
        } else {
            logError(error);
        }
    })
    .finally(() => cleanup());
\end{lstlisting}
\end{KR}

\begin{KR}{Module Patterns}
\begin{lstlisting}[language=JavaScript, style=basesmol]
// 1. ES6 Module Export/Import
// math.js
export const add = (x, y) => x + y;
export const multiply = (x, y) => x * y;
export default class Calculator {
    // ... 
}

// main.js
import Calculator, { add, multiply } from './math.js';
import * as mathUtils from './math.js';

// 2. CommonJS Module Pattern (Node.js)
// utils.js
module.exports = {
    formatDate(date) {
        return date.toISOString();
    },
    calculateTotal(items) {
        return items.reduce((sum, item) => sum + item.price, 0);
    }
};

// app.js
const utils = require('./utils.js');

// 3. Revealing Module Pattern
const counterModule = (() => {
    let count = 0;
    
    const increment = () => ++count;
    const decrement = () => --count;
    const getCount = () => count;
    
    return {
        increment,
        decrement,
        getCount
    };
})();
\end{lstlisting}
\end{KR}

\begin{examplecode}{REST API Implementation}
\begin{lstlisting}[language=JavaScript, style=basesmol]
// Express REST API Example
const express = require('express');
const app = express();
app.use(express.json());

// GET Collection
app.get('/api/users', async (req, res) => {
    try {
        const users = await User.find();
        res.json(users);
    } catch (error) {
        res.status(500).json({ error: error.message });
    }
});

// GET Single Resource
app.get('/api/users/:id', async (req, res) => {
    try {
        const user = await User.findById(req.params.id);
        if (!user) {
            return res.status(404).json({ error: 'User not found' });
        }
        res.json(user);
    } catch (error) {
        res.status(500).json({ error: error.message });
    }
});

// POST New Resource
app.post('/api/users', async (req, res) => {
    try {
        const user = new User(req.body);
        await user.save();
        res.status(201).json(user);
    } catch (error) {
        res.status(400).json({ error: error.message });
    }
});

// PUT Update Resource
app.put('/api/users/:id', async (req, res) => {
    try {
        const user = await User.findByIdAndUpdate(
            req.params.id, 
            req.body,
            { new: true }
        );
        res.json(user);
    } catch (error) {
        res.status(400).json({ error: error.message });
    }
});
\end{lstlisting}
\end{examplecode}