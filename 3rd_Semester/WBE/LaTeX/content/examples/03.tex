\section{UI Technologien}

\begin{example2}{Warum SuiWeb und nicht gleich React?}
    Konzeption von WBE vor ein paar Jahren mit zwei Kollegen (beide IT-Firma aufgebaut, einer in Glarus, einer in Zürich). Sie hatten die Idee, ein eigenes Framework aufzubauen und nicht ein weit verbreitetes Tool zu nehmen, und zwar aus verschiedenen Gründen:
    \begin{itemize}
        \item Vielleicht der wichtigste Grund: Wir sollen Konzepte vermitteln, nicht Tools oder Bibliotheken. Also: die Idee hinter React, Vuew, Svelte, ... verstehen, was vie/ mehr ist, als ein bestimmtes Tool einsetzen zu können.
        \item Dabei auch: verstehen, wie so eine Bibliothek implementiert werden kann, indem Kernkomponenten selbst implementiert oder zumindest verstanden werden. Auch das ermöglicht ein tieferes Verständnis der zugrunde liegenden Konzepte.
        \item Einer der Kritikpunkte am Ökosystem rund um zum Beispiel React ist, dass viele fortgeschrittene Konzepte und Muster verwendet werden, die Einsteiger schnell einmal überfordern können. Solchen Problemen können wir mit einem eigenen, schlanken Werkzeug besser aus dem Weg gehen.
        \item Tools kommen und gehen. Wir haben schon zu oft erlebt, dass Projekte auf dem aktuell gehypten Framework aufgesetzt haben, welches kurz darauf in der Versenkung verschwunden ist. Google Web Toolkit (GWT) anyone?
        \item Ein eigenes Tool erlaubt auch, mit alternativen Notationen (SJDON) und Umsetzungen (etwa bei den Hooks) experimentieren zu können.
    \end{itemize}    
    Konzeptionell orientiert sich SuiWeb allerdings an React.js, speziell was die Funktionskomponenten und Hooks angeht.
\end{example2}

\begin{example2}{FAQ: Warum SJDON und nicht JSX?}
    \begin{itemize}
        \item SJDON ist eine alternative Notation für die Darstellung von DOM-Strukturen, die wir in SuiWeb verwenden.
        \item SJDON ist eine Art Lisp-Syntax für DOM-Strukturen, die sich sehr gut für hierarchische Strukturen eignet.
        \item SJDON ist eine Art S-Expression, die sich sehr gut für die Verarbeitung durch JavaScript eignet.
        \item SJDON ist eine Art JSON, die sich sehr gut für die Verarbeitung durch JavaScript eignet.
        \item Im Gegensatz zu JSX ist SJDON reines JavaScript und kann direkt durch JavaScript verarbeitet werden, ohne vorgängigen Build Step mit Tools wie Babel.
        \item Das vereinfacht auch Live Demos im Unterricht, wo beim Editieren von SJDON-Strukturen direkt ein Live Preview angezeigt wird (ok, geht auch mit JSX, aber mit etwas mehr Aufwand, Live Server, etc.).
        \item Für Leute mit Lisp-Erfahrung ist SJDON eh die natürliche Art und Weise, hierarchische Strukturen darzustellen...
    \end{itemize}
\end{example2}