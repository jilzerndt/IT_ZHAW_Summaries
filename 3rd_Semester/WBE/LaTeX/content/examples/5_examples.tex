\section{Example Exercises}

\subsection{JavaScript Fundamentals}

\begin{example2}{Basic Array Manipulation}
Write a function that takes an array of numbers and returns a new array containing only the even numbers, doubled.

\begin{lstlisting}[language=JavaScript, style=basesmol]
// Example solution
function processArray(numbers) {
    return numbers
        .filter(num => num % 2 === 0)
        .map(num => num * 2);
}

// Test
console.log(processArray([1, 2, 3, 4, 5, 6])); // [4, 8, 12]
\end{lstlisting}
\end{example2}

\begin{example2}{Closure Implementation}
Create a function that generates unique IDs with a given prefix. Each call should return a new ID with an incrementing number.

\begin{lstlisting}[language=JavaScript, style=basesmol]
// Example solution
function createIdGenerator(prefix) {
    let counter = 0;
    return function() {
        counter++;
        return `${prefix}${counter}`;
    };
}

// Test
const generateUserId = createIdGenerator('user_');
console.log(generateUserId()); // "user_1"
console.log(generateUserId()); // "user_2"
\end{lstlisting}
\end{example2}

\begin{example2}{Async Programming}
Write an async function that fetches user data from two different endpoints and combines them. Handle potential errors appropriately.

\begin{lstlisting}[language=JavaScript, style=basesmol]
async function getUserData(userId) {
    try {
        const [profile, posts] = await Promise.all([
            fetch(`/api/profile/${userId}`).then(r => r.json()),
            fetch(`/api/posts/${userId}`).then(r => r.json())
        ]);
        
        return {
            ...profile,
            posts: posts
        };
    } catch (error) {
        console.error('Failed to fetch user data:', error);
        throw new Error('Failed to load user data');
    }
}
\end{lstlisting}
\end{example2}


\subsection{DOM Manipulation}

\begin{example2}{Dynamic List Creation}
Write a function that takes an array of items and creates a numbered list in the DOM. Add a button to each item that removes it from the list.

\begin{lstlisting}[language=JavaScript, style=basesmol]
function createList(items, containerId) {
    const container = document.getElementById(containerId);
    const ul = document.createElement('ul');
    
    items.forEach((item, index) => {
        const li = document.createElement('li');
        li.textContent = `${index + 1}. ${item} `;
        
        const button = document.createElement('button');
        button.textContent = 'Remove';
        button.onclick = () => li.remove();
        
        li.appendChild(button);
        ul.appendChild(li);
    });
    
    container.appendChild(ul);
}
\end{lstlisting}
\end{example2}

\subsection{Component Implementation}

\begin{example2}{Form Component}
Create a form component in SuiWeb that handles user input with validation and submits data to a server.

\begin{lstlisting}[language=JavaScript, style=basesmol]
const UserForm = () => {
    const [formData, setFormData] = useState({
        username: '',
        email: ''
    });
    const [errors, setErrors] = useState({});
    
    const validate = () => {
        const newErrors = {};
        if (!formData.username) {
            newErrors.username = 'Username is required';
        }
        if (!formData.email.includes('@')) {
            newErrors.email = 'Valid email is required';
        }
        setErrors(newErrors);
        return Object.keys(newErrors).length === 0;
    };
    
    const handleSubmit = async (e) => {
        e.preventDefault();
        if (!validate()) return;
        
        try {
            await fetch('/api/users', {
                method: 'POST',
                headers: {'Content-Type': 'application/json'},
                body: JSON.stringify(formData)
            });
        } catch (error) {
            setErrors({submit: 'Failed to submit form'});
        }
    };
    
    return [
        "form",
        {onsubmit: handleSubmit},
        ["div",
            ["label", {for: "username"}, "Username:"],
            ["input", {
                id: "username",
                value: formData.username,
                oninput: (e) => setFormData({
                    ...formData,
                    username: e.target.value
                })
            }],
            errors.username && ["span", {class: "error"}, errors.username]
        ],
        ["div",
            ["label", {for: "email"}, "Email:"],
            ["input", {
                id: "email",
                type: "email",
                value: formData.email,
                oninput: (e) => setFormData({
                    ...formData,
                    email: e.target.value
                })
            }],
            errors.email && ["span", {class: "error"}, errors.email]
        ],
        ["button", {type: "submit"}, "Submit"]
    ];
};
\end{lstlisting}
\end{example2}

\subsection{API Implementation}

\begin{example2}{REST API with Express}
Create a simple REST API for a todo list with Express.js, including error handling and basic validation.

\begin{lstlisting}[language=JavaScript, style=basesmol]
const express = require('express');
const app = express();
app.use(express.json());

let todos = [];

// Get all todos
app.get('/api/todos', (req, res) => {
    res.json(todos);
});

// Create new todo
app.post('/api/todos', (req, res) => {
    const { title } = req.body;
    
    if (!title) {
        return res.status(400).json({
            error: 'Title is required'
        });
    }
    
    const todo = {
        id: Date.now(),
        title,
        completed: false
    };
    
    todos.push(todo);
    res.status(201).json(todo);
});

// Update todo
app.patch('/api/todos/:id', (req, res) => {
    const { id } = req.params;
    const { completed } = req.body;
    
    const todo = todos.find(t => t.id === parseInt(id));
    
    if (!todo) {
        return res.status(404).json({
            error: 'Todo not found'
        });
    }
    
    todo.completed = completed;
    res.json(todo);
});

app.use((err, req, res, next) => {
    console.error(err);
    res.status(500).json({
        error: 'Internal server error'
    });
});

app.listen(3000);
\end{lstlisting}
\end{example2}

\begin{example2}{State Management}
Implement a shopping cart component that manages products, quantities, and total price calculation.

\begin{lstlisting}[language=JavaScript, style=basesmol]
const ShoppingCart = () => {
    const [items, setItems] = useState([]);
    
    const addItem = (product) => {
        setItems(current => {
            const existing = current.find(
                item => item.id === product.id
            );
            
            if (existing) {
                return current.map(item =>
                    item.id === product.id
                        ? {...item, quantity: item.quantity + 1}
                        : item
                );
            }
            
            return [...current, {...product, quantity: 1}];
        });
    };
    
    const removeItem = (productId) => {
        setItems(current =>
            current.filter(item => item.id !== productId)
        );
    };
    
    const total = items.reduce(
        (sum, item) => sum + item.price * item.quantity,
        0
    );
    
    return [
        "div",
        ["h2", "Shopping Cart"],
        ["ul",
            ...items.map(item => [
                "li",
                ["span", `${item.name} x ${item.quantity}`],
                ["span", `$${item.price * item.quantity}`],
                ["button", 
                    {onclick: () => removeItem(item.id)},
                    "Remove"
                ]
            ])
        ],
        ["div", `Total: $${total.toFixed(2)}`]
    ];
};
\end{lstlisting}
\end{example2}

\subsection{Browser APIs and Events}

\begin{example2}{Custom Event System}
Implement a publish/subscribe system using browser events.

\begin{lstlisting}[language=JavaScript, style=basesmol]
class EventBus {
    constructor() {
        this.eventTarget = new EventTarget();
    }

    publish(eventName, data) {
        const event = new CustomEvent(eventName, {
            detail: data,
            bubbles: true
        });
        this.eventTarget.dispatchEvent(event);
    }

    subscribe(eventName, callback) {
        const handler = (e) => callback(e.detail);
        this.eventTarget.addEventListener(eventName, handler);
        return () => {
            this.eventTarget.removeEventListener(eventName, handler);
        };
    }
}

// Usage
const bus = new EventBus();
const unsubscribe = bus.subscribe('userLoggedIn', (user) => {
    console.log(`Welcome, ${user.name}!`);
});

bus.publish('userLoggedIn', { name: 'John' });
unsubscribe(); // Cleanup
\end{lstlisting}
\end{example2}

\begin{example2}{Drag and Drop}
Implement a simple drag and drop system for list items.

\begin{lstlisting}[language=JavaScript, style=basesmol]
function initDragAndDrop(containerId) {
    const container = document.getElementById(containerId);
    let draggedItem = null;

    container.addEventListener('dragstart', (e) => {
        draggedItem = e.target;
        e.target.classList.add('dragging');
    });

    container.addEventListener('dragend', (e) => {
        e.target.classList.remove('dragging');
    });

    container.addEventListener('dragover', (e) => {
        e.preventDefault();
        const afterElement = getDragAfterElement(container, e.clientY);
        if (afterElement) {
            container.insertBefore(draggedItem, afterElement);
        } else {
            container.appendChild(draggedItem);
        }
    });

    function getDragAfterElement(container, y) {
        const draggableElements = [
            ...container.querySelectorAll('li:not(.dragging)')
        ];

        return draggableElements.reduce((closest, child) => {
            const box = child.getBoundingClientRect();
            const offset = y - box.top - box.height / 2;
            
            if (offset < 0 && offset > closest.offset) {
                return { offset, element: child };
            }
            return closest;
        }, { offset: Number.NEGATIVE_INFINITY }).element;
    }
}
\end{lstlisting}
\end{example2}

\subsection{Data Manipulation and Algorithms}

\begin{example2}{Deep Object Comparison}
Implement a function that deeply compares two objects for equality.

\begin{lstlisting}[language=JavaScript, style=basesmol]
function deepEqual(obj1, obj2) {
    // Handle primitives and null
    if (obj1 === obj2) return true;
    if (obj1 == null || obj2 == null) return false;
    if (typeof obj1 !== 'object' || typeof obj2 !== 'object') 
        return false;

    const keys1 = Object.keys(obj1);
    const keys2 = Object.keys(obj2);

    if (keys1.length !== keys2.length) return false;

    return keys1.every(key => {
        if (!keys2.includes(key)) return false;
        return deepEqual(obj1[key], obj2[key]);
    });
}

// Test
const obj1 = {
    a: 1,
    b: { c: 2, d: [3, 4] },
    e: null
};
const obj2 = {
    a: 1,
    b: { c: 2, d: [3, 4] },
    e: null
};
console.log(deepEqual(obj1, obj2)); // true
\end{lstlisting}
\end{example2}

\begin{example2}{Custom Promise Implementation}
Create a simplified version of the Promise API.

\begin{lstlisting}[language=JavaScript, style=basesmol]
class MyPromise {
    constructor(executor) {
        this.state = 'pending';
        this.value = undefined;
        this.handlers = [];

        const resolve = (value) => {
            if (this.state === 'pending') {
                this.state = 'fulfilled';
                this.value = value;
                this.handlers.forEach(handler => this.handle(handler));
            }
        };

        const reject = (error) => {
            if (this.state === 'pending') {
                this.state = 'rejected';
                this.value = error;
                this.handlers.forEach(handler => this.handle(handler));
            }
        };

        try {
            executor(resolve, reject);
        } catch (error) {
            reject(error);
        }
    }

    handle(handler) {
        if (this.state === 'pending') {
            this.handlers.push(handler);
        } else {
            const cb = this.state === 'fulfilled' 
                ? handler.onSuccess 
                : handler.onFail;
            if (cb) {
                try {
                    const result = cb(this.value);
                    handler.resolve(result);
                } catch (error) {
                    handler.reject(error);
                }
            }
        }
    }

    then(onSuccess, onFail) {
        return new MyPromise((resolve, reject) => {
            this.handle({
                onSuccess: onSuccess || (val => val),
                onFail: onFail || (err => { throw err; }),
                resolve,
                reject
            });
        });
    }

    catch(onFail) {
        return this.then(null, onFail);
    }
}

// Usage
new MyPromise((resolve, reject) => {
    setTimeout(() => resolve('Success!'), 1000);
})
.then(result => console.log(result))
.catch(error => console.error(error));
\end{lstlisting}
\end{example2}

\subsection{Component Testing}

\begin{example2}{Unit Testing Components}
Write tests for a form component using Jasmine.

\begin{lstlisting}[language=JavaScript, style=basesmol]
describe('UserForm Component', () => {
    let form;
    
    beforeEach(() => {
        form = new UserForm();
    });

    it('should initialize with empty values', () => {
        expect(form.state.username).toBe('');
        expect(form.state.email).toBe('');
        expect(Object.keys(form.state.errors)).toHaveSize(0);
    });

    it('should validate email format', () => {
        form.state.email = 'invalid-email';
        const isValid = form.validate();
        
        expect(isValid).toBe(false);
        expect(form.state.errors.email)
            .toContain('Valid email is required');
    });

    it('should submit form with valid data', async () => {
        form.state.username = 'testuser';
        form.state.email = 'test@example.com';
        
        spyOn(window, 'fetch').and.returnValue(
            Promise.resolve({ ok: true })
        );
        
        await form.handleSubmit();
        
        expect(window.fetch).toHaveBeenCalledWith(
            '/api/users',
            jasmine.any(Object)
        );
        expect(form.state.errors).toEqual({});
    });
});
\end{lstlisting}
\end{example2}