\subsection{Browser APIs}

\begin{KR}{Geolocation API}
1. Einmalige Position abfragen:
\begin{lstlisting}[language=JavaScript, style=basesmol]
navigator.geolocation.getCurrentPosition(
  (position) => {
    console.log(position.coords.latitude)
    console.log(position.coords.longitude)
    console.log(position.coords.accuracy)
  },
  (error) => {
    console.error(error.message)
  },
  {
    enableHighAccuracy: true,
    timeout: 5000,
    maximumAge: 0
  }
)
\end{lstlisting}

2. Position kontinuierlich überwachen:
\begin{lstlisting}[language=JavaScript, style=basesmol]
const watchId = navigator.geolocation.watchPosition(
  positionCallback,
  errorCallback,
  options
)

// Ueberwachung beenden
navigator.geolocation.clearWatch(watchId)
\end{lstlisting}
\end{KR}

\begin{KR}{History API}
1. Navigation:
\begin{lstlisting}[language=JavaScript, style=basesmol]
// Navigation
history.back()      // Eine Seite zurueck
history.forward()   // Eine Seite vor
history.go(-2)      // 2 Seiten zurueck
\end{lstlisting}

2. History Manipulation:
\begin{lstlisting}[language=JavaScript, style=basesmol]
// Neuen Eintrag hinzufuegen
history.pushState(
  {page: 1},           // State-Objekt
  '',                  // Title (meist ignoriert)
  '/neue-url'          // URL
)

// Aktuellen Eintrag ersetzen
history.replaceState(
  {page: 2},
  '',
  '/andere-url'
)
\end{lstlisting}

3. Auf Änderungen reagieren:
\begin{lstlisting}[language=JavaScript, style=basesmol]
window.addEventListener('popstate', (event) => {
  console.log(event.state)    // State-Objekt
  console.log(location.href)  // Aktuelle URL
})
\end{lstlisting}
\end{KR}

\begin{KR}{Web Workers}
1. Worker erstellen:
\begin{lstlisting}[language=JavaScript, style=basesmol]
// main.js
const worker = new Worker('worker.js')

worker.postMessage({data: someData})

worker.onmessage = (e) => {
  console.log('Nachricht vom Worker:', e.data)
}

// worker.js
self.onmessage = (e) => {
  // Daten verarbeiten
  const result = doSomeHeavyComputation(e.data)
  self.postMessage(result)
}
\end{lstlisting}

2. Worker beenden:
\begin{lstlisting}[language=JavaScript, style=basesmol]
worker.terminate()  // Im Hauptthread
self.close()       // Im Worker
\end{lstlisting}

Wichtig:
\begin{itemize}
  \item Worker laufen in separatem Thread
  \item Kein Zugriff auf DOM
  \item Kommunikation nur über Nachrichten
  \item Gut für rechenintensive Aufgaben
\end{itemize}
\end{KR}