\section{JavaScript}

\subsection{JS-Grundlagen}
\begin{code}{Web-Konsole}
    In JS mit dem Keyword \texttt{console}:
    \begin{itemize}
        \item \texttt{console.log(message)}: Loggt eine Nachricht
        \item \texttt{console.clear()}: Löscht die Konsole
        \item \texttt{console.trace(message)}: Stack trace ausgeben
        \item \texttt{console.error(message)}: stderr ausgeben
        \item \texttt{console.time()}: Startet einen Timer
        \item \texttt{console.timeEnd()}: Stoppt den Timer
    \end{itemize}
    Website für Konsolen-API: https://nodejs.org/api/console.html
\end{code}

\begin{definition}{Datentypen}
    Bekannte Datentypen wie bei Java, spezielle Datentypen:
    \begin{itemize}
        \item \texttt{undefined}: Variable wurde deklariert, aber nicht initialisiert
        \item \texttt{null}: Variable wurde deklariert und initialisiert, aber nicht belegt
        \item \texttt{Symbol}: Eindeutiger, unveränderlicher Wert
        \item \texttt{Number}: Ganze Zahlen, Fließkommazahlen, NaN, Infinity
        \begin{itemize}
            \item Infinity: $1/0$, $-1/0$
            \item NaN: $0/0$, $\sqrt{-1}$
        \end{itemize}
        \item \texttt{BigInt}: Ganze Zahlen beliebiger Größe
        \item \texttt{Object}: Sammlung von Schlüssel-Wert-Paaren
        \item \texttt{Function}: Funktionen sind Objekte
    \end{itemize}
\end{definition}

\begin{concept}{Variablen Definition}
    \begin{itemize}
        \item \texttt{var}: Global oder lokal
        \item \texttt{let}: Nur lokal
        \item \texttt{const}: Konstante
    \end{itemize}
\end{concept}

\begin{definition}{Operatoren}
    \begin{itemize}
        \item Arithmetische Operatoren: $+, -, *, /, \%, ++, --$
        \item Zuweisungsoperatoren: $=, +=, -=, *=, /=, \%=, **=, <<=, >>=, >>>=, &=, ^=, |=$
        \item Vergleichsoperatoren: $==, ===, !=, !==, >, <, >=, <=$
        \item Logische Operatoren: $\&\&, ||, !$
        \item Bitweise Operatoren: $\&, |, ^, ~, <<, >>, >>>$
        \item Sonstige Operatoren: \texttt{typeof}, \texttt{instanceof}
    \end{itemize}
\end{definition}

\begin{formula}{Vergleich mit \texttt{==} und \texttt{===}}
    \begin{itemize}
        \item \texttt{==}: Vergleicht Werte, konvertiert Datentypen
        \item \texttt{===}: Vergleicht Werte und Datentypen ohne Konvertierung
    \end{itemize}
    ebenfalls: \texttt{!=} und \texttt{!==}
\end{formula}

\begin{code}{Verzweigungen\text{,} Wiederholung und Switch Case}
    \begin{itemize}
        \item \texttt{if (condition) \{...\} else \{...\}}
        \item \texttt{switch (expression) \{ case x: ... break; default: ... \}}
        \item \texttt{for (initialization; condition; increment) \{...\}}
        \item \texttt{while (condition) \{...\}}
        \item \texttt{do \{...\} while (condition)}
        \item \texttt{for (let x of iterable) \{...\}}
    \end{itemize}
\end{code}

\begin{code}{Funktionsdefinition}
    \begin{itemize}
        \item \texttt{function name(parameters) \{...\}}
        \item \texttt{const name = (parameters) => \{...\}}
        \item \texttt{const name = parameters => \{...\}}
        \item \texttt{const name = parameters => expression}
    \end{itemize}
\end{code}

\begin{lstlisting}[language=JavaScript]
    // Beispiel einer Funktion
    function add(a, b) {
        return a + b;
    }
    // Beispiel einer Arrow-Funktion
    const add = (a, b) => a + b;
\end{lstlisting}


\subsection{Objekte und Arrays}

\begin{concept}{Objekt vs Array}
    \begin{tabular}{lll}
        Was & Objekt & Array\\
        \hline
        Art & Attribut-Wert-Paar & Sequenz von Werten\\
        Literalnotation & Werte =
        
    \end{tabular}
\end{concept}

\subsection{Funktionen und funktionale Programmierung}

\subsection{Prototypen von Objekten}

\subsection{Asynchrone Programmierung}

\subsection{Webserver}