\section{JavaScript}

\begin{concept}{JavaScript Overview}
    \begin{itemize}
        \item Created by Brendan Eich in 1995 for Netscape Navigator
        \item Dynamic, weakly typed programming language
        \item Multi-paradigm: Object-oriented, functional, imperative
        \item Originally for client-side scripting, now also server-side (Node.js)
        \item Regular updates via ECMAScript standard
    \end{itemize}
\end{concept}

\subsection{Core Language Features}

\begin{definition}{Data Types}
    Primitive Types:
    \begin{itemize}
        \item \texttt{number}: 64-bit floating point (IEEE 754)
        \item \texttt{bigint}: arbitrary precision integers (with n suffix)
        \item \texttt{string}: text in '', "", or ``
        \item \texttt{boolean}: true/false
        \item \texttt{undefined}: uninitialized value
        \item \texttt{null}: intentionally empty value
        \item \texttt{symbol}: unique identifier
    \end{itemize}
\end{definition}

\begin{KR}{Type Checking}
\begin{lstlisting}[language=JavaScript, style=basesmol]
// Type checking with typeof
typeof 42          // 'number'
typeof 42n         // 'bigint'
typeof "text"      // 'string'
typeof true        // 'boolean'
typeof undefined   // 'undefined'
typeof null        // 'object' (historical bug!)
typeof {}          // 'object'
typeof []          // 'object'
typeof (() => {})  // 'function'

// Special number values
console.log(Infinity)    // Division by zero
console.log(NaN)        // Invalid numeric operation
\end{lstlisting}
\end{KR}

\begin{definition}{Variables}
    Three ways to declare variables:
    \begin{itemize}
        \item \texttt{var}: function-scoped, hoisted (avoid)
        \item \texttt{let}: block-scoped, mutable
        \item \texttt{const}: block-scoped, immutable reference
    \end{itemize}
\end{definition}

\begin{KR}{Control Structures}
\begin{lstlisting}[language=JavaScript, style=basesmol]
// Conditionals
if (condition) {
    // code
} else if (otherCondition) {
    // code
} else {
    // code
}

// Switch statement
switch(value) {
    case 1:
        // code
        break;
    default:
        // code
}

// Loops
for (let i = 0; i < n; i++) { }
while (condition) { }
do { } while (condition);
for (let item of array) { }
for (let key in object) { }
\end{lstlisting}
\end{KR}

\subsection{Objects and Arrays}

\begin{concept}{Objects}
    Key characteristics:
    \begin{itemize}
        \item Collections of key-value pairs
        \item Dynamic - properties can be added/removed
        \item Keys are strings or symbols
        \item Values can be any type, including functions (methods)
        \item Prototype-based inheritance
    \end{itemize}
\end{concept}

\begin{KR}{Object Manipulation}
\begin{lstlisting}[language=JavaScript, style=basesmol]
// Object creation
const person = {
    name: "Alice",
    age: 30,
    greet() {
        return `Hello, I'm ${this.name}`;
    }
};

// Property access
person.name           // dot notation
person["age"]         // bracket notation

// Property manipulation
person.job = "Developer";    // add
delete person.age;           // delete
"name" in person;           // check existence

// Object methods
Object.keys(person)         // get keys
Object.values(person)       // get values
Object.entries(person)      // get key-value pairs
Object.assign(target, ...sources)  // merge objects
\end{lstlisting}
\end{KR}

\begin{concept}{Arrays}
    Special objects for ordered collections:
    \begin{itemize}
        \item Zero-based indexing
        \item Dynamic length
        \item Can contain mixed types
        \item Many built-in methods for manipulation
    \end{itemize}
\end{concept}

\begin{KR}{Array Methods}
\begin{lstlisting}[language=JavaScript, style=basesmol]
const arr = [1, 2, 3];

// Modifying arrays
arr.push(4);             // add to end
arr.pop();              // remove from end
arr.unshift(0);         // add to start
arr.shift();            // remove from start
arr.splice(1, 1, 'new'); // remove/insert at position

// Accessing arrays
arr.slice(1, 3);        // get sub-array
arr.indexOf(2);         // find element
arr.includes(2);        // check existence

// Functional methods
arr.map(x => x * 2);    // transform elements
arr.filter(x => x > 2); // filter elements
arr.reduce((a, b) => a + b); // reduce to value
arr.forEach(x => console.log(x)); // iterate
\end{lstlisting}
\end{KR}

\subsection{Functions and Closures}

\begin{concept}{Functions}
    Functions in JavaScript are first-class objects:
    \begin{itemize}
        \item Can be assigned to variables
        \item Passed as arguments
        \item Returned from other functions
        \item Have their own properties and methods
    \end{itemize}
\end{concept}

\begin{KR}{Function Declarations}
\begin{lstlisting}[language=JavaScript, style=basesmol]
// Function declaration
function greet(name) {
    return `Hello, ${name}!`;
}

// Function expression
const greet = function(name) {
    return `Hello, ${name}!`;
};

// Arrow function
const greet = name => `Hello, ${name}!`;

// Arrow function with multiple parameters
const add = (a, b) => a + b;

// Arrow function with block
const calculate = (a, b) => {
    const result = a * b;
    return result;
};
\end{lstlisting}
\end{KR}

\begin{KR}{Function Parameters}
\begin{lstlisting}[language=JavaScript, style=basesmol]
// Default parameters
function greet(name = 'Guest') {
    return `Hello, ${name}!`;
}

// Rest parameters
function sum(...numbers) {
    return numbers.reduce((a, b) => a + b, 0);
}

// Destructuring parameters
function printPerson({name, age}) {
    console.log(`${name} is ${age} years old`);
}

// Spread operator
const numbers = [1, 2, 3];
console.log(Math.max(...numbers));
\end{lstlisting}
\end{KR}

\begin{concept}{Closures}
    A closure is created when a function is defined inside another function:
    \begin{itemize}
        \item Has access to variables in outer function scope
        \item Maintains access even after outer function returns
        \item Used for data privacy and state management
    \end{itemize}
\end{concept}

\begin{KR}{Closure Example}
\begin{lstlisting}[language=JavaScript, style=basesmol]
function createCounter() {
    let count = 0;
    return {
        increment() { return ++count; },
        decrement() { return --count; },
        getCount() { return count; }
    };
}

const counter = createCounter();
counter.increment(); // 1
counter.increment(); // 2
counter.decrement(); // 1
\end{lstlisting}
\end{KR}

\subsection{Asynchronous Programming}

\begin{concept}{Asynchronous JavaScript}
    Methods for handling asynchronous operations:
    \begin{itemize}
        \item Callbacks (traditional)
        \item Promises (modern)
        \item Async/Await (modern, cleaner syntax)
    \end{itemize}
\end{concept}

\begin{KR}{Promises}
\begin{lstlisting}[language=JavaScript, style=basesmol]
// Creating a Promise
const myPromise = new Promise((resolve, reject) => {
    // Async operation
    setTimeout(() => {
        if (success) {
            resolve('Operation completed');
        } else {
            reject('Operation failed');
        }
    }, 1000);
});

// Using a Promise
myPromise
    .then(result => console.log(result))
    .catch(error => console.error(error))
    .finally(() => console.log('Cleanup'));

// Async/Await
async function fetchData() {
    try {
        const result = await myPromise;
        console.log(result);
    } catch (error) {
        console.error(error);
    }
}
\end{lstlisting}
\end{KR}

\subsection{Node.js and Modules}

\begin{concept}{Node.js}
    Server-side JavaScript runtime:
    \begin{itemize}
        \item Built on Chrome's V8 engine
        \item Event-driven, non-blocking I/O
        \item Large ecosystem (npm)
        \item Used for web servers, CLI tools, etc.
    \end{itemize}
\end{concept}

\begin{KR}{Module Systems}
\begin{lstlisting}[language=JavaScript, style=basesmol]
// CommonJS (Node.js)
const fs = require('fs');
module.exports = { /* exports */ };

// ES Modules
import { function1 } from './module.js';
export const variable = 42;
export default class MyClass { /* ... */ }

// Package.json
{
    "name": "my-project",
    "version": "1.0.0",
    "dependencies": {
        "express": "^4.17.1"
    }
}
\end{lstlisting}
\end{KR}