\section{Übungsaufgaben}

\subsection{JavaScript Grundlagen}

\begin{example2}{Datentypen und Operatoren}
    \textbf{Aufgabe 1:} Was ist die Ausgabe folgender Ausdrücke?
    \begin{lstlisting}[language=JavaScript, style=basesmol]
typeof NaN
typeof []
typeof null
typeof undefined
[] == false
null === undefined
"5" + 3
"5" - 3
    \end{lstlisting}
    
    \textbf{Lösung:}
    \begin{lstlisting}[language=JavaScript, style=basesmol]
"number"     // NaN ist vom Typ number
"object"     // Arrays sind Objekte
"object"     // null ist historisch ein Objekt
"undefined"  // undefined ist ein eigener Typ
true         // [] wird zu 0 konvertiert
false        // === vergleicht auch Typen
"53"         // String-Konkatenation
2            // Numerische Subtraktion
    \end{lstlisting}
\end{example2}

\begin{example2}{Funktionen und Scoping}
    \textbf{Aufgabe 2:} Was ist die Ausgabe dieses Codes?
    \begin{lstlisting}[language=JavaScript, style=basesmol]
let x = 1;
const f = () => {
    let x = 2;
    return {
        getX: () => x,
        setX: (val) => { x = val; }
    };
};
const obj = f();
console.log(x);        // ?
console.log(obj.getX()); // ?
obj.setX(3);
console.log(obj.getX()); // ?
console.log(x);        // ?
    \end{lstlisting}
    
    \textbf{Lösung:}
    \begin{lstlisting}[language=JavaScript, style=basesmol]
1    // Globales x bleibt 1
2    // Closure hat Zugriff auf lokales x
3    // Lokales x wird auf 3 gesetzt
1    // Globales x bleibt unveraendert
    \end{lstlisting}
\end{example2}

\subsection{DOM und Events}

\begin{example2}{DOM Manipulation}
    \textbf{Aufgabe 3:} Erstellen Sie eine Funktion, die eine ToDo-Liste verwaltet.
    \begin{lstlisting}[language=JavaScript, style=basesmol]
function createTodoList(containerId) {
    // Container finden
    const container = document.getElementById(containerId);
    
    // Input und Liste erstellen
    const input = document.createElement('input');
    const button = document.createElement('button');
    const list = document.createElement('ul');
    
    // Button konfigurieren
    button.textContent = 'Add';
    button.onclick = () => {
        if (input.value.trim()) {
            const li = document.createElement('li');
            li.textContent = input.value;
            list.appendChild(li);
            input.value = '';
        }
    };
    
    // Elemente zusammenfuegen
    container.appendChild(input);
    container.appendChild(button);
    container.appendChild(list);
}
    \end{lstlisting}
\end{example2}

\begin{example2}{Event Handling}
    \textbf{Aufgabe 4:} Implementieren Sie einen Klick-Zähler mit Event Delegation.
    \begin{lstlisting}[language=JavaScript, style=basesmol]
document.getElementById('container').addEventListener('click', 
    (e) => {
        if (e.target.matches('button')) {
            const count = (
                parseInt(e.target.dataset.count) || 0
            ) + 1;
            e.target.dataset.count = count;
            e.target.textContent = `Clicked ${count} times`;
        }
    }
);
    \end{lstlisting}
\end{example2}

\subsection{Client-Server Kommunikation}

\begin{example2}{Fetch API}
    \textbf{Aufgabe 5:} Implementieren Sie eine Funktion für API-Requests.
    \begin{lstlisting}[language=JavaScript, style=basesmol]
async function apiRequest(url, method = 'GET', data = null) {
    const options = {
        method,
        headers: {
            'Content-Type': 'application/json'
        }
    };
    
    if (data) {
        options.body = JSON.stringify(data);
    }
    
    try {
        const response = await fetch(url, options);
        if (!response.ok) {
            throw new Error(`HTTP error: ${response.status}`);
        }
        return await response.json();
    } catch (error) {
        console.error('API request failed:', error);
        throw error;
    }
}
    \end{lstlisting}
\end{example2}

\begin{example2}{Formular-Validierung}
    \textbf{Aufgabe 6:} Erstellen Sie eine Formular-Validierung.
    \begin{lstlisting}[language=JavaScript, style=basesmol]
function validateForm(formId) {
    const form = document.getElementById(formId);
    
    form.addEventListener('submit', (e) => {
        e.preventDefault();
        
        const formData = new FormData(form);
        const errors = [];
        
        // Email validieren
        const email = formData.get('email');
        if (!email.includes('@')) {
            errors.push('Invalid email');
        }
        
        // Passwort validieren
        const password = formData.get('password');
        if (password.length < 8) {
            errors.push('Password too short');
        }
        
        if (errors.length === 0) {
            // Form submission logic
            console.log('Form valid, submitting...');
            form.submit();
        } else {
            alert(errors.join('\n'));
        }
    });
}
    \end{lstlisting}
\end{example2}

\subsection{UI-Komponenten}

\begin{example2}{SuiWeb Komponente}
    \textbf{Aufgabe 7:} Erstellen Sie eine Counter-Komponente mit SuiWeb.
    \begin{lstlisting}[language=JavaScript, style=basesmol]
const Counter = () => {
    const [count, setCount] = useState(0);
    
    return [
        "div",
        ["h2", `Count: ${count}`],
        ["button", 
            {onclick: () => setCount(count + 1)},
            "Increment"
        ],
        ["button", 
            {onclick: () => setCount(count - 1)},
            "Decrement"
        ]
    ];
};
    \end{lstlisting}
\end{example2}

\begin{example2}{Container Component}
    \textbf{Aufgabe 8:} Implementieren Sie eine UserList-Komponente.
    \begin{lstlisting}[language=JavaScript, style=basesmol]
const UserList = () => {
    const [users, setUsers] = useState([]);
    const [loading, setLoading] = useState(true);
    
    if (loading) {
        fetchUsers()
            .then(data => {
                setUsers(data);
                setLoading(false);
            })
            .catch(error => {
                console.error(error);
                setLoading(false);
            });
    }
    
    if (loading) {
        return ["div", "Loading..."];
    }
    
    return [
        "div",
        ["h2", "Users"],
        ["ul", 
            ...users.map(user => 
                ["li", `${user.name} (${user.email})`)
            ]
        ]
    ];
};
    \end{lstlisting}
\end{example2}

\subsection{Theoriefragen}

\begin{example2}{Konzeptfragen}
    \textbf{1. Erklären Sie den Unterschied zwischen == und === in JavaScript.}
    
    Antwort: == vergleicht Werte mit Typumwandlung, === vergleicht Werte und Typen ohne Umwandlung.
    
    \textbf{2. Was ist Event Bubbling?}
    
    Antwort: Events werden von dem auslösenden Element durch den DOM-Baum nach oben weitergeleitet.
    
    \textbf{3. Was ist der Unterschied zwischen localStorage und sessionStorage?}
    
    Antwort: localStorage persistiert Daten auch nach Schließen des Browsers, sessionStorage nur während der Session.
    
    \textbf{4. Erklären Sie den Unterschied zwischen synchronem und asynchronem Code.}
    
    Antwort: Synchroner Code wird sequentiell ausgeführt, asynchroner Code ermöglicht parallele Ausführung ohne Blockierung.
\end{example2}

\subsection{Praktische Aufgaben}

\begin{example2}{Implementierungsaufgaben}
    \textbf{1. Implementieren Sie eine Funktion zur Deep Copy von Objekten.}
    
    \textbf{2. Erstellen Sie eine Funktion, die prüft ob ein String ein Palindrom ist.}
    
    \textbf{3. Implementieren Sie eine debounce-Funktion.}
    
    \textbf{4. Erstellen Sie eine Komponente für einen Image Slider.}
\end{example2}

\begin{example2}{Debugging-Aufgaben}
    \textbf{1. Finden Sie den Fehler im folgenden Code:}
    \begin{lstlisting}[language=JavaScript, style=basesmol]
const getData = () => {
    fetch('api/data')
        .then(response => response.json())
        .then(data => {
            return data;
        });
}
// Warum kommt undefined zurueck?
    \end{lstlisting}
    
    Antwort: Die Funktion hat kein explizites return Statement. Sie sollte entweder async/await verwenden oder die Promise zurückgeben.
\end{example2}
