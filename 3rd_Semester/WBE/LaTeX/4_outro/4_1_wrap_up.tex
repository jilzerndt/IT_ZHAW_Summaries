\section{Wrap-up}

\subsection{React vs SuiWeb}
\begin{concept}{React.js Key Features}
    \begin{itemize}
        \item JavaScript library for building user interfaces
        \item Released by Facebook in 2013
        \item Key principles:
            \begin{itemize}
                \item Declarative
                \item Component-Based
                \item Learn Once, Write Anywhere
                \item Virtual DOM for efficient rendering
                \item Unidirectional data flow
            \end{itemize}
    \end{itemize}
\end{concept}

\begin{KR}{React Components}
Component patterns frequently used in React:
\begin{enumerate}
    \item Function Components (modern approach):
        \begin{verbatim}
const HelloComponent = (props) => {
    return (<div>Hello {props.name}</div>)
}
        \end{verbatim}
    \item Class Components (legacy):
        \begin{verbatim}
class HelloComponent extends React.Component {
    render() {
        return <div>Hello {this.props.name}</div>
    }
}
        \end{verbatim}
    \item State Management with Hooks:
        \begin{verbatim}
const Counter = () => {
    const [state, setState] = useState(1)
    const handler = () => setState(state + 1)
    return (
        <button onClick={handler}>Count: {state}</button>
    )
}
        \end{verbatim}
\end{enumerate}
\end{KR}

\subsection{Course Overview} 
\begin{formula}{Main Topics}
    \begin{enumerate}
        \item JavaScript Core Concepts
            \begin{itemize}
                \item Basic language features
                \item Objects and Arrays
                \item Functions and Prototypes 
                \item Asynchronous Programming
            \end{itemize}
        \item Browser Technologies
            \begin{itemize}
                \item DOM Manipulation
                \item Events
                \item Web APIs
            \end{itemize}
        \item Client-Side Applications
            \begin{itemize}
                \item Component Design
                \item UI Libraries/Frameworks
                \item Modern Web Development
            \end{itemize}
    \end{enumerate}
\end{formula}

\subsection{Advanced Web Topics}
\begin{concept}{Further Areas of Web Development}
    \begin{itemize}
        \item Mobile Development
            \begin{itemize}
                \item Responsive Design
                \item React Native
                \item Progressive Web Apps
            \end{itemize}
        \item Performance Optimization
            \begin{itemize}
                \item WebAssembly (WASM)
                \item Code Splitting
                \item Client-side Caching
            \end{itemize}
        \item Alternative Technologies
            \begin{itemize}
                \item TypeScript
                \item WebAssembly 
                \item Functional Programming
            \end{itemize}
    \end{itemize}
\end{concept}

\subsection{Course Goals}
\begin{concept}{Key Learning Outcomes}
    \begin{itemize}
        \item Solid understanding of JavaScript fundamentals
        \item Proficiency in core web technologies
        \item Understanding of modern web development practices
        \item Foundation for advanced web development topics
    \end{itemize}
\end{concept}

\begin{remark}{Depth of Topics}
Topics covered in the course are divided into:
\begin{itemize}
    \item In-depth topics: Require thorough understanding and practical application
    \item Overview topics: Require basic understanding of concepts and use cases
\end{itemize}
Detailed lists of what falls into each category are provided in the course materials.
\end{remark}

\subsection{Web Development Landscape}
\begin{concept}{Current State of Web Development}
    \begin{itemize}
        \item JavaScript remains dominant web programming language
        \item React continues as leading UI library
        \item Growing ecosystem of tools and frameworks
        \item Increasing focus on performance and user experience
        \item Emergence of new patterns and technologies
    \end{itemize}
\end{concept}

\subsection{Weekly Schedule}
\begin{table}[h]
\begin{tabular}{|l|l|}
\hline
\textbf{Week} & \textbf{Topic} \\
\hline
1 & Introduction, Web Overview \\
2 & JavaScript Basics \\
3 & Objects and Arrays \\
4 & Functions \\
5 & Prototypes \\
6 & Async Programming \\
7 & Web Servers \\
8-9 & Browser JavaScript \\
10 & Client-Server Interaction \\
11-13 & UI Library Development \\
14 & Conclusion, React \\
\hline
\end{tabular}
\end{table}

\subsection{Additional Resources}
For further learning:
\begin{itemize}
    \item HTML/CSS: \texttt{https://developer.mozilla.org/docs/Web}
    \item React: \texttt{https://react.dev}
    \item JavaScript: \texttt{https://eloquentjavascript.net}
    \item TypeScript: \texttt{https://www.typescriptlang.org}
    \item WebAssembly: \texttt{https://webassembly.org}
\end{itemize}