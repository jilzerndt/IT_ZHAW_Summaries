\section{React.js and Web Development}

\subsection{From SuiWeb to React.js}
SuiWeb is an experimental library designed to teach React.js concepts. React.js is "A JavaScript library for building user interfaces" with key characteristics:
\begin{itemize}
    \item Declarative
    \item Component-Based
    \item Learn Once, Write Anywhere
\end{itemize}

Released by Facebook in 2013, React consists of:
\begin{itemize}
    \item React DOM - Performs actual rendering on web page
    \item React Component API - Handles data, lifecycle methods, events and JSX syntax
\end{itemize}

\subsection{Components and Classes}
Components can be written in multiple ways:

\begin{lstlisting}[language=JavaScript]
// ES5
var HelloComponent = React.createClass({
    render: function() {
        return <div>Hello {this.props.name}</div>
    }
})

// ES6
class HelloComponent extends React.Component {
    render() {
        return <div>Hello {this.props.name}</div>
    }
}

// Function Component
const HelloComponent = (props) => {
    return (<div>Hello {props.name}</div>)
}
\end{lstlisting}

\subsection{Component Examples}
Example of component composition:
\begin{lstlisting}[language=JavaScript]
const MyComponent = () => (
    <section>
        <h1>My Component</h1>
        <List data={["Maria", "Hans", "Eva", "Peter"]} />
    </section>
)

const List = ({data}) => (
    <ul>
        { data.map(item => (<li key={item}>{item}</li>)) }
    </ul>
)
\end{lstlisting}

\subsection{State Management}
Example of state management in React:
\begin{lstlisting}[language=JavaScript]
const Counter = () => {
    const [state, setState] = useState(1)
    const handler = () => setState(c => c + 1)
    return (
        <h1 onclick={handler}>Count {state}</h1>
    )
}
\end{lstlisting}

\subsection{Key React Features}
\begin{itemize}
    \item Supports both function and class components
    \item Function components with Hooks
    \item Virtual DOM optimizations
    \item Developer tools (React DevTools)
    \item Server-side and client-side rendering
    \item React Native support for iOS/Android apps
\end{itemize}

\subsection{React Concepts}
\begin{itemize}
    \item React forms the View layer of applications
    \item Component-based architecture
    \item Virtual DOM for efficient rendering
    \item Small, focused component units
    \item Clear one-way data flow
    \item Props down, events up pattern
\end{itemize}

\section{Web Course Overview}

\subsection{Core Topics}
\begin{enumerate}
    \item JavaScript and Node.js fundamentals
    \item JavaScript in Browser environment
    \item Client-side Web Applications
\end{enumerate}

\subsection{Additional Web Topics}
\begin{itemize}
    \item HTML and CSS fundamentals
    \item Mobile web development
    \item Responsive design
    \item Web APIs and sensors
    \item React Native / Expo
    \item WebAssembly (WASM)
    \item TypeScript and alternatives
    \item Desktop applications with web technologies
\end{itemize}

\subsection{Course Structure}
\begin{center}
\begin{tabular}{|c|l|}
\hline
Week & Topic \\
\hline
1 & Introduction, Web Overview \\
2 & JavaScript Basics \\
3 & Objects and Arrays \\
4 & Functions \\
5 & Object Prototypes \\
6 & Asynchronous Programming \\
7 & Webserver \\
8-9 & Browser JavaScript \\
10 & Client-Server Interaction \\
11-13 & UI Library: Components, Implementation, Usage \\
14 & React, Feedback \\
\hline
\end{tabular}
\end{center}

\subsection{Course Goals}
Primary objectives:
\begin{itemize}
    \item Solid understanding of core web technologies
    \item Proficiency in JavaScript as the web's programming language
    \item Understanding of HTML and CSS fundamentals
    \item Overview of modern web development landscape
\end{itemize}

\section{JavaScript Development Tools}

\subsection{Node.js Environment}
\begin{itemize}
    \item Runtime environment for JavaScript
    \item NPM package management
    \item Module system
    \item Command-line tools
\end{itemize}

\subsection{Browser Development}
\begin{itemize}
    \item Browser Developer Tools
    \item DOM Manipulation
    \item Event Handling
    \item AJAX and Fetch API
\end{itemize}

\subsection{Testing and Quality}
\begin{itemize}
    \item Test-Driven Development
    \item Unit Testing
    \item Integration Testing
    \item Code Quality Tools
\end{itemize}