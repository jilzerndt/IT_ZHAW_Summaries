\section{UI-Bibliotheken und Komponenten}

\subsection{Frameworks und Bibliotheken}

\begin{definition}{Framework vs. Bibliothek}
    \begin{itemize}
        \item \textbf{Bibliothek}: Kontrolle beim eigenen Programm, Funktionen werden nach Bedarf verwendet
        \item \textbf{Framework}: Rahmen für die Anwendung, Kontrolle liegt beim Framework ("Hollywood-Prinzip")
    \end{itemize}
\end{definition}

\begin{concept}{Komponenten-basierte Entwicklung}
    Grundprinzipien:
    \begin{itemize}
        \item UI wird in wiederverwendbare Komponenten aufgeteilt
        \item Komponenten können verschachtelt werden
        \item Datenfluss von oben nach unten (props)
        \item Zustand wird in Komponenten verwaltet
        \item Deklarativer Ansatz: UI als Funktion des Zustands
    \end{itemize}
\end{concept}

\subsection{JSX und SJDON}

\begin{definition}{JSX}
    JSX ist eine Erweiterung der JavaScript-Syntax:
    \begin{itemize}
        \item Erlaubt HTML-ähnliche Syntax in JavaScript
        \item Muss zu JavaScript kompiliert werden
        \item Wird hauptsächlich in React verwendet
    \end{itemize}
\end{definition}

\begin{KR}{JSX Syntax}
\begin{lstlisting}[language=JavaScript, style=basesmol]
// JSX Komponente
const Welcome = ({name}) => (
    <div className="welcome">
        <h1>Hello, {name}</h1>
        <p>Welcome to our site!</p>
    </div>
);

// Verwendung
const element = <Welcome name="Alice" />;
\end{lstlisting}
\end{KR}

\begin{definition}{SJDON}
    Simple JavaScript DOM Notation:
    \begin{itemize}
        \item Alternative zu JSX
        \item Verwendet reine JavaScript Arrays und Objekte
        \item Kein Kompilierungsschritt nötig
    \end{itemize}
\end{definition}

\begin{KR}{SJDON Syntax}
\begin{lstlisting}[language=JavaScript, style=basesmol]
// SJDON Komponente
const Welcome = ({name}) => [
    "div", {className: "welcome"},
    ["h1", `Hello, ${name}`],
    ["p", "Welcome to our site!"]
];

// Verwendung
const element = [Welcome, {name: "Alice"}];
\end{lstlisting}
\end{KR}

\subsection{SuiWeb Framework}

\begin{concept}{SuiWeb Grundkonzepte}
    Simple User Interface Toolkit for Web Exercises:
    \begin{itemize}
        \item Komponentenbasiert wie React
        \item Unterstützt JSX und SJDON
        \item Datengesteuert mit Props und State
        \item Vereinfachte Implementierung für Lernzwecke
    \end{itemize}
\end{concept}

\begin{KR}{Komponenten in SuiWeb}
\begin{lstlisting}[language=JavaScript, style=basesmol]
// Einfache Komponente
const MyButton = ({onClick, children}) => [
    "button",
    {
        onclick: onClick,
        style: "background: khaki"
    },
    ...children
];

// Komponente mit State
const Counter = () => {
    const [count, setCount] = useState(0);
    
    return [
        "div",
        ["button", 
            {onclick: () => setCount(count + 1)},
            `Count: ${count}`
        ]
    ];
};
\end{lstlisting}
\end{KR}

\begin{KR}{Container Komponenten}
\begin{lstlisting}[language=JavaScript, style=basesmol]
const MyContainer = () => {
    let initialState = { items: ["Loading..."] };
    let [state, setState] = useState(initialState);
    
    if (state === initialState) {
        fetchData()
            .then(items => setState({items}));
    }
    
    return [
        MyList, 
        {items: state.items}
    ];
};
\end{lstlisting}
\end{KR}

\begin{formula}{State Management}
    Zustandsverwaltung in SuiWeb:
    \begin{itemize}
        \item \texttt{useState} Hook für lokalen Zustand
        \item State Updates lösen Re-Rendering aus
        \item Asynchrone Updates werden gequeued
        \item Props sind read-only
    \end{itemize}
\end{formula}

\begin{concept}{Styling in SuiWeb}
    Verschiedene Möglichkeiten für Styles:
    \begin{itemize}
        \item Inline Styles als Strings
        \item Style-Objekte
        \item Arrays von Style-Objekten
        \item Externe CSS-Klassen
    \end{itemize}
\end{concept}

\begin{KR}{Styling Beispiele}
\begin{lstlisting}[language=JavaScript, style=basesmol]
// String Style
["div", {style: "color: blue; font-size: 16px"}]

// Style Objekt
const styles = {
    container: {
        backgroundColor: "lightgray",
        padding: "10px"
    },
    text: {
        color: "darkblue",
        fontSize: "14px"
    }
};

// Kombinierte Styles
["div", {
    style: [
        styles.container,
        {borderRadius: "5px"}
    ]
}]
\end{lstlisting}
\end{KR}

\subsection{Komponenten-Design}

\begin{theorem}{Best Practices}
    Grundprinzipien für gutes Komponenten-Design:
    \begin{itemize}
        \item Single Responsibility Principle
        \item Trennung von Container und Präsentation
        \item Vermeidung von tiefer Verschachtelung
        \item Wiederverwendbarkeit fördern
        \item Klare Props-Schnittstelle
    \end{itemize}
\end{theorem}

\begin{KR}{Komponenten-Struktur}
\begin{lstlisting}[language=JavaScript, style=basesmol]
// Container Komponente
const UserContainer = () => {
    const [user, setUser] = useState(null);
    
    useEffect(() => {
        fetchUser().then(setUser);
    }, []);
    
    return [UserProfile, {user}];
};

// Praesentations-Komponente
const UserProfile = ({user}) => {
    if (!user) return ["div", "Loading..."];
    
    return [
        "div",
        ["h2", user.name],
        ["p", user.email],
        [UserDetails, {details: user.details}]
    ];
};
\end{lstlisting}
\end{KR}

\begin{concept}{Event Handling}
    Behandlung von Benutzerinteraktionen:
    \begin{itemize}
        \item Events als Props übergeben
        \item Callback-Funktionen für Events
        \item State Updates in Event Handlern
        \item Vermeidung von direkter DOM-Manipulation
    \end{itemize}
\end{concept}

\begin{KR}{Event Handling Beispiel}
\begin{lstlisting}[language=JavaScript, style=basesmol]
const TodoList = () => {
    const [todos, setTodos] = useState([]);
    
    const addTodo = (text) => {
        setTodos([...todos, {
            id: Date.now(),
            text,
            completed: false
        }]);
    };
    
    const toggleTodo = (id) => {
        setTodos(todos.map(todo =>
            todo.id === id
                ? {...todo, completed: !todo.completed}
                : todo
        ));
    };
    
    return [
        "div",
        [TodoForm, {onSubmit: addTodo}],
        [TodoItems, {
            items: todos,
            onToggle: toggleTodo
        }]
    ];
};
\end{lstlisting}
\end{KR}

\begin{corollary}{Optimierungen}
    Möglichkeiten zur Performanzverbesserung:
    \begin{itemize}
        \item Virtuelles DOM für effizientes Re-Rendering
        \item Batching von State Updates
        \item Memoization von Komponenten
        \item Lazy Loading von Komponenten
    \end{itemize}
\end{corollary}
