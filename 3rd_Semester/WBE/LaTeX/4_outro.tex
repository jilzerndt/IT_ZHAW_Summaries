\section{Wrap-up}

\subsection{Überblick des Kurses}

\begin{formula}{Hauptthemen}
    \begin{enumerate}
        \item JavaScript Grundlagen
            \begin{itemize}
                \item Sprache und Syntax
                \item Objekte und Arrays
                \item Funktionen und Prototypen 
                \item Asynchrone Programmierung
                \item Node.js und Module
            \end{itemize}
        \item Browser-Technologien
            \begin{itemize}
                \item DOM Manipulation
                \item Events und Event Handling
                \item Web Storage
                \item Canvas und SVG
                \item Client-Server Kommunikation
            \end{itemize}
        \item UI-Bibliotheken
            \begin{itemize}
                \item Komponentenbasierte Entwicklung
                \item JSX und SJDON
                \item State Management
                \item SuiWeb Framework
            \end{itemize}
    \end{enumerate}
\end{formula}

\subsection{Von SuiWeb zu React}

\begin{concept}{React.js Kernkonzepte}
    \begin{itemize}
        \item JavaScript-Bibliothek für User Interfaces
        \item Entwickelt von Facebook (2013)
        \item Hauptprinzipien:
            \begin{itemize}
                \item Deklarativ
                \item Komponentenbasiert
                \item Learn Once, Write Anywhere
                \item Virtual DOM für effizientes Rendering
                \item Unidirektionaler Datenfluss
            \end{itemize}
    \end{itemize}
\end{concept}

\begin{KR}{React Components}
\begin{lstlisting}[language=JavaScript, style=basesmol]
// Function Component
const Welcome = ({name}) => {
    return <h1>Hello, {name}</h1>;
};

// State Hook
const Counter = () => {
    const [count, setCount] = useState(0);
    
    return (
        <div>
            <p>Count: {count}</p>
            <button onClick={() => setCount(count + 1)}>
                Increment
            </button>
        </div>
    );
};

// Effect Hook
const DataFetcher = () => {
    const [data, setData] = useState(null);
    
    useEffect(() => {
        fetchData().then(setData);
    }, []);
    
    return data ? <DisplayData data={data} /> : <Loading />;
};
\end{lstlisting}
\end{KR}

\subsection{Weiterführende Themen}

\begin{concept}{Modern Web Development}
    \begin{itemize}
        \item Mobile Development
            \begin{itemize}
                \item Responsive Design
                \item Progressive Web Apps
                \item React Native
            \end{itemize}
        \item Performance
            \begin{itemize}
                \item WebAssembly (WASM)
                \item Code Splitting
                \item Service Workers
            \end{itemize}
        \item Alternative Technologien
            \begin{itemize}
                \item TypeScript
                \item Svelte
                \item Vue.js
            \end{itemize}
    \end{itemize}
\end{concept}

\begin{formula}{JavaScript Ecosystem}
    Wichtige Tools und Frameworks:
    \begin{itemize}
        \item \textbf{Build Tools:}
            \begin{itemize}
                \item Webpack
                \item Vite
                \item Babel
            \end{itemize}
        \item \textbf{Testing:}
            \begin{itemize}
                \item Jest
                \item Testing Library
                \item Cypress
            \end{itemize}
        \item \textbf{State Management:}
            \begin{itemize}
                \item Redux
                \item MobX
                \item Zustand
            \end{itemize}
    \end{itemize}
\end{formula}

\begin{KR}{Best Practices}
    Wichtige Prinzipien für die Web-Entwicklung:
    \begin{itemize}
        \item Clean Code
            \begin{itemize}
                \item DRY (Don't Repeat Yourself)
                \item KISS (Keep It Simple, Stupid)
                \item Single Responsibility Principle
            \end{itemize}
        \item Performance
            \begin{itemize}
                \item Lazy Loading
                \item Code Splitting
                \item Caching Strategien
            \end{itemize}
        \item Security
            \begin{itemize}
                \item HTTPS
                \item CORS
                \item Content Security Policy
            \end{itemize}
    \end{itemize}
\end{KR}

\subsection{Ressourcen}

\begin{concept}{Weiterführende Materialien}
    \begin{itemize}
        \item \textbf{Dokumentation:}
            \begin{itemize}
                \item MDN Web Docs: \url{https://developer.mozilla.org}
                \item React Docs: \url{https://react.dev}
                \item Node.js Docs: \url{https://nodejs.org/docs}
            \end{itemize}
        \item \textbf{Bücher:}
            \begin{itemize}
                \item "Eloquent JavaScript" von Marijn Haverbeke
                \item "You Don't Know JS" von Kyle Simpson
                \item "JavaScript: The Good Parts" von Douglas Crockford
            \end{itemize}
        \item \textbf{Online Kurse:}
            \begin{itemize}
                \item freeCodeCamp
                \item Frontend Masters
                \item Egghead.io
            \end{itemize}
    \end{itemize}
\end{concept}

\begin{theorem}{Kursabschluss}
    Wichtige Lernergebnisse:
    \begin{itemize}
        \item Solides Verständnis von JavaScript
        \item Beherrschung der Browser-APIs
        \item Komponentenbasierte Entwicklung
        \item Moderne Web-Entwicklungspraktiken
        \item Basis für fortgeschrittene Themen
    \end{itemize}
\end{theorem}