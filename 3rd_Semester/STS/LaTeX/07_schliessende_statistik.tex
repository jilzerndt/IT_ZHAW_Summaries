\section{Schliessende Statistik}
\begin{definition}{Erwartungstreue Schätzfunktion}
Eine Schätzfunktion $\Theta$ eines Parameters $\theta$ heisst erwartungstreu, wenn:
$$
E(\Theta)=\theta
$$
\end{definition}

\begin{definition}{Effizienz einer Schätzfunktion}
Gegeben sind zwei erwartungstreue Schätzfunktionen $\Theta_1$ und $\Theta_2$ desselben Parameters $\theta$. Man nennt $\Theta_1$ effizienter als $\Theta_2$, falls:
$$
V(\Theta_1)<V(\Theta_2)
$$
\end{definition}

\begin{definition}{Konsistenz einer Schätzfunktion}
Eine Schätzfunktion $\Theta$ heisst konsistent, wenn:
$$
E(\Theta) \rightarrow \theta \text{ und } V(\Theta) \rightarrow 0 \text{ für } n \rightarrow \infty
$$
\end{definition}

\section{Vertrauensintervalle}
\begin{definition}{Vertrauensintervall}
Wir legen eine grosse Wahrscheinlichkeit $\gamma$ fest (z.B. $\gamma=95\%$). $\gamma$ heisst statistische Sicherheit oder Vertrauensniveau. $\alpha=1-\gamma$ ist die Irrtumswahrscheinlichkeit.

Dann bestimmen wir zwei Zufallsvariablen $\Theta_u$ und $\Theta_o$ so, dass sie den wahren Parameterwert $\Theta$ mit der Wahrscheinlichkeit $\gamma$ einschliessen:
$$
P(\Theta_u \leq \Theta \leq \Theta_o)=\gamma
$$
\end{definition}

\begin{concept}{Intervallschätzung}
Verteilungstypen und zugehörige Quantile:
\begin{center}
		\resizebox{\columnwidth}{!}{
\begin{tabular}{|c|c|c|}
\hline
Verteilung & Parameter & Quantile \\
\hline
Normalverteilung ($\sigma^2$ bekannt) & $\mu$ & $c=u_p, p=\frac{1+\gamma}{2}$ \\
\hline
t-Verteilung ($\sigma^2$ unbekannt) & $\mu$ & $c=t_{(p;f=n-1)}, p=\frac{1+\gamma}{2}$ \\
\hline
Chi-Quadrat-Verteilung & $\sigma^2$ & $c_1=\chi^2_{(\frac{1-\gamma}{2};n-1)}, c_2=\chi^2_{(\frac{1+\gamma}{2};n-1)}$ \\
\hline
\end{tabular}
}
\end{center}
\end{concept}

\begin{example}{Berechnung eines Vertrauensintervalls}
Geben Sie das Vertrauensintervall für $\mu$ an ($\sigma^2$ unbekannt). Gegeben sind:
$$
n=10, \quad \bar{x}=102, \quad s^2=16, \quad \gamma=0.99
$$

\begin{enumerate}
  \item Verteilungstyp mit Param $\mu$ und $\sigma^2$ unbekannt $\rightarrow$ T-Verteilung
  \item $f=n-1=9$, $p=\frac{1+\gamma}{2}=0.995$, $c=t_{(p;f)}=t_{(0.995;9)}=3.25$
  \item $e=c \cdot \frac{S}{\sqrt{n}}=4.111$, $\Theta_u=\bar{X}-e=97.89$, $\Theta_o=\bar{X}+e=106.11$
\end{enumerate}
\end{example}

\section{Likelyhood-Funktion}
\begin{definition}{Likelyhood-Funktion}
Wir betrachten eine Zufallsvariable $X$ und ihre Dichte (PDF) $f_x(x|\theta)$, welche von $x$ und einem oder mehreren Parametern $\theta$ abhängig sind. 

Für eine Stichprobe vom Umfang $n$ mit $x_1,\ldots,x_n$ nennen wir die vom Parameter $\theta$ abhängige Funktion die Likelyhood-Funktion der Stichprobe:
$$
L(\theta)=f_x(x_1|\theta) \cdot f_x(x_2|\theta) \cdot \ldots \cdot f_x(x_n|\theta)
$$
\end{definition}

\begin{concept}{Vorgehen bei Maximum-Likelihood-Schätzung}
\begin{enumerate}
  \item Likelyhood-Funktion bestimmen
  \item Maximalstelle der Funktion bestimmen:
        \begin{itemize}
           \item (Partielle) Ableitung $L'(\theta)=0$
        \end{itemize}
\end{enumerate}
\end{concept}
