\section{Wahrscheinlichkeitsrechnung}
\begin{concept}{Ideen}\\
\begin{itemize}
 \item Berechnung durch Aufteilung in mehrere Kombinationen
 \item Berechnung über Inverse
 \item Prozente = Wahrscheinlichkeit / Gesamt-Wahrscheinlichkeit
\end{itemize}
\end{concept}

\begin{example}{Wahrscheinlichkeit bei Rommé}\\
Beim Rommé spielt man mit \emph{110 Karten: sechs} davon sind \emph{Joker}. Zu Beginn eines Spiels erhält jeder Spieler genau \emph{12 Karten}.

In wieviel Prozent aller möglichen Fälle sind darunter \emph{genau zwei} Joker?
$$\frac{\binom{6}{2} \cdot \binom{104}{10}}{\binom{110}{12}}$$

In wieviel Prozent aller möglichen Fälle ist darunter \emph{mindestens ein} Joker?
$$1 - \frac{\binom{104}{12}}{\binom{110}{12}}$$
\end{example}

\begin{example}{Geschwister und Geburtsmonat}\\
Sind in mehr als 60\% aller Fälle von vier (nicht gleichaltrigen) Geschwistern mindestens zwei im gleichen Monat geboren?
$$1 - \frac{12 \cdot 11 \cdot 10 \cdot 9}{12^4}$$
\end{example}

\begin{example}{Anordnung von Büchern}\\
Auf wie viele Arten lassen sich 10 Bücher in ein Regal reihen?
$$n = 10, \quad k = 10$$
$$\frac{n!}{(n-k)!} = 10!$$
\end{example}

\begin{example}{Glühbirnen auswählen}\\
Von \emph{100 Glühbirnen} sind genau \emph{drei defekt}. Es werden nun \emph{6 Glühbirnen} zufällig ausgewählt.

Wie viele Möglichkeiten gibt es, wenn sich \emph{mindestens eine defekte} Glühbirne in der Auswahl befinden soll?
$$\binom{100}{6} - \binom{97}{6} = 203'880'032$$

Mit wie viel Prozent Chancen ist bei einer Auswahl von 6 Glühbirnen \emph{keine defekt}?
$$\frac{\binom{97}{6}}{\binom{100}{6}}$$
\end{example}

\begin{example}{Buchstabenkombinationen}\\
Wie viele Worte lassen sich aus den Buchstaben des Wortes ABRAKADABRA bilden? (Nur Worte in denen alle Buchstaben vorkommen!)

$A = 5x, \quad B = 2x, \quad R = 2x, \quad D = 1x, \quad K = 1x$
$$\binom{11}{5} \cdot \binom{6}{2} \cdot \binom{4}{2} \cdot \binom{2}{1} \cdot \binom{1}{1} = 83160$$
\end{example}

\subsection{Wahrscheinlichkeitstheorie}
\begin{definition}{Ergebnisraum}\\
Ergebnisraum $\Omega$ ist die Menge aller möglichen Ergebnisse des Zufallsexperiments. Zähldichte $\rho: \Omega \rightarrow[0,1]$ ordnet jedem Ereignis seine Wahrscheinlichkeit zu.

Für ein Laplace-Raum $(\Omega, P)$ gilt:
$$
P(M)=\frac{|M|}{|\Omega|}
$$
\\
$\Omega$ = Ergebnisraum (Menge aller möglichen Ergebnisse)\\
$P(M)$ = Wahrscheinlichkeit des Ereignisses $M$\\
$|M|$ = Anzahl der für $M$ günstigen Ergebnisse\\
$|\Omega|$ = Anzahl aller möglichen Ergebnisse\\
\end{definition}

\begin{theorem}{Stochastische Unabhängigkeit}\\
Zwei Ereignisse $A$ und $B$ heissen stochastisch unabhängig, falls:
$$
P(A \cap B)=P(A) \cdot P(B)
$$
\\
$P(A \cap B)$ = Wahrscheinlichkeit dass beide Ereignisse eintreten\\
$P(A)$ = Wahrscheinlichkeit von Ereignis $A$\\
$P(B)$ = Wahrscheinlichkeit von Ereignis $B$\\

Zwei Zufallsvariablen $X: \Omega \rightarrow \mathbb{R}$ und $Y: \Omega \rightarrow \mathbb{R}$ heissen stochastisch unabhängig, falls:
$$
P(X=x, Y=y)=P(X=x) \cdot P(Y=y), \quad \text{für alle } x, y \in \mathbb{R}
$$
\\
$P(X=x, Y=y)$ = Wahrscheinlichkeit dass $X$ den Wert $x$ und $Y$ den Wert $y$ annimmt\\
$P(X=x)$ = Wahrscheinlichkeit dass $X$ den Wert $x$ annimmt\\
$P(Y=y)$ = Wahrscheinlichkeit dass $Y$ den Wert $y$ annimmt\\
\end{theorem}

\subsection{Bedingte Wahrscheinlichkeit}
\begin{definition}{Bedingte Wahrscheinlichkeit}\\
$$
P(B \mid A)=\frac{P(B \cap A)}{P(A)}
$$
\\
$P(B|A)$ = Wahrscheinlichkeit von $B$ unter der Bedingung dass $A$ eingetreten ist\\
$P(B \cap A)$ = Wahrscheinlichkeit dass beide Ereignisse eintreten\\
$P(A)$ = Wahrscheinlichkeit von Ereignis $A$\\
\end{definition}

\begin{theorem}{Multiplikationssatz}\\
$$
P(A \cap B)=P(A) \cdot P(B \mid A)=P(B) \cdot P(A \mid B)
$$
\\
$P(A \cap B)$ = Wahrscheinlichkeit dass beide Ereignisse eintreten\\
$P(A)$ = Wahrscheinlichkeit von Ereignis $A$\\
$P(B|A)$ = Wahrscheinlichkeit von $B$ unter der Bedingung dass $A$ eingetreten ist\\
$P(A|B)$ = Wahrscheinlichkeit von $A$ unter der Bedingung dass $B$ eingetreten ist\\
\end{theorem}
\begin{concept}{Kenngrössen (Varianz und Erwartungswert)}\\
$$E(X + Y) = E(X) + E(Y), \quad E(\alpha X) = \alpha E(X)$$
$$V(X) = E(X^2) - E(X)^2 = \left[\sum_{x\in\mathbb{R}} P(X = x) \cdot x^2\right] - E(X)^2$$
$$V(\alpha X + \beta) = \alpha^2 \cdot V(X), \quad S(X) = \sqrt{V(X)}$$
\\
$E(X)$ = Erwartungswert der Zufallsvariable $X$\\
$V(X)$ = Varianz der Zufallsvariable $X$\\
$S(X)$ = Standardabweichung der Zufallsvariable $X$\\
$\alpha, \beta$ = Konstanten\\
$P(X = x)$ = Wahrscheinlichkeit, dass $X$ den Wert $x$ annimmt\\
$\sum_{x\in\mathbb{R}}$ = Summe über alle möglichen Werte von $x$ in den reellen Zahlen\\
\end{concept}
\begin{definition}{Verteilungen und Erwartungswerte}\\
Für diskrete Verteilungen:
$$
\begin{gathered}
E(X)=\sum_{x \in \mathbb{R}} f(x) \cdot x \\
V(X)=\sum_{x \in \mathbb{R}} f(x) \cdot(x-E(X))^2
\end{gathered}
$$
$E(X)$ = Erwartungswert der Zufallsvariable $X$\\
$V(X)$ = Varianz der Zufallsvariable $X$\\
$f(x)$ = Wahrscheinlichkeitsfunktion\\
$x$ = Mögliche Werte der Zufallsvariable\\

Für stetige Verteilungen:
$$
\begin{gathered}
E(X)=\int_{-\infty}^{\infty} f(x) \cdot x dx \\
V(X)=\int_{-\infty}^{\infty} f(x) \cdot(x-E(X))^2 dx
\end{gathered}
$$
$E(X)$ = Erwartungswert der Zufallsvariable $X$\\
$V(X)$ = Varianz der Zufallsvariable $X$\\
$f(x)$ = Dichtefunktion\\
$x$ = Mögliche Werte der Zufallsvariable\\
\end{definition}
\begin{theorem}{Satz von der Totalen Wahrscheinlichkeit}\\
$$
P(B)=P(A) \cdot P(B \mid A)+P(\bar{A}) \cdot P(B \mid \bar{A})
$$
\\
$P(B)$ = Wahrscheinlichkeit von Ereignis $B$\\
$P(A)$ = Wahrscheinlichkeit von Ereignis $A$\\
$P(\bar{A})$ = Wahrscheinlichkeit des Gegenereignisses von $A$\\
$P(B|A)$ = Wahrscheinlichkeit von $B$ unter der Bedingung dass $A$ eingetreten ist\\
$P(B|\bar{A})$ = Wahrscheinlichkeit von $B$ unter der Bedingung dass $A$ nicht eingetreten ist\\
\end{theorem}

\begin{theorem}{Satz von Bayes}\\
$$
P(A \mid B)=\frac{P(A) \cdot P(B \mid A)}{P(B)}
$$
\\
$P(A|B)$ = Wahrscheinlichkeit von $A$ unter der Bedingung dass $B$ eingetreten ist\\
$P(A)$ = Wahrscheinlichkeit von Ereignis $A$\\
$P(B|A)$ = Wahrscheinlichkeit von $B$ unter der Bedingung dass $A$ eingetreten ist\\
$P(B)$ = Wahrscheinlichkeit von Ereignis $B$\\
\end{theorem}
