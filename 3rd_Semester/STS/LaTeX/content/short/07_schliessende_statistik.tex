\section{Schliessende Statistik}
\begin{definition}{Erwartungstreue Schätzfunktion}\\
Eine Schätzfunktion $\Theta$ eines Parameters $\theta$ heisst erwartungstreu, wenn:
$$
E(\Theta)=\theta
$$
\end{definition}

\begin{definition}{Effizienz einer Schätzfunktion}\\
Gegeben sind zwei erwartungstreue Schätzfunktionen $\Theta_1$ und $\Theta_2$ desselben Parameters $\theta$. Man nennt $\Theta_1$ effizienter als $\Theta_2$, falls:
$$
V(\Theta_1)<V(\Theta_2)
$$
\end{definition}

\begin{definition}{Konsistenz einer Schätzfunktion}\\
Eine Schätzfunktion $\Theta$ heisst konsistent, wenn:
$$
E(\Theta) \rightarrow \theta \text{ und } V(\Theta) \rightarrow 0 \text{ für } n \rightarrow \infty
$$
\end{definition}
\begin{example}{Erwartungstreue Schätzfunktion}\\
Grundgesamtheit mit Erwartungswert $\mu$, Varianz $\sigma^2$ und Zufallsstichprobe $X_1, X_2, X_3$. Die folgende Schätzfunktion ist gegeben:
$$
\Theta_1=\frac{1}{3} \cdot(2X_1+X_2)
$$
$\Theta_1$ = Schätzfunktion\\
$X_1, X_2$ = Zufallsvariablen aus der Stichprobe\\

Ist diese Schätzfunktion erwartungstreu (Parameter: $\mu$)?
$$
\begin{gathered}
E(\Theta_1)=E(\frac{1}{3} \cdot(2X_1+X_2))=\frac{1}{3} \cdot(2E(X_1)+E(X_2)) \\
E(\Theta_1)=\frac{1}{3} \cdot(2\mu+\mu)=\frac{3\mu}{3}=\mu
\end{gathered}
$$
$E(\Theta_1)$ = Erwartungswert der Schätzfunktion\\
$E(X_1), E(X_2)$ = Erwartungswerte der einzelnen Zufallsvariablen\\
$\mu$ = Wahrer Parameterwert\\

Da $E(\Theta_1)=\mu$ ist die Funktion erwartungstreu.
\end{example}
\begin{example}{Effizienz einer Schätzfunktion}\\
Grundgesamtheit mit Erwartungswert $\mu$, Varianz $\sigma^2$ und Zufallsstichprobe $X_1, X_2, X_3$. Gegeben ist die Schätzfunktion:
$$
\Theta_1=\frac{1}{3} \cdot(2X_1+X_2)
$$

\textbf{Berechnung der Effizienz:}
$$
\begin{aligned}
V(\Theta_1) &= V(\frac{1}{3} \cdot(2X_1+X_2)) \\
&= \frac{1}{9} \cdot V(2X_1+X_2) \\
&= \frac{1}{9} \cdot (V(2X_1) + V(X_2)) \\
&= \frac{1}{9} \cdot (4V(X_1) + V(X_2)) \\
&= \frac{1}{9} \cdot (4\sigma^2 + \sigma^2) \\
&= \frac{5\sigma^2}{9}
\end{aligned}
$$
\\
$V(\Theta_1)$ = Varianz der Schätzfunktion\\
$V(X_1), V(X_2)$ = Varianzen der einzelnen Zufallsvariablen\\
$\sigma^2$ = Varianz der Grundgesamtheit\\

Die Effizienz der Schätzfunktion ist also $\frac{5\sigma^2}{9}$.
\end{example}
\subsection{Vertrauensintervalle}
\begin{definition}{Vertrauensintervall}\\
Wir legen eine grosse Wahrscheinlichkeit $\gamma$ fest (z.B. $\gamma=95\%$). $\gamma$ heisst statistische Sicherheit oder Vertrauensniveau. $\alpha=1-\gamma$ ist die Irrtumswahrscheinlichkeit.

Dann bestimmen wir zwei Zufallsvariablen $\Theta_u$ und $\Theta_o$ so, dass sie den wahren Parameterwert $\Theta$ mit der Wahrscheinlichkeit $\gamma$ einschliessen:
$$
P(\Theta_u \leq \Theta \leq \Theta_o)=\gamma
$$
\end{definition}

\begin{concept}{Intervallschätzung}\\
Verteilungstypen und zugehörige Quantile:
\begin{center}
		\resizebox{\columnwidth}{!}{
\begin{tabular}{|c|c|c|}
\hline
Verteilung & Parameter & Quantile \\
\hline
Normalverteilung ($\sigma^2$ bekannt) & $\mu$ & $c=u_p, p=\frac{1+\gamma}{2}$ \\
\hline
t-Verteilung ($\sigma^2$ unbekannt) & $\mu$ & $c=t_{(p;f=n-1)}, p=\frac{1+\gamma}{2}$ \\
\hline
Chi-Quadrat-Verteilung & $\sigma^2$ & $c_1=\chi^2_{(\frac{1-\gamma}{2};n-1)}, c_2=\chi^2_{(\frac{1+\gamma}{2};n-1)}$ \\
\hline
\end{tabular}
}
\end{center}
\end{concept}

\begin{example}{Berechnung eines Vertrauensintervalls}\\
Geben Sie das Vertrauensintervall für $\mu$ an ($\sigma^2$ unbekannt). Gegeben sind:
$$
n=10, \quad \bar{x}=102, \quad s^2=16, \quad \gamma=0.99
$$

\begin{enumerate}
  \item Verteilungstyp mit Param $\mu$ und $\sigma^2$ unbekannt $\rightarrow$ T-Verteilung
  \item $f=n-1=9$, $p=\frac{1+\gamma}{2}=0.995$, $c=t_{(p;f)}=t_{(0.995;9)}=3.25$
  \item $e=c \cdot \frac{S}{\sqrt{n}}=4.111$, $\Theta_u=\bar{X}-e=97.89$, $\Theta_o=\bar{X}+e=106.11$
\end{enumerate}
\end{example}

\subsection{Likelyhood-Funktion}
\begin{definition}{Likelyhood-Funktion}\\
Wir betrachten eine Zufallsvariable $X$ und ihre Dichte (PDF) $f_x(x|\theta)$, welche von $x$ und einem oder mehreren Parametern $\theta$ abhängig sind. 

Für eine Stichprobe vom Umfang $n$ mit $x_1,\ldots,x_n$ nennen wir die vom Parameter $\theta$ abhängige Funktion die Likelyhood-Funktion der Stichprobe:
$$
L(\theta)=f_x(x_1|\theta) \cdot f_x(x_2|\theta) \cdot \ldots \cdot f_x(x_n|\theta)
$$
\end{definition}

\begin{concept}{Vorgehen bei Maximum-Likelihood-Schätzung}\\
\begin{enumerate}
  \item Likelyhood-Funktion bestimmen
  \item Maximalstelle der Funktion bestimmen:
        \begin{itemize}
           \item (Partielle) Ableitung $L'(\theta)=0$
        \end{itemize}
\end{enumerate}
\end{concept}
\begin{definition}{Erwartungswert und Varianz (Funktion und Wert)}\\
\textbf{Erwartungswert:}
$$
\bar{X}=\frac{1}{n} \cdot \sum_{i=1}^n X_i, \quad \hat{\mu}=\bar{x}=\frac{1}{n} \cdot \sum_{i=1}^n x_i
$$
\\
$\bar{X}$ = Arithmetischer Mittelwert (Zufallsvariable)\\
$\hat{\mu}$ = $\bar{x}$ = Arithmetischer Mittelwert (Stichprobenwert)\\
$n$ = Stichprobenumfang\\
$X_i$ = $i$-te Zufallsvariable\\
$x_i$ = $i$-ter Stichprobenwert\\

\textbf{Varianz:}
$$
S^2=\frac{1}{n-1} \cdot \sum_{i=1}^n (X_i-\bar{X})^2, \quad \hat{\sigma}^2=s^2=\frac{1}{n-1} \cdot \sum_{i=1}^n (x_i-\bar{x})^2
$$
\\
$S^2$ = Stichprobenvarianz (Zufallsvariable)\\
$\hat{\sigma}^2$ = $s^2$ = Stichprobenvarianz (Stichprobenwert)\\
$\bar{X}$ = Arithmetischer Mittelwert (Zufallsvariable)\\
$\bar{x}$ = Arithmetischer Mittelwert (Stichprobenwert)
\end{definition}
\subsection{Detaillierte Vertrauensintervalle}

\begin{concept}{Verteilungstypen und Quantile}\\
\begin{center}
\begin{tabular}{|l|c|c|c|}
\hline
Verteilung & Parameter & Standardisierung & Quantile \\
\hline
Normalverteilung & $\mu$ & $U = \frac{\bar{X}-\mu}{\sigma/\sqrt{n}}$ & $c = u_p, p = \frac{1+\gamma}{2}$ \\
($\sigma^2$ bekannt) & & & \\
\hline
t-Verteilung & $\mu$ & $T = \frac{\bar{X}-\mu}{S/\sqrt{n}}$ & $c = t_{(p;f=n-1)}, p = \frac{1+\gamma}{2}$ \\
($\sigma^2$ unbekannt) & & & \\
\hline
Chi-Quadrat & $\sigma^2$ & $Z = (n-1)\frac{s^2}{\sigma^2}$ & $c_1 = \chi^2_{(\frac{1-\gamma}{2};n-1)}$ \\
& & & $c_2 = \chi^2_{(\frac{1+\gamma}{2};n-1)}$ \\
\hline
\end{tabular}
\end{center}
\end{concept}

\begin{formula}{Konfidenzintervalle}\\
Für verschiedene Verteilungen ergeben sich folgende Intervallgrenzen:

1. Normalverteilung ($\sigma^2$ bekannt):
$$\Theta_u = \bar{X} - c\frac{\sigma}{\sqrt{n}}, \quad \Theta_o = \bar{X} + c\frac{\sigma}{\sqrt{n}}$$

2. t-Verteilung ($\sigma^2$ unbekannt):
$$\Theta_u = \bar{X} - c\frac{S}{\sqrt{n}}, \quad \Theta_o = \bar{X} + c\frac{S}{\sqrt{n}}$$

3. Chi-Quadrat-Verteilung:
$$\Theta_u = \frac{(n-1)s^2}{c_2}, \quad \Theta_o = \frac{(n-1)s^2}{c_1}$$
\end{formula}
