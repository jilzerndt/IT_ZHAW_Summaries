\subsection{Vertrauensintervalle}

\begin{example2}{Beispiel: Berechnung eines Vertrauensintervalls (t-Verteilung)}\\
Geben Sie das Vertrauensintervall für $\mu$ an ($\sigma^2$ unbekannt). Gegeben sind:
$$
n=10, \quad \bar{x}=102, \quad s^2=16, \quad \gamma=0.99
$$

\begin{enumerate}
  \item Verteilungstyp mit Param $\mu$ und $\sigma^2$ unbekannt $\rightarrow$ T-Verteilung
  \item $f=n-1=9$, $p=\frac{1+\gamma}{2}=0.995$, $c=t_{(p;f)}=t_{(0.995;9)}=3.25$
  \item $e=c \cdot \frac{S}{\sqrt{n}}=4.111$, $\Theta_u=\bar{X}-e=97.89$, $\Theta_o=\bar{X}+e=106.11$
\end{enumerate}
\end{example2}