\section{Beispiele}

\begin{example}{Erwartungstreue Schätzfunktion}
Grundgesamtheit mit Erwartungswert $\mu$, Varianz $\sigma^2$ und Zufallsstichprobe $X_1, X_2, X_3$. Die folgende Schätzfunktion ist gegeben:
$$
\Theta_1=\frac{1}{3} \cdot(2X_1+X_2)
$$
$\Theta_1$ = Schätzfunktion\\
$X_1, X_2$ = Zufallsvariablen aus der Stichprobe\\

Ist diese Schätzfunktion erwartungstreu (Parameter: $\mu$)?
$$
\begin{gathered}
E(\Theta_1)=E(\frac{1}{3} \cdot(2X_1+X_2))=\frac{1}{3} \cdot(2E(X_1)+E(X_2)) \\
E(\Theta_1)=\frac{1}{3} \cdot(2\mu+\mu)=\frac{3\mu}{3}=\mu
\end{gathered}
$$
$E(\Theta_1)$ = Erwartungswert der Schätzfunktion\\
$E(X_1), E(X_2)$ = Erwartungswerte der einzelnen Zufallsvariablen\\
$\mu$ = Wahrer Parameterwert\\

Da $E(\Theta_1)=\mu$ ist die Funktion erwartungstreu.
\end{example}

\begin{example}{Intervallschätzung für die Varianz} Für die Varianz $\sigma^2$ einer Normalverteilung mit Stichprobenumfang $n=10$ und Stichprobenvarianz $s^2=16$ soll ein $99\%$-Vertrauensintervall berechnet werden.

\begin{enumerate}
  \item Verteilungstyp: Chi-Quadrat-Verteilung
  \item Freiheitsgrade: $f=n-1=9$
  \item Quantile: $c_1=\chi^2_{(0.005;9)}=1.735$, $c_2=\chi^2_{(0.995;9)}=23.589$
  \item Vertrauensintervall:
  $$
  \frac{(n-1)s^2}{c_2} \leq \sigma^2 \leq \frac{(n-1)s^2}{c_1}
  $$
  $n$ = Stichprobenumfang\\
  $s^2$ = Stichprobenvarianz\\
  $c_1, c_2$ = Chi-Quadrat-Quantile\\
  $\sigma^2$ = Wahre Varianz der Grundgesamtheit\\
  $$
  \frac{9 \cdot 16}{23.589} \leq \sigma^2 \leq \frac{9 \cdot 16}{1.735}
  $$
  $$
  6.10 \leq \sigma^2 \leq 82.99
  $$
\end{enumerate}
\end{example}

\begin{example}{Bernoulli-Anteilsschätzung}
Ein Vertrauensintervall für den Parameter $p$ einer Bernoulli-Verteilung soll aus einer Stichprobe mit $n=100$ und $\bar{x}=0.42$ bei einem Vertrauensniveau von $95\%$ berechnet werden.

\begin{enumerate}
  \item Prüfen der Voraussetzung: $n\hat{p}(1-\hat{p})=100 \cdot 0.42 \cdot 0.58 = 24.36 > 9$
  \item Quantil: $c=u_{0.975}=1.96$
  \item Standardfehler: $\sqrt{\frac{\bar{x}(1-\bar{x})}{n}}=\sqrt{\frac{0.42 \cdot 0.58}{100}}=0.0494$
  \item Vertrauensintervall:
  $$
  0.42 \pm 1.96 \cdot 0.0494 = [0.323; 0.517]
  $$
$n$ = Stichprobenumfang\\
$\bar{x}$ = Stichprobenmittelwert (Anteil der Erfolge)\\
$\hat{p}$ = Geschätzter Parameter der Bernoulli-Verteilung\\
$u_{0.975}$ = 97.5%-Quantil der Standardnormalverteilung\\
\end{enumerate}
\end{example}
