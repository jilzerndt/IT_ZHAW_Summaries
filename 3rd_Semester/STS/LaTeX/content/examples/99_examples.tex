\section{Beispiele}

\subsubsection{Kombinatorik}

\begin{example}{Komplexeres Beispiel: Passwörter}\\
\textbf{Aufgabe:} Ein Passwort muss bestehen aus:
\begin{itemize}
\item Genau 8 Zeichen
\item Mindestens ein Großbuchstabe (26 mögliche)
\item Mindestens eine Ziffer (10 mögliche)
\item Kleine Buchstaben erlaubt (26 mögliche)
\end{itemize}

\textbf{Lösung:}
1. Gesamtzahl aller möglichen 8-stelligen Passwörter mit den Zeichen:
   \begin{itemize}
   \item $n = 26 + 26 + 10 = 62$ Zeichen
   \item Variation mit Wiederholung: $62^8$
   \end{itemize}

2. Abziehen der ungültigen Kombinationen:
   \begin{itemize}
   \item Ohne Großbuchstaben: $(36)^8$
   \item Ohne Ziffern: $(52)^8$
   \item Ohne beide: $(26)^8$
   \end{itemize}

3. Nach dem Inklusions-Exclusions-Prinzip:
   $\text{Gültige Passwörter} = 62^8 - 36^8 - 52^8 + 26^8$
\end{example}

\subsubsection{Wahrscheinlichkeitsrechnung Beispiele}

\begin{example}{Lotterie mit bedingten Gewinnen}\\
\textbf{Aufgabe:} Bei einer Lotterie gewinnt man zunächst mit p = 0.1 einen Bonus-Los. Mit diesem Los kann man dann mit p = 0.2 den Hauptpreis von 1000€ gewinnen. Berechne den Erwartungswert.

\textbf{Lösung:}
\begin{enumerate}
\item \textbf{Ereignisbaum erstellen:}
   \begin{itemize}
   \item P(Bonus) = 0.1
   \item P(Hauptgewinn|Bonus) = 0.2
   \end{itemize}

\item \textbf{Mögliche Ausgänge:}
   \begin{itemize}
   \item 1000€: P = 0.1 · 0.2 = 0.02
   \item 0€: P = 0.98
   \end{itemize}

\item \textbf{Erwartungswert:}
   $E(X) = 1000 \cdot 0.02 + 0 \cdot 0.98 = 20$
\end{enumerate}
\end{example}

\begin{example}{Aktienportfolio}
\textbf{Aufgabe:} Ein Portfolio besteht aus:
\begin{itemize}
\item Aktie A: 60\% Anteil, E(A) = 8\%, V(A) = 25
\item Aktie B: 40\% Anteil, E(B) = 12\%, V(B) = 36
\end{itemize}

\textbf{Lösung:}
\begin{enumerate}
\item \textbf{Erwartungswert des Portfolios:}
   \begin{align*}
   E(P) &= 0.6 \cdot E(A) + 0.4 \cdot E(B) \\
   &= 0.6 \cdot 8\% + 0.4 \cdot 12\% \\
   &= 4.8\% + 4.8\% = 9.6\%
   \end{align*}

\item \textbf{Varianz des Portfolios} (bei Unabhängigkeit):
   \begin{align*}
   V(P) &= (0.6)^2 \cdot V(A) + (0.4)^2 \cdot V(B) \\
   &= 0.36 \cdot 25 + 0.16 \cdot 36 \\
   &= 9 + 5.76 = 14.76
   \end{align*}

\item \textbf{Standardabweichung:}
   $S(P) = \sqrt{14.76} \approx 3.84\%$
\end{enumerate}
\end{example}

\subsubsection{Hypergeometrische Verteilung}

\begin{example}{Ziehung ohne Zurücklegen}
\textbf{Aufgabe:} In einer Urne sind 20 Kugeln, davon 8 rot. Es werden 5 Kugeln ohne Zurücklegen gezogen. 

\textbf{Lösung:}
\begin{enumerate}
\item \textbf{Parameter:} N = 20 (Gesamtanzahl), M = 8 (rote Kugeln), n = 5 (Ziehungen)

\item \textbf{Erwartungswert:}
   $$E(X) = 5 \cdot \frac{8}{20} = 2$$

\item \textbf{Varianz:}
   $$V(X) = 5 \cdot \frac{8}{20} \cdot \frac{12}{20} \cdot \frac{15}{19} \approx 1.184$$

\item \textbf{P(genau 2 rote):}
   $P(X=2) = \frac{\binom{8}{2}\binom{12}{3}}{\binom{20}{5}} \approx 0.3682$
\end{enumerate}
\end{example}

\columnbreak


\subsubsection{Bernoulli-Verteilung}

\begin{example}{Münzwurf}\\
\textbf{Aufgabe:} Faire Münze wird geworfen. X = 1 bei Kopf, X = 0 bei Zahl.

\textbf{Lösung:}
\begin{itemize}
\item p = 0.5 (faire Münze)
\item $E(X) = 0.5$
\item $V(X) = 0.5 \cdot 0.5 = 0.25$
\item $P(X=1) = 0.5$
\item $P(X=0) = 0.5$
\end{itemize}
\end{example}

\subsubsection{Binomialverteilung}

\begin{example}{Qualitätskontrolle mit Binomialverteilung}\\
\textbf{Aufgabe:} Eine Maschine produziert Teile mit Ausschussquote 5\%. In einer Stichprobe von 100 Teilen:
\begin{itemize}
\item a) Wie viele defekte Teile sind zu erwarten?
\item b) Wie groß ist die Wahrscheinlichkeit für genau 3 defekte Teile?
\item c) Wie groß ist die Wahrscheinlichkeit für höchstens 2 defekte Teile?
\end{itemize}

\textbf{Lösung:}
\begin{enumerate}
\item \textbf{Parameter:}
   \begin{itemize}
   \item n = 100 (Stichprobenumfang)
   \item p = 0.05 (Ausschusswahrscheinlichkeit)
   \item X $\thicksim$ B(100, 0.05)
   \end{itemize}

\item \textbf{Erwartungswert:}
   $$E(X) = np = 100 \cdot 0.05 = 5$$

\item \textbf{Genau 3 defekte:}
   $$P(X=3) = \binom{100}{3}(0.05)^3(0.95)^{97} \approx 0.1404$$

\item \textbf{Höchstens 2 defekte:}
   $$P(X \leq 2) = \sum_{k=0}^2 \binom{100}{k}(0.05)^k(0.95)^{100-k} \approx 0.0861$$
\end{enumerate}
\end{example}

\subsubsection{Poisson-Verteilung}



\begin{example}{Poisson-Verteilung in der Praxis}\\
\textbf{Aufgabe:} Ein Callcenter erhält durchschnittlich 3 Anrufe pro 10 Minuten.
\begin{itemize}
\item a) Wahrscheinlichkeit für genau 2 Anrufe in 10 Minuten?
\item b) Wahrscheinlichkeit für mehr als 4 Anrufe?
\end{itemize}

\textbf{Lösung:}
\begin{enumerate}
\item \textbf{Parameter:} $\lambda = 3$ (Erwartungswert), X $\thicksim$ Poi(3)

\item \textbf{Genau 2 Anrufe:}
   $$P(X=2) = \frac{3^2}{2!}e^{-3} \approx 0.2240$$ 

\item \textbf{Mehr als 4 Anrufe:}
   $P(X>4) = 1 - \sum_{k=0}^4 \frac{3^k}{k!}e^{-3} \approx 0.1847$
\end{enumerate}
\end{example}

\subsubsection{Normalverteilung}

\begin{example}{Körpergrößen}\\
\textbf{Aufgabe:} Körpergrößen in einer Population sind normalverteilt mit $\mu = 175$ cm und $\sigma = 10$ cm.

\textbf{Berechnung:}
\begin{itemize}
\item $P(X \leq 185) = \phi(\frac{185-175}{10}) = \phi(1) \approx 0.8413$
\item $P(165 \leq X \leq 185) = \phi(1) - \phi(-1) \approx 0.6826$
\item $P(X > 195) = 1 - \phi(2) \approx 0.0228$
\end{itemize}
\end{example}

\subsubsection{Parameter-/Intervallschätzung}

\begin{example}{Intervallschätzung für die Varianz} \\
  Für die Varianz $\sigma^2$ einer Normalverteilung mit Stichprobenumfang $n=10$ und Stichprobenvarianz $s^2=16$ soll ein $99\%$-Vertrauensintervall berechnet werden.

\begin{enumerate}
  \item Verteilungstyp: Chi-Quadrat-Verteilung
  \item Freiheitsgrade: $f=n-1=9$
  \item Quantile: $c_1=\chi^2_{(0.005;9)}=1.735$, $c_2=\chi^2_{(0.995;9)}=23.589$
  \item Vertrauensintervall:
  $$
  \frac{(n-1)s^2}{c_2} \leq \sigma^2 \leq \frac{(n-1)s^2}{c_1}
  $$
  $n$ = Stichprobenumfang\\
  $s^2$ = Stichprobenvarianz\\
  $c_1, c_2$ = Chi-Quadrat-Quantile\\
  $\sigma^2$ = Wahre Varianz der Grundgesamtheit\\
  $$
  \frac{9 \cdot 16}{23.589} \leq \sigma^2 \leq \frac{9 \cdot 16}{1.735}
  $$
  $$
  6.10 \leq \sigma^2 \leq 82.99
  $$
\end{enumerate}
\end{example}

\begin{example}{Bernoulli-Anteilsschätzung}\\
Ein Vertrauensintervall für den Parameter $p$ einer Bernoulli-Verteilung soll aus einer Stichprobe mit $n=100$ und $\bar{x}=0.42$ bei einem Vertrauensniveau von $95\%$ berechnet werden.

\begin{enumerate}
  \item Prüfen der Voraussetzung: $n\hat{p}(1-\hat{p})=100 \cdot 0.42 \cdot 0.58 = 24.36 > 9$
  \item Quantil: $c=u_{0.975}=1.96$
  \item Standardfehler: $\sqrt{\frac{\bar{x}(1-\bar{x})}{n}}=\sqrt{\frac{0.42 \cdot 0.58}{100}}=0.0494$
  \item Vertrauensintervall:
  $$
  0.42 \pm 1.96 \cdot 0.0494 = [0.323; 0.517]
  $$
$n$ = Stichprobenumfang\\
$\bar{x}$ = Stichprobenmittelwert (Anteil der Erfolge)\\
$\hat{p}$ = Geschätzter Parameter der Bernoulli-Verteilung\\
$u_{0.975}$ = 97.5%-Quantil der Standardnormalverteilung\\
\end{enumerate}
\end{example}
