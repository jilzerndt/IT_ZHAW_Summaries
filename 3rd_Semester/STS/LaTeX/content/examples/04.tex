\subsection{Wahrscheinlichkeitsrechnung Beispiele}

\begin{example2}{Lotterie mit bedingten Gewinnen}
\textbf{Aufgabe:} Bei einer Lotterie gewinnt man zunächst mit p = 0.1 einen Bonus-Los. Mit diesem Los kann man dann mit p = 0.2 den Hauptpreis von 1000€ gewinnen. Berechne den Erwartungswert.

\textbf{Lösung:}
\begin{enumerate}
\item \textbf{Ereignisbaum erstellen:}
   \begin{itemize}
   \item P(Bonus) = 0.1
   \item P(Hauptgewinn|Bonus) = 0.2
   \end{itemize}

\item \textbf{Mögliche Ausgänge:}
   \begin{itemize}
   \item 1000€: P = 0.1 · 0.2 = 0.02
   \item 0€: P = 0.98
   \end{itemize}

\item \textbf{Erwartungswert:}
   $$E(X) = 1000 \cdot 0.02 + 0 \cdot 0.98 = 20$$
\end{enumerate}
\end{example2}

\begin{example2}{Aktienportfolio}
\textbf{Aufgabe:} Ein Portfolio besteht aus:
\begin{itemize}
\item Aktie A: 60\% Anteil, E(A) = 8\%, V(A) = 25
\item Aktie B: 40\% Anteil, E(B) = 12\%, V(B) = 36
\end{itemize}

\textbf{Lösung:}
\begin{enumerate}
\item \textbf{Erwartungswert des Portfolios:}
   \begin{align*}
   E(P) &= 0.6 \cdot E(A) + 0.4 \cdot E(B) \\
   &= 0.6 \cdot 8\% + 0.4 \cdot 12\% \\
   &= 4.8\% + 4.8\% = 9.6\%
   \end{align*}

\item \textbf{Varianz des Portfolios} (bei Unabhängigkeit):
   \begin{align*}
   V(P) &= (0.6)^2 \cdot V(A) + (0.4)^2 \cdot V(B) \\
   &= 0.36 \cdot 25 + 0.16 \cdot 36 \\
   &= 9 + 5.76 = 14.76
   \end{align*}

\item \textbf{Standardabweichung:}
   $$S(P) = \sqrt{14.76} \approx 3.84\%$$
\end{enumerate}
\end{example2}

\subsubsection{Kovarianz und Korrelation}

\begin{example2}{Portfoliorisiko mit Korrelation}
\textbf{Aufgabe:} Zwei Aktien mit:
\begin{itemize}
\item A: E(A) = 10\%, S(A) = 5\%
\item B: E(B) = 8\%, S(B) = 4\%
\item Korrelation: $\rho_{AB} = 0.3$
\end{itemize}

\textbf{Portfolio:} 60\% A, 40\% B

\textbf{Lösung:}
\begin{enumerate}
\item \textbf{Erwartungswert:}
   $$E(P) = 0.6 \cdot 10\% + 0.4 \cdot 8\% = 9.2\%$$

\item \textbf{Varianz mit Korrelation:}
   \begin{align*}
   V(P) &= (0.6)^2V(A) + (0.4)^2V(B) + 2(0.6)(0.4)\rho_{AB}S(A)S(B) \\
   &= 0.36 \cdot (5\%)^2 + 0.16 \cdot (4\%)^2 + 2(0.24)(0.3)(5\%)(4\%) \\
   &= 0.09\% + 0.0256\% + 0.0288\% = 0.1444\%
   \end{align*}

\item \textbf{Standardabweichung:}
   $$S(P) = \sqrt{0.1444\%} \approx 3.8\%$$
\end{enumerate}
\end{example2}