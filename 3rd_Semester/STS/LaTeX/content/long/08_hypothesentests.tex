\section{Hypothesentests}

\begin{definition}{Hypothesentest}\\
Ein statistischer Test zur Überprüfung einer Behauptung bzw. Hypothese über einen oder mehrere Parameter einer Grundgesamtheit.

\begin{itemize}
  \item $H_0$: Nullhypothese (zu überprüfende Behauptung)
  \item $H_A$: Alternativhypothese (Gegenhypothese)
  \item $\alpha$: Signifikanzniveau (Irrtumswahrscheinlichkeit)
\end{itemize}
\end{definition}

\begin{KR}{Ablauf eines Hypothesentests}\\
1. Hypothesen formulieren:
   \begin{itemize}
     \item $H_0$ aufstellen (punktförmig)
     \item $H_A$ aufstellen (ein- oder zweiseitig)
   \end{itemize}

2. Signifikanzniveau $\alpha$ festlegen:
   \begin{itemize}
     \item Meist $\alpha = 0.05$ oder $\alpha = 0.01$
   \end{itemize}

3. Testvariable und Verteilung bestimmen:
   \begin{itemize}
     \item Passende Zeile aus Tabelle 8.2.1 wählen
     \item Standardisierte Testvariable notieren
   \end{itemize}

4. Kritische Werte bestimmen:
   \begin{itemize}
     \item Einseitig: ein Wert $c$
     \item Zweiseitig: zwei Werte $c_u$ und $c_o$
   \end{itemize}

5. Testwert berechnen:
   \begin{itemize}
     \item Stichprobenwerte einsetzen
     \item Standardisierung durchführen
   \end{itemize}

6. Entscheidung treffen:
   \begin{itemize}
     \item $H_0$ annehmen oder ablehnen
     \item Ergebnis interpretieren
   \end{itemize}
\end{KR}

\begin{example2}{Mittelwerttest (bekannte Varianz)}\\
Ein Automobilhersteller gibt den mittleren Verbrauch mit $\mu_0 = 8.2$ l/100km an. In einer Stichprobe von $n=25$ Fahrzeugen wurde ein Mittelwert von $\bar{x}=9.1$ l/100km bei bekannter Standardabweichung $\sigma=2.1$ l/100km gemessen.

1. Hypothesen:
   \begin{itemize}
     \item $H_0: \mu = 8.2$
     \item $H_A: \mu \neq 8.2$ (zweiseitig)
   \end{itemize}

2. Signifikanzniveau: $\alpha = 0.05$

3. Testvariable: $U = \frac{\bar{X}-\mu_0}{\sigma/\sqrt{n}}$ (standardnormalverteilt)

4. Kritische Werte:
   \begin{itemize}
     \item $c_u = -1.96$
     \item $c_o = 1.96$
   \end{itemize}

5. Testwert:
   $$\hat{u} = \frac{9.1-8.2}{2.1/\sqrt{25}} = 2.14$$

6. Entscheidung: $\hat{u} > c_o$, also $H_0$ ablehnen
\end{example2}

\begin{KR}{Unterscheidung abhängiger/unabhängiger Stichproben}\\
1. Abhängige Stichproben:
   \begin{itemize}
     \item Gleicher Stichprobenumfang
     \item Messungen am gleichen Objekt
     \item Paarweise Zuordnung möglich
   \end{itemize}

2. Unabhängige Stichproben:
   \begin{itemize}
     \item Unterschiedliche Objekte
     \item Keine Zuordnung möglich
     \item Stichprobenumfänge können verschieden sein
   \end{itemize}

3. Auswirkung auf Test:
   \begin{itemize}
     \item Abhängig: Test der Differenzen
     \item Unabhängig: Test der einzelnen Stichproben
   \end{itemize}
\end{KR}

\begin{example2}{Abhängige Stichproben (t-Test)}\\
Vergleich zweier Messgeräte an denselben 5 Widerständen:

\begin{center}
\begin{tabular}{|c|c|c|c|c|c|}
\hline
i & 1 & 2 & 3 & 4 & 5 \\
\hline
Gerät 1 & 100.5 & 102.4 & 104.3 & 101.5 & 98.4 \\
\hline
Gerät 2 & 98.2 & 99.1 & 102.4 & 101.1 & 96.2 \\
\hline
\end{tabular}
\end{center}

1. Hypothesen:
   \begin{itemize}
     \item $H_0: \mu_d = 0$
     \item $H_A: \mu_d \neq 0$
   \end{itemize}

2. $\alpha = 0.01$

3. Differenzen bilden:
   \begin{itemize}
     \item $\bar{d} = 2.02$
     \item $s_d = 1.047$
   \end{itemize}

4. Testvariable: $T = \frac{\bar{D}}{S_d/\sqrt{n}}$ (t-verteilt mit $f=4$)

5. Testwert:
   $$\hat{t} = \frac{2.02}{1.047/\sqrt{5}} = 4.313$$

6. Kritische Werte: $c_u = -4.604$, $c_o = 4.604$

7. Entscheidung: $|\hat{t}| < c_o$, also $H_0$ annehmen
\end{example2}

\begin{concept}{Fehlerarten bei Hypothesentests}\\
\begin{center}
\begin{tabular}{|l|c|c|}
\hline
& $H_0$ annehmen & $H_0$ ablehnen \\
\hline
$H_0$ wahr & Richtige Entscheidung & Fehler 1. Art ($\alpha$) \\
\hline
$H_0$ falsch & Fehler 2. Art ($\beta$) & Richtige Entscheidung \\
\hline
\end{tabular}
\end{center}

\begin{itemize}
  \item Fehler 1. Art: $\alpha$ (Signifikanzniveau)
  \item Fehler 2. Art: $\beta$ (abhängig vom wahren Wert)
  \item Teststärke: $1-\beta$ (Wahrscheinlichkeit für richtige Ablehnung)
\end{itemize}
\end{concept}

\begin{KR}{p-Wert berechnen}\\
1. Testvariable und Verteilung bestimmen:
   \begin{itemize}
     \item Aus Tabelle 8.2.1 auswählen
     \item Testwert berechnen
   \end{itemize}

2. p-Wert ermitteln:
   \begin{itemize}
     \item Einseitig: $P(T \geq |\hat{t}|)$ oder $P(T \leq |\hat{t}|)$
     \item Zweiseitig: $2 \cdot P(T \geq |\hat{t}|)$
   \end{itemize}

3. Vergleich mit $\alpha$:
   \begin{itemize}
     \item $p \leq \alpha$: $H_0$ ablehnen
     \item $p > \alpha$: $H_0$ annehmen
   \end{itemize}
\end{KR}

\begin{example2}{p-Wert berechnen}\\
Bei einer Qualitätskontrolle wurden in einer Stichprobe von $n=400$ Teilen 20 Defekte gefunden. Die Nullhypothese lautet $H_0: p = 0.03$ gegen $H_A: p > 0.03$.

1. Testvariable:
   $$U = \frac{\hat{p}-p_0}{\sqrt{\frac{p_0(1-p_0)}{n}}}$$

2. Testwert:
   $$\hat{u} = \frac{0.05-0.03}{\sqrt{\frac{0.03 \cdot 0.97}{400}}} = 2.345$$

3. p-Wert (einseitig):
   $$p = P(U \geq 2.345) = 1 - \Phi(2.345) = 0.0095$$

4. Entscheidung: $p < \alpha = 0.05$, also $H_0$ ablehnen
\end{example2}

\begin{KR}{Typische Prüfungsaufgaben}\\
1. Parametertests:
   \begin{itemize}
     \item Mittelwerttest ($\sigma^2$ bekannt/unbekannt)
     \item Varianztest
     \item Anteilstest
   \end{itemize}

2. Vergleich zweier Stichproben:
   \begin{itemize}
     \item Abhängige Stichproben
     \item Unabhängige Stichproben
     \item Gleiche/verschiedene Varianzen
   \end{itemize}

3. p-Wert Berechnung:
   \begin{itemize}
     \item Ein-/zweiseitige Tests
     \item Verschiedene Verteilungen
   \end{itemize}
\end{KR}

\begin{theorem}{Verteilungen der Testvariablen}\\
\begin{itemize}
  \item Normalverteilung ($\sigma^2$ bekannt):
    $$U = \frac{\bar{X}-\mu_0}{\sigma/\sqrt{n}} \sim N(0,1)$$
  
  \item t-Verteilung ($\sigma^2$ unbekannt):
    $$T = \frac{\bar{X}-\mu_0}{S/\sqrt{n}} \sim t_{n-1}$$
  
  \item Chi-Quadrat (Varianztest):
    $$Z = \frac{(n-1)S^2}{\sigma_0^2} \sim \chi^2_{n-1}$$
  
  \item Anteilstest (für großes n):
    $$U = \frac{\hat{p}-p_0}{\sqrt{\frac{p_0(1-p_0)}{n}}} \sim N(0,1)$$
\end{itemize}
\end{theorem}

\begin{example2}{Varianztest}\\
Die Varianz eines Produktionsprozesses soll $\sigma_0^2 = 25$ nicht überschreiten. Eine Stichprobe von $n=12$ Teilen ergab eine empirische Varianz von $s^2 = 40$.

1. Hypothesen:
   \begin{itemize}
     \item $H_0: \sigma^2 = 25$
     \item $H_A: \sigma^2 > 25$
   \end{itemize}

2. $\alpha = 0.05$

3. Testvariable: $Z = \frac{(n-1)S^2}{\sigma_0^2}$ ($\chi^2$-verteilt mit $f=11$)

4. Testwert:
   $$\hat{z} = \frac{11 \cdot 40}{25} = 17.6$$

5. Kritischer Wert: $c = \chi^2_{(0.95;11)} = 19.675$

6. Entscheidung: $\hat{z} < c$, also $H_0$ annehmen
\end{example2}

\begin{concept}{Einseitige vs. zweiseitige Tests}\\
1. Zweiseitiger Test ($H_A: \theta \neq \theta_0$):
   \begin{itemize}
     \item Kritische Werte: $c_u$ und $c_o$
     \item Ablehnbereich: $(-\infty,c_u) \cup (c_o,\infty)$
     \item p-Wert: $2 \cdot P(T \geq |\hat{t}|)$
   \end{itemize}

2. Rechtsseitiger Test ($H_A: \theta > \theta_0$):
   \begin{itemize}
     \item Kritischer Wert: $c_o$
     \item Ablehnbereich: $(c_o,\infty)$
     \item p-Wert: $P(T \geq \hat{t})$
   \end{itemize}

3. Linksseitiger Test ($H_A: \theta < \theta_0$):
   \begin{itemize}
     \item Kritischer Wert: $c_u$
     \item Ablehnbereich: $(-\infty,c_u)$
     \item p-Wert: $P(T \leq \hat{t})$
   \end{itemize}
\end{concept}


\begin{KR}{Maximum-Likelihood-Schätzung}\\
1. Likelyhood-Funktion aufstellen:
   \begin{itemize}
     \item Dichte der Verteilung identifizieren
     \item Produkt der Einzelwahrscheinlichkeiten bilden
     \item $L(\theta) = \prod_{i=1}^n f_X(x_i|\theta)$
   \end{itemize}

2. Logarithmierte Likelyhood-Funktion bilden:
   \begin{itemize}
     \item $\ln(L(\theta))$ berechnen
     \item Produkte werden zu Summen
   \end{itemize}

3. Ableitung Null setzen:
   \begin{itemize}
     \item $\frac{d}{d\theta}\ln(L(\theta)) = 0$
     \item Nach $\theta$ auflösen
   \end{itemize}

4. Maximum prüfen:
   \begin{itemize}
     \item Zweite Ableitung muss negativ sein
     \item $\frac{d^2}{d\theta^2}\ln(L(\theta)) < 0$
   \end{itemize}
\end{KR}

\begin{example2}{Maximum-Likelihood Exponentialverteilung}\\
Gegeben sei eine Stichprobe $x_1,...,x_n$ aus einer Exponentialverteilung mit Parameter $\lambda$:
$$f(x|\lambda) = \lambda e^{-\lambda x}, \quad x \geq 0$$

1. Likelihood-Funktion:
$$L(\lambda) = \prod_{i=1}^n \lambda e^{-\lambda x_i} = \lambda^n e^{-\lambda \sum x_i}$$

2. Log-Likelihood:
$$\ln(L(\lambda)) = n\ln(\lambda) - \lambda \sum x_i$$

3. Ableitung Null setzen:
$$\frac{d}{d\lambda}\ln(L(\lambda)) = \frac{n}{\lambda} - \sum x_i = 0$$

4. ML-Schätzer:
$$\hat{\lambda} = \frac{n}{\sum x_i} = \frac{1}{\bar{x}}$$
\end{example2}

\begin{KR}{Vertrauensintervalle berechnen}\\
1. Verteilungstyp und Parameter bestimmen:
   \begin{itemize}
     \item Normalverteilung mit $\sigma^2$ bekannt
     \item Normalverteilung mit $\sigma^2$ unbekannt
     \item Chi-Quadrat für Varianz
   \end{itemize}

2. Vertrauensniveau $\gamma$ und Freiheitsgrade:
   \begin{itemize}
     \item $\gamma$ aus Aufgabe entnehmen
     \item $f = n-1$ bei t- und $\chi^2$-Verteilung
   \end{itemize}

3. Kritische Werte bestimmen:
   \begin{itemize}
     \item Normalverteilung: $c = u_p$ mit $p = \frac{1+\gamma}{2}$
     \item t-Verteilung: $c = t_{(p;f)}$ mit $p = \frac{1+\gamma}{2}$
     \item Chi-Quadrat: $c_1 = \chi^2_{(p_1;f)}$ und $c_2 = \chi^2_{(p_2;f)}$
   \end{itemize}

4. Intervallgrenzen berechnen:
   \begin{itemize}
     \item Mittelwert: $[\bar{x} - e; \bar{x} + e]$ mit $e = c \cdot \frac{s}{\sqrt{n}}$
     \item Varianz: $[\frac{(n-1)s^2}{c_2}; \frac{(n-1)s^2}{c_1}]$
   \end{itemize}
\end{KR}

\begin{example2}{Vertrauensintervall für Mittelwert}\\
Eine Maschine produziert Schrauben. Bei einer Stichprobe von $n=16$ Schrauben wurde der Durchmesser gemessen:
\begin{itemize}
  \item Mittelwert: $\bar{x} = 5.2$ mm
  \item Standardabweichung: $s = 0.15$ mm
  \item Vertrauensniveau: $\gamma = 95\%$
\end{itemize}

1. Verteilungstyp:
   \begin{itemize}
     \item $\sigma^2$ unbekannt $\rightarrow$ t-Verteilung
     \item $f = n-1 = 15$ Freiheitsgrade
   \end{itemize}

2. Kritischer Wert:
   \begin{itemize}
     \item $p = \frac{1+0.95}{2} = 0.975$
     \item $c = t_{(0.975;15)} = 2.131$
   \end{itemize}

3. Intervallgrenzen:
   \begin{itemize}
     \item $e = 2.131 \cdot \frac{0.15}{\sqrt{16}} = 0.080$
     \item $[\bar{x} - e; \bar{x} + e] = [5.12; 5.28]$
   \end{itemize}
\end{example2}

\begin{example2}{Vertrauensintervall für Varianz}\\
Für die obigen Schrauben soll ein 95\%-Vertrauensintervall für die Varianz berechnet werden.

1. Verteilungstyp:
   \begin{itemize}
     \item Chi-Quadrat-Verteilung
     \item $f = n-1 = 15$ Freiheitsgrade
   \end{itemize}

2. Kritische Werte:
   \begin{itemize}
     \item $p_1 = \frac{1-0.95}{2} = 0.025$
     \item $p_2 = \frac{1+0.95}{2} = 0.975$
     \item $c_1 = \chi^2_{(0.025;15)} = 6.262$
     \item $c_2 = \chi^2_{(0.975;15)} = 27.488$
   \end{itemize}

3. Intervallgrenzen:
   \begin{itemize}
     \item $s^2 = 0.15^2 = 0.0225$
     \item $\theta_u = \frac{15 \cdot 0.0225}{27.488} = 0.0123$
     \item $\theta_o = \frac{15 \cdot 0.0225}{6.262} = 0.0539$
   \end{itemize}

4. Vertrauensintervall für $\sigma^2$: $[0.0123; 0.0539]$
\end{example2}

\begin{KR}{Bestimmung des Stichprobenumfangs}\\
1. Gegebene Verteilung und Parameter:
   \begin{itemize}
     \item Normalverteilung mit $\sigma^2$ bekannt
     \item Vertrauensniveau $\gamma$
     \item Maximal zulässige Intervallbreite $d$
   \end{itemize}

2. Kritischen Wert bestimmen:
   \begin{itemize}
     \item $p = \frac{1+\gamma}{2}$
     \item $c = u_p$ für Normalverteilung
   \end{itemize}

3. Stichprobenumfang berechnen:
   \begin{itemize}
     \item $n \geq (\frac{2c\sigma}{d})^2$
     \item Auf nächste ganze Zahl aufrunden
   \end{itemize}

4. Bei unbekannter Varianz:
   \begin{itemize}
     \item Vorerhebung durchführen
     \item Varianz schätzen
     \item t-Verteilung statt Normalverteilung
   \end{itemize}
\end{KR}

\begin{example2}{Stichprobenumfang bestimmen}\\
Ein Prozess produziert Teile mit bekannter Standardabweichung $\sigma = 0.5$ mm. Der Mittelwert soll mit einer Genauigkeit von $\pm 0.2$ mm bei einem Vertrauensniveau von 99\% geschätzt werden.

1. Gesucht:
   \begin{itemize}
     \item Intervallbreite $d = 0.4$ mm
     \item $\gamma = 0.99$
   \end{itemize}

2. Kritischer Wert:
   \begin{itemize}
     \item $p = \frac{1+0.99}{2} = 0.995$
     \item $c = u_{0.995} = 2.576$
   \end{itemize}

3. Stichprobenumfang:
   \begin{itemize}
     \item $n \geq (\frac{2 \cdot 2.576 \cdot 0.5}{0.4})^2 = 41.47$
     \item $n = 42$ (aufgerundet)
   \end{itemize}
\end{example2}

\begin{concept}{Übersicht statistische Schätzverfahren}\\
1. Punktschätzung:
   \begin{itemize}
     \item Maximum-Likelihood
     \item Momentenmethode
     \item Kleinste-Quadrate
   \end{itemize}

2. Intervallschätzung:
   \begin{itemize}
     \item Vertrauensintervalle für Mittelwert
     \item Vertrauensintervalle für Varianz
     \item Vertrauensintervalle für Anteilswerte
   \end{itemize}

3. Gütekriterien:
   \begin{itemize}
     \item Erwartungstreue
     \item Effizienz
     \item Konsistenz
     \item Minimale Varianz
   \end{itemize}
\end{concept}

\begin{KR}{Typische Prüfungsaufgaben}\\
1. Maximum-Likelihood-Schätzung:
   \begin{itemize}
     \item Likelihood-Funktion aufstellen
     \item Logarithmieren
     \item Ableitung Null setzen
     \item ML-Schätzer bestimmen
     \item Maximum prüfen
   \end{itemize}

2. Vertrauensintervalle:
   \begin{itemize}
     \item Verteilungstyp bestimmen
     \item Kritische Werte nachschlagen
     \item Intervallgrenzen berechnen
     \item Interpretation der Ergebnisse
   \end{itemize}

3. Stichprobenumfang:
   \begin{itemize}
     \item Genauigkeitsanforderungen
     \item Vertrauensniveau
     \item Minimal notwendigen Umfang bestimmen
   \end{itemize}
\end{KR}