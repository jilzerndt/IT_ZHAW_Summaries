\section{Deskriptive Statistik}

\begin{definition}{Bivariate Daten (Merkmale)}\\
Bivariate Daten beschreiben zwei Merkmale desselben Merkmalsträgers. Die Darstellung hängt von den Merkmalstypen ab:
\begin{itemize}
  \item 2x kategoriell $\rightarrow$ Kontingenztabelle + Mosaikplot
  \item 1x kategoriell + 1x metrisch $\rightarrow$ Boxplot oder Stripchart
  \item 2x metrisch $\rightarrow$ Streudiagramm
\end{itemize}
\end{definition}

\begin{KR}{Erstellen einer Kontingenztabelle}
\begin{enumerate}
    \item Identifiziere die Ausprägungen beider kategorieller Merkmale
    \item Erstelle eine Tabelle mit:
        \begin{itemize}
            \item Zeilen für Ausprägungen des ersten Merkmals
            \item Spalten für Ausprägungen des zweiten Merkmals
        \end{itemize}
    \item Zähle die Häufigkeiten für jede Kombination
    \item Füge Randsummen für Zeilen und Spalten hinzu
    \item Optional: Berechne relative Häufigkeiten
\end{enumerate}
\end{KR}

\begin{example2}{Kontingenztabelle}
Studierende nach Studiengang und Geschlecht:
\begin{center}
\begin{tabular}{|l|c|c|c|}
\hline
 & Männlich & Weiblich & Total \\
\hline
Informatik & 120 & 30 & 150 \\
Wirtschaft & 80 & 70 & 150 \\
\hline
Total & 200 & 100 & 300 \\
\hline
\end{tabular}
\end{center}
\end{example2}

\begin{minipage}{0.5\columnwidth}
\begin{definition}{Absolute Häufigkeiten}\\
$$
H=\sum_{i=1}^{n} h_{i}
$$
\\
$H$: Absolute Häufigkeit, \\
$h_{i}$: Einzelhäufigkeit der $i$-ten Beobachtung, \\
$n$: Anzahl der Beobachtungen.
\end{definition}
\end{minipage}
\begin{minipage}{0.5\columnwidth}
\begin{definition}{Relative Häufigkeiten}\\
$$
F=\sum_{i=1}^{m} f_{i}, \quad F(x)=\frac{H(x)}{n}
$$
\\
$F$: Relative Häufigkeit, \\
$f_{i}$: Einzelrelative Häufigkeit der $i$-ten Beobachtung, \\
$H(x)$: Absolute Häufigkeit eines Wertes $x$, \\
$n$: Anzahl der Beobachtungen.
\end{definition}
\end{minipage}

\subsection{Kennwerte (Lagemasse)}

\begin{KR}{Berechnung von Lagekennwerten}
\begin{enumerate}
    \item Sortiere die Daten aufsteigend
    \item Berechne den Mittelwert:
        \begin{itemize}
            \item Summe aller Werte / Anzahl Werte
        \end{itemize}
    \item Bestimme den Median:
        \begin{itemize}
            \item Bei ungerader Anzahl: mittlerer Wert
            \item Bei gerader Anzahl: Mittelwert der beiden mittleren Werte
        \end{itemize}
    \item Finde den Modus (häufigster Wert)
    \item Berechne die Quartile:
        \begin{itemize}
            \item Q1: 25\%-Quantil
            \item Q2: Median (50\%-Quantil)
            \item Q3: 75\%-Quantil
        \end{itemize}
\end{enumerate}
\end{KR}

\begin{minipage}{0.5\columnwidth}
\begin{definition}{Quantil}\\
$$
i=\lceil n \cdot q\rceil, \quad Q=x_{i}=x_{\lceil n \cdot q\rceil}
$$
\\
$i$: Position des Quantils, \\
$n$: Anzahl der Beobachtungen, \\
$q$: Quantilswert (z. B. 0.25 für das erste Quartil), \\
$x_{i}$: Beobachtung an Position $i$.
\end{definition}
\end{minipage}
\begin{minipage}{0.5\columnwidth}
\begin{definition}{Interquartilsabstand}\\
$$
I Q R=Q_{3}-Q_{1}
$$
\\
$IQR$: Interquartilsabstand, \\
$Q_{3}$: Oberes Quartil (75. Perzentil), \\
$Q_{1}$: Unteres Quartil (25. Perzentil).
\end{definition}
\end{minipage}

\begin{example2}{Berechnung von Quantilen}
Gegeben sei die Datenreihe: 2, 4, 4, 5, 7, 8, 9, 10\\
$n = 8$ Beobachtungen

\textbf{Berechnung Q1 (25\%-Quantil):}
\begin{itemize}
    \item $i = \lceil 8 \cdot 0.25 \rceil = \lceil 2 \rceil = 2$
    \item Q1 = $x_2 = 4$
\end{itemize}

\textbf{Berechnung Q2 (Median):}
\begin{itemize}
    \item $n$ gerade $\rightarrow$ Mittelwert von Position 4 und 5
    \item Q2 = $(5 + 7)/2 = 6$
\end{itemize}

\textbf{Berechnung Q3 (75\%-Quantil):}
\begin{itemize}
    \item $i = \lceil 8 \cdot 0.75 \rceil = \lceil 6 \rceil = 6$
    \item Q3 = $x_6 = 8$
\end{itemize}

\textbf{Interquartilsabstand:}
\begin{itemize}
    \item IQR = Q3 - Q1 = 8 - 4 = 4
\end{itemize}
\end{example2}

\begin{definition}{Modus}\\
$$
x_{\text {mod }}=\text{Häufigste Wert}
$$
\end{definition}

\begin{minipage}{0.5\columnwidth}
\begin{concept}{Arithmetisches Mittel}\\
$$
\bar{x}=\frac{1}{n} \sum_{i=1}^{n} x_{i}=\sum_{i=1}^{m} a_{i} \cdot f_{i}
$$
\\
$\bar{x}$: Arithmetisches Mittel, \\
$n$: Anzahl der Beobachtungen, \\
$x_{i}$: Einzelbeobachtung, \\
$a_{i}$: Klassenmitte, \\
$f_{i}$: Relative Häufigkeit der Klasse $i$.
\end{concept}
\end{minipage}%
\begin{minipage}{0.5\columnwidth}
\begin{concept}{Median}\\
\resizebox{\columnwidth}{!}{
$
\left\{\begin{array}{c}
x_{\left[\frac{n+1}{2}\right]} \quad n \text { ungerade } \\ 
0.5 \cdot\left(x_{\left[\frac{n}{2}\right]}+x_{\left[\frac{n}{2}+1\right]}\right) \quad n \text { gerade }
\end{array}\right.
$
}
\\
\\
$n$: Anzahl der Beobachtungen, \\
$x_{[k]}$: Beobachtung an der $k$-ten Position.
\end{concept}
\end{minipage}

\begin{example2}{Vergleich der Lageparameter}
Gegeben seien folgende Datensätze:
\begin{itemize}
    \item A: 2, 2, 3, 4, 4, 5, 8, 12
    \item B: 2, 4, 4, 4, 4, 4, 6, 8
\end{itemize}

\textbf{Datensatz A:}
\begin{itemize}
    \item Mittelwert: $\bar{x}_A = 5$
    \item Median: $x_{med_A} = 4$
    \item Modus: $x_{mod_A} = 2, 4$ (bimodal)
\end{itemize}

\textbf{Datensatz B:}
\begin{itemize}
    \item Mittelwert: $\bar{x}_B = 4.5$
    \item Median: $x_{med_B} = 4$
    \item Modus: $x_{mod_B} = 4$
\end{itemize}

Vergleich zeigt:
\begin{itemize}
    \item Mittelwert reagiert empfindlich auf Ausreißer (A)
    \item Median ist robuster gegen Ausreißer
    \item Modus zeigt Häufungen, kann mehrfach auftreten
\end{itemize}
\end{example2}

\begin{definition}{Stichprobenvarianz $s^{2}$ (Streumasse)}\\
$$
s^{2}=\frac{1}{n} \sum_{i=1}^{n}\left(x_{i}-\bar{x}\right)^{2}=\overline{x^{2}}-\bar{x}^{2}, \quad\left(s_{\text{kor}}\right)^{2}=\frac{1}{n-1} \sum_{i=1}^{n}\left(x_{i}-\bar{x}\right)^{2}
$$
$$
\left(s_{\text{kor}}\right)^{2}=\frac{n}{n-1} \cdot s^{2}
$$
\\
$s^{2}$: Stichprobenvarianz, \\
$s_{\text{kor}}^{2}$: Korrigierte Stichprobenvarianz, \\
$x_{i}$: Einzelbeobachtung, \\
$\bar{x}$: Arithmetisches Mittel, \\
$n$: Anzahl der Beobachtungen.
\end{definition}

\begin{KR}{Berechnung der Stichprobenvarianz}
\begin{enumerate}
    \item Berechne den Mittelwert $\bar{x}$
    \item Für jeden Wert $x_i$:
        \begin{enumerate}
            \item Berechne Abweichung vom Mittelwert $(x_i - \bar{x})$
            \item Quadriere die Abweichung $(x_i - \bar{x})^2$
        \end{enumerate}
    \item Summiere alle quadrierten Abweichungen
    \item Teile durch $(n-1)$ für korrigierte Varianz
    \item Alternative Berechnung:
        \begin{enumerate}
            \item Berechne $\overline{x^2}$ (Mittelwert der quadrierten Werte)
            \item Berechne $(\bar{x})^2$ (Quadrat des Mittelwerts)
            \item Varianz = $\overline{x^2} - (\bar{x})^2$
        \end{enumerate}
\end{enumerate}
\end{KR}

\begin{definition}{Standardabweichung $s$ (Streumasse)}\\
$$
s=\sqrt{\frac{1}{n} \sum_{i=1}^{n}\left(x_{i}-\bar{x}\right)^{2}}=\sqrt{\overline{x^{2}}-\bar{x}^{2}}, \quad s_{\text{kor}}=\sqrt{\frac{1}{n-1} \sum_{i=1}^{n}\left(x_{i}-\bar{x}\right)^{2}}
$$
\\
$s$: Standardabweichung, \\
$s_{\text{kor}}$: Korrigierte Standardabweichung, \\
$x_{i}$: Einzelbeobachtung, \\
$\bar{x}$: Arithmetisches Mittel, \\
$n$: Anzahl der Beobachtungen.
\end{definition}

\begin{example2}{Berechnung von Varianz und Standardabweichung}
Gegeben sei die Datenreihe: 2, 4, 4, 6, 9

\textbf{Schritt 1:} Mittelwert berechnen
$$\bar{x} = \frac{2 + 4 + 4 + 6 + 9}{5} = 5$$

\textbf{Schritt 2:} Abweichungen quadrieren
\begin{center}
\begin{tabular}{|c|c|c|}
\hline
$x_i$ & $(x_i - \bar{x})$ & $(x_i - \bar{x})^2$ \\
\hline
2 & -3 & 9 \\
4 & -1 & 1 \\
4 & -1 & 1 \\
6 & 1 & 1 \\
9 & 4 & 16 \\
\hline
\end{tabular}
\end{center}

\textbf{Schritt 3:} Varianz berechnen
$$s_{\text{kor}}^2 = \frac{9 +1 + 1 + 1 + 16}{5-1} = \frac{28}{4} = 7$$

\textbf{Schritt 4:} Standardabweichung berechnen
$$s_{\text{kor}} = \sqrt{7} \approx 2.65$$

\textbf{Alternative Berechnung:}
\begin{itemize}
    \item $\overline{x^2} = \frac{4 + 16 + 16 + 36 + 81}{5} = 30.6$
    \item $(\bar{x})^2 = 5^2 = 25$
    \item $s^2 = 30.6 - 25 = 5.6$
    \item $s_{\text{kor}}^2 = \frac{5}{4} \cdot 5.6 = 7$
\end{itemize}
\end{example2}

\subsection{PDF + CDF}

\begin{definition}{Nicht klassierte Daten (PMF und CDF)}\\
Die absolute Häufigkeit kann als Funktion $h: \mathbb{R} \rightarrow \mathbb{R}$ bezeichnet werden.
$$
h_{i}
$$
\\
$h_{i}$: Absolute Häufigkeit der $i$-ten Beobachtung.
\\
\\
Die relative Häufigkeit kann als Funktion $f: \mathbb{R} \rightarrow \mathbb{R}$ bezeichnet werden.
$$
f_{i}=\frac{h_{i}}{n}
$$
\\
$f_{i}$: Relative Häufigkeit der $i$-ten Beobachtung, \\
$h_{i}$: Absolute Häufigkeit der $i$-ten Beobachtung, \\
$n$: Anzahl der Beobachtungen.
\end{definition}

\begin{KR}{Erstellen einer Häufigkeitsverteilung}
\begin{enumerate}
    \item Sammle alle verschiedenen Werte
    \item Zähle absolute Häufigkeiten:
        \begin{itemize}
            \item Wie oft kommt jeder Wert vor?
        \end{itemize}
    \item Berechne relative Häufigkeiten:
        \begin{itemize}
            \item Teile jede absolute Häufigkeit durch $n$
        \end{itemize}
    \item Berechne kumulative Häufigkeiten:
        \begin{itemize}
            \item Absolute: Summiere $h_i$ von links nach rechts
            \item Relative: Summiere $f_i$ von links nach rechts
        \end{itemize}
\end{enumerate}
\end{KR}

\subsection{PMF und CDF für diskrete und stetige Daten}

\begin{concept}{Unterschied zwischen PMF und PDF}\\
\begin{itemize}
    \item \textbf{PMF (Probability Mass Function)}:
        \begin{itemize}
            \item Für diskrete Daten
            \item Wahrscheinlichkeit für exakte Werte
            \item Summe aller Wahrscheinlichkeiten = 1
        \end{itemize}
    \item \textbf{PDF (Probability Density Function)}:
        \begin{itemize}
            \item Für stetige Daten
            \item Fläche unter Kurve gibt Wahrscheinlichkeit
            \item Integral über gesamten Bereich = 1
        \end{itemize}
\end{itemize}
\end{concept}

\begin{definition}{Diskrete Verteilungsfunktionen}\\
Die absolute Häufigkeit kann als Funktion $h: \mathbb{R} \rightarrow \mathbb{R}$ bezeichnet werden:
$$h_i$$

Die relative Häufigkeit kann als Funktion $f: \mathbb{R} \rightarrow \mathbb{R}$ bezeichnet werden:
$$f_i = \frac{h_i}{n}$$

\begin{example}{Diskrete Häufigkeitsverteilung}\\
\renewcommand{\arraystretch}{2}%
\begin{center}
\begin{tabular}{|c|c|c|c|c|c|}
\hline
$a_i$ & 397 & 398 & 399 & 400 & Total \\
\hline
$h_i$ & 1 & 3 & 7 & 5 & 16 \\
\hline
$f_i$ & $\frac{1}{16}$ & $\frac{3}{16}$ & $\frac{7}{16}$ & $\frac{5}{16}$ & 1 \\
\hline
$H_i$ & 1 & 4 & 11 & 16 & \\
\hline
$F_i$ & $\frac{1}{16}$ & $\frac{4}{16}$ & $\frac{11}{16}$ & $\frac{16}{16}$ & \\
\hline
\end{tabular}
\end{center}
\end{example}
\end{definition}

\begin{KR}{Klassenbildung für stetige Daten}
\begin{enumerate}
    \item Bestimme Spannweite (Max - Min)
    \item Wähle Anzahl Klassen $k$:
        \begin{itemize}
            \item $5 \leq k \leq 20$
            \item $k \leq \sqrt{n}$
        \end{itemize}
    \item Berechne Klassenbreite:
        \begin{itemize}
            \item $d = \frac{\text{Spannweite}}{k}$
            \item Runde auf praktische Zahl
        \end{itemize}
    \item Bestimme Klassengrenzen:
        \begin{itemize}
            \item Start bei Min oder praktischem Wert darunter
            \item Ende bei Max oder praktischem Wert darüber
        \end{itemize}
    \item Zähle Häufigkeiten in jeder Klasse
\end{enumerate}
\end{KR}

\begin{concept}{Klassenbildung (Faustregeln)}\\
\begin{itemize}
  \item Die Klassen sollten gleich breit gewählt werden
  \item Die Anzahl der Klassen sollte zwischen 5 und 20 liegen, jedoch $\sqrt{n}$ nicht überschreiben
  \item Klassengrenzen sollten "runde" Zahlen sein
  \item Werte auf Klassengrenzen kommen in die obere Klasse
\end{itemize}
\end{concept}

\begin{definition}{Stetige Verteilungsfunktionen}\\
Die absolute Häufigkeitsdichtefunktion erhält man, indem der Wert der absoluten Häufigkeit $h_i$ durch die Klassenbreite (Säulenbreite) $d_i$ geteilt wird:
$$h(x) = \frac{h_i}{d_i}$$

Die relative Häufigkeitsdichtefunktion (PDF) $f: \mathbb{R} \rightarrow [0,1]$ erhält man aus der absoluten Häufigkeitsdichtefunktion, indem man den Wert durch die Stichprobengrösse $n$ teilt:
$$\text{PDF} = f(x) = \frac{h(x)}{n}$$

\begin{example}{Stetige Häufigkeitsverteilung}\\
\renewcommand{\arraystretch}{2}%
\begin{center}
\begin{tabular}{|c|c|c|c|c|c|}
\hline
Klassen & [100,200) & [200,500) & [500,800) & [800,1000) & Total \\
\hline
$h_i$ & 35 & 182 & 317 & 84 & 618 \\
\hline
$f_i$ & $\frac{35}{618}$ & $\frac{182}{618}$ & $\frac{317}{618}$ & $\frac{84}{618}$ & Area = 1 \\
\hline
$d_i$ & 100 & 300 & 300 & 200 & \\
\hline
$h(x)$ & $\frac{35}{100}$ & $\frac{182}{300}$ & $\frac{317}{300}$ & $\frac{84}{200}$ & \\
\hline
$f(x)$ & $\frac{35}{100 \cdot 618}$ & $\frac{182}{300 \cdot 618}$ & $\frac{317}{300 \cdot 618}$ & $\frac{84}{200 \cdot 618}$ & \\
\hline
\end{tabular}
\end{center}
\end{example}
\end{definition}

\begin{KR}{Berechnung von PDF und CDF für klassierte Daten}
\begin{enumerate}
    \item PDF Berechnung:
        \begin{enumerate}
            \item Bestimme für jede Klasse:
                \begin{itemize}
                    \item Absolute Häufigkeit $h_i$
                    \item Klassenbreite $d_i$
                \end{itemize}
            \item Berechne Häufigkeitsdichte:
                \begin{itemize}
                    \item $h(x) = \frac{h_i}{d_i}$
                \end{itemize}
            \item Berechne relative Häufigkeitsdichte:
                \begin{itemize}
                    \item $f(x) = \frac{h(x)}{n}$
                \end{itemize}
        \end{enumerate}
    \item CDF Berechnung:
        \begin{enumerate}
            \item Bestimme kumulative Häufigkeiten $H_i$
            \item Teile durch Stichprobengröße:
                \begin{itemize}
                    \item $F(x) = \frac{H(x)}{n}$
                \end{itemize}
        \end{enumerate}
\end{enumerate}
\end{KR}

\begin{definition}{Varianz und Kovarianz}\\
\textbf{Varianz $s_x^2, s_y^2$}:
$$(s_x)^2 = \overline{x^2} - \bar{x}^2, \quad (s_y)^2 = \overline{y^2} - \bar{y}^2$$

\textbf{Kovarianz $s_{xy}$}:
$$s_{xy} = \frac{1}{n}\sum_{i=1}^{n}(x_i - \bar{x})(y_i - \bar{y}), \quad s_{xy} = \overline{xy} - \bar{x} \cdot \bar{y}$$
\end{definition}

\begin{KR}{Berechnung der Kovarianz}
\begin{enumerate}
    \item Methode 1 (direkte Formel):
        \begin{enumerate}
            \item Berechne Mittelwerte $\bar{x}$ und $\bar{y}$
            \item Für jedes Paar $(x_i,y_i)$:
                \begin{itemize}
                    \item Berechne $(x_i - \bar{x})(y_i - \bar{y})$
                \end{itemize}
            \item Summiere alle Produkte
            \item Teile durch $n$
        \end{enumerate}
    \item Methode 2 (schnellere Berechnung):
        \begin{enumerate}
            \item Berechne $\overline{xy}$ (Mittelwert der Produkte)
            \item Berechne $\bar{x} \cdot \bar{y}$
            \item Kovarianz = $\overline{xy} - \bar{x} \cdot \bar{y}$
        \end{enumerate}
\end{enumerate}
\end{KR}

\begin{example2}{Berechnung von Kovarianz und Korrelation}
Gegeben seien die Wertepaare:
$$(1,2), (2,4), (3,5), (4,8)$$

\textbf{Schritt 1:} Mittelwerte berechnen
$$\bar{x} = \frac{1+2+3+4}{4} = 2.5, \quad \bar{y} = \frac{2+4+5+8}{4} = 4.75$$

\textbf{Schritt 2:} Kovarianz berechnen
\begin{itemize}
    \item $\overline{xy} = \frac{2+8+15+32}{4} = 14.25$
    \item $\bar{x} \cdot \bar{y} = 2.5 \cdot 4.75 = 11.875$
    \item $s_{xy} = 14.25 - 11.875 = 2.375$
\end{itemize}

\textbf{Schritt 3:} Korrelationskoeffizient berechnen
\begin{itemize}
    \item $s_x^2 = \frac{1+4+9+16}{4} - 2.5^2 = 1.25$
    \item $s_y^2 = \frac{4+16+25+64}{4} - 4.75^2 = 5.6875$
    \item $r_{xy} = \frac{2.375}{\sqrt{1.25} \cdot \sqrt{5.6875}} = 0.894$
\end{itemize}
\end{example2}

\begin{concept}{Abkürzungen}\\
$$\bar{x} = \frac{1}{n}\sum_{i=1}^{n} x_i \quad \text{(Mittelwert der x-Werte)}$$
$$\bar{y} = \frac{1}{n}\sum_{i=1}^{n} y_i \quad \text{(Mittelwert der y-Werte)}$$
$$\overline{xy} = \frac{1}{n}\sum_{i=1}^{n} x_i \cdot y_i \quad \text{(Mittelwert der Produkte)}$$
\end{concept}

\begin{definition}{Rang-Varianz und Kovarianz}\\
\textbf{Varianz (Ränge) $(s_{rg(x)})^2, (s_{rg(y)})^2$}:
$$(s_{rg(x)})^2 = \overline{rg(x)^2} - (\overline{rg(x)})^2, \quad (s_{rg(y)})^2 = \overline{rg(y)^2} - (\overline{rg(y)})^2$$

\textbf{Kovarianz (Ränge) $s_{rg(xy)}$}:
$$s_{rg(xy)} = \overline{rg(xy)} - \overline{rg(x)} \cdot \overline{rg(y)} = \overline{rg(xy)} - \frac{(n+1)^2}{4}$$
\end{definition}

\begin{KR}{Rangberechnung und Bindungen}
\begin{enumerate}
    \item Sortiere die Werte aufsteigend
    \item Weise Ränge zu:
        \begin{itemize}
            \item Kleinster Wert: Rang 1
            \item Zweitkleinster: Rang 2
            \item usw.
        \end{itemize}
    \item Bei Bindungen (gleiche Werte):
        \begin{enumerate}
            \item Identifiziere gleiche Werte
            \item Berechne Durchschnittsrang:
                \begin{itemize}
                    \item $\text{Durchschnittsrang} = \frac{\text{Summe der Rangplätze}}{\text{Anzahl gebundener Werte}}$
                \end{itemize}
            \item Weise allen gleichen Werten diesen Rang zu
        \end{enumerate}
\end{enumerate}
\end{KR}

\begin{example2}{Rangberechnung mit Bindungen}
Gegeben sei die Datenreihe: 3, 7, 7, 4, 9, 7, 2

\textbf{Schritt 1:} Sortieren
$$2, 3, 4, 7, 7, 7, 9$$

\textbf{Schritt 2:} Ränge zuweisen
\begin{itemize}
    \item 2: Rang 1
    \item 3: Rang 2
    \item 4: Rang 3
    \item 7: Durchschnittsrang $\frac{4+5+6}{3} = 5$
    \item 9: Rang 7
\end{itemize}

\textbf{Schritt 3:} Finale Rangzuordnung
\begin{center}
\begin{tabular}{|c|c|c|c|c|c|c|c|}
\hline
Wert & 3 & 7 & 7 & 4 & 9 & 7 & 2 \\
\hline
Rang & 2 & 5 & 5 & 3 & 7 & 5 & 1 \\
\hline
\end{tabular}
\end{center}
\end{example2}

\begin{definition}{Der Korrelationskoeffizient (Pearson) $r_{xy}$}\\
$$r_{xy} = \frac{s_{xy}}{s_x \cdot s_y} = \frac{\overline{xy} - \bar{x} \cdot \bar{y}}{\sqrt{\overline{x^2} - \bar{x}^2} \cdot \sqrt{\overline{y^2} - \bar{y}^2}}$$

Ist der Korrelationskoeffizient $r_{xy}$:
\begin{itemize}
  \item $r_{xy} \approx 1 \rightarrow$ starker positiver linearer Zusammenhang
  \item $r_{xy} \approx -1 \rightarrow$ starker negativer linearer Zusammenhang
  \item $r_{xy} \approx 0 \rightarrow$ keine lineare Korrelation
\end{itemize}
\end{definition}

\begin{example2}{Interpretation des Korrelationskoeffizienten}
Verschiedene Datensätze mit jeweils 20 $(x,y)$-Paaren:

\textbf{Fall A:} $r_{xy} = 0.95$
\begin{itemize}
    \item Starker positiver linearer Zusammenhang
    \item $y$ steigt fast proportional mit $x$
    \item Nur geringe Streuung um die Regressionsgerade
\end{itemize}

\textbf{Fall B:} $r_{xy} = -0.82$
\begin{itemize}
    \item Starker negativer linearer Zusammenhang
    \item $y$ sinkt mit steigendem $x$
    \item Moderate Streuung vorhanden
\end{itemize}

\textbf{Fall C:} $r_{xy} = 0.12$
\begin{itemize}
    \item Kaum linearer Zusammenhang
    \item Starke Streuung der Punkte
    \item Möglicherweise nichtlinearer Zusammenhang
\end{itemize}
\end{example2}

\begin{KR}{Prüfung auf Scheinkorrelation}
\begin{enumerate}
    \item Betrachte die Datenpunkte im Streudiagramm:
        \begin{itemize}
            \item Gibt es Ausreißer?
            \item Ist der Zusammenhang wirklich linear?
        \end{itemize}
    \item Überlege fachlich:
        \begin{itemize}
            \item Gibt es plausible Kausalität?
            \item Könnte ein drittes Merkmal beide beeinflussen?
        \end{itemize}
    \item Prüfe Teilstichproben:
        \begin{itemize}
            \item Bleibt Korrelation in Untergruppen bestehen?
            \item Ändert sich die Stärke deutlich?
        \end{itemize}
    \item Bei Zweifeln:
        \begin{itemize}
            \item Spearman-Korrelation prüfen
            \item Weitere Merkmale einbeziehen
            \item Fachexperten konsultieren
        \end{itemize}
\end{enumerate}
\end{KR}

\begin{remark}{Bemerkungen}\\
Auch wenn zwischen zwei Grössen eine Korrelation besteht, so muss das noch lange nicht einen \emph{kausalen Zusammenhang} bedeuten. Man spricht von \emph{Scheinkorrelation} wenn:
\begin{itemize}
    \item Ein drittes Merkmal beide beeinflusst
    \item Der Zusammenhang zufällig ist
    \item Ausreißer das Ergebnis verzerren
    \item Ein nichtlinearer Zusammenhang vorliegt
\end{itemize}
\end{remark}

[Content continues in next part...]