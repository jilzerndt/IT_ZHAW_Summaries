\section{Deskriptive Statistik Multivarianz}

\begin{concept}{Graphische Darstellung}
\begin{itemize}
 \item \textbf{Form} \hspace{1.5cm} linear / gekrümmt
    \begin{itemize}
        \item Linear: Punkte streuen um Gerade
        \item Gekrümmt: Systematische Abweichung von Gerade
    \end{itemize}
 \item \textbf{Richtung} \hspace{0.85cm} positiver / negativer Zusammenhang
    \begin{itemize}
        \item Positiv: $y$ steigt mit $x$
        \item Negativ: $y$ fällt mit steigendem $x$
    \end{itemize}
 \item \textbf{Stärke} \hspace{1.2cm} starke / schwache Streuung
    \begin{itemize}
        \item Stark: Punkte nahe an Linie/Kurve
        \item Schwach: Große Streuung um Trend
    \end{itemize}
\end{itemize}
\end{concept}

\begin{definition}{Korrelationskoeffizient (Spearman) $r_{sp}$}
$$r_{sp} = \frac{s_{rg(xy)}}{s_{rg(x)} \cdot s_{rg(y)}} = \frac{\overline{rg(xy)} - \overline{rg(x)} \cdot \overline{rg(y)}}{\sqrt{\overline{rg(x)^2} - (\overline{rg(x)})^2} \cdot \sqrt{\overline{rg(y)^2} - (\overline{rg(y)})^2}}$$

Vereinfachte Formel, sofern \emph{alle Ränge unterschiedlich} sind:
$$r_{sp} = 1 - \frac{6 \cdot \sum_{i=1}^n d_i^2}{n \cdot (n^2 - 1)}, \quad \text{mit } d_i = rg(x_i) - rg(y_i)$$

\textbf{Ränge}\\
Der Rang $rg(x_i)$ des Stichprobenwertes $x_i$ ist definiert als der Index von $x_i$ in der nach der Grösse geordneten Stichprobe.

\begin{center}
\begin{tabular}{|c|c|c|c|c|c|c|}
\hline
$i$ & 1 & 2 & 3 & 4 & 5 & 6 \\
\hline
$x_i$ & 23 & 27 & 35 & 35 & 42 & 59 \\
\hline
$rg(x_i)$ & 1 & 2 & 3.5 & 3.5 & 5 & 6 \\
\hline
\end{tabular}
\end{center}
\end{definition}

\begin{KR}{Berechnung des Spearman-Korrelationskoeffizienten}
\begin{enumerate}
    \item Weise beiden Merkmalen Ränge zu:
        \begin{itemize}
            \item Sortiere x-Werte, vergebe Ränge
            \item Sortiere y-Werte, vergebe Ränge
            \item Bei Bindungen: Durchschnittsränge
        \end{itemize}
    \item Falls keine Bindungen vorhanden:
        \begin{enumerate}
            \item Berechne Rangdifferenzen $d_i$
            \item Quadriere Differenzen $d_i^2$
            \item Summiere quadrierte Differenzen
            \item Verwende Formel: $r_{sp} = 1 - \frac{6 \sum d_i^2}{n(n^2-1)}$
        \end{enumerate}
    \item Bei Bindungen:
        \begin{enumerate}
            \item Berechne Rangmittelwerte
            \item Berechne Rangvarianzen und -kovarianz
            \item Verwende allgemeine Formel
        \end{enumerate}
\end{enumerate}
\end{KR}

\begin{example2}{Vergleich Pearson und Spearman}
Gegeben seien die Wertepaare:
$$(1,1), (2,4), (3,9), (4,16), (5,25)$$

\textbf{Pearson-Korrelation:}
\begin{itemize}
    \item Zeigt starken linearen Zusammenhang
    \item $r_{xy} = 0.975$
\end{itemize}

\textbf{Spearman-Korrelation:}
\begin{itemize}
    \item Perfekter monotoner Zusammenhang
    \item $r_{sp} = 1.000$
\end{itemize}

\textbf{Vergleich:}
\begin{itemize}
    \item Pearson erfasst nur linearen Zusammenhang
    \item Spearman erfasst jeden monotonen Zusammenhang
    \item Hier: Quadratischer Zusammenhang
    \item Spearman robuster gegen Ausreißer
\end{itemize}
\end{example2}

\begin{remark}{Wahl des Korrelationskoeffizienten}\\
\begin{itemize}
    \item \textbf{Pearson verwenden wenn:}
        \begin{itemize}
            \item Linearer Zusammenhang vermutet
            \item Keine/wenige Ausreißer
            \item Metrische Daten
        \end{itemize}
    \item \textbf{Spearman verwenden wenn:}
        \begin{itemize}
            \item Nichtlinearer monotoner Zusammenhang
            \item Ausreißer vorhanden
            \item Ordinale Daten
            \item Robustheit wichtig
        \end{itemize}
\end{itemize}
\end{remark}