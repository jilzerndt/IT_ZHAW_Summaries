\section{Deskriptive Statistik Multivarianz}

\subsection{Bivariate Daten}

\begin{definition}{Bivariate Daten (Darstellung)}\\
Die Darstellung hängt von den Merkmalstypen ab:
\begin{itemize}
  \item 2x kategoriell $\rightarrow$ Kontingenztabelle + Mosaikplot
  \item 1x kategoriell + 1x metrisch $\rightarrow$ Boxplot oder Stripchart
  \item 2x metrisch $\rightarrow$ Streudiagramm
\end{itemize}
\end{definition}

\subsubsection{Grafische Darstellung}

\begin{KR}{Erstellen einer Kontingenztabelle} %TODO: is this in the correct place?
\begin{enumerate}
    \item Identifiziere die Ausprägungen beider kategorieller Merkmale
    \item Erstelle eine Tabelle mit:
        \begin{itemize}
            \item Zeilen für Ausprägungen des ersten Merkmals
            \item Spalten für Ausprägungen des zweiten Merkmals
        \end{itemize}
    \item Zähle die Häufigkeiten für jede Kombination
    \item Füge Randsummen für Zeilen und Spalten hinzu
    \item Optional: Berechne relative Häufigkeiten
\end{enumerate}
\end{KR}



\subsubsection{Korrelation}

\paragraph{Varianz und Kovarianz}

\begin{concept}{Abkürzungen}\\
$$\bar{x} = \frac{1}{n}\sum_{i=1}^{n} x_i \quad \text{(Mittelwert der x-Werte)}$$
$$\bar{y} = \frac{1}{n}\sum_{i=1}^{n} y_i \quad \text{(Mittelwert der y-Werte)}$$
$$\overline{xy} = \frac{1}{n}\sum_{i=1}^{n} x_i \cdot y_i \quad \text{(Mittelwert der Produkte)}$$
\end{concept}

\begin{definition}{Varianz und Kovarianz}\\
\textbf{Varianz $s_x^2, s_y^2$}:
$$(s_x)^2 = \overline{x^2} - \bar{x}^2, \quad (s_y)^2 = \overline{y^2} - \bar{y}^2$$

\textbf{Kovarianz $s_{xy}$}:
$$s_{xy} = \frac{1}{n}\sum_{i=1}^{n}(x_i - \bar{x})(y_i - \bar{y}), \quad s_{xy} = \overline{xy} - \bar{x} \cdot \bar{y}$$
\end{definition}

\begin{KR}{Berechnung der Kovarianz}
\begin{enumerate}
    \item Methode 1 (direkte Formel):
        \begin{enumerate}
            \item Berechne Mittelwerte $\bar{x}$ und $\bar{y}$
            \item Für jedes Paar $(x_i,y_i)$:
                \begin{itemize}
                    \item Berechne $(x_i - \bar{x})(y_i - \bar{y})$
                \end{itemize}
            \item Summiere alle Produkte
            \item Teile durch $n$
        \end{enumerate}
    \item Methode 2 (schnellere Berechnung):
        \begin{enumerate}
            \item Berechne $\overline{xy}$ (Mittelwert der Produkte)
            \item Berechne $\bar{x} \cdot \bar{y}$
            \item Kovarianz = $\overline{xy} - \bar{x} \cdot \bar{y}$
        \end{enumerate}
\end{enumerate}
\end{KR}

\begin{example2}{Berechnung von Kovarianz und Korrelation}
Gegeben seien die Wertepaare:
$$(1,2), (2,4), (3,5), (4,8)$$

\textbf{Schritt 1:} Mittelwerte berechnen
$$\bar{x} = \frac{1+2+3+4}{4} = 2.5, \quad \bar{y} = \frac{2+4+5+8}{4} = 4.75$$

\textbf{Schritt 2:} Kovarianz berechnen
\begin{itemize}
    \item $\overline{xy} = \frac{2+8+15+32}{4} = 14.25$
    \item $\bar{x} \cdot \bar{y} = 2.5 \cdot 4.75 = 11.875$
    \item $s_{xy} = 14.25 - 11.875 = 2.375$
\end{itemize}

\textbf{Schritt 3:} Korrelationskoeffizient berechnen
\begin{itemize}
    \item $s_x^2 = \frac{1+4+9+16}{4} - 2.5^2 = 1.25$
    \item $s_y^2 = \frac{4+16+25+64}{4} - 4.75^2 = 5.6875$
    \item $r_{xy} = \frac{2.375}{\sqrt{1.25} \cdot \sqrt{5.6875}} = 0.894$
\end{itemize}
\end{example2}



\begin{definition}{Rang-Varianz und Kovarianz}\\
\textbf{Varianz (Ränge) $(s_{rg(x)})^2, (s_{rg(y)})^2$}:
$$(s_{rg(x)})^2 = \overline{rg(x)^2} - (\overline{rg(x)})^2, \quad (s_{rg(y)})^2 = \overline{rg(y)^2} - (\overline{rg(y)})^2$$

\textbf{Kovarianz (Ränge) $s_{rg(xy)}$}:
$$s_{rg(xy)} = \overline{rg(xy)} - \overline{rg(x)} \cdot \overline{rg(y)} = \overline{rg(xy)} - \frac{(n+1)^2}{4}$$
\end{definition}

\begin{KR}{Rangberechnung und Bindungen}
\begin{enumerate}
    \item Sortiere die Werte aufsteigend
    \item Weise Ränge zu:
        \begin{itemize}
            \item Kleinster Wert: Rang 1
            \item Zweitkleinster: Rang 2
            \item usw.
        \end{itemize}
    \item Bei Bindungen (gleiche Werte):
        \begin{enumerate}
            \item Identifiziere gleiche Werte
            \item Berechne Durchschnittsrang:
                \begin{itemize}
                    \item $\text{Durchschnittsrang} = \frac{\text{Summe der Rangplätze}}{\text{Anzahl gebundener Werte}}$
                \end{itemize}
            \item Weise allen gleichen Werten diesen Rang zu
        \end{enumerate}
\end{enumerate}
\end{KR}

\begin{example2}{Rangberechnung mit Bindungen}
Gegeben sei die Datenreihe: 3, 7, 7, 4, 9, 7, 2

\textbf{Schritt 1:} Sortieren
$$2, 3, 4, 7, 7, 7, 9$$

\textbf{Schritt 2:} Ränge zuweisen
\begin{itemize}
    \item 2: Rang 1
    \item 3: Rang 2
    \item 4: Rang 3
    \item 7: Durchschnittsrang $\frac{4+5+6}{3} = 5$
    \item 9: Rang 7
\end{itemize}

\textbf{Schritt 3:} Finale Rangzuordnung
\begin{center}
\begin{tabular}{|c|c|c|c|c|c|c|c|}
\hline
Wert & 3 & 7 & 7 & 4 & 9 & 7 & 2 \\
\hline
Rang & 2 & 5 & 5 & 3 & 7 & 5 & 1 \\
\hline
\end{tabular}
\end{center}
\end{example2}

\paragraph{Korrelationskoeffizienten}

\begin{definition}{Der Korrelationskoeffizient (Pearson) $r_{xy}$}\\
$$r_{xy} = \frac{s_{xy}}{s_x \cdot s_y} = \frac{\overline{xy} - \bar{x} \cdot \bar{y}}{\sqrt{\overline{x^2} - \bar{x}^2} \cdot \sqrt{\overline{y^2} - \bar{y}^2}}$$

Ist der Korrelationskoeffizient $r_{xy}$:
\begin{itemize}
  \item $r_{xy} \approx 1 \rightarrow$ starker positiver linearer Zusammenhang
  \item $r_{xy} \approx -1 \rightarrow$ starker negativer linearer Zusammenhang
  \item $r_{xy} \approx 0 \rightarrow$ keine lineare Korrelation
\end{itemize}
\end{definition}

\begin{example2}{Interpretation des Korrelationskoeffizienten}
Verschiedene Datensätze mit jeweils 20 $(x,y)$-Paaren:

\textbf{Fall A:} $r_{xy} = 0.95$
\begin{itemize}
    \item Starker positiver linearer Zusammenhang
    \item $y$ steigt fast proportional mit $x$
    \item Nur geringe Streuung um die Regressionsgerade
\end{itemize}

\textbf{Fall B:} $r_{xy} = -0.82$
\begin{itemize}
    \item Starker negativer linearer Zusammenhang
    \item $y$ sinkt mit steigendem $x$
    \item Moderate Streuung vorhanden
\end{itemize}

\textbf{Fall C:} $r_{xy} = 0.12$
\begin{itemize}
    \item Kaum linearer Zusammenhang
    \item Starke Streuung der Punkte
    \item Möglicherweise nichtlinearer Zusammenhang
\end{itemize}
\end{example2}



\begin{definition}{Korrelationskoeffizient (Spearman) $r_{sp}$}
$$r_{sp} = \frac{s_{rg(xy)}}{s_{rg(x)} \cdot s_{rg(y)}} = \frac{\overline{rg(xy)} - \overline{rg(x)} \cdot \overline{rg(y)}}{\sqrt{\overline{rg(x)^2} - (\overline{rg(x)})^2} \cdot \sqrt{\overline{rg(y)^2} - (\overline{rg(y)})^2}}$$

Vereinfachte Formel, sofern \emph{alle Ränge unterschiedlich} sind:
$$r_{sp} = 1 - \frac{6 \cdot \sum_{i=1}^n d_i^2}{n \cdot (n^2 - 1)}, \quad \text{mit } d_i = rg(x_i) - rg(y_i)$$

\textbf{Ränge}\\
Der Rang $rg(x_i)$ des Stichprobenwertes $x_i$ ist definiert als der Index von $x_i$ in der nach der Grösse geordneten Stichprobe.

\begin{center}
\begin{tabular}{|c|c|c|c|c|c|c|}
\hline
$i$ & 1 & 2 & 3 & 4 & 5 & 6 \\
\hline
$x_i$ & 23 & 27 & 35 & 35 & 42 & 59 \\
\hline
$rg(x_i)$ & 1 & 2 & 3.5 & 3.5 & 5 & 6 \\
\hline
\end{tabular}
\end{center}
\end{definition}

\begin{KR}{Berechnung des Spearman-Korrelationskoeffizienten}
\begin{enumerate}
    \item Weise beiden Merkmalen Ränge zu:
        \begin{itemize}
            \item Sortiere x-Werte, vergebe Ränge
            \item Sortiere y-Werte, vergebe Ränge
            \item Bei Bindungen: Durchschnittsränge
        \end{itemize}
    \item Falls keine Bindungen vorhanden:
        \begin{enumerate}
            \item Berechne Rangdifferenzen $d_i$
            \item Quadriere Differenzen $d_i^2$
            \item Summiere quadrierte Differenzen
            \item Verwende Formel: $r_{sp} = 1 - \frac{6 \sum d_i^2}{n(n^2-1)}$
        \end{enumerate}
    \item Bei Bindungen:
        \begin{enumerate}
            \item Berechne Rangmittelwerte
            \item Berechne Rangvarianzen und -kovarianz
            \item Verwende allgemeine Formel
        \end{enumerate}
\end{enumerate}
\end{KR}

\begin{example2}{Vergleich Pearson und Spearman}
Gegeben seien die Wertepaare:
$$(1,1), (2,4), (3,9), (4,16), (5,25)$$

\textbf{Pearson-Korrelation:}
\begin{itemize}
    \item Zeigt starken linearen Zusammenhang
    \item $r_{xy} = 0.975$
\end{itemize}

\textbf{Spearman-Korrelation:}
\begin{itemize}
    \item Perfekter monotoner Zusammenhang
    \item $r_{sp} = 1.000$
\end{itemize}

\textbf{Vergleich:}
\begin{itemize}
    \item Pearson erfasst nur linearen Zusammenhang
    \item Spearman erfasst jeden monotonen Zusammenhang
    \item Hier: Quadratischer Zusammenhang
    \item Spearman robuster gegen Ausreißer
\end{itemize}
\end{example2}

\begin{remark}{Wahl des Korrelationskoeffizienten}\\
\begin{itemize}
    \item \textbf{Pearson verwenden wenn:}
        \begin{itemize}
            \item Linearer Zusammenhang vermutet
            \item Keine/wenige Ausreißer
            \item Metrische Daten
        \end{itemize}
    \item \textbf{Spearman verwenden wenn:}
        \begin{itemize}
            \item Nichtlinearer monotoner Zusammenhang
            \item Ausreißer vorhanden
            \item Ordinale Daten
            \item Robustheit wichtig
        \end{itemize}
\end{itemize}
\end{remark}

\subsubsection{Grenzen der Korrelation}

\begin{KR}{Prüfung auf Scheinkorrelation}
    \begin{enumerate}
        \item Betrachte die Datenpunkte im Streudiagramm:
            \begin{itemize}
                \item Gibt es Ausreißer?
                \item Ist der Zusammenhang wirklich linear?
            \end{itemize}
        \item Überlege fachlich:
            \begin{itemize}
                \item Gibt es plausible Kausalität?
                \item Könnte ein drittes Merkmal beide beeinflussen?
            \end{itemize}
        \item Prüfe Teilstichproben:
            \begin{itemize}
                \item Bleibt Korrelation in Untergruppen bestehen?
                \item Ändert sich die Stärke deutlich?
            \end{itemize}
        \item Bei Zweifeln:
            \begin{itemize}
                \item Spearman-Korrelation prüfen
                \item Weitere Merkmale einbeziehen
                \item Fachexperten konsultieren
            \end{itemize}
    \end{enumerate}
    \end{KR}
    
    \begin{remark}{Bemerkungen}\\
    Auch wenn zwischen zwei Grössen eine Korrelation besteht, so muss das noch lange nicht einen \emph{kausalen Zusammenhang} bedeuten. Man spricht von \emph{Scheinkorrelation} wenn:
    \begin{itemize}
        \item Ein drittes Merkmal beide beeinflusst
        \item Der Zusammenhang zufällig ist
        \item Ausreißer das Ergebnis verzerren
        \item Ein nichtlinearer Zusammenhang vorliegt
    \end{itemize}
    \end{remark}

\subsection{Mehrere Merkmale}
%TODO: add information from script