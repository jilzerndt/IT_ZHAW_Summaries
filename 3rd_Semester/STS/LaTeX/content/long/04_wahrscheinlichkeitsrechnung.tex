\section{Elementare Wahrscheinlichkeitsrechnung}
%TODO: remove redundant information
\subsection{Diskrete Wahrscheinlichkeitsräume}

\begin{definition}{Ergebnisraum und Laplace-Raum}
Der \textbf{Ergebnisraum} $\Omega$ ist die Menge aller möglichen Ergebnisse des Zufallsexperiments. Die \textbf{Zähldichte} $\rho: \Omega \rightarrow[0,1]$ ordnet jedem Ereignis seine Wahrscheinlichkeit zu.

Für einen \textbf{Laplace-Raum} $(\Omega, P)$ gilt:
$$P(M)=\frac{|M|}{|\Omega|}$$

\textbf{Parameter:}
\begin{itemize}
    \item $\Omega$ = Ergebnisraum (Menge aller möglichen Ergebnisse)
    \item $P(M)$ = Wahrscheinlichkeit des Ereignisses $M$
    \item $|M|$ = Anzahl der für $M$ günstigen Ergebnisse
    \item $|\Omega|$ = Anzahl aller möglichen Ergebnisse
\end{itemize}
\end{definition}

\begin{concept}{Begriffe der Wahrscheinlichkeitsrechnung}
   %TODO: add
\end{concept}

\begin{theorem}{Eigenschaften von diskreten Wahrscheinlichkeitsräumen}
   %TODO: add
\end{theorem}

\begin{formula}{Wahrschinlichkeits Ausdrücke}
   \begin{itemize}
      \item $P(A)$ = Wahrscheinlichkeit von Ereignis $A$
      \item $P(B)$ = Wahrscheinlichkeit von Ereignis $B$
      \item $P(\bar{A})$ = Wahrscheinlichkeit des Gegenereignisses von $A$
      \item $P(B|A)$ = Wahrscheinlichkeit von $B$ unter der Bedingung dass $A$ eingetreten ist
      \item $P(B|\bar{A})$ = Wahrscheinlichkeit von $B$ unter der Bedingung dass $A$ nicht eingetreten ist
      \item $P(A \cap B)$ = Wahrscheinlichkeit dass beide Ereignisse eintreten
      \item $P(A \cup B)$ = Wahrscheinlichkeit dass mindestens eines der Ereignisse eintritt
   \end{itemize}
\end{formula}

\begin{corollary}{Wahrscheinlichkeitsregeln}
\begin{itemize}
    \item \textbf{Additionssatz:} $P(A \cup B) = P(A) + P(B) - P(A \cap B)$
    \item \textbf{Komplementärregel:} $P(\bar{A}) = 1 - P(A)$
    \item \textbf{Multiplikationssatz:} $P(A \cap B) = P(A) \cdot P(B|A) = P(B) \cdot P(A|B)$
    \item \textbf{Totale Wahrscheinlichkeit:} $P(B) = P(A) \cdot P(B|A) + P(\bar{A}) \cdot P(B|\bar{A})$
    \item \textbf{Satz von Bayes:} $P(A|B) = \frac{P(A) \cdot P(B|A)}{P(B)}$
    \item \textbf{Bedingte Wahrscheinlichkeit:} $P(B|A) = \frac{P(B \cap A)}{P(A)}$
    \item \textbf{Stochastische Unabhängigkeit:} $P(A \cap B) = P(A) \cdot P(B)$
    \item \textbf{Stochastische Unabhängigkeit (Zufallsvariablen):} $P(X=x, Y=y) = P(X=x) \cdot P(Y=y)$
\end{itemize}   
\end{corollary}

\subsubsection{Strategien zur Berechnung von Wahrscheinlichkeiten}

\begin{concept}{Grundlegende Strategien}
\begin{itemize}
    \item \textbf{Aufteilung in Kombinationen:} Komplexe Probleme in einfachere Teilprobleme zerlegen
    \item \textbf{Berechnung über Inverse:} Manchmal ist es einfacher, die Gegenwahrscheinlichkeit zu berechnen
    \item \textbf{Prozentrechnung:} Wahrscheinlichkeit / Gesamt-Wahrscheinlichkeit $\cdot$ 100\%
    \item \textbf{Vierfeldertafel:} Zur Übersicht bei zwei binären Merkmalen
\end{itemize}
\end{concept}

\begin{KR}{Grundschritte der Wahrscheinlichkeitsberechnung}\\
1. \textbf{Ergebnisraum identifizieren}
   \begin{itemize}
   \item Alle möglichen Ergebnisse auflisten
   \item Prüfen, ob es sich um einen Laplace-Raum handelt
   \end{itemize}

2. \textbf{Ereignis präzisieren}
   \begin{itemize}
   \item Exakte mathematische Beschreibung des gesuchten Ereignisses
   \item Zerlegung in Teilmengen falls nötig
   \end{itemize}

3. \textbf{Berechnungsstrategie wählen}
   \begin{itemize}
   \item Direkte Berechnung: $P(A) = \frac{|A|}{|\Omega|}$
   \item Über Gegenereignis: $P(A) = 1 - P(\bar{A})$
   \item Über bedingte Wahrscheinlichkeit falls abhängig
   \end{itemize}

4. \textbf{Berechnung durchführen}
   \begin{itemize}
   \item Kombinatorische Formeln anwenden
   \item Zwischenergebnisse notieren
   \item Probe durch Plausibilitätskontrolle
   \end{itemize}
\end{KR}

\begin{KR}{Problemlösung mit Gegenereignis}\\
1. \textbf{Prüfe, ob Gegenereignis einfacher ist}
   \begin{itemize}
   \item Original: "Mindestens eine..." oder "Mehr als..."
   \item Gegenereignis: "Keine..." oder "Höchstens..."
   \end{itemize}

2. \textbf{Berechne Wahrscheinlichkeit des Gegenereignis}
   \begin{itemize}
   \item Oft einfacher zu zählen
   \item Weniger Fälle zu berücksichtigen
   \end{itemize}

3. \textbf{Wende Komplementärregel an}
   \begin{itemize}
   \item $P(A) = 1 - P(\bar{A})$
   \item Überprüfe Plausibilität des Ergebnisses
   \end{itemize}
\end{KR}

\subsection{Zufallsvariablen und Verteilungen}
%TODO: add
\subsection{Kenngrössen}

\subsubsection{Erwartungswert und Varianz}

\begin{concept}{Kenngrössen von Zufallsvariablen}
\textbf{Wichtige Eigenschaften:}
\begin{itemize}
    \item \textbf{Erwartungswert:} $E(X + Y) = E(X) + E(Y)$, $E(\alpha X) = \alpha E(X)$
    \item \textbf{Varianz:} $V(X) = E(X^2) - E(X)^2$
    \item \textbf{Standardabweichung:} $S(X) = \sqrt{V(X)}$
    \item \textbf{Lineare Transformation:} $V(\alpha X + \beta) = \alpha^2 \cdot V(X)$
\end{itemize}
\end{concept}

\begin{concept}{Kenngrössen (Varianz und Erwartungswert)}\\
$$E(X + Y) = E(X) + E(Y), \quad E(\alpha X) = \alpha E(X)$$
$$V(X) = E(X^2) - E(X)^2 = \left[\sum_{x\in\mathbb{R}} P(X = x) \cdot x^2\right] - E(X)^2$$
$$V(\alpha X + \beta) = \alpha^2 \cdot V(X), \quad S(X) = \sqrt{V(X)}$$
\\
$E(X)$ = Erwartungswert der Zufallsvariable $X$\\
$V(X)$ = Varianz der Zufallsvariable $X$\\
$S(X)$ = Standardabweichung der Zufallsvariable $X$\\
$\alpha, \beta$ = Konstanten\\
$P(X = x)$ = Wahrscheinlichkeit, dass $X$ den Wert $x$ annimmt\\
$\sum_{x\in\mathbb{R}}$ = Summe über alle möglichen Werte von $x$ in den reellen Zahlen\\
\end{concept}
\begin{definition}{Verteilungen und Erwartungswerte}\\
Für diskrete Verteilungen:
$$
\begin{gathered}
E(X)=\sum_{x \in \mathbb{R}} f(x) \cdot x \\
V(X)=\sum_{x \in \mathbb{R}} f(x) \cdot(x-E(X))^2
\end{gathered}
$$
$E(X)$ = Erwartungswert der Zufallsvariable $X$\\
$V(X)$ = Varianz der Zufallsvariable $X$\\
$f(x)$ = Wahrscheinlichkeitsfunktion\\
$x$ = Mögliche Werte der Zufallsvariable\\

Für stetige Verteilungen:
$$
\begin{gathered}
E(X)=\int_{-\infty}^{\infty} f(x) \cdot x dx \\
V(X)=\int_{-\infty}^{\infty} f(x) \cdot(x-E(X))^2 dx
\end{gathered}
$$
$E(X)$ = Erwartungswert der Zufallsvariable $X$\\
$V(X)$ = Varianz der Zufallsvariable $X$\\
$f(x)$ = Dichtefunktion\\
$x$ = Mögliche Werte der Zufallsvariable\\
\end{definition}

\begin{KR}{Berechnung von Erwartungswert und Varianz}
1. \textbf{Erwartungswert bestimmen}
   \begin{itemize}
   \item Diskret: $E(X) = \sum_{x} x \cdot P(X=x)$
   \item Stetig: $E(X) = \int_{-\infty}^{\infty} x \cdot f(x) \, dx$
   \end{itemize}

2. \textbf{Varianz berechnen (2 Methoden)}
   \begin{itemize}
   \item Direkte Methode: $V(X) = \sum_{x} (x-E(X))^2 \cdot P(X=x)$
   \item Verschiebungssatz: $V(X) = E(X^2) - (E(X))^2$
   \end{itemize}

3. \textbf{Bei Standardabweichung}
   \begin{itemize}
   \item Wurzel aus Varianz ziehen
   \item Einheit beachten (gleich wie Ursprungsdaten)
   \end{itemize}
\end{KR}

\begin{example2}{Erwartungswert bei Würfelspiel}
\textbf{Aufgabe:} Bei einem Würfelspiel gewinnt man:
\begin{itemize}
\item Bei 6: 5€
\item Bei 5: 2€
\item Bei 1-4: verliert man 1€
\end{itemize}

\textbf{Lösung:}
\begin{enumerate}
\item \textbf{Wahrscheinlichkeiten und Werte aufstellen:}
   \begin{itemize}
   \item P(X = 5€) = 1/6
   \item P(X = 2€) = 1/6
   \item P(X = -1€) = 4/6
   \end{itemize}

\item \textbf{Erwartungswert berechnen:}
   $$E(X) = 5 \cdot \frac{1}{6} + 2 \cdot \frac{1}{6} + (-1) \cdot \frac{4}{6} = \frac{5+2-4}{6} = \frac{3}{6} = 0.5$$

\item \textbf{Varianz berechnen:}
   \begin{align*}
   E(X^2) &= 25 \cdot \frac{1}{6} + 4 \cdot \frac{1}{6} + 1 \cdot \frac{4}{6} = \frac{25+4+4}{6} = \frac{33}{6} \\
   V(X) &= E(X^2) - (E(X))^2 = \frac{33}{6} - (\frac{1}{2})^2 = \frac{33}{6} - \frac{1}{4} \approx 5.25
   \end{align*}
\end{enumerate}

\textbf{Interpretation:}
\begin{itemize}
\item Positiver Erwartungswert: Spiel ist langfristig profitabel
\item Hohe Varianz: Große Schwankungen möglich
\end{itemize}
\end{example2}

\begin{example2}{Lotterie mit bedingten Gewinnen}
\textbf{Aufgabe:} Bei einer Lotterie gewinnt man zunächst mit p = 0.1 einen Bonus-Los. Mit diesem Los kann man dann mit p = 0.2 den Hauptpreis von 1000€ gewinnen. Berechne den Erwartungswert.

\textbf{Lösung:}
\begin{enumerate}
\item \textbf{Ereignisbaum erstellen:}
   \begin{itemize}
   \item P(Bonus) = 0.1
   \item P(Hauptgewinn|Bonus) = 0.2
   \end{itemize}

\item \textbf{Mögliche Ausgänge:}
   \begin{itemize}
   \item 1000€: P = 0.1 · 0.2 = 0.02
   \item 0€: P = 0.98
   \end{itemize}

\item \textbf{Erwartungswert:}
   $$E(X) = 1000 \cdot 0.02 + 0 \cdot 0.98 = 20$$
\end{enumerate}
\end{example2}

\begin{KR}{Interpretation von Erwartungswert und Varianz}
1. \textbf{Erwartungswert}
   \begin{itemize}
   \item Langfristiger Durchschnitt
   \item Schwerpunkt der Verteilung
   \item Nicht unbedingt ein möglicher Wert
   \end{itemize}

2. \textbf{Varianz}
   \begin{itemize}
   \item Maß für die Streuung
   \item Quadratische Einheit beachten
   \item Je größer, desto unsicherer die Vorhersage
   \end{itemize}

3. \textbf{Standardabweichung}
   \begin{itemize}
   \item Gleiche Einheit wie Daten
   \item Typische Abweichung vom Mittelwert
   \item Oft für Konfidenzintervalle verwendet
   \end{itemize}
\end{KR}

\begin{example2}{Aktienportfolio}
\textbf{Aufgabe:} Ein Portfolio besteht aus:
\begin{itemize}
\item Aktie A: 60\% Anteil, E(A) = 8\%, V(A) = 25
\item Aktie B: 40\% Anteil, E(B) = 12\%, V(B) = 36
\end{itemize}

\textbf{Lösung:}
\begin{enumerate}
\item \textbf{Erwartungswert des Portfolios:}
   \begin{align*}
   E(P) &= 0.6 \cdot E(A) + 0.4 \cdot E(B) \\
   &= 0.6 \cdot 8\% + 0.4 \cdot 12\% \\
   &= 4.8\% + 4.8\% = 9.6\%
   \end{align*}

\item \textbf{Varianz des Portfolios} (bei Unabhängigkeit):
   \begin{align*}
   V(P) &= (0.6)^2 \cdot V(A) + (0.4)^2 \cdot V(B) \\
   &= 0.36 \cdot 25 + 0.16 \cdot 36 \\
   &= 9 + 5.76 = 14.76
   \end{align*}

\item \textbf{Standardabweichung:}
   $$S(P) = \sqrt{14.76} \approx 3.84\%$$
\end{enumerate}
\end{example2}

\begin{definition}{Kovarianz und Korrelation}
Die \textbf{Kovarianz} zweier Zufallsvariablen ist:
$$Cov(X,Y) = E((X-E(X))(Y-E(Y))) = E(XY) - E(X)E(Y)$$

Der \textbf{Korrelationskoeffizient} ist:
$$\rho_{XY} = \frac{Cov(X,Y)}{\sqrt{V(X)V(Y)}}$$

\textbf{Eigenschaften:}
\begin{itemize}
    \item $-1 \leq \rho_{XY} \leq 1$
    \item $\rho_{XY} = \pm 1$: perfekter linearer Zusammenhang
    \item $\rho_{XY} = 0$: unkorreliert
\end{itemize}
\end{definition}

\begin{KR}{Anwendung von Kovarianz und Korrelation}
1. \textbf{Kovarianz berechnen}
   \begin{itemize}
   \item Direkter Weg: $Cov(X,Y) = E(XY) - E(X)E(Y)$
   \item Alternativ: $\frac{1}{n}\sum(x_i-\bar{x})(y_i-\bar{y})$
   \end{itemize}

2. \textbf{Korrelation bestimmen}
   \begin{itemize}
   \item Kovarianz durch Produkt der Standardabweichungen
   \item Normierung auf [-1,1]
   \end{itemize}

3. \textbf{Interpretation}
   \begin{itemize}
   \item Vorzeichen: Richtung des Zusammenhangs
   \item Betrag: Stärke des Zusammenhangs
   \item Unabhängig von Maßeinheiten
   \end{itemize}
\end{KR}

\begin{example2}{Portfoliorisiko mit Korrelation}
\textbf{Aufgabe:} Zwei Aktien mit:
\begin{itemize}
\item A: E(A) = 10\%, S(A) = 5\%
\item B: E(B) = 8\%, S(B) = 4\%
\item Korrelation: $\rho_{AB} = 0.3$
\end{itemize}

\textbf{Portfolio:} 60\% A, 40\% B

\textbf{Lösung:}
\begin{enumerate}
\item \textbf{Erwartungswert:}
   $$E(P) = 0.6 \cdot 10\% + 0.4 \cdot 8\% = 9.2\%$$

\item \textbf{Varianz mit Korrelation:}
   \begin{align*}
   V(P) &= (0.6)^2V(A) + (0.4)^2V(B) + 2(0.6)(0.4)\rho_{AB}S(A)S(B) \\
   &= 0.36 \cdot (5\%)^2 + 0.16 \cdot (4\%)^2 + 2(0.24)(0.3)(5\%)(4\%) \\
   &= 0.09\% + 0.0256\% + 0.0288\% = 0.1444\%
   \end{align*}

\item \textbf{Standardabweichung:}
   $$S(P) = \sqrt{0.1444\%} \approx 3.8\%$$
\end{enumerate}
\end{example2}

\subsection{Bedingte Wahrscheinlichkeit}

\begin{definition}{Bedingte Wahrscheinlichkeit}
Die bedingte Wahrscheinlichkeit von $B$ unter der Bedingung $A$ ist:
$$P(B|A)=\frac{P(B \cap A)}{P(A)}$$

\textbf{Parameter:}
\begin{itemize}
    \item $P(B \cap A)$ = Wahrscheinlichkeit des Durchschnitts ??
\end{itemize}
\end{definition}

\begin{theorem}{Multiplikationssatz}
$$P(A \cap B)=P(A) \cdot P(B|A)=P(B) \cdot P(A|B)$$

\textbf{Anwendung:}
\begin{itemize}
    \item Berechnung von Schnittwahrscheinlichkeiten
    \item Prüfung auf stochastische Unabhängigkeit
    \item Zerlegung von mehrstufigen Experimenten
\end{itemize}
\end{theorem}

\begin{KR}{Erstellen einer Vierfeldertafel}
1. \textbf{Aufbau der Tabelle}
   \begin{itemize}
   \item Zeilen: Erstes Merkmal (A und nicht A)
   \item Spalten: Zweites Merkmal (B und nicht B)
   \item Randwahrscheinlichkeiten notieren
   \end{itemize}

2. \textbf{Eintragen der Wahrscheinlichkeiten}
   \begin{itemize}
   \item Schnittwahrscheinlichkeiten in die Felder
   \item Zeilensummen = P(A) bzw. P(nicht A)
   \item Spaltensummen = P(B) bzw. P(nicht B)
   \end{itemize}

3. \textbf{Berechnung bedingter Wahrscheinlichkeiten}
   \begin{itemize}
   \item $P(B|A) = \frac{P(A \cap B)}{P(A)}$
   \item $P(A|B) = \frac{P(A \cap B)}{P(B)}$
   \end{itemize}
\end{KR}

\begin{example2}{Medizinischer Test}
\textbf{Aufgabe:} Ein Test auf eine Krankheit hat folgende Eigenschaften:
\begin{itemize}
\item 1\% der Bevölkerung hat die Krankheit
\item Test ist bei Kranken zu 98\% positiv
\item Test ist bei Gesunden zu 95\% negativ
\end{itemize}

\textbf{Lösung mit Vierfeldertafel:}
\begin{center}
\begin{tabular}{|c|c|c|c|}
\hline
 & Test + & Test - & Summe \\
\hline
Krank & 0.0098 & 0.0002 & 0.01 \\
\hline
Gesund & 0.0495 & 0.9405 & 0.99 \\
\hline
Summe & 0.0593 & 0.9407 & 1 \\
\hline
\end{tabular}
\end{center}

\textbf{Berechnung:} Wahrscheinlichkeit krank bei positivem Test:
$$P(\text{krank}|\text{positiv}) = \frac{0.0098}{0.0593} \approx 0.165 = 16.5\%$$
\end{example2}

\subsubsection{Spezielle Sätze}

\begin{theorem}{Satz von der Totalen Wahrscheinlichkeit}
$$P(B)=P(A) \cdot P(B|A)+P(\bar{A}) \cdot P(B|\bar{A})$$

\textbf{Anwendung:}
\begin{itemize}
    \item Berechnung von P(B) durch Fallunterscheidung
    \item Basis für den Satz von Bayes
    \item Wichtig bei Entscheidungsbäumen
\end{itemize}

\end{theorem}

\begin{KR}{Anwendung des Satzes der totalen Wahrscheinlichkeit}
%TODO: add
\end{KR}

\begin{theorem}{Satz von Bayes}
$$P(A|B)=\frac{P(A) \cdot P(B|A)}{P(B)}$$

\textbf{Anwendung:}
\begin{itemize}
    \item Umkehrung bedingter Wahrscheinlichkeiten
    \item Aktualisierung von Wahrscheinlichkeiten
    \item Diagnostische Tests
\end{itemize}
\end{theorem}

\begin{KR}{Anwendung des Satzes von Bayes}
1. \textbf{Identifiziere die bekannten Größen}
   \begin{itemize}
   \item A priori Wahrscheinlichkeit P(A)
   \item Bedingte Wahrscheinlichkeit P(B|A)
   \item Totale Wahrscheinlichkeit P(B)
   \end{itemize}

2. \textbf{Berechne P(B) falls nötig}
   \begin{itemize}
   \item Nutze Satz der totalen Wahrscheinlichkeit
   \item P(B) = P(A)·P(B|A) + P(Ā)·P(B|Ā)
   \end{itemize}

3. \textbf{Berechne P(A|B)}
   \begin{itemize}
   \item Setze in Bayes-Formel ein
   \item Interpretiere das Ergebnis
   \end{itemize}
\end{KR}

\begin{example2}{Qualitätskontrolle}
\textbf{Aufgabe:} Eine Maschine produziert Teile. 
\begin{itemize}
\item 95\% der Teile sind fehlerfrei
\item Ein Test erkennt fehlerhafte Teile zu 98\%
\item Der Test klassifiziert 3\% der guten Teile falsch
\end{itemize}

\textbf{Gesucht:} Wahrscheinlichkeit für tatsächlich fehlerhaftes Teil bei positivem Test

\textbf{Lösung:}
\begin{itemize}
\item P(F) = 0.05 (fehlerhaft)
\item P(T|F) = 0.98 (Test positiv wenn fehlerhaft)
\item P(T|¬F) = 0.03 (Test positiv wenn gut)
\item P(T) = 0.05·0.98 + 0.95·0.03 = 0.0775
\item P(F|T) = $\frac{0.05 \cdot 0.98}{0.0775} \approx 0.632$ = 63.2\%
\end{itemize}
\end{example2}

\subsection{Stochastische Unabhängigkeit}
%TODO: remove redundant information
\begin{theorem}{Stochastische Unabhängigkeit}\\
Zwei Ereignisse $A$ und $B$ heissen stochastisch unabhängig, falls:
$$
P(A \cap B)=P(A) \cdot P(B)
$$

Zwei Zufallsvariablen $X: \Omega \rightarrow \mathbb{R}$ und $Y: \Omega \rightarrow \mathbb{R}$ heissen stochastisch unabhängig, falls:
$$
P(X=x, Y=y)=P(X=x) \cdot P(Y=y), \quad \text{für alle } x, y \in \mathbb{R}
$$
\\
$P(X=x, Y=y)$ = Wahrscheinlichkeit dass $X$ den Wert $x$ und $Y$ den Wert $y$ annimmt\\
$P(X=x)$ = Wahrscheinlichkeit dass $X$ den Wert $x$ annimmt\\
$P(Y=y)$ = Wahrscheinlichkeit dass $Y$ den Wert $y$ annimmt\\
\end{theorem}



\begin{definition}{Stochastische Unabhängigkeit}
Zwei Ereignisse $A$ und $B$ heißen stochastisch unabhängig, falls:
$$P(A \cap B) = P(A) \cdot P(B)$$

Zwei Zufallsvariablen $X$ und $Y$ heißen stochastisch unabhängig, falls für alle $x,y \in \mathbb{R}$:
$$P(X=x, Y=y) = P(X=x) \cdot P(Y=y)$$

\textbf{Eigenschaften:}
\begin{itemize}
    \item Für unabhängige Ereignisse: $P(A|B) = P(A)$
    \item Für unabhängige Zufallsvariablen: $E(XY) = E(X)E(Y)$
    \item Varianz der Summe: $V(X+Y) = V(X) + V(Y)$
\end{itemize}
\end{definition}

\begin{KR}{Prüfung auf stochastische Unabhängigkeit}
1. \textbf{Für Ereignisse}
   \begin{itemize}
   \item Berechne $P(A \cap B)$
   \item Berechne $P(A) \cdot P(B)$
   \item Vergleiche die Werte
   \end{itemize}

2. \textbf{Für Zufallsvariablen}
   \begin{itemize}
   \item Stelle Verbundverteilung auf
   \item Prüfe für alle Wertepaare
   \item Alternative: Prüfe Kovarianz = 0
   \end{itemize}

3. \textbf{Praktische Überlegungen}
   \begin{itemize}
   \item Physikalische/logische Abhängigkeit?
   \item Kausaler Zusammenhang?
   \item Gemeinsame Einflussfaktoren?
   \end{itemize}
\end{KR}

%TODO: add missing formulas/theorems

\begin{example2}{Würfelwurf und Münzwurf}
\textbf{Aufgabe:} Ein Würfel wird geworfen und eine Münze geworfen.
Ereignisse:
\begin{itemize}
\item A: "Würfel zeigt eine 6"
\item B: "Münze zeigt Kopf"
\end{itemize}

\textbf{Lösung:}
\begin{enumerate}
\item \textbf{Einzelwahrscheinlichkeiten:}
   \begin{itemize}
   \item $P(A) = \frac{1}{6}$
   \item $P(B) = \frac{1}{2}$
   \end{itemize}

\item \textbf{Schnittwahrscheinlichkeit:}
   $P(A \cap B) = \frac{1}{12} = \frac{1}{6} \cdot \frac{1}{2} = P(A) \cdot P(B)$

\item \textbf{Schlussfolgerung:} Die Ereignisse sind stochastisch unabhängig
\end{enumerate}
\end{example2}

\begin{example2}{Kartenziehen ohne Zurücklegen}
\textbf{Aufgabe:} Aus einem Kartenspiel werden nacheinander zwei Karten gezogen.
Ereignisse:
\begin{itemize}
\item A: "Erste Karte ist Herz"
\item B: "Zweite Karte ist Herz"
\end{itemize}

\textbf{Lösung:}
\begin{enumerate}
\item \textbf{Wahrscheinlichkeiten:}
   \begin{itemize}
   \item $P(A) = \frac{13}{52} = \frac{1}{4}$
   \item $P(B|A) = \frac{12}{51}$
   \item $P(B|\bar{A}) = \frac{13}{51}$
   \end{itemize}

\item \textbf{Prüfung:}
   $$P(B) = \frac{13}{52} \neq P(B|A)$$

\item \textbf{Schlussfolgerung:} Die Ereignisse sind stochastisch abhängig
\end{enumerate}
\end{example2}



