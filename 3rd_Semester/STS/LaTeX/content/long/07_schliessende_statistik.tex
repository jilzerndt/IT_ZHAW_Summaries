\section{Schliessende Statistik: Parameter- und Intervallschätzung}
%TODO: remove redundant content and add missing information from the script
%TODO: Tabellen aus Skript 

\subsection{Zufallsstichproben}
%TODO: add missing content from the script

\subsection{Parameterschätzungen}

\begin{concept}{Grundlagen der Schätztheorie}\\
Die Schätztheorie befasst sich mit zwei Hauptproblemen:
\begin{itemize}
  \item Punktschätzung: Bestimmung eines einzelnen Schätzwerts
  \item Intervallschätzung: Bestimmung eines Vertrauensbereichs
\end{itemize}

Wichtige Begriffe:
\begin{itemize}
  \item $\theta$: Unbekannter Parameter der Grundgesamtheit
  \item $\Theta$: Schätzfunktion (Zufallsvariable)
  \item $\hat{\theta}$: Schätzwert (konkreter Wert)
  \item $n$: Stichprobenumfang
\end{itemize}
\end{concept}

\subsubsection{Schätzfunktionen}
%TODO: add missing content from the script

\subsubsection{Kriterien für eine optimale Schätzfunktion}

\begin{definition}{Erwartungstreue Schätzfunktion}\\
Eine Schätzfunktion $\Theta$ eines Parameters $\theta$ heisst erwartungstreu, wenn:
$$
E(\Theta)=\theta
$$
$E(\Theta)$: Erwartungswert der Schätzfunktion\\
$\theta$: Wahrer Parameter der Grundgesamtheit
\end{definition}

\begin{definition}{Effizienz einer Schätzfunktion}\\
Gegeben sind zwei erwartungstreue Schätzfunktionen $\Theta_1$ und $\Theta_2$ desselben Parameters $\theta$. Man nennt $\Theta_1$ effizienter als $\Theta_2$, falls:
$$
V(\Theta_1)<V(\Theta_2)
$$
$V(\Theta_1), V(\Theta_2)$: Varianzen der Schätzfunktionen
\end{definition}

\begin{definition}{Konsistenz einer Schätzfunktion}\\
Eine Schätzfunktion $\Theta$ heisst konsistent, wenn:
$$
E(\Theta) \rightarrow \theta \text{ und } V(\Theta) \rightarrow 0 \text{ für } n \rightarrow \infty
$$
$n$: Stichprobenumfang
\end{definition}

\begin{KR}{Prüfen von Schätzfunktionen}\\
1. Erwartungstreue:
   \begin{itemize}
     \item Erwartungswert $E(\Theta)$ berechnen
     \item Mit Parameter $\theta$ vergleichen
     \item Erwartungstreu, wenn $E(\Theta)=\theta$
   \end{itemize}

2. Effizienz:
   \begin{itemize}
     \item Varianzen $V(\Theta_1)$ und $V(\Theta_2)$ berechnen
     \item Varianzen vergleichen
     \item Kleinere Varianz = effizienter
   \end{itemize}

3. Konsistenz:
   \begin{itemize}
     \item Grenzwert für $n \to \infty$ betrachten
     \item $E(\Theta) \to \theta$?
     \item $V(\Theta) \to 0$?
   \end{itemize}
\end{KR}

\begin{example2}{Prüfung auf Erwartungstreue}\\
Gegeben sei die Schätzfunktion $\Theta_1=\frac{1}{3}(2X_1+X_2)$ für den Erwartungswert $\mu$.

1. Erwartungswert berechnen:
   $$E(\Theta_1)=E(\frac{1}{3}(2X_1+X_2))=\frac{1}{3}(2E(X_1)+E(X_2))$$

2. Einsetzen der Erwartungswerte:
   $$E(\Theta_1)=\frac{1}{3}(2\mu+\mu)=\frac{3\mu}{3}=\mu$$

3. Da $E(\Theta_1)=\mu$, ist die Schätzfunktion erwartungstreu.

Effizienzberechnung:
$$
\begin{aligned}
V(\Theta_1) &= V(\frac{1}{3}(2X_1+X_2)) \\
&= \frac{1}{9}V(2X_1+X_2) \\
&= \frac{1}{9}(4V(X_1)+V(X_2)) \\
&= \frac{1}{9}(4\sigma^2+\sigma^2) \\
&= \frac{5\sigma^2}{9}
\end{aligned}
$$
\end{example2}

\begin{example2}{Erwartungstreue Schätzfunktion}\\
Grundgesamtheit mit Erwartungswert $\mu$, Varianz $\sigma^2$ und Zufallsstichprobe $X_1, X_2, X_3$. Die folgende Schätzfunktion ist gegeben:
$$
\Theta_1=\frac{1}{3} \cdot(2X_1+X_2)
$$
$\Theta_1$ = Schätzfunktion\\
$X_1, X_2$ = Zufallsvariablen aus der Stichprobe\\

Ist diese Schätzfunktion erwartungstreu (Parameter: $\mu$)?
$$
\begin{gathered}
E(\Theta_1)=E(\frac{1}{3} \cdot(2X_1+X_2))=\frac{1}{3} \cdot(2E(X_1)+E(X_2)) \\
E(\Theta_1)=\frac{1}{3} \cdot(2\mu+\mu)=\frac{3\mu}{3}=\mu
\end{gathered}
$$
$E(\Theta_1)$ = Erwartungswert der Schätzfunktion\\
$E(X_1), E(X_2)$ = Erwartungswerte der einzelnen Zufallsvariablen\\
$\mu$ = Wahrer Parameterwert\\

Da $E(\Theta_1)=\mu$ ist die Funktion erwartungstreu.
\end{example2}

\begin{example2}{Effizienz einer Schätzfunktion}\\
Grundgesamtheit mit Erwartungswert $\mu$, Varianz $\sigma^2$ und Zufallsstichprobe $X_1, X_2, X_3$. Gegeben ist die Schätzfunktion:
$$
\Theta_1=\frac{1}{3} \cdot(2X_1+X_2)
$$

\textbf{Berechnung der Effizienz:}
$$
\begin{aligned}
V(\Theta_1) &= V(\frac{1}{3} \cdot(2X_1+X_2)) \\
&= \frac{1}{9} \cdot V(2X_1+X_2) \\
&= \frac{1}{9} \cdot (V(2X_1) + V(X_2)) \\
&= \frac{1}{9} \cdot (4V(X_1) + V(X_2)) \\
&= \frac{1}{9} \cdot (4\sigma^2 + \sigma^2) \\
&= \frac{5\sigma^2}{9}
\end{aligned}
$$
\\
$V(\Theta_1)$ = Varianz der Schätzfunktion\\
$V(X_1), V(X_2)$ = Varianzen der einzelnen Zufallsvariablen\\
$\sigma^2$ = Varianz der Grundgesamtheit\\

Die Effizienz der Schätzfunktion ist also $\frac{5\sigma^2}{9}$.
\end{example2}

\subsubsection{Schätzfunktionen für die wichtigsten statistischen Parameter}

\begin{definition}{Erwartungswert und Varianz (Funktion und Wert)}\\
\textbf{Erwartungswert:}
$$
\bar{X}=\frac{1}{n} \cdot \sum_{i=1}^n X_i, \quad \hat{\mu}=\bar{x}=\frac{1}{n} \cdot \sum_{i=1}^n x_i
$$
\\
$\bar{X}$ = Arithmetischer Mittelwert (Zufallsvariable)\\
$\hat{\mu}$ = $\bar{x}$ = Arithmetischer Mittelwert (Stichprobenwert)\\
$n$ = Stichprobenumfang\\
$X_i$ = $i$-te Zufallsvariable\\
$x_i$ = $i$-ter Stichprobenwert\\

\textbf{Varianz:}
$$
S^2=\frac{1}{n-1} \cdot \sum_{i=1}^n (X_i-\bar{X})^2, \quad \hat{\sigma}^2=s^2=\frac{1}{n-1} \cdot \sum_{i=1}^n (x_i-\bar{x})^2
$$
\\
$S^2$ = Stichprobenvarianz (Zufallsvariable)\\
$\hat{\sigma}^2$ = $s^2$ = Stichprobenvarianz (Stichprobenwert)\\
$\bar{X}$ = Arithmetischer Mittelwert (Zufallsvariable)\\
$\bar{x}$ = Arithmetischer Mittelwert (Stichprobenwert)
\end{definition}

\subsubsection{Maximum-Likelihood-Schätzung}

\begin{definition}{Likelyhood-Funktion}\\
Wir betrachten eine Zufallsvariable $X$ und ihre Dichte (PDF) $f_x(x|\theta)$, welche von $x$ und einem oder mehreren Parametern $\theta$ abhängig sind. 

Für eine Stichprobe vom Umfang $n$ mit $x_1,\ldots,x_n$ nennen wir die vom Parameter $\theta$ abhängige Funktion die Likelyhood-Funktion der Stichprobe:
$$
L(\theta)=f_x(x_1|\theta) \cdot f_x(x_2|\theta) \cdot \ldots \cdot f_x(x_n|\theta)
$$
\end{definition}

\begin{concept}{Vorgehen bei Maximum-Likelihood-Schätzung}\\
\begin{enumerate}
  \item Likelyhood-Funktion bestimmen
  \item Maximalstelle der Funktion bestimmen:
        \begin{itemize}
           \item (Partielle) Ableitung $L'(\theta)=0$
        \end{itemize}
\end{enumerate}
\end{concept}

\begin{definition}{Maximum-Likelihood-Schätzung}\\
Die Likelihood-Funktion für eine Stichprobe $x_1,\ldots,x_n$ ist:
$$
L(\theta)=\prod_{i=1}^n f_X(x_i|\theta)
$$
$f_X(x|\theta)$: Wahrscheinlichkeitsdichte\\
$\theta$: zu schätzender Parameter\\
$n$: Stichprobenumfang

Der Maximum-Likelihood-Schätzer maximiert $L(\theta)$ bzw. $\ln(L(\theta))$.
\end{definition}

\begin{KR}{Maximum-Likelihood-Schätzung}\\
1. Likelihood-Funktion aufstellen:
   $$L(\theta)=\prod_{i=1}^n f_X(x_i|\theta)$$

2. Logarithmieren:
   $$\ln(L(\theta))=\sum_{i=1}^n \ln(f_X(x_i|\theta))$$

3. Ableitung Null setzen:
   $$\frac{d}{d\theta}\ln(L(\theta))=0$$

4. Nach $\theta$ auflösen:
   $$\hat{\theta}_{ML}=\text{Lösung}$$

5. Maximum prüfen:
   $$\frac{d^2}{d\theta^2}\ln(L(\theta)) < 0$$
\end{KR}

\begin{example2}{Maximum-Likelihood Normalverteilung}\\
Gegeben sei eine Stichprobe aus einer Normalverteilung $N(\mu,\sigma^2)$.

1. Likelihood-Funktion:
$$L(\mu,\sigma^2)=\prod_{i=1}^n \frac{1}{\sqrt{2\pi\sigma^2}}e^{-\frac{(x_i-\mu)^2}{2\sigma^2}}$$

2. Log-Likelihood:
$$\ln(L)=-\frac{n}{2}\ln(2\pi\sigma^2)-\frac{1}{2\sigma^2}\sum_{i=1}^n(x_i-\mu)^2$$

3. Ableitungen:
$$\frac{\partial \ln(L)}{\partial \mu}=\frac{1}{\sigma^2}\sum_{i=1}^n(x_i-\mu)=0$$
$$\frac{\partial \ln(L)}{\partial \sigma^2}=-\frac{n}{2\sigma^2}+\frac{1}{2(\sigma^2)^2}\sum_{i=1}^n(x_i-\mu)^2=0$$

4. ML-Schätzer:
$$\hat{\mu}_{ML}=\bar{x}$$
$$\hat{\sigma}^2_{ML}=\frac{1}{n}\sum_{i=1}^n(x_i-\bar{x})^2$$
\end{example2}

\begin{KR}{Maximum-Likelihood-Schätzung}\\
1. Likelyhood-Funktion aufstellen:
   \begin{itemize}
     \item Dichte der Verteilung identifizieren
     \item Produkt der Einzelwahrscheinlichkeiten bilden
     \item $L(\theta) = \prod_{i=1}^n f_X(x_i|\theta)$
   \end{itemize}

2. Logarithmierte Likelyhood-Funktion bilden:
   \begin{itemize}
     \item $\ln(L(\theta))$ berechnen
     \item Produkte werden zu Summen
   \end{itemize}

3. Ableitung Null setzen:
   \begin{itemize}
     \item $\frac{d}{d\theta}\ln(L(\theta)) = 0$
     \item Nach $\theta$ auflösen
   \end{itemize}

4. Maximum prüfen:
   \begin{itemize}
     \item Zweite Ableitung muss negativ sein
     \item $\frac{d^2}{d\theta^2}\ln(L(\theta)) < 0$
   \end{itemize}
\end{KR}

\begin{example2}{Maximum-Likelihood Exponentialverteilung}\\
Gegeben sei eine Stichprobe $x_1,...,x_n$ aus einer Exponentialverteilung mit Parameter $\lambda$:
$$f(x|\lambda) = \lambda e^{-\lambda x}, \quad x \geq 0$$

1. Likelihood-Funktion:
$$L(\lambda) = \prod_{i=1}^n \lambda e^{-\lambda x_i} = \lambda^n e^{-\lambda \sum x_i}$$

2. Log-Likelihood:
$$\ln(L(\lambda)) = n\ln(\lambda) - \lambda \sum x_i$$

3. Ableitung Null setzen:
$$\frac{d}{d\lambda}\ln(L(\lambda)) = \frac{n}{\lambda} - \sum x_i = 0$$

4. ML-Schätzer:
$$\hat{\lambda} = \frac{n}{\sum x_i} = \frac{1}{\bar{x}}$$
\end{example2}

\subsection{Vertrauensintervalle}

\begin{definition}{Vertrauensintervall}\\
Ein Vertrauensintervall zum Niveau $\gamma$ ist ein zufälliges Intervall $[\Theta_u,\Theta_o]$ mit:
$$
P(\Theta_u \leq \theta \leq \Theta_o)=\gamma
$$
$\gamma$: Vertrauensniveau (stat. Sicherheit)\\
$\alpha=1-\gamma$: Irrtumswahrscheinlichkeit\\
$\Theta_u, \Theta_o$: Unter- und Obergrenze
\end{definition}

\begin{concept}{Intervallschätzung}\\
Verteilungstypen und zugehörige Quantile:
\begin{center}
		\resizebox{\columnwidth}{!}{
\begin{tabular}{|c|c|c|}
\hline
Verteilung & Parameter & Quantile \\
\hline
Normalverteilung ($\sigma^2$ bekannt) & $\mu$ & $c=u_p, p=\frac{1+\gamma}{2}$ \\
\hline
t-Verteilung ($\sigma^2$ unbekannt) & $\mu$ & $c=t_{(p;f=n-1)}, p=\frac{1+\gamma}{2}$ \\
\hline
Chi-Quadrat-Verteilung & $\sigma^2$ & $c_1=\chi^2_{(\frac{1-\gamma}{2};n-1)}, c_2=\chi^2_{(\frac{1+\gamma}{2};n-1)}$ \\
\hline
\end{tabular}
}
\end{center}
\end{concept}

\begin{concept}{Übersicht statistische Schätzverfahren}\\
1. Punktschätzung:
   \begin{itemize}
     \item Maximum-Likelihood
     \item Momentenmethode
     \item Kleinste-Quadrate
   \end{itemize}

2. Intervallschätzung:
   \begin{itemize}
     \item Vertrauensintervalle für Mittelwert
     \item Vertrauensintervalle für Varianz
     \item Vertrauensintervalle für Anteilswerte
   \end{itemize}

3. Gütekriterien:
   \begin{itemize}
     \item Erwartungstreue
     \item Effizienz
     \item Konsistenz
     \item Minimale Varianz
   \end{itemize}
\end{concept}


\begin{example}{Berechnung eines Vertrauensintervalls}\\
Geben Sie das Vertrauensintervall für $\mu$ an ($\sigma^2$ unbekannt). Gegeben sind:
$$
n=10, \quad \bar{x}=102, \quad s^2=16, \quad \gamma=0.99
$$

\begin{enumerate}
  \item Verteilungstyp mit Param $\mu$ und $\sigma^2$ unbekannt $\rightarrow$ T-Verteilung
  \item $f=n-1=9$, $p=\frac{1+\gamma}{2}=0.995$, $c=t_{(p;f)}=t_{(0.995;9)}=3.25$
  \item $e=c \cdot \frac{S}{\sqrt{n}}=4.111$, $\Theta_u=\bar{X}-e=97.89$, $\Theta_o=\bar{X}+e=106.11$
\end{enumerate}
\end{example}

\begin{KR}{Vertrauensintervalle berechnen}\\
1. Verteilungstyp bestimmen:
   \begin{itemize}
     \item Normalverteilung ($\sigma^2$ bekannt)
     \item t-Verteilung ($\sigma^2$ unbekannt)
     \item Chi-Quadrat (für Varianz)
   \end{itemize}

2. Quantile bestimmen:
   \begin{itemize}
     \item Normalverteilung: $c=u_p$ mit $p=\frac{1+\gamma}{2}$
     \item t-Verteilung: $c=t_{(p;f)}$ mit $f=n-1$
     \item Chi-Quadrat: $c_1=\chi^2_{(p_1;f)}$, $c_2=\chi^2_{(p_2;f)}$
   \end{itemize}

3. Intervallgrenzen berechnen:
   \begin{itemize}
     \item Mittelwert: $[\bar{x} \pm c \cdot \frac{s}{\sqrt{n}}]$
     \item Varianz: $[\frac{(n-1)s^2}{c_2}, \frac{(n-1)s^2}{c_1}]$
   \end{itemize}
\end{KR}

\begin{example2}{Vertrauensintervall für Mittelwert}\\
Gegeben: $n=25$, $\bar{x}=102$, $s=4$, $\gamma=0.95$

1. Verteilung: t-Verteilung mit $f=24$ ($\sigma^2$ unbekannt)

2. Quantil:
   \begin{itemize}
     \item $p=\frac{1+0.95}{2}=0.975$
     \item $c=t_{(0.975;24)}=2.064$
   \end{itemize}

3. Intervallgrenzen:
   \begin{itemize}
     \item $e=2.064 \cdot \frac{4}{\sqrt{25}}=1.652$
     \item $[102-1.652; 102+1.652]=[100.348; 103.652]$
   \end{itemize}
\end{example2}

\subsubsection{Konstruktion von Vertrauensintervallen}

\begin{KR}{Vertrauensintervalle berechnen}\\
1. Verteilungstyp und Parameter bestimmen:
   \begin{itemize}
     \item Normalverteilung mit $\sigma^2$ bekannt
     \item Normalverteilung mit $\sigma^2$ unbekannt
     \item Chi-Quadrat für Varianz
   \end{itemize}

2. Vertrauensniveau $\gamma$ und Freiheitsgrade:
   \begin{itemize}
     \item $\gamma$ aus Aufgabe entnehmen
     \item $f = n-1$ bei t- und $\chi^2$-Verteilung
   \end{itemize}

3. Kritische Werte bestimmen:
   \begin{itemize}
     \item Normalverteilung: $c = u_p$ mit $p = \frac{1+\gamma}{2}$
     \item t-Verteilung: $c = t_{(p;f)}$ mit $p = \frac{1+\gamma}{2}$
     \item Chi-Quadrat: $c_1 = \chi^2_{(p_1;f)}$ und $c_2 = \chi^2_{(p_2;f)}$
   \end{itemize}

4. Intervallgrenzen berechnen:
   \begin{itemize}
     \item Mittelwert: $[\bar{x} - e; \bar{x} + e]$ mit $e = c \cdot \frac{s}{\sqrt{n}}$
     \item Varianz: $[\frac{(n-1)s^2}{c_2}; \frac{(n-1)s^2}{c_1}]$
   \end{itemize}
\end{KR}

\begin{example2}{Vertrauensintervall für Mittelwert}\\
Eine Maschine produziert Schrauben. Bei einer Stichprobe von $n=16$ Schrauben wurde der Durchmesser gemessen:
\begin{itemize}
  \item Mittelwert: $\bar{x} = 5.2$ mm
  \item Standardabweichung: $s = 0.15$ mm
  \item Vertrauensniveau: $\gamma = 95\%$
\end{itemize}

1. Verteilungstyp:
   \begin{itemize}
     \item $\sigma^2$ unbekannt $\rightarrow$ t-Verteilung
     \item $f = n-1 = 15$ Freiheitsgrade
   \end{itemize}

2. Kritischer Wert:
   \begin{itemize}
     \item $p = \frac{1+0.95}{2} = 0.975$
     \item $c = t_{(0.975;15)} = 2.131$
   \end{itemize}

3. Intervallgrenzen:
   \begin{itemize}
     \item $e = 2.131 \cdot \frac{0.15}{\sqrt{16}} = 0.080$
     \item $[\bar{x} - e; \bar{x} + e] = [5.12; 5.28]$
   \end{itemize}
\end{example2}

\begin{example2}{Vertrauensintervall für Varianz}\\
Für die obigen Schrauben soll ein 95\%-Vertrauensintervall für die Varianz berechnet werden.

1. Verteilungstyp:
   \begin{itemize}
     \item Chi-Quadrat-Verteilung
     \item $f = n-1 = 15$ Freiheitsgrade
   \end{itemize}

2. Kritische Werte:
   \begin{itemize}
     \item $p_1 = \frac{1-0.95}{2} = 0.025$
     \item $p_2 = \frac{1+0.95}{2} = 0.975$
     \item $c_1 = \chi^2_{(0.025;15)} = 6.262$
     \item $c_2 = \chi^2_{(0.975;15)} = 27.488$
   \end{itemize}

3. Intervallgrenzen:
   \begin{itemize}
     \item $s^2 = 0.15^2 = 0.0225$
     \item $\theta_u = \frac{15 \cdot 0.0225}{27.488} = 0.0123$
     \item $\theta_o = \frac{15 \cdot 0.0225}{6.262} = 0.0539$
   \end{itemize}

4. Vertrauensintervall für $\sigma^2$: $[0.0123; 0.0539]$
\end{example2}

\begin{KR}{Bestimmung des Stichprobenumfangs}\\
1. Gegebene Verteilung und Parameter:
   \begin{itemize}
     \item Normalverteilung mit $\sigma^2$ bekannt
     \item Vertrauensniveau $\gamma$
     \item Maximal zulässige Intervallbreite $d$
   \end{itemize}

2. Kritischen Wert bestimmen:
   \begin{itemize}
     \item $p = \frac{1+\gamma}{2}$
     \item $c = u_p$ für Normalverteilung
   \end{itemize}

3. Stichprobenumfang berechnen:
   \begin{itemize}
     \item $n \geq (\frac{2c\sigma}{d})^2$
     \item Auf nächste ganze Zahl aufrunden
   \end{itemize}

4. Bei unbekannter Varianz:
   \begin{itemize}
     \item Vorerhebung durchführen
     \item Varianz schätzen
     \item t-Verteilung statt Normalverteilung
   \end{itemize}
\end{KR}

\begin{example2}{Stichprobenumfang bestimmen}\\
Ein Prozess produziert Teile mit bekannter Standardabweichung $\sigma = 0.5$ mm. Der Mittelwert soll mit einer Genauigkeit von $\pm 0.2$ mm bei einem Vertrauensniveau von 99\% geschätzt werden.

1. Gesucht:
   \begin{itemize}
     \item Intervallbreite $d = 0.4$ mm
     \item $\gamma = 0.99$
   \end{itemize}

2. Kritischer Wert:
   \begin{itemize}
     \item $p = \frac{1+0.99}{2} = 0.995$
     \item $c = u_{0.995} = 2.576$
   \end{itemize}

3. Stichprobenumfang:
   \begin{itemize}
     \item $n \geq (\frac{2 \cdot 2.576 \cdot 0.5}{0.4})^2 = 41.47$
     \item $n = 42$ (aufgerundet)
   \end{itemize}
\end{example2}





\begin{KR}{Minimum Stichprobenumfang bestimmen}\\
1. Voraussetzungen:
   \begin{itemize}
     \item Gewünschte Genauigkeit $d$
     \item Vertrauensniveau $\gamma$
     \item Standardabweichung $\sigma$ (wenn bekannt)
   \end{itemize}

2. Kritischen Wert bestimmen:
   \begin{itemize}
     \item $c=u_p$ oder $t_{(p;f)}$
     \item $p=\frac{1+\gamma}{2}$
   \end{itemize}

3. Stichprobenumfang berechnen:
   \begin{itemize}
     \item $n \geq (\frac{2c\sigma}{d})^2$ für Mittelwert
     \item Auf nächste ganze Zahl aufrunden
   \end{itemize}
\end{KR}

\begin{example2}{Stichprobenumfang}\\
Ein Parameter soll mit einer Genauigkeit von $d=0.2$ bei $\gamma=0.99$ geschätzt werden. Die Standardabweichung ist $\sigma=0.5$ bekannt.

1. Kritischer Wert:
   \begin{itemize}
     \item $p=\frac{1+0.99}{2}=0.995$
     \item $c=u_{0.995}=2.576$
   \end{itemize}

2. Mindestumfang:
   \begin{itemize}
     \item $n \geq (\frac{2 \cdot 2.576 \cdot 0.5}{0.2})^2=41.47$
     \item $n=42$ (aufgerundet)
   \end{itemize}
\end{example2}

\begin{concept}{Übersicht Vertrauensintervalle}\\
   %TODO: add missing information from overview table from the script
\begin{center}
\begin{tabular}{|l|l|l|l|}
\hline
Parameter & Verteilung & Test-Statistik & Intervall \\
\hline
$\mu$ ($\sigma^2$ bekannt) & Normal & $U=\frac{\bar{X}-\mu}{\sigma/\sqrt{n}}$ & $[\bar{x} \pm u_p \frac{\sigma}{\sqrt{n}}]$ \\
\hline
$\mu$ ($\sigma^2$ unbek.) & t & $T=\frac{\bar{X}-\mu}{S/\sqrt{n}}$ & $[\bar{x} \pm t_p \frac{s}{\sqrt{n}}]$ \\
\hline
$\sigma^2$ & $\chi^2$ & $Z=\frac{(n-1)S^2}{\sigma^2}$ & $[\frac{(n-1)s^2}{c_2}, \frac{(n-1)s^2}{c_1}]$ \\
\hline
\end{tabular}
\end{center}
\end{concept}

\subsubsection{Detaillierte Vertrauensintervalle}

\begin{concept}{Verteilungstypen und Quantile}\\
\begin{center}
\begin{tabular}{|l|c|c|c|}
\hline
Verteilung & Parameter & Standardisierung & Quantile \\
\hline
Normalverteilung & $\mu$ & $U = \frac{\bar{X}-\mu}{\sigma/\sqrt{n}}$ & $c = u_p, p = \frac{1+\gamma}{2}$ \\
($\sigma^2$ bekannt) & & & \\
\hline
t-Verteilung & $\mu$ & $T = \frac{\bar{X}-\mu}{S/\sqrt{n}}$ & $c = t_{(p;f=n-1)}, p = \frac{1+\gamma}{2}$ \\
($\sigma^2$ unbekannt) & & & \\
\hline
Chi-Quadrat & $\sigma^2$ & $Z = (n-1)\frac{s^2}{\sigma^2}$ & $c_1 = \chi^2_{(\frac{1-\gamma}{2};n-1)}$ \\
& & & $c_2 = \chi^2_{(\frac{1+\gamma}{2};n-1)}$ \\
\hline
\end{tabular}
\end{center}
\end{concept}

\begin{formula}{Konfidenzintervalle}\\
Für verschiedene Verteilungen ergeben sich folgende Intervallgrenzen:

1. Normalverteilung ($\sigma^2$ bekannt):
$$\Theta_u = \bar{X} - c\frac{\sigma}{\sqrt{n}}, \quad \Theta_o = \bar{X} + c\frac{\sigma}{\sqrt{n}}$$

2. t-Verteilung ($\sigma^2$ unbekannt):
$$\Theta_u = \bar{X} - c\frac{S}{\sqrt{n}}, \quad \Theta_o = \bar{X} + c\frac{S}{\sqrt{n}}$$

3. Chi-Quadrat-Verteilung:
$$\Theta_u = \frac{(n-1)s^2}{c_2}, \quad \Theta_o = \frac{(n-1)s^2}{c_1}$$
\end{formula}

\subsection{Typische Prüfungsaufgaben}

\begin{KR}{Typische Prüfungsaufgaben}\\
1. Maximum-Likelihood:
   \begin{itemize}
     \item Likelihood-Funktion aufstellen
     \item Logarithmieren und ableiten
     \item ML-Schätzer bestimmen
   \end{itemize}

2. Vertrauensintervalle:
   \begin{itemize}
     \item Verteilungstyp bestimmen
     \item Quantile nachschlagen
     \item Intervall berechnen
   \end{itemize}

3. Schätzer prüfen:
   \begin{itemize}
     \item Erwartungstreue nachweisen
     \item Effizienz vergleichen
     \item Konsistenz zeigen
   \end{itemize}

4. Stichprobenumfang:
   \begin{itemize}
     \item Genauigkeit berücksichtigen
     \item Vertrauensniveau einbeziehen
     \item Mindestumfang bestimmen
   \end{itemize}
\end{KR}

\begin{KR}{Typische Prüfungsaufgaben}\\
1. Maximum-Likelihood-Schätzung:
   \begin{itemize}
     \item Likelihood-Funktion aufstellen
     \item Logarithmieren
     \item Ableitung Null setzen
     \item ML-Schätzer bestimmen
     \item Maximum prüfen
   \end{itemize}

2. Vertrauensintervalle:
   \begin{itemize}
     \item Verteilungstyp bestimmen
     \item Kritische Werte nachschlagen
     \item Intervallgrenzen berechnen
     \item Interpretation der Ergebnisse
   \end{itemize}

3. Stichprobenumfang:
   \begin{itemize}
     \item Genauigkeitsanforderungen
     \item Vertrauensniveau
     \item Minimal notwendigen Umfang bestimmen
   \end{itemize}
\end{KR}