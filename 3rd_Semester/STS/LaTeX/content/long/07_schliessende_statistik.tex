\section{Parameter- und Intervallschätzung}

\begin{concept}{Grundlagen der Schätztheorie}\\
Die Schätztheorie befasst sich mit zwei Hauptproblemen:
\begin{itemize}
  \item Punktschätzung: Bestimmung eines einzelnen Schätzwerts
  \item Intervallschätzung: Bestimmung eines Vertrauensbereichs
\end{itemize}

Wichtige Begriffe:
\begin{itemize}
  \item $\theta$: Unbekannter Parameter der Grundgesamtheit
  \item $\Theta$: Schätzfunktion (Zufallsvariable)
  \item $\hat{\theta}$: Schätzwert (konkreter Wert)
  \item $n$: Stichprobenumfang
\end{itemize}
\end{concept}

\begin{definition}{Erwartungstreue Schätzfunktion}\\
Eine Schätzfunktion $\Theta$ eines Parameters $\theta$ heisst erwartungstreu, wenn:
$$
E(\Theta)=\theta
$$
$E(\Theta)$: Erwartungswert der Schätzfunktion\\
$\theta$: Wahrer Parameter der Grundgesamtheit
\end{definition}

\begin{definition}{Effizienz einer Schätzfunktion}\\
Gegeben sind zwei erwartungstreue Schätzfunktionen $\Theta_1$ und $\Theta_2$ desselben Parameters $\theta$. Man nennt $\Theta_1$ effizienter als $\Theta_2$, falls:
$$
V(\Theta_1)<V(\Theta_2)
$$
$V(\Theta_1), V(\Theta_2)$: Varianzen der Schätzfunktionen
\end{definition}

\begin{definition}{Konsistenz einer Schätzfunktion}\\
Eine Schätzfunktion $\Theta$ heisst konsistent, wenn:
$$
E(\Theta) \rightarrow \theta \text{ und } V(\Theta) \rightarrow 0 \text{ für } n \rightarrow \infty
$$
$n$: Stichprobenumfang
\end{definition}

\begin{KR}{Prüfen von Schätzfunktionen}\\
1. Erwartungstreue:
   \begin{itemize}
     \item Erwartungswert $E(\Theta)$ berechnen
     \item Mit Parameter $\theta$ vergleichen
     \item Erwartungstreu, wenn $E(\Theta)=\theta$
   \end{itemize}

2. Effizienz:
   \begin{itemize}
     \item Varianzen $V(\Theta_1)$ und $V(\Theta_2)$ berechnen
     \item Varianzen vergleichen
     \item Kleinere Varianz = effizienter
   \end{itemize}

3. Konsistenz:
   \begin{itemize}
     \item Grenzwert für $n \to \infty$ betrachten
     \item $E(\Theta) \to \theta$?
     \item $V(\Theta) \to 0$?
   \end{itemize}
\end{KR}

\begin{example2}{Prüfung auf Erwartungstreue}\\
Gegeben sei die Schätzfunktion $\Theta_1=\frac{1}{3}(2X_1+X_2)$ für den Erwartungswert $\mu$.

1. Erwartungswert berechnen:
   $$E(\Theta_1)=E(\frac{1}{3}(2X_1+X_2))=\frac{1}{3}(2E(X_1)+E(X_2))$$

2. Einsetzen der Erwartungswerte:
   $$E(\Theta_1)=\frac{1}{3}(2\mu+\mu)=\frac{3\mu}{3}=\mu$$

3. Da $E(\Theta_1)=\mu$, ist die Schätzfunktion erwartungstreu.

Effizienzberechnung:
$$
\begin{aligned}
V(\Theta_1) &= V(\frac{1}{3}(2X_1+X_2)) \\
&= \frac{1}{9}V(2X_1+X_2) \\
&= \frac{1}{9}(4V(X_1)+V(X_2)) \\
&= \frac{1}{9}(4\sigma^2+\sigma^2) \\
&= \frac{5\sigma^2}{9}
\end{aligned}
$$
\end{example2}

\begin{definition}{Maximum-Likelihood-Schätzung}\\
Die Likelihood-Funktion für eine Stichprobe $x_1,\ldots,x_n$ ist:
$$
L(\theta)=\prod_{i=1}^n f_X(x_i|\theta)
$$
$f_X(x|\theta)$: Wahrscheinlichkeitsdichte\\
$\theta$: zu schätzender Parameter\\
$n$: Stichprobenumfang

Der Maximum-Likelihood-Schätzer maximiert $L(\theta)$ bzw. $\ln(L(\theta))$.
\end{definition}

\begin{KR}{Maximum-Likelihood-Schätzung}\\
1. Likelihood-Funktion aufstellen:
   $$L(\theta)=\prod_{i=1}^n f_X(x_i|\theta)$$

2. Logarithmieren:
   $$\ln(L(\theta))=\sum_{i=1}^n \ln(f_X(x_i|\theta))$$

3. Ableitung Null setzen:
   $$\frac{d}{d\theta}\ln(L(\theta))=0$$

4. Nach $\theta$ auflösen:
   $$\hat{\theta}_{ML}=\text{Lösung}$$

5. Maximum prüfen:
   $$\frac{d^2}{d\theta^2}\ln(L(\theta)) < 0$$
\end{KR}

\begin{example2}{Maximum-Likelihood Normalverteilung}\\
Gegeben sei eine Stichprobe aus einer Normalverteilung $N(\mu,\sigma^2)$.

1. Likelihood-Funktion:
$$L(\mu,\sigma^2)=\prod_{i=1}^n \frac{1}{\sqrt{2\pi\sigma^2}}e^{-\frac{(x_i-\mu)^2}{2\sigma^2}}$$

2. Log-Likelihood:
$$\ln(L)=-\frac{n}{2}\ln(2\pi\sigma^2)-\frac{1}{2\sigma^2}\sum_{i=1}^n(x_i-\mu)^2$$

3. Ableitungen:
$$\frac{\partial \ln(L)}{\partial \mu}=\frac{1}{\sigma^2}\sum_{i=1}^n(x_i-\mu)=0$$
$$\frac{\partial \ln(L)}{\partial \sigma^2}=-\frac{n}{2\sigma^2}+\frac{1}{2(\sigma^2)^2}\sum_{i=1}^n(x_i-\mu)^2=0$$

4. ML-Schätzer:
$$\hat{\mu}_{ML}=\bar{x}$$
$$\hat{\sigma}^2_{ML}=\frac{1}{n}\sum_{i=1}^n(x_i-\bar{x})^2$$
\end{example2}

\begin{definition}{Vertrauensintervall}\\
Ein Vertrauensintervall zum Niveau $\gamma$ ist ein zufälliges Intervall $[\Theta_u,\Theta_o]$ mit:
$$
P(\Theta_u \leq \theta \leq \Theta_o)=\gamma
$$
$\gamma$: Vertrauensniveau (stat. Sicherheit)\\
$\alpha=1-\gamma$: Irrtumswahrscheinlichkeit\\
$\Theta_u, \Theta_o$: Unter- und Obergrenze
\end{definition}

\begin{KR}{Vertrauensintervalle berechnen}\\
1. Verteilungstyp bestimmen:
   \begin{itemize}
     \item Normalverteilung ($\sigma^2$ bekannt)
     \item t-Verteilung ($\sigma^2$ unbekannt)
     \item Chi-Quadrat (für Varianz)
   \end{itemize}

2. Quantile bestimmen:
   \begin{itemize}
     \item Normalverteilung: $c=u_p$ mit $p=\frac{1+\gamma}{2}$
     \item t-Verteilung: $c=t_{(p;f)}$ mit $f=n-1$
     \item Chi-Quadrat: $c_1=\chi^2_{(p_1;f)}$, $c_2=\chi^2_{(p_2;f)}$
   \end{itemize}

3. Intervallgrenzen berechnen:
   \begin{itemize}
     \item Mittelwert: $[\bar{x} \pm c \cdot \frac{s}{\sqrt{n}}]$
     \item Varianz: $[\frac{(n-1)s^2}{c_2}, \frac{(n-1)s^2}{c_1}]$
   \end{itemize}
\end{KR}

\begin{example2}{Vertrauensintervall für Mittelwert}\\
Gegeben: $n=25$, $\bar{x}=102$, $s=4$, $\gamma=0.95$

1. Verteilung: t-Verteilung mit $f=24$ ($\sigma^2$ unbekannt)

2. Quantil:
   \begin{itemize}
     \item $p=\frac{1+0.95}{2}=0.975$
     \item $c=t_{(0.975;24)}=2.064$
   \end{itemize}

3. Intervallgrenzen:
   \begin{itemize}
     \item $e=2.064 \cdot \frac{4}{\sqrt{25}}=1.652$
     \item $[102-1.652; 102+1.652]=[100.348; 103.652]$
   \end{itemize}
\end{example2}

\begin{KR}{Minimum Stichprobenumfang bestimmen}\\
1. Voraussetzungen:
   \begin{itemize}
     \item Gewünschte Genauigkeit $d$
     \item Vertrauensniveau $\gamma$
     \item Standardabweichung $\sigma$ (wenn bekannt)
   \end{itemize}

2. Kritischen Wert bestimmen:
   \begin{itemize}
     \item $c=u_p$ oder $t_{(p;f)}$
     \item $p=\frac{1+\gamma}{2}$
   \end{itemize}

3. Stichprobenumfang berechnen:
   \begin{itemize}
     \item $n \geq (\frac{2c\sigma}{d})^2$ für Mittelwert
     \item Auf nächste ganze Zahl aufrunden
   \end{itemize}
\end{KR}

\begin{example2}{Stichprobenumfang}\\
Ein Parameter soll mit einer Genauigkeit von $d=0.2$ bei $\gamma=0.99$ geschätzt werden. Die Standardabweichung ist $\sigma=0.5$ bekannt.

1. Kritischer Wert:
   \begin{itemize}
     \item $p=\frac{1+0.99}{2}=0.995$
     \item $c=u_{0.995}=2.576$
   \end{itemize}

2. Mindestumfang:
   \begin{itemize}
     \item $n \geq (\frac{2 \cdot 2.576 \cdot 0.5}{0.2})^2=41.47$
     \item $n=42$ (aufgerundet)
   \end{itemize}
\end{example2}

\begin{concept}{Übersicht Vertrauensintervalle}\\
\begin{center}
\begin{tabular}{|l|l|l|l|}
\hline
Parameter & Verteilung & Test-Statistik & Intervall \\
\hline
$\mu$ ($\sigma^2$ bekannt) & Normal & $U=\frac{\bar{X}-\mu}{\sigma/\sqrt{n}}$ & $[\bar{x} \pm u_p \frac{\sigma}{\sqrt{n}}]$ \\
\hline
$\mu$ ($\sigma^2$ unbek.) & t & $T=\frac{\bar{X}-\mu}{S/\sqrt{n}}$ & $[\bar{x} \pm t_p \frac{s}{\sqrt{n}}]$ \\
\hline
$\sigma^2$ & $\chi^2$ & $Z=\frac{(n-1)S^2}{\sigma^2}$ & $[\frac{(n-1)s^2}{c_2}, \frac{(n-1)s^2}{c_1}]$ \\
\hline
\end{tabular}
\end{center}
\end{concept}

\begin{KR}{Typische Prüfungsaufgaben}\\
1. Maximum-Likelihood:
   \begin{itemize}
     \item Likelihood-Funktion aufstellen
     \item Logarithmieren und ableiten
     \item ML-Schätzer bestimmen
   \end{itemize}

2. Vertrauensintervalle:
   \begin{itemize}
     \item Verteilungstyp bestimmen
     \item Quantile nachschlagen
     \item Intervall berechnen
   \end{itemize}

3. Schätzer prüfen:
   \begin{itemize}
     \item Erwartungstreue nachweisen
     \item Effizienz vergleichen
     \item Konsistenz zeigen
   \end{itemize}

4. Stichprobenumfang:
   \begin{itemize}
     \item Genauigkeit berücksichtigen
     \item Vertrauensniveau einbeziehen
     \item Mindestumfang bestimmen
   \end{itemize}
\end{KR}