\section{Deskriptive Statistik (mehrere Merkmale)}

Bei vielen Fragestellungen sind verschiedene Merkmale bei einem Merkmalsträger interessant. Teilweise ist auch das Ziel der Untersuchungen, Zusammenhänge zwischen diesen Merkmalen zu finden.

\begin{definition}{Multivariate Daten}
\begin{itemize}
    \item \textbf{Bivariate Daten:} Zwei Merkmale pro Merkmalsträger
    \item \textbf{Multivariate Daten:} Mehrere Merkmale pro Merkmalsträger
\end{itemize}
\end{definition}

\subsection{Bivariate Daten}

\subsubsection{Grafische Darstellung}

\begin{concept}{Darstellungsformen nach Merkmalstypen}
\begin{itemize}
    \item \textbf{Zwei kategorielle Merkmale:}
    \begin{itemize}
        \item Kontingenztabellen
        \item Mosaikplots
    \end{itemize}
    \item \textbf{Ein kategorielles + ein metrisches Merkmal:}
    \begin{itemize}
        \item Boxplots
        \item Stripcharts
        \item Kennwerte pro Kategorie
    \end{itemize}
    \item \textbf{Zwei metrische Merkmale:}
    \begin{itemize}
        \item Streudiagramm (Scatterplot)
        \item Punktwolke in der (x,y)-Ebene
    \end{itemize}
\end{itemize}
\end{concept}

\begin{KR}{Analyse von Streudiagrammen}
\begin{enumerate}
    \item Untersuche die \textbf{Form} des Zusammenhangs:
    \begin{itemize}
        \item Linear: Punkte streuen um Gerade
        \item Gekrümmt: Punkte folgen einer Kurve
        \item Mehrere Punktwolken vorhanden?
    \end{itemize}
    \item Bestimme die \textbf{Richtung}:
    \begin{itemize}
        \item Positiv: y-Werte steigen mit x-Werten
        \item Negativ: y-Werte fallen mit x-Werten
        \item Kein Trend erkennbar
    \end{itemize}
    \item Beurteile die \textbf{Stärke}:
    \begin{itemize}
        \item Wenig Streuung: starker Zusammenhang
        \item Große Streuung: schwacher Zusammenhang
        \item Auf Ausreißer achten
    \end{itemize}
\end{enumerate}
\end{KR}

\subsubsection{Korrelation}

\begin{definition}{Kovarianz}
Die Kovarianz ist ein Maß für den linearen Zusammenhang:
$$s_{xy} = \frac{1}{n}\sum_{i=1}^n (x_i - \bar{x})(y_i - \bar{y})$$
Alternative Berechnungsformel:
$$s_{xy} = \overline{xy} - \bar{x}\bar{y}$$
mit $\overline{xy} = \frac{1}{n}\sum_{i=1}^n x_i y_i$
\end{definition}

\begin{definition}{Korrelationskoeffizient nach Pearson}
Der Pearson-Korrelationskoeffizient normiert die Kovarianz:
$$r_{xy} = \frac{s_{xy}}{s_x \cdot s_y}$$
Eigenschaften:
\begin{itemize}
    \item $-1 \leq r_{xy} \leq 1$
    \item $r_{xy} \approx 1$: starker positiver linearer Zusammenhang
    \item $r_{xy} \approx -1$: starker negativer linearer Zusammenhang
    \item $r_{xy} \approx 0$: kein linearer Zusammenhang
\end{itemize}
\end{definition}

\begin{definition}{Rangkorrelationskoeffizient nach Spearman}
Für monotone Zusammenhänge wird der Spearman-Koeffizient verwendet:
$$r_{Sp} = \frac{s_{rg(xy)}}{s_{rg(x)} \cdot s_{rg(y)}}$$
Bei unterschiedlichen Rängen gilt vereinfacht:
$$r_{Sp} = 1 - \frac{6\sum_{i=1}^n d_i^2}{n(n^2-1)}$$
mit $d_i = rg(x_i) - rg(y_i)$ (Rangdifferenzen)
\end{definition}

\begin{concept}{Unterschied Pearson und Spearman}
\begin{itemize}
    \item \textbf{Pearson:}
    \begin{itemize}
        \item Misst linearen Zusammenhang
        \item Empfindlich gegen Ausreißer
        \item Für metrische Daten
    \end{itemize}
    \item \textbf{Spearman:}
    \begin{itemize}
        \item Misst monotonen Zusammenhang
        \item Robust gegen Ausreißer
        \item Auch für ordinale Daten
    \end{itemize}
\end{itemize}
\end{concept}

\begin{remark}{Scheinkorrelation}
Eine Korrelation zwischen zwei Merkmalen bedeutet nicht automatisch einen kausalen Zusammenhang:
\begin{itemize}
    \item Ein drittes Merkmal könnte beide beeinflussen
    \item Der Zusammenhang könnte zufällig sein
    \item Ausreißer können das Ergebnis verzerren
\end{itemize}
\end{remark}

\subsection{Mehrere Merkmale}

\begin{concept}{Darstellung multivariater Daten}
\begin{itemize}
    \item \textbf{Kategorielle Merkmale:}
    \begin{itemize}
        \item Mehrdimensionale Kontingenztabellen
        \item Farbliche Codierung zusätzlicher Dimensionen
    \end{itemize}
    \item \textbf{Metrische Merkmale:}
    \begin{itemize}
        \item Matrix von Streudiagrammen
        \item Korrelationsmatrix
    \end{itemize}
\end{itemize}
\end{concept}