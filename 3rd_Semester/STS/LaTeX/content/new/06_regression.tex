\section{Die Methode der kleinsten Quadrate}

\subsection{Einführung}
\begin{concept}{Einführung}\\
Die Methode der kleinsten Quadrate ist eine weit verbreitete Optimierungsmethode zur Modellierung mathematischer Zusammenhänge in großen Datenmengen. Das Ziel ist es, optimale Parameter zu finden, die den funktionalen Zusammenhang zwischen Messdaten am besten beschreiben. Bei der linearen Regression wird beispielsweise ein linearer Zusammenhang zwischen den Daten vermutet und versucht, eine optimale Gerade in die Datenmenge einzupassen.
\end{concept}

\subsection{Lineare Regression}

\begin{definition}{Lineare Regression}\\
Gegeben sind Datenpunkte $(x_i; y_i)$ mit $1 \leq i \leq n$, die näherungsweise auf einer Geraden liegen. 

Die Residuen oder Fehler $\epsilon_i = y_i - g(x_i)$ dieser Datenpunkte sind die Abstände in $y$-Richtung zwischen $y_i$ und der Geraden $g$.\\

Die''bestmögliche'' Gerade, die Ausgleichs- oder Regressionsgerade, ist diejenige Gerade, für die die Summe der quadrierten Residuen $\sum_{i=1}^n \epsilon_i^2$ am kleinsten ist:

\[\sum_{i=1}^n (y_i - g(x_i))^2 = \sum_{i=1}^n (y_i - \hat{y}_i)^2\]

mit:\\
$y_i$: beobachtete $y$-Werte\\
$\hat{y}_i$: prognostizierte bzw. erklärte $y$-Werte\\
$\epsilon_i$: Residuum (oder auch Fehler/Abweichung) des $i$-ten Datenpunktes\\
$g(x_i)$ = Wert der Regressionsgerade an der Stelle $x_i$\\
$n$ = Anzahl der Datenpunkte\\
$(x_i, y_i)$ = Datenpunkte
\end{definition}

\begin{theorem}{Parameter der Regressionsgerade}\\
Die Regressionsgerade $g(x) = mx + d$ mit den Parametern $m$ und $d$ ist die Gerade, für die die Residualvarianz $\tilde{s}_\epsilon^2$ minimal ist.

\textbf{Parameter:}\\
Steigung: $m = \frac{\tilde{s}_{xy}}{\tilde{s}_x^2}$\\
y-Achsenabschnitt: $d = \bar{y} - m\bar{x}$\\

\textbf{Wichtige Kenngrößen:}\\
Arithmetische Mittel: $\bar{x} = \frac{1}{n}\sum_{i=1}^n x_i$ und $\bar{y} = \frac{1}{n}\sum_{i=1}^n y_i$\\

Varianz der $x_i$-Werte:\\
$\tilde{s}_x^2 = \frac{1}{n}\sum_{i=1}^n (x_i-\bar{x})^2 = (\frac{1}{n}\sum_{i=1}^n x_i^2) - \bar{x}^2$\\

Varianz der $y_i$-Werte:\\
$\tilde{s}_y^2 = \frac{1}{n}\sum_{i=1}^n (y_i-\bar{y})^2 = (\frac{1}{n}\sum_{i=1}^n y_i^2) - \bar{y}^2$\\

Kovarianz:\\
$\tilde{s}_{xy} = \frac{1}{n}\sum_{i=1}^n (x_i-\bar{x})(y_i-\bar{y}) = (\frac{1}{n}\sum_{i=1}^n x_iy_i) - \bar{x}\bar{y}$\\

Residualvarianz:\\
$\tilde{s}_\epsilon^2 = \tilde{s}_y^2 - \frac{\tilde{s}_{xy}^2}{\tilde{s}_x^2}$\\
\end{theorem}

\begin{KR}{Lineare Regression berechnen}
\begin{enumerate}
  \setlength{\itemsep}{1pt}
\item Berechne arithmetische Mittel $\bar{x}$ und $\bar{y}$
\item Berechne Kovarianzen und Varianzen:
   \begin{itemize}
    \setlength{\itemsep}{1pt}
     \item $s_{xy} = \frac{1}{n}\sum_{i=1}^n (x_i-\bar{x})(y_i-\bar{y})$
     \item $s_x^2 = \frac{1}{n}\sum_{i=1}^n (x_i-\bar{x})^2$
     \item $s_y^2 = \frac{1}{n}\sum_{i=1}^n (y_i-\bar{y})^2$
   \end{itemize}
\item Berechne Steigung $m$ und y-Achsenabschnitt $d$:
   \begin{itemize}
     \item $m = \frac{s_{xy}}{s_x^2}$
     \item $d = \bar{y} - m\bar{x}$
   \end{itemize}
\item Regressionsgerade: $g(x) = mx + d$
\end{enumerate}
\end{KR}

\begin{example2}{Lineare Regression}
Gegeben sind die Datenpunkte:
\begin{center}
\begin{tabular}{|c|c|c|c|c|c|}
\hline
$x_i$ & 1 & 2 & 3 & 4 & 5 \\
\hline
$y_i$ & 2.1 & 4.0 & 6.3 & 7.8 & 9.9 \\
\hline
\end{tabular}
\end{center}

1. $\bar{x} = 3$, $\bar{y} = 6.02$

2. Kovarianzen und Varianzen:
   \begin{itemize}
     \item $s_{xy} = 3.945$
     \item $s_x^2 = 2$
     \item $s_y^2 = 8.4916$
   \end{itemize}

3. Parameter:
   \begin{itemize}
     \item $m = \frac{3.945}{2} = 1.9725$
     \item $d = 6.02 - 1.9725 \cdot 3 = 0.1025$
   \end{itemize}

4. Regressionsgerade: $g(x) = 1.9725x + 0.1025$
\end{example2}

\subsubsection{Varianzzerlegung und Bestimmtheitsmass}

\begin{concept}{Varianzzerlegung}\\
Die Totale Varianz setzt sich zusammen aus der Residualvarianz und der Varianz der prognostizierten Werte:

$$\tilde{s}_y^2 = \tilde{s}_\epsilon^2 + \tilde{s}_{\hat{y}}^2$$

mit:\\
$\tilde{s}_y^2$: Totale Varianz\\
$\tilde{s}_{\hat{y}}^2$: prognostizierte (erklärte) Varianz\\
$\tilde{s}_\epsilon^2$: Residualvarianz
\end{concept}

\begin{theorem}{Bestimmtheitsmass}\\
Das Bestimmtheitsmass $R^2$ beurteilt die globale Anpassungsgüte einer Regression über den Anteil der prognostizierten Varianz $s_{\hat{y}}^2$ an der totalen Varianz $s_y^2$:
$$
R^2=\frac{s_{\hat{y}}^2}{s_y^2}
$$
$R^2$ = Bestimmtheitsmass (zwischen 0 und 1)\\
$s_{\hat{y}}^2$ = Varianz der prognostizierten Werte\\
$s_y^2$ = Totale Varianz\\

Das Bestimmtheitsmass $R^2$ entspricht dem Quadrat des Korrelationskoeffizienten (nach Bravais-Pearson):
$$
R^2=\frac{s_{xy}^2}{s_x^2 \cdot s_y^2}=(r_{xy})^2
$$
$s_{xy}$ = Kovarianz von $x$ und $y$\\
$s_x^2$ = Varianz der $x$-Werte\\
$s_y^2$ = Varianz der $y$-Werte\\
$r_{xy}$ = Korrelationskoeffizient\\
\end{theorem}

\begin{corollary}{Interpretation des Bestimmtheitsmasses}
\begin{itemize}
  \item $R^2 = 0.75$ bedeutet, dass 75\% der gesamten Varianz durch die Regression erklärt sind
  \item Die restlichen 25\% sind Zufallsstreuung
\end{itemize}
\end{corollary}

\begin{KR}{Bestimmtheitsmass berechnen}\\
1. Berechne die totale Varianz $s_y^2$
2. Berechne die Residualvarianz $s_{\epsilon}^2$
3. Berechne die erklärte Varianz $s_{\hat{y}}^2$
4. Berechne das Bestimmtheitsmass:
   $$R^2 = \frac{s_{\hat{y}}^2}{s_y^2} = 1 - \frac{s_{\epsilon}^2}{s_y^2}$$
5. Interpretation:
   \begin{itemize}
     \item $R^2 \approx 1$: Sehr gute Anpassung
     \item $R^2 \approx 0$: Schlechte Anpassung
   \end{itemize}
\end{KR}

\subsubsection{Residuenbetrachtung}

\begin{concept}{Residuenplot}\\
Die Residuen werden bezogen auf die prognostizierten y-Werte $\hat{y}$ dargestellt. Auf der horizontalen Achse werden die prognostizierten y-Werte $\hat{y}$ und auf der vertikalen Achse die Residuen angetragen.

Beurteilungskriterien:\\
- Residuen sollten unsystematisch (d.h. zufällig) streuen\\
- Überall etwa gleich um die horizontale Achse streuen\\
- Betragsmäßig kleine Residuen sollten häufiger sein als große
\end{concept}

\begin{KR}{Residuen und Residuenplot analysieren}\\
1. Berechne die Residuen für jeden Datenpunkt:
   \begin{itemize}
     \item $\epsilon_i = y_i - (mx_i + d)$
   \end{itemize}
2. Erstelle Residuenplot:
   \begin{itemize}
     \item x-Achse: Prognostizierte Werte $\hat{y}_i = mx_i + d$
     \item y-Achse: Residuen $\epsilon_i$
   \end{itemize}
3. Prüfe Eigenschaften:
   \begin{itemize}
     \item Residuen sollten zufällig um Null streuen
     \item Keine systematischen Muster erkennbar
     \item Gleiche Streubreite über alle $\hat{y}_i$
   \end{itemize}
\end{KR}

\subsection{Nichtlineares Verhalten}

\begin{concept}{Linearisierung} \textbf{Wichtige Transformationen:}\\
Oft können nichtlineare Regressionsmodelle durch geeignete Transformation auf ein lineares Modell zurückgeführt werden.
\begin{center}
\begin{tabular}{|c|c|}
\hline
Ausgangsfunktion & Transformation \\
\hline
$y = q \cdot x^m$ & $\log(y) = \log(q) + m \cdot \log(x)$ \\
\hline
$y = q \cdot m^x$ & $\log(y) = \log(q) + \log(m) \cdot x$ \\
\hline
$y = q \cdot e^{m \cdot x}$ & $\ln(y) = \ln(q) + m \cdot x$ \\
\hline
$y = \frac{1}{q+m \cdot x}$ & $V = q + m \cdot x; V = \frac{1}{y}$ \\
\hline
$y = q + m \cdot \ln(x)$ & $y = q + m \cdot U; U = \ln(x)$ \\
\hline
$y = \frac{1}{q \cdot m^x}$ & $\log(\frac{1}{y}) = \log(q) + \log(m) \cdot x$ \\
\hline
\end{tabular}
\end{center}
$y$ = Abhängige Variable\\
$x$ = Unabhängige Variable\\
$q, m$ = Parameter der Funktion
\end{concept}

\begin{KR}{Nichtlineare Regression durch Linearisierung}
\begin{enumerate}
  \item Bestimme passende Transformation aus Tabelle
  \item Führe Transformation durch
  \item Wende lineare Regression auf transformierte Daten an
  \item Transformiere Parameter zurück
\end{enumerate}
\end{KR}

\begin{example2}{Exponentielles Wachstum} $y=q \cdot e^{mx}$
mit gegebenen Messwerten:
\begin{center}
\begin{tabular}{|c|c|c|c|c|}
\hline
$x$ & 1 & 2 & 3 & 4 \\
\hline
$y$ & 2.1 & 4.2 & 8.1 & 15.9 \\
\hline
\end{tabular}
\end{center}

1. Transformation $\ln(y)=\ln(q)+mx$ $\rightarrow$ $Y=\ln(y)$, $b=\ln(q)$:
\begin{center}
\begin{tabular}{|c|c|c|c|c|}
\hline
$x$ & 1 & 2 & 3 & 4 \\
\hline
$Y$ & 0.742 & 1.435 & 2.092 & 2.766 \\
\hline
\end{tabular}
\end{center}

2. Lineare Regression: $Y=mx+b$ $rightarrow$ $Y = 0.674x + 0.071$

3. Rücktransformation: $q=e^b$
   \begin{itemize}
     \item $m = 0.674$
     \item $q = e^{0.071} = 1.074$
   \end{itemize}

4. Ergebnis: $y = 1.074 \cdot e^{0.674x}$
\end{example2}

\subsection{Allgemeines Vorgehen bei der Regression}

\begin{concept}{Matrix-Darstellung}\\
Für die Methode der kleinsten Quadrate mit mehreren Variablen wird ein lineares Gleichungssystem aufgestellt:

$y = Xp + \epsilon$

mit:\\
$p$: Vektor der Parameter\\
$y$: Vektor der Messwerte\\
$\epsilon$: Vektor der Residuen\\
$X$: Matrix der Eingangswerte\\

Die Lösung ist:\\
$p = (X^TX)^{-1}X^Ty$\\
falls $(X^TX)$ invertierbar
\end{concept}

\begin{definition}{Matrix-Darstellung}\\
Die Parameter $m$ und $q$ der Regressionsgeraden werden mit der Matrix $A$ berechnet:
$$A = \begin{psmallmatrix} x_1 & 1 \\ \vdots & \vdots \\ x_n & 1 \end{psmallmatrix}, \quad A^T \cdot A \cdot \begin{psmallmatrix} m \\ q \end{psmallmatrix} = A^T \cdot \begin{psmallmatrix} y_1 \\ \vdots \\ y_n \end{psmallmatrix}$$
\end{definition}

\begin{KR}{Matrix-Methode für lineare Regression}\\
1. Erstelle Design-Matrix $A$:
   $$A = \begin{psmallmatrix} x_1 & 1 \\ \vdots & \vdots \\ x_n & 1 \end{psmallmatrix}$$
2. Berechne $A^T \cdot A$
3. Berechne $(A^T \cdot A)^{-1}$
4. Berechne Parameter:
   $$\begin{psmallmatrix} m \\ q \end{psmallmatrix} = (A^T \cdot A)^{-1} \cdot A^T \cdot \vec{y}$$
\end{KR}

\begin{formula}{Residuenberechnung}\\
Die Residuen $\epsilon_i$ ergeben sich als:
$$\epsilon_i = y_i - \hat{y}_i = y_i - (mx_i + q)$$

Die Summe der quadrierten Residuen wird minimiert:
$$\sum_{i=1}^n \epsilon_i^2 = \sum_{i=1}^n (y_i - (mx_i + q))^2 \rightarrow \text{min}$$
\end{formula}

\begin{KR}{Vorgehen bei Mehrfachregression}\\
1. Aufstellen der Matrix $X$ mit den Eingangswerten
2. Berechnung der Parameter $p = (X^TX)^{-1}X^Ty$
3. Berechnung der Residuen $\epsilon = y - Xp$
4. Überprüfung der Modellgüte durch:
   \begin{itemize}
     \item Bestimmtheitsmass $R^2$
     \item Residuenanalyse
     \item Plausibilität der Parameter
   \end{itemize}
\end{KR}

\begin{KR}{Mehrfachregression}\\
1. Aufstellen der Designmatrix:
   $$A = \begin{psmallmatrix} 
   x_{11} & x_{12} & \cdots & x_{1(k-1)} & 1 \\
   x_{21} & x_{22} & \cdots & x_{2(k-1)} & 1 \\
   \vdots & \vdots & \ddots & \vdots & \vdots \\
   x_{n1} & x_{n2} & \cdots & x_{n(k-1)} & 1
   \end{psmallmatrix}$$

2. Berechnung der Parameter:
   $$\vec{p} = (A^T A)^{-1} A^T \vec{y}$$

3. Residuen berechnen:
   $$\vec{\epsilon} = \vec{y} - A\vec{p}$$

4. Bestimmtheitsmass:
   $$R^2 = 1 - \frac{\sum \epsilon_i^2}{\sum(y_i - \bar{y})^2}$$
\end{KR}

\begin{example2}{Mehrfachregression}
Ein Gebrauchtwagenhändler möchte den Preis (P) seiner Autos basierend auf Alter (A) und Kilometerstand (K) berechnen.
Gegeben sind folgende Daten:

\begin{center}
\begin{tabular}{|c|c|c|c|}
\hline
Auto & Alter (Jahre) & km (10000) & Preis (1000 CHF) \\
\hline
1 & 2 & 3 & 25 \\
2 & 3 & 4 & 20 \\
3 & 4 & 6 & 15 \\
4 & 5 & 7 & 12 \\
\hline
\end{tabular}
\end{center}

1. Designmatrix aufstellen:
   $$A = \begin{psmallmatrix}
   2 & 3 & 1 \\
   3 & 4 & 1 \\
   4 & 6 & 1 \\
   5 & 7 & 1
   \end{psmallmatrix}$$

2. Parameter berechnen:
   $$\vec{p} = \begin{psmallmatrix} -3 \\ -1.5 \\ 35 \end{psmallmatrix}$$

3. Resultierende Funktion:
   $$P = -3A - 1.5K + 35$$
\end{example2}

\begin{KR}{Polynomiale Regression}\\
Für Regression mit Polynomen höheren Grades:

1. Erweitere Designmatrix:
   $$A = \begin{psmallmatrix}
   x_1^n & x_1^{n-1} & \cdots & x_1 & 1 \\
   x_2^n & x_2^{n-1} & \cdots & x_2 & 1 \\
   \vdots & \vdots & \ddots & \vdots & \vdots \\
   x_m^n & x_m^{n-1} & \cdots & x_m & 1
   \end{psmallmatrix}$$

2. Löse wie bei linearer Regression:
   $$\vec{p} = (A^T A)^{-1} A^T \vec{y}$$

3. Polynom aufstellen:
   $$y = p_1x^n + p_2x^{n-1} + ... + p_nx + p_{n+1}$$
\end{KR}

\begin{example2}{Quadratische Regression}
Gegeben sind Messwerte:
\begin{center}
\begin{tabular}{|c|c|c|c|c|c|}
\hline
$x$ & 0 & 1 & 2 & 3 & 4 \\
\hline
$y$ & 1 & 2.1 & 5.2 & 10.1 & 17.2 \\
\hline
\end{tabular}
\end{center}

1. Designmatrix für quadratisches Polynom:
   $$A = \begin{psmallmatrix}
   0 & 0 & 1 \\
   1 & 1 & 1 \\
   4 & 2 & 1 \\
   9 & 3 & 1 \\
   16 & 4 & 1
   \end{psmallmatrix}$$

2. Parameter berechnen:
   $$\vec{p} = \begin{psmallmatrix} 1 \\ 0.1 \\ 1 \end{psmallmatrix}$$

3. Quadratische Funktion:
   $$y = x^2 + 0.1x + 1$$
\end{example2}

\begin{concept}{Gütekriterien für Regression}\\
1. Bestimmtheitsmass $R^2$:
   \begin{itemize}
     \item $R^2 > 0.9$: Sehr gute Anpassung
     \item $0.7 < R^2 < 0.9$: Gute Anpassung
     \item $0.5 < R^2 < 0.7$: Mittelmässige Anpassung
     \item $R^2 < 0.5$: Schlechte Anpassung
   \end{itemize}

2. Residuenanalyse:
   \begin{itemize}
     \item Residuen sollten zufällig um 0 schwanken
     \item Keine systematischen Muster erkennbar
     \item Residuen sollten normalverteilt sein
   \end{itemize}

3. Prognosegüte:
   \begin{itemize}
     \item Mittlerer quadratischer Fehler (MSE)
     \item Wurzel des mittleren quadratischen Fehlers (RMSE)
     \item Mittlerer absoluter Fehler (MAE)
   \end{itemize}
\end{concept}

\begin{example2}{Modellwahl durch Residuenanalyse}
Für einen Datensatz wurden drei Modelle getestet:
\begin{itemize}
  \item Linear: $y = 2x + 1$
  \item Quadratisch: $y = x^2 + x + 1$
  \item Exponentiell: $y = 2e^{0.5x}$
\end{itemize}

Bestimmtheitsmasse:
\begin{itemize}
  \item Linear: $R^2 = 0.85$
  \item Quadratisch: $R^2 = 0.98$
  \item Exponentiell: $R^2 = 0.92$
\end{itemize}

Residuenanalyse zeigt:
\begin{itemize}
  \item Linear: Systematische Krümmung in Residuen
  \item Quadratisch: Zufällige Verteilung der Residuen
  \item Exponentiell: Leichte Systematik in Residuen
\end{itemize}

Schlussfolgerung: Das quadratische Modell ist am besten geeignet.
\end{example2}

\begin{KR}{Prüfungsaufgaben lösen}\\
1. Aufgabentyp identifizieren:
   \begin{itemize}
     \item Einfache lineare Regression
     \item Mehrfachregression
     \item Nichtlineare Regression mit Transformation
     \item Polynomiale Regression
   \end{itemize}

2. Vorgehen wählen:
   \begin{itemize}
     \item Linear: Direkte Berechnung mit Formeln
     \item Nichtlinear: Transformation und lineare Regression
     \item Polynomial: Erweiterte Designmatrix
     \item Mehrfach: Matrix-Methode
   \end{itemize}

3. Berechnungen durchführen:
   \begin{itemize}
     \item Parameter bestimmen
     \item Bestimmtheitsmass berechnen
     \item Residuen analysieren
   \end{itemize}

4. Ergebnisse interpretieren:
   \begin{itemize}
     \item Modellgüte bewerten
     \item Residuen beurteilen
     \item Prognosen erstellen
   \end{itemize}
\end{KR}

\begin{example2}{Klausuraufgabe - Linearisierung}
Gegeben sind Messwerte für ein exponentielles Wachstum:
\begin{center}
\begin{tabular}{|c|c|c|c|c|}
\hline
$t$ (h) & 0 & 1 & 2 & 3 \\
\hline
$N$ & 100 & 150 & 225 & 340 \\
\hline
\end{tabular}
\end{center}

Finden Sie eine Funktion der Form $N(t) = N_0 e^{kt}$

1. Transformation:
   $$\ln(N) = \ln(N_0) + kt$$

2. Neue Wertetabelle:
\begin{center}
\begin{tabular}{|c|c|c|c|c|}
\hline
$t$ & 0 & 1 & 2 & 3 \\
\hline
$\ln(N)$ & 4.61 & 5.01 & 5.42 & 5.83 \\
\hline
\end{tabular}
\end{center}

3. Lineare Regression:
   $$\ln(N) = 0.405t + 4.61$$

4. Rücktransformation:
   $$N(t) = 100.4 e^{0.405t}$$

5. Bestimmtheitsmass: $R^2 = 0.999$
\end{example2}