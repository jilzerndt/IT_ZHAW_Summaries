\section{Spezielle Verteilungen}

\subsection{Diskrete und Stetige Zufallsvariablen}

\begin{definition}{Diskrete und Stetige Zufallsvariablen}
Bei einer \textbf{diskreten Zufallsvariable} gibt es immer Lücken zwischen den Werten; sie kann nur bestimmte Werte annehmen.

Eine \textbf{stetige Zufallsvariable} hat ein kontinuierliches Spektrum von möglichen Werten.

\textbf{Berechnung von Wahrscheinlichkeiten:}
\begin{itemize}
    \item Diskret: $P(X=x) = f(x)$ (PMF)
    \item Stetig: $P(X \leq x) = \int_{-\infty}^x f(t)dt$ (CDF)
\end{itemize}
\end{definition}

\begin{concept}{Gegenüberstellung von diskreten und stetigen Zufallsvariablen}
\begin{center}
\begin{tabular}{|l|l|l|}
\hline
 & Diskrete ZV & Stetige ZV \\ \hline
Dichtefunktion & $f(x) = P(X=x)$ & $f(x) = F'(x) \neq P(X=x)$ \\ \hline
Verteilungsfunktion & $F(x) = \sum_{x \leq X} f(x)$ & $F(x) = \int_{-\infty}^x f(t)dt$ \\ \hline
Wahrscheinlichkeiten & $P(a \leq X \leq b) = \sum_{a \leq x \leq b} f(x)$ & $P(a \leq X \leq b) = \int_a^b f(x)dx$ \\ \hline
Erwartungswert & $E(X) = \sum_{x \in \mathbb{R}} x \cdot f(x)$ & $E(X) = \int_{-\infty}^{\infty} x \cdot f(x)dx$ \\ \hline
Varianz & $V(X) = \sum_{x \in \mathbb{R}} (x-E(X))^2 \cdot f(x)$ & $V(X) = \int_{-\infty}^{\infty} (x-E(X))^2 \cdot f(x)dx$ \\ \hline
\end{tabular}
\end{center}
\end{concept}

\subsection{Diskrete Verteilungen}

\begin{concept}{Übersicht der diskreten Verteilungen}
\begin{center}
\begin{tabular}{|l|c|c|c|}
\hline
Verteilung & Notation & $E(X)$ & $V(X)$ \\ \hline
Hypergeometrisch & $H(N,M,n)$ & $n \cdot \frac{M}{N}$ & $n \cdot \frac{M}{N} \cdot (1-\frac{M}{N}) \cdot \frac{N-n}{N-1}$ \\ \hline
Binomial & $B(n,p)$ & $n \cdot p$ & $n \cdot p \cdot q$ \\ \hline
Poisson & $Poi(\lambda)$ & $\lambda$ & $\lambda$ \\ \hline
\end{tabular}
\end{center}
\end{concept}

\begin{definition}{Hypergeometrische Verteilung}
Ziehen \textbf{ohne Zurücklegen} aus einer endlichen Grundgesamtheit.

\textbf{Parameter:}
\begin{itemize}
    \item $N$: Grundgesamtheit
    \item $M$: Anzahl Merkmalsträger
    \item $n$: Stichprobenumfang
\end{itemize}

\textbf{Wahrscheinlichkeitsfunktion:}
$$P(X=k) = \frac{\binom{M}{k} \cdot \binom{N-M}{n-k}}{\binom{N}{n}}$$

\textbf{Kenngrößen:}
\begin{itemize}
    \item $E(X) = n \cdot \frac{M}{N}$
    \item $V(X) = n \cdot \frac{M}{N} \cdot (1-\frac{M}{N}) \cdot \frac{N-n}{N-1}$
\end{itemize}

\textbf{Notation:} $X \sim H(N,M,n)$
\end{definition}

\begin{definition}{Binomialverteilung}
$n$-malige \textbf{unabhängige Wiederholung} eines Bernoulli-Experiments.

\textbf{Parameter:}
\begin{itemize}
    \item $n$: Anzahl Versuche
    \item $p$: Erfolgswahrscheinlichkeit
    \item $q = 1-p$: Gegenwahrscheinlichkeit
\end{itemize}

\textbf{Wahrscheinlichkeitsfunktion:}
$$P(X=k) = \binom{n}{k} \cdot p^k \cdot q^{n-k}$$

\textbf{Kenngrößen:}
\begin{itemize}
    \item $E(X) = n \cdot p$
    \item $V(X) = n \cdot p \cdot q$
\end{itemize}

\textbf{Notation:} $X \sim B(n,p)$
\end{definition}

\begin{definition}{Poissonverteilung}
Modelliert \textbf{seltene Ereignisse} in einem festen Intervall.

\textbf{Parameter:}
\begin{itemize}
    \item $\lambda$: Erwartungswert/Rate pro Intervall
\end{itemize}

\textbf{Wahrscheinlichkeitsfunktion:}
$$P(X=k) = \frac{\lambda^k}{k!} \cdot e^{-\lambda}$$

\textbf{Kenngrößen:}
\begin{itemize}
    \item $E(X) = \lambda$
    \item $V(X) = \lambda$
\end{itemize}

\textbf{Notation:} $X \sim Poi(\lambda)$
\end{definition}

\subsection{Stetige Verteilungen}

\begin{definition}{Normalverteilung}
Die Dichtefunktion der Normalverteilung ist:
$$\varphi_{\mu,\sigma}(x) = \frac{1}{\sqrt{2\pi} \cdot \sigma} \cdot e^{-\frac{1}{2}(\frac{x-\mu}{\sigma})^2}$$

\textbf{Parameter:}
\begin{itemize}
    \item $\mu$: Erwartungswert (Lage)
    \item $\sigma$: Standardabweichung (Streuung)
\end{itemize}

\textbf{Eigenschaften:}
\begin{itemize}
    \item Symmetrisch um $\mu$
    \item Wendepunkte bei $\mu \pm \sigma$
    \item Ca. 68\% der Werte in $[\mu-\sigma, \mu+\sigma]$
    \item Ca. 95\% der Werte in $[\mu-2\sigma, \mu+2\sigma]$
    \item Ca. 99,7\% der Werte in $[\mu-3\sigma, \mu+3\sigma]$
\end{itemize}

\textbf{Notation:} $X \sim N(\mu,\sigma)$
\end{definition}

\subsection{Zentraler Grenzwertsatz und Approximationen}

\begin{theorem}{Zentraler Grenzwertsatz}
Für die Summe $S_n = X_1 + ... + X_n$ von $n$ unabhängigen, identisch verteilten Zufallsvariablen mit $E(X_i)=\mu$ und $V(X_i)=\sigma^2$ gilt:

\begin{itemize}
    \item $S_n$ ist approximativ normalverteilt
    \item $E(S_n) = n\mu$
    \item $V(S_n) = n\sigma^2$
\end{itemize}

Für das arithmetische Mittel $\bar{X}_n = \frac{S_n}{n}$ gilt:
\begin{itemize}
    \item $\bar{X}_n$ ist approximativ normalverteilt
    \item $E(\bar{X}_n) = \mu$
    \item $V(\bar{X}_n) = \frac{\sigma^2}{n}$
\end{itemize}
\end{theorem}

\begin{concept}{Approximationsregeln}
\textbf{Binomialverteilung $\rightarrow$ Normalverteilung:}
\begin{itemize}
    \item Bedingung: $npq > 9$
    \item $B(n,p) \approx N(np, \sqrt{npq})$
    \item Stetigkeitskorrektur beachten!
\end{itemize}

\textbf{Binomialverteilung $\rightarrow$ Poissonverteilung:}
\begin{itemize}
    \item Bedingung: $n \geq 50$ und $p \leq 0.1$
    \item $B(n,p) \approx Poi(np)$
\end{itemize}

\textbf{Hypergeometrisch $\rightarrow$ Binomialverteilung:}
\begin{itemize}
    \item Bedingung: $n \leq \frac{N}{20}$
    \item $H(N,M,n) \approx B(n,\frac{M}{N})$
\end{itemize}
\end{concept}

\begin{KR}{Wahl der richtigen Verteilung}
1. \textbf{Diskrete Verteilungen:}
   \begin{itemize}
   \item Ziehen ohne Zurücklegen: Hypergeometrisch
   \item Unabhängige Versuche: Binomial
   \item Seltene Ereignisse: Poisson
   \end{itemize}

2. \textbf{Approximationen prüfen:}
   \begin{itemize}
   \item $npq > 9$: Normal-Approximation möglich
   \item $n \geq 50, p \leq 0.1$: Poisson-Approximation möglich
   \item $n \leq \frac{N}{20}$: Binomial-Approximation möglich
   \end{itemize}

3. \textbf{Stetigkeitskorrektur:}
   \begin{itemize}
   \item Bei Normal-Approximation: $\pm 0.5$ an den Grenzen
   \item $P(X \leq k) \approx P(X \leq k + 0.5)$
   \item $P(X = k) \approx P(k - 0.5 \leq X \leq k + 0.5)$
   \end{itemize}
\end{KR}