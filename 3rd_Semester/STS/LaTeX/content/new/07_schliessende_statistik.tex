\section{Schliessende Statistik -- Parameter- und Intervallschätzung}

\subsection{Zufallsstichproben}

\begin{concept}{Grundlagen der Zufallsstichproben}\\
Die Grundgesamtheit ist eine Menge von gleichartigen Objekten oder Elementen. Sie kann endlich oder unendlich viele Objekte enthalten.

Eine Stichprobe vom Umfang $n$ wird entnommen, um Informationen über die Grundgesamtheit zu gewinnen. Dies ist oft notwendig weil:
\begin{itemize}
  \item Der Zeit- und Kostenaufwand für eine Vollerhebung zu hoch ist
  \item Die Anzahl Objekte zu groß ist
  \item Die Objekte bei der Untersuchung zerstört werden
\end{itemize}
\end{concept}

\begin{definition}{Einfache Zufallsstichprobe}\\
Eine einfache Zufallsstichprobe vom Umfang $n$ ist eine Folge von Zufallsvariablen $X_1, X_2, \ldots, X_n$ (Stichprobenvariablen). Dabei bezeichnet $X_i$ die Merkmalsausprägung des $i$-ten Elements in der Stichprobe.

Die beobachteten Merkmalswerte $x_1, x_2, \ldots, x_n$ der $n$ Elemente sind Realisierungen der Zufallsvariablen und heißen Stichprobenwerte.

Wichtige Eigenschaften:
\begin{itemize}
  \item Jedes Element hat die gleiche Chance, ausgewählt zu werden
  \item Die Ziehungen sind stochastisch unabhängig 
  \item Alle $X_i$ folgen derselben Verteilung $F(x)$ der Grundgesamtheit
\end{itemize}
\end{definition}

\subsection{Parameterschätzungen}

\subsubsection{Schätzfunktionen}

\begin{definition}{Schätzfunktion}\\
Eine Schätzfunktion $\Theta = g(X_1,X_2,\ldots,X_n)$ ist eine spezielle Stichprobenfunktion zur Schätzung eines Parameters $\theta$ der Grundgesamtheit.

Der Schätzwert $\hat{\theta} = g(x_1,x_2,\ldots,x_n)$ ergibt sich durch Einsetzen der konkreten Stichprobenwerte.

Wichtig: $\theta$ ist der wahre, unbekannte Parameterwert der Grundgesamtheit.
\end{definition}

\subsubsection{Kriterien für eine optimale Schätzfunktion}

\begin{concept}{Optimale Schätzfunktionen}\\
Eine Schätzfunktion sollte folgende Eigenschaften haben:

1. Erwartungstreu: $E(\Theta) = \theta$
2. Effizient: Kleinste Varianz unter allen erwartungstreuen Schätzern
3. Konsistent: $E(\Theta) \to \theta$ und $V(\Theta) \to 0$ für $n \to \infty$

\textbf{Interpretation:}
\begin{itemize}
  \item Erwartungstreue bedeutet, dass im Mittel der richtige Wert geschätzt wird
  \item Effizienz bedeutet möglichst geringe Streuung der Schätzung
  \item Konsistenz bedeutet, dass die Schätzung mit wachsendem Stichprobenumfang immer genauer wird
\end{itemize}
\end{concept}

\subsubsection{Wichtige Schätzfunktionen}

\begin{theorem}{Schätzfunktionen für wichtige Parameter}\\
\textbf{Erwartungswert:}
\[\bar{X} = \frac{1}{n}\sum_{i=1}^n X_i \qquad \hat{\mu} = \bar{x} = \frac{1}{n}\sum_{i=1}^n x_i\]

Eigenschaften:
\begin{itemize}
  \item Erwartungstreu: $E(\bar{X}) = \mu$
  \item Konsistent: $V(\bar{X}) = \frac{\sigma^2}{n} \to 0$ für $n \to \infty$
\end{itemize}

\textbf{Varianz:}
\[S^2 = \frac{1}{n-1}\sum_{i=1}^n (X_i-\bar{X})^2 \qquad \hat{\sigma}^2 = s^2 = \frac{1}{n-1}\sum_{i=1}^n (x_i-\bar{x})^2\]

Eigenschaften:
\begin{itemize}
  \item Erwartungstreu: $E(S^2) = \sigma^2$
  \item Konsistent: $V(S^2) \to 0$ für $n \to \infty$
\end{itemize}

\textbf{Anteilswert:} (bei Bernoulli-Verteilung)
\[\bar{X} = \frac{1}{n}\sum_{i=1}^n X_i \qquad \hat{p} = \bar{x} = \frac{1}{n}\sum_{i=1}^n x_i\]
\end{theorem}

\subsubsection{Maximum-Likelihood-Schätzung}

\begin{definition}{Likelihood-Funktion}\\
Für eine Stichprobe vom Umfang $n$ mit den Werten $x_1,x_2,\ldots,x_n$ ist die Likelihood-Funktion definiert als:

\[L(\theta) = f_X(x_1|\theta) \cdot f_X(x_2|\theta) \cdot \ldots \cdot f_X(x_n|\theta)\]

wobei $f_X(x|\theta)$ die Wahrscheinlichkeitsdichte der Verteilung ist.
\end{definition}

\begin{KR}{Maximum-Likelihood-Schätzung}\\
1. Likelihood-Funktion $L(\theta)$ aufstellen 
2. Log-Likelihood $\ln(L(\theta))$ bilden (vereinfacht die Rechnung)
3. Ableitung $\frac{d}{d\theta}\ln(L(\theta))=0$ setzen
4. Nach $\theta$ auflösen für $\hat{\theta}_{ML}$
5. Maximum überprüfen durch zweite Ableitung

\textbf{Beispiel für Normalverteilung:}
\begin{enumerate}
  \item $L(\mu,\sigma^2) = \prod_{i=1}^n \frac{1}{\sqrt{2\pi\sigma^2}}e^{-\frac{(x_i-\mu)^2}{2\sigma^2}}$
  \item $\ln(L) = -\frac{n}{2}\ln(2\pi\sigma^2) - \frac{1}{2\sigma^2}\sum(x_i-\mu)^2$
  \item $\frac{\partial}{\partial\mu}\ln(L) = 0$ und $\frac{\partial}{\partial\sigma^2}\ln(L) = 0$
  \item Ergibt: $\hat{\mu}_{ML} = \bar{x}$ und $\hat{\sigma}^2_{ML} = \frac{1}{n}\sum(x_i-\bar{x})^2$
\end{enumerate}
\end{KR}

\subsection{Vertrauensintervalle}

\begin{definition}{Vertrauensintervall}\\
Ein Vertrauensintervall $[\Theta_u,\Theta_o]$ zum Niveau $\gamma$ ist ein zufälliges Intervall mit:

\[P(\Theta_u \leq \theta \leq \Theta_o) = \gamma\]

$\gamma$: Vertrauensniveau (statistische Sicherheit)\\
$\alpha = 1-\gamma$: Irrtumswahrscheinlichkeit\\
$\Theta_u, \Theta_o$: Unter- und Obergrenze
\end{definition}

\begin{concept}{Vertrauensintervall-Typen}\\
\begin{center}
\begin{tabular}{|l|l|l|l|}
\hline
Fall & Verteilung & Test-Statistik & Grenzen \\
\hline
$\mu$ ($\sigma^2$ bekannt) & Standard- & $U = \frac{\bar{X}-\mu}{\sigma/\sqrt{n}}$ & $\bar{x} \pm c\frac{\sigma}{\sqrt{n}}$ \\
& normalvert. & & $c = u_p$ \\
\hline
$\mu$ ($\sigma^2$ unbek.) & t-Verteilung & $T = \frac{\bar{X}-\mu}{S/\sqrt{n}}$ & $\bar{x} \pm c\frac{s}{\sqrt{n}}$ \\
& mit $f=n-1$ & & $c = t_{p,f}$ \\
\hline
$\sigma^2$ & $\chi^2$-Verteilung & $Z = \frac{(n-1)S^2}{\sigma^2}$ & $[\frac{(n-1)s^2}{c_2}, \frac{(n-1)s^2}{c_1}]$ \\
& mit $f=n-1$ & & $c_1 = \chi^2_{p_1,f}, c_2 = \chi^2_{p_2,f}$ \\
\hline
\end{tabular}
\end{center}

mit $p = \frac{1+\gamma}{2}$, $p_1 = \frac{1-\gamma}{2}$, $p_2 = \frac{1+\gamma}{2}$
\end{concept}

\begin{KR}{Vertrauensintervalle berechnen}\\
1. Verteilungstyp bestimmen:
   \begin{itemize}
     \item Parameter ($\mu$ oder $\sigma^2$)
     \item $\sigma^2$ bekannt oder unbekannt
   \end{itemize}

2. Quantile bestimmen:
   \begin{itemize}
     \item $\gamma$ und $\alpha$ beachten
     \item Richtige Tabelle wählen
     \item Freiheitsgrade $f=n-1$ beachten
   \end{itemize}

3. Intervallgrenzen berechnen:
   \begin{itemize}
     \item Standardfehler berechnen
     \item Grenzen $\Theta_u$ und $\Theta_o$ bestimmen
   \end{itemize}
\end{KR}

\begin{KR}{Stichprobenumfang bestimmen}\\
1. Bei gegebener Genauigkeit $d$ und Vertrauensniveau $\gamma$:
   \begin{itemize}
     \item $\sigma^2$ bekannt: $n \geq (\frac{2c\sigma}{d})^2$
     \item Auf nächste ganze Zahl aufrunden
     \item $c$ aus entsprechender Verteilung
   \end{itemize}

2. Bei unbekannter Varianz:
   \begin{itemize}
     \item Vorerhebung durchführen
     \item Varianz schätzen
     \item t-Verteilung verwenden
   \end{itemize}
\end{KR}

\begin{example2}{Vertrauensintervall für Mittelwert}\\
Gegeben: $n=25$ Messungen, $\bar{x}=102$, $s=4$, $\gamma=0.95$

1. Verteilungstyp: t-Verteilung ($\sigma^2$ unbekannt)
   \begin{itemize}
     \item $f=24$ Freiheitsgrade
     \item $p=0.975$
     \item $c=t_{(0.975;24)}=2.064$
   \end{itemize}

2. Grenzen berechnen:
   \begin{itemize}
     \item $e=2.064 \cdot \frac{4}{\sqrt{25}}=1.652$
     \item $[102-1.652; 102+1.652]$
     \item $[100.348; 103.652]$
   \end{itemize}
\end{example2}