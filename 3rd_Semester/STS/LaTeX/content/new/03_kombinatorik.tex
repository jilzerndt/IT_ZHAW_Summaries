\section{Kombinatorik}

Bei vielen Wahrscheinlichkeitsproblemen ist es möglich, durch geschicktes Abzählen Wahrscheinlichkeiten zu ermitteln. Dazu sind kombinatorische Überlegungen erforderlich.

\subsection{Grundbegriffe}

\begin{definition}{Fakultät}
Die Fakultät $n!$ ist für eine natürliche Zahl $n$ rekursiv definiert:
\begin{itemize}
    \item Startwert: $0! = 1$
    \item Für $n \geq 1$: $n! = n \cdot (n-1)!$
\end{itemize}
Beispiel: $4! = 4 \cdot 3! = 4 \cdot 3 \cdot 2! = 4 \cdot 3 \cdot 2 \cdot 1! = 4 \cdot 3 \cdot 2 \cdot 1 \cdot 0! = 24$
\end{definition}

\begin{definition}{Binomialkoeffizient}
Der Binomialkoeffizient $\binom{n}{k}$ ist für natürliche Zahlen $0 \leq k \leq n$ definiert als:
$$\binom{n}{k} = \frac{n!}{(n-k)! \cdot k!}$$
Er gibt die Anzahl Möglichkeiten an, aus $n$ Objekten $k$ Objekte auszuwählen.
\end{definition}

\subsection{Grundlegende Abzählmethoden}

\begin{concept}{Systematik der Kombinatorik}
Man unterscheidet vier grundlegende Abzählprobleme:
\begin{center}
\begin{tabular}{|l|c|c|}
\hline 
& \textbf{Mit Wiederholung} & \textbf{Ohne Wiederholung} \\
\hline
\textbf{Variation} & $n^k$ & $\frac{n!}{(n-k)!}$ \\
(Reihenfolge wichtig) & & \\
\hline
\textbf{Kombination} & $\binom{n+k-1}{k}$ & $\binom{n}{k}$ \\
(Reihenfolge unwichtig) & & \\
\hline
\end{tabular}
\end{center}
\end{concept}

\begin{KR}{Bestimmung der Abzählmethode}
1. \textbf{Analysiere das Problem:}
   \begin{itemize}
   \item $n$: Anzahl verfügbarer Objekte
   \item $k$: Anzahl auszuwählender Objekte
   \end{itemize}

2. \textbf{Prüfe die Reihenfolge:}
   \begin{itemize}
   \item Ist die Reihenfolge wichtig? → Variation
   \item Ist nur die Auswahl wichtig? → Kombination
   \end{itemize}

3. \textbf{Prüfe Wiederholungen:}
   \begin{itemize}
   \item Dürfen Objekte mehrfach vorkommen? → Mit Wiederholung
   \item Darf jedes Objekt nur einmal? → Ohne Wiederholung
   \end{itemize}

4. \textbf{Wähle die passende Formel}
\end{KR}

\begin{example2}{Variation mit Wiederholung}
\textbf{Zahlenschloss:} 6 Stellen, Ziffern 0-9 möglich
\begin{itemize}
    \item $n=10$ Ziffern
    \item $k=6$ Stellen
    \item Reihenfolge wichtig
    \item Wiederholung erlaubt
    \item Lösung: $10^6 = 1\,000\,000$ Möglichkeiten
\end{itemize}
\end{example2}

\begin{example2}{Variation ohne Wiederholung}
\textbf{Schwimmwettkampf:} Erste 3 Plätze bei 10 Schwimmern
\begin{itemize}
    \item $n=10$ Schwimmer
    \item $k=3$ Plätze
    \item Reihenfolge wichtig
    \item Keine Wiederholung möglich
    \item Lösung: $\frac{10!}{7!} = 720$ Möglichkeiten
\end{itemize}
\end{example2}

\begin{example2}{Kombination mit Wiederholung}
\textbf{Zahnarzt:} 3 Spielzeuge aus 5 verschiedenen Arten
\begin{itemize}
    \item $n=5$ Arten
    \item $k=3$ Spielzeuge
    \item Reihenfolge unwichtig
    \item Wiederholung möglich
    \item Lösung: $\binom{7}{3} = 35$ Möglichkeiten
\end{itemize}
\end{example2}

\begin{example2}{Kombination ohne Wiederholung}
\textbf{Lotto:} 6 aus 49
\begin{itemize}
    \item $n=49$ Zahlen
    \item $k=6$ Auswahl
    \item Reihenfolge unwichtig
    \item Keine Wiederholung
    \item Lösung: $\binom{49}{6} = 13\,983\,816$ Möglichkeiten
\end{itemize}
\end{example2}

\subsection{Eigenschaften der Binomialkoeffizienten}

\begin{theorem}{Eigenschaften}
Für den Binomialkoeffizienten gelten:
\begin{itemize}
    \item Leere Menge: $\binom{n}{0} = 1$
    \item Symmetrie: $\binom{n}{k} = \binom{n}{n-k}$
    \item Pascal'sche Rekursion: $\binom{n+1}{k+1} = \binom{n}{k} + \binom{n}{k+1}$
    \item Summe: $\sum_{k=0}^n \binom{n}{k} = 2^n$
\end{itemize}
\end{theorem}

\begin{KR}{Berechnung von Binomialkoeffizienten}
1. \textbf{Prüfe Spezialfälle:}
   \begin{itemize}
   \item $\binom{n}{0} = \binom{n}{n} = 1$
   \item $\binom{n}{1} = n$
   \end{itemize}

2. \textbf{Nutze Symmetrie:}
   \begin{itemize}
   \item $\binom{n}{k} = \binom{n}{n-k}$
   \end{itemize}

3. \textbf{Pascal'sches Dreieck}
   \begin{itemize}
   \item Baue schrittweise auf
   \item Nutze Rekursionsformel
   \end{itemize}

4. \textbf{Direkte Berechnung}
   \begin{itemize}
   \item Nur wenn nötig
   \item Kürze vor dem Ausrechnen
   \end{itemize}
\end{KR}