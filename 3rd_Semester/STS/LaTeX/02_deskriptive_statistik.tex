\section{Beschreibende Statistik}
\begin{definition}{Absolute Häufigkeiten}
$$
H=\sum_{1}^{n} h_{i}
$$
\end{definition}

\begin{definition}{Relative Häufigkeiten}
$$
F=\sum_{1}^{m} f_{i}, \quad F(x)=\frac{H(x)}{n}
$$
\end{definition}

\section{Kennwerte (Lagemasse)}
\begin{definition}{Quantil}
$$
i=\lceil n \cdot q\rceil, Q=x_{i}=x_{\lceil n \cdot q\rceil}
$$
\end{definition}

\begin{definition}{Interquartilsabstand}
$$
I Q R=Q_{3}-Q_{1}
$$
\end{definition}

\begin{definition}{Modus}
$x_{\text {mod }}=$ Häufigste Wert
\end{definition}

\begin{concept}{Arithmetisches Mittel und Median}
\begin{center}
\begin{tabular}{|l|l|}
\hline
Arithmetisches Mittel & Median \\
$\bar{x}=\frac{1}{n} \sum_{i=1}^{n} x_{i}=\sum_{i=1}^{m} a_{i} \cdot f_{i}$ & $\left\{\begin{array}{c}x_{\left[\frac{n+1}{2}\right]} \quad n \text { ungerade } \\ 0.5 \cdot\left(x_{\left[\frac{n}{2}\right]}+x_{\left[\frac{n}{2}+1\right]}\right) \quad n \text { gerade }\end{array}\right.$ \\
\hline
\end{tabular}
\end{center}
\end{concept}

\begin{definition}{Stichprobenvarianz $s^{2}$ (Streumasse)}
$$
s^{2}=\frac{1}{n} \sum_{i=1}^{n}\left(x_{i}-\bar{x}\right)^{2}=\overline{x^{2}}-\bar{x}^{2}, \quad\left(s_{k o r}\right)^{2}=\frac{1}{n-1} \sum_{i=1}^{n}\left(x_{i}-\bar{x}\right)^{2}
$$

$$
\left(s_{k o r}\right)^{2}=\frac{n}{n-1} \cdot s^{2}
$$
\end{definition}

\begin{definition}{Standardabweichung $s$ (Streumasse)}
$$
s=\sqrt{\frac{1}{n} \sum_{i=1}^{n}\left(x_{i}-\bar{x}\right)^{2}}=\sqrt{\overline{x^{2}}-\bar{x}^{2}}, \quad s_{k o r}=\sqrt{\frac{1}{n-1} \sum_{i=1}^{n}\left(x_{i}-\bar{x}\right)^{2}}
$$
\end{definition}

\section{PDF + CDF}
\begin{definition}{Nicht klassierte Daten (PMF und CDF)}
Die absolute Häufigkeit kann als Funktion $h: \mathbb{R} \rightarrow \mathbb{R}$ bezeichnet werden.
$$
h_{i}
$$

Die relative Häufigkeit kann als Funktion $f: \mathbb{R} \rightarrow \mathbb{R}$ bezeichnet werden.
$$
f_{i}=\frac{h_{i}}{n}
$$
\end{definition}
