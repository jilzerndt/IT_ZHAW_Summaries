\section{Deskriptive Statistik}

\begin{definition}{Bivariate Daten (Merkmale)}\\
\begin{itemize}
  \item 2x kategoriell $\rightarrow$ Kontingenztabelle + Mosaikplot
  \item 1x kategoriell + 1x metrisch $\rightarrow$ Boxplot oder Stripchart
  \item 2x metrisch $\rightarrow$ Streudiagramm
\end{itemize}
\end{definition}

\begin{minipage}{0.5\columnwidth}
\begin{definition}{Absolute Häufigkeiten}\\
$$
H=\sum_{i=1}^{n} h_{i}
$$
\\
$H$: Absolute Häufigkeit, \\
$h_{i}$: Einzelhäufigkeit der $i$-ten Beobachtung, \\
$n$: Anzahl der Beobachtungen.
\end{definition}
\end{minipage}
\begin{minipage}{0.5\columnwidth}
\begin{definition}{Relative Häufigkeiten}\\
$$
F=\sum_{i=1}^{m} f_{i}, \quad F(x)=\frac{H(x)}{n}
$$
\\
$F$: Relative Häufigkeit, \\
$f_{i}$: Einzelrelative Häufigkeit der $i$-ten Beobachtung, \\
$H(x)$: Absolute Häufigkeit eines Wertes $x$, \\
$n$: Anzahl der Beobachtungen.
\end{definition}
\end{minipage}

\subsection{Kennwerte (Lagemasse)}

\begin{minipage}{0.5\columnwidth}
\begin{definition}{Quantil}\\
$$
i=\lceil n \cdot q\rceil, \quad Q=x_{i}=x_{\lceil n \cdot q\rceil}
$$
\\
$i$: Position des Quantils, \\
$n$: Anzahl der Beobachtungen, \\
$q$: Quantilswert (z. B. 0.25 für das erste Quartil), \\
$x_{i}$: Beobachtung an Position $i$.
\end{definition}
\end{minipage}
\begin{minipage}{0.5\columnwidth}
\begin{definition}{Interquartilsabstand}\\
$$
I Q R=Q_{3}-Q_{1}
$$
\\
$IQR$: Interquartilsabstand, \\
$Q_{3}$: Oberes Quartil (75. Perzentil), \\
$Q_{1}$: Unteres Quartil (25. Perzentil).
\end{definition}
\end{minipage}

\begin{definition}{Modus}\\
$$
x_{\text {mod }}=\text{Häufigste Wert}
$$
\end{definition}

\begin{minipage}{0.5\columnwidth}
\begin{concept}{Arithmetisches Mittel}\\
$$
\bar{x}=\frac{1}{n} \sum_{i=1}^{n} x_{i}=\sum_{i=1}^{m} a_{i} \cdot f_{i}
$$
\\
$\bar{x}$: Arithmetisches Mittel, \\
$n$: Anzahl der Beobachtungen, \\
$x_{i}$: Einzelbeobachtung, \\
$a_{i}$: Klassenmitte, \\
$f_{i}$: Relative Häufigkeit der Klasse $i$.
\end{concept}
\end{minipage}%
\begin{minipage}{0.5\columnwidth}
\begin{concept}{Median}\\
\resizebox{\columnwidth}{!}{
$
\left\{\begin{array}{c}
x_{\left[\frac{n+1}{2}\right]} \quad n \text { ungerade } \\ 
0.5 \cdot\left(x_{\left[\frac{n}{2}\right]}+x_{\left[\frac{n}{2}+1\right]}\right) \quad n \text { gerade }
\end{array}\right.
$
}
\\
\\
$n$: Anzahl der Beobachtungen, \\
$x_{[k]}$: Beobachtung an der $k$-ten Position.
\end{concept}
\end{minipage}

\begin{definition}{Stichprobenvarianz $s^{2}$ (Streumasse)}\\
$$
s^{2}=\frac{1}{n} \sum_{i=1}^{n}\left(x_{i}-\bar{x}\right)^{2}=\overline{x^{2}}-\bar{x}^{2}, \quad\left(s_{\text{kor}}\right)^{2}=\frac{1}{n-1} \sum_{i=1}^{n}\left(x_{i}-\bar{x}\right)^{2}
$$
$$
\left(s_{\text{kor}}\right)^{2}=\frac{n}{n-1} \cdot s^{2}
$$
\\
$s^{2}$: Stichprobenvarianz, \\
$s_{\text{kor}}^{2}$: Korrigierte Stichprobenvarianz, \\
$x_{i}$: Einzelbeobachtung, \\
$\bar{x}$: Arithmetisches Mittel, \\
$n$: Anzahl der Beobachtungen.
\end{definition}

\begin{definition}{Standardabweichung $s$ (Streumasse)}\\
$$
s=\sqrt{\frac{1}{n} \sum_{i=1}^{n}\left(x_{i}-\bar{x}\right)^{2}}=\sqrt{\overline{x^{2}}-\bar{x}^{2}}, \quad s_{\text{kor}}=\sqrt{\frac{1}{n-1} \sum_{i=1}^{n}\left(x_{i}-\bar{x}\right)^{2}}
$$
\\
$s$: Standardabweichung, \\
$s_{\text{kor}}$: Korrigierte Standardabweichung, \\
$x_{i}$: Einzelbeobachtung, \\
$\bar{x}$: Arithmetisches Mittel, \\
$n$: Anzahl der Beobachtungen.
\end{definition}

\subsection{PDF + CDF}

\begin{definition}{Nicht klassierte Daten (PMF und CDF)}\\
Die absolute Häufigkeit kann als Funktion $h: \mathbb{R} \rightarrow \mathbb{R}$ bezeichnet werden.
$$
h_{i}
$$
\\
$h_{i}$: Absolute Häufigkeit der $i$-ten Beobachtung.
\\
\\
Die relative Häufigkeit kann als Funktion $f: \mathbb{R} \rightarrow \mathbb{R}$ bezeichnet werden.
$$
f_{i}=\frac{h_{i}}{n}
$$
\\
$f_{i}$: Relative Häufigkeit der $i$-ten Beobachtung, \\
$h_{i}$: Absolute Häufigkeit der $i$-ten Beobachtung, \\
$n$: Anzahl der Beobachtungen.
\end{definition}
