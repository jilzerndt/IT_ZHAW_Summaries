\section{Deskriptive Statistik}

\begin{definition}{Stichprobe}
    Eine \textbf{Stichprobe} $(x_1,...,x_n)$ der Länge $n$ ist eine Teilmenge einer Grundgesamtheit. 
    Meist handelt es sich um eine Zufallsstichprobe, d.h. die Auswahl ist zufällig und repräsentativ.
\end{definition}

\begin{definition}{Merkmal}
    Ein \textbf{Merkmal} ist eine Eigenschaft, die in einer Stichprobe untersucht wird. 
    Es kann quantitativ (z.B. Größe) oder qualitativ (z.B. Farbe) sein.
    
\end{definition}

\subsection{Mehrere Merkmale}