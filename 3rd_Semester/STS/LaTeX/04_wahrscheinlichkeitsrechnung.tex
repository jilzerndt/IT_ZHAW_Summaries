\section{Wahrscheinlichkeitsrechnung}
\begin{concept}{Spezialfälle der Kombinatorik}
\begin{example}{Romme Beispiel}
Beim Rommé spielt man mit 110 Karten: sechs davon sind Joker. Zu Beginn eines Spiels erhält jeder Spieler genau 12 Karten.

Wahrscheinlichkeit für genau zwei Joker:
$$
\frac{\binom{6}{2} \cdot\binom{104}{10}}{\binom{110}{12}}
$$
\\
$\binom{6}{2}$ = Anzahl Möglichkeiten 2 Joker aus 6 zu wählen\\
$\binom{104}{10}$ = Anzahl Möglichkeiten 10 Nicht-Joker aus 104 zu wählen\\
$\binom{110}{12}$ = Gesamtanzahl Möglichkeiten 12 Karten aus 110 zu wählen\\

Wahrscheinlichkeit für mindestens einen Joker:
$$
1-\frac{\binom{104}{12}}{\binom{110}{12}}
$$
\\
$\binom{104}{12}$ = Anzahl Möglichkeiten 12 Nicht-Joker aus 104 zu wählen\\
$\binom{110}{12}$ = Gesamtanzahl Möglichkeiten 12 Karten aus 110 zu wählen\\
\end{example}

\begin{example}{Glühbirnen Beispiel}
Von 100 Glühbirnen sind genau drei defekt. Es werden nun 6 Glühbirnen zufällig ausgewählt.

Anzahl Möglichkeiten mit mindestens einer defekten Glühbirne:
$$
\binom{100}{6}-\binom{97}{6}=203'880'032
$$
\\
$\binom{100}{6}$ = Gesamtanzahl Möglichkeiten 6 Glühbirnen aus 100 zu wählen\\
$\binom{97}{6}$ = Anzahl Möglichkeiten 6 intakte Glühbirnen aus 97 zu wählen\\

Wahrscheinlichkeit für keine defekte Glühbirne:
$$
\frac{\binom{97}{6}}{\binom{100}{6}}
$$
\\
$\binom{97}{6}$ = Anzahl Möglichkeiten 6 intakte Glühbirnen aus 97 zu wählen\\
$\binom{100}{6}$ = Gesamtanzahl Möglichkeiten 6 Glühbirnen aus 100 zu wählen\\
\end{example}
\end{concept}

\subsection{Wahrscheinlichkeitstheorie}
\begin{definition}{Ergebnisraum}
Ergebnisraum $\Omega$ ist die Menge aller möglichen Ergebnisse des Zufallsexperiments. Zähldichte $\rho: \Omega \rightarrow[0,1]$ ordnet jedem Ereignis seine Wahrscheinlichkeit zu.

Für ein Laplace-Raum $(\Omega, P)$ gilt:
$$
P(M)=\frac{|M|}{|\Omega|}
$$
\\
$\Omega$ = Ergebnisraum (Menge aller möglichen Ergebnisse)\\
$P(M)$ = Wahrscheinlichkeit des Ereignisses $M$\\
$|M|$ = Anzahl der für $M$ günstigen Ergebnisse\\
$|\Omega|$ = Anzahl aller möglichen Ergebnisse\\
\end{definition}

\begin{theorem}{Stochastische Unabhängigkeit}
Zwei Ereignisse $A$ und $B$ heissen stochastisch unabhängig, falls:
$$
P(A \cap B)=P(A) \cdot P(B)
$$
\\
$P(A \cap B)$ = Wahrscheinlichkeit dass beide Ereignisse eintreten\\
$P(A)$ = Wahrscheinlichkeit von Ereignis $A$\\
$P(B)$ = Wahrscheinlichkeit von Ereignis $B$\\

Zwei Zufallsvariablen $X: \Omega \rightarrow \mathbb{R}$ und $Y: \Omega \rightarrow \mathbb{R}$ heissen stochastisch unabhängig, falls:
$$
P(X=x, Y=y)=P(X=x) \cdot P(Y=y), \quad \text{für alle } x, y \in \mathbb{R}
$$
\\
$P(X=x, Y=y)$ = Wahrscheinlichkeit dass $X$ den Wert $x$ und $Y$ den Wert $y$ annimmt\\
$P(X=x)$ = Wahrscheinlichkeit dass $X$ den Wert $x$ annimmt\\
$P(Y=y)$ = Wahrscheinlichkeit dass $Y$ den Wert $y$ annimmt\\
\end{theorem}

\subsection{Bedingte Wahrscheinlichkeit}
\begin{definition}{Bedingte Wahrscheinlichkeit}
$$
P(B \mid A)=\frac{P(B \cap A)}{P(A)}
$$
\\
$P(B|A)$ = Wahrscheinlichkeit von $B$ unter der Bedingung dass $A$ eingetreten ist\\
$P(B \cap A)$ = Wahrscheinlichkeit dass beide Ereignisse eintreten\\
$P(A)$ = Wahrscheinlichkeit von Ereignis $A$\\
\end{definition}

\begin{theorem}{Multiplikationssatz}
$$
P(A \cap B)=P(A) \cdot P(B \mid A)=P(B) \cdot P(A \mid B)
$$
\\
$P(A \cap B)$ = Wahrscheinlichkeit dass beide Ereignisse eintreten\\
$P(A)$ = Wahrscheinlichkeit von Ereignis $A$\\
$P(B|A)$ = Wahrscheinlichkeit von $B$ unter der Bedingung dass $A$ eingetreten ist\\
$P(A|B)$ = Wahrscheinlichkeit von $A$ unter der Bedingung dass $B$ eingetreten ist\\
\end{theorem}

\begin{theorem}{Satz von der Totalen Wahrscheinlichkeit}
$$
P(B)=P(A) \cdot P(B \mid A)+P(\bar{A}) \cdot P(B \mid \bar{A})
$$
\\
$P(B)$ = Wahrscheinlichkeit von Ereignis $B$\\
$P(A)$ = Wahrscheinlichkeit von Ereignis $A$\\
$P(\bar{A})$ = Wahrscheinlichkeit des Gegenereignisses von $A$\\
$P(B|A)$ = Wahrscheinlichkeit von $B$ unter der Bedingung dass $A$ eingetreten ist\\
$P(B|\bar{A})$ = Wahrscheinlichkeit von $B$ unter der Bedingung dass $A$ nicht eingetreten ist\\
\end{theorem}

\begin{theorem}{Satz von Bayes}
$$
P(A \mid B)=\frac{P(A) \cdot P(B \mid A)}{P(B)}
$$
\\
$P(A|B)$ = Wahrscheinlichkeit von $A$ unter der Bedingung dass $B$ eingetreten ist\\
$P(A)$ = Wahrscheinlichkeit von Ereignis $A$\\
$P(B|A)$ = Wahrscheinlichkeit von $B$ unter der Bedingung dass $A$ eingetreten ist\\
$P(B)$ = Wahrscheinlichkeit von Ereignis $B$\\
\end{theorem}
