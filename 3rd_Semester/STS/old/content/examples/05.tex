\subsection{Spezielle Verteilungen Beispiele}

\subsubsection{Hypergeometrische Verteilung}

\begin{example2}{Ziehung ohne Zurücklegen}
\textbf{Aufgabe:} In einer Urne sind 20 Kugeln, davon 8 rot. Es werden 5 Kugeln ohne Zurücklegen gezogen. 

\textbf{Lösung:}
\begin{enumerate}
\item \textbf{Parameter:}
   \begin{itemize}
   \item N = 20 (Gesamtanzahl)
   \item M = 8 (rote Kugeln)
   \item n = 5 (Ziehungen)
   \end{itemize}

\item \textbf{Erwartungswert:}
   $$E(X) = 5 \cdot \frac{8}{20} = 2$$

\item \textbf{Varianz:}
   $$V(X) = 5 \cdot \frac{8}{20} \cdot \frac{12}{20} \cdot \frac{15}{19} \approx 1.184$$

\item \textbf{P(genau 2 rote):}
   $$P(X=2) = \frac{\binom{8}{2}\binom{12}{3}}{\binom{20}{5}} \approx 0.3682$$
\end{enumerate}
\end{example2}

\begin{example2}{Lotterie}
\textbf{Aufgabe:} In einer Urne sind 100 Lose, davon 10 Gewinnerlose. Ein Spieler zieht 5 Lose.

\textbf{Parameter:}
\begin{itemize}
\item N = 100 (Gesamtlose)
\item M = 10 (Gewinnerlose)
\item n = 5 (gezogene Lose)
\end{itemize}

\textbf{Berechnung:}
\begin{itemize}
\item $E(X) = 5 \cdot \frac{10}{100} = 0.5$ Gewinne erwartet
\item $V(X) = 5 \cdot \frac{10}{100} \cdot \frac{90}{100} \cdot \frac{95}{99} \approx 0.432$
\item $P(X=1) = \frac{\binom{10}{1} \cdot \binom{90}{4}}{\binom{100}{5}} \approx 0.3726$
\end{itemize}
\end{example2}

\subsubsection{Bernoulli-Verteilung}

\begin{example2}{Münzwurf}\\
\textbf{Aufgabe:} Faire Münze wird geworfen. X = 1 bei Kopf, X = 0 bei Zahl.

\textbf{Lösung:}
\begin{itemize}
\item p = 0.5 (faire Münze)
\item $E(X) = 0.5$
\item $V(X) = 0.5 \cdot 0.5 = 0.25$
\item $P(X=1) = 0.5$
\item $P(X=0) = 0.5$
\end{itemize}
\end{example2}

\subsubsection{Binomialverteilung}

\begin{example2}{Qualitätskontrolle mit Binomialverteilung}
\textbf{Aufgabe:} Eine Maschine produziert Teile mit Ausschussquote 5\%. In einer Stichprobe von 100 Teilen:
\begin{itemize}
\item a) Wie viele defekte Teile sind zu erwarten?
\item b) Wie groß ist die Wahrscheinlichkeit für genau 3 defekte Teile?
\item c) Wie groß ist die Wahrscheinlichkeit für höchstens 2 defekte Teile?
\end{itemize}

\textbf{Lösung:}
\begin{enumerate}
\item \textbf{Parameter:}
   \begin{itemize}
   \item n = 100 (Stichprobenumfang)
   \item p = 0.05 (Ausschusswahrscheinlichkeit)
   \item X $\thicksim$ B(100, 0.05)
   \end{itemize}

\item \textbf{Erwartungswert:}
   $$E(X) = np = 100 \cdot 0.05 = 5$$

\item \textbf{Genau 3 defekte:}
   $$P(X=3) = \binom{100}{3}(0.05)^3(0.95)^{97} \approx 0.1404$$

\item \textbf{Höchstens 2 defekte:}
   $$P(X \leq 2) = \sum_{k=0}^2 \binom{100}{k}(0.05)^k(0.95)^{100-k} \approx 0.0861$$
\end{enumerate}
\end{example2}

\begin{example2}{Binomialverteilung in der Qualitätskontrolle}
\textbf{Aufgabe:} Ein Produktionsprozess hat eine Fehlerquote von 5\%. In einer Stichprobe von 100 Teilen:

\textbf{Parameter:}
\begin{itemize}
\item n = 100 (Stichprobenumfang)
\item p = 0.05 (Fehlerwahrscheinlichkeit)
\item X $\sim$  B(100, 0.05)
\end{itemize}

\textbf{Berechnung:}
\begin{itemize}
\item $E(X) = 100 \cdot 0.05 = 5$ defekte Teile erwartet
\item $V(X) = 100 \cdot 0.05 \cdot 0.95 = 4.75$
\item $P(X=3) = \binom{100}{3} \cdot 0.05^3 \cdot 0.95^{97} \approx 0.1754$
\item $P(X \leq 2) = \sum_{k=0}^2 \binom{100}{k} \cdot 0.05^k \cdot 0.95^{100-k} \approx 0.1247$
\end{itemize}
\end{example2}

\subsubsection{Poisson-Verteilung}

\begin{example2}{Anrufe in Call-Center}
\textbf{Aufgabe:} Ein Call-Center erhält durchschnittlich 4 Anrufe pro Stunde.

\textbf{Parameter:}
\begin{itemize}
\item $\lambda = 4$ (Anrufe pro Stunde)
\item X $\sim$ Poi(4)
\end{itemize}

\textbf{Berechnung:}
\begin{itemize}
\item $E(X) = 4$ Anrufe erwartet
\item $V(X) = 4$
\item $P(X=3) = \frac{4^3}{3!} \cdot e^{-4} \approx 0.1954$
\item $P(X \leq 2) = e^{-4} \cdot (1 + 4 + \frac{16}{2}) \approx 0.2381$
\end{itemize}
\end{example2}

\begin{example2}{Poisson-Verteilung in der Praxis}
\textbf{Aufgabe:} Ein Callcenter erhält durchschnittlich 3 Anrufe pro 10 Minuten.
\begin{itemize}
\item a) Wahrscheinlichkeit für genau 2 Anrufe in 10 Minuten?
\item b) Wahrscheinlichkeit für mehr als 4 Anrufe?
\end{itemize}

\textbf{Lösung:}
\begin{enumerate}
\item \textbf{Parameter:}
   \begin{itemize}
   \item $\lambda = 3$ (Erwartungswert)
   \item X $\thicksim$ Poi(3)
   \end{itemize}

\item \textbf{Genau 2 Anrufe:}
   $$P(X=2) = \frac{3^2}{2!}e^{-3} \approx 0.2240$$ 

\item \textbf{Mehr als 4 Anrufe:}
   $$P(X>4) = 1 - \sum_{k=0}^4 \frac{3^k}{k!}e^{-3} \approx 0.1847$$
\end{enumerate}
\end{example2}

\subsubsection{Normalverteilung}

\begin{example2}{Körpergrößen}
\textbf{Aufgabe:} Körpergrößen in einer Population sind normalverteilt mit $\mu = 175$ cm und $\sigma = 10$ cm.

\textbf{Berechnung:}
\begin{itemize}
\item $P(X \leq 185) = \phi(\frac{185-175}{10}) = \phi(1) \approx 0.8413$
\item $P(165 \leq X \leq 185) = \phi(1) - \phi(-1) \approx 0.6826$
\item $P(X > 195) = 1 - \phi(2) \approx 0.0228$
\end{itemize}
\end{example2}

\subsubsection{Approximation}

\begin{example2}{Approximation der Binomialverteilung}
\textbf{Aufgabe:} Eine Produktionsanlage produziert mit Ausschusswahrscheinlichkeit 5\%. In einer Charge von 200 Teilen:
\begin{itemize}
\item Wie groß ist die Wahrscheinlichkeit für 15 oder mehr defekte Teile?
\end{itemize}

\textbf{Lösung:}
\begin{enumerate}
\item \textbf{Prüfung Approximationsbedingung:}
   \begin{itemize}
   \item $npq = 200 \cdot 0.05 \cdot 0.95 = 9.5 > 9$
   \item Normalapproximation ist zulässig
   \end{itemize}

\item \textbf{Parameter der Normalverteilung:}
   \begin{itemize}
   \item $\mu = np = 200 \cdot 0.05 = 10$
   \item $\sigma = \sqrt{npq} = \sqrt{9.5} \approx 3.08$
   \end{itemize}

\item \textbf{Berechnung mit Stetigkeitskorrektur:}
   \begin{align*}
   P(X \geq 15) &= 1 - P(X \leq 14) \\
   &= 1 - P(X \leq 14.5) \\
   &= 1 - \phi(\frac{14.5-10}{3.08}) \\
   &= 1 - \phi(1.46) \\
   &\approx 0.0721
   \end{align*}
\end{enumerate}
\end{example2}

\begin{example2}{Approximation durch Poissonverteilung}
\textbf{Aufgabe:} Ein seltener Gendefekt tritt mit Wahrscheinlichkeit p = 0.001 auf. In einer Gruppe von 2000 Menschen:
\begin{itemize}
\item Wie groß ist die Wahrscheinlichkeit für genau 3 Betroffene?
\end{itemize}

\textbf{Lösung:}
\begin{enumerate}
\item \textbf{Prüfung Approximationsbedingung:}
   \begin{itemize}
   \item $n = 2000 \geq 50$ und $p = 0.001 \leq 0.1$
   \item Poissonapproximation ist zulässig
   \end{itemize}

\item \textbf{Parameter:}
   \begin{itemize}
   \item $\lambda = np = 2000 \cdot 0.001 = 2$
   \end{itemize}

\item \textbf{Berechnung:}
   $$P(X = 3) = \frac{2^3}{3!} \cdot e^{-2} \approx 0.180$$

\item \textbf{Vergleich mit Binomialverteilung:}
   $$P_{Bin}(X = 3) = \binom{2000}{3} \cdot 0.001^3 \cdot 0.999^{1997} \approx 0.180$$
\end{enumerate}
\end{example2}