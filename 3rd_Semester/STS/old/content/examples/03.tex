\subsection{Kombinatorik Beispiele}

\begin{example2}{Variation mit Wiederholung}
\textbf{Zahlenschloss:} 6 Stellen, Ziffern 0-9 möglich
\begin{itemize}
    \item $n=10$ Ziffern
    \item $k=6$ Stellen
    \item Reihenfolge wichtig
    \item Wiederholung erlaubt
    \item Lösung: $10^6 = 1\,000\,000$ Möglichkeiten
\end{itemize}
\end{example2}

\begin{example2}{Variation ohne Wiederholung}
\textbf{Schwimmwettkampf:} Erste 3 Plätze bei 10 Schwimmern
\begin{itemize}
    \item $n=10$ Schwimmer
    \item $k=3$ Plätze
    \item Reihenfolge wichtig
    \item Keine Wiederholung möglich
    \item Lösung: $\frac{10!}{7!} = 720$ Möglichkeiten
\end{itemize}
\end{example2}

\begin{example2}{Kombination mit Wiederholung}
\textbf{Zahnarzt:} 3 Spielzeuge aus 5 verschiedenen Arten
\begin{itemize}
    \item $n=5$ Arten
    \item $k=3$ Spielzeuge
    \item Reihenfolge unwichtig
    \item Wiederholung möglich
    \item Lösung: $\binom{7}{3} = 35$ Möglichkeiten
\end{itemize}
\end{example2}

\begin{example2}{Kombination ohne Wiederholung}
\textbf{Lotto:} 6 aus 49
\begin{itemize}
    \item $n=49$ Zahlen
    \item $k=6$ Auswahl
    \item Reihenfolge unwichtig
    \item Keine Wiederholung
    \item Lösung: $\binom{49}{6} = 13\,983\,816$ Möglichkeiten
\end{itemize}
\end{example2}

\begin{example2}{Variation mit Wiederholung (Zahlenschloss)}
\textbf{Aufgabe:} Wie viele Möglichkeiten gibt es bei einem Zahlenschloss (0 -- 9) mit 6 Zahlenkränzen?

\textbf{Lösung:}
\begin{itemize}
\item $n = 10$ (Ziffern 0-9)
\item $k = 6$ (Stellen)
\item Reihenfolge wichtig: Ja (123456 $\neq$ 654321)
\item Wiederholungen erlaubt: Ja (11111 ist möglich)
\item Formel: $n^k = 10^6 = 1\,000\,000$ mögliche Kombinationen
\end{itemize}
\end{example2}

\begin{example2}{Variation ohne Wiederholung (Schwimmwettkampf)}
\textbf{Aufgabe:} Bei einem Schwimmwettkampf starten 10 Teilnehmer. Wie viele mögliche Platzierungen der ersten drei Plätze (Podest) gibt es?

\textbf{Lösung:}
\begin{itemize}
\item $n = 10$ (Teilnehmer)
\item $k = 3$ (Podestplätze)
\item Reihenfolge wichtig: Ja (1., 2., 3. Platz unterschiedlich)
\item Wiederholungen erlaubt: Nein (niemand kann mehrere Plätze belegen)
\item Formel: $\frac{n!}{(n-k)!} = \frac{10!}{7!} = 720$ mögliche Platzierungen
\end{itemize}
\end{example2}

\begin{example2}{Kombination mit Wiederholung (Zahnarzt)}
\textbf{Aufgabe:} 3 Spielzeuge werden aus 5 Töpfen gezogen. Jeder Topf ist mit einer (unterschiedlichen) Art von Spielzeug befüllt. Wie viele Möglichkeiten hat das Kind?

\textbf{Lösung:}
\begin{itemize}
\item $n = 5$ (Arten von Spielzeug)
\item $k = 3$ (zu wählende Spielzeuge)
\item Reihenfolge wichtig: Nein (nur Anzahl pro Art relevant)
\item Wiederholungen erlaubt: Ja (mehrere Spielzeuge gleicher Art möglich)
\item Formel: $\binom{n+k-1}{k} = \binom{7}{3} = 35$ Möglichkeiten
\end{itemize}
\end{example2}

\begin{example2}{Kombination ohne Wiederholung (Lotto)}
\textbf{Aufgabe:} Wie gross sind die Chancen beim Lotto 6 aus 49 Zahlen richtig zu ziehen?

\textbf{Lösung:}
\begin{itemize}
\item $n = 49$ (Zahlen insgesamt)
\item $k = 6$ (zu wählende Zahlen)
\item Reihenfolge wichtig: Nein (nur Auswahl relevant)
\item Wiederholungen erlaubt: Nein (jede Zahl nur einmal)
\item Formel: $\binom{49}{6} = 13\,983\,816$ Möglichkeiten
\item Gewinnwahrscheinlichkeit: $\frac{1}{13\,983\,816} \approx 0.0000000715$
\end{itemize}
\end{example2}

\begin{example2}{Variation mit Wiederholung (Zahlenschloss)}\\
Wie viele Möglichkeiten gibt es bei einem Zahlenschloss (0 -- 9) mit 6 Zahlenkränzen?

$$n = 10, \quad k = 6$$
$$n^k = 10^6$$
\end{example2}

\begin{example2}{Variation ohne Wiederholung (Schimmwettkampf)}\\
Bei einem Schwimmwettkampf starten 10 Teilnehmer. Wie viele mögliche Platzierungen der ersten drei Plätze (Podest) gibt es?

$$n = 10, \quad k = 3$$
$$\frac{n!}{(n-k)!} = \frac{10!}{(10-3)!} = \frac{10!}{7!}$$
\end{example2}

\begin{example2}{Kombination mit Wiederholung (Zahnarzt)}\\
3 Spielzeuge werden aus 5 Töpfen gezogen. Jeder Topf ist mit einer (unterschiedlichen) Art von Spielzeug befüllt.

Wie viele Möglichkeiten hat das Kind?

$$n = 5, \quad k = 3$$
$$\binom{n+k-1}{k} = \binom{5+3-1}{3} = \binom{7}{3}$$
\end{example2}

\begin{example2}{Kombination ohne Wiederholung (Lotto)}\\
Wie gross sind die Chancen beim Lotto 6 aus 49 Zahlen richtig zu ziehen?

Jede Zahl ist nur einmal vorhanden und die Zahlen werden nicht zurückgelegt. Die Reihenfolge in der gezogen wird spielt keine Rolle.

$$n = 49, \quad k = 6$$
$$\binom{n}{k} = \binom{49}{6}$$
\end{example2}

\begin{example2}{Komplexeres Beispiel: Passwörter}
\textbf{Aufgabe:} Ein Passwort muss bestehen aus:
\begin{itemize}
\item Genau 8 Zeichen
\item Mindestens ein Großbuchstabe (26 mögliche)
\item Mindestens eine Ziffer (10 mögliche)
\item Kleine Buchstaben erlaubt (26 mögliche)
\end{itemize}

\textbf{Lösung:}
1. Gesamtzahl aller möglichen 8-stelligen Passwörter mit den Zeichen:
   \begin{itemize}
   \item $n = 26 + 26 + 10 = 62$ Zeichen
   \item Variation mit Wiederholung: $62^8$
   \end{itemize}

2. Abziehen der ungültigen Kombinationen:
   \begin{itemize}
   \item Ohne Großbuchstaben: $(36)^8$
   \item Ohne Ziffern: $(52)^8$
   \item Ohne beide: $(26)^8$
   \end{itemize}

3. Nach dem Inklusions-Exclusions-Prinzip:
   \[ \text{Gültige Passwörter} = 62^8 - 36^8 - 52^8 + 26^8 \]
\end{example2}