\section{Deskriptive Statistik}

\begin{definition}{Bivariate Daten (Merkmale)}\\
\begin{itemize}
  \item 2x kategoriell $\rightarrow$ Kontingenztabelle + Mosaikplot
  \item 1x kategoriell + 1x metrisch $\rightarrow$ Boxplot oder Stripchart
  \item 2x metrisch $\rightarrow$ Streudiagramm
\end{itemize}
\end{definition}

\begin{minipage}{0.5\columnwidth}
\begin{definition}{Absolute Häufigkeiten}\\
$$
H=\sum_{i=1}^{n} h_{i}
$$
\\
$H$: Absolute Häufigkeit, \\
$h_{i}$: Einzelhäufigkeit der $i$-ten Beobachtung, \\
$n$: Anzahl der Beobachtungen.
\end{definition}
\end{minipage}
\begin{minipage}{0.5\columnwidth}
\begin{definition}{Relative Häufigkeiten}\\
$$
F=\sum_{i=1}^{m} f_{i}, \quad F(x)=\frac{H(x)}{n}
$$
\\
$F$: Relative Häufigkeit, \\
$f_{i}$: Einzelrelative Häufigkeit der $i$-ten Beobachtung, \\
$H(x)$: Absolute Häufigkeit eines Wertes $x$, \\
$n$: Anzahl der Beobachtungen.
\end{definition}
\end{minipage}

\subsection{Kennwerte (Lagemasse)}

\begin{minipage}{0.5\columnwidth}
\begin{definition}{Quantil}\\
$$
i=\lceil n \cdot q\rceil, \quad Q=x_{i}=x_{\lceil n \cdot q\rceil}
$$
\\
$i$: Position des Quantils, \\
$n$: Anzahl der Beobachtungen, \\
$q$: Quantilswert (z. B. 0.25 für das erste Quartil), \\
$x_{i}$: Beobachtung an Position $i$.
\end{definition}
\end{minipage}
\begin{minipage}{0.5\columnwidth}
\begin{definition}{Interquartilsabstand}\\
$$
I Q R=Q_{3}-Q_{1}
$$
\\
$IQR$: Interquartilsabstand, \\
$Q_{3}$: Oberes Quartil (75. Perzentil), \\
$Q_{1}$: Unteres Quartil (25. Perzentil).
\end{definition}
\end{minipage}

\begin{definition}{Modus}\\
$$
x_{\text {mod }}=\text{Häufigste Wert}
$$
\end{definition}

\begin{minipage}{0.5\columnwidth}
\begin{concept}{Arithmetisches Mittel}\\
$$
\bar{x}=\frac{1}{n} \sum_{i=1}^{n} x_{i}=\sum_{i=1}^{m} a_{i} \cdot f_{i}
$$
\\
$\bar{x}$: Arithmetisches Mittel, \\
$n$: Anzahl der Beobachtungen, \\
$x_{i}$: Einzelbeobachtung, \\
$a_{i}$: Klassenmitte, \\
$f_{i}$: Relative Häufigkeit der Klasse $i$.
\end{concept}
\end{minipage}%
\begin{minipage}{0.5\columnwidth}
\begin{concept}{Median}\\
\resizebox{\columnwidth}{!}{
$
\left\{\begin{array}{c}
x_{\left[\frac{n+1}{2}\right]} \quad n \text { ungerade } \\ 
0.5 \cdot\left(x_{\left[\frac{n}{2}\right]}+x_{\left[\frac{n}{2}+1\right]}\right) \quad n \text { gerade }
\end{array}\right.
$
}
\\
\\
$n$: Anzahl der Beobachtungen, \\
$x_{[k]}$: Beobachtung an der $k$-ten Position.
\end{concept}
\end{minipage}

\begin{definition}{Stichprobenvarianz $s^{2}$ (Streumasse)}\\
$$
s^{2}=\frac{1}{n} \sum_{i=1}^{n}\left(x_{i}-\bar{x}\right)^{2}=\overline{x^{2}}-\bar{x}^{2}, \quad\left(s_{\text{kor}}\right)^{2}=\frac{1}{n-1} \sum_{i=1}^{n}\left(x_{i}-\bar{x}\right)^{2}
$$
$$
\left(s_{\text{kor}}\right)^{2}=\frac{n}{n-1} \cdot s^{2}
$$
\\
$s^{2}$: Stichprobenvarianz, \\
$s_{\text{kor}}^{2}$: Korrigierte Stichprobenvarianz, \\
$x_{i}$: Einzelbeobachtung, \\
$\bar{x}$: Arithmetisches Mittel, \\
$n$: Anzahl der Beobachtungen.
\end{definition}

\begin{definition}{Standardabweichung $s$ (Streumasse)}\\
$$
s=\sqrt{\frac{1}{n} \sum_{i=1}^{n}\left(x_{i}-\bar{x}\right)^{2}}=\sqrt{\overline{x^{2}}-\bar{x}^{2}}, \quad s_{\text{kor}}=\sqrt{\frac{1}{n-1} \sum_{i=1}^{n}\left(x_{i}-\bar{x}\right)^{2}}
$$
\\
$s$: Standardabweichung, \\
$s_{\text{kor}}$: Korrigierte Standardabweichung, \\
$x_{i}$: Einzelbeobachtung, \\
$\bar{x}$: Arithmetisches Mittel, \\
$n$: Anzahl der Beobachtungen.
\end{definition}

\subsection{PDF + CDF}

\begin{definition}{Nicht klassierte Daten (PMF und CDF)}\\
Die absolute Häufigkeit kann als Funktion $h: \mathbb{R} \rightarrow \mathbb{R}$ bezeichnet werden.
$$
h_{i}
$$
\\
$h_{i}$: Absolute Häufigkeit der $i$-ten Beobachtung.
\\
\\
Die relative Häufigkeit kann als Funktion $f: \mathbb{R} \rightarrow \mathbb{R}$ bezeichnet werden.
$$
f_{i}=\frac{h_{i}}{n}
$$
\\
$f_{i}$: Relative Häufigkeit der $i$-ten Beobachtung, \\
$h_{i}$: Absolute Häufigkeit der $i$-ten Beobachtung, \\
$n$: Anzahl der Beobachtungen.
\end{definition}

\subsection{PMF und CDF für diskrete und stetige Daten}

\begin{definition}{Diskrete Verteilungsfunktionen}\\
Die absolute Häufigkeit kann als Funktion $h: \mathbb{R} \rightarrow \mathbb{R}$ bezeichnet werden:
$$h_i$$

Die relative Häufigkeit kann als Funktion $f: \mathbb{R} \rightarrow \mathbb{R}$ bezeichnet werden:
$$f_i = \frac{h_i}{n}$$

\begin{example}{Diskrete Häufigkeitsverteilung}\\
\renewcommand{\arraystretch}{2}%
\begin{center}
\begin{tabular}{|c|c|c|c|c|c|}
\hline
$a_i$ & 397 & 398 & 399 & 400 & Total \\
\hline
$h_i$ & 1 & 3 & 7 & 5 & 16 \\
\hline
$f_i$ & $\frac{1}{16}$ & $\frac{3}{16}$ & $\frac{7}{16}$ & $\frac{5}{16}$ & 1 \\
\hline
$H_i$ & 1 & 4 & 11 & 16 & \\
\hline
$F_i$ & $\frac{1}{16}$ & $\frac{4}{16}$ & $\frac{11}{16}$ & $\frac{16}{16}$ & \\
\hline
\end{tabular}
\end{center}
\end{example}
\end{definition}

\begin{concept}{Klassenbildung (Faustregeln)}\\
\begin{itemize}
  \item Die Klassen sollten gleich breit gewählt werden
  \item Die Anzahl der Klassen sollte zwischen 5 und 20 liegen, jedoch $\sqrt{n}$ nicht überschreiben.
\end{itemize}
\end{concept}

\begin{definition}{Stetige Verteilungsfunktionen}\\
Die absolute Häufigkeitsdichtefunktion erhält man, indem der Wert der absoluten Häufigkeit $h_i$ durch die Klassenbreite (Säulenbreite) $d_i$ geteilt wird:
$$h(x) = \frac{h_i}{d_i}$$

Die relative Häufigkeitsdichtefunktion (PDF) $f: \mathbb{R} \rightarrow [0,1]$ erhält man aus der absoluten Häufigkeitsdichtefunktion, indem man den Wert durch die Stichprobengrösse $n$ teilt:
$$\text{PDF} = f(x) = \frac{h(x)}{n}$$

\begin{example}{Stetige Häufigkeitsverteilung}\\
\renewcommand{\arraystretch}{2}%
\begin{center}
\begin{tabular}{|c|c|c|c|c|c|}
\hline
Klassen & 100-200 & 200-500 & 500-800 & 800-1000 & Total \\
\hline
$h_i$ & 35 & 182 & 317 & 84 & 618 \\
\hline
$f_i$ & $\frac{35}{618}$ & $\frac{182}{618}$ & $\frac{317}{618}$ & $\frac{84}{618}$ & Area = 1 \\
\hline
$d_i$ & 100 & 300 & 300 & 200 & \\
\hline
$h(x)$ & $\frac{35}{100}$ & $\frac{182}{300}$ & $\frac{317}{300}$ & $\frac{84}{200}$ & \\
\hline
$f(x)$ & $\frac{35}{100 \cdot 618}$ & $\frac{182}{300 \cdot 618}$ & $\frac{317}{300 \cdot 618}$ & $\frac{84}{200 \cdot 618}$ & \\
\hline
\end{tabular}
\end{center}
\end{example}
\end{definition}

\begin{definition}{Varianz und Kovarianz}\\
\textbf{Varianz $s_x^2, s_y^2$}:
$$(s_x)^2 = \overline{x^2} - \bar{x}^2, \quad (s_y)^2 = \overline{y^2} - \bar{y}^2$$

\textbf{Kovarianz $s_{xy}$}:
$$s_{xy} = \frac{1}{n}\sum_{i=1}^{n}(x_i - \bar{x})(y_i - \bar{y}), \quad s_{xy} = \overline{xy} - \bar{x} \cdot \bar{y}$$
\end{definition}

\begin{concept}{Abkürzungen}\\
$$\bar{x} = \frac{1}{n}\sum_{i=1}^{n} x_i$$
$$\bar{y} = \frac{1}{n}\sum_{i=1}^{n} y_i$$
$$\overline{xy} = \frac{1}{n}\sum_{i=1}^{n} x_i \cdot y_i$$
\end{concept}

\begin{definition}{Rang-Varianz und Kovarianz}\\
\textbf{Varianz (Ränge) $(s_{rg(x)})^2, (s_{rg(y)})^2$}:
$$(s_{rg(x)})^2 = \overline{rg(x)^2} - (\overline{rg(x)})^2, \quad (s_{rg(y)})^2 = \overline{rg(y)^2} - (\overline{rg(y)})^2$$

\textbf{Kovarianz (Ränge) $s_{rg(xy)}$}:
$$s_{rg(xy)} = \overline{rg(xy)} - \overline{rg(x)} \cdot \overline{rg(y)} = \overline{rg(xy)} - \frac{(n+1)^2}{4}$$
\end{definition}

\begin{definition}{Der Korrelationskoeffizient (Pearson) $r_{xy}$}\\
$$r_{xy} = \frac{s_{xy}}{s_x \cdot s_y} = \frac{\overline{xy} - \bar{x} \cdot \bar{y}}{\sqrt{\overline{x^2} - \bar{x}^2} \cdot \sqrt{\overline{y^2} - \bar{y}^2}}$$

Ist der Korrelationskoeffizient $r_{xy}$:
\begin{itemize}
  \item $r_{xy} \approx 1 \rightarrow$ starker positiver linearer Zusammenhang
  \item $r_{xy} \approx -1 \rightarrow$ starker negativer linearer Zusammenhang
  \item $r_{xy} \approx 0 \rightarrow$ keine lineare Korrelation
\end{itemize}
\end{definition}

\begin{remark}{Bemerkungen}\\
Auch wenn zwischen zwei Grössen eine Korrelation besteht, so muss das noch lange nicht einen \emph{kausalen Zusammenhang} bedeuten. Man spricht von \emph{Scheinkorrelation}.
\end{remark}

\begin{concept}{Graphische Darstellung}\\
\begin{itemize}
 \item Form \hspace{1.5cm} linear / gekrümmt
 \item Richtung \hspace{0.85cm} positiver / negativer Zusammenhang  
 \item Stärke \hspace{1.2cm} starke / schwache Streuung
\end{itemize}
\end{concept}
\begin{definition}{Korrelationskoeffizient (Spearman) $r_{sp}$}\\
$$r_{sp} = \frac{s_{rg(xy)}}{s_{rg(x)} \cdot s_{rg(y)}} = \frac{\overline{rg(xy)} - \overline{rg(x)} \cdot \overline{rg(y)}}{\sqrt{\overline{rg(x)^2} - (\overline{rg(x)})^2} \cdot \sqrt{\overline{rg(y)^2} - (\overline{rg(y)})^2}}$$

Vereinfachte Formel, sofern \emph{alle Ränge unterschiedlich} sind:
$$r_{sp} = 1 - \frac{6 \cdot \sum_{i=1}^n d_i^2}{n \cdot (n^2 - 1)}, \quad \text{mit } d_i = rg(x_i) - rg(y_i)$$

\textbf{Ränge}\\
Der Rang $rg(x_i)$ des Stichprobenwertes $x_i$ ist definiert als der Index von $x_i$ in der nach der Grösse geordneten Stichprobe.

\begin{center}
\begin{tabular}{|c|c|c|c|c|c|c|}
\hline
$i$ & 1 & 2 & 3 & 4 & 5 & 6 \\
\hline
$x_i$ & 23 & 27 & 35 & 35 & 42 & 59 \\
\hline
$rg(x_i)$ & 1 & 2 & 3.5 & 3.5 & 5 & 6 \\
\hline
\end{tabular}
\end{center}
\end{definition}
