\section{Kombinatorik}
\begin{definition}{Fakultät}\\
$$
n!=1 \cdot 2 \cdot \ldots \cdot n=\prod_{k=1}^{n} k
$$

   $n$ = Die positive ganze Zahl, für die die Fakultät berechnet wird\\
   $k$ = Laufvariable in der Produktnotation\\
   $\prod$ = Produkt aller Terme von $k=1$ bis $n$\\
\end{definition}

\begin{definition}{Binomialkoeffizient}\\
Wie viele Möglichkeiten gibt es $k$ Objekte aus einer Gesamtheit von $n$ Objekten auszuwählen.
$$
\binom{n}{k}=\frac{n!}{(n-k)!\cdot k!}
$$
   $n$ = Gesamtanzahl der Objekte in der Menge\\
   $k$ = Anzahl der auszuwählenden Objekte\\
   $n!$ = Fakultät von $n$\\
   $(n-k)!$ = Fakultät von $(n-k)$\\
   $k!$ = Fakultät von $k$\\
\end{definition}

\subsection{Systematik}
\begin{concept}{Grundbegriffe}\\
\begin{center}
\resizebox{\columnwidth}{!}{
\begin{tabular}{|c|c|c|c|}
\hline
\multicolumn{2}{|c|}{Variation (mit Reihenfolge)} & \multicolumn{2}{c|}{Kombination (ohne Reihenfolge)} \\
\hline
Mit Wiederholung & Ohne Wiederholung & Mit Wiederholung & Ohne Wiederholung \\
\hline
$n^{k}$ & $\frac{n!}{(n-k)!}$ & $\binom{n+k-1}{k}$ & $\binom{n}{k}$ \\
\hline
Zahlenschloss & Schwimmwettkampf & Zahnarzt & Lotto \\
\hline
\end{tabular}
}
\end{center}
\end{concept}
\begin{example}{Variation mit Wiederholung (Zahlenschloss)}\\
Wie viele Möglichkeiten gibt es bei einem Zahlenschloss (0 -- 9) mit 6 Zahlenkränzen?

$$n = 10, \quad k = 6$$
$$n^k = 10^6$$
\end{example}

\begin{example}{Variation ohne Wiederholung (Schimmwettkampf)}\\
Bei einem Schwimmwettkampf starten 10 Teilnehmer. Wie viele mögliche Platzierungen der ersten drei Plätze (Podest) gibt es?

$$n = 10, \quad k = 3$$
$$\frac{n!}{(n-k)!} = \frac{10!}{(10-3)!} = \frac{10!}{7!}$$
\end{example}

\begin{example}{Kombination mit Wiederholung (Zahnarzt)}\\
3 Spielzeuge werden aus 5 Töpfen gezogen. Jeder Topf ist mit einer (unterschiedlichen) Art von Spielzeug befüllt.

Wie viele Möglichkeiten hat das Kind?

$$n = 5, \quad k = 3$$
$$\binom{n+k-1}{k} = \binom{5+3-1}{3} = \binom{7}{3}$$
\end{example}

\begin{example}{Kombination ohne Wiederholung (Lotto)}\\
Wie gross sind die Chancen beim Lotto 6 aus 49 Zahlen richtig zu ziehen?

Jede Zahl ist nur einmal vorhanden und die Zahlen werden nicht zurückgelegt. Die Reihenfolge in der gezogen wird spielt keine Rolle.

$$n = 49, \quad k = 6$$
$$\binom{n}{k} = \binom{49}{6}$$
\end{example}
