\section{Kombinatorik}


\subsection{Grundbegriffe}

\begin{definition}{Fakultät}
$n!$ einer natürlichen Zahl $n$ ist definiert als das Produkt aller positiven ganzen Zahlen bis zu dieser Zahl:
\vspace{-2mm}\\
$$
n!=1 \cdot 2 \cdot \ldots \cdot n=\prod_{a=1}^{n} a \text{ mit } 0! = 1 \text{ als Definitionsvereinbarung}
$$
\vspace{-4mm}\\
\textbf{Parameter:}
\begin{itemize}
    \item $n$ = Die positive ganze Zahl, für die die Fakultät berechnet wird
    \item $a$ = Laufvariable in der Produktnotation
    \item $\prod$ = Produkt aller Terme von $a=1$ bis $n$
\end{itemize}
\end{definition}

\begin{definition}{Binomialkoeffizient}
$\binom{n}{k}$ für natürliche Zahlen $0 \leq k \leq n$:

\begin{minipage}{0.5\textwidth}
$$\binom{n}{k} = \frac{n!}{(n-k)! \cdot k!}$$
\end{minipage}
\begin{minipage}{0.5\textwidth}
$\rightarrow$ Anzahl Möglichkeiten, aus \\$n$ Objekten $k$ Objekte auszuwählen.
\end{minipage}
\end{definition}

\begin{theorem}{Eigenschaften}
Für den Binomialkoeffizienten gelten:

\begin{minipage}{0.5\textwidth}
   \textbf{Leere Menge:} $\binom{n}{0} = \binom{n}{n} = 1$
   \vspace{1mm}\\
   \textbf{Symmetrie:} $\binom{n}{k} = \binom{n}{n-k}$
\end{minipage}
\begin{minipage}{0.5\textwidth}
   $$\text{\textbf{Summe:}} \sum_{k=0}^n \binom{n}{k} = 2^n$$
\end{minipage}

\textbf{Pascal'sche Rekursion:} $\binom{n+1}{k+1} = \binom{n}{k} + \binom{n}{k+1}$

\end{theorem}

\begin{KR}{Berechnung von Binomialkoeffizienten}\\
1. \textbf{Prüfe Spezialfälle:} $\binom{n}{0} = \binom{n}{n} = 1$ und $\binom{n}{1} = n$

2. \textbf{Nutze Symmetrie:} $\binom{n}{k} = \binom{n}{n-k}$

3. \textbf{Pascal'sches Dreieck:} Baue schrittweise auf, nutze Rekursionsformel

4. \textbf{Direkte Berechnung:} Nur wenn nötig, kürze vor dem Ausrechnen
\end{KR}

\subsubsection{Grundlegende Abzählmethoden}

\begin{concept}{Systematik der Kombinatorik}
\vspace{1mm}\\
\resizebox{\textwidth}{!}{
\begin{tabular}{|l|c|c|}
\hline 
& \textbf{Mit Wiederholung} & \textbf{Ohne Wiederholung} \\
\hline
\textbf{Variation} & \multirow{2}{3mm}{$n^k$} & \multirow{2}{3mm}{$\frac{n!}{(n-k)!}$} \\
(Reihenfolge wichtig) & & \\
\hline
\textbf{Kombination} & \multirow{2}{3mm}{$\binom{n+k-1}{k}$} & \multirow{2}{3mm}{$\binom{n}{k}$} \\
(Reihenfolge unwichtig) & & \\
\hline
\end{tabular}}
\end{concept}

\begin{KR}{Bestimmung der Abzählmethode}\\
1. \textbf{Analysiere das Problem:}
   \begin{itemize}
   \item $n$: Anzahl verfügbarer Objekte, $k$: Anzahl auszuwählender Objekte
   \end{itemize}

2. \textbf{Prüfe die Reihenfolge:}
   \begin{itemize}
   \item Ist die Reihenfolge wichtig? $\rightarrow$ Variation
   \item Ist nur die Auswahl wichtig? $\rightarrow$ Kombination
   \end{itemize}

3. \textbf{Prüfe Wiederholungen:}
   \begin{itemize}
   \item Dürfen Objekte mehrfach vorkommen? $\rightarrow$ Mit Wiederholung
   \item Darf jedes Objekt nur einmal? $\rightarrow$ Ohne Wiederholung
   \end{itemize}
\end{KR}

\begin{KR}{Lösen komplexer kombinatorischer Probleme}\\
1. \textbf{Problem zerlegen}
   \begin{itemize}
   \item Teile das Problem in unabhängige Teilprobleme
   \item Identifiziere abhängige Entscheidungen
   \end{itemize}

2. \textbf{Für jedes Teilproblem:} Wähle passende Abzählmethode

3. \textbf{Kombiniere Teillösungen}
   \begin{itemize}
   \item Unabhängige Ereignisse: Multipliziere
   \item Sich ausschließende Ereignisse: Addiere
   \item Prüfe Überlappungen (Inklusions-Exclusions)
   \end{itemize}
\end{KR}



