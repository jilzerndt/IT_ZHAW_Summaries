\section{Weitere Beispiele}

\subsection{Deskriptive Statistik}

\begin{example2}{Merkmalstypen - Praktische Beispiele}
\begin{itemize}
    \item \textbf{Nominal}: Geschlecht, Automarke, Blutgruppe
    \item \textbf{Ordinal}: Bildungsabschluss, Zufriedenheit (1-5), Kaufkraft (tief/mittel/hoch)
    \item \textbf{Diskret metrisch}: Anzahl Kinder, Würfelaugen, Stockwerke
    \item \textbf{Stetig metrisch}: Temperatur, Gewicht, Länge
\end{itemize}
\end{example2}

\subsection{Kombinatorik}

\begin{example2}{Zahlenschloss}
Ein Zahlenschloss hat 6 Stellen, jede mit Ziffern 0-9.
\begin{itemize}
\item Reihenfolge wichtig? Ja (Variation)
\item Wiederholungen erlaubt? Ja
\item Formel: $n^k = 10^6$ mögliche Kombinationen
\end{itemize}
\end{example2}

\begin{example2}{Lotto}
6 aus 49 ziehen:
\begin{itemize}
\item Reihenfolge wichtig? Nein (Kombination)
\item Wiederholungen erlaubt? Nein
\item Formel: $\binom{49}{6} = \frac{49!}{6!(49-6)!}$ mögliche Kombinationen
\end{itemize}
\end{example2}

\begin{example2}{Zahnarztproblem}
3 Spielzeuge aus 5 Töpfen ziehen:
\begin{itemize}
\item Reihenfolge wichtig? Nein (Kombination)
\item Wiederholungen erlaubt? Ja
\item Formel: $\binom{5+3-1}{3} = \binom{7}{3}$ mögliche Kombinationen
\end{itemize}
\end{example2}

\begin{example2}{Komplexeres Beispiel: Mannschaftsauswahl}
In einer Klasse von 20 Studierenden sollen:
\begin{itemize}
\item Eine 11er Fußballmannschaft gebildet werden
\item Mit genau 6 Frauen und 5 Männern
\item Es gibt 8 Frauen und 12 Männer in der Klasse
\end{itemize}

\textbf{Lösung:}
1. Wähle 6 aus 8 Frauen: $\binom{8}{6}$
2. Wähle 5 aus 12 Männern: $\binom{12}{5}$
3. Multipliziere: $\binom{8}{6} \cdot \binom{12}{5}$ = 22,176 Möglichkeiten
\end{example2}

\subsection{Wahrscheinlichkeitsrechnung}

\begin{example2}{Wahrscheinlichkeit bei Rommé}\\
Beim Rommé spielt man mit \emph{110 Karten: sechs} davon sind \emph{Joker}. Zu Beginn eines Spiels erhält jeder Spieler genau \emph{12 Karten}.

In wieviel Prozent aller möglichen Fälle sind darunter \emph{genau zwei} Joker?
$$\frac{\binom{6}{2} \cdot \binom{104}{10}}{\binom{110}{12}}$$

In wieviel Prozent aller möglichen Fälle ist darunter \emph{mindestens ein} Joker?
$$1 - \frac{\binom{104}{12}}{\binom{110}{12}}$$
\end{example2}

\begin{example2}{Geschwister und Geburtsmonat}\\
Sind in mehr als 60\% aller Fälle von vier (nicht gleichaltrigen) Geschwistern mindestens zwei im gleichen Monat geboren?
$$1 - \frac{12 \cdot 11 \cdot 10 \cdot 9}{12^4}$$
\end{example2}

\begin{example2}{Anordnung von Büchern}\\
Auf wie viele Arten lassen sich 10 Bücher in ein Regal reihen?
$$n = 10, \quad k = 10$$
$$\frac{n!}{(n-k)!} = 10!$$
\end{example2}

\begin{example2}{Glühbirnen auswählen}\\
Von \emph{100 Glühbirnen} sind genau \emph{drei defekt}. Es werden nun \emph{6 Glühbirnen} zufällig ausgewählt.

Wie viele Möglichkeiten gibt es, wenn sich \emph{mindestens eine defekte} Glühbirne in der Auswahl befinden soll?
$$\binom{100}{6} - \binom{97}{6} = 203'880'032$$

Mit wie viel Prozent Chancen ist bei einer Auswahl von 6 Glühbirnen \emph{keine defekt}?
$$\frac{\binom{97}{6}}{\binom{100}{6}}$$
\end{example2}

\begin{example2}{Buchstabenkombinationen}\\
Wie viele Worte lassen sich aus den Buchstaben des Wortes ABRAKADABRA bilden? (Nur Worte in denen alle Buchstaben vorkommen!)

$A = 5x, \quad B = 2x, \quad R = 2x, \quad D = 1x, \quad K = 1x$
$$\binom{11}{5} \cdot \binom{6}{2} \cdot \binom{4}{2} \cdot \binom{2}{1} \cdot \binom{1}{1} = 83160$$
\end{example2}

\begin{example2}{Kartenspiel: Rommé}
\textbf{Aufgabe:} Beim Rommé spielt man mit 110 Karten, davon sind 6 Joker. Jeder Spieler erhält 12 Karten.

\textbf{Teil 1:} Berechne die Wahrscheinlichkeit für genau zwei Joker.
\begin{itemize}
\item \textbf{Ergebnisraum:} Alle möglichen 12-Karten-Hände: $|\Omega| = \binom{110}{12}$
\item \textbf{Günstige Ereignisse:}
    \begin{itemize}
    \item 2 Joker aus 6: $\binom{6}{2}$
    \item 10 Nicht-Joker aus 104: $\binom{104}{10}$
    \end{itemize}
\item \textbf{Berechnung:} $P(\text{2 Joker}) = \frac{\binom{6}{2} \cdot \binom{104}{10}}{\binom{110}{12}}$
\end{itemize}

\textbf{Teil 2:} Berechne die Wahrscheinlichkeit für mindestens einen Joker.
\begin{itemize}
\item \textbf{Strategie:} Berechnung über Gegenereignis (kein Joker)
\item \textbf{Berechnung:} $P(\text{mind. 1 Joker}) = 1 - \frac{\binom{104}{12}}{\binom{110}{12}}$
\end{itemize}
\end{example2}

\begin{example2}{Glühbirnen-Problem}
\textbf{Aufgabe:} Von 100 Glühbirnen sind 3 defekt. Es werden 6 zufällig ausgewählt.

\textbf{Teil 1:} Anzahl Möglichkeiten mit mindestens einer defekten Glühbirne.
\begin{itemize}
\item \textbf{Gesamtmöglichkeiten:} $\binom{100}{6}$
\item \textbf{Gegenereignis:} Keine defekte = $\binom{97}{6}$
\item \textbf{Lösung:} $\binom{100}{6} - \binom{97}{6} = 203'880'032$
\end{itemize}

\textbf{Teil 2:} Wahrscheinlichkeit für keine defekte Glühbirne.
$$P(\text{keine defekt}) = \frac{\binom{97}{6}}{\binom{100}{6}}$$
\end{example2}
