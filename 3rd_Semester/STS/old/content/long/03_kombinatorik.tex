% Stochastik und Statistik Summary
% Based on STS_CheatSheet_Islikpas.pdf

\section{Kombinatorik}

\subsection{Grundlegende Methoden des Abzählens}

\begin{KR}{Entscheidungsweg für kombinatorische Probleme}\\
1. \textbf{Bestimme die relevanten Parameter}
   \begin{itemize}
   \item $n$: Wie viele Objekte gibt es insgesamt?
   \item $k$: Wie viele Objekte sollen ausgewählt werden?
   \end{itemize}

2. \textbf{Prüfe die Reihenfolge}
   \begin{itemize}
   \item Spielt die Reihenfolge eine Rolle? $\rightarrow$ Variation
   \item Ist nur die Auswahl wichtig? $\rightarrow$ Kombination
   \end{itemize}

3. \textbf{Prüfe Wiederholungen}
   \begin{itemize}
   \item Können Objekte mehrfach gewählt werden? $\rightarrow$ Mit Wiederholung
   \item Darf jedes Objekt nur einmal vorkommen? $\rightarrow$ Ohne Wiederholung
   \end{itemize}

4. \textbf{Wähle die passende Formel}
   \begin{itemize}
   \item Variation mit Wiederholung: $n^k$
   \item Variation ohne Wiederholung: $\frac{n!}{(n-k)!}$
   \item Kombination mit Wiederholung: $\binom{n+k-1}{k}$
   \item Kombination ohne Wiederholung: $\binom{n}{k}$
   \end{itemize}
\end{KR}

\subsubsection{Systematik}
\begin{concept}{Grundbegriffe der Kombinatorik}
\begin{center}
\resizebox{\columnwidth}{!}{
\begin{tabular}{|c|c|c|c|}
\hline
\multicolumn{2}{|c|}{Variation (mit Reihenfolge)} & \multicolumn{2}{c|}{Kombination (ohne Reihenfolge)} \\
\hline
Mit Wiederholung & Ohne Wiederholung & Mit Wiederholung & Ohne Wiederholung \\
\hline
$n^{k}$ & $\frac{n!}{(n-k)!}$ & $\binom{n+k-1}{k}$ & $\binom{n}{k}$ \\
\hline
Zahlenschloss & Schwimmwettkampf & Zahnarzt & Lotto \\
\hline
\end{tabular}
}
\end{center}
\end{concept}



\begin{formula}{Systematik der Kombinatorik}
\begin{itemize}
\item \textbf{Variation mit Wiederholung}: $n^k$
\item \textbf{Variation ohne Wiederholung}: $\frac{n!}{(n-k)!}$
\item \textbf{Kombination mit Wiederholung}: $\binom{n+k-1}{k}$
\item \textbf{Kombination ohne Wiederholung}: $\binom{n}{k}$
\end{itemize}
\end{formula}

\subsection{Binomialkoeffizienten}

\begin{definition}{Fakultät}
Die Fakultät einer natürlichen Zahl $n$ ist definiert als das Produkt aller positiven ganzen Zahlen bis zu dieser Zahl:
$$
n!=1 \cdot 2 \cdot \ldots \cdot n=\prod_{k=1}^{n} k
$$
mit $0! = 1$ als Definitionsvereinbarung

\textbf{Parameter:}
\begin{itemize}
    \item $n$ = Die positive ganze Zahl, für die die Fakultät berechnet wird
    \item $k$ = Laufvariable in der Produktnotation
    \item $\prod$ = Produkt aller Terme von $k=1$ bis $n$
\end{itemize}
\end{definition}

\begin{KR}{Berechnung von Fakultäten}
1. Prüfe Spezialfälle:
   \begin{itemize}
   \item $0! = 1$ (Definition)
   \item $1! = 1$
   \end{itemize}
2. Für $n > 1$:
   \begin{itemize}
   \item Schreibe alle Zahlen von 1 bis $n$ auf
   \item Multipliziere der Reihe nach
   \item Alternative: Nutze rekursive Definition $n! = n \cdot (n-1)!$
   \end{itemize}
\end{KR}

\begin{definition}{Binomialkoeffizient}
Der Binomialkoeffizient $\binom{n}{k}$ gibt an, wie viele Möglichkeiten es gibt, $k$ Objekte aus einer Gesamtheit von $n$ Objekten auszuwählen:
$$
\binom{n}{k}=\frac{n!}{(n-k)!\cdot k!}
$$

\textbf{Parameter:}
\begin{itemize}
    \item $n$ = Gesamtanzahl der Objekte in der Menge
    \item $k$ = Anzahl der auszuwählenden Objekte ($0 \leq k \leq n$)
    \item $n!$ = Fakultät von $n$
    \item $(n-k)!$ = Fakultät von $(n-k)$
    \item $k!$ = Fakultät von $k$
\end{itemize}
\end{definition}

\begin{theorem}{Wichtige Eigenschaften:}
   %TODO: add missing properties
\begin{itemize}
    \item $\binom{n}{0} = \binom{n}{n} = 1$
    \item $\binom{n}{k} = \binom{n}{n-k}$ (Symmetrie)
    \item $\binom{n+1}{k} = \binom{n}{k-1} + \binom{n}{k}$ (Pascal'sche Rekursion)
\end{itemize}
\end{theorem}

\begin{KR}{Schritte zur Berechnung von Binomialkoeffizienten}\\
   %TODO: correctly format with itemize
1. \textbf{Prüfe Grundfälle}
   - $\binom{n}{0} = 1$
   - $\binom{n}{n} = 1$
   - $\binom{n}{1} = n$
2. \textbf{Nutze Symmetrie}
   - $\binom{n}{k} = \binom{n}{n-k}$
3. \textbf{Pascal'sches Dreieck}
   - $\binom{n+1}{k} = \binom{n}{k-1} + \binom{n}{k}$
4. \textbf{Direkte Berechnung}
   - $\binom{n}{k} = \frac{n!}{k!(n-k)!}$
\end{KR}















\begin{example2}{Variation mit Wiederholung (Zahlenschloss)}
\textbf{Aufgabe:} Wie viele Möglichkeiten gibt es bei einem Zahlenschloss (0 -- 9) mit 6 Zahlenkränzen?

\textbf{Lösung:}
\begin{itemize}
\item $n = 10$ (Ziffern 0-9)
\item $k = 6$ (Stellen)
\item Reihenfolge wichtig: Ja (123456 $\neq$ 654321)
\item Wiederholungen erlaubt: Ja (11111 ist möglich)
\item Formel: $n^k = 10^6 = 1\,000\,000$ mögliche Kombinationen
\end{itemize}
\end{example2}

\begin{example2}{Variation ohne Wiederholung (Schwimmwettkampf)}
\textbf{Aufgabe:} Bei einem Schwimmwettkampf starten 10 Teilnehmer. Wie viele mögliche Platzierungen der ersten drei Plätze (Podest) gibt es?

\textbf{Lösung:}
\begin{itemize}
\item $n = 10$ (Teilnehmer)
\item $k = 3$ (Podestplätze)
\item Reihenfolge wichtig: Ja (1., 2., 3. Platz unterschiedlich)
\item Wiederholungen erlaubt: Nein (niemand kann mehrere Plätze belegen)
\item Formel: $\frac{n!}{(n-k)!} = \frac{10!}{7!} = 720$ mögliche Platzierungen
\end{itemize}
\end{example2}

\begin{example2}{Kombination mit Wiederholung (Zahnarzt)}
\textbf{Aufgabe:} 3 Spielzeuge werden aus 5 Töpfen gezogen. Jeder Topf ist mit einer (unterschiedlichen) Art von Spielzeug befüllt. Wie viele Möglichkeiten hat das Kind?

\textbf{Lösung:}
\begin{itemize}
\item $n = 5$ (Arten von Spielzeug)
\item $k = 3$ (zu wählende Spielzeuge)
\item Reihenfolge wichtig: Nein (nur Anzahl pro Art relevant)
\item Wiederholungen erlaubt: Ja (mehrere Spielzeuge gleicher Art möglich)
\item Formel: $\binom{n+k-1}{k} = \binom{7}{3} = 35$ Möglichkeiten
\end{itemize}
\end{example2}

\begin{example2}{Kombination ohne Wiederholung (Lotto)}
\textbf{Aufgabe:} Wie gross sind die Chancen beim Lotto 6 aus 49 Zahlen richtig zu ziehen?

\textbf{Lösung:}
\begin{itemize}
\item $n = 49$ (Zahlen insgesamt)
\item $k = 6$ (zu wählende Zahlen)
\item Reihenfolge wichtig: Nein (nur Auswahl relevant)
\item Wiederholungen erlaubt: Nein (jede Zahl nur einmal)
\item Formel: $\binom{49}{6} = 13\,983\,816$ Möglichkeiten
\item Gewinnwahrscheinlichkeit: $\frac{1}{13\,983\,816} \approx 0.0000000715$
\end{itemize}
\end{example2}

\begin{example2}{Variation mit Wiederholung (Zahlenschloss)}\\
Wie viele Möglichkeiten gibt es bei einem Zahlenschloss (0 -- 9) mit 6 Zahlenkränzen?

$$n = 10, \quad k = 6$$
$$n^k = 10^6$$
\end{example2}

\begin{example2}{Variation ohne Wiederholung (Schimmwettkampf)}\\
Bei einem Schwimmwettkampf starten 10 Teilnehmer. Wie viele mögliche Platzierungen der ersten drei Plätze (Podest) gibt es?

$$n = 10, \quad k = 3$$
$$\frac{n!}{(n-k)!} = \frac{10!}{(10-3)!} = \frac{10!}{7!}$$
\end{example2}

\begin{example2}{Kombination mit Wiederholung (Zahnarzt)}\\
3 Spielzeuge werden aus 5 Töpfen gezogen. Jeder Topf ist mit einer (unterschiedlichen) Art von Spielzeug befüllt.

Wie viele Möglichkeiten hat das Kind?

$$n = 5, \quad k = 3$$
$$\binom{n+k-1}{k} = \binom{5+3-1}{3} = \binom{7}{3}$$
\end{example2}

\begin{example2}{Kombination ohne Wiederholung (Lotto)}\\
Wie gross sind die Chancen beim Lotto 6 aus 49 Zahlen richtig zu ziehen?

Jede Zahl ist nur einmal vorhanden und die Zahlen werden nicht zurückgelegt. Die Reihenfolge in der gezogen wird spielt keine Rolle.

$$n = 49, \quad k = 6$$
$$\binom{n}{k} = \binom{49}{6}$$
\end{example2}



\begin{KR}{Lösen komplexer kombinatorischer Probleme}
1. \textbf{Problem zerlegen}
   \begin{itemize}
   \item Teile das Problem in unabhängige Teilprobleme
   \item Identifiziere abhängige Entscheidungen
   \end{itemize}

2. \textbf{Für jedes Teilproblem}
   \begin{itemize}
   \item Bestimme $n$ und $k$
   \item Prüfe Reihenfolge und Wiederholung
   \item Wähle passende Formel
   \end{itemize}

3. \textbf{Kombiniere Teillösungen}
   \begin{itemize}
   \item Unabhängige Ereignisse: Multipliziere
   \item Sich ausschließende Ereignisse: Addiere
   \item Prüfe Überlappungen (Inklusions-Exclusions)
   \end{itemize}
\end{KR}

\begin{example2}{Komplexeres Beispiel: Passwörter}
\textbf{Aufgabe:} Ein Passwort muss bestehen aus:
\begin{itemize}
\item Genau 8 Zeichen
\item Mindestens ein Großbuchstabe (26 mögliche)
\item Mindestens eine Ziffer (10 mögliche)
\item Kleine Buchstaben erlaubt (26 mögliche)
\end{itemize}

\textbf{Lösung:}
1. Gesamtzahl aller möglichen 8-stelligen Passwörter mit den Zeichen:
   \begin{itemize}
   \item $n = 26 + 26 + 10 = 62$ Zeichen
   \item Variation mit Wiederholung: $62^8$
   \end{itemize}

2. Abziehen der ungültigen Kombinationen:
   \begin{itemize}
   \item Ohne Großbuchstaben: $(36)^8$
   \item Ohne Ziffern: $(52)^8$
   \item Ohne beide: $(26)^8$
   \end{itemize}

3. Nach dem Inklusions-Exclusions-Prinzip:
   \[ \text{Gültige Passwörter} = 62^8 - 36^8 - 52^8 + 26^8 \]
\end{example2}



