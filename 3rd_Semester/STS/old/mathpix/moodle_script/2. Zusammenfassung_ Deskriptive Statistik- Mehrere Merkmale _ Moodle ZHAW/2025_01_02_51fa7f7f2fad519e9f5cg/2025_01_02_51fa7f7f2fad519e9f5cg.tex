\documentclass[10pt]{article}
\usepackage[ngerman]{babel}
\usepackage[utf8]{inputenc}
\usepackage[T1]{fontenc}
\usepackage{amsmath}
\usepackage{amsfonts}
\usepackage{amssymb}
\usepackage[version=4]{mhchem}
\usepackage{stmaryrd}

\begin{document}
\section*{Kovarianz und Korrelation}
Generell setzen wir eine bivariate (d.h. 2-merkmalige) Stichprobe $\left(x_{1}, y_{1}\right), \ldots,\left(x_{n}, y_{n}\right)$ der Länge $n$ von metrischen Merkmalen voraus. Für die Berechnung der bivariaten Kennwerte legen wir die folgenden Abkürzungen für arithmetische Mittel fest:

Arithmetische Mittel der x-und y-Merkmale: $\bar{x}=\frac{1}{n} \sum_{i=1}^{n} x_{i}$, und $\bar{y}=\frac{1}{n} \sum_{i=1}^{n} y_{i}$.\\
Arithmetische Mittel der quadrierten x-und y-Merkmale: $\overline{x^{2}}=\frac{1}{n} \sum_{i=1}^{n} x_{i}{ }^{2}$, und $\overline{y^{2}}=\frac{1}{n} \sum_{i=1}^{n} y_{i}{ }^{2}$.\\
Arithmetische Mittel des Produktes der x-und y-Merkmale: $\overline{x y}=\frac{1}{n} \sum_{i=1}^{n} x_{i} y_{i}$.\\
Für Varianz und Standardabweichung benutzen wir die bekannten Abkürzungen und Zusammenhänge:\\
$\tilde{s}_{x}^{2}=\overline{x^{2}}-\bar{x}^{2}, \tilde{s}_{x}=\sqrt{\tilde{s}_{x}^{2}}$.

\section*{Definition}
(Empirische) Kovarianz: $\tilde{s}_{x y}=\frac{1}{n} \sum_{i=1}^{n}\left(x_{i}-\bar{x}\right) \cdot\left(y_{i}-\bar{y}\right)$.\\
(Empirischer) Korrelationskoeffizient nach Pearson: $r_{x y}=\frac{\tilde{s}_{x y}}{\tilde{s}_{x} \cdot \tilde{s}_{y}}$, für $\tilde{s}_{x} \neq 0$ und $\tilde{s}_{y} \neq 0$.

\section*{Wichtige Eigenschaften der Kovarianz und Korrelation}
(1) $\tilde{s}_{x y}$ und $r_{x y}$ messen den linearen Zusammenhang der beiden Merkmale, d.h. wie nahe die Datenpunkte einer Geraden mit der Gleichung $y=m \cdot x+q$ mit $m \neq 0$ kommen.\\
(2) $-1 \leq r_{x y} \leq 1$\\
(3) $r_{x y}=0$ bzw. $\tilde{s}_{x y}=0$ oder nahe bei Null, bedeutet, dass die Punkte $\left(x_{1}, y_{1}\right), \ldots,\left(x_{n}, y_{n}\right)$ gleichmässig um den Schwerpunkt $(\bar{x}, \bar{y})$ verteilt sind.\\
(4) $r_{x y}=1$ bedeutet, dass ein positiver linearer Zusammenhang zwischen den Merkmalen besteht.\\
(5) $r_{x y}=-1$ bedeutet, dass ein negativer linearer Zusammenhang zwischen den Merkmalen besteht.\\
(6) $r_{x y}>0$ : Die Punkte liegen tendenziell um eine Gerade mit positiver Steigung (gleichsinniger linearer Zusammenhang, positive Korrelation).\\
(7) $r_{x y}<0$ : Die Punkte liegen tendenziell um eine Gerade mit negativer Steigung (gleichsinniger linearer Zusammenhang, negative Korrelation).\\
(8) $r_{x y}$ ist nicht robust, d.h. bereits ein Ausreisser in den Daten kann den Wert stark beeinflussen.\\
(9) $\tilde{s}_{x y}=\overline{x y}-\bar{x} \cdot \bar{y}$.\\
(10) $r_{x y}=\frac{\overline{x y}-\bar{x} \cdot \bar{y}}{\sqrt{\overline{x^{2}}-\bar{x}^{2}} \cdot \sqrt{\overline{\overline{y^{2}}-\bar{y}^{2}}}}$.

\section*{Rangkorrelation}
Sei $\left(x_{1}, \ldots, x_{n}\right)$ eine Stichprobe eines metrischen Merkmals.\\
Der Rang $\operatorname{rg}\left(x_{i}\right)$ des Stichprobenwertes $x_{i}$ ist definiert als der Index von $x_{i}$ in der nach der Grösse geordneten Stichprobe, wenn dieser Wert nur einmal auftritt. Tritt der Stichprobenwert $x_{i}$ mehrmals auf (man sagt dann, dass Bindungen in den Daten auftreten), so ist $r g\left(x_{i}\right)$ das arithmetische Mittel der Indizes aller Stichprobenwerte $x_{i}$ in der geordneten Stichprobe. Für die Stichprobe $(6,3,4,3,3,2,6)$ erhält man die nach der Grösse geordnete Stichprobe $(2,3,3,3,4,6,6)$ und die Ränge $\operatorname{rg}(2)=1, r g(3)=\frac{1}{3}(2+3+4)=3, r g(4)=5$, $r g(6)=\frac{1}{2}(6+7)=6.5$.

\section*{Definition}
Wir setzen eine bivariate Stichprobe $\left(x_{1}, y_{1}\right), \ldots,\left(x_{n}, y_{n}\right)$ der Länge $n$ von metrischen Merkmalen voraus. Dazu bilden wir die zugehörigen Rangfolgen:

Die Folge der Rangpaare: $r g(x y)=\left(\left(r g\left(x_{1}\right), r g\left(y_{1}\right)\right), \ldots,\left(r g\left(x_{n}\right), r g\left(y_{n}\right)\right)\right.$\\
Die Folge der Ränge der x-Werte: $r g(x)=\left(r g\left(x_{1}\right), \ldots, r g\left(x_{n}\right)\right)$\\
Die Folge der Ränge der y-Werte: $r g(y)=\left(r g\left(y_{1}\right), \ldots, r g\left(y_{n}\right)\right.$.\\
Der (empirische) Korrelationskoeffizient nach Spearman (Rangkorrelationskoeffizient) ist definiert als der Pearson Korrelationskoeffizient der Rangfolgen: $r_{S p}=r_{r g(x y)}$.

\section*{Wichtige Eigenschaften der Rangkorrelation}
(1) $\tilde{s}_{r g(x y)}$ und $r_{S p}$ messen den monotonen Zusammenhang der beiden Merkmale, d.h. wie nahe die Datenpunkte einer streng monotonen Funktion kommen. In anderen Worten, es wird gemessen wie gut die Rangordnungen in den $x$ und $y$ Werten sich entsprechen.\\
(2) $-1 \leq r_{S p} \leq 1$\\
(3) $r_{S p}=0$ bzw. $\tilde{s}_{r g(x y)}=0$ bedeutet, dass die Punkte $\left(x_{1}, y_{1}\right), \ldots,\left(x_{n}, y_{n}\right)$ gleichmässig um den Schwerpunkt der Ränge $\overline{(r g(x)}, \overline{r g(y)})=\left(\frac{n+1}{2}, \frac{n+1}{2}\right)$ verteilt sind.\\
(4) $r_{S p}=1$ bedeutet, dass ein streng monoton wachsender funktionaler Zusammenhang zwischen den Merkmalen besteht.\\
(5) $r_{S p}=-1$ bedeutet, dass ein streng monoton fallender funktionaler Zusammenhang zwischen den Merkmalen besteht.\\
(6) $r_{S p}$ ist robust, d.h. Ausreisser in den Datenpunkten beeinflussen den Wert nicht.\\
(7) $\tilde{s}_{r g(x y)}=\overline{r g(x y)}-\overline{r g(x)} \cdot \overline{r g(y)}=\overline{r g(x y)}-\frac{(n+1)^{2}}{4}$.\\
(8) $r_{S p}=\frac{\overline{r g(x y)}-\overline{r g(x)} \cdot \overline{r g(y)}}{\sqrt{{\overline{r g(x)^{2}}}^{2}-\overline{r g(x)}} \cdot \sqrt{\overline{{\overline{r g(y)^{2}}}^{2}-\overline{r g(y)}}}{ }^{2}}=\frac{\overline{r g(x y)}-\frac{(n+1)^{2}}{4}}{\sqrt{\overline{\overline{r g(x)^{2}}}-\frac{(n+1)^{2}}{4}} \cdot \sqrt{\overline{\overline{r g(y)}^{2}}-\frac{(n+1)^{2}}{4}}}$.\\
(9) Falls jeder Stichprobenwert der $x$ und der $y$-Werte nur einmal vorkommt, in anderen Worten, falls die Ränge in den $x$ Werte und auch in den $y$-Werte alle verschieden sind (also keine Bindungen auftreten), gibt es eine einfache Berechnungsformel: $\quad r_{S p}=1-\frac{6 \cdot \sum_{i=1}^{n} d_{i}{ }^{2}}{n \cdot\left(n^{2}-1\right)}$ mit $d_{i}=r g\left(y_{i}\right)-r g\left(x_{i}\right)$.

Zuletzt geändert: Montag, 12. September 2022, 14:37\\
41. Zusammenfassung: Deskriptive Statistik

\section*{Direkt zu:}
\begin{enumerate}
  \setcounter{enumi}{2}
  \item Zusammenfassung: Kombinatorik
\end{enumerate}

\end{document}