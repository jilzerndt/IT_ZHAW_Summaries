\documentclass[10pt]{article}
\usepackage[ngerman]{babel}
\usepackage[utf8]{inputenc}
\usepackage[T1]{fontenc}
\usepackage{amsmath}
\usepackage{amsfonts}
\usepackage{amssymb}
\usepackage[version=4]{mhchem}
\usepackage{stmaryrd}
\usepackage{bbold}

\begin{document}
\section*{Vier elementare Abzählprobleme}
Bei allen hier betrachteten elementaren Abzählproblemen geht es um eine Auswahl von k Objekten aus einer Gesamtheit von n Objekten. Man teilt die Probleme nach zwei Kriterien ein: Ob in der Auswahl die Reihenfolge der Objekte eine Rolle spielt und ob Wiederholung der Objekte in der Auswahl möglich ist. Eine Auswahl von k Objekten aus $n$ Objekten bei der die Reihenfolge eine Rolle spielt, heisst Variation von $\mathbf{k}$ Objekten aus $\mathbf{n}$. Eine Auswahl von $k$ Objekten aus $n$ Objekten bei der die Reihenfolge keine Rolle spielt, nennt man Kombination von $\mathbf{k}$ Objekten aus $\mathbf{n}$ Objekten.

\begin{itemize}
  \item Variation von k aus n Objekten mit Wiederholung.
  \item Variation von $k$ aus $n$ Objekten ohne Wiederholung.
  \item Kombination von $k$ aus $n$ Objekten mit Wiederholung.
  \item Kombination von $k$ aus $n$ Objekten ohne Wiederholung.
\end{itemize}

\begin{center}
\begin{tabular}{|l|l|l|l|l|}
\hline
\multicolumn{3}{|c|}{Auswahlen von k Objekten aus einer Gesamtheit von n Objekten} &  \\
\hline
Variation (=mit Reihenfolge) & Kombination (=ohne Reihenfolge) &  &  \\
\hline
Mit Wiederholung & Ohne Wiederholung & Mit Wiederholung & Ohne Wiederholung \\
\hline
\begin{tabular}{l}
Zahlenschloss (1) \\
Bitproblem (4) \\
Buchstabenproblem \\
(7b) \\
\end{tabular} & \begin{tabular}{l}
Schwimmwettkampf (2) \\
Buchstabenproblem \\
(7a) \\
Napoleon (9) \\
\end{tabular} & \begin{tabular}{l}
Zahnarztproblem (5) \\
Tellschiessen (8) \\
\end{tabular} & \begin{tabular}{l}
Lotto (3) \\
Fussballmannschaft (6a) \\
Teilmengenproblem (11a) \\
\end{tabular} \\
\hline
$n^{k}$ & $\frac{n!}{(n-k)!}$ & $\binom{n+k-1}{k}$ & $\frac{n!}{(n-k)!k!}=\binom{n}{k}$ \\
\hline
\end{tabular}
\end{center}

\section*{Binomialkoeffizienten}
Die Fakultät $n$ ! ist für eine natürliche Zahl $n$ rekursiv definiert. Für den Startwert 0 als $0!=1$ und für jeden Nachfolger $n \geq 1$ definiert man $n!=n \cdot(n-1)$ !.

Der Binomialkoeffizient $\binom{n}{k}$ ist für natürliche Zahlen $0 \leq k \leq n$, mit Hilfe der Fakultät definiert als

$$
\binom{n}{k}=\frac{n!}{(n-k)!\cdot k!}
$$

Dahinter steckt die anschauliche Bedeutung des Binomialkoeffizienten als die Anzahl aller Auswahlen (ohne Reihenfolge!) von $k$ Objekten aus einer Gesamtheit von $n$ unterscheidbaren Objekten.

Allgemein gilt die folgende Konstruktionsformel der Binomialkoeffizienten:\\
$\binom{n+1}{k+1}=\binom{n}{k}+\binom{n}{k+1}$\\
Mit Hilfe dieser rekursiven Vorschrift können die Binomialkoeffizienten bestimmt werden, ohne die nötigen Fakultäten und Quotienten explizit ausrechnen zu müssen. Dieses Berechnungsverfahren für Binomialkoeffizienten ist als das Pascal'sche Dreieck bekannt.

\begin{center}
\begin{tabular}{|c|c|c|c|c|c|c|c|c|c|c|}
\hline
$n$ &  &  &  &  &  & $k=0$ &  &  &  &  \\
\hline
0 &  &  &  &  & $\binom{0}{0}=1$ &  & $k=1$ &  &  &  \\
\hline
1 &  &  &  & $\binom{1}{0}=1$ & + & $\binom{1}{1}=1$ &  & $k=2$ &  &  \\
\hline
2 &  &  & $\binom{2}{0}=1$ & 71 & $\binom{2}{1}=2$ & 1 & $\binom{2}{2}=1$ &  & $k=3$ &  \\
\hline
3 &  & $\binom{3}{0}=1$ & + & $\binom{3}{1}=3$ & +1 & $\binom{3}{2}=3$ & $\frac{1}{1}$ & $\binom{3}{3}=1$ &  & $k=4$ \\
\hline
4 & $\binom{4}{0}=1$ & + & $\binom{4}{1}=4$ & + & $\binom{4}{2}=6$ & +1 & $\binom{4}{3}=4$ & + & $\binom{4}{4}=1$ &  \\
\hline
$\ldots$ &  & $\ldots$ &  & $\ldots$ &  &  &  & $\ldots$ &  &  \\
\hline
\end{tabular}
\end{center}

Die absteigenden Diagonalen von rechts nach links gehören jeweils zum gleichen $k$ Wert. Sie sind in der gleichen Farbe gehalten. Die Zeilen gehören jeweils zu einem bestimmten $n$ Wert. Man beachte, wie die Einträge jeder Zeile aus den Werten der jeweils vorhergehenden Zeile nach obiger Formel berechnet werden.\\
Wichtige Eigenschaften der Binomialkoeffizienten:

\begin{itemize}
  \item \hspace{0pt} [Leere Menge]
\end{itemize}

$$
\binom{n}{0}=1
$$

\begin{itemize}
  \item \hspace{0pt} [Symmetrie]
\end{itemize}

$$
\binom{n}{k}=\binom{n}{n-k}
$$

\begin{itemize}
  \item \hspace{0pt} [Pascal'sches Dreieck - Rekursion]
\end{itemize}

$$
\binom{n+1}{k+1}=\binom{n}{k}+\binom{n}{k+1}
$$

\begin{itemize}
  \item \hspace{0pt} [Binomischer Lehrsatz]\\
$(a+b)^{n}=\binom{n}{0} a^{0} b^{n}+\binom{n}{1} a b^{n-1}+\binom{n}{2} a^{2} b^{n-2}+\ldots+\binom{n}{n-1} a^{n-1} b+\binom{n}{n} a^{n} b^{0}=\sum_{k=0}^{n}\binom{n}{k} a^{k} b^{n-k}$\\
für komplexe Zahlen $a, b \in \mathbb{C}$.
  \item \hspace{0pt} [Summe]
\end{itemize}

$$
\sum_{k=0}^{n}\binom{n}{k}=\binom{n}{0}+\binom{n}{1}+\binom{n}{2}+\ldots+\binom{n}{n-1}+\binom{n}{n}=2^{n}
$$

Zuletzt geändert: Donnerstag, 10. September 2020, 11:56\\
2. Zusammenfassung: Deskriptive Statistik- Mehrere Merkmale

Direkt zu:\\
4. Zusammenfassung: Elementare Wahrscheinlichkeitsrechnung


\end{document}