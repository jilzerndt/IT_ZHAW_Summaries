\section{Use Case Realization}

\begin{concept}{Use Case Realization}\\
Die Umsetzung von Use Cases erfolgt durch:
\begin{itemize}
    \item Detaillierte Szenarien aus den Use Cases
    \item Systemantworten müssen realisiert werden
    \item UI statt System im SSD
    \item Systemoperationen sind die zu implementierenden Elemente
\end{itemize}
\end{concept}

\begin{theorem}{Use Case Realization Ziele}
\begin{itemize}
    \item Umsetzung der fachlichen Anforderungen in Code
    \item Einhaltung der Architekturvorgaben
    \item Implementierung der GRASP-Prinzipien
    \item Erstellung wartbaren und testbaren Codes
    \item Dokumentation der Design-Entscheidungen
\end{itemize}
\end{theorem}

\begin{definition}{Verantwortlichkeiten (Responsibilities)}
Im objektorientierten Design unterscheiden wir zwei Arten von Verantwortlichkeiten:

\textbf{Doing-Verantwortlichkeiten:}
\begin{itemize}
    \item Selbst etwas tun
    \item Aktionen anderer Objekte anstoßen
    \item Aktivitäten anderer Objekte kontrollieren
\end{itemize}

\textbf{Knowing-Verantwortlichkeiten:}
\begin{itemize}
    \item Private eingekapselte Daten kennen
    \item Verwandte Objekte kennen
    \item Dinge berechnen/ableiten können
\end{itemize}
\end{definition}

\begin{concept}{Architekturbezogene Aspekte}
Bei der Use Case Realization müssen folgende architektonische Aspekte beachtet werden:

\textbf{Schichtenarchitektur:}
\begin{itemize}
    \item Presentation Layer (UI)
    \item Application Layer (Use Cases)
    \item Domain Layer (Business Logic)
    \item Infrastructure Layer (Persistence, External Services)
\end{itemize}

\textbf{Abhängigkeitsregeln:}
\begin{itemize}
    \item Abhängigkeiten nur nach unten
    \item Interfaces für externe Services
    \item Dependency Injection für lose Kopplung
\end{itemize}

\textbf{Cross-Cutting Concerns:}
\begin{itemize}
    \item Logging
    \item Security
    \item Transaction Management
    \item Exception Handling
\end{itemize}
\end{concept}



\begin{KR}{Vorgehen bei der Use Case Realization}\\
\textbf{1. Vorbereitung:}
\begin{itemize}
    \item Use Case auswählen und SSD ableiten
    \item Systemoperation identifizieren
    \item Operation Contract erstellen/prüfen
\end{itemize}

\textbf{2. Analyse:}
\begin{itemize}
    \item Aktuellen Code/Dokumentation analysieren
    \item DCD überprüfen/aktualisieren
    \item Vergleich mit Domänenmodell
    \item Neue Klassen gemäß Domänenmodell erstellen
\end{itemize}

\textbf{3. Realisierung:}
\begin{itemize}
    \item Controller Klasse bestimmen
    \item Zu verändernde Klassen festlegen
    \item Weg zu diesen Klassen festlegen:
    \begin{itemize}
        \item Parameter für Wege definieren
        \item Klassen bei Bedarf erstellen
        \item Verantwortlichkeiten zuweisen
        \item Verschiedene Varianten evaluieren
    \end{itemize}
    \item Veränderungen implementieren
    \item Review durchführen
\end{itemize}
\end{KR}

\begin{corollary}{Use Case Realization Dokumentation}\\
\textbf{1. Analysephase}
\begin{itemize}
    \item Use Case und Systemoperationen dokumentieren
    \item Domänenmodell-Ausschnitt zeigen
    \item Relevante Anforderungen auflisten
\end{itemize}

\textbf{2. Design}
\begin{itemize}
    \item Design Class Diagram erstellen
    \item Sequenzdiagramme für komplexe Abläufe
    \item GRASP-Prinzipien begründen
\end{itemize}

\textbf{3. Implementation}
\begin{itemize}
    \item Code-Struktur dokumentieren
    \item Wichtige Algorithmen erläutern
    \item Test-Strategie beschreiben
\end{itemize}
\end{corollary}


\columnbreak

\subsubsection{Diagramme und Modelle in der Use Case Realization}

\begin{definition}{UML im Implementierungsprozess}\\
UML dient als:
\begin{itemize}
    \item Zwischenschritt bei wenig Erfahrung
    \item Kompakter Ersatz für Programmiercode
    \item Kommunikationsmittel (auch für Nicht-Techniker)
\end{itemize}
\end{definition}

\begin{concept}{System Sequence Diagrams (SSD)}
    %todo ADD
\end{concept}

\begin{KR}{System Sequence Diagrams (SSD) erstellen}
    %todo ADD
\end{KR}


\begin{example2}{Interaction Diagrams in der Use Case Realization}\\
    %todo ADD a better one
\textbf{Sequenzdiagramm für enterItem():}

\begin{lstlisting}[language=Java, style=basesmol]
:Register -> :ProductCatalog: getDescription(itemId)
:ProductCatalog --> :Register: desc
:Register -> currentSale: makeLineItem(desc, quantity)
currentSale -> :SalesLineItem: create(desc, quantity)
currentSale -> lineItems: add(sl)
\end{lstlisting}

\textbf{Begründung der Interaktionen:}
\begin{itemize}
    \item Register als Controller empfängt Systemoperation
    \item ProductCatalog als Information Expert für Produkte
    \item Sale als Creator für SalesLineItem
    \item Sale als Container verwaltet seine LineItems
\end{itemize}
\end{example2}

\begin{concept}{Design Class Diagram (DCD)}
    %todo ADD
    
\end{concept}

\begin{KR}{Design Class Diagram (DCD) erstellen}
\textbf{1. Klassen identifizieren}
\begin{itemize}
    \item Aus Domänenmodell übernehmen
    \item Technische Klassen ergänzen
    \item Controller bestimmen
\end{itemize}

\textbf{2. Attribute definieren}
\begin{itemize}
    \item Datentypen festlegen
    \item Sichtbarkeiten bestimmen
    \item Validierungen vorsehen
\end{itemize}

\textbf{3. Methoden hinzufügen}
\begin{itemize}
    \item Systemoperationen verteilen
    \item GRASP-Prinzipien anwenden
    \item Signaturen definieren
\end{itemize}

\textbf{4. Beziehungen modellieren}
\begin{itemize}
    \item Assoziationen aus Domänenmodell
    \item Navigierbarkeit festlegen
    \item Abhängigkeiten minimieren
\end{itemize}
\end{KR}

\begin{example2}{Design Class Diagram}
    %todo ADD
\end{example2}

\columnbreak

\subsubsection{GRASP in der Use Case Realization}

\begin{concept}{GRASP Prinzipien in der Use Case Realization}
GRASP (General Responsibility Assignment Software Patterns) dient als Leitfaden für die Zuweisung von Verantwortlichkeiten:

\textbf{Zentrale Prinzipien:}
\begin{itemize}
    \item Information Expert: Verantwortlichkeit dort, wo die Information ist
    \item Creator: Objekte werden von eng verbundenen anderen Objekten erstellt
    \item Controller: Erste Anlaufstelle für Systemoperationen
    \item Low Coupling: Minimale Abhängigkeiten zwischen Klassen
    \item High Cohesion: Fokussierte Verantwortlichkeiten pro Klasse
\end{itemize}
\end{concept}

\begin{KR}{GRASP Prinzipien in der Use Case Realization}
    %todo ADD
\end{KR}

\begin{example2}{GRASP-basierte Implementation}\\
\textbf{Szenario:} Implementierung einer Bestellverwaltung mit GRASP-Prinzipien

\textbf{Information Expert:}
\begin{lstlisting}[language=Java, style=basesmol]
public class Order {
    private List<OrderItem> items = new ArrayList<>();
    
    public BigDecimal calculateTotal() {
        return items.stream()
                   .map(OrderItem::getSubtotal)
                   .reduce(BigDecimal.ZERO, BigDecimal::add);
    }
}
\end{lstlisting}

\textbf{Creator:}
\begin{lstlisting}[language=Java, style=basesmol]
public class Order {
    public OrderItem createOrderItem(Product product, int quantity) {
        OrderItem item = new OrderItem(product, quantity);
        items.add(item);
        return item;
    }
}
\end{lstlisting}

\textbf{Controller:}
\begin{lstlisting}[language=Java, style=basesmol]
public class OrderController {
    private OrderService orderService;
    
    public OrderDTO createOrder(String customerId) {
        Order order = orderService.initializeOrder(customerId);
        return OrderMapper.toDTO(order);
    }
}
\end{lstlisting}
\end{example2}

\subsubsection{Implementierung und Prüfung}

\begin{KR}{Typische Prüfungsaufgaben}\\
\textbf{1. Use Case Realization dokumentieren}
\begin{itemize}
    \item System Sequence Diagram erstellen
    \item Operation Contracts definieren
    \item Design Class Diagram zeichnen
    \item GRASP Prinzipien begründen
    \item Sequenzdiagramm für wichtige Operationen
\end{itemize}

\textbf{2. Implementation analysieren}
\begin{itemize}
    \item GRASP Verletzungen identifizieren
    \item Verbesserungen vorschlagen
    \item Alternative Designs diskutieren
\end{itemize}

\textbf{3. Architektur evaluieren}
\begin{itemize}
    \item Schichtenarchitektur prüfen
    \item Kopplung analysieren
    \item Kohäsion bewerten
\end{itemize}
\end{KR}




\begin{example2}{Vollständige Use Case Realization}\\
    %todo better example?
\textbf{Use Case:} Bestellung aufgeben

\textbf{1. Systemoperationen:}
\begin{itemize}
    \item createOrder()
    \item addItem(productId, quantity)
    \item removeItem(itemId)
    \item submitOrder()
\end{itemize}

\textbf{2. Design-Entscheidungen:}
\begin{itemize}
    \item OrderController als Fassade
    \item Order aggregiert OrderItems
    \item OrderService für Geschäftslogik
    \item Repository für Persistenz
\end{itemize}

\textbf{3. GRASP-Anwendung:}
\begin{itemize}
    \item Information Expert:
    \begin{itemize}
        \item Order berechnet Gesamtsumme
        \item OrderItem verwaltet Produktdaten
    \end{itemize}
    \item Creator:
    \begin{itemize}
        \item Order erstellt OrderItems
        \item OrderService erstellt Orders
    \end{itemize}
    \item Low Coupling:
    \begin{itemize}
        \item Repository-Interface für Persistenz
        \item Service-Interface für Geschäftslogik
    \end{itemize}
\end{itemize}

\textbf{4. Implementierung:}
\begin{lstlisting}[language=Java, style=basesmol]
public class OrderController {
    private OrderService orderService;
    private Order currentOrder;
    
    public void createOrder() {
        currentOrder = orderService.createOrder();
    }
    
    public void addItem(String productId, int quantity) {
        currentOrder.addItem(productId, quantity);
    }
    
    public void submitOrder() {
        orderService.submitOrder(currentOrder);
    }
}
\end{lstlisting}
\end{example2}



\begin{KR}{Typische Implementierungsfehler vermeiden}
\begin{itemize}
    \item \textbf{Architekturverletzungen:}
    \begin{itemize}
        \item Schichtentrennung beachten
        \item Abhängigkeiten richtig setzen
    \end{itemize}
    
    \item \textbf{GRASP-Verletzungen:}
    \begin{itemize}
        \item Information Expert beachten
        \item Creator Pattern richtig anwenden
        \item High Cohesion erhalten
    \end{itemize}
    
    \item \textbf{Testbarkeit:}
    \begin{itemize}
        \item Klassen isoliert testbar halten
        \item Abhängigkeiten mockbar gestalten
    \end{itemize}
\end{itemize}
\end{KR}

\begin{example2}{Typische Implementierungsfehler}\\
    %todo better example?
\textbf{1. Verletzung von GRASP-Prinzipien}
\begin{lstlisting}[language=Java, style=basesmol]
// Falsch: Information Expert verletzt
public class Register {
    public BigDecimal calculateTotal(Sale sale) {
        // Register berechnet Total statt Sale
        return sale.getItems().stream()
                  .map(item -> item.getPrice())
                  .reduce(BigDecimal.ZERO, BigDecimal::add);
    }
}
// Richtig: Sale ist Information Expert
public class Sale {
    public BigDecimal getTotal() {
        return items.stream()
                   .map(item -> item.getSubtotal())
                   .reduce(BigDecimal.ZERO, BigDecimal::add);
    }
}
\end{lstlisting}

\textbf{2. Architekturverletzungen}
\begin{lstlisting}[language=Java, style=basesmol]
// Falsch: Domain-Objekt mit UI-Abhaengigkeit
public class Sale {
    private JFrame frame;
    public void complete() {
        // Domain-Logik vermischt mit UI
        frame.showMessage("Sale completed");
    }
}
// Richtig: Trennung der Schichten
public class Sale {
    public void complete() {
        // Reine Domain-Logik
        this.status = SaleStatus.COMPLETED;
        this.completionTime = LocalDateTime.now();
    }
}
\end{lstlisting}
\end{example2}

\columnbreak

\subsubsection{Testing und Refactoring}

\begin{KR}{Implementierung prüfen}\\
\textbf{1. Funktionale Prüfung}
\begin{itemize}
    \item Use Case Szenarien durchspielen
    \item Randfälle testen
    \item Fehlersituationen prüfen
\end{itemize}

\textbf{2. Strukturelle Prüfung}
\begin{itemize}
    \item Architekturkonformität
    \item GRASP-Prinzipien
    \item Clean Code Regeln
\end{itemize}

\textbf{3. Qualitätsprüfung}
\begin{itemize}
    \item Testabdeckung
    \item Wartbarkeit
    \item Performance
\end{itemize}
\end{KR}







\begin{KR}{Refactoring in Use Case Realization}\\
\textbf{1. Code Smells erkennen}
\begin{itemize}
    \item Duplizierter Code
    \item Lange Methoden
    \item Große Klassen
    \item Feature Envy
    \item Data Class
\end{itemize}

\textbf{2. Refactoring durchführen}
\begin{itemize}
    \item Extract Method
    \item Move Method
    \item Extract Class
    \item Introduce Parameter Object
    \item Replace Conditional with Polymorphism
\end{itemize}

\textbf{3. Beispiel Refactoring:}
\begin{lstlisting}[language=Java, style=basesmol]
// Vor Refactoring
public class Sale {
    public void complete() {
        BigDecimal total = BigDecimal.ZERO;
        for(SaleLineItem item : items) {
            total = total.add(item.getPrice()
                   .multiply(new BigDecimal(item.getQuantity())));
        }
        this.total = total;
        this.status = "COMPLETED";
        // ... weitere 20 Zeilen Code
    }
}
// Nach Refactoring
public class Sale {
    public void complete() {
        calculateTotal();
        updateStatus();
        generateReceipt();
        notifyInventory();
    }
    private void calculateTotal() {
        this.total = items.stream()
            .map(SaleLineItem::getSubtotal)
            .reduce(BigDecimal.ZERO, BigDecimal::add);
    }
    // ... weitere private Methoden
}
\end{lstlisting}
\end{KR}


\begin{KR}{Testing in Use Case Realization}\\
\textbf{1. Unit Tests}
\begin{itemize}
    \item Isolierte Tests für einzelne Klassen
    \item Mocking von Abhängigkeiten
    \item Tests für Standardfälle und Ausnahmen
    \item ATRIP-Prinzipien beachten:
    \begin{itemize}
        \item Automatic: Tests müssen automatisch ausführbar sein
        \item Thorough: Vollständige Testabdeckung wichtiger Funktionen
        \item Repeatable: Tests müssen reproduzierbar sein
        \item Independent: Tests dürfen sich nicht gegenseitig beeinflussen
        \item Professional: Tests müssen wartbar und lesbar sein
    \end{itemize}
\end{itemize}
\end{KR}

\begin{example2}{Vollständiges Use Case Realization Beispiel}\\
\textbf{Use Case:} Warenkorb zum Check-out freigeben

\textbf{1. System Sequence Diagram}
\begin{itemize}
    \item validateCart()
    \item prepareCheckout()
    \item getCheckoutURL()
\end{itemize}

\textbf{2. Design Class Diagram}
\begin{itemize}
    \item CartController
    \item ShoppingCart
    \item CartValidator
    \item CheckoutService
\end{itemize}

\textbf{3. Implementation:}
\begin{lstlisting}[language=Java, style=basesmol]
public class CartController {
    private CartValidator validator;
    private CheckoutService checkoutService;
    
    public CheckoutResult prepareCheckout(String cartId) {
        ShoppingCart cart = findCart(cartId);
        // Validierung (Information Expert)
        ValidationResult result = validator.validate(cart);
        if (!result.isValid()) {
            throw new ValidationException(result.getErrors());
        }
        // Checkout vorbereiten
        String checkoutUrl = checkoutService.initiate(cart);
        return new CheckoutResult(checkoutUrl);
    }
}
\end{lstlisting}

\textbf{4. Tests:}
\begin{lstlisting}[language=Java, style=basesmol]
@Test
public void shouldPrepareCheckoutForValidCart() {
    // Given
    ShoppingCart cart = createValidCart();
    when(validator.validate(cart))
        .thenReturn(ValidationResult.valid());
    when(checkoutService.initiate(cart))
        .thenReturn("https://checkout/123");
    // When
    CheckoutResult result = controller.prepareCheckout(cart.getId());
    // Then
    assertNotNull(result.getCheckoutUrl());
    verify(checkoutService).initiate(cart);
}
\end{lstlisting}
\end{example2}


















