\begin{example}{Extract Method Refactoring}\\
\textbf{Vorher:}
\begin{lstlisting}[language=Java, style=base]
void printOwing() {
    printBanner();
    
    // calculate outstanding
    double outstanding = 0.0;
    for (Order order : orders) {
        outstanding += order.getAmount();
    }
    
    // print details
    System.out.println("name: " + name);
    System.out.println("amount: " + outstanding);
}
\end{lstlisting}

\textbf{Nachher:}
\begin{lstlisting}[language=Java, style=base]
void printOwing() {
    printBanner();
    double outstanding = calculateOutstanding();
    printDetails(outstanding);
}

double calculateOutstanding() {
    double result = 0.0;
    for (Order order : orders) {
        result += order.getAmount();
    }
    return result;
}

void printDetails(double outstanding) {
    System.out.println("name: " + name);
    System.out.println("amount: " + outstanding);
}
\end{lstlisting}
\end{example}

\begin{example}{Unit Test}\\
\textbf{Zu testende Klasse:}
\begin{lstlisting}[language=Java, style=base]
public class Calculator {
    public int add(int a, int b) {
        return a + b;
    }
}
\end{lstlisting}

\textbf{Test:}
\begin{lstlisting}[language=Java, style=base]
@Test
public class CalculatorTest {
    private Calculator calc;
    
    @Before
    public void setup() {
        calc = new Calculator();
    }
    
    @Test
    public void testAdd() {
        assertEquals(4, calc.add(2, 2));
        assertEquals(0, calc.add(-2, 2));
        assertEquals(-4, calc.add(-2, -2));
    }
}
\end{lstlisting}
\end{example}

\begin{example}{BDD Test}\\
\textbf{Feature File:}
\begin{lstlisting}[style=basesmol]
Feature: Calculator Addition
  Scenario: Add two positive numbers
    Given I have a calculator
    When I add 2 and 2
    Then the result should be 4
    
  Scenario: Add positive and negative numbers
    Given I have a calculator
    When I add -2 and 2
    Then the result should be 0
\end{lstlisting}

\textbf{Step Definitions:}
\begin{lstlisting}[language=Java, style=base]
public class CalculatorSteps {
    private Calculator calc;
    private int result;
    
    @Given("I have a calculator")
    public void createCalculator() {
        calc = new Calculator();
    }
    
    @When("I add {int} and {int}")
    public void addNumbers(int a, int b) {
        result = calc.add(a, b);
    }
    
    @Then("the result should be {int}")
    public void checkResult(int expected) {
        assertEquals(expected, result);
    }
}
\end{lstlisting}
\end{example}

\begin{example}{Client-Server Implementation}\\
\textbf{Aufgabe:} Implementieren Sie einen einfachen Echo-Server mit Java.

\textbf{Lösung:}
\begin{lstlisting}[language=Java, style=base]
// Server
public class EchoServer {
    public static void main(String[] args) {
        try (ServerSocket server = new ServerSocket(8080)) {
            while (true) {
                Socket client = server.accept();
                new Thread(() -> handleClient(client)).start();
            }
        }
    }
    
    private static void handleClient(Socket client) {
        try (
            BufferedReader in = new BufferedReader(
                new InputStreamReader(client.getInputStream()));
            PrintWriter out = new PrintWriter(
                client.getOutputStream(), true)
        ) {
            String line;
            while ((line = in.readLine()) != null) {
                out.println("Echo: " + line);
            }
        } catch (IOException e) {
            e.printStackTrace();
        }
    }
}

// Client
public class EchoClient {
    public static void main(String[] args) {
        try (
            Socket socket = new Socket("localhost", 8080);
            PrintWriter out = new PrintWriter(
                socket.getOutputStream(), true);
            BufferedReader in = new BufferedReader(
                new InputStreamReader(socket.getInputStream()))
        ) {
            out.println("Hello Server!");
            System.out.println(in.readLine());
        } catch (IOException e) {
            e.printStackTrace();
        }
    }
}
\end{lstlisting}
\end{example}

\begin{example}{Publish-Subscribe Pattern}\\
\textbf{Aufgabe:} Implementieren Sie ein einfaches Event-System.

\textbf{Lösung:}
\begin{lstlisting}[language=Java, style=base]
public class EventBus {
    private Map<String, List<EventHandler>> handlers = new HashMap<>();
    
    public void subscribe(String event, EventHandler handler) {
        handlers.computeIfAbsent(event, k -> new ArrayList<>())
               .add(handler);
    }
    
    public void publish(String event, String data) {
        if (handlers.containsKey(event)) {
            handlers.get(event)
                   .forEach(handler -> handler.handle(data));
        }
    }
}

interface EventHandler {
    void handle(String data);
}

// Verwendung
EventBus bus = new EventBus();
bus.subscribe("userLogin", data -> 
    System.out.println("User logged in: " + data));
bus.publish("userLogin", "john_doe");
\end{lstlisting}
\end{example}

\begin{example}{JDBC Basisbeispiel}
\begin{lstlisting}[language=Java, style=base]
import java.sql.*;

public class DbTest {
    public static void main(String[] args) 
            throws SQLException {
        // Verbindung aufbauen
        Connection con = DriverManager.getConnection(
            "jdbc:postgresql://test.zhaw.ch/testdb",
            "user", "password");
            
        // Statement erstellen und ausfuehren
        Statement stmt = con.createStatement();
        ResultSet rs = stmt.executeQuery(
            "SELECT * FROM test ORDER BY name");
            
        // Ergebnisse verarbeiten
        while (rs.next()) {
            System.out.println(
                "Name: " + rs.getString("name"));
        }
        
        // Aufraeumen
        rs.close();
        stmt.close();
        con.close();
    }
}
\end{lstlisting}
\end{example}

\begin{example2}{DAO Implementation}
\begin{lstlisting}[language=Java, style=base]
public interface ArticleDAO {
    void insert(Article item);
    void update(Article item);
    void delete(Article item);
    Article findById(int id);
    Collection<Article> findAll();
    Collection<Article> findByName(String name);
}

public class Article {
    private long id;
    private String name;
    private float price;
    
    // Getter/Setter
}

public class JdbcArticleDAO implements ArticleDAO {
    private Connection conn;
    
    public void insert(Article item) {
        PreparedStatement stmt = conn.prepareStatement(
            "INSERT INTO articles (name, price) VALUES (?, ?)");
        stmt.setString(1, item.getName());
        stmt.setFloat(2, item.getPrice());
        stmt.executeUpdate();
    }
    // weitere Implementierungen
}
\end{lstlisting}
\end{example2}

\begin{example}{Parent-Child Beziehung mit JPA}
\begin{lstlisting}[language=Java, style=base]
@Entity
public class Department {
    @Id @GeneratedValue
    private Long id;
    
    private String name;
    
    @OneToMany(mappedBy = "department")
    private List<Employee> employees;
}

@Entity
public class Employee {
    @Id @GeneratedValue
    private Long id;
    
    @ManyToOne
    @JoinColumn(name = "department_id")
    private Department department;
    
    private String name;
    private double salary;
}
\end{lstlisting}
\end{example}

\begin{example}{Spring Data Repository}
\begin{lstlisting}[language=Java, style=base]
@Repository
public interface SaleRepository 
        extends CrudRepository<Sale, String> {
    
    List<Sale> findOrderByDateTime();
    
    List<Sale> findByDateTime(
        final LocalDateTime dateTime);
}

@Service
public class ProcessSaleHandler {
    private final ProductDescriptionRepository catalog;
    private final SaleRepository saleRepository;
    
    @Transactional
    public void endSale() {
        assert(currentSale != null 
            && !currentSale.isComplete());
        this.currentSale.becomeComplete();
        this.saleRepository.save(currentSale);
    }
}
\end{lstlisting}
\end{example}

\begin{example}{Abstract Factory: POS Terminal}
\begin{lstlisting}[language=Java, style=base]
public interface IJavaPOSDevicesFactory {
    CashDrawer getNewCashDrawer();
    CoinDispenser getNewCoinDispenser();
    // weitere Methoden
}

public class IBMJavaPOSDevicesFactory 
        implements IJavaPOSDevicesFactory {
    public CashDrawer getNewCashDrawer() {
        return new com.ibm.pos.jpos.CashDrawer();
    }
    // weitere Implementierungen
}
\end{lstlisting}
\end{example}

\begin{example}{Command: Persistenz}
\begin{lstlisting}[language=Java, style=base]
public interface ICommand {
    void execute();
    void undo();
}

public class DBUpdateCommand implements ICommand {
    private PersistentObject object;
    
    public void execute() {
        // Update in Datenbank
    }
    
    public void undo() {
        // Aenderung rueckgaengig machen
    }
}
\end{lstlisting}
\end{example}

\begin{example}{Template Method: GUI Framework}
\begin{lstlisting}[language=Java, style=base]
public abstract class GUIComponent {
    // Template Method
    public final void update() {
        clearBackground();
        repaint(); // Hook Method
    }
    
    protected abstract void repaint();
}

public class MyButton extends GUIComponent {
    protected void repaint() {
        // Button-spezifische Implementation
    }
}
\end{lstlisting}
\end{example}

\begin{example}{Spring Data Repository}
\begin{lstlisting}[language=Java, style=base]
@Repository
public interface UserRepository 
        extends JpaRepository<User, Long> {
    // Methode wird automatisch implementiert
    List<User> findByLastNameOrderByFirstNameAsc(
        String lastName);
    
    // SQL-Query via Annotation
    @Query("SELECT u FROM User u WHERE u.active = true")
    List<User> findActiveUsers();
}
\end{lstlisting}
\end{example}