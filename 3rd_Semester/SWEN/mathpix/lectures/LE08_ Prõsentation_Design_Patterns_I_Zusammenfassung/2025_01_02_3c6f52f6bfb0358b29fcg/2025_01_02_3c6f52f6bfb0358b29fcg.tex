\documentclass[10pt]{article}
\usepackage[ngerman]{babel}
\usepackage[utf8]{inputenc}
\usepackage[T1]{fontenc}
\usepackage{fvextra, csquotes}
\usepackage{graphicx}
\usepackage[export]{adjustbox}
\graphicspath{ {./images/} }

\title{Bachelor of Science (BSc) in Informatik }

\author{}
\date{}


\begin{document}
\maketitle
\section*{Modul Software-Entwicklung 1 (SWEN1) }
\section*{LE 08 - Entwurf mit Design Pattern I Zusammenfassung}
SWEN1/PM3 Team:\\
R. Ferri (feit), D. Liebhart (lieh), K. Bleisch (bles), G. Wyder (wydg)

\section*{Lernziele LE 08 - Entwurf mit Design Patterns I}
\begin{itemize}
  \item Sie sind in der Lage:
  \item Den allgemeinen Aufbau und Zweck von Design Patterns (Entwurfsmuster) zu erklären.
  \item Den Aufbau und Einsatz der folgenden Design Patterns zu erklären und anzuwenden:
  \item Adaptor
  \item Factory
  \item Singleton
  \item Dependency Injection
  \item Proxy
  \item Chain of Responsibility
\end{itemize}

\section*{1. Einführung in Design Patterns}
\begin{enumerate}
  \setcounter{enumi}{1}
  \item Repetition GRASP
  \item Design Pattern
  \item Wrap-up und Ausblick
\end{enumerate}

\section*{Bewährte Lösungsschablonen für wiederkehrende Entwurfsprobleme }
\begin{displayquote}
Rad nicht neu erfinden\\
Gemeinsame Sprache/Verständnis\\
Best-practices lernen
\end{displayquote}

\section*{Aufbau Design Patterns}
\begin{itemize}
  \item Beschreibungsschema
  \item Name
  \item Beschreibung Problem
  \item Beschreibung Lösung
  \item Hinweise für Anwendung
  \item Beispiele
\end{itemize}

\section*{GRASP und GoF}
\begin{itemize}
  \item GRASP
  \item Grundlegende Prinzipien der Zuweisung von Verantwortlichkeiten, formuliert als Design Patterns
  \item GoF
  \item Buch «Design Patterns», herausgekommen 1995
  \item 23 Patterns, 15 sind gebräuchlich
  \item Erste systematische Publikation von Design Patterns
  \item Können als Spezialisierungen von GRASP interpretiert werden
\end{itemize}

\begin{verbatim}
Design Patterns
Elements of Reusable
Object-Oriented Software
Erich Gamma
\
**)
\end{verbatim}

\begin{center}
\includegraphics[max width=\textwidth]{2025_01_02_3c6f52f6bfb0358b29fcg-06}
\end{center}

\begin{verbatim}
FForword by Grady Booch

\author{
1. Einführung in Design Patterns \\ 2. Repetition GRASP \\ 3. Design Pattern \\ 4. Wrap-up und Ausblick
}

\section*{GRASP Prinzipien und GoF Design Patterns}

School of

Low coupling is a way to achieve protection at a variation point.

Polymorphism is a way to achieve protection at a variation point, and a way to achieve low coupling.

An indirection is a way to achieve low coupling.
The Adapter design pattern is a kind of Indirection and a Pure Fabrication, that uses Polymorphism.

GoF Patterns sind
Spezialfälle der
GRASP Prinzipien
![](https://cdn.mathpix.com/cropped/2025_01_02_3c6f52f6bfb0358b29fcg-08.jpg?height=590&width=1668&top_left_y=1142&top_left_x=1254)

\title{
1. Einführung in Design Patterns
}
2. Repetition GRASP
3. Design Patterns
4. Wrap-up und Ausblick

\section*{Liste der Design Patterns für die Lerneinheit}
- Adapter
- Simple Factory
- Singleton
- Dependency Injection
- Proxy
- Chain of Responsibility

\section*{Adapter: Problem und Lösung}
- Problem
- Eine Klasse soll eingesetzt werden, die aber inkompatibel mit einem bereits definierten domänenspezifischem Interface ist.
- Lösung
- Eine eigene Adapter Klasse wird dazwischengeschaltet.
![](https://cdn.mathpix.com/cropped/2025_01_02_3c6f52f6bfb0358b29fcg-11.jpg?height=1200&width=1362&top_left_y=438&top_left_x=1828)
- Hinweise
- Oft wird mit einem Adapter ein externer Dienst in die eigene Anwendung integriert, insbesondere wenn der Dienst austauschbar sein soll.
- Das Target Interface ist bewusst für die Domänenlogik optimiert, während der Adaptee oft von extern bezogen wird.
- Falls es sich beim Adaptee um einen externen Dienst handelt, kann im Adapter allenfalls die Kommunikation integriert
![](https://cdn.mathpix.com/cropped/2025_01_02_3c6f52f6bfb0358b29fcg-12.jpg?height=1205&width=1370&top_left_y=431&top_left_x=1820) werden.

\section*{Simple Factory: Problem und Lösung}
- Problem
- Das Erzeugen eines neuen Objekts ist aufwändig.
- Lösung
- Eine eigene Klasse für das Erzeugen eines neuen Objekts wird geschrieben.
![](https://cdn.mathpix.com/cropped/2025_01_02_3c6f52f6bfb0358b29fcg-13.jpg?height=765&width=1006&top_left_y=502&top_left_x=2172)

\section*{Simple Factory: Hinweise}
- Hinweise
- Oft ist die Erzeugung des neuen Objekts von irgendeiner Art von Konfiguration abhängig.
- Es ist auch möglich, die create() Methode mit Parametern zu ergänzen.
- Die Factory kann allenfalls die erzeugten Objekte zwischenspeichern und später wiederverwenden.
![](https://cdn.mathpix.com/cropped/2025_01_02_3c6f52f6bfb0358b29fcg-14.jpg?height=753&width=1009&top_left_y=508&top_left_x=2166)

\section*{Singleton: Problem und Lösung}
- Problem
- Man benötigt von einer Klasse nur eine einzige Instanz.
- Diese Instanz muss global sichtbar sein.
- Lösung
- Klasse mit einer statischen Methode, , die immer dasselbe Objekt zurückliefert.
/ / Instanz-Methoden ...
$\}$
- Statische Methode wird public deklariert.

\section*{Singleton: Hinweise}
- Hinweise
- Globale Sichtbarkeit wird heutzutage sehr kritisch betrachtet.
- Lazy Creation für die Instanz ist möglich, dann sollte aber die getInstance() Methode synchronisiert
\begin{tabular}{|l|}
\hline \multicolumn{1}{|c|}{\begin{tabular}{c} 
«Singleton» \\
Singleton
\end{tabular}} \\
\hline -instance : Singleton \\
$\ldots$ normale Instanz Attribute
\end{tabular}$|$\begin{tabular}{l}
$\pm$ getInstance() : Singleton \\
$\ldots$. normale Instanz Methoden \\
\hline
\end{tabular} werden.

\section*{Dependency Injection (DI): Problem und Lösung}
- Problem
- Eine Klasse braucht eine Referenz auf ein anderes Objekt. Dieses Objekt muss ein bestimmtes Interface definieren, je nach Konfiguration aber mit einer anderen Funktionalität.
- Lösung
- Anstelle, dass die Klasse das abhängige Objekt selber erzeugt, wird dieses Objekt von aussen (Injector) gesetzt.
![](https://cdn.mathpix.com/cropped/2025_01_02_3c6f52f6bfb0358b29fcg-17.jpg?height=1317&width=1136&top_left_y=412&top_left_x=1907)

\section*{Dependency Injection (DI): Hinweise (1/2)}
- Hinweise
- Ersatz für das Factory Pattern.
- Direkter Widerspruch zum GRASP Creator Prinzip.
- Viele Frameworks unterstützen inzwischen DI (z.B. Spring), kann aber problemlos auch ohne Framework angewendet werden.
- Erleichtert das Schreiben von Testfällen, insbesondere den Gebrauch von Mocks.
![](https://cdn.mathpix.com/cropped/2025_01_02_3c6f52f6bfb0358b29fcg-18.jpg?height=1313&width=1122&top_left_y=412&top_left_x=1913)

\section*{Proxy: Problem und Lösung}
- Problem
- Ein Objekt ist nicht oder noch nicht im selben Adressraum verfügbar.
- Lösung
- Ein Stellvertreter Objekt («Proxy») mit demselben Interface wird anstelle des richtigen Objekts verwendet.
- Das «Proxy» Objekt leitet alle Methodenaufrufe zum richtigen Objekt weiter.
![](https://cdn.mathpix.com/cropped/2025_01_02_3c6f52f6bfb0358b29fcg-19.jpg?height=1017&width=1166&top_left_y=546&top_left_x=1828)
- Einsatz als (Struktur ist dieselbe!)
- «Remote Proxy» ist ein Stellvertreter für ein Objekt in einem anderen Adressraum und übernimmt die Kommunikation mit diesem.
- «Virtual Proxy» verzögert das Erzeugen des richtigen Objekts auf das erste Mal, dass dieses benutzt wird.
- «Protection Proxy» kontrolliert den
![](https://cdn.mathpix.com/cropped/2025_01_02_3c6f52f6bfb0358b29fcg-20.jpg?height=1013&width=1166&top_left_y=548&top_left_x=1930)
- Sieht ähnlich aus wie ein Adapter, der Unterschied ist aber, dass der «Adaptee», in diesem Fall das RealSubject, auch dasselbe Interface implementiert wie der «Adapter» resp. Subject
- Vom Aufbau her identisch mit dem Decorator Pattern (siehe LE09), hat aber einen anderen Zweck.
![](https://cdn.mathpix.com/cropped/2025_01_02_3c6f52f6bfb0358b29fcg-21.jpg?height=1022&width=1170&top_left_y=548&top_left_x=1822)

\section*{Chain of Responsibility: Problem und Lösung}
- Problem
- Für eine Anfrage gibt es potentiell mehrere Handler, aber von vornherein ist es nicht möglich (oder nur sehr schwer), den richtigen Handler herauszufinden.
- Lösung
- Die Handler werden in einer einfach verketteten Liste hintereinandergeschaltet.
- Jeder Handler entscheidet dann, ob der die Anfrage selber beantworten möchte oder sie an den nächsten Handler weiterleitet.
![](https://cdn.mathpix.com/cropped/2025_01_02_3c6f52f6bfb0358b29fcg-22.jpg?height=1085&width=1311&top_left_y=474&top_left_x=1909)

\section*{Chain of Responsibility: Hinweise}
- Hinweise
- Als Variante davon leitet jeder Handler die Anfrage an den nächsten Handler weiter, unabhängig davon, ob er sie selber behandelt oder nicht.
- Es könnte sein, dass gar kein Handler die Anfrage behandelt.
![](https://cdn.mathpix.com/cropped/2025_01_02_3c6f52f6bfb0358b29fcg-23.jpg?height=1089&width=1303&top_left_y=472&top_left_x=1917)

\title{
1. Einführung in Design Patterns
}
2. Repetition GRASP
3. Design Pattern
4. Wrap-up und Ausblick
- Design Patterns sind wichtige Werkzeuge um gut strukturierten, wartbaren Code zu schreiben.
- Die Kombination von Singleton, Factory und Adapter wurde traditionell oft eingesetzt, um externe Dienste anzusprechen.
- Anstelle von Singleton und Factory ist vermehrt Dependency Injection (DI) vorzuziehen.
- Ein Proxy kapselt den Zugriff auf ein anderes Objekt vollständig ab und ist wie ein Stellvertreter.
- Chain of Responsibility ist dann angebracht, wenn eine Aufgabe potentiell von mehreren Handlern übernommen werden kann, aber für eine konkrete Aufgabe im voraus nicht klar ist, welcher Handler wirklich zuständig ist.

\section*{Ausblick}
- In der nächsten Lerneinheit werden wir:
- weitere Design Patterns kennenlernen und anwenden.

\section*{Quellenverzeichnis}
[1] Larman, C.: UML 2 und Patterns angewendet, mitp Professional, 2005
[2] Seidel, M. et al.: UML @ Classroom: Eine Einführung in die objektorientierte Modellierung, dpunkt.verlag, 2012
[3] Martin, R. C.: Clean Architecture: A Craftsman's Guide to Software Structure and Design, mitp Professional, 2018
[4] Gamma, E et al.: Design Patterns: Elements of Reusable Object-Oriented Software Addison Wesley Longman, 1995
[5] McDonald, J: DZone Refcardz: Design Patterns, www.dzone.com, 2008
\end{verbatim}


\end{document}