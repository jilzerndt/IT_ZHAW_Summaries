\section{Persistenz}

\begin{concept}{Persistenz Grundlagen}\\
Persistenz bezeichnet die dauerhafte Speicherung von Daten über das Programmende hinaus:
\begin{itemize}
    \item Speicherung in Datenbankmanagementsystemen (DBMS)
    \item Haupttypen:
    \begin{itemize}
        \item Relationale Datenbanksysteme (RDBMS)
        \item NoSQL-Datenbanken (ohne fixes Schema)
    \end{itemize}
    \item O/R-Mapping (Object Relational Mapping)
    \begin{itemize}
        \item Abbildung zwischen Objekten und Datensätzen
        \item Überwindung des Strukturbruchs (Impedance Mismatch)
    \end{itemize}
\end{itemize}
\end{concept}

\begin{definition}{O/R-Mismatch}\\
Der Strukturbruch zwischen objektorientierter und relationaler Welt:
\begin{itemize}
    \item \textbf{Typen-Systeme:}
    \begin{itemize}
        \item Unterschiedliche NULL-Behandlung
        \item Datum/Zeit-Darstellung
    \end{itemize}
    \item \textbf{Beziehungen:}
    \begin{itemize}
        \item Richtung der Beziehungen
        \item Mehrfachbeziehungen
        \item Vererbung
    \end{itemize}
    \item \textbf{Identität:}
    \begin{itemize}
        \item OO: Implizite Objektidentität
        \item DB: Explizite Identität (Primary Key)
    \end{itemize}
\end{itemize}
\end{definition}

\subsection{JDBC - Java Database Connectivity}

\begin{concept}{JDBC Grundlagen}\\
JDBC ist die standardisierte Schnittstelle für Datenbankzugriffe in Java:
\begin{itemize}
    \item Seit JDK 1.1 (1997)
    \item Plattformunabhängig
    \item Datenbankunabhängig
    \item Aktuelle Version: 4.2
\end{itemize}
\end{concept}

\begin{KR}{JDBC Verwendung}
Grundlegende Schritte für Datenbankzugriff:
\begin{enumerate}
    \item JDBC-Treiber installieren und laden
    \item Verbindung zur Datenbank aufbauen
    \item SQL-Statements ausführen
    \item Ergebnisse verarbeiten
    \item Transaktion abschließen (Commit/Rollback)
    \item Verbindung schließen
\end{enumerate}
\end{KR}

\begin{example}{JDBC Basisbeispiel}
\begin{lstlisting}[language=Java, style=base]
import java.sql.*;

public class DbTest {
    public static void main(String[] args) 
            throws SQLException {
        // Verbindung aufbauen
        Connection con = DriverManager.getConnection(
            "jdbc:postgresql://test.zhaw.ch/testdb",
            "user", "password");
            
        // Statement erstellen und ausfuehren
        Statement stmt = con.createStatement();
        ResultSet rs = stmt.executeQuery(
            "SELECT * FROM test ORDER BY name");
            
        // Ergebnisse verarbeiten
        while (rs.next()) {
            System.out.println(
                "Name: " + rs.getString("name"));
        }
        
        // Aufraeumen
        rs.close();
        stmt.close();
        con.close();
    }
}
\end{lstlisting}
\end{example}

\subsection{Design Patterns für Persistenz}

\begin{theorem}{Persistenz Design Patterns}\\
Drei grundlegende Ansätze für die Persistenzschicht:
\begin{itemize}
    \item \textbf{Active Record (Anti-Pattern):}
    \begin{itemize}
        \item Entität verwaltet eigene Persistenz
        \item Vermischung von Fachlichkeit und Technik
        \item Schlechte Testbarkeit
    \end{itemize}
    \item \textbf{Data Access Object (DAO):}
    \begin{itemize}
        \item Kapselung des Datenbankzugriffs
        \item Trennung von Fachlichkeit und Technik
        \item Gute Testbarkeit durch Mocking
    \end{itemize}
    \item \textbf{Repository (DDD):}
    \begin{itemize}
        \item Abstraktionsschicht über Data-Mapper
        \item Zentralisierung von Datenbankabfragen
        \item Komplexere Implementierung
    \end{itemize}
\end{itemize}
\end{theorem}

\begin{KR}{DAO Implementation}\\
Schritte zur Implementierung eines DAOs:
\begin{enumerate}
    \item Interface definieren:
    \begin{itemize}
        \item CRUD-Methoden (Create, Read, Update, Delete)
        \item Spezifische Suchmethoden
    \end{itemize}
    \item Domänenklasse erstellen:
    \begin{itemize}
        \item Nur fachliche Attribute
        \item Keine Persistenzlogik
    \end{itemize}
    \item DAO-Implementierung:
    \begin{itemize}
        \item Datenbankzugriff kapseln
        \item O/R-Mapping implementieren
        \item Transaktionshandling
    \end{itemize}
\end{enumerate}
\end{KR}

\begin{example2}{DAO Implementation}
\begin{lstlisting}[language=Java, style=base]
public interface ArticleDAO {
    void insert(Article item);
    void update(Article item);
    void delete(Article item);
    Article findById(int id);
    Collection<Article> findAll();
    Collection<Article> findByName(String name);
}

public class Article {
    private long id;
    private String name;
    private float price;
    
    // Getter/Setter
}

public class JdbcArticleDAO implements ArticleDAO {
    private Connection conn;
    
    public void insert(Article item) {
        PreparedStatement stmt = conn.prepareStatement(
            "INSERT INTO articles (name, price) VALUES (?, ?)");
        stmt.setString(1, item.getName());
        stmt.setFloat(2, item.getPrice());
        stmt.executeUpdate();
    }
    // weitere Implementierungen
}
\end{lstlisting}
\end{example2}

\subsection{Java Persistence API (JPA)}

\begin{concept}{JPA Grundkonzepte}\\
JPA ist der Java-Standard für O/R-Mapping:
\begin{itemize}
    \item \textbf{Entity-Klassen:}
    \begin{itemize}
        \item Plain Old Java Objects (POJOs)
        \item Annotation @Entity
        \item Keine JPA-spezifischen Abhängigkeiten
    \end{itemize}
    \item \textbf{Referenzen:}
    \begin{itemize}
        \item Eager/Lazy Loading
        \item Automatisches Nachladen
    \end{itemize}
    \item \textbf{Provider:}
    \begin{itemize}
        \item Hibernate
        \item EclipseLink
        \item OpenJPA
    \end{itemize}
\end{itemize}
\end{concept}

\begin{concept}{JPA Technologie-Stack}\\
\begin{itemize}
    \item Java Application
    \item Java Persistence API
    \item JPA Provider (Hibernate, EclipseLink, etc.)
    \item JDBC Driver
    \item Relationale Datenbank
\end{itemize}
\includegraphics[width=0.8\linewidth]{images/2025_01_02_5ba1dc702e9f94ba8e06g-29.jpg}
\end{concept}

\begin{KR}{JPA Entity Erstellung}
\begin{enumerate}
    \item Entity-Klasse definieren:
    \begin{itemize}
        \item @Entity Annotation
        \item ID-Feld mit @Id markieren
    \end{itemize}
    \item Beziehungen definieren:
    \begin{itemize}
        \item @OneToMany, @ManyToOne etc.
        \item Navigationsrichtung festlegen
    \end{itemize}
    \item Validierung hinzufügen:
    \begin{itemize}
        \item @NotNull, @Size etc.
        \item Geschäftsregeln
    \end{itemize}
\end{enumerate}
\end{KR}

\begin{example}{Parent-Child Beziehung mit JPA}
\begin{lstlisting}[language=Java, style=base]
@Entity
public class Department {
    @Id @GeneratedValue
    private Long id;
    
    private String name;
    
    @OneToMany(mappedBy = "department")
    private List<Employee> employees;
}

@Entity
public class Employee {
    @Id @GeneratedValue
    private Long id;
    
    @ManyToOne
    @JoinColumn(name = "department_id")
    private Department department;
    
    private String name;
    private double salary;
}
\end{lstlisting}
\end{example}

\subsection{Repository Pattern}

\begin{concept}{Repository Pattern}\\
Das Repository Pattern bietet eine zusätzliche Abstraktionsschicht über der Data-Mapper-Schicht:
\begin{itemize}
    \item Zentralisierung von Datenbankabfragen
    \item Domänenorientierte Schnittstelle
    \item Unterstützung komplexer Abfragen
    \item Häufig in Kombination mit Spring Data
\end{itemize}
\end{concept}

\begin{example}{Spring Data Repository}
\begin{lstlisting}[language=Java, style=base]
@Repository
public interface SaleRepository 
        extends CrudRepository<Sale, String> {
    
    List<Sale> findOrderByDateTime();
    
    List<Sale> findByDateTime(
        final LocalDateTime dateTime);
}

@Service
public class ProcessSaleHandler {
    private final ProductDescriptionRepository catalog;
    private final SaleRepository saleRepository;
    
    @Transactional
    public void endSale() {
        assert(currentSale != null 
            && !currentSale.isComplete());
        this.currentSale.becomeComplete();
        this.saleRepository.save(currentSale);
    }
}
\end{lstlisting}
\end{example}

\begin{remark}
Spring Data unterstützt die automatische Generierung von Repository-Implementierungen basierend auf Methodennamen. Dies reduziert den Implementierungsaufwand erheblich.
\end{remark}