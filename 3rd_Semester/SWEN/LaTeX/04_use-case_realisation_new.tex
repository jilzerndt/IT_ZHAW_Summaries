\section{Use Case Realisation}

\begin{concept}{Use Case Realization}\\
Die Umsetzung von Use Cases erfolgt durch:
\begin{itemize}
    \item Detaillierte Szenarien aus den Use Cases
    \item Systemantworten müssen realisiert werden
    \item UI statt System im SSD
    \item Systemoperationen sind die zu implementierenden Elemente
\end{itemize}
\end{concept}

\begin{definition}{UML im Implementierungsprozess}\\
UML dient als:
\begin{itemize}
    \item Zwischenschritt bei wenig Erfahrung
    \item Kompakter Ersatz für Programmiercode
    \item Kommunikationsmittel (auch für Nicht-Techniker)
\end{itemize}
\end{definition}

\begin{KR}{Vorgehen bei der Use Case Realization}\\
\textbf{1. Vorbereitung:}
\begin{itemize}
    \item Use Case auswählen und SSD ableiten
    \item Systemoperation identifizieren
    \item Operation Contract erstellen/prüfen
\end{itemize}

\textbf{2. Analyse:}
\begin{itemize}
    \item Aktuellen Code/Dokumentation analysieren
    \item DCD überprüfen/aktualisieren
    \item Vergleich mit Domänenmodell
    \item Neue Klassen gemäß Domänenmodell erstellen
\end{itemize}

\textbf{3. Realisierung:}
\begin{itemize}
    \item Controller Klasse bestimmen
    \item Zu verändernde Klassen festlegen
    \item Weg zu diesen Klassen festlegen:
    \begin{itemize}
        \item Parameter für Wege definieren
        \item Klassen bei Bedarf erstellen
        \item Verantwortlichkeiten zuweisen
        \item Verschiedene Varianten evaluieren
    \end{itemize}
    \item Veränderungen implementieren
    \item Review durchführen
\end{itemize}
\end{KR}

\begin{example}{Use Case Realization: Verkauf abwickeln}
\textbf{1. Vorbereitung:}
\begin{itemize}
    \item \textbf{Use Case:} Verkauf abwickeln
    \item \textbf{Systemoperation:} makeNewSale()
    \item \textbf{Contract:} Neue Sale-Instanz wird erstellt
\end{itemize}

\textbf{2. Analyse:}
\begin{itemize}
    \item \textbf{Klassen:} Register, Sale
    \item \textbf{DCD:} Beziehung Register-Sale prüfen
    \item \textbf{Neue Klassen:} Payment, SaleLineItem
\end{itemize}

\textbf{3. Implementierung:}
\begin{itemize}
    \item Register als Controller
    \item Sale-Klasse erweitern
    \item Beziehungen implementieren
\end{itemize}
\end{example}

\begin{KR}{Typische Implementierungsfehler vermeiden}\\
\begin{itemize}
    \item \textbf{Architekturverletzungen:}
    \begin{itemize}
        \item Schichtentrennung beachten
        \item Abhängigkeiten richtig setzen
    \end{itemize}
    
    \item \textbf{GRASP-Verletzungen:}
    \begin{itemize}
        \item Information Expert beachten
        \item Creator Pattern richtig anwenden
        \item High Cohesion erhalten
    \end{itemize}
    
    \item \textbf{Testbarkeit:}
    \begin{itemize}
        \item Klassen isoliert testbar halten
        \item Abhängigkeiten mockbar gestalten
    \end{itemize}
\end{itemize}
\end{KR}
