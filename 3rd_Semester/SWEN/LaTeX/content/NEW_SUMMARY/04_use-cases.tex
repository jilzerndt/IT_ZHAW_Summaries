\section{Use Case Realization}

\begin{concept}{Use Case Realization}
Die Umsetzung von Use Cases erfolgt durch:
\begin{itemize}
    \item Detaillierte Szenarien aus den Use Cases
    \item Systemantworten müssen realisiert werden
    \item UI statt System im SSD
    \item Systemoperationen sind die zu implementierenden Elemente
\end{itemize}
\end{concept}

\subsection{Analyse-Artefakte}

\begin{concept}{Einfluss der Analyse-Artefakte}
\begin{itemize}
    \item \textbf{Use Cases:}
    \begin{itemize}
        \item Standardszenario
        \item Erweiterungen
        \item Definiert Systemoperationen
    \end{itemize}
    \item \textbf{Systemverträge:}
    \begin{itemize}
        \item Definieren Vorbedingungen
        \item Definieren Nachbedingungen
        \item Legen Invarianten fest
    \end{itemize}
    \item \textbf{Domänenmodell:}
    \begin{itemize}
        \item Inspiriert Softwareklassen
        \item Definiert Attribute
        \item Zeigt Beziehungen auf
    \end{itemize}
\end{itemize}
\end{concept}

\begin{KR}{Vorgehen bei der Use Case Realization}
\textbf{1. Vorbereitung}
\begin{itemize}
    \item Use Case auswählen und SSD ableiten
    \item Systemoperation identifizieren
    \item Operation Contract erstellen/prüfen
    \item Code/Dokumentation analysieren
    \item DCD überprüfen/aktualisieren
\end{itemize}

\textbf{2. Realisierung}
\begin{itemize}
    \item Controller Klasse bestimmen
    \item Zu verändernde Klassen festlegen
    \item Weg zu diesen Klassen festlegen
    \item Veränderungen implementieren
    \item Review durchführen
\end{itemize}
\end{KR}

\subsection{Design und Implementation}

\begin{concept}{UML im Design-Prozess}
UML dient als:
\begin{itemize}
    \item Zwischenschritt bei wenig Erfahrung
    \item Kompakter Ersatz für Programmiercode
    \item Kommunikationsmittel (auch für Nicht-Techniker)
\end{itemize}

\textbf{Arten der Modellierung:}
\begin{itemize}
    \item \textbf{Statische Modelle:} Klassenstruktur
    \item \textbf{Dynamische Modelle:} Verhalten
\end{itemize}
\end{concept}

\begin{KR}{Design to Code}
\textbf{Aus Design-Klassen-Diagramm (DCD):}
\begin{itemize}
    \item Klassen und deren Attribute
    \item Methoden und deren Signaturen
    \item Beziehungen zwischen Klassen
\end{itemize}

\textbf{Aus Interaktionsdiagrammen:}
\begin{itemize}
    \item Methodenaufrufe
    \item Parameter
    \item Sequenz der Aufrufe
\end{itemize}
\end{KR}

\subsection{GRASP Anwendung}

\begin{KR}{GRASP Patterns in der Realization}
\textbf{Die 5 wichtigsten Prinzipien:}
\begin{itemize}
    \item \textbf{Information Expert:}
    \begin{itemize}
        \item Verantwortlichkeit dort, wo Information liegt
        \item Basis für Methodenzuweisung
    \end{itemize}
    \item \textbf{Creator:}
    \begin{itemize}
        \item Regeln für Objekterzeugung
        \item Container erzeugt Inhalt
    \end{itemize}
    \item \textbf{Controller:}
    \begin{itemize}
        \item Erste Systemanlaufstelle
        \item Koordiniert Systemoperationen
    \end{itemize}
    \item \textbf{Low Coupling:}
    \begin{itemize}
        \item Minimale Abhängigkeiten
        \item Entscheidungshilfe bei Alternativen
    \end{itemize}
    \item \textbf{High Cohesion:}
    \begin{itemize}
        \item Fokussierte Verantwortlichkeiten
        \item Zusammengehörige Funktionalität
    \end{itemize}
\end{itemize}
\end{KR}

%todo: Add example of complete use case realization with all steps

\begin{KR}{Testing und Refactoring}
\textbf{1. Funktionale Prüfung}
\begin{itemize}
    \item Use Case Szenarien durchspielen
    \item Randfälle testen
    \item Fehlersituationen prüfen
\end{itemize}

\textbf{2. Strukturelle Prüfung}
\begin{itemize}
    \item Architekturkonformität
    \item GRASP-Prinzipien
    \item Clean Code Regeln
\end{itemize}

\textbf{3. Qualitätsprüfung}
\begin{itemize}
    \item Testabdeckung
    \item Wartbarkeit
    \item Performance
\end{itemize}
\end{KR}

%todo: Add testing examples and common refactoring scenarios