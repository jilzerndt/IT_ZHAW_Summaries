\begin{example}{Adapter Pattern}\\
\textbf{Szenario:} Altbestand an Drittanbieter-Bibliothek integrieren
\begin{lstlisting}[language=Java, style=base] 
// Bestehende Schnittstelle
interface ModernPrinter {
    void printDocument(String content);
}

// Alte Drittanbieter-Klasse
class LegacyPrinter {
    public void print(String[] pages) {
        for(String page : pages) {
            System.out.println(page);
        }
    }
}

// Adapter
class PrinterAdapter implements ModernPrinter {
    private LegacyPrinter legacyPrinter;
    
    public PrinterAdapter(LegacyPrinter printer) {
        this.legacyPrinter = printer;
    }
    
    public void printDocument(String content) {
        String[] pages = content.split("\n");
        legacyPrinter.print(pages);
    }
}
\end{lstlisting}
\end{example}

\begin{example}{Simple Factory}\\
\textbf{Szenario:} Erzeugung von verschiedenen Datenbankverbindungen
\begin{lstlisting}[language=Java, style=base]
class DatabaseFactory {
    public static Database createDatabase(String type) {
        switch(type) {
            case "MySQL":
                return new MySQLDatabase();
            case "PostgreSQL":
                return new PostgreSQLDatabase();
            default:
                throw new IllegalArgumentException("Unknown DB type");
        }
    }
}

// Verwendung
Database db = DatabaseFactory.createDatabase("MySQL");
\end{lstlisting}
\end{example}

\begin{example}{Singleton}\\
\textbf{Szenario:} Globale Konfigurationsverwaltung
\begin{lstlisting}[language=Java, style=base]
public class Configuration {
    private static Configuration instance;
    private Map<String, String> config;
    
    private Configuration() {
        config = new HashMap<>();
    }
    
    public static Configuration getInstance() {
        if(instance == null) {
            instance = new Configuration();
        }
        return instance;
    }
}
\end{lstlisting}
\end{example}

\begin{example}{Dependency Injection}\\
\textbf{Szenario:} Flexible Logger-Implementation
\begin{lstlisting}[language=Java, style=base]
interface Logger {
    void log(String message);
}

class FileLogger implements Logger {
    public void log(String message) {
        // Log to file
    }
}

class UserService {
    private final Logger logger;
    
    public UserService(Logger logger) { // Dependency Injection
        this.logger = logger;
    }
}
\end{lstlisting}
\end{example}

\begin{example}{Proxy}\\
\textbf{Szenario:} Verzögertes Laden eines großen Bildes
\begin{lstlisting}[language=Java, style=base]
interface Image {
    void display();
}

class RealImage implements Image {
    private String filename;
    
    public RealImage(String filename) {
        this.filename = filename;
        loadFromDisk();
    }
    
    private void loadFromDisk() {
        System.out.println("Loading " + filename);
    }
    
    public void display() {
        System.out.println("Displaying " + filename);
    }
}

class ImageProxy implements Image {
    private RealImage realImage;
    private String filename;
    
    public ImageProxy(String filename) {
        this.filename = filename;
    }
    
    public void display() {
        if(realImage == null) {
            realImage = new RealImage(filename);
        }
        realImage.display();
    }
}
\end{lstlisting}
\end{example}

\begin{example}{Chain of Responsibility}\\
\textbf{Szenario:} Authentifizierungskette
\begin{lstlisting}[language=Java, style=base]
abstract class AuthHandler {
    protected AuthHandler next;
    
    public void setNext(AuthHandler next) {
        this.next = next;
    }
    
    public abstract boolean handle(String username, String password);
}

class LocalAuthHandler extends AuthHandler {
    public boolean handle(String username, String password) {
        if(checkLocalDB(username, password)) {
            return true;
        }
        return next != null ? next.handle(username, password) : false;
    }
}

class LDAPAuthHandler extends AuthHandler {
    public boolean handle(String username, String password) {
        if(checkLDAP(username, password)) {
            return true;
        }
        return next != null ? next.handle(username, password) : false;
    }
}
\end{lstlisting}
\end{example}

\begin{example}{Decorator}\\
\textbf{Szenario:} Dynamische Erweiterung eines Text-Editors
\begin{lstlisting}[language=Java, style=base]
interface TextComponent {
    String render();
}

class SimpleText implements TextComponent {
    private String text;
    
    public SimpleText(String text) {
        this.text = text;
    }
    
    public String render() {
        return text;
    }
}

class BoldDecorator implements TextComponent {
    private TextComponent component;
    
    public BoldDecorator(TextComponent component) {
        this.component = component;
    }
    
    public String render() {
        return "<b>" + component.render() + "</b>";
    }
}
\end{lstlisting}
\end{example}

\begin{example}{Observer}\\
\textbf{Szenario:} News-Benachrichtigungssystem
\begin{lstlisting}[language=Java, style=base]
interface NewsObserver {
    void update(String news);
}

class NewsAgency {
    private List<NewsObserver> observers = new ArrayList<>();
    
    public void addObserver(NewsObserver observer) {
        observers.add(observer);
    }
    
    public void notifyObservers(String news) {
        for(NewsObserver observer : observers) {
            observer.update(news);
        }
    }
}

class NewsChannel implements NewsObserver {
    private String name;
    
    public NewsChannel(String name) {
        this.name = name;
    }
    
    public void update(String news) {
        System.out.println(name + " received: " + news);
    }
}
\end{lstlisting}
\end{example}

\begin{example}{Strategy}\\
\textbf{Szenario:} Verschiedene Zahlungsmethoden
\begin{lstlisting}[language=Java, style=base]
interface PaymentStrategy {
    void pay(int amount);
}

class CreditCardPayment implements PaymentStrategy {
    private String cardNumber;
    
    public void pay(int amount) {
        System.out.println("Paid " + amount + " using Credit Card");
    }
}

class PayPalPayment implements PaymentStrategy {
    private String email;
    
    public void pay(int amount) {
        System.out.println("Paid " + amount + " using PayPal");
    }
}
\end{lstlisting}
\end{example}

\begin{example}{Composite}\\
\textbf{Szenario:} Dateisystem-Struktur
\begin{lstlisting}[language=Java, style=base]
interface FileSystemComponent {
    void list(String prefix);
}

class File implements FileSystemComponent {
    private String name;
    
    public void list(String prefix) {
        System.out.println(prefix + name);
    }
}

class Directory implements FileSystemComponent {
    private String name;
    private List<FileSystemComponent> children = new ArrayList<>();
    
    public void add(FileSystemComponent component) {
        children.add(component);
    }
    
    public void list(String prefix) {
        System.out.println(prefix + name);
        for(FileSystemComponent child : children) {
            child.list(prefix + "  ");
        }
    }
}
\end{lstlisting}
\end{example}

\begin{example}{State}\\
\textbf{Szenario:} Verkaufsautomat
\begin{lstlisting}[language=Java, style=base]
interface VendingMachineState {
    void insertCoin();
    void ejectCoin();
    void selectProduct();
    void dispense();
}

class HasCoinState implements VendingMachineState {
    private VendingMachine machine;
    
    public void selectProduct() {
        System.out.println("Product selected");
        machine.setState(machine.getSoldState());
    }
    
    public void insertCoin() {
        System.out.println("Already have coin");
    }
}

class VendingMachine {
    private VendingMachineState currentState;
    
    public void setState(VendingMachineState state) {
        this.currentState = state;
    }
    
    public void insertCoin() {
        currentState.insertCoin();
    }
}
\end{lstlisting}
\end{example}

\begin{example}{Visitor}\\
\textbf{Szenario:} Dokumentstruktur mit verschiedenen Operationen
\begin{lstlisting}[language=Java, style=base]
interface DocumentElement {
    void accept(Visitor visitor);
}

interface Visitor {
    void visit(Paragraph paragraph);
    void visit(Heading heading);
}

class HTMLVisitor implements Visitor {
    public void visit(Paragraph p) {
        System.out.println("<p>" + p.getText() + "</p>");
    }
    
    public void visit(Heading h) {
        System.out.println("<h1>" + h.getText() + "</h1>");
    }
}
\end{lstlisting}
\end{example}

\begin{example}{Facade}\\
\textbf{Szenario:} Vereinfachte Multimedia-Bibliothek
\begin{lstlisting}[language=Java, style=base]
class MultimediaFacade {
    private AudioSystem audio;
    private VideoSystem video;
    private SubtitleSystem subtitles;
    
    public void playMovie(String movie) {
        audio.initialize();
        video.initialize();
        subtitles.load(movie);
        video.play(movie);
        audio.play();
    }
}
\end{lstlisting}
\end{example}

\begin{example}{Abstract Factory}\\
\textbf{Szenario:} GUI-Elemente für verschiedene Betriebssysteme
\begin{lstlisting}[language=Java, style=base]
interface GUIFactory {
    Button createButton();
    Checkbox createCheckbox();
}

class WindowsFactory implements GUIFactory {
    public Button createButton() {
        return new WindowsButton();
    }
    
    public Checkbox createCheckbox() {
        return new WindowsCheckbox();
    }
}

class MacFactory implements GUIFactory {
    public Button createButton() {
        return new MacButton();
    }
    
    public Checkbox createCheckbox() {
        return new MacCheckbox();
    }
}
\end{lstlisting}
\end{example}

\begin{example}{Factory Method Implementation}\\
\textbf{Aufgabe:} Implementieren Sie eine Factory für verschiedene Dokumenttypen (PDF, Word, Text)

\textbf{Lösung:}
\begin{lstlisting}[language=Java, style=base]
// Interface fuer Produkte
interface Document {
    void open();
    void save();
}

// Konkrete Produkte
class PdfDocument implements Document {
    public void open() { /* ... */ }
    public void save() { /* ... */ }
}

// Factory Method Pattern
abstract class DocumentCreator {
    abstract Document createDocument();
    
    // Template Method
    final void processDocument() {
        Document doc = createDocument();
        doc.open();
        doc.save();
    }
}

// Konkrete Factory
class PdfDocumentCreator extends DocumentCreator {
    Document createDocument() {
        return new PdfDocument();
    }
}
\end{lstlisting}
\end{example}

\begin{example}{Observer Pattern Implementation}\\
\textbf{Aufgabe:} Implementieren Sie ein Benachrichtigungssystem für Aktienkurse

\textbf{Lösung:}
\begin{lstlisting}[language=Java, style=base]
interface StockObserver {
    void update(String stock, double price);
}

class StockMarket {
    private List<StockObserver> observers = new ArrayList<>();
    
    public void attach(StockObserver observer) {
        observers.add(observer);
    }
    
    public void notifyObservers(String stock, double price) {
        for(StockObserver observer : observers) {
            observer.update(stock, price);
        }
    }
}

class StockDisplay implements StockObserver {
    public void update(String stock, double price) {
        System.out.println("Stock: " + stock + 
                         " updated to " + price);
    }
}
\end{lstlisting}
\end{example}