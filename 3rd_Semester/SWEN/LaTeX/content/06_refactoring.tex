\section{Implementation, Refactoring und Testing}

\subsection{Von Design zu Code}

\begin{concept}{Implementierungsstrategien}
\textbf{1. Bottom-Up Entwicklung:}
\begin{itemize}
    \item Implementierung beginnt mit Basisbausteinen
    \item Schrittweise Integration zu größeren Komponenten
    \item \textbf{Vorteile:} Gründlich, solide Basis
    \item \textbf{Nachteile:} Spätes Feedback
\end{itemize}

\textbf{2. Agile Entwicklung:}
\begin{itemize}
    \item Inkrementelle Entwicklung in Sprints
    \item Kontinuierliche Integration und Auslieferung
    \item \textbf{Vorteile:} Flexibilität, schnelles Feedback
    \item \textbf{Nachteile:} Mögliche Restrukturierung nötig
\end{itemize}
\end{concept}

\begin{example}{Bottom-Up vs. Agile Entwicklung}
\textbf{Szenario: Entwicklung eines Onlineshops}

\begin{lstlisting}[language=Java]
// Bottom-Up Ansatz
// 1. Basisklassen
public class Product {
    private String id;
    private String name;
    private BigDecimal price;
}

public class OrderItem {
    private Product product;
    private int quantity;
}

// 2. Zusammengesetzte Klassen
public class Order {
    private List<OrderItem> items;
    private Customer customer;
}

// Agiler Ansatz
// 1. Minimales funktionierendes System
public class SimpleOrder {
    public void addProduct(String productId) {
        // Minimale Implementation
    }
}

// 2. Inkrementelle Erweiterung
public class EnhancedOrder {
    public void addProduct(String productId, int qty) {
        // Erweiterte Funktionalitaet
    }
    
    public BigDecimal calculateTotal() {
        // Neue Funktion
    }
}
\end{lstlisting}
\end{example}

\begin{KR}{Test-Driven Development (TDD)}
\textbf{Red-Green-Refactor Zyklus:}
\begin{lstlisting}[language=Java]
// 1. Red: Test schreiben
@Test
void calculatesOrderTotal() {
    Order order = new Order();
    order.addItem(new Product("p1", new Money(10)));
    order.addItem(new Product("p2", new Money(20)));
    
    Money total = order.getTotal();
    
    assertEquals(new Money(30), total);
}

// 2. Green: Minimale Implementation
public class Order {
    private List<Product> items = new ArrayList<>();
    
    public void addItem(Product p) {
        items.add(p);
    }
    
    public Money getTotal() {
        return items.stream()
            .map(Product::getPrice)
            .reduce(Money.ZERO, Money::add);
    }
}

// 3. Refactor: Code verbessern
public class Order {
    private List<OrderItem> items = new ArrayList<>();
    
    public void addItem(Product p) {
        addItem(p, 1);
    }
    
    public void addItem(Product p, int quantity) {
        items.add(new OrderItem(p, quantity));
    }
    
    public Money getTotal() {
        return items.stream()
            .map(OrderItem::getSubtotal)
            .reduce(Money.ZERO, Money::add);
    }
}
\end{lstlisting}
\end{KR}

\begin{example}{Behavior-Driven Development (BDD)}
\begin{lstlisting}[style=basesmol]
Feature: Order Calculation
  As a customer
  I want to see my order total
  So that I know how much I need to pay

  Scenario: Calculate order with multiple items
    Given I have an empty shopping cart
    When I add 2 units of product "P1" at $10 each
    And I add 1 unit of product "P2" at $20
    Then my order total should be $40

@Test
void calculatesOrderWithMultipleItems() {
    // Given
    ShoppingCart cart = new ShoppingCart();
    
    // When
    cart.addItem(new Product("P1", 10.00), 2);
    cart.addItem(new Product("P2", 20.00), 1);
    
    // Then
    assertEquals(40.00, cart.getTotal());
}
\end{lstlisting}
\end{example}

\begin{KR}{Effektives Refactoring}
\textbf{1. Code Smell: Lange Methode}
\begin{lstlisting}[language=Java]
// Vor Refactoring
public class OrderProcessor {
    public void processOrder(Order order) {
        // Validierung
        if (order == null) throw new IllegalArgumentException();
        if (order.getItems().isEmpty()) 
            throw new EmptyOrderException();
        
        // Lagerpruefung
        for (OrderItem item : order.getItems()) {
            if (!inventory.hasStock(item.getProduct(), 
                                  item.getQuantity())) {
                throw new OutOfStockException();
            }
        }
        
        // Bezahlung
        PaymentResult result = 
            paymentService.process(order.getTotal());
        if (!result.isSuccessful()) {
            throw new PaymentFailedException();
        }
        
        // Versand
        shippingService.schedule(order);
    }
}

// Nach Refactoring
public class OrderProcessor {
    public void processOrder(Order order) {
        validateOrder(order);
        checkInventory(order);
        processPayment(order);
        scheduleShipping(order);
    }
    
    private void validateOrder(Order order) {
        if (order == null) 
            throw new IllegalArgumentException();
        if (order.getItems().isEmpty()) 
            throw new EmptyOrderException();
    }
    
    private void checkInventory(Order order) {
        order.getItems().forEach(this::checkItemStock);
    }
    
    private void checkItemStock(OrderItem item) {
        if (!inventory.hasStock(item.getProduct(), 
                              item.getQuantity())) {
            throw new OutOfStockException();
        }
    }
    
    private void processPayment(Order order) {
        PaymentResult result = 
            paymentService.process(order.getTotal());
        if (!result.isSuccessful()) {
            throw new PaymentFailedException();
        }
    }
    
    private void scheduleShipping(Order order) {
        shippingService.schedule(order);
    }
}
\end{lstlisting}
\end{KR}

[Continue with Testing section examples...]

\begin{example}{Unit Testing Best Practices}
\begin{lstlisting}[language=Java]
public class OrderTest {
    private Order order;
    private Product product;
    
    @BeforeEach
    void setUp() {
        order = new Order();
        product = new Product("test", new Money(10));
    }
    
    @Test
    void newOrderHasNoItems() {
        assertTrue(order.isEmpty());
    }
    
    @Test
    void addingItemIncreasesTotal() {
        order.addItem(product, 2);
        assertEquals(new Money(20), order.getTotal());
    }
    
    @Test
    void throwsExceptionForNegativeQuantity() {
        assertThrows(IllegalArgumentException.class, 
            () -> order.addItem(product, -1));
    }
    
    @Test
    void appliesDiscountCorrectly() {
        order.addItem(product, 10); // $100 total
        order.applyDiscount(0.1);   // 10% discount
        assertEquals(new Money(90), order.getTotal());
    }
}
\end{lstlisting}
\end{example}

