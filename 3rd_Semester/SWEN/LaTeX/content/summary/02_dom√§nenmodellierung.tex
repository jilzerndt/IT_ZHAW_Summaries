\section{Domänenmodellierung}

[Previous content about Domänenmodell concept remains unchanged]

\begin{theorem}{Domänenmodell Zweck}
\begin{itemize}
    \item Visualisierung der Fachdomäne für alle Stakeholder
    \item Grundlage für das spätere Softwaredesign
    \item Gemeinsames Verständnis der Begriffe und Zusammenhänge
    \item Dokumentation der fachlichen Strukturen
    \item Basis für die Kommunikation zwischen Entwicklung und Fachbereich
\end{itemize}
\end{theorem}

[Previous KR about Domänenmodell creation remains]

\begin{example}{Prüfungsaufgabe: Konzeptidentifikation}
\textbf{Aufgabentext:} 
"Ein Bibliothekssystem verwaltet Bücher, die von Mitgliedern ausgeliehen werden können. Jedes Buch hat eine ISBN und mehrere Exemplare. Mitglieder können maximal 5 Bücher gleichzeitig für 4 Wochen ausleihen. Bei Überschreitung wird eine Mahngebühr fällig."

\textbf{Identifizierte Konzepte:}
\begin{itemize}
    \item Buch (Beschreibungsklasse)
    \item Exemplar (Physisches Objekt)
    \item Mitglied (Rolle)
    \item Ausleihe (Transaktion)
    \item Mahnung (Artefakt)
\end{itemize}

\textbf{Begründung:}
\begin{itemize}
    \item Buch/Exemplar: Trennung wegen mehrfacher Exemplare (Beschreibungsmuster)
    \item Ausleihe: Verbindet Exemplar und Mitglied, hat Zeitbezug
    \item Mahnung: Entsteht bei Fristüberschreitung
\end{itemize}
\end{example}

\begin{KR}{Analysemuster Anwendung}
Systematisches Vorgehen bei der Anwendung von Analysemustern:
\begin{enumerate}
    \item \textbf{Muster identifizieren}
    \begin{itemize}
        \item Beschreibungsklassen bei gleichartigen Objekten
        \item Generalisierung bei "ist-ein"-Beziehungen
        \item Komposition bei existenzabhängigen Teilen
        \item Zustände bei Objektlebenszyklen
        \item Rollen bei verschiedenen Funktionen
        \item Assoziationsklassen bei Beziehungsattributen
        \item Wertobjekte bei komplexen Werten
    \end{itemize}
    
    \item \textbf{Muster anwenden}
    \begin{itemize}
        \item Struktur des Musters übernehmen
        \item An Kontext anpassen
        \item Konsistenz prüfen
    \end{itemize}
    
    \item \textbf{Muster kombinieren}
    \begin{itemize}
        \item Überschneidungen identifizieren
        \item Konflikte auflösen
        \item Gesamtmodell harmonisieren
    \end{itemize}
\end{enumerate}
\end{KR}

\begin{example}{Komplexes Domänenmodell: Reisebuchungssystem}
\textbf{Anforderung:} Modellieren Sie ein System für Pauschalreisen mit Flügen, Hotels und Aktivitäten.

\textbf{Verwendete Analysemuster:}
\begin{itemize}
    \item \textbf{Beschreibungsklassen:}
    \begin{itemize}
        \item Flugverbindung vs. konkreter Flug
        \item Hotelkategorie vs. konkretes Zimmer
        \item Aktivitätstyp vs. konkrete Durchführung
    \end{itemize}
    
    \item \textbf{Zustände:}
    \begin{itemize}
        \item Buchungszustände: angefragt, bestätigt, storniert
        \item Zahlungszustände: offen, teilbezahlt, vollständig
    \end{itemize}
    
    \item \textbf{Rollen:}
    \begin{itemize}
        \item Person als: Kunde, Reiseleiter, Kontaktperson
    \end{itemize}
    
    \item \textbf{Wertobjekte:}
    \begin{itemize}
        \item Geldbetrag mit Währung
        \item Zeitraum für Reisedauer
    \end{itemize}
\end{itemize}
[Hier könnte ein Beispiel-Diagramm folgen]
\end{example}

\begin{KR}{Review eines Domänenmodells}
Checkliste für die Überprüfung:
\begin{itemize}
    \item \textbf{Fachliche Korrektheit}
    \begin{itemize}
        \item Alle relevanten Konzepte vorhanden?
        \item Begriffe aus der Fachdomäne verwendet?
        \item Beziehungen fachlich sinnvoll?
    \end{itemize}
    
    \item \textbf{Technische Korrektheit}
    \begin{itemize}
        \item UML-Notation korrekt?
        \item Multiplizitäten angegeben?
        \item Assoziationsnamen vorhanden?
    \end{itemize}
    
    \item \textbf{Modellqualität}
    \begin{itemize}
        \item Angemessener Detaillierungsgrad?
        \item Analysemuster sinnvoll eingesetzt?
        \item Keine Implementation vorweggenommen?
    \end{itemize}
\end{itemize}
\end{KR}

[Previous content about common modeling errors remains]

\begin{example}{Typische Prüfungsaufgabe: Modell verbessern}
\textbf{Fehlerhaftes Modell:}
\begin{itemize}
    \item Klasse "UserManager" mit CRUD-Operationen
    \item Attribute "customerID" und "orderID" statt Assoziationen
    \item Operation "calculateTotal()" in Bestellung
    \item Technische Klasse "DatabaseConnection"
\end{itemize}

\textbf{Verbesserungen:}
\begin{itemize}
    \item "UserManager" entfernen, stattdessen Beziehungen modellieren
    \item IDs durch direkte Assoziationen ersetzen
    \item Operationen entfernen (gehören ins Design)
    \item Technische Klassen entfernen
\end{itemize}
\end{example}