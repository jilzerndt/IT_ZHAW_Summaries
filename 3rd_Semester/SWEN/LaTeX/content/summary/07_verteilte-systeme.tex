[Previous content until initial definitions remains unchanged]

\begin{example}{Prüfungsaufgabe: Architekturstil-Analyse}
\textbf{Szenario:}
Ein Messaging-System soll entwickelt werden, das folgende Anforderungen erfüllt:
\begin{itemize}
    \item Hohe Skalierbarkeit
    \item Keine zentrale Komponente (Single Point of Failure)
    \item Direkter Nachrichtenaustausch zwischen Nutzern
    \item Offline-Fähigkeit
\end{itemize}

\textbf{Analysieren Sie die Architekturstile:}

\textbf{1. Client-Server}
\begin{itemize}
    \item \textbf{Vorteile:}
    \begin{itemize}
        \item Zentrale Verwaltung
        \item Einfache Konsistenzsicherung
    \end{itemize}
    \item \textbf{Nachteile:}
    \begin{itemize}
        \item Single Point of Failure
        \item Skalierungsprobleme
    \end{itemize}
\end{itemize}

\textbf{2. Peer-to-Peer}
\begin{itemize}
    \item \textbf{Vorteile:}
    \begin{itemize}
        \item Keine zentrale Komponente
        \item Direkte Kommunikation
        \item Gute Skalierbarkeit
    \end{itemize}
    \item \textbf{Nachteile:}
    \begin{itemize}
        \item Komplexe Konsistenzsicherung
        \item Schwierige Verwaltung
    \end{itemize}
\end{itemize}

\textbf{Empfehlung:} Peer-to-Peer mit hybriden Elementen
\end{example}

\begin{KR}{Verteilungsprobleme analysieren}
\textbf{1. Probleme identifizieren}
\begin{itemize}
    \item \textbf{Netzwerk:}
    \begin{itemize}
        \item Latenz
        \item Bandbreite
        \item Ausfälle
    \end{itemize}
    \item \textbf{Daten:}
    \begin{itemize}
        \item Konsistenz
        \item Replikation
        \item Synchronisation
    \end{itemize}
    \item \textbf{System:}
    \begin{itemize}
        \item Skalierung
        \item Verfügbarkeit
        \item Wartbarkeit
    \end{itemize}
\end{itemize}

\textbf{2. Lösungsstrategien entwickeln}
\begin{itemize}
    \item \textbf{Netzwerk:}
    \begin{itemize}
        \item Caching
        \item Compression
        \item Redundanz
    \end{itemize}
    \item \textbf{Daten:}
    \begin{itemize}
        \item Eventual Consistency
        \item Master-Slave Replikation
        \item Konfliktauflösung
    \end{itemize}
    \item \textbf{System:}
    \begin{itemize}
        \item Load Balancing
        \item Service Discovery
        \item Circuit Breaker
    \end{itemize}
\end{itemize}
\end{KR}

\begin{example}{Typische Prüfungsaufgabe: CAP-Theorem}
\textbf{Aufgabe:}
Analysieren Sie für ein verteiltes Datenbanksystem die Auswirkungen des CAP-Theorems.

\textbf{CAP-Theorem Komponenten:}
\begin{itemize}
    \item \textbf{Consistency:} Alle Knoten sehen dieselben Daten
    \item \textbf{Availability:} Jede Anfrage erhält eine Antwort
    \item \textbf{Partition Tolerance:} System funktioniert trotz Netzwerkausfällen
\end{itemize}

\textbf{Analyse der Trade-offs:}
\begin{itemize}
    \item \textbf{CA-System:}
    \begin{itemize}
        \item Hohe Konsistenz und Verfügbarkeit
        \item Keine Netzwerkpartitionierung möglich
        \item Beispiel: Traditionelle RDBMS
    \end{itemize}
    \item \textbf{CP-System:}
    \begin{itemize}
        \item Konsistenz und Partitionstoleranz
        \item Eingeschränkte Verfügbarkeit
        \item Beispiel: MongoDB
    \end{itemize}
    \item \textbf{AP-System:}
    \begin{itemize}
        \item Verfügbarkeit und Partitionstoleranz
        \item Eventual Consistency
        \item Beispiel: Cassandra
    \end{itemize}
\end{itemize}
\end{example}

\begin{KR}{Verteilte System-Design}
\textbf{1. Anforderungsanalyse}
\begin{itemize}
    \item \textbf{Funktional:}
    \begin{itemize}
        \item Kernfunktionalitäten
        \item Datenmodell
        \item Schnittstellen
    \end{itemize}
    \item \textbf{Nicht-funktional:}
    \begin{itemize}
        \item Skalierbarkeit
        \item Verfügbarkeit
        \item Latenz
    \end{itemize}
\end{itemize}

\textbf{2. Architekturentscheidungen}
\begin{itemize}
    \item \textbf{Kommunikation:}
    \begin{itemize}
        \item Synchron vs. Asynchron
        \item Push vs. Pull
        \item Protokollwahl
    \end{itemize}
    \item \textbf{Datenmanagement:}
    \begin{itemize}
        \item Sharding
        \item Replikation
        \item Caching
    \end{itemize}
\end{itemize}

\textbf{3. Implementierungsaspekte}
\begin{itemize}
    \item \textbf{Fehlerbehandlung:}
    \begin{itemize}
        \item Retry-Strategien
        \item Fallbacks
        \item Monitoring
    \end{itemize}
    \item \textbf{Sicherheit:}
    \begin{itemize}
        \item Authentifizierung
        \item Verschlüsselung
        \item Autorisierung
    \end{itemize}
\end{itemize}
\end{KR}

[Previous content about middleware and error sources remains]