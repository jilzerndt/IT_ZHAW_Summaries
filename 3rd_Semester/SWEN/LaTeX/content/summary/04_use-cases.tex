\section{Use Case Realisation}

[Previous concept about Use Case Realization remains]

\begin{theorem}{Use Case Realization Ziele}
\begin{itemize}
    \item Umsetzung der fachlichen Anforderungen in Code
    \item Einhaltung der Architekturvorgaben
    \item Implementierung der GRASP-Prinzipien
    \item Erstellung wartbaren und testbaren Codes
    \item Dokumentation der Design-Entscheidungen
\end{itemize}
\end{theorem}

[Previous definition about UML remains]

\begin{example}{Analyse eines System Sequence Diagrams}
\textbf{Use Case:} Geld abheben am Bankomat

\textbf{Systemoperationen identifizieren:}
\begin{itemize}
    \item validateCard(cardNumber)
    \item verifyPIN(pin)
    \item selectAmount(amount)
    \item withdrawMoney()
    \item printReceipt()
\end{itemize}

\textbf{Operation Contract für withdrawMoney():}
\begin{itemize}
    \item \textbf{Vorbedingungen:}
    \begin{itemize}
        \item Karte validiert
        \item PIN korrekt
        \item Betrag ausgewählt
    \end{itemize}
    \item \textbf{Nachbedingungen:}
    \begin{itemize}
        \item Kontosaldo aktualisiert
        \item Transaktion protokolliert
        \item Geld ausgegeben
    \end{itemize}
\end{itemize}
\end{example}

[Previous KR about procedure remains]

\begin{KR}{Design Class Diagram (DCD) erstellen}
\textbf{1. Klassen identifizieren}
\begin{itemize}
    \item Aus Domänenmodell übernehmen
    \item Technische Klassen ergänzen
    \item Controller bestimmen
\end{itemize}

\textbf{2. Attribute definieren}
\begin{itemize}
    \item Datentypen festlegen
    \item Sichtbarkeiten bestimmen
    \item Validierungen vorsehen
\end{itemize}

\textbf{3. Methoden hinzufügen}
\begin{itemize}
    \item Systemoperationen verteilen
    \item GRASP-Prinzipien anwenden
    \item Signaturen definieren
\end{itemize}

\textbf{4. Beziehungen modellieren}
\begin{itemize}
    \item Assoziationen aus Domänenmodell
    \item Navigierbarkeit festlegen
    \item Abhängigkeiten minimieren
\end{itemize}
\end{KR}

\begin{example}{Vollständige Use Case Realization}
\textbf{Use Case:} Bestellung aufgeben

\textbf{1. Systemoperationen:}
\begin{itemize}
    \item createOrder()
    \item addItem(productId, quantity)
    \item removeItem(itemId)
    \item submitOrder()
\end{itemize}

\textbf{2. Design-Entscheidungen:}
\begin{itemize}
    \item OrderController als Fassade
    \item Order aggregiert OrderItems
    \item OrderService für Geschäftslogik
    \item Repository für Persistenz
\end{itemize}

\textbf{3. GRASP-Anwendung:}
\begin{itemize}
    \item Information Expert:
    \begin{itemize}
        \item Order berechnet Gesamtsumme
        \item OrderItem verwaltet Produktdaten
    \end{itemize}
    \item Creator:
    \begin{itemize}
        \item Order erstellt OrderItems
        \item OrderService erstellt Orders
    \end{itemize}
    \item Low Coupling:
    \begin{itemize}
        \item Repository-Interface für Persistenz
        \item Service-Interface für Geschäftslogik
    \end{itemize}
\end{itemize}

\textbf{4. Implementierung:}
\begin{lstlisting}[language=Java, style=basesmol]
public class OrderController {
    private OrderService orderService;
    private Order currentOrder;
    
    public void createOrder() {
        currentOrder = orderService.createOrder();
    }
    
    public void addItem(String productId, int quantity) {
        currentOrder.addItem(productId, quantity);
    }
    
    public void submitOrder() {
        orderService.submitOrder(currentOrder);
    }
}
\end{lstlisting}
\end{example}

\begin{KR}{Implementierung prüfen}
\textbf{1. Funktionale Prüfung}
\begin{itemize}
    \item Use Case Szenarien durchspielen
    \item Randfälle testen
    \item Fehlersituationen prüfen
\end{itemize}

\textbf{2. Strukturelle Prüfung}
\begin{itemize}
    \item Architekturkonformität
    \item GRASP-Prinzipien
    \item Clean Code Regeln
\end{itemize}

\textbf{3. Qualitätsprüfung}
\begin{itemize}
    \item Testabdeckung
    \item Wartbarkeit
    \item Performance
\end{itemize}
\end{KR}

\begin{example}{Typische Prüfungsaufgabe}
\textbf{Aufgabe:} 
Gegeben ist folgender Use Case: "Kunde meldet sich an". Erstellen Sie:
\begin{itemize}
    \item System Sequence Diagram
    \item Operation Contracts
    \item Design Class Diagram
    \item Implementierung der wichtigsten Methoden
\end{itemize}

\textbf{Bewertungskriterien:}
\begin{itemize}
    \item Vollständigkeit der Modellierung
    \item Korrekte Anwendung der GRASP-Prinzipien
    \item Sinnvolle Verteilung der Verantwortlichkeiten
    \item Testbare Implementierung
\end{itemize}
\end{example}

[Previous KR about implementation errors remains]