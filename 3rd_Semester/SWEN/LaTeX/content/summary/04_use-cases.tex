\begin{example2}{GRASP-basierte Implementation}
\textbf{Szenario:} Implementierung einer Bestellverwaltung mit GRASP-Prinzipien

\textbf{Information Expert:}
\begin{lstlisting}[language=Java, style=basesmol]
public class Order {
    private List<OrderItem> items = new ArrayList<>();
    
    public BigDecimal calculateTotal() {
        return items.stream()
                   .map(OrderItem::getSubtotal)
                   .reduce(BigDecimal.ZERO, BigDecimal::add);
    }
}
\end{lstlisting}

\textbf{Creator:}
\begin{lstlisting}[language=Java, style=basesmol]
public class Order {
    public OrderItem createOrderItem(Product product, int quantity) {
        OrderItem item = new OrderItem(product, quantity);
        items.add(item);
        return item;
    }
}
\end{lstlisting}

\textbf{Controller:}
\begin{lstlisting}[language=Java, style=basesmol]
public class OrderController {
    private OrderService orderService;
    
    public OrderDTO createOrder(String customerId) {
        Order order = orderService.initializeOrder(customerId);
        return OrderMapper.toDTO(order);
    }
}
\end{lstlisting}
\end{example2}

\begin{KR}{Use Case Realization Dokumentation}
\textbf{1. Analysephase}
\begin{itemize}
    \item Use Case und Systemoperationen dokumentieren
    \item Domänenmodell-Ausschnitt zeigen
    \item Relevante Anforderungen auflisten
\end{itemize}

\textbf{2. Design}
\begin{itemize}
    \item Design Class Diagram erstellen
    \item Sequenzdiagramme für komplexe Abläufe
    \item GRASP-Prinzipien begründen
\end{itemize}

\textbf{3. Implementation}
\begin{itemize}
    \item Code-Struktur dokumentieren
    \item Wichtige Algorithmen erläutern
    \item Test-Strategie beschreiben
\end{itemize}
\end{KR}

\begin{example2}{Design Decisions Documentation}
\textbf{Use Case:} Warenkorb verwalten

\textbf{Design Entscheidungen:}
\begin{itemize}
    \item \textbf{Pattern: Repository}
    \begin{itemize}
        \item Grund: Abstraktion der Datenpersistenz
        \item Alternative: DAO Pattern verworfen
    \end{itemize}
    
    \item \textbf{Pattern: Factory}
    \begin{itemize}
        \item Grund: Komplexe Objekterstellung
        \item Vorteil: Testbarkeit verbessert
    \end{itemize}
\end{itemize}

\textbf{Implementation:}
\begin{lstlisting}[language=Java, style=basesmol]
public class CartFactory {
    public ShoppingCart createCart(String customerId) {
        ShoppingCart cart = new ShoppingCart(customerId);
        cart.setCreatedAt(LocalDateTime.now());
        cart.setStatus(CartStatus.NEW);
        return cart;
    }
}
\end{lstlisting}
\end{example2}

\begin{KR}{Testing in Use Case Realization}\\
\textbf{1. Unit Tests}
\begin{itemize}
    \item Einzelne Klassen testen
    \item Mocks für Abhängigkeiten
    \item Edge Cases abdecken
\end{itemize}

\textbf{2. Integration Tests}
\begin{itemize}
    \item Zusammenspiel der Komponenten
    \item Use Case Szenarien testen
    \item Fehlerszenarien prüfen
\end{itemize}

\textbf{Beispiel Test:}
\begin{lstlisting}[language=Java, style=basesmol]
@Test
public void shouldCalculateCartTotal() {
    // Given
    ShoppingCart cart = new ShoppingCart("customer123");
    cart.addItem(new Product("p1", "Book", 29.99), 2);
    cart.addItem(new Product("p2", "DVD", 19.99), 1);
    
    // When
    BigDecimal total = cart.calculateTotal();
    
    // Then
    assertEquals(new BigDecimal("79.97"), total);
}
\end{lstlisting}
\end{KR}

\begin{example2}{Kompletter Use Case Realization Prozess}
\textbf{Use Case:} Benutzerregistrierung

\textbf{1. System Sequence Diagram}
% add SSD image for user registration

\textbf{2. Operation Contract}
\begin{itemize}
    \item \textbf{Operation:} registerUser(userDTO)
    \item \textbf{Vorbedingungen:} Benutzer nicht registriert
    \item \textbf{Nachbedingungen:} 
    \begin{itemize}
        \item Benutzer erstellt
        \item Bestätigungsmail versendet
    \end{itemize}
\end{itemize}

\textbf{3. Implementation:}
\begin{lstlisting}[language=Java, style=basesmol]
public class UserRegistrationService {
    private UserRepository userRepo;
    private EmailService emailService;
    
    public User registerUser(UserDTO dto) {
        // Validate input
        validateUserInput(dto);
        
        // Create user
        User user = new User();
        user.setEmail(dto.getEmail());
        user.setPassword(passwordEncoder.encode(dto.getPassword()));
        
        // Save user
        user = userRepo.save(user);
        
        // Send confirmation
        emailService.sendConfirmation(user.getEmail());
        
        return user;
    }
}
\end{lstlisting}
\end{example2}

\section{Use Case Realisation old}

\begin{concept}{Use Case Realization}\\
Die Umsetzung von Use Cases erfolgt durch:
\begin{itemize}
    \item Detaillierte Szenarien aus den Use Cases
    \item Systemantworten müssen realisiert werden
    \item UI statt System im SSD
    \item Systemoperationen sind die zu implementierenden Elemente
\end{itemize}
\end{concept}

\begin{theorem}{Use Case Realization Ziele}
\begin{itemize}
    \item Umsetzung der fachlichen Anforderungen in Code
    \item Einhaltung der Architekturvorgaben
    \item Implementierung der GRASP-Prinzipien
    \item Erstellung wartbaren und testbaren Codes
    \item Dokumentation der Design-Entscheidungen
\end{itemize}
\end{theorem}

\begin{definition}{UML im Implementierungsprozess}\\
UML dient als:
\begin{itemize}
    \item Zwischenschritt bei wenig Erfahrung
    \item Kompakter Ersatz für Programmiercode
    \item Kommunikationsmittel (auch für Nicht-Techniker)
\end{itemize}
\end{definition}

\begin{example2}{Analyse eines System Sequence Diagrams}
\textbf{Use Case:} Geld abheben am Bankomat

\textbf{Systemoperationen identifizieren:}
\begin{itemize}
    \item validateCard(cardNumber)
    \item verifyPIN(pin)
    \item selectAmount(amount)
    \item withdrawMoney()
    \item printReceipt()
\end{itemize}

\textbf{Operation Contract für withdrawMoney():}
\begin{itemize}
    \item \textbf{Vorbedingungen:}
    \begin{itemize}
        \item Karte validiert
        \item PIN korrekt
        \item Betrag ausgewählt
    \end{itemize}
    \item \textbf{Nachbedingungen:}
    \begin{itemize}
        \item Kontosaldo aktualisiert
        \item Transaktion protokolliert
        \item Geld ausgegeben
    \end{itemize}
\end{itemize}
\end{example2}

\begin{KR}{Vorgehen bei der Use Case Realization}\\
\textbf{1. Vorbereitung:}
\begin{itemize}
    \item Use Case auswählen und SSD ableiten
    \item Systemoperation identifizieren
    \item Operation Contract erstellen/prüfen
\end{itemize}

\textbf{2. Analyse:}
\begin{itemize}
    \item Aktuellen Code/Dokumentation analysieren
    \item DCD überprüfen/aktualisieren
    \item Vergleich mit Domänenmodell
    \item Neue Klassen gemäß Domänenmodell erstellen
\end{itemize}

\textbf{3. Realisierung:}
\begin{itemize}
    \item Controller Klasse bestimmen
    \item Zu verändernde Klassen festlegen
    \item Weg zu diesen Klassen festlegen:
    \begin{itemize}
        \item Parameter für Wege definieren
        \item Klassen bei Bedarf erstellen
        \item Verantwortlichkeiten zuweisen
        \item Verschiedene Varianten evaluieren
    \end{itemize}
    \item Veränderungen implementieren
    \item Review durchführen
\end{itemize}
\end{KR}

\begin{example2}{Use Case Realization: Verkauf abwickeln}
\textbf{1. Vorbereitung:}
\begin{itemize}
    \item \textbf{Use Case:} Verkauf abwickeln
    \item \textbf{Systemoperation:} makeNewSale()
    \item \textbf{Contract:} Neue Sale-Instanz wird erstellt
\end{itemize}

\textbf{2. Analyse:}
\begin{itemize}
    \item \textbf{Klassen:} Register, Sale
    \item \textbf{DCD:} Beziehung Register-Sale prüfen
    \item \textbf{Neue Klassen:} Payment, SaleLineItem
\end{itemize}

\textbf{3. Implementierung:}
\begin{itemize}
    \item Register als Controller
    \item Sale-Klasse erweitern
    \item Beziehungen implementieren
\end{itemize}
\end{example2}

\begin{KR}{Design Class Diagram (DCD) erstellen}
\textbf{1. Klassen identifizieren}
\begin{itemize}
    \item Aus Domänenmodell übernehmen
    \item Technische Klassen ergänzen
    \item Controller bestimmen
\end{itemize}

\textbf{2. Attribute definieren}
\begin{itemize}
    \item Datentypen festlegen
    \item Sichtbarkeiten bestimmen
    \item Validierungen vorsehen
\end{itemize}

\textbf{3. Methoden hinzufügen}
\begin{itemize}
    \item Systemoperationen verteilen
    \item GRASP-Prinzipien anwenden
    \item Signaturen definieren
\end{itemize}

\textbf{4. Beziehungen modellieren}
\begin{itemize}
    \item Assoziationen aus Domänenmodell
    \item Navigierbarkeit festlegen
    \item Abhängigkeiten minimieren
\end{itemize}
\end{KR}

\begin{example2}{Vollständige Use Case Realization}
\textbf{Use Case:} Bestellung aufgeben

\textbf{1. Systemoperationen:}
\begin{itemize}
    \item createOrder()
    \item addItem(productId, quantity)
    \item removeItem(itemId)
    \item submitOrder()
\end{itemize}

\textbf{2. Design-Entscheidungen:}
\begin{itemize}
    \item OrderController als Fassade
    \item Order aggregiert OrderItems
    \item OrderService für Geschäftslogik
    \item Repository für Persistenz
\end{itemize}

\textbf{3. GRASP-Anwendung:}
\begin{itemize}
    \item Information Expert:
    \begin{itemize}
        \item Order berechnet Gesamtsumme
        \item OrderItem verwaltet Produktdaten
    \end{itemize}
    \item Creator:
    \begin{itemize}
        \item Order erstellt OrderItems
        \item OrderService erstellt Orders
    \end{itemize}
    \item Low Coupling:
    \begin{itemize}
        \item Repository-Interface für Persistenz
        \item Service-Interface für Geschäftslogik
    \end{itemize}
\end{itemize}

\textbf{4. Implementierung:}
\begin{lstlisting}[language=Java, style=basesmol]
public class OrderController {
    private OrderService orderService;
    private Order currentOrder;
    
    public void createOrder() {
        currentOrder = orderService.createOrder();
    }
    
    public void addItem(String productId, int quantity) {
        currentOrder.addItem(productId, quantity);
    }
    
    public void submitOrder() {
        orderService.submitOrder(currentOrder);
    }
}
\end{lstlisting}
\end{example2}

\begin{KR}{Implementierung prüfen}
\textbf{1. Funktionale Prüfung}
\begin{itemize}
    \item Use Case Szenarien durchspielen
    \item Randfälle testen
    \item Fehlersituationen prüfen
\end{itemize}

\textbf{2. Strukturelle Prüfung}
\begin{itemize}
    \item Architekturkonformität
    \item GRASP-Prinzipien
    \item Clean Code Regeln
\end{itemize}

\textbf{3. Qualitätsprüfung}
\begin{itemize}
    \item Testabdeckung
    \item Wartbarkeit
    \item Performance
\end{itemize}
\end{KR}

\begin{example2}{Typische Prüfungsaufgabe}
\textbf{Aufgabe:} 
Gegeben ist folgender Use Case: "Kunde meldet sich an". Erstellen Sie:
\begin{itemize}
    \item System Sequence Diagram
    \item Operation Contracts
    \item Design Class Diagram
    \item Implementierung der wichtigsten Methoden
\end{itemize}

\textbf{Bewertungskriterien:}
\begin{itemize}
    \item Vollständigkeit der Modellierung
    \item Korrekte Anwendung der GRASP-Prinzipien
    \item Sinnvolle Verteilung der Verantwortlichkeiten
    \item Testbare Implementierung
\end{itemize}
\end{example2}

\begin{KR}{Typische Implementierungsfehler vermeiden}\\
\begin{itemize}
    \item \textbf{Architekturverletzungen:}
    \begin{itemize}
        \item Schichtentrennung beachten
        \item Abhängigkeiten richtig setzen
    \end{itemize}
    
    \item \textbf{GRASP-Verletzungen:}
    \begin{itemize}
        \item Information Expert beachten
        \item Creator Pattern richtig anwenden
        \item High Cohesion erhalten
    \end{itemize}
    
    \item \textbf{Testbarkeit:}
    \begin{itemize}
        \item Klassen isoliert testbar halten
        \item Abhängigkeiten mockbar gestalten
    \end{itemize}
\end{itemize}
\end{KR}



