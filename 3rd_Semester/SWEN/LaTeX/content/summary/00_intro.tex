\section{Einführung und Überblick}

\begin{definition}{Software Engineering}
\begin{itemize}
    \item Disziplinen: Anforderungen, Architektur, Implementierung, Test und Wartung.
    \item Ziel: Strukturierte Prozesse für Qualität, Risiko- und Fehlerminimierung.
    \item «Zielorientierte Bereitstellung und systematische Verwendung von Prinzipien, Methoden und Werkzeugen für die arbeitsteilige, ingenieurmäßige Entwicklung und Anwendung von umfangreichen Softwaresystemen.» (H. Balzert)
    \item Aufgrund des hohen Aufwandes zur Erstellung und Wartung komplexer Software erfolgt die Entwicklung durch Softwareentwickler anhand eines strukturierten (Projekt-)Planes.
\end{itemize}
\end{definition}

\begin{definition}{Modellierung in der Softwareentwicklung}
\begin{itemize}
    \item Modelle als Abstraktionen: Anforderungen, Architekturen, Testfälle.
    \item Einsatz von UML: Skizzen, detaillierte Blueprints, vollständige Spezifikationen.
    \item Zweck:
    \begin{itemize}
        \item Verstehen eines Gebildes
        \item Kommunizieren über ein Gebilde
        \item Gedankliches Hilfsmittel zum Gestalten, Bewerten oder Kritisieren
        \item Spezifikation von Anforderungen
        \item Durchführung von Experimenten
    \end{itemize}
\end{itemize}
\end{definition}

\begin{KR}{Modellierungsumfang bestimmen}
Folgende Fragen zur Bestimmung des notwendigen Modellierungsumfangs:
\begin{itemize}
    \item Wie komplex ist die Problemstellung?
    \item Wie viele Stakeholder sind involviert?
    \item Wie kritisch ist das System?
    \item Analogie: Planung einer Hundehütte vs. Haus vs. Wolkenkratzer
\end{itemize}
\end{KR}

\begin{example}{Prüfungsfrage zur Modellierung}
Erklären Sie anhand eines selbst gewählten Beispiels, warum der Modellierungsaufwand je nach Projekt stark variieren kann. Nennen Sie mindestens drei Faktoren, die den Modellierungsumfang beeinflussen.

Mögliche Antwort:
\begin{itemize}
    \item Beispiel: Entwicklung einer Smartphone-App vs. Medizinisches Gerät
    \item Faktoren:
    \begin{itemize}
        \item Komplexität der Domäne
        \item Regulatorische Anforderungen
        \item Anzahl beteiligter Stakeholder
        \item Sicherheitsanforderungen
    \end{itemize}
\end{itemize}
\end{example}

[Previous content continues...]

\begin{concept}{Charakteristiken iterativ-inkrementeller Prozesse}
\begin{itemize}
    \item Projekt-Abwicklung in Iterationen (Mini-Projekte)
    \item In jeder Iteration wird ein Stück der Software entwickelt (Inkrement)
    \item Ziele der Iterationen sind Risiko-getrieben
    \item Iterationen werden reviewed und die Learnings fliessen in die nächsten Iterationen ein
    \item Demming-Cycle: Plan, Do, Check, Act
\end{itemize}
\end{concept}

[Rest of previous content...]

\begin{example}{Typische Prüfungsaufgabe: Prozessmodelle vergleichen}
Vergleichen Sie das Wasserfallmodell mit einem iterativ-inkrementellen Ansatz anhand folgender Kriterien:
\begin{itemize}
    \item Umgang mit sich ändernden Anforderungen
    \item Risikomanagement
    \item Planbarkeit
    \item Kundeneinbindung
\end{itemize}

Musterlösung:
\begin{itemize}
    \item Wasserfall:
    \begin{itemize}
        \item Änderungen schwierig zu integrieren
        \item Risiken erst spät erkennbar
        \item Gut planbar durch feste Phasen
        \item Kunde hauptsächlich am Anfang und Ende involviert
    \end{itemize}
    \item Iterativ-inkrementell:
    \begin{itemize}
        \item Flexibel bei Änderungen
        \item Frühes Erkennen von Risiken
        \item Planung pro Iteration
        \item Kontinuierliches Kundenfeedback
    \end{itemize}
\end{itemize}
\end{example}

[Previous formulas and diagrams remain as they were...]