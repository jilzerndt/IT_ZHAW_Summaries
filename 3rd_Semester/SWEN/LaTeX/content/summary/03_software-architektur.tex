\section{Softwarearchitektur und Design}

\subsection{Beispiele}

\subsubsection{Beispiele Architekturentwurf}

\begin{example2}{Typische Prüfungsaufgabe: Architekturanalyse und Entscheidungen}\\
\textbf{Aufgabenstellung:}
Analysieren Sie folgende Anforderungen und leiten Sie architektonische Konsequenzen ab:
\begin{itemize}
    \item System muss 24/7 verfügbar sein
    \item 10.000 gleichzeitige Benutzer
    \item Reaktionszeit unter 1 Sekunde
    \item Jährliche Wartungsfenster maximal 4 Stunden
\end{itemize}

\textbf{Lösung:}
\begin{itemize}
    \item \textbf{Architekturentscheidungen:}
    \begin{itemize}
        \item Verteilte Architektur für Hochverfügbarkeit
        \item Load Balancing für gleichzeitige Benutzer
        \item Caching-Strategien für Performanz
        \item Blue-Green Deployment für Wartung
    \end{itemize}
    
    \item \textbf{Begründungen:}
    \begin{itemize}
        \item Verteilung minimiert Single Points of Failure
        \item Load Balancer verteilt Last gleichmäßig
        \item Caching reduziert Datenbankzugriffe
        \item Blue-Green erlaubt Updates ohne Downtime
    \end{itemize}
\end{itemize}
\end{example2}

\begin{example2}{Architekturentwurf}\\
\textbf{Aufgabe:} Entwerfen Sie die grundlegende Architektur für ein Online-Banking-System.

\textbf{Lösung:}
\begin{itemize}
    \item \textbf{Anforderungsanalyse:}
    \begin{itemize}
        \item Sicherheit (ISO 25010)
        \item Performance (FURPS+)
        \item Skalierbarkeit
    \end{itemize}
    
    \item \textbf{Architekturentscheidungen:}
    \begin{itemize}
        \item Mehrschichtige Architektur
        \item Microservices für Skalierbarkeit
        \item Sicherheitsschicht
    \end{itemize}
    
    \item \textbf{Module:}
    \begin{itemize}
        \item Authentifizierung
        \item Transaktionen
        \item Kontoführung
    \end{itemize}
\end{itemize}
\end{example2}

\begin{example2}{Architektur-Evaluation: Performance}\\
\textbf{Szenario:} Online-Shop während Black Friday

\textbf{Analyse:}
\begin{itemize}
    \item \textbf{Last-Annahmen:}
    \begin{itemize}
        \item 10.000 gleichzeitige Nutzer
        \item 1.000 Bestellungen pro Minute
        \item 100.000 Produktaufrufe pro Minute
    \end{itemize}
    
    \item \textbf{Architektur-Maßnahmen:}
    \begin{itemize}
        \item Caching-Strategie für Produkte
        \item Load Balancing für Anfragen
        \item Asynchrone Bestellverarbeitung
        \item Datenbank-Replikation
    \end{itemize}
    
    \item \textbf{Monitoring:}
    \begin{itemize}
        \item Response-Zeiten
        \item Server-Auslastung
        \item Cache-Hit-Rate
        \item Fehlerraten
    \end{itemize}
\end{itemize}

\begin{lstlisting}[language=Java, style=basesmol]
// Performance-optimierte Produktabfrage
@Cacheable(value = "products")
public ProductDTO getProduct(String id) {
    ProductDTO product = cache.get(id);
    if (product == null) {
        product = repository.findById(id)
                          .map(this::toDTO)
                          .orElseThrow();
        cache.put(id, product);
    }
    return product;
}
\end{lstlisting}
\end{example2}




