[Previous content remains unchanged until after the initial concepts]

\begin{KR}{Architekturentscheidungen treffen}
Systematischer Ansatz für Architekturentscheidungen:
\begin{enumerate}
    \item \textbf{Anforderungen analysieren}
    \begin{itemize}
        \item Funktionale Anforderungen gruppieren
        \item Nicht-funktionale Anforderungen priorisieren
        \item Randbedingungen identifizieren
    \end{itemize}
    
    \item \textbf{Einflussfaktoren bewerten}
    \begin{itemize}
        \item Technische Faktoren
        \item Organisatorische Faktoren
        \item Wirtschaftliche Faktoren
    \end{itemize}
    
    \item \textbf{Alternativen evaluieren}
    \begin{itemize}
        \item Vor- und Nachteile abwägen
        \item Proof of Concepts durchführen
        \item Risiken analysieren
    \end{itemize}
    
    \item \textbf{Entscheidung dokumentieren}
    \begin{itemize}
        \item Begründung festhalten
        \item Verworfene Alternativen dokumentieren
        \item Annahmen dokumentieren
    \end{itemize}
\end{enumerate}
\end{KR}

\begin{example}{Typische Prüfungsaufgabe: Architekturanalyse}
\textbf{Aufgabenstellung:}
Analysieren Sie folgende Anforderungen und leiten Sie architektonische Konsequenzen ab:
\begin{itemize}
    \item System muss 24/7 verfügbar sein
    \item 10.000 gleichzeitige Benutzer
    \item Reaktionszeit unter 1 Sekunde
    \item Jährliche Wartungsfenster maximal 4 Stunden
\end{itemize}

\textbf{Lösung:}
\begin{itemize}
    \item \textbf{Architekturentscheidungen:}
    \begin{itemize}
        \item Verteilte Architektur für Hochverfügbarkeit
        \item Load Balancing für gleichzeitige Benutzer
        \item Caching-Strategien für Performanz
        \item Blue-Green Deployment für Wartung
    \end{itemize}
    
    \item \textbf{Begründungen:}
    \begin{itemize}
        \item Verteilung minimiert Single Points of Failure
        \item Load Balancer verteilt Last gleichmäßig
        \item Caching reduziert Datenbankzugriffe
        \item Blue-Green erlaubt Updates ohne Downtime
    \end{itemize}
\end{itemize}
\end{example}

[Previous content about N+1 View Model remains]

\begin{KR}{UML Diagrammauswahl}
Entscheidungshilfe für die Wahl des UML-Diagrammtyps:

\textbf{1. Strukturbeschreibung benötigt:}
\begin{itemize}
    \item Klassendiagramm für Typen und Beziehungen
    \item Paketdiagramm für Modularisierung
    \item Komponentendiagramm für Bausteinsicht
    \item Verteilungsdiagramm für Deployment
\end{itemize}

\textbf{2. Verhaltensbeschreibung benötigt:}
\begin{itemize}
    \item Sequenzdiagramm für Interaktionsabläufe
    \item Aktivitätsdiagramm für Workflows
    \item Zustandsdiagramm für Objektlebenszyklen
    \item Kommunikationsdiagramm für Objektkollaborationen
\end{itemize}

\textbf{3. Abstraktionsebene wählen:}
\begin{itemize}
    \item Analyse: Konzeptuelle Diagramme
    \item Design: Detaillierte Spezifikation
    \item Implementation: Codenahes Design
\end{itemize}
\end{KR}

\begin{example}{GRASP Anwendung}
\textbf{Szenario:} Online-Shop Warenkorb-Funktionalität

\textbf{GRASP-Prinzipien angewandt:}
\begin{itemize}
    \item \textbf{Information Expert:}
    \begin{itemize}
        \item Warenkorb kennt seine Positionen
        \item Berechnet selbst Gesamtsumme
    \end{itemize}
    
    \item \textbf{Creator:}
    \begin{itemize}
        \item Warenkorb erstellt Warenkorbpositionen
        \item Bestellung erstellt aus Warenkorb
    \end{itemize}
    
    \item \textbf{Controller:}
    \begin{itemize}
        \item ShoppingController koordiniert UI und Domain
        \item Keine Geschäftslogik im Controller
    \end{itemize}
    
    \item \textbf{Low Coupling:}
    \begin{itemize}
        \item UI kennt nur Controller
        \item Domain unabhängig von UI
    \end{itemize}
\end{itemize}
\end{example}

\begin{KR}{Architektur-Review durchführen}
\textbf{Vorgehen:}
\begin{enumerate}
    \item \textbf{Vorbereitung}
    \begin{itemize}
        \item Architektur-Dokumentation zusammenstellen
        \item Review-Team zusammenstellen
        \item Checklisten vorbereiten
    \end{itemize}
    
    \item \textbf{Durchführung}
    \begin{itemize}
        \item Architektur vorstellen
        \item Anforderungen prüfen
        \item Entscheidungen hinterfragen
        \item Risiken identifizieren
    \end{itemize}
    
    \item \textbf{Nachbereitung}
    \begin{itemize}
        \item Findings dokumentieren
        \item Maßnahmen definieren
        \item Follow-up planen
    \end{itemize}
\end{enumerate}

\textbf{Prüfkriterien:}
\begin{itemize}
    \item Anforderungserfüllung
    \item Technische Machbarkeit
    \item Zukunftssicherheit
    \item Best Practices
\end{itemize}
\end{KR}

[Previous examples and definitions remain]

\begin{example}{Prüfungsaufgabe: UML-Modellierung}
\textbf{Aufgabe:} 
Modellieren Sie für ein Bibliothekssystem die Ausleihe eines Buches mit:
\begin{itemize}
    \item Klassendiagramm der beteiligten Klassen
    \item Sequenzdiagramm des Ausleihvorgangs
    \item Zustandsdiagramm für ein Buchexemplar
\end{itemize}

\textbf{Bewertungskriterien:}
\begin{itemize}
    \item Korrekte UML-Notation
    \item Vollständigkeit der Modellierung
    \item Konsistenz zwischen Diagrammen
    \item Angemessener Detaillierungsgrad
\end{itemize}
\end{example}