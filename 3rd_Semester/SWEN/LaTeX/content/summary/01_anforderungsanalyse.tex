\section{Anforderungsanalyse}

\subsubsection{Usability und User Experience}

\begin{concept}{Usability und User Experience}\\
Die drei Säulen der Benutzererfahrung:
\begin{itemize}
    \item \textbf{Usability (Gebrauchstauglichkeit):} \\ Grundlegende Nutzbarkeit des Systems
    \item \textbf{User Experience:} Usability + Desirability (Attraktivität)
    \item \textbf{Customer Experience:} \\ UX + Brand Experience (Markenwahrnehmung)
\end{itemize}
\includegraphics[width=0.9\linewidth]{images/2024_12_29_0d1d7b5551ea1b4b41bdg-02}
\end{concept}

\begin{definition}{Usability-Dimensionen}\\
Die drei Hauptdimensionen der Usability:
\begin{itemize}
    \item \textbf{Effektivität:}
    \begin{itemize}
        \item Vollständige Aufgabenerfüllung
        \item Gewünschte Genauigkeit
    \end{itemize}
    \end{itemize}

\begin{minipage}{0.5\linewidth}
    \begin{itemize}
    \item \textbf{Effizienz:} Minimaler Aufwand
    \begin{itemize}
        \item Mental
        \item Physisch
        \item Zeitlich
    \end{itemize}
    \end{itemize}
\end{minipage}
\begin{minipage}{0.5\linewidth}
    \begin{itemize}
    \item \textbf{Zufriedenheit:}
    \begin{itemize}
        \item Minimum: Keine Verärgerung
        \item Standard: Zufriedenheit
        \item Optimal: Begeisterung
    \end{itemize}
\end{itemize}
\end{minipage}
\end{definition}

\begin{theorem}{ISO 9241-110: Usability-Anforderungen}

    \begin{minipage}{0.45\linewidth}
        Die sieben Grundprinzipien:
        \begin{itemize}
            \item Aufgabenangemessenheit
            \item Selbstbeschreibungsfähigkeit
            \item Steuerbarkeit
        \end{itemize}
    \end{minipage}
    \begin{minipage}{0.45\linewidth}
        \begin{itemize}
            \item Erwartungskonformität
            \item Fehlertoleranz
            \item Individualisierbarkeit
            \item Lernförderlichkeit
        \end{itemize}
    \end{minipage}
\end{theorem}

\subsubsection{User-Centered Design}

\begin{concept}{UCD (User-Centered Design)}

\begin{minipage}{0.3\linewidth}
Ein iterativer Prozess \\ zur nutzerzentrierten \\ Entwicklung
\end{minipage}
\begin{minipage}{0.65\linewidth}
    \includegraphics[width=\linewidth]{images/2024_12_29_0d1d7b5551ea1b4b41bdg-03}
    \end{minipage}
\end{concept}

\begin{theorem}{Wichtige Artefakte}
\begin{itemize}
    \item Personas: Repräsentative Nutzerprofile
    \item Usage-Szenarien: Konkrete Anwendungsfälle
    \item Mentales Modell: Nutzerverständnis
    \item Domänenmodell: Fachliches Verständnis
    \item Service Blueprint: Geschäftsprozessmodell
    \item Stakeholder Map: Beteiligte und Betroffene
    \item UI-Artefakte: Skizzen, Wireframes, Designs
\end{itemize}
\end{theorem}

\begin{KR}{User Research durchführen}\\
Systematisches Vorgehen für User und Domain Research:
\begin{enumerate}
    \item Zielgruppe identifizieren
    \begin{itemize}
        \item Wer sind die Benutzer?
        \item Was sind ihre Aufgaben/Ziele?
        \item Wie sieht ihre Arbeitsumgebung aus?
    \end{itemize}

    \item Daten sammeln durch
    \begin{itemize}
        \item Contextual Inquiry
        \item Interviews
        \item Beobachtung
        \item Fokusgruppen
        \item Nutzungsauswertung
    \end{itemize}
    \item Ergebnisse dokumentieren in
    \begin{itemize}
        \item Personas
        \item Usage-Szenarien
        \item Mentales Modell
    \end{itemize}
\end{enumerate}
\end{KR}

\begin{example2}{Persona erstellen}\\
\textbf{Aufgabe:} Erstellen Sie eine Persona für ein Online-Banking-System.

\textbf{Lösung:} 
\textbf{Sarah Schmidt, 34, Projektmanagerin}
\begin{itemize}
    \item \textbf{Hintergrund:}
    \begin{itemize}
        \item Arbeitet Vollzeit in IT-Firma
        \item Technik-affin, aber keine Expertin
        \item Nutzt Smartphone für die meisten Aufgaben
    \end{itemize}
    \item \textbf{Ziele:}
    \begin{itemize}
        \item Schnelle Überweisungen zwischen Konten
        \item Überblick über Ausgaben
        \item Sichere Authentifizierung
    \end{itemize}
    \item \textbf{Frustrationen:}
    \begin{itemize}
        \item Komplexe Menüführung
        \item Lange Ladezeiten
        \item Mehrfache Login-Prozesse
    \end{itemize}
\end{itemize}
\end{example2}

\begin{example2}{Stakeholder Map}\\
Zeigt die wichtigsten Stakeholder im Umfeld der Problemdomäne.\\
\includegraphics[width=0.8\linewidth]{images/stakeholdermap.png}
\end{example2}

\subsubsection{Requirements Engineering}

\begin{definition}{Requirements (Anforderungen)}
\begin{itemize}
    \item Leistungsfähigkeiten oder Eigenschaften
    \item Explizit oder implizit
    \item Müssen mit allen Stakeholdern erarbeitet werden
    \item Entwickeln sich während des Projekts
\end{itemize}
\includegraphics[width=\linewidth]{images/user_anforderungen.png}
\end{definition}

\begin{formula}{Anforderungsarten}\\
\textbf{Funktionale Anforderungen:}
\begin{itemize}
    \item Was soll das System tun?
    \item Dokumentiert in Use Cases
    \item Messbar und testbar
\end{itemize}

\textbf{Nicht-funktionale Anforderungen:}
\begin{itemize}
    \item Wie soll das System sein?
    \item Performance, Sicherheit, Usability
    \item Nach ISO 25010 kategorisiert
\end{itemize}

\textbf{Randbedingungen:}
\begin{itemize}
    \item Technische Einschränkungen
    \item Gesetzliche Vorgaben
    \item Budget und Zeitrahmen
\end{itemize}
\end{formula}


\columnbreak

\subsection{Use Cases}

\begin{definition}{Use Case (Anwendungsfall)}\\
Textuelle Beschreibung einer konkreten Interaktion zwischen Akteur und System:
\begin{itemize}
    \item Aus Sicht des Akteurs
    \item Aktiv formuliert
    \item Konkreter Nutzen
    \item Essentieller Stil (Logik statt Implementierung)
\end{itemize}
\end{definition}

\begin{theorem}{Akteure in Use Cases}
\begin{itemize}
    \item \textbf{Primärakteur:} Initiiert den Use Case, erhält Hauptnutzen
    \item \textbf{Unterstützender Akteur:} Hilft bei der Durchführung
    \item \textbf{Offstage-Akteur:} Indirekt beteiligter Stakeholder
\end{itemize}
\end{theorem}

\begin{KR}{Use Case Erstellung}\\
Schritte zur Erstellung eines vollständigen Use Cases:
\begin{enumerate}
    \item \textbf{Identifikation:}
    \begin{itemize}
        \item Systemgrenzen definieren
        \item Primärakteure identifizieren
        \item Ziele der Akteure ermitteln
    \end{itemize}
    \item \textbf{Dokumentation:}
    \begin{itemize}
        \item Brief/Casual für erste Analyse
        \item Fully-dressed für wichtige Use Cases
        \item Standardablauf und Erweiterungen
    \end{itemize}
    \item \textbf{Review:}
    \begin{itemize}
        \item Mit Stakeholdern abstimmen
        \item Auf Vollständigkeit prüfen
        \item Konsistenz sicherstellen
    \end{itemize}
\end{enumerate}
\end{KR}

\begin{example2}{Typische Prüfungsaufgabe: Use Case Analyse}\\
\textbf{Aufgabe:} Analysieren Sie den folgenden Use Case und identifizieren Sie mögliche Probleme:

\textbf{Use Case:} "Der Benutzer loggt sich ein und das System zeigt die Startseite. Er klickt auf den Button und die Daten werden in der Datenbank gespeichert."

\textbf{Probleme:}
\begin{itemize}
    \item Zu technisch/implementierungsnah
    \item Fehlende Akteurperspektive
    \item Unklarer Nutzen/Ziel
    \item Fehlende Alternativszenarien
    \item Keine Fehlerbehandlung
\end{itemize}

\textbf{Verbesserter Use Case:}
"Der Kunde möchte seine Bestelldaten speichern. Er bestätigt die Eingaben und erhält eine Bestätigung über die erfolgreiche Speicherung."
\end{example2}

\begin{concept}{Use Case Granularität}\\
Schritte zur Verfeinerung eines Use Cases:
\begin{enumerate}
    \item \textbf{Brief Use Case}
    \begin{itemize}
        \item Kurze Zusammenfassung
        \item Hauptablauf skizzieren
        \item Keine Details zu Varianten
    \end{itemize}
    \item \textbf{Casual Use Case}
    \begin{itemize}
        \item Mehrere Absätze
        \item Hauptvarianten beschreiben
        \item Informeller Stil
    \end{itemize}
    \item \textbf{Fully-dressed Use Case}
    \begin{itemize}
        \item Vollständige Struktur
        \item Alle Varianten
        \item Vor- und Nachbedingungen
        \item Garantien definieren
    \end{itemize}
\end{enumerate}
\end{concept}

\begin{example2}{Brief Use Case}
\textbf{Verkauf abwickeln}

Kunde kommt mit Waren zur Kasse. Kassier erfasst alle Produkte. System berechnet Gesamtbetrag. Kassier nimmt Zahlung entgegen und gibt ggf. Wechselgeld. System druckt Beleg.
\end{example2}

\begin{example2}{Fully-dressed Use Case}
\textbf{UC: Verkauf abwickeln}
\begin{itemize}
    \item \textbf{Umfang:} Kassensystem
    \item \textbf{Primärakteur:} Kassier
    \item \textbf{Stakeholder:} Kunde (schnelle Abwicklung), Geschäft (korrekte Abrechnung)
    \item \textbf{Vorbedingung:} Kasse ist geöffnet
    \item \textbf{Standardablauf:}
    \begin{enumerate}
        \item Kassier startet neuen Verkauf
        \item System initialisiert neue Transaktion
        \item Kassier erfasst Produkte
        \item System zeigt Zwischensumme
        \item Kassier schliesst Verkauf ab
        \item System zeigt Gesamtbetrag
        \item Kunde bezahlt
        \item System druckt Beleg
    \end{enumerate}
\end{itemize}
\end{example2}

\begin{example2}{Casual Use Case}
\textbf{UC: Verkauf abwickeln}
%TODO: Add casual use case
\end{example2}

\begin{example2}{Fully-dressed Use Case}
\textbf{Aufgabe:} Erstellen Sie einen fully-dressed Use Case für ein Online-Bibliothekssystem. Fokus: "Buch ausleihen"

\textbf{Lösung:}
\begin{itemize}
    \item \textbf{Umfang:} Online-Bibliothekssystem
    \item \textbf{Primärakteur:} Bibliotheksnutzer
    \item \textbf{Stakeholder:} 
    \begin{itemize}
        \item Bibliotheksnutzer: Möchte Buch einfach ausleihen
        \item Bibliothek: Korrekte Erfassung der Ausleihe
    \end{itemize}
    \item \textbf{Vorbedingung:} Nutzer ist eingeloggt
    \item \textbf{Standardablauf:}
    \begin{enumerate}
        \item Nutzer sucht Buch
        \item System zeigt Verfügbarkeit
        \item Nutzer wählt Ausleihe
        \item System prüft Ausleihberechtigung
        \item System registriert Ausleihe
        \item System zeigt Bestätigung
    \end{enumerate}
    \item \textbf{Erweiterungen:}
    \begin{itemize}
        \item 2a: Buch nicht verfügbar
        \item 4a: Keine Ausleihberechtigung
    \end{itemize}
\end{itemize}
\end{example2}







\subsubsection{Systemsequenzdiagramme}

\begin{definition}{Systemsequenzdiagramm (SSD)}\\
Formalisierte Darstellung der System-Interaktionen:
\begin{itemize}
    \item Zeigt Input/Output-Events
    \item Identifiziert Systemoperationen
    \item Basis für API-Design
\end{itemize}
\includegraphics[width=\linewidth]{images/ssd.png}
\end{definition}

\begin{KR}{SSD Erstellung}
\begin{enumerate}
    \item Use Case als Grundlage wählen
    \item Akteur und System identifizieren
    \item Methodenaufrufe definieren:
    \begin{itemize}
        \item Namen aussagekräftig wählen
        \item Parameter festlegen
        \item Rückgabewerte bestimmen
    \end{itemize}
    \item Zeitliche Abfolge modellieren
    \item Optional: Externe Systeme einbinden
\end{enumerate}
\end{KR}

\begin{example2}{SSD Übungsaufgabe}
\textbf{Aufgabe:} Erstellen Sie ein Systemsequenzdiagramm für den Use Case "Geld abheben" an einem Bankautomaten.

\textbf{Wichtige Aspekte:}
\begin{itemize}
    \item Kartenvalidierung
    \item PIN-Eingabe
    \item Betragseingabe
    \item Kontostandsprüfung
    \item Geldausgabe
    \item Belegdruck
\end{itemize}

\textbf{Essentielle Systemoperationen:}
\begin{itemize}
    \item validateCard(cardNumber)
    \item checkPIN(pin)
    \item withdrawMoney(amount)
    \item printReceipt()
\end{itemize}
%TODO: add SSD graphic
\end{example2}








