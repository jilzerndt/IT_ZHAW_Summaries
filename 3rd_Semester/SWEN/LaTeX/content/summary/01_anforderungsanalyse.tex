\section{Anforderungsanalyse}

[Previous content remains unchanged until User-Centered Design]

\begin{KR}{User Research durchführen}
Systematisches Vorgehen für User und Domain Research:
\begin{enumerate}
    \item Zielgruppe identifizieren
    \begin{itemize}
        \item Wer sind die Benutzer?
        \item Was sind ihre Aufgaben/Ziele?
        \item Wie sieht ihre Arbeitsumgebung aus?
    \end{itemize}
    \item Daten sammeln durch
    \begin{itemize}
        \item Contextual Inquiry
        \item Interviews
        \item Beobachtung
        \item Fokusgruppen
        \item Nutzungsauswertung
    \end{itemize}
    \item Ergebnisse dokumentieren in
    \begin{itemize}
        \item Personas
        \item Usage-Szenarien
        \item Mentales Modell
    \end{itemize}
\end{enumerate}
\end{KR}

[Previous content about UCD remains]

\begin{example}{Persona erstellen}
\textbf{Aufgabe:} Erstellen Sie eine Persona für ein Online-Banking-System.

\textbf{Lösung:} 
\textbf{Sarah Schmidt, 34, Projektmanagerin}
\begin{itemize}
    \item \textbf{Hintergrund:}
    \begin{itemize}
        \item Arbeitet Vollzeit in IT-Firma
        \item Technik-affin, aber keine Expertin
        \item Nutzt Smartphone für die meisten Aufgaben
    \end{itemize}
    \item \textbf{Ziele:}
    \begin{itemize}
        \item Schnelle Überweisungen zwischen Konten
        \item Überblick über Ausgaben
        \item Sichere Authentifizierung
    \end{itemize}
    \item \textbf{Frustrationen:}
    \begin{itemize}
        \item Komplexe Menüführung
        \item Lange Ladezeiten
        \item Mehrfache Login-Prozesse
    \end{itemize}
\end{itemize}
\end{example}

[Previous content about Requirements Engineering]

\begin{formula}{Anforderungsarten}
\textbf{Funktionale Anforderungen:}
\begin{itemize}
    \item Was soll das System tun?
    \item Dokumentiert in Use Cases
    \item Messbar und testbar
\end{itemize}

\textbf{Nicht-funktionale Anforderungen:}
\begin{itemize}
    \item Wie soll das System sein?
    \item Performance, Sicherheit, Usability
    \item Nach ISO 25010 kategorisiert
\end{itemize}

\textbf{Randbedingungen:}
\begin{itemize}
    \item Technische Einschränkungen
    \item Gesetzliche Vorgaben
    \item Budget und Zeitrahmen
\end{itemize}
\end{formula}

[Previous content about Use Cases]

\begin{KR}{Use Case Verfeinerung}
Schritte zur Verfeinerung eines Use Cases:
\begin{enumerate}
    \item \textbf{Brief Use Case}
    \begin{itemize}
        \item Kurze Zusammenfassung
        \item Hauptablauf skizzieren
        \item Keine Details zu Varianten
    \end{itemize}
    \item \textbf{Casual Use Case}
    \begin{itemize}
        \item Mehrere Absätze
        \item Hauptvarianten beschreiben
        \item Informeller Stil
    \end{itemize}
    \item \textbf{Fully-dressed Use Case}
    \begin{itemize}
        \item Vollständige Struktur
        \item Alle Varianten
        \item Vor- und Nachbedingungen
        \item Garantien definieren
    \end{itemize}
\end{enumerate}
\end{KR}

\begin{example}{Typische Prüfungsaufgabe: Use Case Analyse}
\textbf{Aufgabe:} Analysieren Sie den folgenden Use Case und identifizieren Sie mögliche Probleme:

\textbf{Use Case:} "Der Benutzer loggt sich ein und das System zeigt die Startseite. Er klickt auf den Button und die Daten werden in der Datenbank gespeichert."

\textbf{Probleme:}
\begin{itemize}
    \item Zu technisch/implementierungsnah
    \item Fehlende Akteurperspektive
    \item Unklarer Nutzen/Ziel
    \item Fehlende Alternativszenarien
    \item Keine Fehlerbehandlung
\end{itemize}

\textbf{Verbesserter Use Case:}
"Der Kunde möchte seine Bestelldaten speichern. Er bestätigt die Eingaben und erhält eine Bestätigung über die erfolgreiche Speicherung."
\end{example}

[Previous content about SSD remains]

\begin{example}{SSD Übungsaufgabe}
\textbf{Aufgabe:} Erstellen Sie ein Systemsequenzdiagramm für den Use Case "Geld abheben" an einem Bankautomaten.

\textbf{Wichtige Aspekte:}
\begin{itemize}
    \item Kartenvalidierung
    \item PIN-Eingabe
    \item Betragseingabe
    \item Kontostandsprüfung
    \item Geldausgabe
    \item Belegdruck
\end{itemize}

\textbf{Essentielle Systemoperationen:}
\begin{itemize}
    \item validateCard(cardNumber)
    \item checkPIN(pin)
    \item withdrawMoney(amount)
    \item printReceipt()
\end{itemize}
\end{example}