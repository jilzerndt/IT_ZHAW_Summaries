[Previous content until Implementation Strategies remains unchanged]

\begin{example}{Prüfungsaufgabe: Entwicklungsansätze vergleichen}
\textbf{Szenario:}
Ein Team soll eine neue Webanwendung entwickeln. Diskutieren Sie die Vor- und Nachteile 
von TDD gegenüber CDD für dieses Projekt.

\textbf{Musterlösung:}
\begin{itemize}
    \item \textbf{TDD Vorteile:}
    \begin{itemize}
        \item Testbare Architektur von Anfang an
        \item Frühe Fehlererkennung
        \item Dokumentation durch Tests
        \item Sicherheit bei Refactoring
    \end{itemize}
    
    \item \textbf{TDD Nachteile:}
    \begin{itemize}
        \item Initial höherer Zeitaufwand
        \item Lernkurve für das Team
        \item Schwierig bei unklaren Anforderungen
    \end{itemize}
    
    \item \textbf{Empfehlung:}
    \begin{itemize}
        \item TDD für kritische Kernkomponenten
        \item CDD für Prototypen und UI
        \item Hybridansatz je nach Modulkritikalität
    \end{itemize}
\end{itemize}
\end{example}

\begin{KR}{Code Review Durchführung}
\textbf{1. Vorbereitung}
\begin{itemize}
    \item Code-Guidelines bereitstellen
    \item Checkliste erstellen
    \item Scope definieren
\end{itemize}

\textbf{2. Review durchführen}
\begin{itemize}
    \item Lesbarkeit prüfen
    \item Naming Conventions
    \item Architekturkonformität
    \item Testabdeckung
\end{itemize}

\textbf{3. Feedback geben}
\begin{itemize}
    \item Konstruktiv formulieren
    \item Priorisieren
    \item Lösungen vorschlagen
\end{itemize}
\end{KR}

\begin{example}{Typische Prüfungsaufgabe: Code Smells}
\textbf{Analysieren Sie folgenden Code auf Code Smells:}

\textbf{Problematischer Code:}
\begin{itemize}
    \item Klasse "UserManager" mit 1000 Zeilen
    \item Methode "processData" mit 200 Zeilen
    \item Variable "data" wird in 15 Methoden verwendet
    \item Duplizierte Validierungslogik in mehreren Klassen
\end{itemize}

\textbf{Identifizierte Smells:}
\begin{itemize}
    \item \textbf{God Class:} UserManager zu groß
    \item \textbf{Long Method:} processData zu komplex
    \item \textbf{Global Variable:} data zu weit verbreitet
    \item \textbf{Duplicate Code:} Validierungslogik
\end{itemize}

\textbf{Refactoring-Vorschläge:}
\begin{itemize}
    \item Aufteilen in spezialisierte Klassen
    \item Extract Method für processData
    \item Einführen einer Validierungsklasse
    \item Dependency Injection für data
\end{itemize}
\end{example}

[Previous content about Testing basics remains]

\begin{example}{Prüfungsaufgabe: Teststrategie}
\textbf{Szenario:}
Ein Onlineshop-System soll getestet werden. Entwickeln Sie eine Teststrategie.

\textbf{Lösung:}
\begin{itemize}
    \item \textbf{Unit Tests:}
    \begin{itemize}
        \item Warenkorb-Berechnungen
        \item Preis-Kalkulationen
        \item Validierungsfunktionen
    \end{itemize}
    
    \item \textbf{Integrationstests:}
    \begin{itemize}
        \item Bestellprozess
        \item Zahlungsabwicklung
        \item Lagerverwaltung
    \end{itemize}
    
    \item \textbf{System Tests:}
    \begin{itemize}
        \item Performance unter Last
        \item Sicherheitsaspekte
        \item Datenbankinteraktionen
    \end{itemize}
    
    \item \textbf{Akzeptanztests:}
    \begin{itemize}
        \item Benutzerszenarien
        \item Geschäftsprozesse
        \item Reporting
    \end{itemize}
\end{itemize}
\end{example}

\begin{KR}{Testabdeckung optimieren}
\textbf{1. Analyse der Testabdeckung}
\begin{itemize}
    \item Code Coverage messen
    \item Kritische Pfade identifizieren
    \item Lücken dokumentieren
\end{itemize}

\textbf{2. Priorisierung}
\begin{itemize}
    \item Geschäftskritische Funktionen
    \item Fehleranfällige Bereiche
    \item Komplexe Algorithmen
\end{itemize}

\textbf{3. Ergänzung der Tests}
\begin{itemize}
    \item Randfall-Tests
    \item Negativtests
    \item Performance-Tests
\end{itemize}

\textbf{4. Wartung}
\begin{itemize}
    \item Regelmäßige Überprüfung
    \item Anpassung an Änderungen
    \item Entfernung veralteter Tests
\end{itemize}
\end{KR}

\begin{example}{Prüfungsaufgabe: Testfälle entwerfen}
\textbf{Aufgabe:}
Entwickeln Sie Testfälle für eine Methode zur Validierung einer Email-Adresse.

\textbf{Testfälle:}
\begin{itemize}
    \item \textbf{Positive Tests:}
    \begin{itemize}
        \item Standard Email (user@domain.com)
        \item Subdomain (user@sub.domain.com)
        \item Mit Punkten (first.last@domain.com)
    \end{itemize}
    
    \item \textbf{Negative Tests:}
    \begin{itemize}
        \item Fehlende @ (userdomain.com)
        \item Mehrere @ (user@@domain.com)
        \item Ungültige Zeichen (user\#@domain.com)
    \end{itemize}
    
    \item \textbf{Randfälle:}
    \begin{itemize}
        \item Leerer String
        \item Nur Whitespace
        \item Sehr lange Adressen
    \end{itemize}
\end{itemize}
\end{example}