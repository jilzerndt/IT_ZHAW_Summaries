\section{Domänenmodellierung}

\begin{definition}{Domänenmodellierung}
Die Domänenmodellierung ist ein essentieller Schritt zwischen Anforderungsanalyse und Software-Design:
\begin{itemize}
    \item Abbildung der Fachdomäne in strukturierter Form
    \item Basis für späteres Objektdesign
    \item Kommunikationsmittel mit Stakeholdern
    \item Dokumentation des Problemverständnisses
\end{itemize}
\end{definition}

\begin{concept}{Domänenmodell}
Ein Domänenmodell ist ein vereinfachtes UML-Klassendiagramm zur Darstellung der Fachdomäne:
\begin{itemize}
    \item Konzepte als Klassen
    \item Eigenschaften als Attribute (ohne Typangabe)
    \item Beziehungen als Assoziationen mit Multiplizitäten
    \item Optional: Aggregationen/Kompositionen
\end{itemize}
\end{concept}

\begin{KR}{Vorgehen bei der Domänenmodellierung}
\begin{enumerate}
    \item \textbf{Analyse der Dokumentation}
    \begin{itemize}
        \item Use Cases durcharbeiten
        \item Glossar berücksichtigen
        \item Stakeholder-Interviews auswerten
    \end{itemize}

    \item \textbf{Konzepte identifizieren}
    \begin{itemize}
        \item Substantive markieren
        \item Kategorisieren (siehe Checkliste)
        \item Redundanzen eliminieren
    \end{itemize}

    \item \textbf{Beziehungen analysieren}
    \begin{itemize}
        \item Verben zwischen Konzepten suchen
        \item Art der Beziehung bestimmen
        \item Multiplizitäten festlegen
    \end{itemize}

    \item \textbf{Attribute hinzufügen}
    \begin{itemize}
        \item Relevante Eigenschaften identifizieren
        \item Vermeidung von Redundanz
        \item Angemessene Detailtiefe wählen
    \end{itemize}

    \item \textbf{Review und Verfeinerung}
    \begin{itemize}
        \item Mit Stakeholdern abstimmen
        \item Konsistenz prüfen
        \item Analysemuster anwenden
    \end{itemize}
\end{enumerate}
\end{KR}

[Previous content for Analysemuster remains...]

\begin{example}{Beschreibungsklassen}
\textbf{Problem:} Modellierung eines Bibliothekssystems

\begin{lstlisting}[language=Java]
// Schlechte Loesung: Redundante Daten
public class Book {
    private String title;
    private String author;
    private String isbn;
    private String description;
    private boolean isLent;
    private Date dueDate;
}

// Bessere Loesung: Beschreibungsklasse
public class BookDescription {
    private String title;
    private String author;
    private String isbn;
    private String description;
}

public class BookCopy {
    private BookDescription description;
    private boolean isLent;
    private Date dueDate;
}
\end{lstlisting}
\end{example}

\begin{example}{Wertobjekte}
\textbf{Problem:} Modellierung von Geldbeträgen

\begin{lstlisting}[language=Java]
// Schlechte Loesung: Primitive Obsession
public class Order {
    private double amount;  // Problematisch!
    private String currency;
}

// Bessere Loesung: Money Value Object
public class Money {
    private BigDecimal amount;
    private Currency currency;
    
    public Money add(Money other) {
        if (!this.currency.equals(other.currency)) {
            throw new IllegalArgumentException("Different currencies");
        }
        return new Money(amount.add(other.amount), currency);
    }
}

public class Order {
    private Money price;  // Besser!
}
\end{lstlisting}
\end{example}

\begin{KR}{Validierung des Domänenmodells}
\textbf{Checkliste für die Qualitätssicherung:}
\begin{enumerate}
    \item \textbf{Vollständigkeit}
    \begin{itemize}
        \item Alle wichtigen Konzepte vorhanden?
        \item Alle relevanten Beziehungen modelliert?
        \item Wichtige Attribute berücksichtigt?
    \end{itemize}

    \item \textbf{Korrektheit}
    \begin{itemize}
        \item Konzepte richtig kategorisiert?
        \item Beziehungen korrekt typisiert?
        \item Multiplizitäten stimmen?
    \end{itemize}

    \item \textbf{Angemessenheit}
    \begin{itemize}
        \item Abstraktionsniveau passend?
        \item Detaillierungsgrad einheitlich?
        \item Komplexität handhabbar?
    \end{itemize}

    \item \textbf{Verständlichkeit}
    \begin{itemize}
        \item Eindeutige Bezeichnungen?
        \item Klare Strukturierung?
        \item Nachvollziehbare Beziehungen?
    \end{itemize}
\end{enumerate}
\end{KR}

\begin{example}{Komplettes Domänenmodell: E-Commerce System}
\textbf{Identifizierte Konzepte:}
\begin{itemize}
    \item \textbf{Beschreibungsklassen:}
    \begin{itemize}
        \item ProductCatalog
        \item ProductDescription
    \end{itemize}
    
    \item \textbf{Wertobjekte:}
    \begin{itemize}
        \item Money
        \item Address
    \end{itemize}
    
    \item \textbf{Entitäten:}
    \begin{itemize}
        \item Customer
        \item Order
        \item OrderLine
    \end{itemize}
    
    \item \textbf{Zustände:}
    \begin{itemize}
        \item OrderStatus
        \item PaymentStatus
    \end{itemize}
\end{itemize}

\begin{lstlisting}[language=Java]
// Beispielhafte Implementierung der Kernkonzepte
public class Order {
    private OrderId id;
    private Customer customer;
    private List<OrderLine> lines;
    private Money total;
    private OrderStatus status;
    private Address shippingAddress;
}

public enum OrderStatus {
    CREATED, CONFIRMED, PAID, SHIPPED, DELIVERED
}

public class OrderLine {
    private ProductDescription product;
    private int quantity;
    private Money lineTotal;
}
\end{lstlisting}
\end{example}

[Your previous content for avoiding modeling errors remains...]