\section{Verteilte Systeme}

\begin{definition}{Verteiltes System}
Ein Netzwerk aus autonomen Computern und Softwarekomponenten, die als einheitliches System erscheinen:
\begin{itemize}
    \item Autonome Knoten und Komponenten
    \item Netzwerkverbindung
    \item Erscheint als ein System
    \item Gemeinsame Ressourcennutzung
    \item Transparente Verteilung
\end{itemize}
\end{definition}

\begin{example}{Verteiltes System in der Praxis}
\begin{lstlisting}[language=Java]
// Microservice-Architektur Beispiel
@RestController
public class OrderService {
    private final ProductService productService;
    private final PaymentService paymentService;
    
    @PostMapping("/orders")
    public OrderResponse createOrder(
            @RequestBody OrderRequest request) {
        // Synchrone Kommunikation mit Product-Service
        ProductInfo product = 
            productService.getProduct(request.getProductId());
            
        // Asynchrone Kommunikation via Message Broker
        paymentService.processPaymentAsync(
            new PaymentRequest(request.getPaymentDetails()));
            
        return new OrderResponse(/* ... */);
    }
}
\end{lstlisting}
\end{example}

\begin{KR}{Fehlerbehandlung in verteilten Systemen}
\begin{lstlisting}[language=Java]
public class ResilientServiceCaller {
    private final CircuitBreaker circuitBreaker;
    private final RetryTemplate retryTemplate;
    
    public <T> T callService(ServiceCall<T> serviceCall) {
        return circuitBreaker.run(() -> 
            retryTemplate.execute(context -> {
                try {
                    return serviceCall.execute();
                } catch (NetworkException e) {
                    // Exponential Backoff
                    long waitTime = 
                        Math.pow(2, context.getRetryCount());
                    Thread.sleep(waitTime * 1000);
                    throw e;
                }
            })
        );
    }
}

// Verwendung
OrderInfo order = resilientCaller.callService(() ->
    orderService.getOrder(orderId));
\end{lstlisting}
\end{KR}

\begin{concept}{Kommunikationsmuster}
\textbf{1. Synchrone Kommunikation:}
\begin{lstlisting}[language=Java]
// REST-Client mit synchronem Aufruf
@FeignClient("product-service")
public interface ProductClient {
    @GetMapping("/products/{id}")
    ProductDTO getProduct(@PathVariable String id);
}

// Synchroner Service-Aufruf
ProductDTO product = productClient.getProduct(id);
\end{lstlisting}

\textbf{2. Asynchrone Kommunikation:}
\begin{lstlisting}[language=Java]
// Message Producer
@Service
public class OrderEventPublisher {
    private final KafkaTemplate<String, OrderEvent> kafka;
    
    public void publishOrderCreated(Order order) {
        OrderEvent event = new OrderEvent(order);
        kafka.send("order-events", event);
    }
}

// Message Consumer
@KafkaListener(topics = "order-events")
public void handleOrderEvent(OrderEvent event) {
    // Asynchrone Verarbeitung
}
\end{lstlisting}
\end{concept}

\begin{example}{Implementierung von Konsistenzstrategien}
\begin{lstlisting}[language=Java]
@Entity
public class Product {
    @Version
    private Long version;
    
    @Lock(LockModeType.OPTIMISTIC)
    public void updateStock(int quantity) {
        // Optimistic Locking durch @Version
        this.stockQuantity += quantity;
    }
}

// Verwendung mit Retry bei Konflikt
@Transactional
public void processOrder(Order order) {
    try {
        Product product = productRepo.findById(
            order.getProductId());
        product.updateStock(-order.getQuantity());
        productRepo.save(product);
    } catch (OptimisticLockException e) {
        // Retry mit neuem Versuch
        retryTemplate.execute(context -> {
            // Wiederhole Operation
            return null;
        });
    }
}
\end{lstlisting}
\end{example}

\begin{KR}{Service Discovery und Load Balancing}
\begin{lstlisting}[language=Java]
// Service Registration
@SpringBootApplication
@EnableEurekaClient
public class ProductService {
    public static void main(String[] args) {
        SpringApplication.run(ProductService.class, args);
    }
}

// Load Balancer Configuration
@Configuration
public class LoadBalancerConfig {
    @Bean
    @LoadBalanced
    public RestTemplate restTemplate() {
        return new RestTemplate();
    }
}

// Service Discovery verwendung
@Service
public class ProductServiceClient {
    private final RestTemplate restTemplate;
    
    public ProductInfo getProduct(String id) {
        return restTemplate.getForObject(
            "http://product-service/products/" + id,
            ProductInfo.class
        );
    }
}
\end{lstlisting}
\end{KR}

\begin{concept}{CAP-Theorem in der Praxis}
\begin{itemize}
    \item \textbf{Consistency (C):} Alle Knoten sehen dieselben Daten
    \item \textbf{Availability (A):} Jede Anfrage erhält eine Antwort
    \item \textbf{Partition Tolerance (P):} System funktioniert trotz Netzwerkausfällen
\end{itemize}

\textbf{Beispiel-Implementierungen:}
\begin{itemize}
    \item \textbf{CP-System:} Distributed Database (z.B. MongoDB)
    \item \textbf{AP-System:} Content Delivery Network (CDN)
    \item \textbf{CA-System:} Traditional RDBMS (z.B. PostgreSQL)
\end{itemize}
\end{concept}

[Previous content about Middleware Technologies and Error Sources remains...]