\section{Use Case Realisation}

\begin{concept}{Use Case Realization}
Die Umsetzung von Use Cases erfolgt durch:
\begin{itemize}
    \item Detaillierte Szenarien aus den Use Cases
    \item Systemantworten müssen realisiert werden
    \item UI statt System im SSD (System Sequence Diagram)
    \item Systemoperationen sind die zu implementierenden Elemente
    \item Berücksichtigung der Softwarearchitektur
\end{itemize}
\end{concept}

\begin{definition}{UML im Implementierungsprozess}
UML dient als:
\begin{itemize}
    \item Zwischenschritt bei wenig Erfahrung
    \item Kompakter Ersatz für Programmiercode
    \item Kommunikationsmittel (auch für Nicht-Techniker)
    \item Dokumentation von Design-Entscheidungen
\end{itemize}
\end{definition}

\begin{KR}{Vorgehen bei der Use Case Realization}
\textbf{1. Vorbereitung und Analyse:}
\begin{lstlisting}[language=Java]
// Beispiel: Systemoperation aus SSD
public interface POSSystem {
    // Use Case: Process Sale
    void makeNewSale();
    void enterItem(String itemId, int quantity);
    void endSale();
    void makePayment(BigDecimal amount);
}
\end{lstlisting}

\textbf{2. Design der Controller-Klasse:}
\begin{lstlisting}[language=Java]
// Fassaden-Controller nach GRASP
public class Register implements POSSystem {
    private Sale currentSale;
    private ProductCatalog catalog;
    
    @Override
    public void makeNewSale() {
        currentSale = new Sale();
    }
    
    @Override
    public void enterItem(String itemId, int quantity) {
        ProductDescription desc = 
            catalog.getProductDescription(itemId);
        currentSale.makeLineItem(desc, quantity);
    }
}
\end{lstlisting}

\textbf{3. Implementierung der Domänenklassen:}
\begin{lstlisting}[language=Java]
public class Sale {
    private List<SaleLineItem> items = new ArrayList<>();
    private boolean isComplete = false;
    
    public void makeLineItem(ProductDescription desc, 
                           int quantity) {
        SaleLineItem item = new SaleLineItem(desc, quantity);
        items.add(item);
    }
    
    public BigDecimal getTotal() {
        return items.stream()
            .map(SaleLineItem::getSubtotal)
            .reduce(BigDecimal.ZERO, BigDecimal::add);
    }
}
\end{lstlisting}
\end{KR}

\begin{example}{Komplette Use Case Realization: Bestellung aufgeben}
\textbf{1. System Sequence Diagram (SSD):}
\begin{lstlisting}[language=Java]
// Systemoperationen aus SSD
public interface OrderSystem {
    String startOrder(String customerId);
    void addItem(String orderId, String productId, int qty);
    OrderSummary submitOrder(String orderId);
}
\end{lstlisting}

\textbf{2. Controller Implementation:}
\begin{lstlisting}[language=Java]
@Service // Beispiel mit Spring Framework
public class OrderController implements OrderSystem {
    private final OrderRepository orders;
    private final CustomerService customers;
    private final ProductService products;
    
    @Override
    public String startOrder(String customerId) {
        Customer customer = customers.findById(customerId);
        Order order = new Order(customer);
        return orders.save(order).getId();
    }
    
    @Override
    public void addItem(String orderId, 
                       String productId, 
                       int qty) {
        Order order = orders.findById(orderId);
        Product product = products.findById(productId);
        order.addItem(product, qty);
        orders.save(order);
    }
}
\end{lstlisting}

\textbf{3. Domänenklassen:}
\begin{lstlisting}[language=Java]
public class Order {
    private String id;
    private Customer customer;
    private List<OrderItem> items = new ArrayList<>();
    private OrderStatus status = OrderStatus.NEW;
    
    public void addItem(Product product, int quantity) {
        OrderItem item = new OrderItem(product, quantity);
        items.add(item);
    }
    
    public OrderSummary createSummary() {
        return new OrderSummary(
            id,
            customer.getName(),
            calculateTotal(),
            items.size()
        );
    }
}
\end{lstlisting}
\end{example}

\begin{KR}{GRASP-konforme Implementierung}
\begin{enumerate}
    \item \textbf{Information Expert}
    \begin{lstlisting}[language=Java]
    // Sale ist Expert fuer Preisberechnung
    public class Sale {
        private List<SaleLineItem> items;
        
        public BigDecimal getTotal() {
            // Sale kennt seine Items
            return items.stream()
                .map(SaleLineItem::getSubtotal)
                .reduce(BigDecimal.ZERO, 
                       BigDecimal::add);
        }
    }
    \end{lstlisting}

    \item \textbf{Creator}
    \begin{lstlisting}[language=Java]
    // Sale erstellt SaleLineItems
    public class Sale {
        public void makeLineItem(ProductDescription desc, 
                               int quantity) {
            // Sale erstellt und enthaelt SaleLineItems
            SaleLineItem item = 
                new SaleLineItem(desc, quantity);
            items.add(item);
        }
    }
    \end{lstlisting}

    \item \textbf{Low Coupling}
    \begin{lstlisting}[language=Java]
    // Verwendung von Interfaces
    public class Register {
        private ProductCatalog catalog;
        // Kopplung nur ueber Interface
        private PaymentProcessor paymentProcessor;
        
        public void makePayment(BigDecimal amount) {
            // Lose Kopplung durch Interface
            paymentProcessor.process(amount);
        }
    }
    \end{lstlisting}
\end{enumerate}
\end{KR}

[Your previous content for avoiding implementation errors remains...]