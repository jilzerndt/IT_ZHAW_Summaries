\section{Implementation, Refactoring und Testing}

\subsection{Design to Code}

\begin{concept}{Umsetzungsstrategien}
\textbf{Code-Driven Development:}
\begin{itemize}
    \item Direkte Implementierung der Klassen
    \item Nachträgliches Testing
\end{itemize}

\textbf{Test-Driven Development (TDD):}
\begin{itemize}
    \item Tests vor Implementation
    \item Red-Green-Refactor Zyklus
\end{itemize}

\textbf{Behavior-Driven Development (BDD):}
\begin{itemize}
    \item Testbeschreibung aus Anwendersicht
    \item Gherkin-Syntax für Szenarios
\end{itemize}
\end{concept}

\begin{KR}{Implementation Grundsätze}
\textbf{1. Code-Guidelines:}
\begin{itemize}
    \item Einheitliche Formatierung
    \item Klare Namenskonventionen
    \item Dokumentationsrichtlinien
\end{itemize}

\textbf{2. Fehlerbehandlung:}
\begin{itemize}
    \item Exceptions statt Fehlercodes
    \item Sinnvolle Error Messages
    \item Logging-Strategie
\end{itemize}

\textbf{3. Namensgebung:}
\begin{itemize}
    \item Aussagekräftige Namen
    \item Konsistente Begriffe
    \item Domain-Driven Naming
\end{itemize}
\end{KR}

\subsection{Refactoring}

\begin{definition}{Refactoring Grundlagen}
Strukturierte Verbesserung des Codes ohne Änderung des externen Verhaltens:
\begin{itemize}
    \item Kleine, kontrollierte Schritte
    \item Erhaltung der Funktionalität 
    \item Verbesserung der Codequalität und interner Struktur
    \item Ziel: Bessere Wartbarkeit und Erweiterbarkeit
\end{itemize}
\end{definition}

\begin{definition}{Code Smells}
Anzeichen für mögliche Probleme im Code:
\begin{itemize}
    \item Duplizierter Code
    \item Lange Methoden
    \item Klassen mit vielen Instanzvariablen
    \item Klassen mit sehr viel Code
    \item Auffällig ähnliche Unterklassen
    \item Keine Interfaces
    \item Hohe Kopplung zwischen Klassen
\end{itemize}
\end{definition}

\begin{KR}{Refactoring Patterns}
\textbf{1. Extract Method}
\begin{itemize}
    \item Code in eigene Methode auslagern
    \item Verbessert Lesbarkeit und Wiederverwendbarkeit
    \item Reduziert Duplikation
\end{itemize}

\textbf{2. Move Method/Field}
\begin{itemize}
    \item Methode/Feld in andere Klasse verschieben
    \item Verbessert Kohäsion
    \item Folgt Information Expert
\end{itemize}

\textbf{3. Extract Class}
\begin{itemize}
    \item Teil einer Klasse in neue Klasse auslagern
    \item Trennt Verantwortlichkeiten
    \item Erhöht Kohäsion
\end{itemize}

\textbf{4. Rename Method/Class/Variable}
\begin{itemize}
    \item Bessere Namen für besseres Verständnis
    \item Dokumentiert Zweck
    \item Erleichtert Wartung
\end{itemize}
\end{KR}

\subsection{Testing}

\begin{concept}{Teststufen}
\begin{itemize}
\item \textbf{Unit Tests:} 
\begin{itemize}
    \item Einzelne Komponenten
    \item Isolation durch Mocks/Stubs
    \item Schnelle Ausführung
\end{itemize}

\item \textbf{Integrationstests:} 
\begin{itemize} 
    \item Zusammenspiel mehrerer Komponenten
    \item Schnittstellen-Tests
    \item Externe Systeme
\end{itemize}

\item \textbf{Systemtests:} 
\begin{itemize}
    \item End-to-End Szenarien
    \item Nicht-funktionale Anforderungen
\end{itemize}

\item \textbf{Abnahmetests:} 
\begin{itemize}
    \item Gegen Kundenanforderungen
    \item User Acceptance Testing (UAT)
\end{itemize}
\end{itemize}
\end{concept}

\begin{KR}{Testdesign}
\textbf{1. Funktionaler Test (Black-Box)}
\begin{itemize}
    \item Test aus Benutzersicht
    \item Ohne Codekenntnis
    \item Fokus auf Input/Output
\end{itemize}

\textbf{2. Strukturbezogener Test (White-Box)}
\begin{itemize}
    \item Test mit Codekenntnis
    \item Code Coverage
    \item Pfadtests
\end{itemize}

\textbf{3. Änderungsbezogener Test}
\begin{itemize}
    \item Regressionstest
    \item Verifizierung von Änderungen
    \item Sicherstellung der Gesamtfunktionalität
\end{itemize}
\end{KR}

\begin{definition}{Testbegriffe}
\begin{itemize}
    \item \textbf{Testling/Testobjekt:} Das zu testende Element
    \item \textbf{Fehler:} Fehler des Entwicklers bei der Implementation
    \item \textbf{Fehlerwirkung/Bug:} Abweichung vom spezifizierten Verhalten
    \item \textbf{Testfall:} Spezifische Testkonfiguration mit Testdaten
    \item \textbf{Testtreiber:} Programm zur Testausführung
\end{itemize}
\end{definition}

%todo: Add examples for:
% - Refactoring before/after
% - Unit test with mocks
% - Integration test setup
% - Test case design examples