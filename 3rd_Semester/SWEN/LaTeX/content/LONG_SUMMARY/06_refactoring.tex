
\section{Implementation, Refactoring und Testing}

\subsection{Von Design zu Code}

\begin{KR}{Clean Code}\\
\textbf{1. Code-Guidelines:}
\begin{itemize}
    \item Einheitliche Formatierung
    \item Klare Namenskonventionen
    \item Dokumentationsrichtlinien
\end{itemize}

\textbf{2. Fehlerbehandlung:}
\begin{itemize}
    \item Exceptions statt Fehlercodes
    \item Sinnvolle Error Messages
    \item Logging-Strategie
\end{itemize}

\textbf{3. Namensgebung:}
\begin{itemize}
    \item Aussagekräftige Namen
    \item Konsistente Begriffe
    \item Domain-Driven Naming
\end{itemize}
\end{KR}

\begin{concept}{Implementierungsstrategien}\\
\textbf{1. Bottom-Up Entwicklung:}
\begin{itemize}
    \item Implementierung beginnt mit Basisbausteinen
    \item Schrittweise Integration zu größeren Komponenten
    \item \textbf{Vorteile:} Gründlich, solide Basis
    \item \textbf{Nachteile:} Spätes Feedback
\end{itemize}

\textbf{2. Agile Entwicklung:}
\begin{itemize}
    \item Inkrementelle Entwicklung in Sprints
    \item Kontinuierliche Integration und Auslieferung
    \item \textbf{Vorteile:} Flexibilität, schnelles Feedback
    \item \textbf{Nachteile:} Mögliche Restrukturierung nötig
\end{itemize}
\end{concept}

\begin{definition}{Entwicklungsansätze}\\
\textbf{Code-Driven Development (CDD):}
\begin{itemize}
    \item Direkte Implementierung der Klassen
    \item Nachträgliches Testing
\end{itemize}

\textbf{Test-Driven Development (TDD):}
\begin{itemize}
    \item Tests vor Implementation
    \item Red-Green-Refactor Zyklus
\end{itemize}

\textbf{Behavior-Driven Development (BDD):}
\begin{itemize}
    \item Testbeschreibung aus Anwendersicht
    \item Gherkin-Syntax für Szenarios
\end{itemize}
\end{definition}

\begin{example2}{Prüfungsaufgabe: Entwicklungsansätze vergleichen}
\textbf{Szenario:}
Ein Team soll eine neue Webanwendung entwickeln. Diskutieren Sie die Vor- und Nachteile 
von TDD gegenüber CDD für dieses Projekt.

\textbf{Musterlösung:}
\begin{itemize}
    \item \textbf{TDD Vorteile:}
    \begin{itemize}
        \item Testbare Architektur von Anfang an
        \item Frühe Fehlererkennung
        \item Dokumentation durch Tests
        \item Sicherheit bei Refactoring
    \end{itemize}
    
    \item \textbf{TDD Nachteile:}
    \begin{itemize}
        \item Initial höherer Zeitaufwand
        \item Lernkurve für das Team
        \item Schwierig bei unklaren Anforderungen
    \end{itemize}
    
    \item \textbf{Empfehlung:}
    \begin{itemize}
        \item TDD für kritische Kernkomponenten
        \item CDD für Prototypen und UI
        \item Hybridansatz je nach Modulkritikalität
    \end{itemize}
\end{itemize}
\end{example2}

\columnbreak


\subsection{Refactoring}

\begin{definition}{Refactoring Grundlagen}\\
Strukturierte Verbesserung des Codes ohne Änderung des externen Verhaltens:
\begin{itemize}
    \item Kleine, kontrollierte Schritte
    \item Erhaltung der Funktionalität (Externes Verhalten bleibt gleich)
    \item Verbesserung der Codequalität und interner Struktur
    \item Ziel: Bessere Wartbarkeit und Erweiterbarkeit
\end{itemize}
\end{definition}

\begin{definition}{Code Smells}
Anzeichen für mögliche Probleme im Code:
\begin{itemize}
    \item \textbf{Duplizierter Code:} Gleicher Code an mehreren Stellen
    \item \textbf{Lange Methoden:} Methoden mit zu vielen Verantwortlichkeiten
    \item \textbf{Große Klassen:} Klassen mit vielen Instanzvariablen
    \item \textbf{Auffällig ähnliche Unterklassen:} Potential für Abstraktion
    \item \textbf{Hohe Kopplung:} Zu viele Abhängigkeiten zwischen Klassen
    \item \textbf{Keine Interfaces:} Mangelnde Abstraktion
\end{itemize}
\end{definition}

\begin{concept}{Laufzeit-Optimierung}\\
\textbf{Grundregeln:}
\begin{itemize}
    \item Zuerst messen, dann optimieren
    \item Performance-Profile nutzen
    \item Bottlenecks identifizieren
\end{itemize}

\textbf{Häufige Probleme:}
\begin{itemize}
    \item Datenbank-Zugriffe
    \item Ineffiziente Algorithmen
    \item Speicherlecks
\end{itemize}
\end{concept}

\begin{KR}{Refactoring Durchführung}\\
\textbf{1. Code Smells identifizieren:}
\begin{itemize}
    \item Duplizierter Code
    \item Lange Methoden
    \item Große Klassen
    \item Hohe Kopplung
\end{itemize}

\textbf{2. Refactoring durchführen:}
\begin{itemize}
    \item Tests sicherstellen
    \item Änderungen vornehmen
    \item Tests ausführen
\end{itemize}

\textbf{3. Patterns anwenden}
\end{KR}

\begin{formula}{Refactoring Patterns}\\
\textbf{1. Extract Method}
\begin{itemize}
    \item Code in eigene Methode auslagern
    \item Verbessert Lesbarkeit und Wiederverwendbarkeit
    \item Reduziert Duplikation
\end{itemize}

\textbf{2. Move Method/Field}
\begin{itemize}
    \item Methode/Feld in andere Klasse verschieben 
    \item Verbessert Kohäsion
    \item Folgt Information Expert
\end{itemize}

\textbf{3. Extract Class}
\begin{itemize}
    \item Teil einer Klasse in neue Klasse auslagern
    \item Trennt Verantwortlichkeiten
    \item Erhöht Kohäsion
\end{itemize}

\textbf{4. Rename Method/Class/Variable}
\begin{itemize}
    \item Bessere Namen für besseres Verständnis
    \item Dokumentiert Zweck
    \item Erleichtert Wartung
\end{itemize}
\end{formula}


\begin{example2}{Typische Prüfungsaufgabe: Code Smells}\\
\textbf{Analysieren Sie folgenden Code auf Code Smells:}

\textbf{Problematischer Code:}
\begin{itemize}
    \item Klasse "UserManager" mit 1000 Zeilen
    \item Methode "processData" mit 200 Zeilen
    \item Variable "data" wird in 15 Methoden verwendet
    \item Duplizierte Validierungslogik in mehreren Klassen
\end{itemize}

\textbf{Identifizierte Smells:}
\begin{itemize}
    \item \textbf{God Class:} UserManager zu groß
    \item \textbf{Long Method:} processData zu komplex
    \item \textbf{Global Variable:} data zu weit verbreitet
    \item \textbf{Duplicate Code:} Validierungslogik
\end{itemize}

\textbf{Refactoring-Vorschläge:}
\begin{itemize}
    \item Aufteilen in spezialisierte Klassen
    \item Extract Method für processData
    \item Einführen einer Validierungsklasse
    \item Dependency Injection für data
\end{itemize}
\end{example2}




\begin{example2}{Refactoring: Extract Method}
\textbf{Vor dem Refactoring:}
\begin{lstlisting}[language=Java, style=basesmol]
public void printOwing() {
    // print banner
    System.out.println("*****************");
    System.out.println("**Customer Owes**");
    System.out.println("*****************");
    
    // calculate total
    double total = 0;
    for (Order order : orders) {
        total += order.getAmount();
    }
    
    // print details
    System.out.println("name: " + name);
    System.out.println("amount: " + total);
}
\end{lstlisting}

\textbf{Nach dem Refactoring:}
\begin{lstlisting}[language=Java, style=basesmol]
public void printOwing() {
    printBanner();
    double total = calculateTotal();
    printDetails(total);
}

private void printBanner() {
    System.out.println("*****************");
    System.out.println("**Customer Owes**");
    System.out.println("*****************");
}

private double calculateTotal() {
    double total = 0;
    for (Order order : orders) {
        total += order.getAmount();
    }
    return total;
}

private void printDetails(double total) {
    System.out.println("name: " + name);
    System.out.println("amount: " + total);
}
\end{lstlisting}
\end{example2}

\pagebreak

\subsection{Testing}

\begin{definition}{Testarten} (Nach Sicht)\\
\begin{itemize}
    \item \textbf{Funktionaler Test (Black-Box Verfahren):} 
    \begin{itemize}
        \item Test aus Benutzersicht
        \item Ohne Codekenntnis (Kenntnis der internen Implementation)
        \item Fokus auf Input/Output
    \end{itemize}
    \item \textbf{Strukturbezogener Test (White-Box Verfahren):}
    \begin{itemize}
        \item Test mit Codekenntnis (Kenntnis der Implementation)
        \item Code Coverage Pfadtests
    \end{itemize}
\end{itemize}
\end{definition}

\begin{concept}{Teststufen} (Nach Umfang)
\begin{itemize}
\item \textbf{1. Unit Tests} Einzelne Komponenten
\begin{itemize}
    \item Isolation durch Mocks/Stubs
    \item Schnelle Ausführung und Automatisierte Tests
\end{itemize}

\item \textbf{2. Integrationstests} Zusammenspiel mehrer Komponenten
\begin{itemize} 
    \item Fokus auf Schnittstellen
    \item Kann externe Systeme einbeziehen
\end{itemize}

\item \textbf{3. Systemtests} Gesamtsystem testen
\begin{itemize}
    \item End-to-End Szenarien
    \item Auch nicht-funktionale Anforderungen
\end{itemize}

\item \textbf{4. Abnahmetests} Gegen Kundenanforderungen
\begin{itemize}
    \item Oft manuelle Tests
    \item User Acceptance Testing (UAT)
\end{itemize}
\end{itemize}
\end{concept}

\begin{theorem}{Wichtige Testbegriffe}
\begin{itemize}
    \item \textbf{Testling/Testobjekt:} Das zu testende Element
    \item \textbf{Fehler:} Fehler des Entwicklers bei der Implementation
    \item \textbf{Fehlerwirkung/Bug:} Abweichung vom spezifizierten Verhalten
    \item \textbf{Testfall:} Spezifische Testkonfiguration mit Testdaten
    \item \textbf{Testtreiber:} Programm zur Testausführung
\end{itemize}
\end{theorem}

\begin{corollary}{Testmerkmale}
\textbf{Was wird getestet?}
\begin{itemize}
    \item Eine Einheit/Klasse (Unit-Test)
    \item Zusammenarbeit mehrerer Klassen
    \item Die gesamte Applikationslogik (ohne UI)
    \item Die gesamte Anwendung (über UI)
\end{itemize}

\textbf{Wie wird getestet?}
\begin{itemize}
    \item Dynamisch: Testfall wird ausgeführt
    \begin{itemize}
        \item Black-Box Test
        \item White-Box Test
    \end{itemize}
    \item Statisch: Quelltext wird analysiert
    \begin{itemize}
        \item Walkthrough
        \item Review
        \item Inspektion
    \end{itemize}
\end{itemize}

\textbf{Wann wird der Test geschrieben?}
\begin{itemize}
    \item Vor dem Implementieren (Test-Driven Development)
    \item Nach dem Implementieren
\end{itemize}

\textbf{Wer testet?}
\begin{itemize}
    \item Entwickler
    \item Tester, Qualitätssicherungsabteilung
    \item Kunde, Endbenutzer
\end{itemize}
\end{corollary}

\begin{KR}{Testentwicklung}\\
\textbf{1. Testfall definieren:}
\begin{itemize}
    \item Vorbedingungen festlegen
    \item Testdaten vorbereiten
    \item Erwartetes Ergebnis definieren
\end{itemize}

\textbf{2. Test implementieren:}
\begin{itemize}
    \item Setup vorbereiten
    \item Testlogik schreiben
    \item Assertions definieren
\end{itemize}

\textbf{3. Test ausführen:}
\begin{itemize}
    \item Automatisiert ausführen
    \item Ergebnisse prüfen
    \item Dokumentation erstellen
\end{itemize}
\end{KR}

\begin{example2}{Prüfungsaufgabe: Teststrategie}\\
\textbf{Szenario:}
Ein Onlineshop-System soll getestet werden. Entwickeln Sie eine Teststrategie.

\textbf{Lösung:}
\begin{itemize}
    \item \textbf{Unit Tests:}
    \begin{itemize}
        \item Warenkorb-Berechnungen
        \item Preis-Kalkulationen
        \item Validierungsfunktionen
    \end{itemize}
    
    \item \textbf{Integrationstests:}
    \begin{itemize}
        \item Bestellprozess
        \item Zahlungsabwicklung
        \item Lagerverwaltung
    \end{itemize}
    
    \item \textbf{System Tests:}
    \begin{itemize}
        \item Performance unter Last
        \item Sicherheitsaspekte
        \item Datenbankinteraktionen
    \end{itemize}
    
    \item \textbf{Akzeptanztests:}
    \begin{itemize}
        \item Benutzerszenarien
        \item Geschäftsprozesse
        \item Reporting
    \end{itemize}
\end{itemize}
\end{example2}

\begin{KR}{Testabdeckung optimieren}\\
\textbf{1. Analyse der Testabdeckung}
\begin{itemize}
    \item Code Coverage messen
    \item Kritische Pfade identifizieren
    \item Lücken dokumentieren
\end{itemize}

\textbf{2. Priorisierung}
\begin{itemize}
    \item Geschäftskritische Funktionen
    \item Fehleranfällige Bereiche
    \item Komplexe Algorithmen
\end{itemize}

\textbf{3. Ergänzung der Tests}
\begin{itemize}
    \item Randfall-Tests
    \item Negativtests
    \item Performance-Tests
\end{itemize}

\textbf{4. Wartung}
\begin{itemize}
    \item Regelmäßige Überprüfung
    \item Anpassung an Änderungen
    \item Entfernung veralteter Tests
\end{itemize}
\end{KR}

\begin{example2}{Prüfungsaufgabe: Testfälle entwerfen}\\
\textbf{Aufgabe:}
Entwickeln Sie Testfälle für eine Methode zur Validierung einer Email-Adresse.

\textbf{Testfälle:}
\begin{itemize}
    \item \textbf{Positive Tests:}
    \begin{itemize}
        \item Standard Email (user@domain.com)
        \item Subdomain (user@sub.domain.com)
        \item Mit Punkten (first.last@domain.com)
    \end{itemize}
    
    \item \textbf{Negative Tests:}
    \begin{itemize}
        \item Fehlende @ (userdomain.com)
        \item Mehrere @ (user@@domain.com)
        \item Ungültige Zeichen (user\#@domain.com)
    \end{itemize}
    
    \item \textbf{Randfälle:}
    \begin{itemize}
        \item Leerer String
        \item Nur Whitespace
        \item Sehr lange Adressen
    \end{itemize}
\end{itemize}
\end{example2}

\begin{example2}{Common Testing Pitfalls}\\
\textbf{1. Flaky Tests}
\begin{lstlisting}[language=Java, style=basesmol]
// Schlecht: Test koennte fehlschlagen
@Test
public void testAsyncOperation() throws Exception {
   service.startAsyncOperation();
   Thread.sleep(1000); // Fragil!
   assertTrue(service.isOperationComplete());
}
// Besser: Explizites Warten
@Test
public void testAsyncOperation() {
   service.startAsyncOperation();
   await()
       .atMost(5, TimeUnit.SECONDS)
       .until(() -> service.isOperationComplete());
   assertTrue(service.isOperationComplete());
}
\end{lstlisting}

\textbf{2. Zu viele Mocks}
\begin{lstlisting}[language=Java, style=basesmol]
// Schlecht: Uebermassiges Mocking
@Test
public void testOverMocked() {
   Database db = mock(Database.class);
   Logger logger = mock(Logger.class);
   EmailService email = mock(EmailService.class);
   UserService users = mock(UserService.class);
   
   // Test wird unuebersichtlich und fragil
}

// Besser: Sociable Unit Testing
@Test
public void testSociable() {
   // Nur externe Abhaengigkeiten mocken
   EmailService email = mock(EmailService.class);
   UserService service = new UserService(email);
   
   User user = service.createUser("test@test.com");
   assertNotNull(user.getId());
}
\end{lstlisting}
\end{example2}



\subsubsection{more Testing stuff}

\begin{KR}{Testing Frameworks and Tools}\\
\textbf{1. Unit Testing}
\begin{itemize}
   \item \textbf{JUnit:}
   \begin{itemize}
       \item Assertions für Vergleiche
       \item Test Lifecycle (@Before, @After)
       \item Test Suites organisieren
   \end{itemize}
   \item \textbf{Mockito:}
   \begin{itemize}
       \item Mocking von Abhängigkeiten
       \item Verifikation von Aufrufen
       \item Stubbing von Verhalten
   \end{itemize}
\end{itemize}
\end{KR}

\begin{example2}{Unit Testing Best Practices}\\
\textbf{FIRST Prinzipien:}
\begin{itemize}
   \item \textbf{Fast:} Tests müssen schnell ausführbar sein
   \item \textbf{Independent:} Tests dürfen nicht voneinander abhängen
   \item \textbf{Repeatable:} Gleiche Ergebnisse bei mehrfacher Ausführung
   \item \textbf{Self-validating:} Automatische Ergebnisüberprüfung
   \item \textbf{Timely:} Tests werden zeitnah geschrieben
\end{itemize}
\end{example2}

\begin{example2}{Test Coverage Analysis}\\
\textbf{1. Line Coverage}
\begin{lstlisting}[language=Java, style=basesmol]
public class PaymentProcessor {
   public boolean processPayment(Payment payment) {
       if (payment == null) {         // Covered
           return false;              // Covered
       }
       
       if (payment.getAmount() <= 0) { // Not covered
           return false;               // Not covered
       }
       
       return doProcessPayment(payment); // Covered
   }
}
\end{lstlisting}

\textbf{2. Branch Coverage}
\begin{lstlisting}[language=Java, style=basesmol]
@Test
public void testPaymentProcessing() {
   // Test 1: null payment
   assertFalse(processor.processPayment(null));
   
   // Test 2: valid payment
   Payment payment = new Payment(100.00);
   assertTrue(processor.processPayment(payment));
   
   // Missing: Test fuer payment.getAmount() <= 0
}
\end{lstlisting}
\end{example2}

\begin{KR}{Integration Testing}\\
\textbf{1. Teststrategie:}
\begin{itemize}
   \item Bottom-up Integration
   \item Top-down Integration
   \item Big Bang Integration
\end{itemize}

\textbf{2. Testumfang:}
\begin{itemize}
   \item Komponenteninteraktion
   \item Schnittstellentests
   \item Datenfluss
\end{itemize}

\textbf{3. Testaufbau:}
\begin{lstlisting}[language=Java, style=basesmol]
@SpringBootTest
public class OrderIntegrationTest {
   @Autowired
   private OrderService orderService;
   
   @Autowired
   private CustomerService customerService;
   
   @Test
   public void shouldProcessOrderEndToEnd() {
       // Setup test data
       Customer customer = customerService.createCustomer(
           "Test Customer", "test@example.com");
       
       Order order = new Order(customer);
       order.addItem(new OrderItem("Test Product", 29.99, 1));
       
       // Execute process
       OrderResult result = orderService.processOrder(order);
       
       // Verify results
       assertTrue(result.isSuccess());
       assertEquals("COMPLETED", result.getStatus());
       // Verify database state
       Order savedOrder = orderService.findOrder(result.getOrderId());
       assertEquals("COMPLETED", savedOrder.getStatus());
   }
}
\end{lstlisting}
\end{KR}

\begin{KR}{Continuous Integration Testing}\\
\textbf{1. Build Pipeline}
\begin{itemize}
   \item Compile und Unit Tests
   \item Integration Tests
   \item Code Quality Checks
   \item Security Scans
\end{itemize}

\textbf{2. Test Automatisierung}
\begin{itemize}
   \item Smoke Tests
   \item Regression Tests
   \item Performance Tests
\end{itemize}

\textbf{3. Reporting}
\begin{itemize}
   \item Test Coverage Reports
   \item Performance Metrics
   \item Error Logs
\end{itemize}
\end{KR}



\begin{KR}{System Testing}\\
\textbf{1. Funktionale Tests:}
\begin{itemize}
   \item End-to-End Szenarien
   \item Geschäftsprozesse
   \item Use Case Tests
\end{itemize}

\textbf{2. Nicht-funktionale Tests:}
\begin{itemize}
   \item Performance Tests
   \item Last Tests
   \item Sicherheitstests
   \item Usability Tests
\end{itemize}

\textbf{3. Dokumentation:}
\begin{itemize}
   \item Testfälle
   \item Testdaten
   \item Testergebnisse
   \item Fehlerberichte
\end{itemize}
\end{KR}


\begin{KR}{Performance Testing}\\
\textbf{1. Last Tests (Load Testing)}
\begin{itemize}
   \item Normales Benutzerverhalten simulieren
   \item Durchschnittliche Antwortzeiten messen
   \item Ressourcenverbrauch überwachen
\end{itemize}

\textbf{2. Stress Tests}
\begin{itemize}
   \item System an Grenzen bringen
   \item Verhalten bei Überlast prüfen
   \item Wiederherstellung testen
\end{itemize}

\textbf{3. Endurance Tests}
\begin{itemize}
   \item Langzeitverhalten prüfen
   \item Memory Leaks aufdecken
   \item Ressourcenverbrauch über Zeit
\end{itemize}

\textbf{Beispiel JMeter Test Plan:}
\begin{lstlisting}[language=Java, style=basesmol]
// Thread Group Configuration
Number of Threads (Users): 100
Ramp-up Period: 60 seconds
Loop Count: 10

// HTTP Request
GET /api/products
Headers:
 Content-Type: application/json
 Authorization: Bearer ${token}

// Response Assertions
Response Code: 200
Response Time: < 500ms
\end{lstlisting}
\end{KR}







