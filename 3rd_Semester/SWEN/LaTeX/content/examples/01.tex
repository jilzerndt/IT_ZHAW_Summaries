\section{KR und Beispiele für Anforderungsanalyse}

\begin{KR}{Use Case Erstellung}\\
\textbf{Vorgehen bei der Erstellung eines vollständigen Use Cases:}
\begin{enumerate}
    \item \textbf{Identifikation:}
    \begin{itemize}
        \item Primärakteur bestimmen
        \item Scope festlegen
        \item Ebene definieren (Summary, User-Goal, Subfunction)
    \end{itemize}
    
    \item \textbf{Stakeholder und Interessen:}
    \begin{itemize}
        \item Alle betroffenen Parteien identifizieren
        \item Interessen pro Stakeholder beschreiben
        \item Priorisierung der Interessen
    \end{itemize}
    
    \item \textbf{Bedingungen:}
    \begin{itemize}
        \item Vorbedingungen definieren
        \item Nachbedingungen (Erfolgsfall)
        \item Minimalgarantien festlegen
    \end{itemize}
    
    \item \textbf{Standardablauf:}
    \begin{itemize}
        \item Schritte durchnummerieren
        \item Akteur-System Interaktion
        \item Klare, aktive Formulierung
    \end{itemize}
    
    \item \textbf{Erweiterungen/Alternativen:}
    \begin{itemize}
        \item Fehlerfälle identifizieren
        \item Alternative Abläufe beschreiben
        \item Auf Standardablauf referenzieren
    \end{itemize}
\end{enumerate}
\end{KR}

\begin{KR}{Usability-Requirements analysieren}\\
    \small
\textbf{Analyseschritte:}
\begin{enumerate}
    \item \textbf{Benutzergruppen identifizieren:}
    \begin{itemize}
        \item Primäre/sekundäre Nutzer
        \item Erfahrungsniveau
        \item Nutzungskontext
    \end{itemize}
    
    \item \textbf{Anforderungen nach ISO 9241-110:}
    \begin{itemize}
        \item Aufgabenangemessenheit
        \item Selbstbeschreibungsfähigkeit
        \item Steuerbarkeit
        \item Erwartungskonformität
        \item Fehlertoleranz
        \item Individualisierbarkeit
        \item Lernförderlichkeit
    \end{itemize}
    
    \item \textbf{Messbare Kriterien definieren:}
    \begin{itemize}
        \item Erfolgsrate bei Aufgaben
        \item Bearbeitungszeit
        \item Fehlerrate
        \item Nutzerzufriedenheit
    \end{itemize}
\end{enumerate}
\end{KR}

\begin{example2}{Usability-Analyse: Online-Banking}\\
    \small
\textbf{Aufgabe:} Analysieren Sie die Usability-Anforderungen für eine Online-Banking App.

\textbf{Analyse nach ISO 9241-110:}
\begin{itemize}
    \item \textbf{Aufgabenangemessenheit:}
    \begin{itemize}
        \item Schneller Zugriff auf häufige Funktionen
        \item Klare Übersicht über Kontostände
        \item Effiziente Überweisungsprozesse
    \end{itemize}
    
    \item \textbf{Selbstbeschreibungsfähigkeit:}
    \begin{itemize}
        \item Klare Status-Anzeigen
        \item Verständliche Fehlermeldungen
        \item Hilfe-Funktion
    \end{itemize}
    
    \item \textbf{Fehlertoleranz:}
    \begin{itemize}
        \item Bestätigung bei kritischen Aktionen
        \item Korrekturmöglichkeiten
        \item Plausibilitätsprüfungen
    \end{itemize}
\end{itemize}

\textbf{Messbare Kriterien:}
\begin{itemize}
    \item Überweisung in < 60 Sekunden
    \item Fehlerrate < 1\%
    \item Nutzerzufriedenheit > 4/5
\end{itemize}
\end{example2}

\begin{example2}{Fully-dressed Use Case}\\
\textbf{Aufgabe:} Schreiben Sie einen vollständigen Use Case für "Ticket buchen" in einem Hotelreservierungssystem.

\textbf{Use Case:} Hotelzimmer buchen

\textbf{Scope:} Hotelbuchungssystem

\textbf{Level:} User Goal

\textbf{Primary Actor:} Hotel-Gast

\textbf{Stakeholder und Interessen:}
\begin{itemize}
    \item \textbf{Hotel-Gast:}
    \begin{itemize}
        \item Schnelle, einfache Buchung
        \item Bestätigung der Buchung
        \item Korrekte Preisberechnung
    \end{itemize}
    \item \textbf{Hotel:}
    \begin{itemize}
        \item Korrekte Zimmerbelegung
        \item Zahlungsgarantie
        \item Vollständige Gästeinformationen
    \end{itemize}
\end{itemize}

\textbf{Vorbedingungen:}
\begin{itemize}
    \item Gast ist im System angemeldet
    \item Mindestens ein Zimmer verfügbar
\end{itemize}

\textbf{Nachbedingungen:}
\begin{itemize}
    \item Buchung ist gespeichert
    \item Zimmer ist reserviert
    \item Bestätigung ist versendet
\end{itemize}

\textbf{Standardablauf:}
\begin{enumerate}
    \item Gast wählt Reisedaten aus
    \item System zeigt verfügbare Zimmer
    \item Gast wählt Zimmer aus
    \item System zeigt Buchungsdetails und Gesamtpreis
    \item Gast gibt Zahlungsinformationen ein
    \item System validiert Zahlungsdaten
    \item System bestätigt Buchung
    \item System sendet Buchungsbestätigung
\end{enumerate}

\textbf{Erweiterungen:}
\begin{itemize}
    \item 2a. Keine Zimmer verfügbar:
    \begin{enumerate}
        \item System zeigt alternative Daten
        \item Gast wählt neue Daten oder bricht ab
    \end{enumerate}
    \item 6a. Zahlung fehlgeschlagen:
    \begin{enumerate}
        \item System zeigt Fehlermeldung
        \item Gast kann neue Zahlungsdaten eingeben oder abbrechen
    \end{enumerate}
\end{itemize}
\end{example2}

\begin{KR}{Systemsequenzdiagramme erstellen}\\
\textbf{Vorgehen für SSD:}
\begin{enumerate}
    \item \textbf{Akteure identifizieren:}
    \begin{itemize}
        \item Primärakteur festlegen
        \item System als Black Box
        \item Zeitachse definieren
    \end{itemize}
    
    \item \textbf{Operationen definieren:}
    \begin{itemize}
        \item Systemoperationen identifizieren
        \item Parameter festlegen
        \item Rückgabewerte bestimmen
    \end{itemize}
    
    \item \textbf{Ablauf modellieren:}
    \begin{itemize}
        \item Nachrichten einzeichnen
        \item Alternative Pfade markieren
        \item Schleifen kennzeichnen
    \end{itemize}
    
    \item \textbf{Dokumentation:}
    \begin{itemize}
        \item Beschriftungen prüfen
        \item Alternatives ergänzen
        \item Bezug zum Use Case herstellen
    \end{itemize}
\end{enumerate}
\end{KR}

\begin{example2}{Systemsequenzdiagramm: Hotelbuchung}\\
\textbf{Aufgabe:} Erstellen Sie ein SSD für den Use Case "Hotelzimmer buchen".

\textbf{Systemoperationen:}
\begin{lstlisting}[language=Java, style=base]
// Verfuegbarkeit pruefen
checkAvailability(dates: DateRange): List<Room>

// Zimmer reservieren
bookRoom(roomId: RoomId, 
         dates: DateRange): Reservation

// Zahlung durchfuehren
processPayment(reservationId: ReservationId,
              paymentInfo: PaymentDetails): boolean

// Buchung bestaetigen
confirmBooking(reservationId: ReservationId): 
    BookingConfirmation
\end{lstlisting}

\textbf{Alternative Pfade:}
\begin{itemize}
    \item Keine Verfügbarkeit → Alternative Daten
    \item Zahlungsfehler → Neue Zahlungsdaten
    \item Systembuchungsfehler → Fehlermeldung
\end{itemize}
\end{example2}

\begin{KR}{Contracts für Systemoperationen}\\
\textbf{Contract-Elemente:}
\begin{enumerate}
    \item \textbf{Operation:}
    \begin{itemize}
        \item Name und Parameter
        \item Rückgabetyp
        \item Exceptions
    \end{itemize}
    
    \item \textbf{Vorbedingungen:}
    \begin{itemize}
        \item Systemzustand
        \item Gültige Parameter
        \item Benutzerkontext
    \end{itemize}
    
    \item \textbf{Nachbedingungen:}
    \begin{itemize}
        \item Zustandsänderungen
        \item Objekterzeugung
        \item Attributänderungen
        \item Assoziationen
    \end{itemize}
\end{enumerate}
\end{KR}

\begin{example2}{Contract: Hotelbuchung}\\
\textbf{Operation:} bookRoom(roomId: RoomId, dates: DateRange): Reservation

\textbf{Querverweis:} UC "Hotelzimmer buchen"

\textbf{Vorbedingungen:}
\begin{itemize}
    \item Benutzer ist authentifiziert
    \item Zimmer ist im spezifizierten Zeitraum verfügbar
    \item Zimmer existiert
\end{itemize}

\textbf{Nachbedingungen:}
\begin{itemize}
    \item Reservierungs-Instanz wurde erstellt
    \item Reservierung ist mit Zimmer verknüpft
    \item Zimmer ist als reserviert markiert
    \item Reservierungszeitraum ist gesetzt
    \item Benutzer ist mit Reservierung verknüpft
\end{itemize}
\end{example2}

