\section{Examples}

\subsection{Software Engineering}

\begin{example2}{Typische Prüfungsaufgabe: Prozessmodelle vergleichen}\\
Vergleichen Sie das Wasserfallmodell mit einem iterativ-inkrementellen Ansatz anhand folgender Kriterien:
\begin{itemize}
    \item Umgang mit sich ändernden Anforderungen
    \item Risikomanagement
    \item Planbarkeit
    \item Kundeneinbindung
\end{itemize}

Musterlösung:
\begin{itemize}
    \item Wasserfall:
    \begin{itemize}
        \item Änderungen schwierig zu integrieren
        \item Risiken erst spät erkennbar
        \item Gut planbar durch feste Phasen
        \item Kunde hauptsächlich am Anfang und Ende involviert
    \end{itemize}
    \item Iterativ-inkrementell:
    \begin{itemize}
        \item Flexibel bei Änderungen
        \item Frühes Erkennen von Risiken
        \item Planung pro Iteration
        \item Kontinuierliches Kundenfeedback
    \end{itemize}
\end{itemize}
\end{example2}

\begin{KR}{Modellierungsumfang bestimmen}\\
Folgende Fragen zur Bestimmung des notwendigen Modellierungsumfangs:
\begin{itemize}
    \item Wie komplex ist die Problemstellung?
    \item Wie viele Stakeholder sind involviert?
    \item Wie kritisch ist das System?
    \item Analogie: Planung einer Hundehütte vs. Haus vs. Wolkenkratzer
\end{itemize}
\end{KR}

\begin{example2}{Prüfungsfrage zur Modellierung}\\
Erklären Sie anhand eines selbst gewählten Beispiels, warum der Modellierungsaufwand je nach Projekt stark variieren kann. Nennen Sie mindestens drei Faktoren, die den Modellierungsumfang beeinflussen.
\vspace{3mm}\\
Mögliche Antwort:
\begin{itemize}
    \item Beispiel: Entwicklung einer Smartphone-App vs. Medizinisches Gerät
    \item Faktoren:
    \begin{itemize}
        \item Komplexität der Domäne
        \item Regulatorische Anforderungen
        \item Anzahl beteiligter Stakeholder
        \item Sicherheitsanforderungen
    \end{itemize}
\end{itemize}
\end{example2}

\subsection{Architekturmuster}

\begin{KR}{Event-Bus Pattern}\\
\textbf{Implementierung eines Event-Bus Systems:}

\textbf{1. Event-Bus}

\begin{itemize}
    \item Publisher-Subscriber Mechanismus implementieren
    \item Event-Routing einrichten
    \item Event-Persistenz berücksichtigen
    \item Ordering und Filtering ermöglichen
\end{itemize}

\textbf{2. Publisher}

\begin{itemize}
    \item Event-Typen definieren
    \item Event-Publikation implementieren
    \item Transaktionshandling berücksichtigen
    \item Fehlerbehandlung vorsehen
\end{itemize}

\textbf{3. Subscriber}

\begin{itemize}
    \item Event-Handler implementieren
    \item Idempotenz sicherstellen
    \item Fehlertoleranz einbauen
    \item Dead-Letter-Queue vorsehen
\end{itemize}
\end{KR}

\begin{example2}{Event-Bus Implementation}
    %Keine Umlaute in lstlisting blocks!! e.g. replace ä with ae
\begin{lstlisting}[language=Java, style=basesmol]
// Event Bus
public class EventBus {
    private Map<Class<?>, List<EventSubscriber>> subscribers = new HashMap<>();
    
    public void publish(Event event) {
        List<EventSubscriber> eventSubscribers = subscribers
            .getOrDefault(event.getClass(), Collections.emptyList());
            
        for (EventSubscriber subscriber : eventSubscribers) {
            try {
                subscriber.onEvent(event);
            } catch (Exception e) {
                handleSubscriberError(subscriber, event, e);
            }
        }
    }
    public void subscribe(Class<?> eventType, EventSubscriber subscriber) {
        subscribers.computeIfAbsent(eventType, k -> new ArrayList<>())
                  .add(subscriber);
    }
}
// Publisher
public class OrderService {
    private EventBus eventBus;
    
    public void createOrder(OrderRequest request) {
        Order order = orderRepository.save(
            new Order(request));
            
        eventBus.publish(new OrderCreatedEvent(
            order.getId(),
            order.getCustomerId(),
            order.getTotalAmount(),
            LocalDateTime.now()
        ));
    }
}
// Subscriber
@Component
public class NotificationService implements EventSubscriber {
    private ProcessedEventRepository processedEvents;
    
    @Override
    public void onEvent(Event event) {
        if (!(event instanceof OrderCreatedEvent)) return;
        
        String eventId = event.getId();
        if (processedEvents.exists(eventId)) return;
        
        try {
            sendNotification((OrderCreatedEvent) event);
            processedEvents.save(eventId);
        } catch (Exception e) {
            sendToDeadLetterQueue(event, e);
        }
    }
}
\end{lstlisting}
\end{example2}

\subsection{Framework Design}

\begin{KR}{Framework Design Principles}

\begin{minipage}[t]{0.5\textwidth}
\textbf{1. Abstraktionsebenen definieren}
\begin{itemize}
    \item \textbf{Core API:}
    \begin{itemize}
        \item Zentrale Interfaces
        \item Hauptfunktionalität
        \item Erweiterungspunkte
    \end{itemize}
    
    \item \textbf{Extensions:}
    \begin{itemize}
        \item Plugin-Mechanismen
        \item Callback-Interfaces
        \item Event-Systeme
    \end{itemize}
    
    \item \textbf{Implementierung:}
    \begin{itemize}
        \item Standard-Implementierungen
        \item Utility-Klassen
        \item Helper-Funktionen
    \end{itemize}
\end{itemize}
\end{minipage}
\begin{minipage}[t]{0.5\textwidth}
\textbf{2. Erweiterungsmechanismen}
\begin{itemize}
    \item \textbf{Interface-basiert:}
    \begin{itemize}
        \item Klare Verträge
        \item Lose Kopplung
        \item Einfache Erweiterung
    \end{itemize}
    
    \item \textbf{Annotations:}
    \begin{itemize}
        \item Deklarative Konfiguration
        \item Metadaten-getrieben
        \item Runtime-Processing
    \end{itemize}
    
    \item \textbf{Composition:}
    \begin{itemize}
        \item Plugin-System
        \item Service-Loader
        \item Dependency Injection
    \end{itemize}
\end{itemize}
\end{minipage}
\end{KR}

\begin{KR}{Analyse von Framework-Anforderungen}

\begin{minipage}[t]{0.5\textwidth}
\textbf{1. Fachliche Analyse}
\begin{itemize}
    \item \textbf{Core Features:}
    \begin{itemize}
        \item Zentrale Funktionalität
        \item Gemeinsame Abstraktionen
        \item Standardverhalten
    \end{itemize}
    \item \textbf{Variationspunkte:}
    \begin{itemize}
        \item Kundenspezifische\\ Anpassungen
        \item Optionale Features
        \item Erweiterungsmöglichkeiten
    \end{itemize}
\end{itemize}
\end{minipage}
\begin{minipage}[t]{0.5\textwidth}
\textbf{2. Technische Analyse}
\begin{itemize}
    \item \textbf{Architektur-Entscheidungen:}
    \begin{itemize}
        \item Erweiterungsmechanismen
        \item Integration in\\ bestehende Systeme
        \item Schnittstellen-Design
    \end{itemize}
    \item \textbf{Qualitätsanforderungen:}
    \begin{itemize}
        \item Performance
        \item Wartbarkeit
        \item Testbarkeit
    \end{itemize}
\end{itemize}
\end{minipage}
\end{KR}


\begin{example2}{Prüfungsaufgabe: Framework-Analyse}\\
\textbf{Szenario:}
Ein Framework für die Verarbeitung verschiedener Dokumentformate (PDF, DOC, TXT) 
soll entwickelt werden.

\textbf{Aufgabe:}
Analysieren Sie die Design-Entscheidungen.

\textbf{Lösung:}
\begin{itemize}
    \item \textbf{Erweiterungspunkte:}
    \begin{itemize}
        \item Dokumenttyp-Erkennung
        \item Parser für Formate
        \item Konvertierungslogik
    \end{itemize}
    
    \item \textbf{Design Patterns:}
    \begin{itemize}
        \item Factory für Parser-Erzeugung
        \item Strategy für Verarbeitungsalgorithmen
        \item Template Method für Konvertierung
    \end{itemize}
    
    \item \textbf{Schnittstellen:}
    \begin{itemize}
        \item DocumentParser Interface
        \item ConversionStrategy Interface
        \item DocumentMetadata Klasse
    \end{itemize}
\end{itemize}
\end{example2}