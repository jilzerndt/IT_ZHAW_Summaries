\section{Kochrezepte und Übungen}

\subsection{Implementation, Refactoring and Testing}

\begin{KR}{Test-Driven Development (TDD)}\\
\textbf{Schritte:}
\begin{enumerate}
    \item \textbf{Red:} Test schreiben der fehlschlägt
    \begin{itemize}
        \item Testfall definieren
        \item Erwartetes Verhalten spezifizieren
        \item Test implementieren
    \end{itemize}
    
    \item \textbf{Green:} Minimale Implementation
    \begin{itemize}
        \item Nur das Nötigste implementieren
        \item Test soll grün werden
        \item Keine Optimierungen
    \end{itemize}
    
    \item \textbf{Refactor:} Code verbessern
    \begin{itemize}
        \item Code aufräumen
        \item Duplizierung entfernen
        \item Tests müssen grün bleiben
    \end{itemize}
\end{enumerate}
\end{KR}

\begin{KR}{Typische Refactoring Patterns}\\
\textbf{Methoden:}
\begin{itemize}
    \item \textbf{Extract Method:}
    \begin{itemize}
        \item Code in neue Methode auslagern
        \item Gemeinsame Funktionalität zusammenfassen
        \item Lesbarkeit verbessern
    \end{itemize}
    
    \item \textbf{Move Method:}
    \begin{itemize}
        \item Methode in andere Klasse verschieben
        \item Näher an verwendeten Daten
        \item Kohäsion verbessern
    \end{itemize}
    
    \item \textbf{Replace Conditional with Polymorphism:}
    \begin{itemize}
        \item Switch/if durch Vererbung ersetzen
        \item Flexibilität erhöhen
        \item Wartbarkeit verbessern
    \end{itemize}
\end{itemize}
\end{KR}

\begin{example2}[breakable]{Typische Prüfungsaufgabe: Code Refactoring}\\
\textbf{Aufgabe:} Refactoren Sie den folgenden Code unter Anwendung geeigneter Patterns

\textbf{Vorher:}
\begin{lstlisting}[language=Java, style=base]
public class Order {
    private List<OrderItem> items;
    private double totalAmount;
    private String status;
    
    public void calculateTotal() {
        totalAmount = 0;
        for(OrderItem item : items) {
            totalAmount += item.getPrice() * item.getQuantity();
            // Komplexe Rabattberechnung
            if(item.getQuantity() > 10) {
                totalAmount *= 0.9;  // 10% Rabatt
            }
            if(totalAmount > 1000) {
                totalAmount *= 0.95; // 5% Rabatt
            }
        }
    }
    
    public void processOrder() {
        calculateTotal();
        // Komplexe Statusberechnung
        if(totalAmount < 100) {
            status = "SMALL_ORDER";
        } else if(totalAmount < 1000) {
            status = "MEDIUM_ORDER";
        } else {
            status = "LARGE_ORDER";
        }
        // Weitere 20 Zeilen Status-Logik...
    }
}
\end{lstlisting}

\textbf{Nachher:}
\begin{lstlisting}[language=Java, style=base]
public class Order {
    private List<OrderItem> items;
    private OrderStatus status;
    private DiscountStrategy discountStrategy;
    
    public Money calculateTotal() {
        Money total = items.stream()
            .map(OrderItem::getSubtotal)
            .reduce(Money.ZERO, Money::add);
            
        return discountStrategy.applyDiscount(total);
    }
    
    public void processOrder() {
        Money total = calculateTotal();
        status = OrderStatus.forAmount(total);
        status.process(this);
    }
}

public interface DiscountStrategy {
    Money applyDiscount(Money amount);
}

public class QuantityBasedDiscount implements DiscountStrategy {
    public Money applyDiscount(Money amount) {
        if(quantity > 10) {
            return amount.multiply(0.9);
        }
        return amount;
    }
}

public enum OrderStatus {
    SMALL_ORDER(amount -> amount.isLessThan(Money.of(100))),
    MEDIUM_ORDER(amount -> amount.isLessThan(Money.of(1000))),
    LARGE_ORDER(amount -> true);
    
    private final Predicate<Money> matcher;
    
    public static OrderStatus forAmount(Money amount) {
        return Stream.of(values())
            .filter(s -> s.matcher.test(amount))
            .findFirst()
            .orElseThrow();
    }
    
    public void process(Order order) {
        // Status-spezifische Verarbeitung
    }
}
\end{lstlisting}

\textbf{Angewendete Refactorings:}
\begin{itemize}
    \item Extract Method für Berechnungslogik
    \item Strategy Pattern für Rabattberechnung
    \item State Pattern für Orderstatus
    \item Replace Conditional with Polymorphism
    \item Value Objects für Geldbeträge
\end{itemize}
\end{example2}

\begin{KR}{Unit Testing Best Practices}\\
\textbf{1. Test-Struktur (AAA):}
\begin{itemize}
    \item \textbf{Arrange:} Testdaten vorbereiten
    \item \textbf{Act:} Testobjekt ausführen
    \item \textbf{Assert:} Ergebnis prüfen
\end{itemize}

\textbf{2. Namenskonventionen:}
\begin{itemize}
    \item methodName\_testCase\_expectedResult
    \item should\_doSomething\_when\_condition
    \item given\_when\_then\ Format
\end{itemize}

\textbf{3. Coverage:}
\begin{itemize}
    \item Happy Path testen
    \item Edge Cases abdecken
    \item Fehlerfälle prüfen
\end{itemize}
\end{KR}

\begin{example2}[breakable]{Typische Prüfungsaufgabe: Unit Tests}\\
\textbf{Aufgabe:} Schreiben Sie Unit Tests für eine Warenkorb-Komponente

\begin{lstlisting}[language=Java, style=base]
public class ShoppingCartTest {
    private ShoppingCart cart;
    private Product testProduct;
    
    @BeforeEach
    void setUp() {
        cart = new ShoppingCart();
        testProduct = new Product("Test", Money.of(10));
    }
    
    @Test
    void shouldCalculateTotal_whenEmpty() {
        assertEquals(Money.ZERO, cart.getTotal());
    }
    
    @Test
    void shouldCalculateTotal_withOneItem() {
        cart.addItem(testProduct, 1);
        assertEquals(Money.of(10), cart.getTotal());
    }
    
    @Test 
    void shouldApplyQuantityDiscount_whenOver10Items() {
        cart.addItem(testProduct, 11);
        Money expected = Money.of(10 * 11 * 0.9);
        assertEquals(expected, cart.getTotal());
    }
    
    @Test
    void shouldThrowException_whenNegativeQuantity() {
        assertThrows(IllegalArgumentException.class, 
            () -> cart.addItem(testProduct, -1));
    }
}
\end{lstlisting}
\end{example2}

\begin{KR}{Code Review}\\
\textbf{Review Checklist:}
\begin{enumerate}
    \item \textbf{Funktionalität:}
    \begin{itemize}
        \item Anforderungen erfüllt?
        \item Edge Cases behandelt?
        \item Fehlerbehandlung korrekt?
    \end{itemize}
    
    \item \textbf{Code Qualität:}
    \begin{itemize}
        \item Clean Code Prinzipien
        \item SOLID Prinzipien
        \item Naming Conventions
    \end{itemize}
    
    \item \textbf{Tests:}
    \begin{itemize}
        \item Testabdeckung ausreichend?
        \item Tests aussagekräftig?
        \item Testfälle vollständig?
    \end{itemize}
    
    \item \textbf{Best Practices:}
    \begin{itemize}
        \item Design Patterns korrekt?
        \item Logging vorhanden?
        \item Dokumentation aktuell?
    \end{itemize}
\end{enumerate}
\end{KR}

\begin{example2}[breakable]{Code Review Szenario}\\
\textbf{Aufgabe:} Führen Sie ein Code Review für folgende Implementierung durch:

\begin{lstlisting}[language=Java, style=base]
// Original Code
public class DataManager {
    private static DataManager instance;
    private Connection conn;
    
    private DataManager() {
        try {
            conn = DriverManager.getConnection("db_url");
        } catch(Exception e) {
            e.printStackTrace();
        }
    }
    
    public static DataManager getInstance() {
        if(instance == null) {
            instance = new DataManager();
        }
        return instance;
    }
    
    public void saveData(String data) {
        try {
            Statement stmt = conn.createStatement();
            stmt.execute("INSERT INTO data VALUES('" + data + "')");
        } catch(Exception e) {
            System.out.println("Error: " + e.getMessage());
        }
    }
}
\end{lstlisting}

\textbf{Review Feedback:}
\begin{itemize}
    \item \textbf{Probleme:}
    \begin{itemize}
        \item Singleton nicht thread-safe
        \item SQL Injection Gefahr
        \item Schlechte Exception-Behandlung
        \item Keine Ressourcen-Freigabe
        \item Keine Konfigurierbarkeit
    \end{itemize}
    
    \item \textbf{Verbesserungsvorschläge:}
    \begin{itemize}
        \item Dependency Injection statt Singleton
        \item Prepared Statements verwenden
        \item Proper Exception Handling
        \item Try-with-resources für Statements
        \item Konfiguration externalisieren
    \end{itemize}
\end{itemize}

\textbf{Verbesserte Version:}
\begin{lstlisting}[language=Java, style=base]
@Component
public class DataManager implements AutoCloseable {
    private final DataSource dataSource;
    private final Logger logger = LoggerFactory.getLogger(DataManager.class);
    
    public DataManager(DataSource dataSource) {
        this.dataSource = dataSource;
    }
    
    public void saveData(String data) {
        String sql = "INSERT INTO data VALUES(?)";
        try (Connection conn = dataSource.getConnection();
             PreparedStatement stmt = conn.prepareStatement(sql)) {
            
            stmt.setString(1, data);
            stmt.execute();
            
        } catch (SQLException e) {
            logger.error("Failed to save data", e);
            throw new DataAccessException("Could not save data", e);
        }
    }
    
    @Override
    public void close() {
        // Cleanup code
    }
}
\end{lstlisting}
\end{example2}