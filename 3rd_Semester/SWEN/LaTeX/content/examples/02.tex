\section{KR und Beispiele für Domänenmodellierung}

\begin{KR}{Domänenmodell erstellen}\\
\textbf{1. Konzepte identifizieren}
\begin{itemize}
    \item \textbf{Substantive aus Text extrahieren:}
    \begin{itemize}
        \item Physische oder virtuelle Objekte
        \item Rollen und Akteure
        \item Ereignisse und Transaktionen
        \item Kataloge und Spezifikationen
    \end{itemize}
    
    \item \textbf{Konzeptkategorien prüfen:}
    \begin{itemize}
        \item Geschäftsobjekte
        \item Container/Sammlungen
        \item Beschreibungen/Spezifikationen
        \item Orte/Standorte
        \item Transaktionen/Ereignisse
    \end{itemize}
\end{itemize}

\textbf{2. Attribute zuordnen}
\begin{itemize}
    \item \textbf{Attributregeln:}
    \begin{itemize}
        \item Nur atomare Werte
        \item Keine abgeleiteten Attribute
        \item Keine IDs als Attribute
        \item Keine Referenzen als Attribute
    \end{itemize}
    
    \item \textbf{Typische Attribute:}
    \begin{itemize}
        \item Beschreibende Eigenschaften
        \item Status und Zustände
        \item Mengen und Werte
        \item Zeitangaben
    \end{itemize}
\end{itemize}

\textbf{3. Beziehungen modellieren}
\begin{itemize}
    \item \textbf{Assoziationstypen:}
    \begin{itemize}
        \item Einfache Assoziation
        \item Aggregation/Komposition
        \item Vererbung
    \end{itemize}
    
    \item \textbf{Multiplizitäten festlegen:}
    \begin{itemize}
        \item 1:1, 1:n, n:m
        \item Optionalität (0..1)
        \item Mindest-/Maximalwerte
    \end{itemize}
\end{itemize}
\end{KR}

\begin{example2}{Domänenmodell: Restaurant-System}\\
\textbf{Aufgabentext aus SEP-Muster:}
Eine Restaurantkette möchte ihr Bestellsystem modernisieren. Gäste bestellen Speisen und Getränke, die von der Küche bzw. Bar zubereitet werden. Bestellungen werden pro Tisch gesammelt und später gemeinsam oder getrennt bezahlt.

\textbf{Konzept-Analyse:}
\begin{itemize}
    \item \textbf{Physische Objekte:}
    \begin{itemize}
        \item Tisch
        \item Speisen/Getränke
    \end{itemize}
    
    \item \textbf{Rollen:}
    \begin{itemize}
        \item Gast
        \item Servicepersonal
        \item Koch/Barkeeper
    \end{itemize}
    
    \item \textbf{Transaktionen:}
    \begin{itemize}
        \item Bestellung
        \item Bezahlung
    \end{itemize}
\end{itemize}

\textbf{Domänenmodell:}
%include image of the domain model here


\textbf{Begründung der Modellierungsentscheidungen:}
\begin{itemize}
    \item \textbf{Beschreibungsklassen:}
    \begin{itemize}
        \item MenueItem für Speisen/Getränke (Trennung von konkreten Bestellpositionen)
        \item Produktkategorien für Gruppierung
    \end{itemize}
    
    \item \textbf{Aggregationen:}
    \begin{itemize}
        \item Bestellung aggregiert Bestellpositionen
        \item Tisch aggregiert Plätze
    \end{itemize}
    
    \item \textbf{Beziehungen:}
    \begin{itemize}
        \item Gast sitzt an Platz (1:1)
        \item Bestellung gehört zu Tisch (n:1)
        \item Position referenziert MenueItem (n:1)
    \end{itemize}
\end{itemize}
\end{example2}

\begin{KR}{Analysemuster anwenden}\\
\textbf{1. Beschreibungsklassen}
\begin{itemize}
    \item \textbf{Anwendung bei:}
    \begin{itemize}
        \item Trennung von Typ und Instanz
        \item Gemeinsame unveränderliche Eigenschaften
        \item Mehrere gleichartige Objekte
    \end{itemize}
    
    \item \textbf{Implementierung:}
    \begin{itemize}
        \item Beschreibungsklasse für Typinformationen
        \item Instanzklasse für konkrete Objekte
        \item 1:n Beziehung zwischen beiden
    \end{itemize}
\end{itemize}

\textbf{2. Zustandsmodellierung}
\begin{itemize}
    \item \textbf{Anwendung bei:}
    \begin{itemize}
        \item Komplexe Zustandsübergänge
        \item Zustandsabhängiges Verhalten
        \item Viele verschiedene Status
    \end{itemize}
    
    \item \textbf{Implementierung:}
    \begin{itemize}
        \item Abstrakte Zustandsklasse
        \item Konkrete Zustände als Subklassen
        \item Assoziationen zum Hauptobjekt
    \end{itemize}
\end{itemize}

\textbf{3. Rollen}
\begin{itemize}
    \item \textbf{Anwendung bei:}
    \begin{itemize}
        \item Verschiedene Funktionen eines Objekts
        \item Dynamische Rollenzuordnung
        \item Unterschiedliche Verantwortlichkeiten
    \end{itemize}
    
    \item \textbf{Implementierung:}
    \begin{itemize}
        \item Rolleninterface oder abstrakte Klasse
        \item Konkrete Rollenklassen
        \item Assoziation zum Basisobjekt
    \end{itemize}
\end{itemize}
\end{KR}

\begin{example2}{Analysemuster: Bibliothekssystem}\\
\textbf{Aufgabentext:}
Modellieren Sie ein Bibliothekssystem mit Büchern, die in mehreren Exemplaren vorliegen können. Benutzer können Bücher ausleihen und reservieren.

\textbf{Musterlösung mit Analysemuster:}
\begin{itemize}
    \item \textbf{Beschreibungsklassen:}
    \begin{itemize}
        \item Book (Beschreibung: ISBN, Titel, Autor)
        \item BookCopy (Instanz: Inventarnummer, Status)
    \end{itemize}
    
    \item \textbf{Zustandsmodellierung:}
    \begin{itemize}
        \item LendingState (abstrakt)
        \item Available, Borrowed, Reserved als Subklassen
    \end{itemize}
    
    \item \textbf{Rollen:}
    \begin{itemize}
        \item Person (Basis)
        \item Member, Librarian als Rollen
    \end{itemize}
\end{itemize}

\textbf{Begründung der Muster:}
\begin{itemize}
    \item Beschreibungsklassen trennen unveränderliche Buchdaten von Exemplaren
    \item Zustandsmuster ermöglicht komplexe Statusübergänge und -validierung
    \item Rollenmuster erlaubt verschiedene Berechtigungen und Funktionen
\end{itemize}
\end{example2}

\begin{KR}{Typische Modellierungsfehler vermeiden}\\
\textbf{1. Konzeptuelle Fehler}
\begin{itemize}
    \item \textbf{Vermeiden:}
    \begin{itemize}
        \item Technische statt fachliche Klassen
        \item Prozesse als Klassen
        \item Operationen im Domänenmodell
    \end{itemize}
    
    \item \textbf{Richtig:}
    \begin{itemize}
        \item Fachliche Konzepte modellieren
        \item Prozesse durch Beziehungen
        \item Nur Attribute im Modell
    \end{itemize}
\end{itemize}

\textbf{2. Strukturelle Fehler}
\begin{itemize}
    \item \textbf{Vermeiden:}
    \begin{itemize}
        \item IDs als Attribute
        \item Referenzen als Attribute
        \item Redundante Attribute
    \end{itemize}
    
    \item \textbf{Richtig:}
    \begin{itemize}
        \item Assoziationen statt IDs
        \item Abgeleitete Attribute weglassen
        \item Informationen zentralisieren
    \end{itemize}
\end{itemize}
\end{KR}

\begin{example2}{Modellierungsfehler erkennen}\\
\textbf{Fehlerhaftes Modell:}
\begin{itemize}
    \item Klasse 'OrderManager' mit CRUD-Operationen
    \item Attribut 'customerID' statt Assoziation
    \item Klasse 'PaymentProcess' für Ablauf
    \item Operation 'calculateTotal()' in Order
\end{itemize}

\textbf{Korrigiertes Modell:}
\begin{itemize}
    \item Klasse 'Order' mit fachlichen Attributen
    \item Assoziation zwischen Order und Customer
    \item Payment als eigenständiges Konzept
    \item Keine Operationen im Modell
\end{itemize}

\textbf{Begründung:}
\begin{itemize}
    \item Technische Manager-Klassen gehören nicht ins Domänenmodell
    \item IDs werden durch Assoziationen ersetzt
    \item Prozesse werden durch Beziehungen und Status modelliert
    \item Operationen gehören ins Designmodell
\end{itemize}
\end{example2}