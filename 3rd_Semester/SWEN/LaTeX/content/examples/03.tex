\section{KR und Beispiele für Softwarearchitektur und Design}

\begin{KR}{Architekturanalyse durchführen}\\
\textbf{1. Requirements analysieren}
\begin{itemize}
    \item \textbf{Funktionale Anforderungen:}
    \begin{itemize}
        \item Use Cases gruppieren
        \item Systemgrenzen definieren
        \item Schnittstellen identifizieren
    \end{itemize}
    
    \item \textbf{Nicht-funktionale Anforderungen:}
    \begin{itemize}
        \item FURPS+ Kategorien prüfen
        \item ISO 25010 Qualitätsmerkmale
        \item Priorisierung vornehmen
    \end{itemize}
    
    \item \textbf{Randbedingungen:}
    \begin{itemize}
        \item Technische Constraints
        \item Organisatorische Vorgaben
        \item Budget und Zeitrahmen
    \end{itemize}
\end{itemize}

\textbf{2. Architekturentscheidungen:}
\begin{itemize}
    \item \textbf{Variationspunkte analysieren:}
    \begin{itemize}
        \item Veränderliche Komponenten
        \item Austauschbare Teile
        \item Erweiterungsmöglichkeiten
    \end{itemize}
    
    \item \textbf{Entscheidungen dokumentieren:}
    \begin{itemize}
        \item Problem beschreiben
        \item Alternativen evaluieren
        \item Entscheidung begründen
    \end{itemize}
\end{itemize}
\end{KR}

\begin{example2}[breakable]{Architekturanalyse: Hotelsystem}\\
    \small
\textbf{Aus Muster-SEP:} Analysieren Sie die Anforderungen zur Ansteuerung des automatischen Zimmerverschlusssystems (ZVS).

\textbf{Analyse:}
\begin{itemize}
    \item \textbf{System-Schnittstelle:}
    \begin{itemize}
        \item ZVS arbeitet autonom
        \item Benötigt nur Karten-ID und Zimmer-Zuordnung
        \item Hauslieferant für ZVS festgelegt
    \end{itemize}
    
    \item \textbf{Architektur-Entscheidung:}
    \begin{itemize}
        \item \textbf{Kein Variationspunkt nötig da:}
        \begin{itemize}
            \item Gleicher Lieferant für alle Hotels
            \item Schnittstelle stabil
            \item Keine Änderungen geplant
        \end{itemize}
        
        \item \textbf{Trotzdem Adaption empfohlen:}
        \begin{itemize}
            \item Technische Änderungen möglich
            \item Adapter-Pattern einfach umsetzbar
            \item Zukünftige Flexibilität
        \end{itemize}
    \end{itemize}
\end{itemize}
\end{example2}

\begin{KR}{Layered Architecture Design}\\
    \small
\textbf{1. Schichten definieren}
\begin{itemize}
    \item \textbf{Presentation Layer:}
    \begin{itemize}
        \item UI-Komponenten
        \item Controller
        \item View Models
    \end{itemize}
    
    \item \textbf{Application Layer:}
    \begin{itemize}
        \item Services
        \item Use Case Implementation
        \item Koordination
    \end{itemize}
    
    \item \textbf{Domain Layer:}
    \begin{itemize}
        \item Business Objects
        \item Domain Logic
        \item Domain Services
    \end{itemize}
    
    \item \textbf{Infrastructure Layer:}
    \begin{itemize}
        \item Persistence
        \item External Services
        \item Technical Services
    \end{itemize}
\end{itemize}

\textbf{2. Regeln definieren:}
\begin{itemize}
    \item Abhängigkeiten nur nach unten
    \item Schichten über Interfaces verbinden
    \item Domänenlogik isolieren
\end{itemize}
\end{KR}

\begin{example2}[breakable]{Schichtenarchitektur: SafeHome}\\
    \small
\textbf{Aus Muster-SEP:} Entwerfen Sie die Architektur für ein Sicherheitssystem.

\textbf{Schichtenmodell:}
\begin{lstlisting}[language=Java, style=basesmol]
// Presentation Layer
public class SecurityController {
    private SecurityService securityService;
    
    public void arm(String code) {
        securityService.armSystem(code);
    }
}

// Application Layer
public class SecurityService {
    private AlarmSystem alarmSystem;
    private CodeValidator validator;
    
    public void armSystem(String code) {
        if (validator.isValid(code)) {
            alarmSystem.arm();
        }
    }
}

// Domain Layer
public class AlarmSystem {
    private List<Sensor> sensors;
    private AlarmState state;
    
    public void arm() {
        validateSystemState();
        state = state.arm();
        activateSensors();
    }
}

// Infrastructure Layer
public class SensorRepository {
    public List<Sensor> getActiveSensors() {
        // DB access
    }
}
\end{lstlisting}
\end{example2}

\begin{KR}{Architekturdokumentation erstellen}\\
\textbf{1. Überblick}
\begin{itemize}
    \item \textbf{Systemkontext:}
    \begin{itemize}
        \item Externe Systeme
        \item Akteure/Benutzer
        \item Schnittstellen
    \end{itemize}
    
    \item \textbf{Architekturziele:}
    \begin{itemize}
        \item Qualitätsattribute
        \item Randbedingungen
        \item Designprinzipien
    \end{itemize}
\end{itemize}

\textbf{2. Entscheidungen}
\begin{itemize}
    \item \textbf{Architekturentscheidungen:}
    \begin{itemize}
        \item Problem/Kontext
        \item Alternativen
        \item Begründung
    \end{itemize}
    
    \item \textbf{Muster/Konzepte:}
    \begin{itemize}
        \item Verwendete Patterns
        \item Architekturstile
        \item Prinzipien
    \end{itemize}
\end{itemize}
\end{KR}

\begin{example2}[breakable]{Architekturdokumentation: SafeHome}\\
\textbf{Aus Muster-SEP:} Dokumentieren Sie die Architekturentscheidungen für das Sicherheitssystem.

\textbf{1. Systemkontext}
\begin{itemize}
    \item \textbf{Externe Systeme:}
    \begin{itemize}
        \item Sensoren (Bewegung, Feuer, Wasser)
        \item Aktoren (Alarm, Licht)
        \item Mobile Clients
    \end{itemize}
    
    \item \textbf{Schnittstellen:}
    \begin{itemize}
        \item REST API für Remote-Zugriff
        \item Hardware-Protokolle für Sensoren
        \item Event-Bus für Benachrichtigungen
    \end{itemize}
\end{itemize}

\textbf{2. Architekturmuster}
\begin{itemize}
    \item \textbf{State Pattern:}
    \begin{itemize}
        \item Für Systemzustände (Armed, Disarmed)
        \item Zustandsübergänge kontrollieren
        \item Zustandsspezifisches Verhalten
    \end{itemize}
    
    \item \textbf{Observer Pattern:}
    \begin{itemize}
        \item Für Sensor-Events
        \item Benachrichtigung bei Alarmen
        \item Lose Kopplung
    \end{itemize}
\end{itemize}

\textbf{Code-Beispiel:}
\begin{lstlisting}[language=Java, style=basesmol]
// State Pattern Implementation
public interface SystemState {
    void arm();
    void disarm(String code);
    void handleSensorTriggered(Sensor sensor);
}

public class ArmedState implements SystemState {
    private AlarmSystem system;
    
    @Override
    public void handleSensorTriggered(Sensor sensor) {
        if (sensor.requiresImmediateResponse()) {
            system.triggerAlarm();
        } else {
            system.startEntryTimer();
        }
    }
}

// Observer Pattern Implementation
public interface SensorListener {
    void onSensorTriggered(SensorEvent event);
}

public class AlarmSystem implements SensorListener {
    @Override
    public void onSensorTriggered(SensorEvent event) {
        currentState.handleSensorTriggered(
            event.getSensor());
    }
}
\end{lstlisting}
\end{example2}

\begin{KR}{Qualitätsanforderungen prüfen}\\
\textbf{1. ISO 25010 Kategorien}
\begin{itemize}
    \item \textbf{Performance:}
    \begin{itemize}
        \item Response Times
        \item Throughput
        \item Resource Usage
    \end{itemize}
    
    \item \textbf{Reliability:}
    \begin{itemize}
        \item Fault Tolerance
        \item Recovery
        \item Availability
    \end{itemize}
    
    \item \textbf{Security:}
    \begin{itemize}
        \item Authentication
        \item Authorization
        \item Data Protection
    \end{itemize}
\end{itemize}

\textbf{2. Szenario-basierte Bewertung}
\begin{itemize}
    \item \textbf{Szenario definieren:}
    \begin{itemize}
        \item Auslöser/Stimulus
        \item Umgebung/Kontext
        \item Erwartete Reaktion
    \end{itemize}
    
    \item \textbf{Architektur prüfen:}
    \begin{itemize}
        \item Mechanismen identifizieren
        \item Risiken bewerten
        \item Maßnahmen definieren
    \end{itemize}
\end{itemize}
\end{KR}

\begin{example2}[breakable]{Qualitätsszenarien: SafeHome}\\
\textbf{Performance-Szenario:}
\begin{itemize}
    \item \textbf{Stimulus:} Sensor meldet Einbruch
    \item \textbf{Umgebung:} System im Armed-Zustand
    \item \textbf{Response:} Alarm innerhalb 500ms
    \item \textbf{Messung:} Reaktionszeit < 500ms in 99.9%
\end{itemize}

\textbf{Verfügbarkeits-Szenario:}
\begin{itemize}
    \item \textbf{Stimulus:} Hardware-Komponente fällt aus
    \item \textbf{Umgebung:} Normalbetrieb
    \item \textbf{Response:} System bleibt funktionsfähig
    \item \textbf{Messung:} 99.99% Verfügbarkeit
\end{itemize}

\textbf{Architekturmaßnahmen:}
\begin{lstlisting}[language=Java, style=basesmol]
// Performance Optimization
@Component
public class SensorEventProcessor {
    private BlockingQueue<SensorEvent> eventQueue;
    private ThreadPoolExecutor executor;
    
    @Async
    public void processSensorEvent(SensorEvent event) {
        eventQueue.offer(event);
        executor.execute(() -> processEvent(event));
    }
}

// High Availability
public class RedundantAlarmSystem {
    private List<AlarmDevice> alarmDevices;
    
    public void triggerAlarm() {
        for (AlarmDevice device : alarmDevices) {
            try {
                device.activate();
                return; // Success
            } catch (DeviceFailureException e) {
                // Try next device
                continue;
            }
        }
    }
}
\end{lstlisting}
\end{example2}

\section{KR und Beispiele für Architekturmuster}

\begin{KR}{Architekturmuster auswählen}\\
\textbf{1. Pattern-Analyse}
\begin{itemize}
    \item \textbf{Problemkontext:}
    \begin{itemize}
        \item Art der Anwendung
        \item Verteilungsanforderungen
        \item Qualitätsattribute
    \end{itemize}
    
    \item \textbf{Pattern-Katalog durchsuchen:}
    \begin{itemize}
        \item Layered Architecture
        \item Client-Server
        \item Pipe-Filter
        \item Event-Bus
        \item Master-Slave
    \end{itemize}
    
    \item \textbf{Trade-offs evaluieren:}
    \begin{itemize}
        \item Vorteile/Nachteile
        \item Komplexität vs. Flexibilität
        \item Implementierungsaufwand
    \end{itemize}
\end{itemize}

\textbf{2. Implementierung planen}
\begin{itemize}
    \item \textbf{Struktur:}
    \begin{itemize}
        \item Komponenten definieren
        \item Schnittstellen festlegen
        \item Interaktionen beschreiben
    \end{itemize}
    
    \item \textbf{Qualitätssicherung:}
    \begin{itemize}
        \item Testbarkeit berücksichtigen
        \item Performance-Aspekte
        \item Wartbarkeit sicherstellen
    \end{itemize}
\end{itemize}
\end{KR}

\begin{example2}[breakable]{Event-Bus Pattern: SafeHome}\\
\textbf{Aufgabe:} Implementieren Sie das Event-Bus Pattern für die Sensor-Kommunikation im SafeHome System.

\begin{lstlisting}[language=Java, style=basesmol]
// Event Klasse
public class SensorEvent {
    private String sensorId;
    private SensorType type;
    private LocalDateTime timestamp;
    private Map<String, Object> data;
    
    // Constructor und Getter
}

// Event Bus
public class SecurityEventBus {
    private Map<Class<?>, List<EventHandler>> handlers = 
        new HashMap<>();
    
    public void publish(SensorEvent event) {
        List<EventHandler> eventHandlers = 
            handlers.getOrDefault(
                event.getClass(), 
                Collections.emptyList()
            );
        
        for (EventHandler handler : eventHandlers) {
            try {
                handler.handle(event);
            } catch (Exception e) {
                handleError(handler, event, e);
            }
        }
    }
    
    public void subscribe(Class<?> eventType, 
                         EventHandler handler) {
        handlers.computeIfAbsent(
            eventType, 
            k -> new ArrayList<>()
        ).add(handler);
    }
}

// Event Handler Implementation
@Component
public class MotionSensorHandler 
    implements EventHandler<MotionEvent> {
    
    private AlarmSystem alarmSystem;
    
    @Override
    public void handle(MotionEvent event) {
        if (alarmSystem.isArmed()) {
            evaluateMotion(event);
        }
    }
    
    private void evaluateMotion(MotionEvent event) {
        if (event.getIntensity() > THRESHOLD) {
            alarmSystem.startEntryTimer();
        }
    }
}
\end{lstlisting}
\end{example2}

\begin{KR}{Client-Server Pattern implementieren}\\
\textbf{1. Komponenten definieren}
\begin{itemize}
    \item \textbf{Server:}
    \begin{itemize}
        \item Service-Interfaces
        \item Request-Handling
        \item Resource-Management
    \end{itemize}
    
    \item \textbf{Client:}
    \begin{itemize}
        \item Service-Proxies
        \item Error-Handling
        \item UI/Interaktion
    \end{itemize}
    
    \item \textbf{Protokoll:}
    \begin{itemize}
        \item Nachrichten-Format
        \item Statushandling
        \item Security
    \end{itemize}
\end{itemize}

\textbf{2. Implementierungsaspekte}
\begin{itemize}
    \item \textbf{Kommunikation:}
    \begin{itemize}
        \item Synchron/Asynchron
        \item Serialisierung
        \item Fehlerbehandlung
    \end{itemize}
    
    \item \textbf{Skalierung:}
    \begin{itemize}
        \item Load Balancing
        \item Caching
        \item Connection Pooling
    \end{itemize}
\end{itemize}
\end{KR}

\begin{example2}[breakable]{Client-Server: SafeHome Mobile App}\\
\textbf{Aufgabe:} Implementieren Sie die Client-Server Architektur für die mobile SafeHome App.

\begin{lstlisting}[language=Java, style=basesmol]
// Server-Side Controller
@RestController
@RequestMapping("/api/v1/security")
public class SecurityController {
    private SecurityService securityService;
    
    @PostMapping("/arm")
    public ResponseEntity<SystemStatus> armSystem(
            @RequestBody ArmRequest request) {
        try {
            SystemStatus status = 
                securityService.armSystem(
                    request.getCode());
            return ResponseEntity.ok(status);
        } catch (InvalidCodeException e) {
            return ResponseEntity
                .status(HttpStatus.UNAUTHORIZED)
                .build();
        }
    }
    
    @GetMapping("/status")
    public ResponseEntity<SystemStatus> getStatus() {
        return ResponseEntity.ok(
            securityService.getCurrentStatus());
    }
}

// Client-Side Service
public class SecurityClient {
    private final WebClient webClient;
    private final String baseUrl;
    
    public Mono<SystemStatus> armSystem(String code) {
        return webClient.post()
            .uri(baseUrl + "/arm")
            .body(new ArmRequest(code))
            .retrieve()
            .bodyToMono(SystemStatus.class)
            .timeout(Duration.ofSeconds(5))
            .retry(3)
            .onErrorResume(this::handleError);
    }
    
    private Mono<SystemStatus> handleError(Throwable error) {
        if (error instanceof TimeoutException) {
            return Mono.error(
                new ConnectionException("Timeout"));
        }
        return Mono.error(error);
    }
}

// Mobile App Component
public class SecurityViewModel {
    private SecurityClient securityClient;
    private MutableLiveData<SystemStatus> status = 
        new MutableLiveData<>();
    
    public void armSystem(String code) {
        securityClient.armSystem(code)
            .subscribe(
                status::setValue,
                this::handleError
            );
    }
}
\end{lstlisting}
\end{example2}

\begin{KR}{Master-Slave Pattern implementieren}\\
\textbf{1. Komponenten}
\begin{itemize}
    \item \textbf{Master:}
    \begin{itemize}
        \item Aufgabenverteilung
        \item Ergebnissammlung
        \item Fehlerhandling
    \end{itemize}
    
    \item \textbf{Slaves:}
    \begin{itemize}
        \item Aufgabenausführung
        \item Statusmeldungen
        \item Autonome Arbeit
    \end{itemize}
\end{itemize}

\textbf{2. Koordination}
\begin{itemize}
    \item \textbf{Verteilung:}
    \begin{itemize}
        \item Load Balancing
        \item Resource Management
        \item Failover
    \end{itemize}
    
    \item \textbf{Monitoring:}
    \begin{itemize}
        \item Health Checks
        \item Performance Metrics
        \item Error Tracking
    \end{itemize}
\end{itemize}
\end{KR}

\begin{example2}[breakable]{Master-Slave: Sensor Processing}\\
\textbf{Aufgabe:} Implementieren Sie eine Master-Slave Architektur für die Verarbeitung von Sensordaten.

\begin{lstlisting}[language=Java, style=basesmol]
// Master Component
public class SensorMaster {
    private List<SensorSlave> slaves;
    private Map<String, SensorData> results = 
        new ConcurrentHashMap<>();
    
    public void processSensorData(List<SensorInput> inputs) {
        // Distribute work
        Map<SensorSlave, List<SensorInput>> distribution = 
            distributeWork(inputs);
            
        // Start processing
        List<Future<SensorData>> futures = 
            submitTasks(distribution);
            
        // Collect results
        collectResults(futures);
    }
    
    private Map<SensorSlave, List<SensorInput>> 
            distributeWork(List<SensorInput> inputs) {
        Map<SensorSlave, List<SensorInput>> distribution = 
            new HashMap<>();
            
        int slaveIndex = 0;
        for (SensorInput input : inputs) {
            SensorSlave slave = slaves.get(
                slaveIndex % slaves.size());
            distribution.computeIfAbsent(
                slave, 
                k -> new ArrayList<>()
            ).add(input);
            slaveIndex++;
        }
        
        return distribution;
    }
}

// Slave Component
public class SensorSlave {
    private String slaveId;
    private SensorProcessor processor;
    
    public SensorData process(SensorInput input) {
        try {
            return processor.process(input);
        } catch (Exception e) {
            handleError(e);
            return SensorData.error(input, e);
        }
    }
    
    public HealthStatus checkHealth() {
        return new HealthStatus(
            slaveId,
            processor.getStatus(),
            getResourceMetrics()
        );
    }
}

// Coordinator
public class SensorCoordinator {
    private SensorMaster master;
    private HealthMonitor monitor;
    
    @Scheduled(fixedRate = 5000)
    public void checkSlaveHealth() {
        List<HealthStatus> statuses = 
            master.getSlaveStatuses();
        monitor.updateHealth(statuses);
        
        // Handle unhealthy slaves
        List<String> unhealthySlaves = 
            findUnhealthySlaves(statuses);
        if (!unhealthySlaves.isEmpty()) {
            reconfigureMaster(unhealthySlaves);
        }
    }
}
\end{lstlisting}
\end{example2}

\section{KR und Beispiele für Architektur-Evaluation}

\begin{KR}{Architektur-Evaluation durchführen}\\
\textbf{1. Evaluationskriterien definieren}
\begin{itemize}
    \item \textbf{Qualitätsattribute:}
    \begin{itemize}
        \item Erfüllung der ISO 25010 Kriterien
        \item Priorisierte Qualitätsziele
        \item Messbare Metriken
    \end{itemize}
    
    \item \textbf{Business-Ziele:}
    \begin{itemize}
        \item Time-to-Market
        \item Entwicklungskosten
        \item Wartungsaufwand
    \end{itemize}
    
    \item \textbf{Technische Ziele:}
    \begin{itemize}
        \item Skalierbarkeit
        \item Integrationsfähigkeit
        \item Technologie-Stack
    \end{itemize}
\end{itemize}

\textbf{2. Evaluation durchführen}
\begin{itemize}
    \item \textbf{Szenarien prüfen:}
    \begin{itemize}
        \item Use Case Realisierung
        \item Qualitätsszenarien
        \item Änderungsszenarien
    \end{itemize}
    
    \item \textbf{Risiken analysieren:}
    \begin{itemize}
        \item Technische Risiken
        \item Abhängigkeiten
        \item Komplexität
    \end{itemize}
\end{itemize}
\end{KR}

\section{KR und Beispiele für Architektur-Evaluation}

\begin{KR}{Architektur-Evaluation durchführen}\\
\textbf{1. Evaluationskriterien definieren}
\begin{itemize}
    \item \textbf{Qualitätsattribute:}
    \begin{itemize}
        \item Erfüllung der ISO 25010 Kriterien
        \item Priorisierte Qualitätsziele
        \item Messbare Metriken
    \end{itemize}
    
    \item \textbf{Business-Ziele:}
    \begin{itemize}
        \item Time-to-Market
        \item Entwicklungskosten
        \item Wartungsaufwand
    \end{itemize}
    
    \item \textbf{Technische Ziele:}
    \begin{itemize}
        \item Skalierbarkeit
        \item Integrationsfähigkeit
        \item Technologie-Stack
    \end{itemize}
\end{itemize}

\textbf{2. Evaluation durchführen}
\begin{itemize}
    \item \textbf{Szenarien prüfen:}
    \begin{itemize}
        \item Use Case Realisierung
        \item Qualitätsszenarien
        \item Änderungsszenarien
    \end{itemize}
    
    \item \textbf{Risiken analysieren:}
    \begin{itemize}
        \item Technische Risiken
        \item Abhängigkeiten
        \item Komplexität
    \end{itemize}
\end{itemize}
\end{KR}

\begin{example2}[breakable]{Architektur-Evaluation: SafeHome}\\
\textbf{Aufgabe:} Evaluieren Sie die Architektur des SafeHome Systems bezüglich Zuverlässigkeit und Skalierbarkeit.

\textbf{1. Qualitätsszenarien}
\begin{itemize}
    \item \textbf{Zuverlässigkeit:}
    \begin{itemize}
        \item \textbf{Szenario:} Sensorausfall
        \item \textbf{Stimulus:} Sensor antwortet nicht
        \item \textbf{Response:} System bleibt funktionsfähig
        \item \textbf{Metrik:} 99.99% Verfügbarkeit
    \end{itemize}
    
    \item \textbf{Skalierbarkeit:}
    \begin{itemize}
        \item \textbf{Szenario:} Erhöhte Sensoranzahl
        \item \textbf{Stimulus:} 100 neue Sensoren
        \item \textbf{Response:} Performance stabil
        \item \textbf{Metrik:} < 500ms Reaktionszeit
    \end{itemize}
\end{itemize}

\textbf{2. Architektur-Review}
\begin{lstlisting}[language=Java, style=basesmol]
// Reliability Pattern Implementation
public class SensorManager {
    private Map<String, Sensor> sensors;
    private CircuitBreaker circuitBreaker;
    
    @Retry(maxAttempts = 3)
    public SensorStatus checkSensor(String sensorId) {
        return circuitBreaker.execute(() -> 
            sensors.get(sensorId)
                .getStatus());
    }
    
    public void handleSensorFailure(String sensorId) {
        // Deactivate failed sensor
        sensors.get(sensorId).deactivate();
        
        // Notify system
        notifySystemStatus(
            String.format(
                "Sensor %s failed, system operating in degraded mode",
                sensorId
            )
        );
        
        // Adjust monitoring
        updateMonitoringStrategy();
    }
}

// Scalability Pattern Implementation
public class SensorEventProcessor {
    private Queue<SensorEvent> eventQueue;
    private ThreadPoolExecutor executor;
    
    public SensorEventProcessor(int maxThreads) {
        this.eventQueue = new LinkedBlockingQueue<>();
        this.executor = new ThreadPoolExecutor(
            2, maxThreads,
            60L, TimeUnit.SECONDS,
            new LinkedBlockingQueue<>()
        );
    }
    
    public void processEvent(SensorEvent event) {
        executor.submit(() -> {
            try {
                handleEvent(event);
            } catch (Exception e) {
                handleProcessingError(event, e);
            }
        });
    }
    
    private void handleEvent(SensorEvent event) {
        // Process event based on type
        switch (event.getType()) {
            case MOTION:
                processMotionEvent((MotionEvent) event);
                break;
            case FIRE:
                processFireEvent((FireEvent) event);
                break;
            // ... other event types
        }
    }
}

\end{lstlisting}

\textbf{3. Evaluation Ergebnisse}
\begin{itemize}
    \item \textbf{Zuverlässigkeit:}
    \begin{itemize}
        \item + Circuit Breaker verhindert Kaskadeneffekte
        \item + Retry-Mechanismus für Stabilität
        \item + Degraded Mode für Teilausfälle
        \item - Komplexe Fehlerbehandlung nötig
    \end{itemize}
    
    \item \textbf{Skalierbarkeit:}
    \begin{itemize}
        \item + Asynchrone Event-Verarbeitung
        \item + Dynamischer Thread-Pool
        \item + Event-Queue als Buffer
        \item - Monitoring Overhead bei vielen Sensoren
    \end{itemize}
\end{itemize}
\end{example2}

\begin{KR}{Variationspunkte analysieren}\\
    \small
\textbf{1. Identifikation}
\begin{itemize}
    \item \textbf{Technische Variation:}
    \begin{itemize}
        \item Datenbankanbindung
        \item UI-Framework
        \item Protokolle
    \end{itemize}
    
    \item \textbf{Fachliche Variation:}
    \begin{itemize}
        \item Geschäftsregeln
        \item Workflows
        \item Berechnungen
    \end{itemize}
    
    \item \textbf{Konfiguration:}
    \begin{itemize}
        \item Parameter
        \item Grenzwerte
        \item Features
    \end{itemize}
\end{itemize}

\textbf{2. Design}
\begin{itemize}
    \item \textbf{Pattern-Auswahl:}
    \begin{itemize}
        \item Strategy Pattern
        \item Factory Pattern
        \item Plugin-System
    \end{itemize}
    
    \item \textbf{Interface-Design:}
    \begin{itemize}
        \item Abstraktion
        \item Stabilität
        \item Erweiterbarkeit
    \end{itemize}
\end{itemize}
\end{KR}

\begin{example2}[breakable][breakable]{Variationspunkte: SafeHome}\\
\textbf{Aufgabe:} Identifizieren und implementieren Sie Variationspunkte im SafeHome System.

\textbf{1. Analyse}
\begin{itemize}
    \item \textbf{Sensor-Protokolle:}
    \begin{itemize}
        \item Verschiedene Hersteller
        \item Unterschiedliche Kommunikationsprotokolle
        \item Erweiterbarkeit für neue Sensoren
    \end{itemize}
    
    \item \textbf{Alarmierung:}
    \begin{itemize}
        \item Multiple Benachrichtigungskanäle
        \item Konfigurierbare Eskalation
        \item Kundenspezifische Regeln
    \end{itemize}
\end{itemize}

\textbf{2. Implementation}
\begin{lstlisting}[language=Java, style=basesmol]
// Sensor Protocol Variation
public interface SensorProtocol {
    void initialize();
    SensorData readData();
    void sendCommand(Command cmd);
}

public class ZigBeeSensor implements SensorProtocol {
    @Override
    public void initialize() {
        // ZigBee specific initialization
    }
    
    @Override
    public SensorData readData() {
        // Read from ZigBee sensor
        return new SensorData(/* ... */);
    }
}

// Notification Variation
public interface NotificationChannel {
    void sendAlert(AlertLevel level, String message);
    boolean isAvailable();
    Priority getPriority();
}

public class NotificationService {
    private List<NotificationChannel> channels;
    private NotificationConfig config;
    
    public void notify(Alert alert) {
        // Sort channels by priority
        List<NotificationChannel> availableChannels = 
            channels.stream()
                .filter(NotificationChannel::isAvailable)
                .sorted(comparing(
                    NotificationChannel::getPriority))
                .collect(toList());
                
        // Try channels until successful
        for (NotificationChannel channel : availableChannels) {
            try {
                channel.sendAlert(
                    alert.getLevel(), 
                    alert.getMessage()
                );
                return;
            } catch (NotificationException e) {
                // Try next channel
                continue;
            }
        }
        
        // No channel available
        handleNotificationFailure(alert);
    }
}
\end{lstlisting}

\textbf{3. Bewertung}
\begin{itemize}
    \item \textbf{Vorteile:}
    \begin{itemize}
        \item Neue Sensor-Protokolle einfach integrierbar
        \item Flexible Alarmierungskonfiguration
        \item Ausfallsicherheit durch Multiple Channels
    \end{itemize}
    
    \item \textbf{Nachteile:}
    \begin{itemize}
        \item Erhöhte Komplexität
        \item Testing Overhead
        \item Configuration Management
    \end{itemize}
\end{itemize}
\end{example2}