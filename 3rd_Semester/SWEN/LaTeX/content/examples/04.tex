\section{KR und Beispiele für Use Case Realization}

\begin{KR}{Use Case Realization durchführen}\\
\textbf{1. Vorbereitung}
\begin{itemize}
    \item \textbf{Use Case analysieren:}
    \begin{itemize}
        \item Standardablauf identifizieren
        \item Wichtige Erweiterungen identifizieren
        \item Systemoperationen aus SSD extrahieren
    \end{itemize}
    
    \item \textbf{Domänenmodell prüfen:}
    \begin{itemize}
        \item Benötigte Klassen identifizieren
        \item Beziehungen validieren
        \item Fehlende Konzepte ergänzen
    \end{itemize}
\end{itemize}

\textbf{2. Design}
\begin{itemize}
    \item \textbf{Controller bestimmen:}
    \begin{itemize}
        \item Use Case oder Fassaden Controller
        \item Verantwortlichkeiten definieren
        \item Schnittstellen festlegen
    \end{itemize}
    
    \item \textbf{Interaktionen modellieren:}
    \begin{itemize}
        \item Sequenzdiagramm erstellen
        \item GRASP Prinzipien anwenden
        \item Patterns einsetzen
    \end{itemize}
\end{itemize}

\textbf{3. Implementation}
\begin{itemize}
    \item \textbf{Code strukturieren:}
    \begin{itemize}
        \item Package-Struktur
        \item Schichtenarchitektur
        \item Abhängigkeiten
    \end{itemize}
    
    \item \textbf{Tests erstellen:}
    \begin{itemize}
        \item Unit Tests
        \item Integrationstests
        \item Use Case Tests
    \end{itemize}
\end{itemize}
\end{KR}

\begin{example2}[breakable]{Use Case Realization: Forum}\\
\textbf{Use Case:} Neue Diskussion erstellen

\textbf{Systemoperationen:}
\begin{itemize}
    \item addNewDiscussion(title, content)
    \item getNumberOfContributions()
\end{itemize}

\begin{lstlisting}[language=Java, style=basesmol]
// Controller
public class ForumController {
    private ForumService forumService;
    private AccessValidator accessValidator;
    
    public Discussion addNewDiscussion(
            String topicId, 
            DiscussionRequest request) {
        // Validate access
        accessValidator.validateUserAccess(
            request.getUserId());
            
        // Create discussion
        return forumService.createDiscussion(
            topicId, 
            request.getTitle(),
            request.getContent()
        );
    }
}

// Domain Model
public class Discussion {
    private String title;
    private String content;
    private User author;
    private List<Comment> comments;
    private LocalDateTime createdAt;
    
    public void addComment(Comment comment) {
        validateComment(comment);
        comments.add(comment);
    }
    
    public int getContributionCount() {
        return 1 + comments.size(); // Discussion + Comments
    }
}

// Service Layer
public class ForumService {
    private DiscussionRepository repository;
    private TopicRepository topicRepository;
    
    @Transactional
    public Discussion createDiscussion(
            String topicId,
            String title,
            String content) {
        // Find topic
        Topic topic = topicRepository
            .findById(topicId)
            .orElseThrow(TopicNotFoundException::new);
            
        // Create discussion
        Discussion discussion = new Discussion(
            title, content);
            
        // Add to topic
        topic.addDiscussion(discussion);
        
        // Save
        return repository.save(discussion);
    }
}

// Unit Test
public class ForumControllerTest {
    @Mock
    private ForumService forumService;
    
    @Mock
    private AccessValidator accessValidator;
    
    @InjectMocks
    private ForumController controller;
    
    @Test
    public void shouldCreateNewDiscussion() {
        // Given
        DiscussionRequest request = new DiscussionRequest(
            "userId", "title", "content");
        
        // When
        controller.addNewDiscussion("topicId", request);
        
        // Then
        verify(accessValidator)
            .validateUserAccess("userId");
        verify(forumService)
            .createDiscussion(
                "topicId", "title", "content");
    }
}
\end{lstlisting}
\end{example2}

\begin{KR}{GRASP Prinzipien anwenden}\\
\textbf{1. Information Expert}
\begin{itemize}
    \item \textbf{Identifikation:}
    \begin{itemize}
        \item Welche Klasse hat die Information?
        \item Wo liegen die relevanten Daten?
        \item Wer kennt die Berechnungsgrundlagen?
    \end{itemize}
    
    \item \textbf{Anwendung:}
    \begin{itemize}
        \item Methoden dort platzieren wo die Daten sind
        \item Berechnungen in Expertenklasse
        \item Delegieren wenn nötig
    \end{itemize}
\end{itemize}

\textbf{2. Low Coupling}
\begin{itemize}
    \item \textbf{Analyse:}
    \begin{itemize}
        \item Abhängigkeiten identifizieren
        \item Kritische Kopplungen erkennen
        \item Alternatives Design prüfen
    \end{itemize}
    
    \item \textbf{Maßnahmen:}
    \begin{itemize}
        \item Interfaces einführen
        \item Dependency Injection
        \item Vermittler einsetzen
    \end{itemize}
\end{itemize}

\textbf{3. High Cohesion}
\begin{itemize}
    \item \textbf{Prüfung:}
    \begin{itemize}
        \item Zusammengehörigkeit der Methoden
        \item Fokus der Klasse
        \item Aufgabenteilung
    \end{itemize}
    
    \item \textbf{Verbesserung:}
    \begin{itemize}
        \item Klassen aufteilen
        \item Verantwortlichkeiten gruppieren
        \item Hilfsklassen einführen
    \end{itemize}
\end{itemize}
\end{KR}

\begin{example2}[breakable]{GRASP Anwendung: Online Shop}\\
\textbf{Use Case:} Bestellung aufgeben

\begin{lstlisting}[language=Java, style=basesmol]
// Information Expert: Order berechnet eigenen Gesamtbetrag
public class Order {
    private List<OrderLine> lines;
    private Customer customer;
    
    public Money calculateTotal() {
        return lines.stream()
            .map(OrderLine::getSubtotal)
            .reduce(Money.ZERO, Money::add);
    }
}

// Low Coupling: Interface statt konkreter Implementierung
public interface PaymentGateway {
    PaymentResult processPayment(Money amount);
}

public class OrderService {
    private final PaymentGateway paymentGateway;
    
    public OrderResult createOrder(OrderRequest request) {
        Order order = createOrderFromRequest(request);
        
        PaymentResult payment = 
            paymentGateway.processPayment(
                order.calculateTotal());
                
        if (payment.isSuccessful()) {
            return OrderResult.success(order);
        } else {
            return OrderResult.failed(payment.getReason());
        }
    }
}

// High Cohesion: Spezialisierte Services
public class OrderValidator {
    public void validate(Order order) {
        validateCustomer(order.getCustomer());
        validateOrderLines(order.getLines());
        validateDeliveryAddress(
            order.getDeliveryAddress());
    }
}

public class InventoryService {
    public void reserveStock(Order order) {
        for (OrderLine line : order.getLines()) {
            reserveItem(
                line.getProduct(), 
                line.getQuantity());
        }
    }
}
\end{lstlisting}
\end{example2}

\begin{KR}{Interaction Diagrams erstellen}\\
\textbf{1. Sequenzdiagramm}
\begin{itemize}
    \item \textbf{Elemente:}
    \begin{itemize}
        \item Lebenslinien für Objekte
        \item Nachrichten (synchron/asynchron)
        \item Aktivierungsbalken
        \item Alternative Abläufe
    \end{itemize}
    
    \item \textbf{Best Practices:}
    \begin{itemize}
        \item Von links nach rechts lesen
        \item Wichtige Objekte links
        \item Klare Beschriftungen
        \item Rückgabewerte zeigen
    \end{itemize}
\end{itemize}

\textbf{2. Kommunikationsdiagramm}
\begin{itemize}
    \item \textbf{Elemente:}
    \begin{itemize}
        \item Objekte als Rechtecke
        \item Nummerierte Nachrichten
        \item Assoziationen als Linien
    \end{itemize}
    
    \item \textbf{Best Practices:}
    \begin{itemize}
        \item Übersichtliche Anordnung
        \item Klare Nummerierung
        \item Wichtige Beziehungen hervorheben
    \end{itemize}
\end{itemize}
\end{KR}

\begin{example2}[breakable]{Interaction Diagrams: Bestellprozess}\\
\textbf{Use Case:} Bestellung aufgeben

\textbf{Sequenzdiagramm Code:}
\begin{lstlisting}[language=Java, style=basesmol]
@RestController
public class OrderController {
    private OrderService orderService;
    private PaymentService paymentService;
    
    public OrderResponse createOrder(
            OrderRequest request) {
        // 1: Validiere Bestellung
        OrderValidator.validate(request);
        
        // 2: Erstelle Order
        Order order = orderService.createOrder(request);
        
        // 3: Prozessiere Zahlung
        PaymentResult payment = 
            paymentService.processPayment(
                order.getId(), 
                order.getTotal());
        
        // 4: Bestaetige Order
        if (payment.isSuccessful()) {
            order.confirm();
            orderService.save(order);
            return OrderResponse.success(order);
        } else {
            return OrderResponse.failed(
                payment.getReason());
        }
    }
}

public class OrderService {
    @Transactional
    public Order createOrder(OrderRequest request) {
        // 2.1: Create order entity
        Order order = new Order(
            request.getCustomerId());
            
        // 2.2: Add items
        for (OrderItemRequest item : request.getItems()) {
            Product product = productRepository
                .findById(item.getProductId())
                .orElseThrow(ProductNotFoundException::new);
                
            order.addItem(
                product, 
                item.getQuantity());
        }
        
        // 2.3: Save order
        return orderRepository.save(order);
    }
}
\end{lstlisting}

\textbf{Sequenzdiagramm Analyse:}
\begin{itemize}
    \item \textbf{Objekte:}
    \begin{itemize}
        \item OrderController als Fassade
        \item OrderService für Geschäftslogik
        \item PaymentService für Zahlungen
        \item Order als Domain Object
    \end{itemize}
    
    \item \textbf{Verantwortlichkeiten:}
    \begin{itemize}
        \item Controller: Koordination
        \item Service: Transaktionslogik
        \item Domain: Business Rules
    \end{itemize}
    
    \item \textbf{GRASP Patterns:}
    \begin{itemize}
        \item Controller Pattern
        \item Information Expert
        \item Creator Pattern
    \end{itemize}
\end{itemize}
\end{example2}