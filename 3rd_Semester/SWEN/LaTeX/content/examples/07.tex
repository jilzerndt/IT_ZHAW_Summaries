\section{Kochrezepte und Übungen}

\subsection{Verteilte Systeme}

\begin{KR}{Verteiltes System entwerfen}\\
\textbf{1. Architekturanalyse}
\begin{itemize}
    \item Anforderungen analysieren
    \begin{itemize}
        \item Skalierbarkeit
        \item Verfügbarkeit
        \item Konsistenz
    \end{itemize}
    \item Kommunikationsmuster bestimmen
    \begin{itemize}
        \item Synchron vs. Asynchron
        \item Request-Response vs. Messaging
        \item Push vs. Pull
    \end{itemize}
    \item Fehlerszenarien identifizieren
    \begin{itemize}
        \item Netzwerkfehler
        \item Server-Ausfälle
        \item Datenverlust
    \end{itemize}
\end{itemize}

\textbf{2. Design}
\begin{itemize}
    \item Architekturmuster wählen
    \begin{itemize}
        \item Client-Server
        \item Peer-to-Peer
        \item Event-basiert
    \end{itemize}
    \item Verteilungsstruktur festlegen
    \begin{itemize}
        \item Service-Grenzen
        \item Datenverteilung
        \item Load Balancing
    \end{itemize}
    \item Konsistenzmodell definieren
    \begin{itemize}
        \item Strong vs. Eventual Consistency
        \item ACID vs. BASE
        \item CAP Trade-offs
    \end{itemize}
\end{itemize}

\textbf{3. Implementation}
\begin{itemize}
    \item Kommunikation implementieren
    \begin{itemize}
        \item Protokolle definieren
        \item Serialisierung festlegen
        \item Error Handling
    \end{itemize}
    \item Monitoring einrichten
    \begin{itemize}
        \item Logging
        \item Metriken
        \item Alarme
    \end{itemize}
    \item Testing
    \begin{itemize}
        \item Unit Tests
        \item Integration Tests
        \item Chaos Testing
    \end{itemize}
\end{itemize}
\end{KR}

\begin{KR}{Remote Procedure Call (RPC) implementieren}\\
\textbf{1. Interface definieren}
\begin{itemize}
    \item Services spezifizieren
    \item Parameter definieren
    \item Rückgabewerte festlegen
    \item Fehlerbehandlung planen
\end{itemize}

\textbf{2. Stub/Skeleton generieren}
\begin{itemize}
    \item IDL verwenden
    \item Marshalling implementieren
    \item Proxies erstellen
\end{itemize}

\textbf{3. Netzwerkkommunikation}
\begin{itemize}
    \item Protokoll wählen
    \item Serialisierung implementieren
    \item Timeouts einbauen
    \item Retry-Logik
\end{itemize}

\textbf{4. Fehlerbehandlung}
\begin{itemize}
    \item Netzwerkfehler behandeln
    \item Timeout-Handling
    \item Circuit Breaker einbauen
    \item Fallback-Strategien
\end{itemize}
\end{KR}

\begin{example2}{Typische Prüfungsaufgabe: Verteiltes System}\\
\textbf{Aufgabe:} Entwerfen Sie ein verteiltes Chat-System

\textbf{Anforderungen:}
\begin{itemize}
    \item Unterstützung für 100.000 gleichzeitige Nutzer
    \item Nachrichtenhistorie speichern
    \item Offline-Nachrichten möglich
    \item Maximale Latenz 500ms
\end{itemize}

\textbf{Lösung:}
\begin{lstlisting}[language=Java, style=base]
// Message Broker Interface
public interface MessageBroker {
    void publish(String topic, Message message);
    void subscribe(String topic, MessageHandler handler);
}

// Chat Service
@Service
public class ChatService {
    private final MessageBroker broker;
    private final MessageRepository repository;
    
    public void sendMessage(ChatMessage message) {
        // Persist message
        repository.save(message);
        
        // Publish to online users
        broker.publish(
            "chat." + message.getRoomId(), 
            message
        );
        
        // Handle offline users
        message.getRecipients().stream()
            .filter(user -> !userIsOnline(user))
            .forEach(user -> 
                queueOfflineMessage(user, message));
    }
    
    public void joinRoom(String userId, String roomId) {
        broker.subscribe("chat." + roomId, 
            message -> deliverToUser(userId, message));
    }
    
    private void queueOfflineMessage(
            String userId, ChatMessage message) {
        broker.publish(
            "offline." + userId,
            message
        );
    }
}

// Load Balancer Configuration
@Configuration
public class LoadBalancerConfig {
    @Bean
    public LoadBalancer chatLoadBalancer() {
        return LoadBalancer.builder()
            .algorithm(RoundRobin.class)
            .maxConnections(10000)
            .healthCheck("/health")
            .build();
    }
}
\end{lstlisting}

\textbf{Architekturentscheidungen:}
\begin{itemize}
    \item Event-driven Architecture mit Message Broker
    \item Sharding nach Chat-Räumen
    \item In-Memory Caching für aktive Chats
    \item Message Queue für Offline-Nachrichten
\end{itemize}
\end{example2}

\begin{example2}{Message Oriented Middleware}\\
\textbf{Aufgabe:} Implementieren Sie ein Nachrichtensystem mit JMS

\begin{lstlisting}[language=Java, style=base]
// Message Producer
@Component
public class OrderProducer {
    @Autowired
    private JmsTemplate jmsTemplate;
    
    public void sendOrder(Order order) {
        try {
            jmsTemplate.convertAndSend("orders", order, message -> {
                message.setStringProperty(
                    "orderType", 
                    order.getType().toString()
                );
                return message;
            });
        } catch (JmsException e) {
            handleMessageError(order, e);
        }
    }
}

// Message Consumer
@Component
public class OrderConsumer {
    @JmsListener(
        destination = "orders",
        selector = "orderType = 'PREMIUM'"
    )
    public void processPremiumOrder(Order order) {
        try {
            // Process order with high priority
            processOrderWithPriority(order);
        } catch (Exception e) {
            // Dead Letter Queue
            handleFailedOrder(order, e);
        }
    }
    
    @JmsListener(
        destination = "orders",
        selector = "orderType = 'STANDARD'"
    )
    public void processStandardOrder(Order order) {
        // Process normal order
    }
}

// Error Handling
@Component
public class DeadLetterQueueHandler {
    @JmsListener(destination = "DLQ")
    public void handleFailedMessages(Message failedMessage) {
        // Analyze failure
        // Retry with backoff
        // Alert if necessary
    }
}
\end{lstlisting}
\end{example2}

\begin{KR}{Fehlerbehandlung in verteilten Systemen}\\
\textbf{1. Netzwerkfehler}
\begin{itemize}
    \item Timeouts implementieren
    \item Retry-Strategien definieren
    \item Circuit Breaker einsetzen
\end{itemize}

\textbf{2. Dateninkonsistenzen}
\begin{itemize}
    \item Eventual Consistency
    \item Konfliktauflösung
    \item Versioning
\end{itemize}

\textbf{3. Ausfallsicherheit}
\begin{itemize}
    \item Redundanz einbauen
    \item Fallback-Strategien
    \item Graceful Degradation
\end{itemize}

\textbf{4. Monitoring}
\begin{itemize}
    \item Logging
    \item Metriken sammeln
    \item Alerting einrichten
\end{itemize}
\end{KR}

\begin{example2}{Circuit Breaker Implementation}\\
\textbf{Aufgabe:} Implementieren Sie einen Circuit Breaker für einen Microservice

\begin{lstlisting}[language=Java, style=base]
public class CircuitBreaker {
    private final long timeout;
    private final int failureThreshold;
    private final long resetTimeout;
    
    private AtomicInteger failures = new AtomicInteger();
    private AtomicReference<State> state = 
        new AtomicReference<>(State.CLOSED);
    private AtomicLong lastFailureTime = new AtomicLong();
    
    public enum State {
        CLOSED, OPEN, HALF_OPEN
    }
    
    public <T> T execute(Supplier<T> action) throws Exception {
        if (shouldExecute()) {
            try {
                T result = action.get();
                reset();
                return result;
            } catch (Exception e) {
                handleFailure();
                throw e;
            }
        }
        throw new CircuitBreakerException("Circuit open");
    }
    
    private boolean shouldExecute() {
        State currentState = state.get();
        if (currentState == State.CLOSED) {
            return true;
        }
        if (currentState == State.OPEN) {
            if (hasResetTimeoutExpired()) {
                state.compareAndSet(State.OPEN, State.HALF_OPEN);
                return true;
            }
            return false;
        }
        return true; // HALF_OPEN
    }
    
    private void handleFailure() {
        lastFailureTime.set(System.currentTimeMillis());
        if (failures.incrementAndGet() >= failureThreshold) {
            state.set(State.OPEN);
        }
    }
    
    private void reset() {
        failures.set(0);
        state.set(State.CLOSED);
    }
    
    private boolean hasResetTimeoutExpired() {
        return System.currentTimeMillis() - lastFailureTime.get() 
            > resetTimeout;
    }
}

// Usage Example
@Service
public class UserService {
    private final CircuitBreaker circuitBreaker = new CircuitBreaker(
        1000,  // timeout
        5,     // failureThreshold
        60000  // resetTimeout
    );
    
    private final RestTemplate restTemplate;
    
    public UserDTO getUser(String id) {
        return circuitBreaker.execute(() ->
            restTemplate.getForObject(
                "/api/users/" + id,
                UserDTO.class
            )
        );
    }
}
\end{lstlisting}
\end{example2}

\begin{KR}{Service Discovery implementieren}\\
\textbf{1. Registry Service}
\begin{itemize}
    \item Service-Registrierung
    \item Health Checking
    \item Load Balancing
\end{itemize}

\textbf{2. Service Registration}
\begin{itemize}
    \item Startup Registration
    \item Heartbeat Mechanism
    \item Graceful Shutdown
\end{itemize}

\textbf{3. Service Discovery}
\begin{itemize}
    \item Cache Management
    \item Failure Detection
    \item Load Balancing
\end{itemize}
\end{KR}

\begin{example2}{Service Discovery mit Spring Cloud}\\
\textbf{Aufgabe:} Implementieren Sie Service Discovery für Microservices

\begin{lstlisting}[language=Java, style=base]
// Eureka Server
@SpringBootApplication
@EnableEurekaServer
public class ServiceRegistryApplication {
    public static void main(String[] args) {
        SpringApplication.run(
            ServiceRegistryApplication.class, 
            args
        );
    }
}

// Service Registration
@SpringBootApplication
@EnableDiscoveryClient
public class UserServiceApplication {
    public static void main(String[] args) {
        SpringApplication.run(
            UserServiceApplication.class, 
            args
        );
    }
}

// Service Discovery
@Service
public class UserServiceClient {
    @Autowired
    private DiscoveryClient discoveryClient;
    
    @Autowired
    private RestTemplate restTemplate;
    
    public UserDTO getUser(String id) {
        // Get service instance
        List<ServiceInstance> instances = 
            discoveryClient.getInstances("user-service");
            
        if (instances.isEmpty()) {
            throw new ServiceNotFoundException(
                "user-service not available");
        }
        
        // Load balance
        ServiceInstance instance = 
            loadBalance(instances);
            
        // Make request
        return restTemplate.getForObject(
            instance.getUri() + "/api/users/" + id,
            UserDTO.class
        );
    } 
    
    private ServiceInstance loadBalance(
            List<ServiceInstance> instances) {
        // Simple round-robin
        int index = requestCounter.incrementAndGet() 
            % instances.size();
        return instances.get(index);
    }
}
\end{lstlisting}
\end{example2}