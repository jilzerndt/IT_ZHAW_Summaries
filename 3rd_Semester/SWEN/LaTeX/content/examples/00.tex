\section{KR und Beispiele für Einführung}

\begin{KR}{Prozessmodelle vergleichen}
\textbf{Vorgehen bei der Analyse:}
\begin{enumerate}
    \item \textbf{Kriterien identifizieren:}
    \begin{itemize}
        \item Zeit/Budget/Scope
        \item Risikomanagement
        \item Kundeneinbindung
        \item Änderungsmanagement
    \end{itemize}
    
    \item \textbf{Modelle analysieren:}
    \begin{itemize}
        \item Charakteristiken der Modelle verstehen
        \item Stärken und Schwächen erkennen
        \item Anwendungsszenarien berücksichtigen 
    \end{itemize}
    
    \item \textbf{Gegenüberstellung:}
    \begin{itemize}
        \item Direkte Vergleiche für jedes Kriterium
        \item Unterschiede herausarbeiten
        \item Mit Beispielen unterlegen
    \end{itemize}
\end{enumerate}
\end{KR}

\begin{example2}{Prozessmodell-Analyse}\\
\textbf{Aufgabe:} Vergleichen Sie das Wasserfallmodell mit einem iterativ-inkrementellen Ansatz anhand folgender Kriterien:
\begin{itemize}
    \item Umgang mit sich ändernden Anforderungen
    \item Risikomanagement
    \item Planbarkeit
    \item Kundeneinbindung
\end{itemize}

\textbf{Musterlösung:}
\begin{itemize}
    \item \textbf{Wasserfall:}
    \begin{itemize}
        \item Änderungen schwierig zu integrieren
        \item Risiken erst spät erkennbar
        \item Gut planbar durch feste Phasen
        \item Kunde hauptsächlich am Anfang und Ende involviert
    \end{itemize}
    \item \textbf{Iterativ-inkrementell:}
    \begin{itemize}
        \item Flexibel bei Änderungen durch kurze Zyklen
        \item Frühes Erkennen von Risiken durch regelmäßige Reviews
        \item Planung pro Iteration, mehr Flexibilität
        \item Kontinuierliches Kundenfeedback in jeder Iteration
    \end{itemize}
\end{itemize}
\end{example2}

\begin{KR}{Modellierungsumfang bestimmen}
\textbf{Analyseschritte:}
\begin{enumerate}
    \item \textbf{Projektkontext analysieren:}
    \begin{itemize}
        \item Projektgröße und Komplexität
        \item Anzahl beteiligter Stakeholder
        \item Kritikalität des Systems
        \item Domänenwissen im Team
    \end{itemize}
    
    \item \textbf{Faktoren bewerten:}
    \begin{itemize}
        \item Risiko bei Fehlern
        \item Änderungshäufigkeit
        \item Dokumentationspflichten
        \item Teamverteilung
    \end{itemize}
    
    \item \textbf{Umfang festlegen:}
    \begin{itemize}
        \item Minimaler vs. maximaler Modellierungsumfang
        \item Kosten-Nutzen Abwägung
        \item Verfügbare Ressourcen
    \end{itemize}
\end{enumerate}
\end{KR}

\begin{example2}{Modellierungsumfang: Beispielaufgabe}\\
\textbf{Aufgabe:} Ein Softwaresystem soll die Verwaltung von Patientenakten in einer Arztpraxis unterstützen. Das System muss verschiedene gesetzliche Auflagen erfüllen. Bestimmen Sie den notwendigen Modellierungsumfang.

\textbf{Analyse:}
\begin{itemize}
    \item \textbf{Hoher Modellierungsumfang notwendig wegen:}
    \begin{itemize}
        \item Medizinische Domäne mit hoher Komplexität
        \item Gesetzliche Anforderungen (Datenschutz, Dokumentation)
        \item Kritische Daten und Prozesse
        \item Verschiedene Stakeholder (Ärzte, Personal, Patienten)
    \end{itemize}
    
    \item \textbf{Erforderliche Modelle:}
    \begin{itemize}
        \item Detailliertes Domänenmodell
        \item Vollständige Use Cases
        \item Ausführliche Systemarchitektur
        \item Sicherheits- und Datenschutzkonzepte
        \item Prozessmodelle für kritische Abläufe
    \end{itemize}
\end{itemize}

\textbf{Begründung:}
Bei einem medizinischen System überwiegen die Vorteile einer ausführlichen Modellierung klar die Kosten:
\begin{itemize}
    \item Fehler können schwerwiegende Folgen haben
    \item Nachträgliche Änderungen sind aufwändig
    \item Dokumentationspflichten müssen erfüllt werden
    \item Zertifizierungen erfordern genaue Modelle
\end{itemize}
\end{example2}

\begin{KR}{Requirements vs. Technology Matrix}
\textbf{Analyse-Schritte:}
\begin{enumerate}
    \item \textbf{Einordnung Anforderungen:}
    \begin{itemize}
        \item Known Requirements → Klare Spezifikation
        \item Unknown Requirements → Agile Exploration
    \end{itemize}
    
    \item \textbf{Einordnung Technologie:}
    \begin{itemize}
        \item Known Technology → Bewährte Tools/Methoden
        \item Unknown Technology → Prototypen/Spikes
    \end{itemize}
    
    \item \textbf{Quadranten analysieren:}
    \begin{itemize}
        \item Known/Known → Wasserfall möglich
        \item Known/Unknown → Technische Prototypen
        \item Unknown/Known → Agile Methoden
        \item Unknown/Unknown → Extreme Prototyping
    \end{itemize}
\end{enumerate}
\end{KR}

\begin{example2}{Requirements vs. Technology: Projektanalyse}\\
\textbf{Aufgabe:} Analysieren Sie folgende Projekte und ordnen Sie sie in die Requirements/Technology Matrix ein:

\textbf{Projekt 1: Online-Shop Update}
\begin{itemize}
    \item Bekannte E-Commerce Plattform
    \item Standard-Funktionalitäten
    \item Bewährte Technologien
    \textbf{→ Known/Known Quadrant}
\end{itemize}

\textbf{Projekt 2: KI-basierte Diagnose}
\begin{itemize}
    \item Neue Anwendungsdomäne
    \item Unklare Nutzeranforderungen
    \item Innovative KI-Technologie
    \textbf{→ Unknown/Unknown Quadrant}
\end{itemize}

\textbf{Projekt 3: Legacy-System Migration}
\begin{itemize}
    \item Klare Funktionsanforderungen
    \item Neue Cloud-Technologie
    \item Unbekannte Performance-Charakteristik
    \textbf{→ Known/Unknown Quadrant}
\end{itemize}

\textbf{Empfohlene Vorgehensweise:}
\begin{itemize}
    \item Projekt 1: Strukturierter Wasserfall-Ansatz
    \item Projekt 2: Extreme Prototyping mit kurzen Iterationen
    \item Projekt 3: Technische Prototypen, dann inkrementelle Migration
\end{itemize}
\end{example2}