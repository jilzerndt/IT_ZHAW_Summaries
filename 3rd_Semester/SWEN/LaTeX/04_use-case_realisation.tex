\section{Use Case Realisation}

\section*{7 Vorlesung 07}
\subsection*{7.1 Use Cases und Use-Case-Realization}
Die Planung erfolgt anhand von Use-Cases, Realisierung v Use-Cases\\
Der wichtigste Teil sind die detaillierten Szenarien (Standardszenario und Erweiterungen), und davon die Systemantworten. Diese müssen schlussendlich realisiert werden.

\subsection*{7.1.1 Vergleich SSD}
UI statt System, Systemoperationen sind Elemente die realisiert werden

\subsection*{7.1.2 Warum UML}
\begin{itemize}
  \item Zwischenschritt bei wenig Erfahrung
  \item Ersatz für Programmiersprache, Kompakt
  \item auch für Laien zu verstehen
\end{itemize}

\subsection*{7.2 Vorgehen UC Realization}
\begin{enumerate}
  \item Use Case auswählen, offene Fragen klären, SSD ableiten
  \item Systemoperation auswählen
  \item Operation Contract (Systemvertrag) erstellen/überlegen
  \item Aktuellen Code/Dokumentation analysieren\\
(a) DCD überprüfen/aktualisieren\\
(b) Vergleich mit Domänenmodell durchführen\\
(c) Neue Klassen gemäß Domänenmodell erstellen
  \item Falls notwendig, Refactorings durchführen\\
(a) Controller Klasse bestimmen\\
(b) Zu verändernde Klassen festlegen\\
(c) Weg zu Klassen festlegen\\
i. Weg mit Parametern wählen\\
ii. Klassen ggf. neu erstellen\\
iii. Verantwortlichkeiten zuweisen\\
iv. Varianten bewerten\\
(d) Veränderungen programmieren\\
(e) Review durchführen
\end{enumerate}