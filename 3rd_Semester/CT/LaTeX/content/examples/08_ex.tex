\subsection{Examples}

\begin{concept}{ARM Parameter Passing}\\
Key rules for parameter passing:
\begin{itemize}
  \item \textbf{Register Parameters}:
    \begin{itemize}
      \item First four parameters in R0-R3
      \item Additional parameters on stack
      \item Return value in R0 (or R0/R1 for 64-bit)
    \end{itemize}
  \item \textbf{Stack Parameters}:
    \begin{itemize}
      \item Pushed right-to-left
      \item 8-byte aligned
      \item Caller responsible for cleanup
    \end{itemize}
  \item \textbf{Return Values}:
    \begin{itemize}
      \item 32-bit or less in R0
      \item 64-bit in R0 and R1
      \item Larger values via pointer
    \end{itemize}
\end{itemize}
\end{concept}

\begin{KR}{Parameter Passing Guidelines}\\
Rules for implementing function calls:

1. Caller responsibilities:
\begin{lstlisting}[language=armasm, style=basesmol]
    ; Save any needed registers
    PUSH    {R4-R6, LR}      ; Save registers

    ; Load parameters into R0-R3
    MOV     R0, R4           ; First parameter
    MOV     R1, R5           ; Second parameter
    MOV     R2, R6           ; Third parameter

    ; Call function
    BL      function

    ; Save return value if needed
    MOV     R7, R0           ; Save result

    ; Restore registers
    POP     {R4-R6, PC}      ; Return
\end{lstlisting}

2. Callee responsibilities:
\begin{lstlisting}[language=armasm, style=basesmol]
function
    ; Save any registers we'll modify
    PUSH    {R4, LR}
    
    ; Process parameters in R0-R3
    ; Put return value in R0
    
    ; Restore registers
    POP     {R4, PC}
\end{lstlisting}

3. Reference parameter handling:
\begin{lstlisting}[language=armasm, style=basesmol]
    ; Loading from pointer
    LDR     R2, [R0]         ; Get value at address
    ; Storing to pointer
    STR     R2, [R1]         ; Store to address
    ; Incrementing pointer
    ADD     R0, R0, #4       ; Next word
\end{lstlisting}
\end{KR}

\begin{remark}
Important considerations:
\begin{itemize}
  \item Track register usage and preservation
  \item Consider stack alignment requirements
  \item Be aware of parameter passing limits
  \item Handle return values consistently
  \item Monitor stack growth in recursion
\end{itemize}
\end{remark}


\begin{example2}{Pass by Value vs Reference} Parameter passing issues:
\begin{lstlisting}[language=C, style=basesmol]
// Incorrect swap - pass by value
void swap_bad(int32_t c, int32_t d) {
    int32_t temp = c;
    c = d;
    d = temp;
}
// Correct swap - pass by reference
void swap_good(int32_t *c, int32_t *d) {
    int32_t temp = *c;
    *c = *d;
    *d = temp;
}
int32_t main(void) {
    int32_t a = 3, b = 5;
    
    swap_bad(a, b);   // Doesn't work
    swap_good(&a, &b); // Works correctly
}
\end{lstlisting}

Assembly implementation of swap\_good:
\begin{lstlisting}[language=armasm, style=basesmol]
swap_good
    PUSH    {LR}             ; Save return address
    
    LDR     R2, [R0]         ; Load *c into R2
    LDR     R3, [R1]         ; Load *d into R3
    
    STR     R3, [R0]         ; Store R3 to *c
    STR     R2, [R1]         ; Store R2 to *d
    
    POP     {PC}             ; Return
\end{lstlisting}
\end{example2}





\begin{example2}{Complex Parameter Example}
Function with mixed parameter types:
\begin{lstlisting}[language=C, style=basesmol]
typedef struct {
    int32_t x;
    int32_t y;
} point_t;

int32_t calculate(point_t* p, int32_t scale, 
                  int32_t* result);
\end{lstlisting}

Assembly implementation:
\begin{lstlisting}[language=armasm, style=basesmol]
; R0 = point_t* p
; R1 = scale
; R2 = result pointer
calculate
    PUSH    {R4-R5, LR}     ; Save registers
    
    ; Load structure members
    LDR     R4, [R0, #0]    ; Load p->x
    LDR     R5, [R0, #4]    ; Load p->y
    
    ; Perform calculation
    MULS    R4, R1, R4      ; x * scale
    MULS    R5, R1, R5      ; y * scale
    
    ; Store result
    STR     R4, [R2, #0]    ; *result = x
    ADDS    R0, R4, R5      ; Return sum
    
    POP     {R4-R5, PC}     ; Return
\end{lstlisting}
\end{example2}



\begin{example2}{Data Structure Access}
Working with structures and arrays:
\begin{lstlisting}[language=C, style=basesmol]
typedef struct {
    uint32_t minutes;
    uint32_t seconds;
} time_t;

time_t time;
\end{lstlisting}

Assembly implementation:
\begin{lstlisting}[language=armasm, style=basesmol]
    ; Access structure members
    LDR     R0, =time       ; Get structure address
    LDR     R1, [R0, #0]    ; Load minutes
    LDR     R2, [R0, #4]    ; Load seconds
    
    ; Modify structure
    ADDS    R2, #1          ; Increment seconds
    CMP     R2, #60         ; Check for overflow
    BLT     store_back
    MOVS    R2, #0          ; Reset seconds
    ADDS    R1, #1          ; Increment minutes
store_back
    STR     R1, [R0, #0]    ; Store minutes
    STR     R2, [R0, #4]    ; Store seconds
\end{lstlisting}
\end{example2}

\begin{example2}{Recursive Function Example}\\
Factorial calculation showing stack usage:

\begin{lstlisting}[language=C, style=basesmol]
int32_t fakultaet_recursive(int32_t n) {
    if(n < 2) {
        return 1;
    } else {
        return n * fakultaet_recursive(n-1);
    }
}

int32_t main(void) {
    int32_t n = 20;
    int32_t result = fakultaet_recursive(n);
}
\end{lstlisting}

Assembly implementation showing stack growth:
\begin{lstlisting}[language=armasm, style=basesmol]
fakultaet_recursive
    PUSH    {R4, LR}         ; Each call adds 8 bytes
    MOV     R4, R0           ; Save n
    
    CMP     R0, #2           ; Check base case
    BLT     return_one
    
    SUB     R0, R0, #1       ; n-1
    BL      fakultaet_recursive
    MUL     R0, R4, R0       ; n * result
    
    POP     {R4, PC}         ; Return
    
return_one
    MOV     R0, #1           ; Return 1
    POP     {R4, PC}         ; Return
\end{lstlisting}

Maximum stack size = 8 bytes * 19 calls = 152 bytes
\end{example2}



\begin{example2}{Function Call Example}\\
C function and its parameter passing:

\begin{lstlisting}[language=C, style=basesmol]
uint32_t logical_and(uint32_t a, uint32_t b, uint32_t c) {
    return a & b & c;
}

int32_t main(void) {
    uint32_t x = 0x11223344;  // In R4
    uint32_t y = 0xFFFF0000;  // In R5
    uint32_t z = 0x33661122;  // In R6
    uint32_t result;          // In R7
    
    result = logical_and(x, y, z);
}
\end{lstlisting}

Beim Start des Programmes wird die Variable x in R4, y in R 5 und z in R 6 abgelegt. Die Variable result wird in R7 abgelegt.
\vspace{1mm}\\
1. Welche Schritte führt der Caller (main) vor dem Aufruf der Funktion logical\_and() durch? Wie werden die Parameter übergeben?\\
Die Variablen in den Registern R4 bis R6 werden nach R0 bis $R 2$ kopiert und so der Funktion übergeben. $R 6 \rightarrow R 2, R 5 \rightarrow R 1, R 4 \rightarrow R 0$ 
\vspace{1mm}\\
2. Wie gibt die Funktion logical\_and() den Rückgabewert zurück?\\
Der Rückgabewert wird via R0 zurückgegeben.
\vspace{1mm}\\
3. Welche Operation führt der Call nach dem Aufruf der Funktion logical\_and() durch?\\
Der Rückgabewert wird von R0 nach R7 kopiert
\vspace{1mm}\\
Assembly implementation showing parameter passing:
\begin{lstlisting}[language=armasm, style=basesmol]
main
    ; Initial register setup
    LDR     R4, =0x11223344   ; x
    LDR     R5, =0xFFFF0000   ; y
    LDR     R6, =0x33661122   ; z
    
    ; Parameter passing
    MOV     R0, R4            ; a = x
    MOV     R1, R5            ; b = y
    MOV     R2, R6            ; c = z
    BL      logical_and       
    MOV     R7, R0            ; result = return value

logical_and
    AND     R0, R0, R1        ; a & b
    AND     R0, R0, R2        ; & c
    BX      LR                ; Return
\end{lstlisting}
\end{example2}

\begin{KR}{Parameter Passing Techniques}
Methods for passing parameters to subroutines:

1. Register parameters (up to 4):
\begin{lstlisting}[language=armasm, style=basesmol]
    ; Caller side
    MOVS    R0, #1          ; First parameter
    MOVS    R1, #2          ; Second parameter
    BL      my_function     ; Call function
    
    ; Callee side
my_function
    PUSH    {LR}
    ADD     R0, R0, R1      ; Use parameters
    POP     {PC}            ; Return
\end{lstlisting}

2. Stack parameters (more than 4):
\begin{lstlisting}[language=armasm, style=basesmol]
    ; Caller side
    PUSH    {R0}            ; Push fifth parameter
    MOVS    R0, #1          ; First parameter
    BL      my_function     ; Call function
    ADD     SP, SP, #4      ; Clean up stack
    
    ; Callee side
my_function
    PUSH    {R4, LR}
    LDR     R4, [SP, #8]    ; Access fifth parameter
    POP     {R4, PC}        ; Return
\end{lstlisting}

3. Return values:
\begin{lstlisting}[language=armasm, style=basesmol]
    ; Single value return
    MOVS    R0, #42         ; Return in R0
    BX      LR
    
    ; Multiple value return
    MOVS    R0, #42         ; First return value
    MOVS    R1, #24         ; Second return value
    BX      LR
\end{lstlisting}
\end{KR}

\begin{KR}{AAPCS Compliant Function Implementation}
Guidelines for implementing ARM Procedure Call Standard:

1. Register parameter passing:
\begin{lstlisting}[language=armasm, style=basesmol]
; Function using multiple parameters
my_func                     ; void my_func(int a, int b, int c)
    PUSH    {R4-R6, LR}     ; Save registers
    
    ; Parameters already in:
    ; R0 = a
    ; R1 = b
    ; R2 = c
    
    MOV     R4, R0          ; Save parameters to
    MOV     R5, R1          ; preserved registers
    MOV     R6, R2          ; if needed long-term
    
    ; Function body...
    
    POP     {R4-R6, PC}     ; Return
\end{lstlisting}

2. Stack parameter access:
\begin{lstlisting}[language=armasm, style=basesmol]
; Function with 5+ parameters
complex_func    ; void complex(int a, int b, int c, int d, int e)
    PUSH    {R4-R6, LR}
    
    ; First four params in R0-R3
    MOV     R4, R0          ; Save a
    MOV     R5, R1          ; Save b
    
    ; Fifth parameter on stack
    LDR     R6, [SP, #16]   ; Load e (after saved regs)
    
    ; Function body...
    
    POP     {R4-R6, PC}
\end{lstlisting}

3. Return value handling:
\begin{lstlisting}[language=armasm, style=basesmol]
; Function returning multiple values
return_pair   ; void return_pair(int* out1, int* out2)
    PUSH    {R4, LR}
    
    ; Parameters are pointers in R0, R1
    MOVS    R2, #42         ; First value
    MOVS    R3, #24         ; Second value
    
    STR     R2, [R0]        ; Store first result
    STR     R3, [R1]        ; Store second result
    
    POP     {R4, PC}
\end{lstlisting}
\end{KR}

\begin{example2}{Structure Parameter Handling}
Example of passing and returning structures:
\begin{lstlisting}[language=armasm, style=basesmol]
; Structure with two fields
; typedef struct {
;     int32_t x;
;     int32_t y;
; } point_t;

update_point  ; point_t update_point(point_t* p, int32_t dx)
    PUSH    {R4-R5, LR}
    
    ; R0 = point pointer, R1 = dx
    LDR     R4, [R0, #0]    ; Load p->x
    LDR     R5, [R0, #4]    ; Load p->y
    
    ; Update x coordinate
    ADDS    R4, R1          ; Add dx to x
    STR     R4, [R0, #0]    ; Store back to p->x
    
    ; Return original y in R0
    MOV     R0, R5          
    
    POP     {R4-R5, PC}     
\end{lstlisting}

Key points:
\begin{itemize}
  \item Structures passed by pointer
  \item Fields accessed via offsets
  \item Return value in R0
\end{itemize}
\end{example2}

\begin{KR}{Array Parameter Handling}
Guidelines for passing and processing arrays:

1. Simple array processing:
\begin{lstlisting}[language=armasm, style=basesmol]
; Calculate sum of array
; int sum_array(int* arr, int count)
sum_array
    PUSH    {R4-R5, LR}
    
    ; R0 = array pointer, R1 = count
    MOVS    R4, #0          ; Initialize sum
    MOVS    R5, #0          ; Initialize index
    
loop
    CMP     R5, R1          ; Check index < count
    BGE     done
    
    LDR     R2, [R0, R5, LSL #2] ; Load arr[i]
    ADDS    R4, R2          ; Add to sum
    ADDS    R5, #1          ; Increment index
    B       loop
    
done
    MOV     R0, R4          ; Return sum
    POP     {R4-R5, PC}
\end{lstlisting}

2. Array with size parameter:
\begin{lstlisting}[language=armasm, style=basesmol]
; Find maximum in array
; int find_max(int* arr, int size)
find_max
    PUSH    {R4-R5, LR}
    
    ; R0 = array, R1 = size
    CMP     R1, #0          ; Check empty array
    BEQ     return_zero
    
    LDR     R4, [R0]        ; Initialize max
    MOVS    R5, #1          ; Start from second element
    
max_loop
    CMP     R5, R1
    BGE     max_done
    
    LDR     R2, [R0, R5, LSL #2]
    CMP     R2, R4
    BLE     not_larger
    MOV     R4, R2          ; Update max
not_larger
    ADDS    R5, #1
    B       max_loop
    
max_done
    MOV     R0, R4          ; Return maximum
    POP     {R4-R5, PC}
    
return_zero
    MOVS    R0, #0
    POP     {R4-R5, PC}
\end{lstlisting}
\end{KR}

\begin{example2}{Variable Output Parameters}
Example handling multiple output values:
\begin{lstlisting}[language=armasm, style=basesmol]
; Calculate min and max
; void minmax(int* arr, int size, int* min, int* max)
minmax
    PUSH    {R4-R7, LR}
    
    ; R0 = array, R1 = size
    ; R2 = min ptr, R3 = max ptr
    
    LDR     R4, [R0]        ; Initial value
    MOV     R5, R4          ; min = first
    MOV     R6, R4          ; max = first
    MOVS    R7, #1          ; index
    
loop
    CMP     R7, R1
    BGE     done
    
    LDR     R4, [R0, R7, LSL #2]
    CMP     R4, R5
    BGE     check_max
    MOV     R5, R4          ; Update min
check_max
    CMP     R4, R6
    BLE     next
    MOV     R6, R4          ; Update max
next
    ADDS    R7, #1
    B       loop
    
done
    STR     R5, [R2]        ; Store min result
    STR     R6, [R3]        ; Store max result
    
    POP     {R4-R7, PC}
\end{lstlisting}

Key points:
\begin{itemize}
  \item Multiple outputs via pointers
  \item Results stored to provided addresses
  \item No return value needed
\end{itemize}
\end{example2}

\begin{example2}{String Parameter Handling}
String manipulation example:
\begin{lstlisting}[language=armasm, style=basesmol]
; Find string length
; int strlen(const char* str)
strlen
    PUSH    {R4, LR}
    
    MOV     R4, R0          ; Save string pointer
    MOVS    R0, #0          ; Initialize length
    
count_loop
    LDRB    R1, [R4, R0]    ; Load character
    CMP     R1, #0          ; Check for null
    BEQ     done
    ADDS    R0, #1          ; Increment length
    B       count_loop
    
done
    POP     {R4, PC}        ; Return length in R0
\end{lstlisting}

\begin{itemize}
  \item String passed as pointer
  \item Null terminator check
  \item Length returned in R0
\end{itemize}
\end{example2}

\begin{KR}{Return Value Patterns}
Guidelines for handling different return types:

1. Simple values:
\begin{lstlisting}[language=armasm, style=basesmol]
; Return 32-bit value
    MOV     R0, R4          ; Result in R0
    BX      LR
    
; Return boolean
    MOVS    R0, #1          ; Return true
    BX      LR
\end{lstlisting}

2. Large values via pointer:
\begin{lstlisting}[language=armasm, style=basesmol]
; Fill structure with result
    STR     R4, [R0, #0]    ; Store first field
    STR     R5, [R0, #4]    ; Store second field
    BX      LR
\end{lstlisting}

3. Multiple return values:
\begin{lstlisting}[language=armasm, style=basesmol]
; Return pair of values
    MOV     R0, R4          ; First return value
    MOV     R1, R5          ; Second return value
    BX      LR
\end{lstlisting}
\end{KR}

