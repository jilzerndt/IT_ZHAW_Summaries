\subsection{Examples}

\begin{concept}{ARM Parameter Passing}\\
Key rules for parameter passing:
\begin{itemize}
  \item \textbf{Register Parameters}:
    \begin{itemize}
      \item First four parameters in R0-R3
      \item Additional parameters on stack
      \item Return value in R0 (or R0/R1 for 64-bit)
    \end{itemize}
  \item \textbf{Stack Parameters}:
    \begin{itemize}
      \item Pushed right-to-left
      \item 8-byte aligned
      \item Caller responsible for cleanup
    \end{itemize}
  \item \textbf{Return Values}:
    \begin{itemize}
      \item 32-bit or less in R0
      \item 64-bit in R0 and R1
      \item Larger values via pointer
    \end{itemize}
\end{itemize}
\end{concept}

\begin{KR}{Parameter Passing Guidelines}\\
Rules for implementing function calls:

1. Caller responsibilities:
\begin{lstlisting}[language=armasm, style=basesmol]
    ; Save any needed registers
    PUSH    {R4-R6, LR}      ; Save registers

    ; Load parameters into R0-R3
    MOV     R0, R4           ; First parameter
    MOV     R1, R5           ; Second parameter
    MOV     R2, R6           ; Third parameter

    ; Call function
    BL      function

    ; Save return value if needed
    MOV     R7, R0           ; Save result

    ; Restore registers
    POP     {R4-R6, PC}      ; Return
\end{lstlisting}

2. Callee responsibilities:
\begin{lstlisting}[language=armasm, style=basesmol]
function
    ; Save any registers we'll modify
    PUSH    {R4, LR}
    
    ; Process parameters in R0-R3
    ; Put return value in R0
    
    ; Restore registers
    POP     {R4, PC}
\end{lstlisting}

3. Reference parameter handling:
\begin{lstlisting}[language=armasm, style=basesmol]
    ; Loading from pointer
    LDR     R2, [R0]         ; Get value at address
    ; Storing to pointer
    STR     R2, [R1]         ; Store to address
    ; Incrementing pointer
    ADD     R0, R0, #4       ; Next word
\end{lstlisting}
\end{KR}

\begin{remark}
Important considerations:
\begin{itemize}
  \item Track register usage and preservation
  \item Consider stack alignment requirements
  \item Be aware of parameter passing limits
  \item Handle return values consistently
  \item Monitor stack growth in recursion
\end{itemize}
\end{remark}


\begin{example2}{Pass by Value vs Reference} Parameter passing issues:
\begin{lstlisting}[language=C, style=basesmol]
// Incorrect swap - pass by value
void swap_bad(int32_t c, int32_t d) {
    int32_t temp = c;
    c = d;
    d = temp;
}
// Correct swap - pass by reference
void swap_good(int32_t *c, int32_t *d) {
    int32_t temp = *c;
    *c = *d;
    *d = temp;
}
int32_t main(void) {
    int32_t a = 3, b = 5;
    
    swap_bad(a, b);   // Doesn't work
    swap_good(&a, &b); // Works correctly
}
\end{lstlisting}

Assembly implementation of swap\_good:
\begin{lstlisting}[language=armasm, style=basesmol]
swap_good
    PUSH    {LR}             ; Save return address
    
    LDR     R2, [R0]         ; Load *c into R2
    LDR     R3, [R1]         ; Load *d into R3
    
    STR     R3, [R0]         ; Store R3 to *c
    STR     R2, [R1]         ; Store R2 to *d
    
    POP     {PC}             ; Return
\end{lstlisting}
\end{example2}





\begin{example2}{Complex Parameter Example}
Function with mixed parameter types:
\begin{lstlisting}[language=C, style=basesmol]
typedef struct {
    int32_t x;
    int32_t y;
} point_t;

int32_t calculate(point_t* p, int32_t scale, 
                  int32_t* result);
\end{lstlisting}

Assembly implementation:
\begin{lstlisting}[language=armasm, style=basesmol]
; R0 = point_t* p
; R1 = scale
; R2 = result pointer
calculate
    PUSH    {R4-R5, LR}     ; Save registers
    
    ; Load structure members
    LDR     R4, [R0, #0]    ; Load p->x
    LDR     R5, [R0, #4]    ; Load p->y
    
    ; Perform calculation
    MULS    R4, R1, R4      ; x * scale
    MULS    R5, R1, R5      ; y * scale
    
    ; Store result
    STR     R4, [R2, #0]    ; *result = x
    ADDS    R0, R4, R5      ; Return sum
    
    POP     {R4-R5, PC}     ; Return
\end{lstlisting}
\end{example2}



\begin{example2}{Data Structure Access}
Working with structures and arrays:
\begin{lstlisting}[language=C, style=basesmol]
typedef struct {
    uint32_t minutes;
    uint32_t seconds;
} time_t;

time_t time;
\end{lstlisting}

Assembly implementation:
\begin{lstlisting}[language=armasm, style=basesmol]
    ; Access structure members
    LDR     R0, =time       ; Get structure address
    LDR     R1, [R0, #0]    ; Load minutes
    LDR     R2, [R0, #4]    ; Load seconds
    
    ; Modify structure
    ADDS    R2, #1          ; Increment seconds
    CMP     R2, #60         ; Check for overflow
    BLT     store_back
    MOVS    R2, #0          ; Reset seconds
    ADDS    R1, #1          ; Increment minutes
store_back
    STR     R1, [R0, #0]    ; Store minutes
    STR     R2, [R0, #4]    ; Store seconds
\end{lstlisting}
\end{example2}

\begin{example2}{Recursive Function Example}\\
Factorial calculation showing stack usage:

\begin{lstlisting}[language=C, style=basesmol]
int32_t fakultaet_recursive(int32_t n) {
    if(n < 2) {
        return 1;
    } else {
        return n * fakultaet_recursive(n-1);
    }
}

int32_t main(void) {
    int32_t n = 20;
    int32_t result = fakultaet_recursive(n);
}
\end{lstlisting}

Assembly implementation showing stack growth:
\begin{lstlisting}[language=armasm, style=basesmol]
fakultaet_recursive
    PUSH    {R4, LR}         ; Each call adds 8 bytes
    MOV     R4, R0           ; Save n
    
    CMP     R0, #2           ; Check base case
    BLT     return_one
    
    SUB     R0, R0, #1       ; n-1
    BL      fakultaet_recursive
    MUL     R0, R4, R0       ; n * result
    
    POP     {R4, PC}         ; Return
    
return_one
    MOV     R0, #1           ; Return 1
    POP     {R4, PC}         ; Return
\end{lstlisting}

Maximum stack size = 8 bytes * 19 calls = 152 bytes
\end{example2}



\begin{example2}{Function Call Example}\\
C function and its parameter passing:

\begin{lstlisting}[language=C, style=basesmol]
uint32_t logical_and(uint32_t a, uint32_t b, uint32_t c) {
    return a & b & c;
}

int32_t main(void) {
    uint32_t x = 0x11223344;  // In R4
    uint32_t y = 0xFFFF0000;  // In R5
    uint32_t z = 0x33661122;  // In R6
    uint32_t result;          // In R7
    
    result = logical_and(x, y, z);
}
\end{lstlisting}

Beim Start des Programmes wird die Variable x in R4, y in R 5 und z in R 6 abgelegt. Die Variable result wird in R7 abgelegt.
\vspace{1mm}\\
1. Welche Schritte führt der Caller (main) vor dem Aufruf der Funktion logical\_and() durch? Wie werden die Parameter übergeben?\\
Die Variablen in den Registern R4 bis R6 werden nach R0 bis $R 2$ kopiert und so der Funktion übergeben. $R 6 \rightarrow R 2, R 5 \rightarrow R 1, R 4 \rightarrow R 0$ 
\vspace{1mm}\\
2. Wie gibt die Funktion logical\_and() den Rückgabewert zurück?\\
Der Rückgabewert wird via R0 zurückgegeben.
\vspace{1mm}\\
3. Welche Operation führt der Call nach dem Aufruf der Funktion logical\_and() durch?\\
Der Rückgabewert wird von R0 nach R7 kopiert
\vspace{1mm}\\
Assembly implementation showing parameter passing:
\begin{lstlisting}[language=armasm, style=basesmol]
main
    ; Initial register setup
    LDR     R4, =0x11223344   ; x
    LDR     R5, =0xFFFF0000   ; y
    LDR     R6, =0x33661122   ; z
    
    ; Parameter passing
    MOV     R0, R4            ; a = x
    MOV     R1, R5            ; b = y
    MOV     R2, R6            ; c = z
    BL      logical_and       
    MOV     R7, R0            ; result = return value

logical_and
    AND     R0, R0, R1        ; a & b
    AND     R0, R0, R2        ; & c
    BX      LR                ; Return
\end{lstlisting}
\end{example2}

