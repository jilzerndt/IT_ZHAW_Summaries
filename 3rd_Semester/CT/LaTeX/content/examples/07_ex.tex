\subsection{Examples}







\begin{KR}{Stack Initialization and Memory Layout}

1. Stack Organization:
\begin{lstlisting}[language=armasm, style=basesmol]
; Memory Layout
0x20030000  Stack-base (initial SP value)
           | Stack grows downward
0x20020000  |
           | Available stack space
0x20010000  |
0x20000000  Start of SRAM (Stack-limit)
\end{lstlisting}

2. Initialization Process:
\begin{itemize}
  \item At reset, processor loads initial SP from address 0x00000000
  \item Vector table first entry contains Stack-base address
  \item SP initialized to this value before any code execution
\end{itemize}

3. Setting Stack Base Address:
\begin{lstlisting}[language=armasm, style=basesmol]
    AREA    RESET, DATA, READONLY
    EXPORT  __Vectors
__Vectors
    DCD     0x20030000      ; Initial SP value
    DCD     Reset_Handler   ; Reset vector
    ; Rest of vector table...
\end{lstlisting}

4. Determining Stack Size:
\begin{itemize}
  \item Calculate maximum stack depth:
    \begin{itemize}
      \item Count local variables
      \item Consider nested function calls
      \item Include interrupt handler needs
    \end{itemize}
  \item Add safety margin (e.g., 20\%)
  \item Ensure stack stays within SRAM bounds
\end{itemize}

5. Stack Pointer Rules:
\begin{itemize}
  \item Must be word-aligned (multiple of 4)
  \item Must point to valid SRAM
  \item Must not overlap with other memory regions
  \item Stack-limit < SP <= Stack-base always true
\end{itemize}

6. Example Stack Size Calculation:
\begin{lstlisting}[language=armasm, style=basesmol]
; Example for a simple program
Local vars:        20 bytes
Function params:   16 bytes
Saved registers:   32 bytes
Nested calls:      24 bytes
ISR overhead:      32 bytes
Safety margin:     25 bytes (20%)
------------------------
Total needed:     149 bytes

; Round up to nearest word boundary
Actual allocation: 152 bytes (38 words)
\end{lstlisting}
\end{KR}



\begin{KR}{Stack Analysis in Nested Subroutine Calls} from exercise sheet 8\\
Steps to analyze stack content in nested subroutine calls:

\textbf{1. Track initial stack pointer value}: Note starting SP value
\begin{itemize}
  \item Stack grows downward in memory
\end{itemize}

\textbf{2. For each PUSH instruction}: Subtract 4 bytes per register
\begin{itemize}
  \item Write values in order specified
  \item Remember which subroutine saved what
\end{itemize}

\textbf{3. For subroutine calls (BL)}: LR gets set to return address
\begin{itemize}
  \item If LR needs to be saved (nested calls), it must be PUSHed
\end{itemize}

\textbf{4. Keep track through nested calls}:
\begin{itemize}
  \item Each nested level maintains its own stack frame
  \item Stack unwinds in reverse order when returning
\end{itemize}
\end{KR}

\begin{example2}{Simple Stack Analysis}\\
Consider this simple nested call sequence:

\begin{lstlisting}[language=armasm, style=basesmol]
main    ; SP = 0x20002000
        LDR     R0, =0x11111111
        LDR     R1, =0x22222222
        BL      funcA          ; Z1
        B       endless

funcA   PUSH    {R0, R1, LR}  ; Z2
        BL      funcB
        POP     {R0, R1, PC}

funcB   PUSH    {R4, LR}      ; Z3
        MOVS    R4, #42
        POP     {R4, PC}
\end{lstlisting}

Stack contents at Z1, Z2, Z3:

\begin{lstlisting}[style=basesmol]
Z1: Stack is empty, SP = 0x20002000

Z2: Stack from top (higher address):           
0x20001FF4: Return addr to main
0x20001FF0: 0x22222222 (R1)
0x20001FEC: 0x11111111 (R0)
SP = 0x20001FEC

Z3: Stack from top (higher address):
0x20001FE4: Return addr to funcA
0x20001FE0: R4 value
0x20001FF4: Return addr to main (from Z2)
0x20001FF0: 0x22222222 (R1 from Z2)
0x20001FEC: 0x11111111 (R0 from Z2)
SP = 0x20001FE0
\end{lstlisting}
\end{example2}

\begin{KR}{Stack Frame Management Patterns}
Guidelines for managing stack frames:

1. Basic frame setup:
\begin{lstlisting}[language=armasm, style=basesmol]
function_start
    ; Save registers and create frame
    PUSH    {R4-R7, LR}     ; Save modified registers
    SUB     SP, SP, #12     ; Allocate local variables
    
    ; Function body...
    
    ; Clean up frame and return
    ADD     SP, SP, #12     ; Deallocate locals
    POP     {R4-R7, PC}     ; Restore and return
\end{lstlisting}

2. Accessing stack variables:
\begin{lstlisting}[language=armasm, style=basesmol]
    ; Local variable access
    STR     R0, [SP, #0]    ; First local variable
    STR     R1, [SP, #4]    ; Second local variable
    
    ; Later access
    LDR     R0, [SP, #0]    ; Load first variable
    LDR     R1, [SP, #4]    ; Load second variable
\end{lstlisting}

3. Nested function calls:
\begin{lstlisting}[language=armasm, style=basesmol]
outer_func
    PUSH    {R4-R6, LR}     ; Save registers
    ; Function body...
    BL      inner_func      ; Call inner function
    ; Continue processing...
    POP     {R4-R6, PC}     ; Return
\end{lstlisting}
\end{KR}

\begin{example2}{Stack-Based Local Variables}
Example of managing local array on stack:
\begin{lstlisting}[language=armasm, style=basesmol]
process_data
    PUSH    {R4-R7, LR}     ; Save registers
    SUB     SP, SP, #16     ; Space for 4-word array
    
    ; Initialize array on stack
    MOVS    R0, #1          
    STR     R0, [SP, #0]    ; array[0] = 1
    MOVS    R0, #2
    STR     R0, [SP, #4]    ; array[1] = 2
    
    ; Process array elements
    LDR     R1, [SP, #0]    ; Load array[0]
    LDR     R2, [SP, #4]    ; Load array[1]
    ADDS    R0, R1, R2      ; Compute sum
    
    ; Clean up and return
    ADD     SP, SP, #16     ; Remove local array
    POP     {R4-R7, PC}     ; Return
\end{lstlisting}

Key points:
\begin{itemize}
  \item Allocate space with SUB SP
  \item Access relative to SP
  \item Clean up stack before return
\end{itemize}
\end{example2}

\begin{KR}{Reentrant Function Implementation}
Guidelines for writing reentrant functions:

1. Save all modified registers:
\begin{lstlisting}[language=armasm, style=basesmol]
    ; R0-R3 are caller-saved
    PUSH    {R4-R7, LR}     ; Save callee-saved
    
    ; Use R4-R7 for local variables
    MOV     R4, R0          ; Save parameter
    
    ; Restore before return
    POP     {R4-R7, PC}
\end{lstlisting}

2. Use stack for local storage:
\begin{lstlisting}[language=armasm, style=basesmol]
    ; Allocate local storage
    SUB     SP, SP, #8      ; Two local variables
    STR     R0, [SP, #0]    ; Save first value
    STR     R1, [SP, #4]    ; Save second value
    
    ; Access locals via SP
    LDR     R0, [SP, #0]    ; Load first value
\end{lstlisting}

3. Handle nested calls:
\begin{lstlisting}[language=armasm, style=basesmol]
    ; Save state before nested call
    PUSH    {R4, LR}        ; Save working register
    BL      other_func      ; Make nested call
    POP     {R4, PC}        ; Restore and return
\end{lstlisting}
\end{KR}

\begin{example2}{Recursive Function}
Implementation of recursive factorial calculation:
\begin{lstlisting}[language=armasm, style=basesmol]
factorial
    ; Input in R0, result in R0
    PUSH    {R4, LR}        ; Save registers
    MOVS    R4, R0          ; Save n
    
    CMP     R0, #1          ; Check base case
    BLE     fact_end        ; Return 1 if n <= 1
    
    SUBS    R0, R0, #1      ; n-1
    BL      factorial       ; Recursive call
    MULS    R0, R4, R0      ; n * factorial(n-1)
    B       fact_return
    
fact_end
    MOVS    R0, #1          ; Return 1
    
fact_return
    POP     {R4, PC}        ; Return
\end{lstlisting}

Stack growth pattern:
\begin{itemize}
  \item Each call adds 8 bytes (R4, LR)
  \item Maximum depth = n-1 calls
  \item Stack unwinds on returns
\end{itemize}
\end{example2}

