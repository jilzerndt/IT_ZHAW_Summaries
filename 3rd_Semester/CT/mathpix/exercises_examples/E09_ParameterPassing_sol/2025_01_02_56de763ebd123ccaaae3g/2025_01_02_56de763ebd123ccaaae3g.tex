\documentclass[10pt]{article}
\usepackage[ngerman]{babel}
\usepackage[utf8]{inputenc}
\usepackage[T1]{fontenc}
\usepackage{amsmath}
\usepackage{amsfonts}
\usepackage{amssymb}
\usepackage[version=4]{mhchem}
\usepackage{stmaryrd}

\begin{document}
\section*{CT1 Übungsaufgaben}
\section*{Datenübergabe / Schnittstelle zu Hochsprachen}
\section*{Aufgabe 1}
Gegeben ist folgernder C-Code:

\begin{verbatim}
uint32_t logical_and(uint32_t a, uint32_t b, uint32_t c)
{
    return a & b & c;
}
int32_t main(void)
{
    uint32_t result;
    uint32-t x = 0x11223344;
    uint32-t y = 0xFFFF0000;
    uint32_t z = 0x33661122;
    result = logical_and(x, y, z);
}
\end{verbatim}

Beim Start des Programmes wird die Variable x in R4, y in R 5 und z in R 6 abgelegt. Die Variable result wird in R7 abgelegt.

\begin{enumerate}
  \item Welche Schritte führt der Caller (main) vor dem Aufruf der Funktion logical\_and() durch? Wie werden die Parameter übergeben?\\
Die Variablen in den Registern R4 bis R6 werden nach R0 bis $R 2$ kopiert und so der Funktion übergeben. ( $R 6=>R 2, R 5=>R 1, R 4=>R 0$ )\\
$\qquad$
  \item Wir gibt die Funktion logical\_and() den Rückgabewert zurück?
\end{enumerate}

Der Rückgabewert wird via R0 zurückgegeben.\\
3. Welche Operation führt der Call nach dem Aufruf der Funktion logical\_and() durch?

Der Rückgabewert wird von R0 nach R7 kopiert

\section*{Aufgabe 2}
Gegeben ist das folgende C-Programm:

\begin{verbatim}
#include <utils_ctboard.h>
void swap_bad(int32_t c, int32_t d)
{
    /*WARNING: This code does not work*/
    int32_t temp = c;
    c = d;
    d = temp;
}
int32_t main(void)
{
    int32_t a = 3, b = 5;
    swap_bad(a,b);
    write_word(0x60000300, a);
}
\end{verbatim}

a. Die Swap-Funktion swap\_bad() lässt sich ohne Fehler kompilieren und ausführen. Nach ihrem Aufruf in main() hat die Variable a jedoch immer noch den Wert 3. Erläutern Sie anhand der Calling Convention wo das Problem liegt.\\
Beim Aufruf der Funktion swap bad() werden die übergebenen Parameter in die Register R0, R1 kopiert und in der Funktion auch korrekt verändert. Jedoch werden die Änderungen nicht an den Caller zurückgegeben. Daher sind die Änderungen in main() nicht sichtbar.\\
b. Schreiben Sie eine Funktion swap\_good() die diesen Fehler behebt und korrekt funktioniert

\begin{verbatim}
void swap_good(int32_t *c, int32_t *d) {
    int32_t temp = *c;
    *c = *d;
    *d = temp;
}
int32_t main(void) {
    int\overline{3}2_t a = 3, b = 5;
    swap_good(&a, &b);
    write_word(0x60000300, a);
}
\end{verbatim}

\section*{Aufgabe 3}
Gegeben sei das folgende C-Programm:

\begin{verbatim}
int32_t fakultaet_recursive(int32_t n)
{
    if(n< 2){
        return 1;
        }
        else{
                return n * fakultaet_recursive(n-1); // Break Point set here
    }
}
int32_t main(void)
{
    int32_t n = 20;
    int32_t result = fakultaet_recursive(n);
}
\end{verbatim}

Das C-Programm berechnet die Fakultät von 20 und verwendet dazu Rekursion. Dabei ruft sich die Funktion selbst wieder auf. Dies wird so lange wiederholt, bis die Abbruchbedingung ( $n<2$ ) erreicht wird.

Wenn das Programm zum zweiten Mal beim Breakpoint in der Funktion fakultaet\_recursive() gestoppt wird, ist das Stackframe der main()-Funktion und der aufgerufenen fakultaet\_recursive()-Funktionen 16 Byte gross. Die Variable n wird in einem Register gespeichert.\\
Wie gross wird das Stackframe beim Ausführen des Programmes maximal?\\
Da die Variable $n$ in einem Register gespeichert ist und das Stackframe der main()Funktion somit 0 ist, wird bei jedem Aufruf der Funktion fakultaet recursive() das Stackframe um 8 Byte (R4 und LR) erhöht. Nach zwanzig Aufrufen ist das Stackframe also 160 Byte gross.


\end{document}