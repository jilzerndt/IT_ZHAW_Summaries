\documentclass[10pt]{article}
\usepackage[ngerman]{babel}
\usepackage[utf8]{inputenc}
\usepackage[T1]{fontenc}
\usepackage{amsmath}
\usepackage{amsfonts}
\usepackage{amssymb}
\usepackage[version=4]{mhchem}
\usepackage{stmaryrd}

\begin{document}
\section*{CT1 Übungsaufgaben Arithmetische und Logische Instruktionen}
\section*{Aufgabe 1 - Codieren von arithmetischen Ausdrücken}
Für die folgenden Variablen wurde entsprechend Speicherplatz im Datenbereich reserviert:

\begin{center}
\begin{tabular}{lrr}
 & AREA & progData, DATA, READWRITE \\
Zahl1 & DCD & $0 ;(32$ bit $)$ \\
Zahl2 & DCD & $0 ;(32$ bit $)$ \\
Zahl3 & DCW & $0 ;(16$ bit) \\
Zahl4 & DCW & $0 ;(16$ bit $)$ \\
Zahl5 & DCB & $0 ;(8$ bit) \\
Zahl6 & DCB & $0 ;(8$ bit) \\
\end{tabular}
\end{center}

Codieren Sie die folgenden Ausdrücke in Assembler.\\
a) $\mathrm{R} 0=\mathrm{R} 1+\mathrm{R} 2$

$$
\text { ADDS } \quad \mathrm{R} 0, \mathrm{R} 1, \mathrm{R} 2
$$

b) $\mathrm{R} 0=\mathrm{R} 1+\mathrm{R} 2+\mathrm{R} 3$

\begin{center}
\begin{tabular}{ll}
ADDS & R0, R1, R2 \\
ADDS & R0, R0, R3 \\
\end{tabular}
\end{center}

c) $\mathrm{R} 8=\mathrm{R} 8+1$

MOVS R0,\#1\\
ADD R8,R8,R0\\
d) $\mathrm{R} 1=\mathrm{R} 2+200$\\
; ADDS R1,R2,\#200 falsch, rd = rn\\
MOVS R1,\#200\\
ADD R1,R1,R2\\
e) $\mathrm{R} 10=\mathrm{R} 8-\mathrm{R} 7$

\begin{center}
\begin{tabular}{ll}
MOV & R0, R8 \\
SUBS & R0, R0, R7 \\
MOV & R10,R0 \\
\end{tabular}
\end{center}

f) $R 8=-R 8$

MOV R7, R8\\
RSBS R7,\#0\\
MOV R8,R7\\
g) Zahl1 = Zahl1 - 1

\begin{center}
\begin{tabular}{ll}
LDR & R1, =Zahl1 \\
LDR & R0, [R1] \\
SUBS & R0,\#1 \\
STR & R0, [r1] \\
\end{tabular}
\end{center}

h) Zahl1 = Zahl1 + Zahl2

\begin{verbatim}
LDR R0, =Zahl1
LDR R1, [R0]
LDR R2, =Zahl2
LDR R3, [R2]
ADDS R1, R1, R3
STR R1, [R0]
\end{verbatim}

i) Zahl3 = Zahl3 - Zahl4 (unsigned)\\
LDR R0,=Zahl3\\[0pt]
LDRH R1, [R0]\\
LDR R2,=Zah14\\[0pt]
LDRH R3,[R2]\\
SUBS R1,R1,R3\\[0pt]
STRH R1, [R0]\\
j) Zahl1 = R5 * Zahl3 (unsigned)\\
LDR R0,=Zahl1\\
LDR R1,=Zahl3\\[0pt]
LDRH R2,[R1]\\
MULS R2,R5,R2\\[0pt]
STR R2,[R0]

\section*{Aufgabe 2 - Multiword Addition/Subtraktion}
Für die folgenden 64-bit Variablen wurde entsprechend Speicherplatz im Datenbereich reserviert:

\begin{center}
\begin{tabular}{llc}
 & AREA & progData, DATA, READWRITE \\
Long1 & DCD & $0 ;$ low word \\
 & Long2 & DCD \\
 & 0; high word &  \\
 & LCD & $0 ;$ low word \\
 & DCD & $0 ;$ high word \\
 & DCD & $0 ;$ low word \\
 & $0 ;$ high word &  \\
\end{tabular}
\end{center}

Implementieren Sie die folgenden 64-bit Operationen in Assembler.\\
a) Long3 $=$ Long1 + Long2

\begin{center}
\begin{tabular}{ll}
LDR & R0, =Long1 \\
LDM & R0!, \{R1-R4\} \\
ADDS & R1, R1, R3 \\
ADCS & R2, R2,R4 \\
LDR & R0, =Long3 \\
STM & R0!, \{R1,R2\} \\
\end{tabular}
\end{center}

b) Long3 $=$ Long2 - Long1

\begin{center}
\begin{tabular}{ll}
LDR & R0, =Long1 \\
LDM & R0!, \{R1-R4\} \\
SUBS & R3, R3, R1 \\
SBCS & R4, R4, R2 \\
LDR & R0, =Long3 \\
STM & R0!, \{R3,R4\} \\
\end{tabular}
\end{center}


\end{document}