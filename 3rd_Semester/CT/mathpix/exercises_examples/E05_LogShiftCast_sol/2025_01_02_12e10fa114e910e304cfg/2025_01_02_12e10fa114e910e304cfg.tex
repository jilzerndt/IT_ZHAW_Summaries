% This LaTeX document needs to be compiled with XeLaTeX.
\documentclass[10pt]{article}
\usepackage[utf8]{inputenc}
\usepackage{ucharclasses}
\usepackage{amsmath}
\usepackage{amsfonts}
\usepackage{amssymb}
\usepackage[version=4]{mhchem}
\usepackage{stmaryrd}
\usepackage{polyglossia}
\usepackage{fontspec}
\setmainlanguage{german}
\setotherlanguages{hindi}
\newfontfamily\hindifont{Noto Serif Devanagari}
\newfontfamily\lgcfont{CMU Serif}
\setDefaultTransitions{\lgcfont}{}
\setTransitionsFor{Hindi}{\hindifont}{\lgcfont}

\begin{document}
\section*{CT1 Übungsaufgaben Log- Shift-Operationen und Casting}
\section*{Aufgabe 1 - Logische Instruktionen}
Geben Sie die Assemblerinstruktionen für die folgenden Fälle an.\\
a) Invertieren des Inhaltes des Registers R1 (Bilden des Einerkomplementes)

\begin{verbatim}
MVNS R1,R1
\end{verbatim}

b) Verändern Sie den Inhalt des Registers R1 so, dass

\begin{itemize}
  \item die Bits $3 . .0$ alle eins,
  \item die Bits $7 . .4$ alle null,
  \item die Bits 17-16 alle invertiert und
  \item alle übrigen Bits unverändert sind.
\end{itemize}

\begin{verbatim}
MOVS RO,#0xF
ORRS R1,R0 ; 3-0 -> 1
MOVS RO,#0xFO
BICS R1,R0 ; 7-4 -> 0
MOVS R0,#0x30
LSLS R0,#12 ;0x30000 ,shift 4 nibbles = 12bit
EORS R1,R0 ; 17-16 invertiert
\end{verbatim}

\section*{Aufgabe 2 - Multiplikation mit einer Konstanten}
Schreiben Sie eine Codesequenz in Assembler welche den vorzeichenlosen Inhalt des Registers R7 mit der Dezimalzahl 43 multipliziert und im Register R7 abspeichert.\\
a) Verwenden Sie die Multiplikationsinstruktion

\begin{verbatim}
MOVS RO,#43
MULS R7,R0,R7
\end{verbatim}

b) Ersetzen Sie die Multiplikationsinstruktion durch Shift- und Additionsbefehle

\begin{verbatim}
MOVS R7,R0 ;1
LSLS RO,RO,#1 ;2
ADDS R7,R7,R0
LSLS RO,RO,#2 ; 8=4*2
ADDS R7,R7,R0
LSLS RO,RO,#2 ; 32=8*4
ADDS R7,R7,R0
\end{verbatim}

\section*{Aufgabe 3-Reverse Engineering}
Die folgenden Assemblersequenzen wurden durch den bcc Compiler aus einem C Programm erzeugt. Schreiben Sie einen möglichen C-Code, welcher diese Assemblersequenz erzeugt.

Die Speicherstellen im Assembler werden den ursprünglichen Variablen im C-Programm wie folgt zugeordnet:

\begin{center}
\begin{tabular}{|c|c|c|c|c|}
\hline
ux & DCD & ? & प & uint32\_t ux \\
\hline
uy & DCD & ? & [ & uint $32-$ t uy \\
\hline
uz & DCD & ? & [ & uint32 t uz \\
\hline
\end{tabular}
\end{center}

\begin{center}
\begin{tabular}{|c|c|c|}
\hline
 & Assembler & C \\
\hline
Beispiel & \( \begin{array}{ll} \text { LDR } & \text { R0, }=\text { ux } \\ \text { LDR } & \text { R1, [R0] } \\ \text { ADDS } & \text { R1,\#1 } \\ \text { STR } & \text { R1, }[R 0] \end{array} \) & ux++; \\
\hline
a) & \begin{tabular}{ll}
LDR & R0, =ux \\
LDR & R1, [R0] \\
LSLS & R1,R1, \#1 \\
 &  \\
LDR & R2, =uy \\
LDR & R3,[R2] \\
ADDS & R3,R1 \\
LSLS & R3,R3,\#3 \\
STR & R3, [R2] \\
\end{tabular} & \texttt{Lsg: uy = 8 * (2 * ux + uy); oder uy = ((ux << 1) + uy) << 3;} \\
\hline
b) & \begin{tabular}{ll}
LDR & R0, =ux \\
LDR & R1, [R0] \\
LDR & R2, =uy \\
LDR & R3, [R2] \\
LDR & R4, =uz \\
LDR & R5,[R4] \\
LSRS & R1,R1,\#3 \\
LSLS & R3, R3,\#4 \\
ORRS & R1,R3 \\
MVNS & R1,R1 \\
ANDS & $R 1, R 5$ \\
STR & R1,[R4] \\
\end{tabular} & \texttt{Lsg: uz = \~{}( (ux >> 3) | (uy << 4)) \& uz; oder uz = \~{}( (ux / 8) | (uy * 16)) \& uz;} \\
\hline
\end{tabular}
\end{center}

\section*{Aufgabe 4 - Explicit Casting in C}
a) Gegeben ist der folgende C Code:

\begin{verbatim}
uint8_t ux = 100;
int8_t sx = (int8_t)ux;
\end{verbatim}

Als welche Dezimalzahl wird der Inhalt der Variable sx nach dem Cast interpretiert?

\begin{verbatim}
100d --> ux hat Speicherinhalt 0x64 oder Binär 0110'0100
sx hat denselben Speicherinhalt, wird aber als signed
interpretiert. Da das höchstwertigste Bit nicht gesetzt ist, hat die
Interpretation aber in diesem Fall keinen Einfluss.
sx wird ebenfalls als 100d interpretiert
\end{verbatim}

b) Gegeben ist der folgende C Code:

\begin{verbatim}
int8_t sx = -10;
uint8_t ux = (uint_8)sx;
\end{verbatim}

Als welche Dezimalzahl wird der Inhalt der Variable ux nach dem Cast interpretiert?

\begin{verbatim}
-10d --> sx hat Speicherinhalt 0xF6 oder Binär 1111'0110
(-128+64+32+16+4 + 2)
ux hat denselben Speicherinhalt, wird aber als unsigned
interpretiert. So ergibt sich 128 + 64 + 32 + 16 + 4 + 2 = 246d
\end{verbatim}


\end{document}