\section{Network Layer}
\paragraph{Schicht 3: Internet}

\includegraphics[width=1\linewidth]{images/orientierung_network_layer.png}

\begin{definition}{Die Netzwerkschicht}
    \begin{itemize}
        \item verbindet verschiedene Netze
        \item ist in Endsystemen und den verbindenden Elementen (Routern) implementiert
        \item verbirgt die Besonderheiten der einzelnen Netze
    \end{itemize}
    Grundlegende Unterscheidung zwischen
    \begin{itemize}
        \item Forwarding (Weiterleitung der Daten)
        \begin{itemize}
            \item Aufgrund von Routingtabellen
        \end{itemize}
        \item Routing (Aufbau der Routingtabellen)
        \begin{itemize}
            \item Durch statische Konfiguration oder
            \item Dynamisch durch Routing-Protokolle
        \end{itemize}
    \end{itemize}
\end{definition}

\begin{concept}{Was braucht es damit das Internet "funktioniert"?}
    \begin{itemize}
        \item Information, zu welchem Knoten Daten geschickt werden sollen → Adressierung
        \item Geräte, die die verschiedenen LANs miteinander verbinden → Router
        \item Wie weiss ein Router, wohin die Daten weitergeleitet werden sollen? → Routing
        \item Was geschieht, wenn unterschiedliche LANs unterschiedliche Framegrössen unterstützen? → Fragmentierung
        \item Welche Helferlein fürs Internet gibt es? Was kann ich machen, wenn etwas nicht wie erwartet funktioniert? → Internet Control Message Protocol (ICMP)
    \end{itemize}
\end{concept}

\begin{lemma}{Grundsätze des Internets}
    \begin{itemize}
        \item Jedes Netzwerk soll für sich selbst funktionsfähig sein
        \item Die Kommunikation basiert auf «best effort»
        \item Die Verbindung der Netze erfolgt durch Black Boxes
        \item Keine zentrale Funktionssteuerung wird benötigt
    \end{itemize}
\end{lemma}

\begin{definition}{Kommunikationsobjekte}\\
    Die vier Grundsätze führten zur Wahl eines paketvermittelnden Netzes auf der Grundlage von vier Layern (siehe OSI Modell)

    \begin{itemize}
        \item \textcolor{orange}{(Application-)Message/Stream} Layer 5-7
        \item \textcolor{green}{(Transport-)Paket, Datagram} Layer 4
        \item \textcolor{blue}{(IP-)Paket (früher Datagram)} Layer 3
        \item \textcolor{yellow}{(HW-specific) Frame} Layer 1-2
    \end{itemize}
\end{definition}

\begin{concept}{Eigenschaften des Internet Layers}
    \begin{itemize}
        \item Die Basis bildet ein verbindungsloser Network Layer
        \begin{itemize}
            \item Er hält das virtuelle Netz von Teilnetzen zusammen.
            \item Er leitet IP-Pakete zwischen zwei beliebigen Hosts weiter; egal in welchen Teilnetzen sich die Hosts befinden.
        \end{itemize}
        \item Kümmert sich nur um den Transport der IP-Pakete
        \begin{itemize}
            \item Es ist nicht die Aufgabe des Internet Layers, den Verlust oder die fehlerhafte Übertragung von Paketen zu korrigieren.
            \item Die Reihenfolge der Pakete kann ändern, falls Pakete auf verschiedenen Pfaden übertragen werden.
        \end{itemize}
        \item Bei IP müssen die höheren Layer zusätzlich erwünschte Funktionalität übernehmen. Beispiele:
        \begin{itemize}
            \item Fehlerfreie, komplette Übertragung
            \item Richtige Reihenfolge
            \item Flusskontrolle
        \end{itemize}
    \end{itemize}
\end{concept}

\subsection{Netzwerk Applikationen und Protokolle}

\subsubsection{Routing}

\begin{definition}{Router}\\
     Komponenten, die es erlauben Subnetze miteinander zu verbinden. Router haben eine ähnliche Funktion wie Bridges, allerdings arbeiten sie auf dem Network Layer.
    \begin{itemize}
        \item Router empfangen nur Pakete, die direkt an sie adressiert sind.
        \item Die Weiterleitung erfolgt anhand der Network Layer Adresse.
        \item Benutzen immer den optimalen Pfad.
    \end{itemize}
        \includegraphics[width=0.7\linewidth]{images/router.png}
\end{definition}
    \includegraphics[width=1\linewidth]{images/router2.png}

\begin{concept}{Routing and Forwarding}
    \begin{itemize}
        \item Routing: Aufbau und Update der Routingtabellen in den Knoten
            \begin{itemize}
                \item dazu müssen die Router den optimalen Pfad zu jedem Host kennen
                \item In kleinen oder Teilnetzen: Statische Konfiguration
                \item in grösseren Netzen: Dynamisch durch Routing-Protokolle die Topologie des Netzes ermitteln, daraus ideale Pfade bestimmen
            \end{itemize}
        \item Forwarding: Weiterleiten der Daten
        \begin{itemize}
            \item Aufgrund von Routingtabellen Datenpakete weiterleiten
        \end{itemize}
    \end{itemize}
    Jeder Knoten auf dem Weg, einschliesslich dem Sender, wertet seine Routingtabelle aus und trifft Forwarding-Entscheidungen
\end{concept}

\begin{definition}{Routing-Tabelle}\\
    Enthält Informationen, wie jedes Netz (und damit jedes Interface) erreicht werden kann
    \begin{itemize}
        \item Für Weiterleitungsentscheidung notwendige Informationen
        \begin{itemize}
            \item Welche Netze (Netzadresse, Subnetzmaske) gibt es? Wichtig für die Skalierbarkeit: Netz-Adressen, nicht Interface Adressen!
            \item Über welches Interface sind diese Netze erreichbar?
            \item Ist das Zielnetz erreicht, oder muss das Paket an einen nächsten Router (IP-Adresse) weitergegeben werden?
        \end{itemize}
        \item Sortiert nach der Länge der Netzmaske
        \item Von oben nach unten durchsucht
        \item Verglichen werden die Netzadressen
        \item erster Eintrag der passt wird verwendet, default Eintrag am Schluss passt immer
    \end{itemize}
        \includegraphics[width=0.75\linewidth]{images/bsp_routing_tabelle.png}
\end{definition}

\begin{concept}{Flaches Routing}
    \begin{itemize}
        \item Router kennt explizite Wege zu jedem einzelnen Zielnetz
        \begin{itemize}
            \item Pakete an unbekannte Netze werden verworfen
        \end{itemize}
        \item Redundanz möglich durch Speichern mehrerer Wege ins gleichen Netz
        \begin{itemize}
            \item Wege müssen nicht gleich gut sein
        \end{itemize}
        \item Einsatz in stark vermaschten Netzen oder im zentralen Bereich (Backbone)
        \item Sehr grosse Routing-Tabellen
    \end{itemize}
        \includegraphics[width=1\linewidth]{images/flaches_routing.png}
\end{concept}

\begin{example2}{Flaches Routing Übung}\\
Was geschieht mit dem IP-Paket?
    \begin{itemize}
        \item Kein Unterbruch?\\ Es wird nach gemäss dem 4. Eintrag der Routingtabelle von Router B an p0 weitergeleitet
        \item Unterbruch von p0 / Router B ? \\ Es wird gemäss Eintrag 5 in der Routingtabelle von Router B an p2 weitergeleitet.
        \item zusätzlicher Unterbruch p0 / Router C ?\\ Router C kann das IP-Paket nicht weiterleiten, es IP-Paket erreicht den Empfänger nicht.
    \end{itemize}
        \includegraphics[width=1\linewidth]{images/flaches_routing_bsp.png}
\end{example2}

\begin{concept}{Hierarchisches Routing (Default)}
    \begin{itemize}
        \item Router kennt die direkt angeschlossenen Netze seiner Interfaces und genau einen anderen Router, an den er alles schickt, was für andere Netze bestimmt ist
        \begin{itemize}
            \item Der nächste Router geht genau gleich vor
        \end{itemize}
        \item Einsatz am „Rand“ von Netzen Hosts, ccess Router
        \item Kleine Routing-Tabellen mit jeweils einem Default-Eintrag
    \end{itemize}
        \includegraphics[width=1\linewidth]{images/hierarchisches_routing.png}
\end{concept}

\subsubsection{Internet Protokolle (IP)}

\begin{definition}{Hierarchische Adressierung}\\
IP-Adressen sind zweistufig hierarchisch\\
    \includegraphics[width=1\linewidth]{images/hierarchische_ip_adressierung.png} 
\end{definition}

\begin{concept}{Darstellung von Protokoll Daten}
    \begin{itemize}
        \item Aufbau der IP-Adresse
    \end{itemize}
        \includegraphics[width=1\linewidth]{images/aufbau_ip_adresse.png}
    \begin{itemize}
        \item Prinzip der Darstellung in RFCs
    \end{itemize}
        \includegraphics[width=1\linewidth]{images/darstellung_ip_rfc.png}
\end{concept}

\begin{definition}{Terminologie}
    \begin{itemize}
        \item Sender und Empfänger werden im TCP/IP Referenzmodell als Hosts bezeichnet
        \item Der Internet Layer stellt ein virtuelles Netz mit einer einheitlichen Adressierung zur Verfügung $\rightarrow$ IP-Adressen
        \begin{itemize}
            \item Die IP-Adresse identifiziert ein Host-Interface (und nicht den Host) eindeutig innerhalb eines Netzwerks
            \item Jeder Host hat mindestens eine Adresse
            \item Multi-Homed Hosts haben mehrere IP-Adressen
        \end{itemize}
    \end{itemize}
    \includegraphics[width=1\linewidth]{images/ip_adressen_terminologie.png}
\end{definition}

\begin{formula}{Schreibweise}
    \begin{itemize}
        \item IP-Adressen werden durch vier Zahlen, unterbrochen durch Punkte, dargestellt
        \item Jede Zahl entspricht einem Byte in dezimaler Darstellung
        \item Jede Zahl hat einen Wert im Bereich 0 – 255
    \end{itemize}
    \includegraphics[width=1\linewidth]{images/schreibweise_ip.png}
\end{formula}

\begin{formula}{Netzadresse}
    \begin{itemize}
        \item Netzadresse ist reserviert: Darf nicht für Interfaces verwendet werden.
        \item Tiefste Adresse im Subnetz (Interface-Adressbits alle 0)
        \item Berechnet durch:
    \end{itemize}
    \includegraphics[width=1\linewidth]{images/netzadresse.png}
\end{formula}

\begin{formula}{Broadcast-Adresse}
    \begin{itemize}
        \item Broadcast-Adresse adressiert alle Interfaces in einem Subnetze (reserviert)
        \item Höchste Adresse im Subnetz (All Ones Broadcast)
        \item Berechnet durch:
    \end{itemize}
        \includegraphics[width=1\linewidth]{images/broadcastadresse.png}
\end{formula}

\begin{concept}{Subnetzmaske}
    bestimmt die Grenze zwischen Netz- und Interface-Adressbits:\\
        \includegraphics[width=1\linewidth]{images/subnetzmaske.png}\\
    \includegraphics[width=1\linewidth]{images/subnetzmaske_bsp.png}   
\end{concept}

\begin{KR}{Rechnen mit Netzmasken}\\
    Typische Internet-Adressen Aufgabenstellung: Berechnen Sie die fehlenden Informationen\\
        \includegraphics[width=1\linewidth]{images/rechenne_mit_netzmasken.png}
\end{KR}

\begin{formula}{Netzmasken}\\
        \includegraphics[width=0.2\linewidth]{images/numbers.png}
\end{formula}

\begin{example}
    Welche Werte können in einer Subnetzmaske vorkommen? Betrachten Sie das dritte Byte als Beispiel und geben Sie für jeden Wert an, wie viele Interfaces in diesem Fall adressiert werden können. \\
        \includegraphics[width=0.5\linewidth]{images/subnetzmaske_example.png}\\
    Wie kommt diese Zahl zustande?\\
    \includegraphics[width=0.5\linewidth]{images/subnetzmaske_bsp_2.png}
\end{example}

\begin{KR}{Internet-Adressierung (IPv4)}
    \begin{itemize}
        \item Netzadresse: Tiefste Adresse im Subnetz 
        \begin{itemize}
            \item Interface-Adresse AND Subnetzmaske
        \end{itemize}
        \item Broadcast: Höchste Adresse im Subnetz 
        \begin{itemize}
            \item Interface-Adresse OR Invertierte Subnetzmaske
        \end{itemize}
    \end{itemize}
        \includegraphics[width=1\linewidth]{images/ipv4.png}
\end{KR}

\begin{example}
    \begin{itemize}
        \item Interface 000…000 
        \begin{itemize}
            \item 32 – Länge vom Subnetz
        \end{itemize}
        \item Subnetzmaske 255.255.240.0 
        \begin{itemize}
            \item 1111’1111.1111’1111.1111’0000.0000’0000
        \end{itemize}
        \item Subnetz 160.85.16.0/20 
        \begin{itemize}
            \item 20 = Länge
        \end{itemize}
    \end{itemize}
    \includegraphics[width=1\linewidth]{images/ipv4_example.png}
\end{example}

\subsubsection{Classful Routing: Sub-/Supernetting}

\begin{concept}{Classful Routing}\\
    Ursprünglich war der IP Adressbereich in fünf Netzklassen (A - E) eingeteilt
    \begin{itemize}
        \item Eine Prefix (die ersten 4 Adress-Bits) erlaubt die Bestimmung der Klasse
    \end{itemize}
        \includegraphics[width=1\linewidth]{images/classfulrouitng.png}
\end{concept}

\begin{formula}{Adressbereiche für Classful Routing}
    \begin{itemize}
        \item Die klassischen Netze fixer Grösse sind unflexibel
        \begin{itemize}
            \item Klasse C Netze sind für Unternehmen zu klein
            \item Klasse A Netze sind zu gross
            \item Klasse B Netze sind zu wenig
        \end{itemize}
        \item Abhilfe schafft CIDR – Classless Inter-Domain Routing
        \begin{itemize}
            \item Flexible Verwendung von Netzmasken beliebiger Länge
            \item Aufteilung grosser Netze in kleinere Subnetze, Zusammenfassen mehrerer kleiner Netze zu einem gemeinsamen grösseren Netz
        \end{itemize}
    \end{itemize}
\end{formula}

\begin{example2}{Classful Routing}\\
    Beispiel von 4 zusammengeschlossenen Netzen:\\
        \includegraphics[width=1\linewidth]{images/classful_routing.png}
\end{example2}

\begin{concept}{Supernetting}
    Zusammenfügen von kleinen Netzen\\
    Hintereinanderliegende Class C Netze können zu einem Netz zusammengefügt werden. \\
    Kann ebenfalls helfen, Routingtabellen in Routern zu verkleinern (Aggregate Routes)
\end{concept}

\begin{example}
    Beispiel: Zusammenfassen von 4 Class C Netzen (22 = 2 Bits der Subnetzmaske)\\
        \includegraphics[width=1\linewidth]{images/example_supernetting.png}
\end{example}

\begin{concept}{Subnetting}
    Aufteilung in kleinere Netze\\
    Die ZHAW besitzt das Klasse B Netz 160.85.0.0
    \begin{itemize}
        \item Total $2^{16} \cong 65000$  Hosts
        \item Die ZHAW möchte dieses in 8 kleinere Subnetze aufteilen → Subnetting
    \end{itemize}
    Verschieben der Netzmasken-Bits: $8 = 2^3$, es werden 3 "1" in der binären Netzmaske ergänzt
    \begin{itemize}
        \item 3 Bits identifizieren 8 Subnetze (000 → 111)
        \item Die Netzmaske verändert sich von /16 zu /19 (255.255.0.0 → 255.255.224.0)
        \item Der Interface-Anteil verändert sich von $2^{16}$ zu $2^{13}$ = 8192 IP Adressen pro Subnetz
    \end{itemize}
        \includegraphics[width=1\linewidth]{images/subnetting1.png}\\
    Damit haben wir 8 neue Subnetze mit den folgenden Netzadressen:\\
        \includegraphics[width=0.75\linewidth]{images/subnetting2.png}
    \begin{itemize}
        \item Der Netz-Anteil der binären Netzmaske hat nun 19 statt 16 "1" → Subnetzmaske: 255.255.224.0 oder /19
        \item Der Host-Anteil der binären Nutzmaske hat nun 13 statt 16 "0" → Anzahl Hostadressen 8'192
    \end{itemize}
    Das zweite Netz oben wird deshalb korrekt wie folgt gekennzeichnet:
    \begin{itemize}
        \item 160.85.32.0 / 255.255.224.0 oder 160.85.32.0 /19
    \end{itemize}
    Das fünfte Netz wird wie folgt gekennzeichnet:
    \begin{itemize}
        \item 160.85.128.0 / 255.255.224.0 oder 160.85.128.0 /19
    \end{itemize}
    \textcolor{pink}{Wichtige Regel: Eine Netzwerkadresse ist immer ein Vielfaches der Netzgrösse!}
\end{concept}

\begin{example2}{Flexible Aufteilung eines Netzbereiches}\\
    Ein KMU mit 4 Standorten hat von seinem ISP das Netz 193.72.32.0 /21 erhalten. Das KMU hat 3 grössere und einen kleineren Standort und will diese redundant verbinden.\\
        \includegraphics[width=1\linewidth]{images/flexible_aufteilung_netzbereich.png}    
\end{example2}

\subsubsection{Spezielle IP Adressen}

\begin{definition}{localhost}\\
    Loopback-Adressen
    \begin{itemize}
        \item Das gesamte A-Netz 127.0.0.0/8 ist für Loopback-Test reserviert
        \item Daten werden an ein emuliertes Loopback-Gerät geschickt, das sie direkt zurück gibt (kein Netzwerk/-Interface nötig).
    \end{itemize}
    \includegraphics[width=0.5\linewidth]{images/localhost.png}
\end{definition}

\begin{KR}{Key Takes Internet/IP}
    \begin{itemize}
        \item Die Netzwerkschicht verbindet einzelne, auf Schicht zwei verbundenen Netze, zu einem grossen virtuellen Netzwerk (einem Internet)
        \begin{itemize}
            \item Interna der einzelnen Netzwerke bleiben für die Endknoten verborgen
            \item Die einzelnen Netzwerke können ganz unterschiedliche Technologien verwenden
        \end{itemize}
        \item Die Netzwerkschicht erfüllt hierbei zwei Hauptaufgaben
        \begin{itemize}
            \item Routing (Bestimmung der Wege durch das Netz, Aufbau von Routingtabellen)
            \item Forwarding (Weiterleitung von Packets gemäss der Routingtabellen)
        \end{itemize}
        \item Um global Kommunikationsteilnehmer identifizieren (adressieren) zu können, wird ein hierarchisches Adress-Schema verwendet; Routing und Forwarding erfolgt aufgrund der Netzzugehörigkeit von Knoten, nicht aufgrund der Knotenadressen selbst
        \item Gemäss den Designkriterien bei der Entwicklung der TCP/IP Protokollsuite bietet IP einen verbindungslosen, unzuverlässigen Dienst
        \begin{itemize}
            \item Erlaubt dafür relativ einfache Implementierung von Routern und bietet ein robustes Verhalten bei Fehlern im Netz (Link- / Komponentenausfall)
        \end{itemize}
        \item IPv4 Adressen bestehen aus 32 Bit; diese sind in eine Netz- und einen Interface-Teil unterteilt
        \begin{itemize}
            \item Bitweises AND von Netzmaske und IP-Adresse ergibt die Netzadresse
            \item Sind alle Bits des Interfaces-Adressteils gleich „1“, so erhält man die Broadcastadresse im jeweiligen Netz
            \item Knoten mit übereinstimmender Netzadresse gehören zum gleichen Netz und können direkt (über Layer 2) miteinander kommunizieren, die Kommunikation zu allen anderen Knoten muss über Router verlaufen
        \end{itemize}
        \item Routingtabellen definieren, über welche Netzwerk-Interfaces und Router welche Netze erreichbar sind.
        \begin{itemize}
            \item Sie werden in den Hosts und Routern nach der Grösse der Netzmaske abgearbeitet
            \item Bei flachem Routing umfasst die Routingtabelle alle bekannten (im Internetwork vorhandenen) Netze
            \item Hierarchisches Routing arbeitet mit Default-Einträgen
        \end{itemize}
        \item Der Default-Eintrag in der Routingtabelle definiert, wohin IP-Pakete geroutet werden sollen, für die keine Eintrag in der Routing Tabelle passt.
    \end{itemize}
\end{KR}

\subsubsection{IPv4}

\begin{KR}{IP-Header Format}
    Ein IP-Packet besteht aus einem Header (min. 20 Byte) und Nutzdaten.
    \begin{itemize}
        \item \textcolor{blue}{Version} IPv4 / IPv6
        \item \textcolor{blue}{IHL} Header Length in 4-Byte (20 Byte → IHL = 5)
        \item \textcolor{blue}{Type of Service} neu Differentiated Services (DS), Erlaubt Priorisierung, Einteilung der Daten in Verkehrsklassen
        \begin{itemize}
            \item DSCP: spez. Verhalten bzgl. Weiterleitung
            \item ECN: kann drohende Überlast markieren
        \end{itemize}
        \item \textcolor{blue}{Total Length} Länge des IP-Packets (Header + Nutzdaten)
        \item \textcolor{yellow}{ID Number} Identifikation des IP-Pakets / Fragmente, erlaubt Identifikation zusammengehöriger Fragmente
        \item \textcolor{yellow}{Flags} Kontroll-Flags für Fragmentierung (0/DF/MF)
        \item \textcolor{yellow}{Fragment Offset} Gibt an, wo ein Fragment hingehört
        \item \textcolor{green}{Time to Live} anz. Sek, Hop-Counter, 0 → Paket wird verworfen
        \item \textcolor{green}{Protocol} Übergeordnetes Protokoll
        \item \textcolor{purple}{Header Checksum} verhindert fehlgeleitete Pakete (nicht Nutzdaten)
        \item \textcolor{purple}{Source Address} Wer das Paket ursprünglich abgesendet hat
        \item \textcolor{purple}{Destination Address} Wer das Paket schliesslich erhalten soll
        \item \textcolor{purple}{Options/Padding} variabel, füllt auf ein Vielfaches von 32Bits auf
    \end{itemize}
        \includegraphics[width=1\linewidth]{images/internet_protokoll_format_ip.png}\\
    Das unterliegende Netz limitiert die Grösse eines Pakets (Maximum Transfer Unit). Der Sender kennt die MTU der Netze nicht.\\
\end{KR}

\begin{definition}{Fragmentierung}
    \begin{itemize}
        \item Länge der Nutzdaten = Vielfaches von 8 Bytes
        \item Die Pakete haben die gleiche und grösstmögliche Länge
    \end{itemize}
\end{definition}

\begin{formula}{Reassembly}
    nutze Flags (0/DF/MF) und Fragment Offset
    \begin{itemize}
        \item Zusammensetzen beim Zielhost
        \item Letztes Fragment: MF = 0
    \end{itemize}
        \includegraphics[width=0.75\linewidth]{images/reassembly.png}\\
        Kombination mit DF und MF erlaubt vollständige Rekonstruktion ohne explizite Übertragung der ursprünglichen Paketgrösse
\end{formula}

\begin{concept}{IP-Fragmentierung in heutigen Systemen}\\
    Übertragung durch unterliegendes Netz limitiert (maximale Payload)
    \begin{itemize}
        \item Im IP-Kontext als Maximum Transfer Unit (MTU) bezeichnet
        \item Unterschiedlich für verschiedene Technologien
    \end{itemize}
    Fragmentierung in Routern wird vermieden
    \begin{itemize}
        \item Fragmentierung findet im Sender statt
        \item Entlastet Router von dieser Aufgabe
        \item Hierzu muss der Sender die kleinste MTU auf dem gesamten Pfad kennen
        \begin{itemize}
            \item Funktionalität siehe ICMP
        \end{itemize}
        \item Jedes Fragment ist ein vollständiges IP-Paket inklusive Header und wird an den Empfänger weitergeleitet
    \end{itemize}
    Das Reassemblieren findet erst im Ziel-Host statt
    \begin{itemize}
        \item Pakete nehmen eventuell unterschiedliche Pfade
        \item Pakete müssten sonst eventuell später wieder fragmentiert werden
    \end{itemize}
    In IPv6 ist keine Fragmentierung "unterwegs" vorgesehen
\end{concept}

\subsubsection{Kapselung und Adressauflösung}

\begin{definition}{Kapselung eines IP-Pakets im Ehternet Frame}\\
    Meist wird heute Ethernet-Encapsulation verwendet
    \begin{itemize}
        \item Das IP-Paket wird direkt im Nutzdatenteil des Frames übertragen
        \item Das Type Feld des Ethernet Frames erhält den Wert 0800 (hex)
        \item Die MTU ist damit 1500 Bytes
    \end{itemize}
        \includegraphics[width=1\linewidth]{images/kapselung_ip_paket.png}
\end{definition}

\begin{example2}{Übertragung eines IP-Pakets mit Encapsulation}\\
    \includegraphics[width=0.75\linewidth]{images/übertragung_eines_ip_pakets_bsp.png}\\
    Was geschieht bei der Übertragung genau?
    \begin{itemize}
        \item Knoten a sendet ein IP-Paket an Knoten c
        \begin{itemize}
            \item das Paket enthält die IP-Adressen von a und c
        \end{itemize}
        \item Knoten a konsultiert die Routing Tabelle und sieht:
        \begin{itemize}
            \item dass c über den Router AB erreicht werden kann, und
            \item Kennt nun die IP-Adresse von Router AB
        \end{itemize}
        \item Knoten a generiert ein Ethernet Frame, welches an die Hardware-adresse S von Router AB gesendet wird
        \begin{itemize}
            \item a muss aus der IP-Adresse von Router AB die Hardware-Adresse S herausfinden
            \item \textbf{Adressauflösung}
        \end{itemize}
        \item Router AB empfängt das Ethernet Frame, packt das IP-Paket aus und modifiziert den Header (TTL)
        \item Router AB konsultiert die Routing Tabelle und sieht:
        \begin{itemize}
            \item dass c über den Router BC erreicht werden kann, und
            \item Kennt nun die IP-Adresse von Router BC
        \end{itemize}
    \end{itemize}
    \textcolor{pink}{Die IP-Adressen a und c bleiben während der gesamten Übertragung unverändert}\\
    \includegraphics[width=0.5\linewidth]{images/encapsulation_bsp.png}
\end{example2}

\begin{KR}{Kapselung und Adressauflösung}\\
    ARP (Address Resolution Protocol)
    \begin{itemize}
        \item Ermittelt HW-Adresse (MAC) zu einer IP-Adresse
    \end{itemize}
        \includegraphics[width=0.75\linewidth]{images/arp.png}\\
    Internet Control Message Protocol (ICMP)
    \begin{itemize}
        \item Übertragungen von Fehlermeldungen oder Informationsaustausch
    \end{itemize}
\end{KR}

\begin{concept}{ARP}
    Grundprinzip von ARP
    \begin{itemize}
        \item Für das Senden von Daten an einen durch seine IP-Adresse identifizierten Knoten im lokalen Netz wird dessen Hardwareadresse benötigt.
        \item Ist diese nicht bekannt, werden alle Knoten im Netz per Broadcast angefragt.
        \item Der Knoten mit der angefragten IP-Adresse kennt seine eigene Hardwareadresse und sendet sie an den fragenden Knoten zurück
        \item Die ARP-Tabelle speichert bekannte <IP-MAC> Kombinationen für eine gewisse Zeit
    \end{itemize}
        \includegraphics[width=1\linewidth]{images/arp_concept.png}
\end{concept}

\begin{formula}{ARP Nachrichtenstruktur}\\
    ARP-Request und ARP-Response sind je in genau einem Ethernet Frame enthalten mit Type 0806
    \begin{itemize}
        \item Beim Request ist die Destination Address FF-FF-FF-FF-FF-FF (Broadcast Frame) und die Hardware Address of Target ist 0
    \end{itemize}
        \includegraphics[width=1\linewidth]{images/arp_nachrichtenstruktur.png}
\end{formula}

\begin{KR}{ARP Implementierung und Verwendung}
    \begin{itemize}
        \item Ein ARP-Request/Response für jedes IP-Paket wäre sehr ineffizient
        \begin{itemize}
            \item Jeder Knoten führt eine Tabelle (ARP-Cache) mit bekannten HW-Adressen
        \end{itemize}
        \item Aufgelöste (bekannte) Mappings IP Adresse $\rightarrow$ Hardwareadresse werden im ARP-Cache für gespeichert
        \begin{itemize}
            \item Erneuerung nach Ablauf eines Timers, typisch: einige Minuten
        \end{itemize}
        \item Abfrage/Modifizieren des ARP-Cache mit arp (Windows):
        \begin{itemize}
            \item arp -a: Anzeigen aller Einträge
            \item arp -d ip\_addr: Löschen eines Eintrags
            \item arp –s ip\_addr hw\_addr: Setzen eines Eintrags
        \end{itemize}
        \item Neue / empfohlene Befehle für Linux:
        \begin{itemize}
            \item ip neigh \{ add | del | show\}
        \end{itemize}
    \end{itemize}
    Weitere Verwendung:
    \begin{itemize}
        \item Erkennung von Adresskonflikten
        \begin{itemize}
            \item Nach einer Adresszuweisung (manuell oder per DHCP) wird ein ARP-Request an die eigene IPAdresse gerichtet, um zu prüfen, ob kein anderer Host im LAN die Adresse verwendet
            \item Falls eine Antwort kommt, liegt ein Adresskonflikt vor
        \end{itemize}
        \item Erneuerung von Einträgen im ARP-cache
        \begin{itemize}
            \item Linux Systeme senden in diesem Fall einen ARP-Request als Unicast
            \item Reduziert Broadcast-Last im Netz
        \end{itemize}
    \end{itemize}
\end{KR}

\begin{concept}{Internet Control Message Protocol (ICMP)}\\
    Übertragung von Fehlermeldungen oder Informationsaustausch auf Internet Layer, z.B.
    \begin{itemize}
        \item Time to live (TTL) hat den Wert 0 erreicht
        \item Ein Host möchte testen, ob ein anderer Host „up“ ist ICMP Meldungen werden in IP Paketen gekapselt
        \item Sieht aus wie ein Protokoll eines höheren Layers, welches den Internet Layer verwendet
        \item ICMP ist aber so eng mit IP verbunden, dass es zum Network Layer gezählt wird
    \end{itemize}
        \includegraphics[width=0.75\linewidth]{images/icmp.png}
\end{concept}

\begin{KR}{ICMP Format}\\
    Header:
    \begin{itemize}
        \item \textcolor{blue}{Type} ICMP Typ
        \item \textcolor{green}{Code} Message Details
        \item \textcolor{green}{Checksum} Prüfsumme über die ICMP Meldung
        \item \textcolor{pink}{depends on code} Unterschiedliche Werte und Verwendung je nach ICMP Typ
    \end{itemize}
    \textcolor{purple}{Datenbereich} IP-Header und 64 Bits of Original Datagram
    \includegraphics[width=1\linewidth]{images/icmp_details.png}
\end{KR}

\begin{definition}{ICMP Meldungstypen}
    \begin{itemize}
        \item ICMP benutzt direkt IP - keine Garantie, dass die Meldungen je ankommen
        \item Meldungen sind informativ gedacht; Ziel ist nicht eine zuverlässige Übertragung von IP Paketen
        \item Für eine fehlerfreie Übertragung inklusive Flusssteuerung sind höhere Layer verantwortlich!
    \end{itemize}
        \includegraphics[width=0.75\linewidth]{images/icmp_medlungstypen.png}
    \begin{itemize}
        \item Destination Unreachable (Fehler)
        \begin{itemize}
            \item IP-Paket kann nicht zum Ziel gebracht werden
            \item Beispiel: Keine Route zum Ziel-Host vorhanden
        \end{itemize}
        \item Redirect (Optimierung)
        \begin{itemize}
            \item Ein Host H sendet ein IP-Paket an einen ersten Router R1
            \item R1 stellt fest, dass der nächste Router auf dem Weg zum Ziel R2 ist; R2 ist aber im gleichen Netz wie H und R1 (möglicherweise unvollständige Routingtabelle im Host H)
            \item R1 sendet an H eine Redirect-Meldung, damit H Pakete fortan direkt an R2 sendet
        \end{itemize}
        \item Time Exceeded (Fehler)
        \begin{itemize}
            \item Router ändert das TTL-Feld im IP-Header von 1 auf 0
            \item Host hat nicht alle Fragmente erhalten, bevor der Timer abläuft
        \end{itemize}
        \item Parameter Problem: Bad IP Header (Fehler)
        \begin{itemize}
            \item IP Packet Header enthält ungültigen Wert, der nicht verarbeitet werden kann (z.B. nicht existierende IP-Option)
        \end{itemize}
        \item Echo Request/Reply (Information)
        \begin{itemize}
            \item Host sendet Echo-Request, der adressierte Host antwortet mit Echo-Reply; Reply enthält die gleichen Daten wie Request
        \end{itemize}
        \item Timestamp Request/Reply (Information)
        \begin{itemize}
            \item Wie Echo, aber zusätzlich wird die aktuelle Zeit der Hosts ausgetauscht (32-Bit Wert, Millisekunden seit Mitternacht GMT)
        \end{itemize}
    \end{itemize}
\end{definition}

\begin{definition}{Echo Request/Reply Messages}
    Test, ob Host erreichbar ist
    \begin{itemize}
        \item Host antwortet auf Echo Request (Type 8) mit Echo Reply (Type 0), mit gleichem Inhalt wie der Echo Request
    \end{itemize}
    Format
    \begin{itemize}
        \item Identifier: Erlaubt Zuordnung von Reply zu Echo-Request
        \item Sequence Number: Wird innerhalb eines Identifiers jeweils um 1 erhöht
        \item Data: Beliebige Daten, werden vom Empfänger gespiegelt
    \end{itemize}
        \includegraphics[width=0.75\linewidth]{images/icmp_echorequest.png}\\
    \textbf{ping} verwendet Echo und Echo Reply, um die Erreichbarkeit eines Routers/Hosts zu prüfen; ebenfalls wird die Round-Trip Zeit gemessen\\
        \includegraphics[width=0.5\linewidth]{images/ping.png}
\end{definition}

\begin{definition}{ICMP Destination Unreachable}\\
    Vom Router/Zielhost an Absender gesendet, wenn Paket nicht weitergeleitet werden kann\\
        \includegraphics[width=1\linewidth]{images/destination_unreachable.png}\\
    \textbf{Path MTU Discovery:}\\
    Ziel
    \begin{itemize}
        \item Erkennung der kleinsten MTU auf Pfad zwischen Sender und Empfänger (Path-MTU, PMTU)
        \item RFC 1191 → Path MTU discovery
    \end{itemize}
    Zweck
    \begin{itemize}
        \item Vermeidung von Fragmentierung «unterwegs»
    \end{itemize}
\end{definition}

\begin{example}
        \includegraphics[width=0.75\linewidth]{images/dest_unreachable_ex1.png}\\
    Welche Codes werden von einem Router (0,1,4) und welche vom Zielhost (2,3) generiert?
    Welche vermutlich von einer Firewall (13)?
\end{example}

\begin{KR}{Path MTU discovery}\\
    Annahme, dass die PMTU gleich der lokalen MTU ist
    \begin{itemize}
        \item Senden von IP-Paketen mit Länge=PMTU und mit DF=1
        \item Empfang von «Destination Unreachable» mit Code 4 «fragmentation needed and DF set»
        \item PMTU reduzieren auf «Next-Hop MTU»
    \end{itemize}
    Die «Next-Hop MTU» erkennt man: Enthalten in Octet 5..8 ("must be zero" stimmt nur, wenn wirklich "unused")
\end{KR}

\begin{example2}{ICMP Destination Unreachable}\\
    Host 160.85.31.3 versucht, das folgende Paket an Host 160.85.29.99 zu senden (Farben siehe IP-Header def.):
    \begin{itemize}
        \item \textcolor{blue}{4500 0028} \textcolor{yellow}{8b10 0000} \textcolor{green}{0711} \textcolor{purple}{a8a4 \colorbox{lightgrey}{a055 1f03} \colorbox{lightgrey}{a055 1d63} 8b0d 829d 0014 a348 030a 0000 7504 1137 407c 0800}
        \item Erkennen Sie in diesem Paket die IP Adressen von Sender und Destination?
        \begin{itemize}
            \item \colorbox{lightgrey}{Sender}: a055 1f03, \colorbox{lightgrey}{Destination}: a055 1d63
        \end{itemize}
    \end{itemize}
    Ein Router kennt keinen Weg und sendet diese Destination Unreachable Message zurück (Farben siehe ICMP-Header def):
    \begin{itemize}
        \item 4500 0038 8038 0000 fd\textbf{01} 5bc0 \colorbox{lightgrey}{a055 821e a055 1f03} \textcolor{blue}{03}\textcolor{green}{01 4bf7} \textcolor{pink}{0000 0000} \textcolor{purple}{4500 0028 8b10 0000 0711 a8a4 a055 1f03 a055 1d63 8b0d 829d 0014 a348}
        \item Wie erkennen Sie, dass es sich um eine ICMP Message handelt? \textbf{Protocol}: 01
        \item Wie erkennen Sie den ICMP Typ? \textcolor{blue}{Type}: 03
        \item Erkennen Sie die "64 Bytes of Original Datagram"? \textcolor{purple}{Original Data}
    \end{itemize}
\end{example2}

\begin{definition}{ICMP Time Exceeded}
    \begin{itemize}
        \item \textcolor{blue}{Type = 11}
        \item \textcolor{pink}{unused (must be 0)}
    \end{itemize}
    Wird in diesen 2 Fällen gesendet:
    \begin{itemize}
        \item Router setzt TTL-Feld von 1 auf 0
        \begin{itemize}
            \item Paket wird verworfen und der Absender informiert (Code = 0)
        \end{itemize}
        \item Zielhost kann ein fragmentiertes Paket nicht innerhalb nützlicher Zeit reassemblieren
        \begin{itemize}
            \item Fragmente werden verworfen und der Absender informiert (Code = 1)
        \end{itemize}
    \end{itemize}
    \textbf{traceroute} erlaubt, den Weg zu einem beliebigen Host (oder einem fehlerhaft konfigurierten Router auf diesem Weg) zu finden
    \begin{itemize}
        \item UDP Datagramme an hohe Destination Portnummer (zufällig gewählt, default 33434)
        \item Erstes Datagramm: TTL := 1
        \begin{itemize}
            \item Erster Router setzt TTL auf 0, verwirft Paket und sendet Time Exceeded Message zurück
            \item Erste Router ist bekannt
        \end{itemize}
        \item Nächstes Datagramm: TTL := 2
        \begin{itemize}
            \item Zweiter Router ist bekannt etc...
        \end{itemize}
        \item …
        \item Zielhost kann Zielport nicht erreichen
        \begin{itemize}
            \item Destination Unreachable Message (Code = 1) an Absender
            \item Zielhost ist erreicht
        \end{itemize}
        \item Um die "Entfernung" zu den einzelnen Routern/Zielhosts zu bestimmen, wird zugleich noch die Round-Trip Zeit gemessen
    \end{itemize}
        \includegraphics[width=1\linewidth]{images/traceroute.png}
\end{definition}

\subsection{IPv6}

\begin{definition}{IPv6}
    \begin{itemize}
        \item IPv6 ist in RFC 2460 spezifiziert
        \item 128-bit Adressen; diese werden mit je zwei-Bytes in Hex-Darstellung notiert und durch Doppelpunkte getrennt
        \item IPv6 verwendet Extension Headers, um den Basic Header zu vereinfachen
        \item Ein Interface kann mehr als eine IPv6 Adresse haben
        \begin{itemize}
            \item Ein Interface hat in der Regel eine lokale und zwei globale IPv6 Adressen:
            \item Eine MAC-basierte und eine nicht von der Hardware abhängige.
        \end{itemize}
        \item IPv6 verwendet zur Abfrage der Layer-2 Adressen NDP statt ARP
        \item Domain Name Service (DNS)
        \begin{itemize}
            \item IPv4 stellt an den Resolver Anfragen nach A-Records
            \item IPv6 stellt an den Resolver Anfragen nach AAAA-Records
        \end{itemize}
        \item Link-Local Adressen sind nur im lokalen Layer-2 Segment gültig; sie werden nicht geroutet
    \end{itemize}
    hat sich nicht durchgesetzt weil:
    \begin{itemize}
        \item nicht so einfach lesbar wie IPv4
        \item Viele Probleme von IPv4 konnten gelöst werden
        \item IPv6 ist nicht rückwärtskompatibel, daher teuer zu implementieren, und benötigt viel zusätzliches Fachwissen von Netzwerkadministratoren
    \end{itemize}
\end{definition}

\begin{KR}{Key Takes}
    \begin{itemize}
        \item Der IP-Header besteht aus 20 Bytes (ohne Optionen)
        \item Um über Netze mit verschiedenen Maximum Transfer Units (MTU) arbeiten zu können, unterstützt IP Fragmentierung und Reassembly
        \begin{itemize}
            \item Heute wird in der Regel beim Sender fragmentiert und im Ziel-Host reassembliert
            \item Path MTU discovery mittels ICMP kann verwendet werden, um die kleinste MTU auf dem Weg zum Ziel-Host zu identifizieren
        \end{itemize}
        \item IP-Pakete werden in Ethernet Frames gekapselt und von jedem Router wieder ausgepackt und erneut gekapselt.
        \begin{itemize}
            \item Dazu muss der Router die Layer-2 Adresse (MAC-Adresse) des nächsten Routers/Hosts kennen (ARP-Cache) oder erfragen (ARP-Request)
        \end{itemize}
        \item ICMP wird verwendet, um Fehler innerhalb der Netzwerkschicht zu behandeln (keine Retransmissions)
        \begin{itemize}
            \item ICMP-Nachrichten werden in IP-Pakete gekapselt, werden aber dennoch der NetzwerkSchicht zugeordnet
        \end{itemize}
    \end{itemize}
\end{KR}









