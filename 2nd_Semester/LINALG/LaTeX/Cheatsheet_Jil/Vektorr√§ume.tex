\graphicspath{{images/}}

\section{Vektorräume}

\begin{definition}{Vektorraum}
    Menge $V$ mit zwei Verknüpfungen:
    \begin{itemize}
        \item Vektoraddition: $\overrightarrow{a} + \overrightarrow{b} \in V$
        \item Skalarmultiplikation: $\lambda \cdot \overrightarrow{a} \in V$
        \item Nullpunkt: $\overrightarrow{0} \in V$
    \end{itemize}
\end{definition}

\begin{definition}{Unterraum}
    Teilmenge $U$ eines Vektorraums $V$, die selbst ein Vektorraum ist.
    \begin{itemize}
        \item $\overrightarrow{0} \in U$
        \item $\forall \vec{a}, \vec{b} \in U$ gilt $\overrightarrow{a} + \overrightarrow{b} \in U$
        \item $\forall \lambda \in \R$ und $\forall \vec{a} \in U$ gilt $\lambda \cdot \overrightarrow{a} \in U$
    \end{itemize}
    
\end{definition}

\subsubsection*{Basis und Dimension}

\begin{definition}{Linearer Span}\\
    Menge aller Linearkombinationen der Vektoren $\overrightarrow{b_{1}}, \overrightarrow{b_{2}}, \ldots, \overrightarrow{b_{n}}$ in einem reellen Vektorraum $V$.

    $$
    \operatorname{span}\left(\overrightarrow{b_{1}}, \ldots, \overrightarrow{b_{n}}\right)=\left\{\lambda_{1} \cdot \overrightarrow{b_{1}}+ \ldots +\lambda_{n} \cdot \overrightarrow{b_{n}} \mid \lambda_{1}, \ldots, \lambda_{n} \in \mathbb{R}\right\}
    $$

    Schreibt man die Vektoren $\overrightarrow{b_{k}} \in \mathbb{R}^{m}$ nebeneinander so entsteht die $m \times n-$ Matrix $B$

    Folgende Aussagen sind dann äquivalent:

    \begin{enumerate}
    \item Die Vektoren $\overrightarrow{b_{1}}, \overrightarrow{b_{2}}, \ldots, \overrightarrow{b_{n}}$ sind linear unabhängig

    \item Das LGS $B \cdot \vec{x}=\overrightarrow{0}$ hat nur eine Lösung nämlich $\vec{x}=\overrightarrow{0}$

    \item Es gilt $\operatorname{rg}(B)=n$\\
    Eine Teilmenge $U$ eine Vektorraums $V$ heisst Unterraum von $V$, wenn $U$ selbst auch ein Vektorraum ist.

    \end{enumerate}
\end{definition}

\begin{concept}{Erzeugendensystem}
    Menge von Vektoren, die den gesamten Vektorraum aufspannen.\\
    Eine Menge $\left\{\overrightarrow{b_{1}}, \overrightarrow{b_{2}}, \ldots, \overrightarrow{b_{n}}\right\}$ von Vektoren $\overrightarrow{b_{k}}$ im Vektorraum $V$ heisst Erzeugendensystem von $V$, wenn gilt:

    $$
    V=\operatorname{span}\left(\overrightarrow{b_{1}}, \overrightarrow{b_{2}}, \ldots, \overrightarrow{b_{n}}\right)
    $$

    Schreibt man die Vektoren $\overrightarrow{b_{k}} \in \mathbb{R}^{m}$ nebeneinander so entsteht die $m \times n$ - Matrix $B$.

    Folgende Aussagen sind dann äquivalent:

    \begin{enumerate}
    \item Die Vektoren $\overrightarrow{b_{k}}$ bilden ein Erzeugendensystem $\mathbb{R}^{m}$

    \item Das LGS $B \cdot \vec{x}=\vec{a}$ ist für jedes $\vec{a} \in \mathbb{R}^{m}$ Iösbar

    \item Es gilt $r g(B)=m$

    \end{enumerate}
    
\end{concept}

\begin{definition}{Dimensionen}
    Für jeden reellen Vektorraum $V$ gilt: Jede Basis von $V$ hat gleich viele Elemente.

    Die Anzahl Vektoren, die eine Basis von $V$ bilden, heisst Dimension von $V=\operatorname{dim}(V)$.

    \begin{itemize}
    \item Eine Basis von $\mathbb{R}^{n}$ hat $n$ Elemente $\rightarrow \operatorname{dim}\left(\mathbb{R}^{n}\right)=n$
    \end{itemize}
\end{definition}

\begin{definition}{Basis}
    Eine Menge $B=\left\{\overrightarrow{b_{1}}, \overrightarrow{b_{2}}, \ldots, \overrightarrow{b_{n}}\right\}$ von Vektoren $\vec{b}_{k}$ im Vektorraum $V$ heisst Basis von $V$,
    wenn gilt:
    \begin{itemize}
        \item $B=\left\{\overrightarrow{b_{1}}, \overrightarrow{b_{2}}, \ldots, \overrightarrow{b_{n}}\right\}$ ist ein Erzeugendensystem von $V$
        \item Die Vektoren $\overrightarrow{b_{1}}, \overrightarrow{b_{2}}, \ldots, \overrightarrow{b_{n}}$ sind linear unabhängig
    \end{itemize}
\end{definition}

\begin{theorem}{Basis und Dimensionen}
    Folgende Aussagen sind äquivalent:
    \begin{itemize}
    \item Die Vektoren $\overrightarrow{b_{1}}, \overrightarrow{b_{2}}, \ldots, \overrightarrow{b_{n}}$ bilden eine Basis von $\mathbb{R}^{n}$
    \item $\operatorname{rg}(B)=n$
    \item $\operatorname{det}(B) \neq 0$
    \item $B$ ist invertierbar
    \item Das $\operatorname{LGS} B \cdot \vec{x}=\vec{c}$ hat eine eindeutige Lösung
    \end{itemize}
\end{theorem}

\begin{KR}{Basiswechsel}
    Beliebige Basis $B \rightarrow$ Standard-Basis $S$

    $$
    B = \left\{ \begin{pmatrix} x_{1} \\ \scalebox{0.5}{\vdots} \\ z_{1} \end{pmatrix}_{S} ; \begin{pmatrix} x_{2} \\ \scalebox{0.5}{\vdots} \\ z_{2} \end{pmatrix}_{S} \right\}, \quad \vec{a} = \begin{pmatrix} a_{1} \\ \scalebox{0.5}{\vdots} \\ a_{n} \end{pmatrix}_{B}
    $$
    $$ 
    \vec{a}=a_{1} \cdot \overrightarrow{b_{1}}+a_{2} \cdot \overrightarrow{b_{2}}+\cdots+a_{n} \cdot \overrightarrow{b_{n}}
    $$
\end{KR}

\begin{example}
    $$B=\left\{\binom{3}{1}_{S} ;\binom{-1}{0}_{S}\right\}, \vec{a}=\binom{2}{3}_{B} \Rightarrow \vec{a}=2 \cdot \binom{3}{1}+3 \cdot \binom{-1}{0} = \binom{3}{2}_S$$
\end{example}

\begin{KR}{Basiswechsel}
    Standard-Basis $S \rightarrow$ Beliebige Basis $B$
    $$
    B = \left\{ \begin{pmatrix} x_{1} \\ \scalebox{0.5}{\vdots} \\ z_{1} \end{pmatrix} ; \begin{pmatrix} x_{2} \\ \scalebox{0.5}{\vdots} \\ z_{2} \end{pmatrix} \right\}, \quad \vec{a} = \begin{pmatrix} a_{1} \\ \scalebox{0.5}{\vdots} \\ a_{n} \end{pmatrix}_{S} \Rightarrow B \cdot \begin{pmatrix} a_{1} \\ \scalebox{0.5}{\vdots} \\ a_{n} \end{pmatrix}_{B} = \begin{pmatrix} a_{1} \\ \scalebox{0.5}{\vdots} \\ a_{n} \end{pmatrix}_{S}
    $$

    $$
    \begin{pmatrix} a_{1} \\ \scalebox{0.5}{\vdots} \\ a_{n} \end{pmatrix}_{B} = B \cdot B^{-1} \cdot \begin{pmatrix} a_{1} \\ \scalebox{0.5}{\vdots} \\ a_{n} \end{pmatrix}_{B} = B^{-1} \cdot \begin{pmatrix} a_{1} \\ \scalebox{0.5}{\vdots} \\ a_{n} \end{pmatrix}_{S}
    $$
\end{KR}

\begin{example}
    $$
    B=\left\{\binom{1}{1}_{S} ;\binom{-1}{0}_{S}\right\}, \quad \vec{a}=\binom{-7}{-4}_{S} \Rightarrow \begin{pmatrix}
        1 & -1 \\
        1 & 0
        \end{pmatrix} \cdot\binom{a_{1}}{a_{2}}_{B}=\binom{-7}{-4}_{S}
    $$

    $$
    \begin{pmatrix} 1 & -1 \\ 1 & 0 \end{pmatrix} \cdot\begin{pmatrix} 1 & -1 \\ 1 & 0 \end{pmatrix}^{-1} \cdot\binom{a_{1}}{a_{2}}_{B}=\begin{pmatrix} 1 & -1 \\ 1 & 0 \end{pmatrix}^{-1} \cdot\binom{-7}{-4}_{S}
    $$

    $$
    \binom{a_{1}}{a_{2}}_{B} = \begin{pmatrix} 1 & -1 \\ 1 & 0 \end{pmatrix}^{-1} \cdot\binom{-7}{-4}_{S} 
    \Rightarrow \begin{pmatrix} 0 \cdot -7 + 1 \cdot -4 \\ -1 \cdot -7 + 1 \cdot -4 \end{pmatrix} = \binom{-4}{3}_{B}
    $$
\end{example}