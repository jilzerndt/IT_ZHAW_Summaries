\graphicspath{{images/}}
\section*{LGS und Matrizen}

\subsubsection*{Matrizen}

    \begin{definition}{{Matrix, Element, Zeilen, Spalten und Typ}}\\
        Eine \textit{Matrix} ist (simpel gesagt) ein Vektor mit mehreren Spalten 
        und wird mit Grossbuchstaben bezeichnet.
        Ein \textit{Element} $a_ij$ ist ein Wert aus dieser Matrix,
        auf den über die Zeile und Spalte zugegriffen wird (\textbf{Z}eile \textbf{z}uerst,
        \textbf{Sp}alte \textbf{Sp}äter).
        Der \textit einer Matrix ergibt sich aus der Anzahl ihren Zeilen und Spalten.
        Matrizen mit $m$-Zeilen und $n$-Spalten werden $m\times n$-Matrizen genannt.
    \end{definition}

    \begin{definition}{Matrix}
        Tabelle mit $m$ Zeilen und $n$ Spalten.
        \begin{itemize}
            \item $m \times n$-Matrix
            \item $a_{ij}$: Element in der $i$-ten Zeile und $j$-ten Spalte
        \end{itemize}
    \end{definition}

    \begin{definition}{Nullmatrix}
        Eine Matrix, deren Elemente alle gleich $0$ sind, heisst \textit{Nullmatrix} und wird mit $0$ bezeichnet.
    \end{definition}

    \begin{definition}{Spaltenmatrix}
        \begin{wrapfigure}{hr!}{0.2\textwidth}
            \vspace{-20pt}
            \begin{equation*}
                \vec{a}=\begin{pmatrix}
                    a_1\\a_2\\\vdots\\a_n
                \end{pmatrix}
            \end{equation*}
        \end{wrapfigure}
        Besteht eine Matrix nur aus einer Spalte, so heisst diese \textit{Spaltenmatrix}.
        Spaltenmatrix können als Vektoren aufgefasst werden und können mit einem kleinen Buchstaben 
        sowie einem Pfeil darüber notiert werden ($\vec{a}$). 
    \end{definition}
    
    \begin{minipage}{0.5\linewidth}
    \begin{formula}{Addition und Subtraktion}
        \begin{itemize}
            \item $A + B = C$
            \item $c_{ij} = a_{ij} + b_{ij}$
        \end{itemize}
    \end{formula}
    \end{minipage}
    \begin{minipage}{0.48\linewidth}
    \begin{formula}{Skalarmultiplikation}
        \begin{itemize}
            \item $k \cdot A = B$
            \item $b_{ij} = k \cdot a_{ij}$
        \end{itemize}
    \end{formula}
    \end{minipage}

    \begin{theorem}{Rechenregeln für die Addition und skalare Multiplikation von Matrizen}
        \begin{itemize}
            \item Kommutativ-Gesetz: $A+B=B+A$
            \item Assoziativ-Gesetz: $A+(B+C)=(A+B)+C$
            \item Distributiv-Gesetz:\\ 
                $\lambda\cdot(A+B)=\lambda\cdot A+\lambda\cdot B$
                sowie $(\lambda + \mu)\cdot A=\lambda\cdot A+\mu\cdot A$
        \end{itemize}    
    \end{theorem}
    
    \begin{formula}{Matrixmultiplikation} $A^{m \times n}$, $B^{k \times n}$\\
        \begin{minipage}{0.6\linewidth}
        Bedingung: $A$ hat $n$ Spalten und $B$ hat $n$ Zeilen.\\
        Resultat: $C$ hat $m$ Zeilen und $k$ Spalten.
        \begin{itemize}
            \item $A \cdot B = C$
            \item $c_{ij} = a_{i1} \cdot b_{1j} + a_{i2} \cdot b_{2j} + \ldots + a_{in} \cdot b_{nj}$
            \item $A \cdot B \neq B \cdot A$
        \end{itemize}  
        \end{minipage}
        \begin{minipage}{0.35\linewidth} 
        \begin{center}
        \includegraphics[width=0.8\linewidth]{matrixmultiplikation.png}
        \end{center}
        \end{minipage}
    \end{formula}


    
    
    \begin{theorem}{Rechenregeln für die Multiplikation von Matrizen}
        \begin{itemize}
            \item Assoziativ-Gesetz: $A\cdot(B\cdot C)=(A\cdot B)\cdot C$
            \item Distributiv-Gesetz: \\
                $A\cdot(B+C)=A\cdot B+A\cdot C$ und $(A+B)\cdot C=A\cdot C+B\cdot C$
            \item Skalar-Koeffizient: $(\lambda\cdot A)\cdot B=\lambda\cdot (A\cdot B)=A\cdot(\lambda\cdot B)$
        \end{itemize}
    \end{theorem}

    \begin{definition}{Transponierte Matrix}
        \begin{itemize}
            \item $A^T$: Spalten und Zeilen vertauscht
            \item $(A^T)_{ij} = A_{ji}$
        \end{itemize}
    \end{definition}
    
    \begin{definition}{Transponieren}
        \begin{wrapfigure}{r}{0.4\textwidth}
            \vspace{-13pt}
            \includegraphics[width=0.9\linewidth]{mat-transpos.png}
        \end{wrapfigure}
        Die \textit{Transponierte} einer $m\times n$-Matrix ist eine $n\times m$-Matrix. 
        Diese wird erhalten, indem die Zeilen zu Spalten und die Spalten zu Zeilen gemacht werden.
    \end{definition}

    \begin{theorem}{Transposition Regeln}
        \begin{equation*}
            {(A\cdot B)}^T = B^T\cdot A^T
        \end{equation*}
    \end{theorem}




\subsubsection*{Lineare Gleichungssysteme (LGS)}

\paragraph{Lineare Unabhängigkeit}
    \begin{definition}{Lineare Unabhängigkeit}
        Wir betrachten Vektoren $\vec{a_1},\vec{a_2},\ldots,\vec{a_k}$ mit $n$ Komponenten.
        Diese Vektoren heissen \textit{linear unabhängig} wenn $\sum_{i=1}^k 0\cdot a_i$ die
        einzige Linearkombination deren ist, die $\vec{0}$ ergibt. 
        Anderenfalls heissen sie \textit{linear abhängig}.
    \end{definition}

    \begin{theorem}{Koeffizientenmatrix{,} Determinante{,} Lösbarkeit des LGS }
        Für eine quadratische $n\times n$-Matrix $A$ sind die folgenden Aussagen äquivalent:
        \begin{enumerate}
            \item $\det(A)\neq 0$
            \item $rg(A)=n$
            \item $A$ ist invertierbar
            \item Das LGS $A\cdot\vec{c}=\vec{c}$ hat eine eindeutige Lösung.
            \item Die Spalten von $A$ sind linear unabhängig.
            \item Die Zeilen von $A$ sind linear unabhängig.
        \end{enumerate}
    \end{theorem}

\begin{theorem}{Rang einer Matrix}\\
    Anzahl linear unabhängiger Zeilen- oder Spaltenvektoren.\\
    Rang $rg(A)$ einer Matrix $A^{m \times n}$:
    $$rg(A) = \text{Anzahl Zeilen - Anzahl Nullzeilen}$$
    \begin{itemize}
        \item Lösbar: $rg(A) = rg(A|b)$
        \item Nicht lösbar: $rg(A) \neq rg(A|b)$
        \item genau eine Lösung: $rg(A) = n$
        \item unendlich viele Lösungen: $rg(A) < n$
    \end{itemize}
\end{theorem}

\paragraph{Gauss-Verfahren}

\begin{concept}{Zeilenstufenform (Gauss)}
    \begin{itemize}
        \item Alle Nullen stehen unterhalb der Diagonalen, Nullzeilen zuunterst
        \item Die erste Zahl $\neq 0$ in jeder Zeile ist eine führende Eins
        \item Führende Einsen, die weiter unten stehen $\rightarrow$ stehen weiter rechts
    \end{itemize}
    \textbf{Reduzierte Zeilenstufenform: (Gauss-Jordan)}\\
    Alle Zahlen links und rechts der führenden Einsen sind Nullen.
\end{concept}



\begin{KR}{Parameterdarstellung} bei unendlich vielen Lösungen
    \begin{itemize}
        \item Führende Unbekannte: Spalte mit führender Eins
        \item Freie Unbekannte: Spalten ohne führende Eins
    \end{itemize}
    \includegraphics[width=0.3\linewidth]{parameterdarstellung_lgs.png}\\
    Auflösung nach der führenden Unbekannten:
    \begin{itemize}
        \item $1 x_1 - 2 x_2 + 0 x_3 + 3 x_4 = 5 \quad x_2 = \lambda \rightarrow x_1 = 5 + 2 \cdot \lambda - 3 \cdot \mu$
        \item $0 x_1 + 0 x_2 + 1 x_3 + 1 x_4 = 3 \quad x_4 = \mu \rightarrow x_3 = 3 - \mu$    
    \end{itemize}
    \vspace*{2mm}
    $$ \vec{x} = \begin{pmatrix} x_1 \\ x_2 \\ x_3 \\ x_4 \end{pmatrix} 
    = \begin{pmatrix} 5 + 2 \lambda - 3 \mu \\ \lambda \\ 3 - \mu \\ \mu \end{pmatrix} 
    = \begin{pmatrix} 5 \\ 0 \\ 3 \\ 0 \end{pmatrix} + \lambda \begin{pmatrix} 2 \\ 1 \\ 0 \\ 0 \end{pmatrix} + \mu \begin{pmatrix} -3 \\ 0 \\ -1 \\ 1 \end{pmatrix}$$
\end{KR}
    
    \begin{formula}{Gauss-Verfahren}\\
        Für das Gauss-Verfahren wird Schritt 1.-4 des Gauss-Jordan-Verfahren angewendet.
        Das resultierende LGS wird durch Rückwärtssubstitution gelöst.
    \end{formula}
    
    \begin{formula}{Gauss-Jordan-Verfahren}
        \begin{enumerate}
            \item Wir bestimmen die am weitesten links stehende Spalte mit Elementen $\neq 0$.
                Wir nennen diese Spalte die \textit{Pivot-Spalte}.
            \item Ist die oberste Zahl in der Pivot-Spalte $= 0$, 
                dann vertauschen wir die erste Zeile mit der obersten, die in der Pivot-Spalte ein Element $\neq 0$ hat.
            \item Die oberste Zahl in der Pivot-Spalte ist nun eine Zahl $a\neq 0$.
                Wir dividieren die erste Zeile durch $a$. So erhalten wir die führende Eins.
            \item Nun wollen wir unterhalb der führenden Eins lauter Nullen erzeugen. 
                Dazu addieren wir passende Vielfache der ersten Zeile zu den übrigen Zeilen. 
        \end{enumerate}
        Wir lassen nun die erste Zeile aussen vor 
        und wenden die ersten vier Schritte auf den verleibenden Teil der Matrix an.
        Dieses Verfahren wiederholen wir so oft, bis die erweiterte Koeffizientenmatrix Zeilenstufenform hat.
        \begin{enumerate}
            \setcounter{enumi}{4}
            \item Nun arbeiten wir von unten nach oben und addieren jeweils geeignete Vielfache jeder
                Zeile zu den darüber liegenden Zeilen, um über den führenden Einsen Nullen zu erzeugen.
        \end{enumerate}
    \end{formula}
    
    \begin{theorem}{Lösbarkeit von linearen Gleichungssystemen}
        \begin{enumerate}
            \item Ein LGS $A\cdot\vec{x}=\vec{c}$ ist genau dann lösbar, wenn $rg(A)=rg(A\mid\vec{c})$
            \item Es hat genau eine Lösung, falls \textbf{zusätzlich} zu 1. gilt: $rg(A)=n$
            \item Es hat unendlich viele Lösungen, falls \textbf{zusätzlich} gilt: $(rg(A)<n)$
        \end{enumerate}
        \begin{highlight}{i}
            Bei einem homogenen LGS ist nach Definition $\vec{c}=\vec{0}$; deswegen gilt immer: $rg(A)=rg(A\mid\vec{c})$.
            Daher gibt es bei homogenen LGS nur zwei Möglichkeiten:
            \begin{itemize}
                \item Das LGS hat \textbf{eine Lösung} $x_1=x_2=\cdots=x_n=0$, die sog. \textit{triviale Lösung}.
                \item Das LGS hat \textbf{unendlich viele Lösungen}.
            \end{itemize}
        \end{highlight}
    \end{theorem}

    \begin{definition}{Homogenes LGS}\\
        Ein LGS heisst \textit{homogen}, wenn die rechte Seite $=\vec{0}$ ist: $A\cdot\vec{x}=\vec{0}$.
    \end{definition}
    \begin{lemma}{Zusammenhänge invertierbarkeit und Homogenes LGS}\\
        Ist $A$ invertierbar, so hat das homogene LGS $A\cdot\vec{x}=\vec{0}$ die eindeutige Lösung $x=A^{-1}\cdot 0=0$
    \end{lemma}    

\subsubsection*{Quadratische Matrizen}

\begin{formula}{Matrizen umformen}
    bestimme die Matrix $X$:
    $$A \cdot X + B = 2 \cdot X$$
    \begin{itemize}
        \item $A \cdot X=2 \cdot X-B$
        \item $A \cdot X-2 \cdot X=-B$
        \item $(A-2 E) \cdot X=-B$
        \item $(A-2 E) \cdot(A-2 E)^{-1} \cdot X=(A-2 E)^{-1} \cdot-B$
        \item $X=(A-2 E)^{-1} \cdot-B$
    \end{itemize}
\end{formula}

\paragraph{Inverse}
    \begin{definition}{Inverse}
        Die Inverse einer quadratischen Matrix $A$ ist eine Matrix $A^{-1}$, für die gilt:
        \begin{equation*}
            A\cdot A^{-1}=A^{-1}\cdot A=E
        \end{equation*}.
        Eine Matrix heisst \textit{invertierbar / regulär}, wenn sie eine Inverse hat. 
        Andernfalls heisst sie \textit{singulär}.
    \end{definition}
    \begin{definition}{Inverse einer quadratischen Matrix $A$}
        \begin{itemize}
            \item $A \cdot A^{-1} = A^{-1} \cdot A = E$
            \item $A^{-1}$ existiert, wenn $rg(A) = n$
        \end{itemize}
    \end{definition}
    \begin{theorem}{Eigenschaften invertierbarer Matrizen}
        \begin{enumerate}
            \item Die Inverse einer invertierbaren Matrix ist eindeutig bestimmt.
            \item Die Inverse einer invertierbaren Matrix $A$ ist invertierbar und es gilt: $(A^{-1})^{-1}=A$.
            \item Multiplizieren wir zwei invertierbare Matrizen $A$ und $B$ miteinander, 
                so ist das Produkt auch invertierbar und es gilt: ${(A\cdot B)}^{-1}=B^{-1}\cdot A^{-1}$.\\
                Die Reihenfolge ist relevant!
            \item Die Transponierte $A^T$ einer quadratischen Matrix $A$ ist genau dann invertierbar, 
                wenn $A$ invertierbar ist. In diesem Fall gilt: ${(A^T)^{-1}}={(A^{-1})}^T$.
        \end{enumerate}
    \end{theorem}
    \begin{lemma}{Zusammenhänge invertierbarkeit}\\
        Gegeben eines LGS $A\cdot\vec{x}=\vec{c}$ mit $n\times n$-Koeffizientenmatrix $A$.
        Dann sind folgende Aussagen äquivalent ($\Leftrightarrow$):
        \begin{enumerate}
            \item $A$ ist invertierbar.
            \item $A\cdot\vec{x}=\vec{c}$ hat genau eine Lösung.
            \item $rg(A)=n$
        \end{enumerate}
    \end{lemma}








\begin{theorem}{Inverse einer $2 \times 2$-Matrix}\\
    \begin{minipage}{0.45\linewidth}
    $$A = \begin{pmatrix} a & b \\ c & d \end{pmatrix}$$
    mit $det(A) = ad - bc$
    \end{minipage}
    \begin{minipage}{0.5\linewidth}
        $$A^{-1} = \frac{1}{\det(A)} \cdot \begin{pmatrix} d & -b \\ -c & a \end{pmatrix}$$
        NUR Invertierbar falls $ad - bc \neq 0$
    \end{minipage}
    
\end{theorem}

\begin{KR}{Inverse berechnen} einer quadratischen Matrix $A^{n \times n}$
    $$A \cdot A^{-1} = E \rightarrow \left( A | E \right) \leadsto \text{Zeilenoperationen} \leadsto \left( E | A^{-1}\right)$$
\end{KR}

\begin{example}
    $$
\underbrace{\left(\begin{array}{ccc}
4 & -1 & 0 \\
0 & 2 & 1 \\
3 & -5 & -2
\end{array}\right)}_{A} \cdot \underbrace{\left(\begin{array}{lll}
x_{1} & y_{1} & z_{1} \\
x_{2} & y_{2} & z_{2} \\
x_{3} & y_{3} & z_{3}
\end{array}\right)}_{A^{-1}}=\underbrace{\left(\begin{array}{lll}
1 & 0 & 0 \\
0 & 1 & 0 \\
0 & 0 & 1
\end{array}\right)}_{E}$$
$$ \rightarrow\left(\begin{array}{ccc|ccc}
4 & -1 & 0 & 1 & 0 & 0 \\
0 & 2 & 1 & 0 & 1 & 0 \\
3 & -5 & -2 & 0 & 0 & 1
\end{array}\right)
$$

Zeilenstufenform (linke Seite)

$$ \leadsto 
\left(\begin{array}{ccc|ccc}
1 & -1 / 4 & 0 & 1 / 4 & 0 & 0 \\
0 & 1 & 1 / 2 & 0 & 1 / 2 & 0 \\
0 & 0 & 1 & -6 & 17 & 8
\end{array}\right)
$$

Reduzierte Zeilenstufenform (linke Seite)

$$ \leadsto 
\left(\begin{array}{ccc|ccc}
1 & 0 & 0 & 1 & -2 & -1 \\
0 & 1 & 0 & 3 & -8 & -4 \\
0 & 0 & 1 & -6 & 17 & 8
\end{array}\right) \Rightarrow A^{-1}=\left(\begin{array}{ccc}
1 & -2 & -1 \\
3 & -8 & -4 \\
-6 & 17 & 8
\end{array}\right)
$$
\end{example}



\begin{concept}{LGS mit Inverse lösen}
    $A \cdot \vec{x} = \vec{b}$
    $$A^{-1} \cdot A \cdot \vec{x} = A^{-1} \cdot \vec{b} \rightarrow \vec{x} = A^{-1} \cdot \vec{b}$$
    \textbf{Beispiel:}\\
    \includegraphics[width=0.5\linewidth]{lgs_inverse.png}
\end{concept}



