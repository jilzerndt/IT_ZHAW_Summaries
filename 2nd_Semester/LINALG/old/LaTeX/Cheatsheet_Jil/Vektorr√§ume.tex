\graphicspath{{images/}}

\section{Vektorräume}




    \begin{definition}{Reeller Vektorraum}
        ist eine Menge $V\neq\emptyset$ mit zwei Verknüpfungen:
        \begin{center}
        Addition: $+ : V\times V\rightarrow V : (\vec{a};\vec{b})\mapsto\vec{a}+\vec{b}$ \\
        Skalarmultiplikation: $\cdot : \R\times V\rightarrow V : (\lambda;\vec{a})\mapsto\lambda\cdot\vec{a}$
        \end{center}
        Eigenschaften: {\small ($V$ = die Menge aller Vektoren, $\vec{a},\vec{b},\vec{c}\in V$, $\lambda,\mu\in \R$)}
        \begin{itemize}
            \item Addition: $\forall \vec{a},\vec{b}\in V\,:(\vec{a}+\vec{b})\in V$
            \item Skalarmultiplikation: $\forall \lambda\in \R, \vec{a}\in V\,:(\lambda\cdot\vec{a})\in V$
            \item Neutralelement $\vec{0}\in V$: $\forall \vec{a}\in V\,:\vec{a}+\vec{0}=\vec{a}$
            \item Inverses Element $-\vec{a}\in V$: $\forall \vec{a}\in V$ $\exists-\vec{a}$ so dass $\vec{a}+(-\vec{a})=0$
            \item $1\cdot\vec{a}=\vec{a}$
            \item Kommutativgesetz: $\vec{a}+\vec{b}=\vec{b}+\vec{a}$
            \item Assoziativgesetz:
                $\vec{a}+(\vec{b}+\vec{c})=(\vec{a}+\vec{b})+\vec{c}$ und
                $\lambda\cdot(\mu\cdot\vec{a})=(\lambda\cdot\mu)\cdot\vec{a}$
            \item Distributivgesetz: 
                $\lambda\cdot(\vec{a}+\vec{b})=\lambda\cdot\vec{a}+\lambda\cdot\vec{b}$ und
                $(\lambda+\mu)\cdot\vec{a}=\lambda\cdot\vec{a}+\mu\cdot\vec{a}$
        \end{itemize}
    \end{definition}

    \begin{definition}{Unterraum} $U\subseteq V$\\
        Teilmenge $U$ eines Vektorraums $V$, die selbst ein Vektorraum ist.
    \end{definition}

    \begin{definition}{Nullvektorraum}
        Die Teilmenge $U=\{\vec{0}\}\subseteq V$, die nur den Nullvektor aus einem Vektorraum V enthält, 
        heisst der \textit{Nullvektorraum} und ist immer ein Unterraum von $V$.
    \end{definition}

    \begin{KR}{Unterraumkriterien} {\small für Beweis dass $U$ ein Unterraum von $V$ ist}

        \vspace{1mm}

        Eine Teilmenge $U\neq\emptyset$ eines Vektorraums $V$ ist genau dann ein Unterraum von $V$, wenn gilt:

        \begin{minipage}{0.5\linewidth}
        \begin{itemize}
            \item $\overrightarrow{0} \in U$
            \item $\overrightarrow{a} + \overrightarrow{b} \in U \quad \forall \vec{a}, \vec{b} \in U$
            \item $\lambda \cdot \overrightarrow{a} \in U \quad \forall \lambda \in \R$ und $\forall \vec{a} \in U$
        \end{itemize}
        \end{minipage}
        \hspace{2mm}
        \begin{minipage}{0.4\linewidth}
        {\small \raggedleft Um zu zeigen dass eine Menge $V$ ein Vektorraum ist, beweisen wir genau auch diese Kriterien!!}
        \end{minipage}
    \end{KR}

    \begin{example2}{Unterraum Beweis}
        $U = \{f(x) = a \cdot x^2 + b \cdot x + c \mid a, b, c \in \R\}$

        \vspace{1mm}

        \begin{itemize}
            \item $\overrightarrow{0} = 0 \cdot x^2 + 0 \cdot x + 0 \in U$
            \item $\overrightarrow{a} = a_1 \cdot x^2 + b_1 \cdot x + c_1$ und $\overrightarrow{b} = a_2 \cdot x^2 + b_2 \cdot x + c_2$\\
                $\Rightarrow \overrightarrow{a} + \overrightarrow{b} = (a_1 + a_2) \cdot x^2 + (b_1 + b_2) \cdot x + (c_1 + c_2) \in U$
            \item $\lambda \cdot \overrightarrow{a} = \lambda \cdot (a_1 \cdot x^2 + b_1 \cdot x + c_1) = (\lambda \cdot a_1) \cdot x^2 + (\lambda \cdot b_1) \cdot x + (\lambda \cdot c_1) \in U$
        \end{itemize}
        \textcolor{pink}{$\Rightarrow U \subseteq \R$ ist ein Unterraum von $\R$}
    \end{example2}







\subsubsection*{Basis und Dimension}

\begin{definition}{Linearer Span} $\operatorname{span}(\vec{b_{1}}, ..., \vec{b_{n}})=\{\lambda_{1} \cdot \vec{b_{1}}+ ... +\lambda_{n} \cdot \vec{b_{n}} \mid \lambda_{1}, ..., \lambda_{n} \in \mathbb{R}\}$
    
    \vspace{1mm}
    
    {\small Menge aller Linearkombinationen von $\vec{b_{1}}, ..., \vec{b_{n}}$ in Vektorraum $V$}

    \vspace{1mm}

    {\small $\overrightarrow{b_{k}} \in \mathbb{R}^{m}$ nebeneinander schreiben $\Rightarrow$ Matrix $B^{m \times n}$ }

    \vspace{1mm}

    \begin{itemize}
        \item Die Vektoren $\overrightarrow{b_{1}}, \overrightarrow{b_{2}}, \ldots, \overrightarrow{b_{n}}$ sind linear unabhängig
        \item Das LGS $B \cdot \vec{x}=\overrightarrow{0}$ hat nur eine Lösung nämlich $\vec{x}=\overrightarrow{0}$
        \item Es gilt $\operatorname{rg}(B)=n$
    \end{itemize}
\end{definition}

\begin{concept}{Erzeugendensystem} $V=\operatorname{span}(\overrightarrow{b_{1}}, \overrightarrow{b_{2}}, \ldots, \overrightarrow{b_{n}})$
    
    \vspace{1mm}
    
    {\small Menge von Vektoren, die den gesamten Vektorraum $V$ aufspannen}

    \vspace{1mm}

    {\small $\overrightarrow{b_{k}} \in \mathbb{R}^{m}$ nebeneinander schreiben $\Rightarrow$ Matrix $B^{m \times n}$ }

    \vspace{1mm}

    \begin{itemize}
        \item Die Vektoren $\overrightarrow{b_{k}}$ bilden ein Erzeugendensystem $\mathbb{R}^{m}$
        \item Das LGS $B \cdot \vec{x}=\vec{a}$ ist für jedes $\vec{a} \in \mathbb{R}^{m}$ Iösbar
        \item Es gilt $r g(B)=m$
    \end{itemize}
\end{concept}

\begin{definition}{Dimension} $\operatorname{dim}(V)$
    {\small $\quad \quad$ Anzahl Vektoren, die eine Basis von $V$ bilden}

    Eine Basis von $\mathbb{R}^{n}$ hat $n$ Elemente $\rightarrow \operatorname{dim}\left(\mathbb{R}^{n}\right)=n$
\end{definition}

\begin{definition}{Basis}
    $B=\{\vec{b_{1}}, ..., \vec{b_{n}}\}$ von $\vec{b}_{k} \in V$ heisst Basis von $V$ wenn gilt:
    \begin{itemize}
        \item $B$ ist ein Erzeugendensystem von $V$
        \item Die Vektoren $\overrightarrow{b_{1}}, \overrightarrow{b_{2}}, \ldots, \overrightarrow{b_{n}}$ sind linear unabhängig
        \item $\vec{b}_k$ nebeneinander schreiben $\Rightarrow$ Matrix $B^{n \times n}$
    \end{itemize}

    Für $B^{n \times n}$ gilt:
    \begin{itemize}
        \item $\operatorname{rg}(B)=n$
        \item $\operatorname{det}(B) \neq 0$
        \item $B$ ist invertierbar
        \item Das LGS $B \cdot \vec{x}=\vec{c}$ hat eine eindeutige Lösung
        \end{itemize}
\end{definition}

\begin{KR}{Ist B eine Basis von V?} 
    Gegeben: $\overrightarrow{b_{1}}, \overrightarrow{b_{2}}, \ldots, \overrightarrow{b_{n}}$
    \begin{enumerate}
        \item Stelle die Vektoren als Spalten einer Matrix $B$ dar
        \item Gauss Algorithmus: Stelle $B$ in Zeilenstufenform
        \item Berechne den Rang $\Rightarrow r g(B) = \operatorname{dim}(\operatorname{span}(B))$
    \end{enumerate}

    Nun gilt:
    \begin{itemize}
        \item $r g(B)=n \Rightarrow \{\vec{b_{1}}, \vec{b_{2}}, \ldots, \vec{b_{n}}\}$ sind linear unabhängig
        \item $r g(B)<n \Rightarrow \{\vec{b_{1}}, \vec{b_{2}}, \ldots, \vec{b_{n}}\}$ sind linear abhängig
        \item $r g(B)=\operatorname{dim}(V) \Rightarrow \{\vec{b_{1}}, \vec{b_{2}}, \ldots, \vec{b_{n}}\}$ = Erzeugendensystem von $V$
    \end{itemize}
    \begin{center}
    \textcolor{pink}{$r g(B)=\operatorname{dim}(V)=n$} $\Rightarrow \{\vec{b_{1}}, \vec{b_{2}}, \ldots, \vec{b_{n}}\}$ ist eine Basis von $V$
    \end{center}
\end{KR}

\begin{formula}{Basiswechsel}
    Beliebige Basis $B \rightarrow$ Standard-Basis $S$

    \vspace{1mm}

    Gegeben: Basis B, aus $\vec{b}_S$ und Vektor $\vec{a}_B$ in Basis B\\
    {\small (die Vektoren $\vec{b}$ der Basis B sind in Standardbasis!)}
    $$
    B ={\vec{b}_1, ... \vec{b}_n} = \left\{ \begin{psmallmatrix} x_{1} \\ \scalebox{0.5}{\vdots} \\ z_{1} \end{psmallmatrix}_{S} \cdots \begin{psmallmatrix} x_{2} \\ \scalebox{0.5}{\vdots} \\ z_{2} \end{psmallmatrix}_{S} \right\}, \quad \vec{a}_B = \begin{psmallmatrix} a_{1} \\ \scalebox{0.5}{\vdots} \\ a_{n} \end{psmallmatrix}_{B}
    $$
    $\vec{a}_B \rightarrow \vec{a}_S$ umrechnen:
    $$ 
    B \cdot \begin{psmallmatrix} a_{1} \\ \scalebox{0.5}{\vdots} \\ a_{n} \end{psmallmatrix}_{B} = \begin{psmallmatrix} a_{1} \\ \scalebox{0.5}{\vdots} \\ a_{n} \end{psmallmatrix}_{S}
    \Rightarrow \vec{a}_S=\vec{b_{1}} \cdot (a_{1})_B + \cdots + \vec{b_{n}} \cdot (a_{n})_B 
    $$
\end{formula}

\begin{example}
    $$B=\left\{\binom{3}{1}_{S} ;\binom{-1}{0}_{S}\right\}, \vec{a}=\binom{2}{3}_{B} \Rightarrow \vec{a}=2 \cdot \binom{3}{1}+3 \cdot \binom{-1}{0} = \binom{3}{2}_S$$
\end{example}

\begin{formula}{Basiswechsel}
    Standard-Basis $S \rightarrow$ Beliebige Basis $B$
    \vspace{1mm}

    Gegeben: Basis B, aus $\vec{b}_S$ und Vektor $\vec{a}_S$ in Standardbasis
    $$
    B = \left\{ \begin{psmallmatrix} x_{1} \\ \scalebox{0.5}{\vdots} \\ z_{1} \end{psmallmatrix} ; \begin{psmallmatrix} x_{2} \\ \scalebox{0.5}{\vdots} \\ z_{2} \end{psmallmatrix} \right\}, \quad \vec{a}_S = \begin{psmallmatrix} a_{1} \\ \scalebox{0.5}{\vdots} \\ a_{n} \end{psmallmatrix}_{S} 
    $$
    \begin{minipage}{0.5\linewidth}
    $\vec{a}_S \rightarrow \vec{a}_B$ umrechnen: \\

    1. Finde Inverse $B^{-1}$ von $B$
    \end{minipage}
    \begin{minipage}{0.5\linewidth}
    $$
    \Rightarrow B^{-1} \cdot \begin{psmallmatrix} a_{1} \\ \scalebox{0.5}{\vdots} \\ a_{n} \end{psmallmatrix}_{S}=\begin{psmallmatrix} a_{1} \\ \scalebox{0.5}{\vdots} \\ a_{n} \end{psmallmatrix}_{B} 
    $$
    \end{minipage}
\end{formula}

\begin{example}
    $$
    B=\left\{\binom{1}{1}_{S} ;\binom{-1}{0}_{S}\right\}, \quad \vec{a}=\binom{-7}{-4}_{S} \Rightarrow \begin{psmallmatrix}
        1 & -1 \\
        1 & 0
        \end{psmallmatrix} \cdot\binom{a_{1}}{a_{2}}_{B}=\binom{-7}{-4}_{S}
    $$

    $$
    \begin{psmallmatrix} 1 & -1 \\ 1 & 0 \end{psmallmatrix} \cdot\begin{psmallmatrix} 1 & -1 \\ 1 & 0 \end{psmallmatrix}^{-1} \cdot\binom{a_{1}}{a_{2}}_{B}=\begin{psmallmatrix} 1 & -1 \\ 1 & 0 \end{psmallmatrix}^{-1} \cdot\binom{-7}{-4}_{S}
    $$

    $$
    \binom{a_{1}}{a_{2}}_{B} = \begin{psmallmatrix} 1 & -1 \\ 1 & 0 \end{psmallmatrix}^{-1} \cdot\binom{-7}{-4}_{S} 
    \Rightarrow \begin{psmallmatrix} 0 \cdot -7 + 1 \cdot -4 \\ -1 \cdot -7 + 1 \cdot -4 \end{psmallmatrix} = \binom{-4}{3}_{B}
    $$
\end{example}