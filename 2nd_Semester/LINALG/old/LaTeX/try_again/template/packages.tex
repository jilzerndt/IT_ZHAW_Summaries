\usepackage[hidelinks]{hyperref}
\usepackage{import}
\usepackage{caption}
\usepackage{subcaption}

%info:
\hypersetup{
	pdfinfo={
		Author={Jil Zerndt},
		Title={<Title>},
		Subject={<Subject>},
		Keywords={ZHAW}
	}
}

%font and language
\usepackage[utf8]{inputenc}
\usepackage[english, ngerman]{babel}
\usepackage[T1]{fontenc}

%"normal" text font:
\usepackage{lmodern} 

%for code snippets use this:
%{\fontfamily{cmtt}\selectfont "code you want in different font" }

%page layout/margins:
\usepackage[left=5mm, right=5mm, bottom=5mm, top=5mm]{geometry}
%for page numbering
\usepackage{lastpage}

%for titles:
\usepackage[explicit]{titlesec}
\usepackage{titling}

%for underlining:
\usepackage{ulem}

%headers and footers:
\usepackage{fancyhdr}

%lists and tables:
\usepackage{enumitem}
\setlist{nolistsep, leftmargin = 4mm}

%how lists are numbered:
%\setlist[enumerate]{label=(\roman*)}
\setenumerate[2]{label*=\arabic*}
\newlist{enumerateroman}{enumerate}{1}
\setlist[enumerateroman]{label=(\roman*)}

%using multiple columns/rows
\usepackage{multicol}
\usepackage{multirow}

%to adjust boxes & other layout stuff
\usepackage{graphicx}
\usepackage{csvsimple}
\usepackage{colortbl}
\usepackage{wrapfig}

%see pdf for instructions on tcolorbox
%for instructions and examples on different boxes, see boxes.tex
\usepackage{tcolorbox}
\tcbuselibrary{all}

%use if content extends over multiple pages
\usepackage{supertabular}

%math symbols:
\usepackage{bm}
\usepackage{calc}
\usepackage{textcomp}
\usepackage{amsmath}
\usepackage{esint}
\usepackage{amssymb}    % \cdots
\usepackage{commath}
\usepackage{cancel}     % strike-through terms
\usepackage{gensymb}
\usepackage{empheq}
\usepackage{aligned-overset}
\usepackage{mathtools}

%used together with tcolorbox/skins
\usepackage{tikz}
\usetikzlibrary{positioning, calc, fit, matrix, decorations, angles, quotes, babel, 3d, arrows.meta, shadings, shapes.geometric}

% pgfplot layers
\usepackage{pgfplots}
\usepgfplotslibrary{groupplots}
\usepackage{wrapfig}
\pgfplotsset{compat=1.7}                                                                
% Plots  (idk yet how this works i'll figure it out at some point)
\pgfdeclarelayer{bg}
\pgfdeclarelayer{l1}
\pgfdeclarelayer{l2}
\pgfdeclarelayer{l3}
\pgfdeclarelayer{tl1}
\pgfdeclarelayer{tl2}
\pgfdeclarelayer{tl3}
\pgfsetlayers{bg,l3,l2,l1,main,tl1,tl2,tl3}

\usepackage{tabularx}