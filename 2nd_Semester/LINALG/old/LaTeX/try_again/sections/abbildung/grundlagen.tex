\begin{definition}{Lineare Abbildung}
    Gegeben sind zwei reelle Vektorräume $V$ und $W$ ($V$ und $W$ können auch gleich sein).
    Eine Abbildung $f: V \to W$ heisst \textbf{linear Abbildung}, wenn für alle $x, y \in V$ und jedes $\lambda \in \mathbb{R}$ gilt:
    \begin{align}
        f(\vec{x}+\vec{y}) &= f(\vec{x}) + f(\vec{y}) \\
        f(\lambda \vec{x}) &= \lambda f(\vec{x})
    \end{align}
    Der Vektor $f(\vec{x})\in W$, der herauskommt, wenn $f$ auf einen Vektor $\vec{x}\in V$ angewendet wird, heisst \textbf{Bild} von $\vec{x}$ unter $f$.
    \begin{heighlight}{i}
        Linearität ist etwas Besonderes. Die allermeistne Abbildungen/Funktionen sind nicht linear.
    \end{heighlight}
\end{definition}

\begin{theorem}{Lineare Abbildung - Darstellung}
    Wir betrachten die Vektorräume $R'''$ und $R''$, versehen mit den jeweiligen Standardbasen.
    Dann lässt sich jede lineare Abbildung $f: R''\to R'''$ durch eine $m\times n$-Matrix $A$ darstellen:
    \begin{equation*}
        f(\vec{x})=A\times \vec{x}
    \end{equation*}
    Die Spalten der Matrix $A$ sind die Bilder der Basisvektoren von $R''$:
    \begin{equation*}
        A=\pmatrix{\vec{a}_1 & \vec{a}_2 & \cdots & \vec{a}_n}
    \end{equation*}
\end{theorem}

\begin{formula}{Zentrische Streckung}
    \begin{equation*}
        \begin{pmatrix}
            \lambda & 0 & 0\\
            0 & \lambda & 0\\
            0 & 0 & 1
        \end{pmatrix}
    \end{equation*}
\end{formula}

\begin{definition}{Kern}
    Der \textit{Kern} $\ker(A)$ einer $m\times n$-Matrix $A$ ist die Menge aller 
    Vektoren $\vec{x}\in R^n$ ist die Lösungsmenge des homogenen linearen Gleichungssystems
    $A\cdot\ve{x}=\vec{0}$.
\end{definition}

\begin{definition}{Bild}
    Das \textit{Bild} (auch Spaltenraum) $\Ima(A)$ einer $m\times n$-Matrix $A$,
    ist der Unterraum des $m-$dimensionalen Vektorraums $W$, der von den Spalten
    $\vec{a_1}, vec{a_2}, \ldots, \vec{a_n}$ der Matrix (aufgefasst als Vektoren in $W$) aufgespannt wird.
    \begin{equation*}
        \Ima(A)=span(\vec{a_1}, \vec{a_2}, \ldots, \vec{a_n})
        = \{\lambda\vec{a_1}+\lambda\vec{a_2}+\ldots+\vec{a_n}\mid \lambda_k\in\R\}
    \end{equation*}
\end{definition}

\begin{theorem}{Beziehung Kern und Bild}
    Für jede $m\times n$-Matrix $A$ gilt:
    \begin{equation*}
         \dim(im(A))=rg(A)\text{ und } \dim(ker(A))+dim(im(A))=n
    \end{equation*}
\end{theorem}