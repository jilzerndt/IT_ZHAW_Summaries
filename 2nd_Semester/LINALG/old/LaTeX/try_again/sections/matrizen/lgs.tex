\subsection{LGS}
    \begin{definition}{Rang}\\
        Der Rang $rg(A)$ einer Matrix $A$ in Stufenform gibt an, wie viele Zeilen von $A$ linear unabhängig sind.
        Der Rang entspricht der Anzahl der Zeilen von $A$ minus der Anzahl der Nullzeilen.
        \begin{equation*}
            rg(A) = \text{Gesamtanzahl Zeilen} - \text{Anzahl Nullzeilen}
        \end{equation*}
        Der Rang einer Matrix ist für die Lösbarkeit von Gleichungssystemen von Bedeutung.
    \end{definition}
    
    \begin{definition}{Matrizengleichung}
        
    \end{definition}
    
    \begin{definition}{Zeilenstufenform und reduzierte Zeilenstufenform}
        
    \end{definition}
    
    \begin{formula}{Gauss-Verfahren}\\
        Für das Gauss-Verfahren wird Schritt 1.-4 des Gauss-Jordan-Verfahren angewendet.
        Das resultierende LGS wird durch Rückwärtssubstitution gelöst.
    \end{formula}
    
    \begin{formula}{Gauss-Jordan-Verfahren}
        \begin{enumerate}
            \item Wir bestimmen die am weitesten links stehende Spalte mit Elementen $\neq 0$.
                Wir nennen diese Spalte die \textit{Pivot-Spalte}.
            \item Ist die oberste Zahl in der Pivot-Spalte $= 0$, 
                dann vertauschen wir die erste Zeile mit der obersten, die in der Pivot-Spalte ein Element $\neq 0$ hat.
            \item Die oberste Zahl in der Pivot-Spalte ist nun eine Zahl $a\neq 0$.
                Wir dividieren die erste Zeile durch $a$. So erhalten wir die führende Eins.
            \item Nun wollen wir unterhalb der führenden Eins lauter Nullen erzeugen. 
                Dazu addieren wir passende Vielfache der ersten Zeile zu den übrigen Zeilen. 
        \end{enumerate}
        Wir lassen nun die erste Zeile aussen vor 
        und wenden die ersten vier Schritte auf den verleibenden Teil der Matrix an.
        Dieses Verfahren wiederholen wir so oft, bis die erweiterte Koeffizientenmatrix Zeilenstufenform hat.
        \begin{enumerate}
            \setcounter{enumi}{4}
            \item Nun arbeiten wir von unten nach oben und addieren jeweils geeignete Vielfache jeder
                Zeile zu den darüber liegenden Zeilen, um über den führenden Einsen Nullen zu erzeugen.
        \end{enumerate}
    \end{formula}
    
    \begin{theorem}{Lösbarkeit von linearen Gleichungssystemen}
        \begin{enumerate}
            \item Ein LGS $A\cdot\vec{x}=\vec{c}$ ist genau dann lösbar, wenn $rg(A)=rg(A\mid\vec{c})$
            \item Es hat genau eine Lösung, falls \textbf{zusätzlich} zu 1. gilt: $rg(A)=n$
            \item Es hat unendlich viele Lösungen, falls \textbf{zusätzlich} gilt: $(rg(A)<n)$
        \end{enumerate}
        \begin{highlight}{i}
            Bei einem homogenen LGS ist nach Definition $\vec{c}=\vec{0}$; deswegen gilt immer: $rg(A)=rg(A\mid\vec{c})$.
            Daher gibt es bei homogenen LGS nur zwei Möglichkeiten:
            \begin{itemize}
                \item Das LGS hat \textbf{eine Lösung} $x_1=x_2=\cdots=x_n=0$, die sog. \textit{triviale Lösung}.
                \item Das LGS hat \textbf{unendlich viele Lösungen}.
            \end{itemize}
        \end{highlight}
    \end{theorem}