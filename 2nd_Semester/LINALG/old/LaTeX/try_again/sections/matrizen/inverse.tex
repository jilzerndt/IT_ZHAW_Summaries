\subsection{Inverse}
    \begin{definition}{Inverse}
        Die Inverse einer quadratischen Matrix $A$ ist eine Matrix $A^{-1}$, für die gilt:
        \begin{equation*}
            A\cdot A^{-1}=A^{-1}\cdot A=E
        \end{equation*}.
        Eine Matrix heisst \textit{invertierbar / regulär}, wenn sie eine Inverse hat. 
        Andernfalls heisst sie \textit{singulär}.
    \end{definition}
    \begin{theorem}{Eigenschaften invertierbarer Matrizen}
        \begin{enumerate}
            \item Die Inverse einer invertierbaren Matrix ist eindeutig bestimmt.
            \item Die Inverse einer invertierbaren Matrix $A$ ist invertierbar und es gilt: $(A^{-1})^{-1}=A$.
            \item Multiplizieren wir zwei invertierbare Matrizen $A$ und $B$ miteinander, 
                so ist das Produkt auch invertierbar und es gilt: ${(A\cdot B)}^{-1}=B^{-1}\cdot A^{-1}$.\\
                Die Reihenfolge ist relevant!
            \item Die Transponierte $A^T$ einer quadratischen Matrix $A$ ist genau dann invertierbar, 
                wenn $A$ invertierbar ist. In diesem Fall gilt: ${(A^T)^{-1}}={(A^{-1})}^T$.
        \end{enumerate}
    \end{theorem}
    \begin{lemma}{Zusammenhänge invertierbarkeit}\\
        Gegeben eines LGS $A\cdot\vec{x}=\vec{c}$ mit $n\times n$-Koeffizientenmatrix $A$.
        Dann sind folgende Aussagen äquivalent ($\Leftrightarrow$):
        \begin{enumerate}
            \item $A$ ist invertierbar.
            \item $A\cdot\vec{x}=\vec{c}$ hat genau eine Lösung.
            \item $rg(A)=n$
        \end{enumerate}
    \end{lemma}

    \begin{formula}{Inverse einer $2\times2$-Matrix}
        \begin{equation*}
            {\begin{pmatrix}
                a & b\\
                c & d 
            \end{pmatrix}}^{-1}
            =
            \frac{1}{ad-bc}\cdot
            \begin{pmatrix}
                d & -b\\
                -c & a
            \end{pmatrix}
        \end{equation*}
    \end{formula}

    \begin{formula}{Inverse einer $3\times 3$-Matrix}
        
    \end{formula}

    \begin{definition}{Homogenes LGS}\\
        Ein LGS heisst \textit{homogen}, wenn die rechte Seite $=\vec{0}$ ist: $A\cdot\vec{x}=\vec{0}$.
    \end{definition}
    \begin{lemma}{Zusammenhänge invertierbarkeit und Homogenes LGS}\\
        Ist $A$ invertierbar, so hat das homogene LGS $A\cdot\vec{x}=\vec{0}$ die eindeutige Lösung $x=A^{-1}\cdot 0=0$
    \end{lemma}