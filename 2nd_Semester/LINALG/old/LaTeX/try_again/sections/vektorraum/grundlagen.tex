\subsection{Grundlagen}
    \begin{definition}{Vektorraum}\\
        Ein \textit{reeller Vektorraum} ist eine Menge $V\neq\emptyset$ mit zwei Verknüpfungen:
        \begin{align*}
            +\,&:\,V\times V\rightarrow V:\,(\vec{a};\vec{b})\mapsto\vec{a}+\vec{b}\\
            \cdot\,&:\,\R\,\times\rightarrow V:\,(\lambda;\vec{a})\mapsto\lambda\cdot\vec{a}
        \end{align*}
        mit folgenden Eigenschaften:
        Gegeben $\vec{a},\vec{b},\vec{c}\in V$, $\lambda,\mu\in \R$, die Menge aller Vektoren $V$ 
        sowie dem Neutralelement $\vec{0}$ gilt:
        \begin{enumerate}
            \item Es gibt ein Element $\vec{0}\in V$, für das gilt: $\forall \vec{a}\in V\,:\vec{a}+\vec{0}=\vec{a}$
            \item Für jedes Element in $\vec{a}\in V$ gibt es genau ein $-\vec{a}\in V$, so dass $\vec{a}+(-\vec{a})=0$
            \item Es gilt $\forall \vec{a}\in V\,:1\cdot\vec{a}=\vec{a}$
            \item Kommutativgesetz: $\vec{a}+\vec{b}=\vec{b}+\vec{a}$
            \item Assoziativgesetz: \\
                $\vec{a}+(\vec{b}+\vec{c})=(\vec{a}+\vec{b})+\vec{c}$\\
                $\lambda\cdot(\mu\cdot\vec{a})=(\lambda\cdot\mu)\cdot\vec{a}$
            \item Distributivgesetz: \\
                $\lambda\cdot(\vec{a}+\vec{b})=\lambda\cdot\vec{a}+\lambda\cdot\vec{b}$\\
                $(\lambda+\mu)\cdot\vec{a}=\lambda\cdot\vec{a}+\mu\cdot\vec{a}$
        \end{enumerate}
        \textbf{Wichtig:} Die Betrachtung, dass ein Vektor ein Objekt mit \textit{Betrag} und \textit{Richtung} ist,
        stimmt in dieser allgemeinen Sichtweise nicht mehr umbedingt.
    \end{definition}

    \begin{theorem}{Eigenschaften eines Vektorraums}\\
        Dammit eine Menge $V$ mit einer Addition und skalaren Multiplikation ein Vektorraum ist, muss gelten:
        \begin{enumerate}
            \item Die Regeln (1)-(8) aus der Definition werden eingehalten.
            \item $\forall \vec{a},\vec{b}\in V\,:(\vec{a}+\vec{b})\in V$
            \item $\forall \lambda\in \R, \vec{a}\in V\,:(\lambda\cdot\vec{a})\in V$
        \end{enumerate}
    \end{theorem}

    \begin{definition}{Unterraum}
        Eine Teilmenge $U$ eines Vektorraums $V$ heisst \textit{Unterraum}, wenn $U$ selber auch ein Vektorraum ist.
        Nich jede Teilmenge $U\subseteq V$ ist ein Unterraum von $V$. Zwar erfüllt sie die Vektorraum-Eigenschaften
        aus der Definition, jedoch ist nicht garantiert, dass für $\vec{a},\vec{b}\in U$ $\vec{a}+\vec{b}\in U$ gilt. 
    \end{definition}

    \begin{theorem}{Unterraumkriterien}
        Eine Teilmenge $U\neq\emptyset$ eines Vektorraums $V$ ist genau dann ein Unterraum von $V$, wenn gilt:
        \begin{enumerate}
            \item $\forall \vec{a},\vec{a}\in U\,:\vec{a}+\vec{b}\in U$
            \item $\forall \lambda\in\R,\vec{a}\in U\,:\lambda\cdot\vec{a}\in U$
        \end{enumerate}
        \textbf{Wichtig:} $U$ enthält $\vec{0}$. Falls $\vec{0}\notin U$, ist $U$ kein Unterraum. 
    \end{theorem}

    \begin{definition}{Unterraum}
        Die Teilmenge $U=\{\vec{0}\}\subseteq V$, die nur den Nullvektor aus einem Vektorraum V enthält, 
        heisst der \textit{Nullvektorraum} und ist immer ein Unterraum von $V$.
    \end{definition}