\begin{definition}{Reguläre Ausdrücke}
    Wörter, die Sprachen beschreiben
\end{definition}

\begin{concept}{$R A_{\Sigma}$}
    Sprache der Regulären Ausdrücke über $\{\emptyset, \epsilon, *,(),, \mid\} \cup \Sigma$ 
    
    \begin{minipage}{0.5\linewidth}
    \begin{itemize}
        \item $R \in R A_{\Sigma}$  $\Rightarrow\left(R^{*}\right) \in R A_{\Sigma}$
        \item $R, S \in R A_{\Sigma}$ $\Rightarrow(R S) \in R A_{\Sigma}$
        \item $R, S \in R A_{\Sigma}$ $\Rightarrow(R \mid S) \in R A_{\Sigma}$   
    \end{itemize}
    \end{minipage}
    \begin{minipage}{0.3\linewidth}
        \begin{itemize}
            \item $\emptyset, \epsilon \in R A_{\Sigma}$
            \item $\Sigma \subset R A_{\Sigma}$
        \end{itemize}
    \end{minipage}

    \vspace*{1mm}

    \textbf{Priorisierung von Operatoren}
    
    (1) $*=$ Wiederholung $\rightarrow$ (2) Konkatenation $\rightarrow$ (3) $\mid=$ Oder
    
    \textbf{Erweiterter Syntax}

        $
        \begin{array}{lcr}
            R^{+} = R (R^{*}) & R ?=(R \mid \epsilon) & \quad \left[R_{1}, \ldots, R_{k}\right]=R_{1}\left|R_{2}\right| \ldots \mid R_{k} 
        \end{array}
        $
\end{concept}



\begin{definition}{Reguläre Sprache}
    $A$ über dem Alphabet $\Sigma$ heisst regulär, falls \textcolor{pink}{$A=L(R)$} für einen regulären Ausdruck $R \in R A_{\Sigma}$ gilt.
\end{definition} 

\begin{concept}
    {$\forall  R \in R A_{\Sigma}$} definieren wir die Sprache $L(R)$ von $R$ wie folgt:
    \begin{itemize}
        \item Leere Sprache: $L(\emptyset)=\emptyset$
        \item Sprache, die nur das leere Wort enthält: $L(\varepsilon)=\{\varepsilon\}$
        \item Beschreibt die Sprache $\{a\}$: $L(a)=\{a\} \quad \forall a \in \Sigma$
        \item Kombiniert die Wörter von R: $L(R^{*})=L(R)^{*}$
        \item Verkettung von Wörtern (R = prefix): $L(R S)=L(R) \circ L(S)$
        \item Wörter in R oder S: $L(R \mid S)=L(R) \cup L(S)$
    \end{itemize}
\end{concept}








