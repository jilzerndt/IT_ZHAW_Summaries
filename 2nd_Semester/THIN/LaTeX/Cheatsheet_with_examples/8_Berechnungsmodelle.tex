\graphicspath{{images/}}

\section*{Berechnungsmodelle}

\begin{definition}{Turing-berechenbar} = kann von Turing-Maschine gelöst werden
    
    Turing-berechenbare Funktion $T: \Sigma^* \rightarrow \delta^*$

    $T(\omega) = \begin{cases}
        u & \text{falls T auf } \omega \in \Sigma^* \text{ angesetzt, nach endlich vielen}\\ 
        &\text{Schritten mit u auf dem Band anhält}\\
        \uparrow & \text{falls T bei Input } \omega \in \Sigma^* \text{ nicht hält}
    \end{cases}$
\end{definition}

\begin{remark}
    Jedes algorithmisch lösbare Berechnungsproblem ist turing-berechenbar $\Rightarrow$ Computer $\equiv$ TM
\end{remark}

\begin{theorem}{Primitiv rekursive Grundfunktionen} $\forall n \in \mathbb{N}$, $\forall k \in \mathbb{N}$ {\footnotesize (k = Konstante)}
    
    \vspace{1mm}

    n-stellige konstante Funktion: $c_k^n = \mathbb{N}^n \rightarrow \mathbb{N} \text{ mit } c_k^n (x_1, ... , x_n) = k$

    \vspace{1mm}

    Nachfolgerfunktion: $\eta : \mathbb{N} \rightarrow \mathbb{N} \text{ mit } \eta (x) = x + 1$

    \vspace{1mm}
    
    n-stellige Projektion auf die k-te Komponente: {\footnotesize ($1 < k < n$)}
    $$\pi_k^n : \mathbb{N}^n \rightarrow \mathbb{N} \text{ mit } \pi_k^n (x_1, ... ,x_k,..., x_n) = k$$
    {\small n = Anzahl der Argumente, k = Position des Arguments}
\end{theorem}

\begin{minipage}{0.5\linewidth}
    \begin{KR}{Loop (primitiv-rekursiv)}
        \begin{itemize}
            \item Zuweisungen:\\ $x = y + c$ und $x = y - c$
            \item Sequenzen: $P$ und $Q \rightarrow P; Q$
            \item Schleifen:\\ $P \rightarrow$ Loop $x$ do $P$ until End
        \end{itemize}
    \end{KR}
    \begin{KR}{While (Turing vollständig)}\\
        Erweiterung deer Sprache Loop
        \begin{itemize}
            \item While $x_i > 0$ do ... until End
        \end{itemize}
    \end{KR}
\end{minipage}
\begin{minipage}{0.5\linewidth}
    \begin{KR}{GoTo (Turing vollständig)}
        \begin{itemize}
            \item Zuweisungen:\\ $x_i = x_j + c$ und $x_i = x_j - c$
            \item Sprunganweisung:\\ IF $x_i = c$ THEN GOTO $L_k$ ELSE GOTO $L_t$
            \begin{itemize}
                \item or simple: GOTO $L_k$
            \end{itemize}
            \item Schleifen:\\ WHILE $x_i > 0$ DO ... HALT
        \end{itemize}
        \vspace{3mm}
    \end{KR}
\end{minipage}
\raggedcolumns