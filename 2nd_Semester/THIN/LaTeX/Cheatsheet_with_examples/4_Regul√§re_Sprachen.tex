\graphicspath{{images/}}

\paragraph{Reguläre Sprachen und endliche Automaten}

\begin{concept}{Reguläre Sprachen}durch äquivalente Mechanismen beschreibbar
    \begin{itemize}
        \item Akzeptierender Mechanismus
         DEA, NEA, $\varepsilon$-NEA
        \item Beschreibender Mechanismus RA
    \end{itemize}
\end{concept}

\begin{definition}{Zustandsklasse}
    $\Sigma^{*}=\bigcup_{p \in Q}[p] \quad [p] \cap[q]=\emptyset, \forall p \neq q, p, q \in Q$

    Jedes Wort landet in einem Zustand, aber kein Wort landet nach dem Lesen in zwei Zuständen!
\end{definition}

\begin{theorem}{Eigenschaften}
    Seien $L$, $L_{1}$ und $L_{2}$ reguläre Sprachen über $\Sigma$
    \begin{itemize}
        \item Vereinigung: $L_{1} \cup L_{2}=\{\omega \mid \omega \in L_{1} \vee \omega \in L_{2}\}$
        \item Schnitt: $L_{1} \cap L_{2}=\{\omega \mid \omega \in L_{1} \wedge \omega \in L_{2}\}$
        \item Differenz: $L_{1}-L_{2}=\{\omega \mid \omega \in L_{1} \wedge \omega \notin L_{2}\}$
        \item Komplement: $\bar{L}=\Sigma^{*}-L=\{\omega \in \Sigma^{*} \mid \omega \notin L\}$
        \item Konkatenation:\\
        $L_{1} \cdot L_{2}= L_{1} L_{2} = \{\omega=\omega_{1} \omega_{2} \mid \omega_{1} \in L_{1} \wedge \omega_{1} \in L_{2}\}$
        \item Kleenesche Hülle:\\
        $L^{*}=\{\omega=\omega_{1} \omega_{2} \ldots \omega_{n} \mid \omega_{i} \in L \text { für alle } i \in\{1,2, \ldots, n\}\}$
    \end{itemize}
\end{theorem}

\begin{remark}
    $L\left(R_{1}\right)$ : Menge der ganzen Zahlen in Dezimaldarstellung
    \begin{itemize}
    \item $((-\mid \varepsilon)(1,2,3,4,5,6,7,8,9)(0,1,2,3,4,5,6,7,8,9) \mid 0) .0$
    \end{itemize}
\end{remark}

