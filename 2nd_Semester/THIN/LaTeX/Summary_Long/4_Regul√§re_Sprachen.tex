\graphicspath{{images/}}
\section*{Reguläre Sprachen}

\begin{concept}{Äquvivalente Mechanismen}\\
    \begin{itemize}
        \item Akzeptierender Mechanismus DEA, NEA, $\varepsilon$-NEA
        \item Beschreibender Mechanismus RA
    \end{itemize}
      \includegraphics[width=0.2\linewidth]{reguläre_sprachen.png}\\
    \textbf{Äquivalenz DEA und RA}
    \begin{itemize}
    \item Es gibt einen DEA, der die Sprache $L$ akzeptiert
    \item Es gibt einen RA, der die Sprache $L$ akzeptiert.
    \end{itemize}
\end{concept}

\begin{theorem}{Abschlusseigenschaften regulärer Sprachen}\\
    Seien $L_{1}$ und $L_{2}$ zwei reguläre Sprachen über $\Sigma$. Dann ist die Vereinigung ... regulär.
    $$
    L_{1} \cup L_{2}=\left\{\omega \mid \omega \in L_{1} \vee \omega \in L_{2}\right\}
    $$
    Sei $L$ eine reguläre Sprache über $\Sigma$. Dann ist auch das Komplement ... regulär.
    $$
    \bar{L}=\Sigma^{*}-L=\left\{\omega \in \Sigma^{*} \mid \omega \notin L\right\}
    $$
    \begin{itemize}
        \item Schnitt: $L_{1} \cap L_{2}=\left\{\omega \mid \omega \in L_{1} \wedge \omega \in L_{2}\right.$
        \item Differenz: $L_{1}-L_{2}=\left\{\omega \mid \omega \in L_{1} \wedge \omega \notin L_{2}\right\}$
        \item Konkatenation:\\
        $L_{1} \cdot L_{2}=L_{1} L_{2}=\left\{\omega=\omega_{1} \omega_{2} \mid \omega_{1} \in L_{1} \wedge \omega_{1} \in L_{2}\right\}$
        \item Kleenesche Hülle:\\ 
        $L^{*}=\left\{\omega=\omega_{1} \omega_{2} \ldots \omega_{n} \mid \omega_{i} \in L \text { für alle } i \in\{1,2, \ldots, n\}\right\}$
    \end{itemize}
\end{theorem}

\begin{definition}{Zustandsklasse}\\
    Jedes Wort landet in einem Zustand
    $$
    \Sigma^{*}=\bigcup_{p \in Q}[p]
    $$
    Kein Wort landet nach dem Lesen in zwei Zuständen
    $$
    [p] \cap[q]=\emptyset, \text { für alle } p \neq q, p, q \in Q
    $$
\end{definition}

\begin{example}
    Nach dem Lesen von $\omega$ landet man im Zustand $p$.
    $$
    \operatorname{Klasse}\left[q_{0}\right]=\left\{\left.\omega \in\{0,1\}^{*}|| \omega\right|_{0} \bmod (3)=1\right\}
    $$
    Von $M$ akzeptierte Sprache
    $$
    L(M)=\bigcup_{p \in F}[p]
    $$
\end{example}

\begin{example}
    $$
    L(M)=\left\{\left.\forall \omega \in\{0,1\}^{*}|| \omega\right|_{0} \bmod (3)=1\right\}
    $$
\end{example}