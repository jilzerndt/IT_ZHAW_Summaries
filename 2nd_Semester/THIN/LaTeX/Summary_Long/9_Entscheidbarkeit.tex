\graphicspath{{images/}}

\section*{Entscheidbarkeit}

\begin{theorem}{Entscheidbarkeit}\\
    \begin{itemize}
        \item Ein Problem ist entscheidbar, wenn es einen Algorithmus gibt, der für jede Eingabe eine Antwort liefert.
        \item Ein Problem ist semi-entscheidbar, wenn es einen Algorithmus gibt, der für jede Eingabe eine Antwort liefert, falls die Antwort ja ist.
    \end{itemize}
    Eine Sprache $A \subset \Sigma^{*}$ ist genau dann entscheidbar, wenn sowohl $A$ als auch $\bar{A}$ semi-entscheidbar ist.
    \begin{itemize}
        \item $\bar{A}$ steht für das Komplement von $A$ in $\Sigma^{*}: \quad \bar{A}=\Sigma^{*} \backslash A=\left\{\omega \in \Sigma^{*} \mid \omega \notin A\right\}$
    \end{itemize}
\end{theorem}

\begin{corollary}{Entscheidbarkeit und Turingmaschinen}
    Eine Sprache $A \subset \Sigma^{*}$ heisst entscheidbar, wenn eine TM $T$ existiert, die sich wie folgt verhält:
    \begin{itemize}
    \item Bandinhalt $x \in A \quad T$ hält mit Bandinhalt «1» (Ja) an
    \item Bandinhalt $x \in \Sigma^{*} \backslash A \quad T$ hält mit Bandinhalt «0» (Nein) an
    \end{itemize}

    Äquivalente Aussagen:
    \begin{itemize}
        \item $A \subset \Sigma^{*}$ ist entscheidbar
        \item Es existiert eine $T M$, die das Entscheidungsproblem $T(\Sigma, A)$ löst
        \item Es existiert ein WHILE-Programm, dass bei einem zu $A$ gehörenden Wort stets terminiert $\rightarrow$ Entscheidungsverfahren für $A$
    \end{itemize}
\end{corollary}

\begin{corollary}{Semi-Entscheidbarkeit Turingmaschinen}\\
    Eine Sprache $A \subset \Sigma^{*}$ heisst semi-entscheidbar, wenn eine TM $T$ existiert, die sich wie folgt verhält:

    \begin{itemize}
    \item Bandinhalt $x \in A \quad T$ hält mit Bandinhalt «1» (Ja) an
    \item Bandinhalt $x \in \Sigma^{*} \backslash A \quad T$ hält nie an
    \end{itemize}

    Äquivalente Aussagen
    \begin{itemize}
    \item $A \subset \Sigma^{*}$ ist semi-entscheidbar
    \item $A \subset \Sigma^{*}$ ist rekursiv aufzählbar
    \item Es gibt eine TM, die zum Entscheidungsproblem $T(\Sigma, A)$ nur die positiven («Ja») Antworten liefert und sonst gar keine Antwort
    \item Es gibt ein WHILE-Programm, dass bei einem zu $A$ gehörenden Wort stets terminiert und bei Eingabe von Wörtern die nicht zu $A$ gehören nicht terminiert
    \end{itemize}
\end{corollary}

\begin{theorem}{Reduzierbarkeit}\\
    Eine Sprache $A \subset \Sigma^{*}$ heisst auf eine Sprache $B \subset \Gamma^{*}$ reduzierbar, wenn es eine totale, Turing-berechenbare Funktion $F: \Sigma^{*} \rightarrow \Gamma^{*}$ gibt, so dass für alle $\omega \in \Sigma^{*}$
    $$
    \omega \in A \Leftrightarrow F(\omega) \in B
    $$
    \begin{itemize}
    \item $A \preccurlyeq B \quad A$ ist reduzierbar auf $B$
    \item $A \preccurlyeq B$ und $B \preccurlyeq C \rightarrow A \preccurlyeq C$
    \end{itemize}
\end{theorem}

\begin{concept}{Halteproblem}\\
    Das allgemeine Halteproblem $H$ ist die Sprache (\# = Delimiter)
    \begin{itemize}
    \item $\quad H:=\left\{\omega \# x \in\{0,1, \#\}^{*} \mid T_{\omega}\right.$ angesetzt auf $x$ hält $\}$
    \end{itemize}
    Sprachen der Halteprobleme (HP): leeres $H P H_{0}$ und spezielles HP $H_{S}$
    \begin{itemize}
    \item $H_{0}:=\left\{\omega \in\{0,1\}^{*} \mid T_{\omega}\right.$ angesetzt auf das leere Band hält $\}$
    \item $H_{S}:=\left\{\omega \in\{0,1\}^{*} \mid T_{\omega}\right.$ angesetzt auf $\omega$ hält $\}$
    \end{itemize}
    $H_{0}, H_{S}$ und $H$ sind semi-entscheidbar.
\end{concept}