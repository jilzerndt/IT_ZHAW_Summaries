\section*{Alphabete, Wörter, Sprachen}

\begin{definition}{Alphabete}
    sind endliche, nichtleere Mengen von Symbolen.

\begin{itemize}
  \item $\Sigma=\{a, b, c\} \quad$ Mengen von drei Symbolen
  \item $\Sigma_{\text {Bool }}=\{0,1\} \quad$ Boolsches Alphabet
\end{itemize}
\textbf{Keine Alphabete}
\begin{itemize}
  \item $\mathbb{N}, \mathbb{R}, \mathbb{Z}$ usw. (unendliche Mächtigkeit)
\end{itemize}
\end{definition}

\begin{definition}{Wort}
    ist eine endliche Folge von Symbolen eines bestimmten Alphabets.
    \begin{itemize}
    \item $a b c \quad$ Wort über dem Alphabet $\Sigma_{\text {lat }}$ (oder über $\Sigma=\{a, b, c\}$ )
    \item 100111 Wort über dem Alphabet $\{0,1\}$
    \item $\varepsilon \quad$ Leeres Wort (über jedem Alphabet)
    \end{itemize}
\end{definition}

\begin{definition}{Schreibweisen}
    \textbf{$|\omega|=$ Länge eines Wortes}
    \begin{itemize}
    \item $|100111|=6$
    \item $|\varepsilon|=0$
    \end{itemize}

    \textbf{$|\omega|_{x}=$ Häufigkeit eines Symbols} $x$ in einem Wort
    \begin{itemize}
    \item $|100111|_{1}=4$
    \item $|\varepsilon|_{0}=0$
    \item $|\varepsilon|_{\varepsilon}=1$
    \end{itemize}

    \textbf{$\omega^{R}=$ Spiegelwort/Reflection} zu $\omega$
    \begin{itemize}
    \item $(a b c)^{R}=c b a$
    \item $(100111)^{R}=111001$
    \item $\varepsilon^{R}=\varepsilon$
    \end{itemize}
\end{definition}

\begin{definition}{Teilwort (Infix)}
    $v$ ist ein Teilwort (Infix) von $\omega$ ist, wenn man $\omega$ als $\omega=x v y$.
    $$
    \omega \neq v \rightarrow \text { Echtes Teilwort }
    $$
    \begin{itemize}
        \item Teilwörter von $a b b a \quad \varepsilon, a, b, a b, a b b, b b, b b a, a b b a, b a$
        \item Präfixe von $a b b a \quad \varepsilon, a, a b, a b b, a b b a$
        \item Suffixe von $a b b a \quad \varepsilon, a, b a, b b a, a b b a$
    \end{itemize}
\end{definition}

\begin{definition}{Mengen von Wörtern}
    $\Sigma^{k}=$ Die Menge aller Wörter der Länge $\boldsymbol{k}$ über einem Alphabet $\Sigma$

    \begin{itemize}
      \item $\quad \Sigma=\{a, b, c\} \quad \Sigma^{2}=\{a a, a b, a c, b a, b b, b c, c a, c b, c c\}$
      \item $\Sigma=\{0,1\}$ \quad $\{0,1\}^{3}=\{000,001,010,011,100,101,110,111\}$
      \item $\Sigma^{0}=\{\varepsilon\}$
    \end{itemize}

    \begin{itemize}
        \item $\Sigma^{*}=\underbrace{\Sigma^{0}}_{1} \cup \underbrace{\Sigma^{1}} \cup \underbrace{\Sigma^{2}} \cup \underbrace{\Sigma^{3}} \cdots \quad$ Kleensche Hülle
        \item $\Sigma^{+}=\underbrace{\Sigma^{1}}_{2} \cup \underbrace{\sum^{2}}_{4} \cup \underbrace{\Sigma^{3}}_{8} \ldots=\Sigma^{*} \backslash\{\varepsilon\} \quad$ Positive Hülle
      \end{itemize}
\end{definition}

\begin{definition}{Konkatenation}
    $=$ Verkettung von zwei beliebigen Wörtern $x$ und $y$
    $$
    x \circ y=x y:=\left(x_{1}, x_{2} \ldots x_{n}, y_{1}, y_{2} \ldots y_{m}\right)
    $$
\end{definition}

\begin{definition}{Wortpotenzen}
    Sei $x$ ein Wort über einem Alphabet $\Sigma$.
    \begin{itemize}
    \item $x^{0}=\varepsilon$
    \item $x^{n+1}=x^{n} \circ x=x^{n} x$
    \item bbababababbaaaabab $=b^{2}(a b)^{4} b a^{3}(a b)^{2}$
    \end{itemize}
\end{definition}

\begin{definition}{Sprache}
    über einem Alphabet $\Sigma=$ Eine Teilmenge $L \subseteq \Sigma^{*}$ von Wörtern.
    \begin{itemize}
    \item $\Sigma_{1} \subseteq \Sigma_{2} \wedge L$ Sprache über $\Sigma_{1} \rightarrow L$ Sprache über $\Sigma_{2}$
    \item $\quad \Sigma^{*}$ Sprache über jedem Alphabet $\Sigma$
    \item \{\}$=\emptyset$ ist die leere Sprache
    \end{itemize}
    \textbf{Konkatenation} von zwei Sprachen $A \subset \Sigma^{*}$ und $B \subset \Gamma^{*}$
    $$
    A B=\{u v \mid u \in A \text { und } v \in B\}
    $$
    Die \textbf{Kleenesche Hülle} $A^{*}$ einer Sprache $A=\{\varepsilon\} \cup A \cup A A \cup A A A \cup \ldots$
\end{definition}