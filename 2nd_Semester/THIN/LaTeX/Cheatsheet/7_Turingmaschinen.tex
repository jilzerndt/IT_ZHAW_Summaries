\graphicspath{{images/}}

\section*{Turingmaschinen}

\begin{theorem}{Turingmaschine (TM)} $M=\left(Q, \Sigma, \Gamma, \delta, q_{0}, \sqcup , F\right)$

    \begin{minipage}{0.45\linewidth}
        $Q$: Menge von Zuständen

        $\Sigma$: Alphabet der Eingabe

        $\Gamma$ und $\Sigma \subset \Gamma$: Bandalphabet
    \end{minipage}
    \begin{minipage}{0.55\linewidth}
        $q_{0} \in Q$: Anfangszustand

        $F \subseteq Q$: Akzeptierende Zustände

        ${ }_{\sqcup }$: Leerzeichen mit ${ }_{\mu} \in \Gamma$ und ${ }_{\mu} \notin \Sigma$
    \end{minipage}

    Übergangsfunktion: $\boldsymbol{\delta}: \boldsymbol{Q} \times \boldsymbol{\Gamma} \rightarrow \boldsymbol{Q} \times \boldsymbol{\Gamma} \times \boldsymbol{D}, \boldsymbol{D}=\{\boldsymbol{L}, \boldsymbol{R}\}$
    
    \vspace{1mm}

    Sie bestehen aus einem Lese-/Schreibkopf und einem unendlichen Band von Zellen.

    \vspace{1mm}

    Sie bildet das 2-Tupel $(q, X)$ auf das Tripel $(p, Y, D)$

    \begin{minipage}{0.45\linewidth}
        $D=$ Direction

        $X=$ Read

        $Y=$ Overwrite
    \end{minipage}
    \begin{minipage}{0.5\linewidth}
       $q, p \in Q$ und $X, Y \in \Gamma$

       \emph{$q-X / Y, D \rightarrow p$}
    \end{minipage}

\end{theorem}

\begin{definition}{Band} Unterteilt in Zellen mit jeweils einem beliebigen Symbol
    
    Beinhaltet zu Beginn die Eingabe, d.h. ein endliches Wort aus $\Sigma^{*}$. Alle anderen Zellen enthalten das besondere Symbol $\sqcup$
\end{definition}

\begin{definition}{Konfiguration}einer TM $M$ wird eindeutig spezifiziert durch:

    Zustand der Zustandssteuerung, Position des Lese-/Schreibkopfes und Bandinhalt
\end{definition}





