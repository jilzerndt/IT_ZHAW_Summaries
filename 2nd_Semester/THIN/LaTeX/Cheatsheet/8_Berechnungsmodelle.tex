\graphicspath{{images/}}

\section*{Berechnungsmodelle}

\begin{definition}{Turing-berechenbar} = kann von Turing-Maschine gelöst werden
    
    Turing-berechenbare Funktion $T: \Sigma^* \rightarrow \delta^*$

    $T(\omega) = \begin{cases}
        u & \text{falls T auf } \omega \in \Sigma^* \text{ angesetzt, nach endlich vielen}\\ 
        &\text{Schritten mit u auf dem Band anhält}\\
        \uparrow & \text{falls T bei Input } \omega \in \Sigma^* \text{ nicht hält}
    \end{cases}$
\end{definition}

\begin{remark}
    {\footnotesize
    $\forall$ algorithmisch lösbare Problem ist turing-berechenbar $\Rightarrow$ Computer $\equiv$ TM
    }
\end{remark}

\begin{theorem}{Primitiv rekursive Grundfunktionen} $\forall n \in \mathbb{N}$, $\forall k \in \mathbb{N}$ {\footnotesize (k = Konstante)}
    
    \vspace{1mm}

    n-stellige konstante Funktion: $c_k^n = \mathbb{N}^n \rightarrow \mathbb{N} \text{ mit } c_k^n (x_1, ... , x_n) = k$

    \vspace{1mm}

    Nachfolgerfunktion: $\eta : \mathbb{N} \rightarrow \mathbb{N} \text{ mit } \eta (x) = x + 1$

    \vspace{1mm}
    
    n-stellige Projektion auf die k-te Komponente: 

    \vspace{1mm}

    $\quad \pi_k^n : \mathbb{N}^n \rightarrow \mathbb{N} \text{ mit } \pi_k^n (x_1, ... ,x_k,..., x_n) = k$ {\footnotesize $\quad$ ($1 < k < n$)}
    
    {\small n = Anzahl der Argumente, k = Position des Arguments}
\end{theorem}

\begin{minipage}
    {0.62\linewidth}
\begin{concept}{LOOP{,} WHILE{,} GOTO}

        Zuweisung: $xi = xj + c$ oder $xi = xj - c$\\
        Nach Ablauf eines Programms steht der Wert der Berechnung in der Variable $x_0$
\end{concept}
\end{minipage}
\begin{minipage}{0.34\linewidth}
    {\small
    Variablen: $x0, x1, x2, ...$\\
Konstanten: $0, 1, 2, ...$\\
Operationszeichen: $+, -$\\
Trennzeichen: $;$}
\end{minipage}

\begin{minipage}{0.5\linewidth}
    \begin{KR}{Loop (primitiv-rekursiv)}

            Schlüsselwörter: Loop, Do, End

            Sequenzen: $P$ und $Q \rightarrow P; Q$

            Schleifen: Loop $x$ do $P$ ... End
    \end{KR}
    \end{minipage}
    \begin{minipage}{0.5\linewidth}
    \begin{example}    
\begin{lstlisting}[style=Pseudocode, aboveskip=-0.5\baselineskip, belowskip=-0.5\baselineskip]
Loop x1 Do
    x2 = x2 + 1
End;
x0 = x2 + 0
\end{lstlisting}
    \end{example}
\end{minipage}

\begin{minipage}{0.5\linewidth}
    \begin{KR}{While (Turing vollständig)}

        Erweiterung der Sprache Loop

        While $xi > 0$ Do ... End
    \end{KR}
    \begin{example}        
\begin{lstlisting}[style=Pseudocode, aboveskip=-0.5\baselineskip, belowskip=-0.5\baselineskip]
While x1 > 0 Do
    x1 = x1 - 1;
    Loop x2 Do
        x0 = x0 + 1
    End
End
\end{lstlisting}
\end{example}
\end{minipage}
\begin{minipage}{0.5\linewidth}
\begin{KR}{GoTo (Turing vollständig)}\\
        Schlüsselwörter: Goto, If, Else
        
        Marker Mk: M1, M2, ...
        
        Sprunganweisung:\\ If $xi = c$ Then Goto $Mk$ 
\end{KR}
\begin{example}
\begin{lstlisting}[style=Pseudocode, aboveskip=-0.5\baselineskip, belowskip=-0.5\baselineskip]
M1: x0 = x3 + 0;
M2: If x1 = 0 Then Goto M4;
M3: x0 = x2 + 0;
M4: Halt
\end{lstlisting}
\end{example}
\end{minipage}
