\graphicspath{{images/}}

\section*{Berechnungsmodelle}

\begin{definition}{Turing-berechenbar} = kann von Turing-Maschine gelöst werden
    
    Turing-berechenbare Funktion für TM $T: \Sigma^* \rightarrow \delta^*$
    $$T(\omega) = \begin{cases}
        u & \text{falls T auf } \omega \in \Sigma^* \text{ angesetzt, nach endlich vielen}\\ 
        &\text{Schritten mit u auf dem Band anhält}\\
        \uparrow & \text{falls T bei Input } \omega \in \Sigma^* \text{ nicht hält}
    \end{cases}$$
\end{definition}

\begin{remark}
    Jedes algorithmisch lösbare Berechnungsproblem ist turing-berechenbar $\Rightarrow$ Computer $\equiv$ TM
\end{remark}

\begin{theorem}{Primitiv rekursive Grundfunktionen}\\
    $\forall n \in \mathbb{N}$ und jede Konstante $k \in \mathbb{N}$ die n-stellige konstante Funktion:
    $$c_k^n = \mathbb{N}^n \rightarrow \mathbb{N} \text{ mit } c_k^n (x_1, ... , x_n) = k$$
    Nachfolgerfunktion: $\eta : \mathbb{N} \rightarrow \mathbb{N} \text{ mit } \eta (x) = x + 1$

    \vspace{1mm}
    
    $\forall n \in \mathbb{N}$, $1 < k < n$ die n-stellige Projektion auf die k-te Komponente:
    $$\pi_k^n : \mathbb{N}^n \rightarrow \mathbb{N} \text{ mit } \pi_k^n (x_1, ... ,x_k,..., x_n) = k$$
    n = Anzahl der Argumente, k = Position des Arguments
\end{theorem}

\begin{minipage}{0.5\linewidth}
    \begin{KR}{Loop (primitiv-rekursiv)}
        \begin{itemize}
            \item Zuweisungen:\\ $x = y + c$ und $x = y - c$
            \item Sequenzen: $P$ und $Q \rightarrow P; Q$
            \item Schleifen:\\ $P \rightarrow$ Loop $x$ do $P$ until End
        \end{itemize}
    \end{KR}
    
    \begin{example2}{Addition von natürlichen Zahlen}
    
        Add(x, y) = x + y
    \begin{lstlisting}[style=Pseudocode]
    LOOP x1 DO
        x2 = x2 + 1
    END
    x0 = x2 + 0
    \end{lstlisting}
    \end{example2}
\end{minipage}
\begin{minipage}{0.5\linewidth}
    \begin{KR}{While (Turing vollständig)}\\
        Erweiterung deer Sprache Loop
        \begin{itemize}
            \item While $x_i > 0$ do ... until End
        \end{itemize}
    \end{KR}
    
    \begin{example2}{Multiplikation \\von natürlichen Zahlen}
        
        Mul(x, y) = x * y
    \begin{lstlisting}[style=Pseudocode]
    WHILE x1 > 0 DO
        x1 = x1 - 1
        LOOP x2 DO
            x0 = x0 + 1
        END
    END
    \end{lstlisting}
    \end{example2}
\end{minipage}




\begin{KR}{GoTo (Turing vollständig)}
    \begin{itemize}
        \item Zuweisungen: $x_i = x_j + c$ und $x_i = x_j - c$
        \item Sprunganweisung: IF $x_i = c$ THEN GOTO $L_k$ ELSE GOTO $L_t$
        \begin{itemize}
            \item or simple: GOTO $L_k$
        \end{itemize}
        \item Schleifen: WHILE $x_i > 0$ DO ... HALT
    \end{itemize}
\end{KR}

\begin{example2}{Case distinction}
    \begin{lstlisting}[style=Pseudocode]
        M1: x0 = x3 + 0
        M2: IF x1 = 0 THEN GOTO M4
        M3: x0 = x2 +0
        M4: HALT
    \end{lstlisting}
\end{example2}