\paragraph{Reguläre Ausdrücke und Sprachen}
\begin{definition}{Reguläre Sprache}
    $A$ über dem Alphabet $\Sigma$ heisst regulär, falls
    \begin{itemize}
        \item $A=L(R)$ für einen regulären Ausdruck $R \in R A_{\Sigma}$ gilt.
    \end{itemize}
    $L\left(R_{1}\right)$ : Menge der ganzen Zahlen in Dezimaldarstellung
    \begin{itemize}
    \item $((-\mid \varepsilon)(1,2,3,4,5,6,7,8,9)(0,1,2,3,4,5,6,7,8,9) \mid 0) .0$
    \end{itemize}
\end{definition}    

\begin{concept}{Reguläre Ausdrücke}
    Wörter, die Sprachen beschreiben
    
    \begin{minipage}{0.3\linewidth}
        \begin{itemize}
            \item $\emptyset, \epsilon \in R A_{\Sigma}$
            \item $\Sigma \subset R A_{\Sigma}$
        \end{itemize}
    \end{minipage}
    \begin{minipage}{0.5\linewidth}
    \begin{itemize}
        \item $R \in R A_{\Sigma}$  $\Rightarrow\left(R^{*}\right) \in R A_{\Sigma}$
        \item $R, S \in R A_{\Sigma}$ $\Rightarrow(R S) \in R A_{\Sigma}$
        \item $R, S \in R A_{\Sigma}$ $\Rightarrow(R \mid S) \in R A_{\Sigma}$   
    \end{itemize}
    \end{minipage}

    $R A_{\Sigma}$: Sprache der Regulären Ausdrücke über $\{\emptyset, \epsilon, *,(),, \mid\} \cup \Sigma$ 
\end{concept}






\begin{KR}{Eigenschaften und Konventionen} $R A_{\Sigma}$\\
    
    \textbf{Priorisierung von Operatoren}
    \begin{itemize}
    \item (1) $*=$ Wiederholung $\rightarrow$ (2) Konkatenation $\rightarrow$ (3) $\mid=$ Oder
    \end{itemize}
    \textbf{Erweiterter Syntax}

        $
        \begin{array}{lcr}
            R^{+} = R (R^{*}) & R ?=(R \mid \epsilon) & \quad \left[R_{1}, \ldots, R_{k}\right]=R_{1}\left|R_{2}\right| \ldots \mid R_{k} 
        \end{array}
        $
\end{KR}