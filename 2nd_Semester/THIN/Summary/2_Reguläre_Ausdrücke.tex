\subsection*{Reguläre Ausdrücke}

\begin{concept}{Reguläre Ausdrücke}
    sind Wörter, die Sprachen beschreiben.\\
    Die Sprache $R A_{\Sigma}$ der Regulären Ausdrücke über einem Alphabet $\Sigma$ ist wie folgt definiert:

    \begin{itemize}
    \item $\emptyset, \epsilon \in R A_{\Sigma}$
    \item $\quad \Sigma \subset R A_{\Sigma}$
    \item $\quad R \in R A_{\Sigma}$  $\Rightarrow\left(R^{*}\right) \in R A_{\Sigma}$
    \item $R, S \in R A_{\Sigma}$ $\Rightarrow(R S) \in R A_{\Sigma}$
    \item $R, S \in R A_{\Sigma}$ $\Rightarrow(R \mid S) \in R A_{\Sigma}$   
    \end{itemize}
\end{concept}

\begin{example}
    Für jeden regulären Ausdruck $R \in R A_{\Sigma}$ definieren wir die Sprache $L(R)$ von $R$ wie folgt:
    \begin{itemize}
        \item Leere Sprache: $L(\emptyset)=\emptyset$
        \item Sprache, die nur das leere Wort enthält: $L(\varepsilon)=\{\varepsilon\}$
        \item Beschreibt die Sprache $\{a\}$: $L(a)=\{a\} \quad \forall a \in \Sigma$
        \item Kombiniert die Wörter von R: $L(R^{*})=L(R)^{*}$
        \item Verkettung von Wörtern (R = prefix): $L(R S)=L(R) \circ L(S)$
        \item Wörter die in R oder S beschrieben werden: $L(R \mid S)=L(R) \cup L(S)$
    \end{itemize}
\end{example}

\begin{definition}{Reguläre Sprache}\\
    Eine Sprache $A$ über dem Alphabet $\Sigma$ heisst regulär, falls
    \begin{itemize}
        \item $A=L(R)$ für einen regulären Ausdruck $R \in R A_{\Sigma}$ gilt.
    \end{itemize}

    Beispiele
    \begin{itemize}
    \item $\quad R_{1}=a^{*} b \quad L\left(R_{1}\right)=\{b, a b, a a b, a a a b, \ldots\}$
    \item $R_{2}=(a a)^{*} b^{*} a b a \quad L\left(R_{2}\right)=\{a b a, b a b a, a a a b a, a a b a b a, \ldots\}$
    \item $R_{3}=(a \mid a b)^{*} \quad L\left(R_{3}\right)=\{\varepsilon, a, a b, a a, a b a b, \ldots\}$
    \end{itemize}

    $L\left(R_{1}\right)$ : Menge der ganzen Zahlen in Dezimaldarstellung
    \begin{itemize}
    \item $((-\mid \varepsilon)(1,2,3,4,5,6,7,8,9)(0,1,2,3,4,5,6,7,8,9) \mid 0) .0$
    \end{itemize}
\end{definition}    


\begin{KR}{Eigenschaften und Konventionen}
    Die Menge $R A_{\Sigma}$ über dem Alphabet $\Sigma$ ist eine Sprache über dem Alphabet 
    $$\{\emptyset, \epsilon, *,(),, \mid\} \cup \Sigma$$

    \textbf{Priorisierung von Operatoren}
    \begin{itemize}
    \item (1) $*=$ Wiederholung $\rightarrow$ (2) Konkatenation $\rightarrow$ (3) $\mid=$ Oder
    \end{itemize}

    \textbf{Beispiele}
    \begin{itemize}
    \item $(a a)^{*} b^{*} a b a=(a a)^{*} b^{*} a b a$
    \item $(a b)|(b a) \quad=a b| b a$
    \item $a(b(b a))|b \quad=a b b a| b$
    \end{itemize}

    \textbf{Erweiterte Syntax}
    \begin{itemize}
    \item $R^{+}=R\left(R^{*}\right)$
    \item $R ?=(R \mid \epsilon)$
    \item $\left[R_{1}, \ldots, R_{k}\right]=R_{1}\left|R_{2}\right| \ldots \mid R_{k}$
    \end{itemize}
\end{KR}