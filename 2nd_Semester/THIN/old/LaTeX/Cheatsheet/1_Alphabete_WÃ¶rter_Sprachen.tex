\section*{Alphabete, Wörter, Sprachen}

\begin{definition}{Alphabete} endliche, nichtleere Mengen von Symbolen.
\begin{itemize}
  \item $\Sigma_{\text {Bool }}=\{0,1\} \quad$ Boolsches Alphabet
\end{itemize}
\textbf{Keine Alphabete}: $\mathbb{N}, \mathbb{R}, \mathbb{Z}$ usw. (unendliche Mächtigkeit)
\end{definition}

\begin{definition}{Wort}
    endliche Folge von Symbolen eines bestimmten Alphabets.
\end{definition}

\begin{definition}{Schreibweisen}
    $|\omega|=$ Länge eines Wortes

    $|\omega|_{x}=$ Häufigkeit eines Symbols $x$ in einem Wort

    $\omega^{R}=$ Spiegelwort/Reflection zu $\omega$
\end{definition}

\begin{definition}{Teilwort (Infix)}
    $v$ ist ein Teilwort (Infix) von $\omega$ ist, wenn $\omega=x v y$.

    $\omega \neq v \rightarrow \text { Echtes Teilwort }$, Präfix = Anfang, Suffix = Ende
\end{definition}

\begin{definition}{Mengen von Wörtern}
    $\Sigma^{k}=$ Wörter der Länge $\boldsymbol{k}$ über Alphabet $\Sigma$
    \begin{itemize}
        \item $\Sigma^{*}=\underbrace{\Sigma^{0}}_{1} \cup \underbrace{\Sigma^{1}} \cup \underbrace{\Sigma^{2}} \cup \underbrace{\Sigma^{3}} \cdots \quad$ Kleensche Hülle
        \item $\Sigma^{+}=\underbrace{\Sigma^{1}}_{2} \cup \underbrace{\Sigma^{2}}_{4} \cup \underbrace{\Sigma^{3}}_{8} \ldots=\Sigma^{*} \backslash\{\varepsilon\} \quad$ Positive Hülle
      \end{itemize}
\end{definition}

\begin{remark}
    $\varepsilon$ Leeres Wort (über jedem Alphabet) $\quad \Sigma^{0}=\{\varepsilon\}$
\end{remark}

\begin{definition}{Konkatenation}
    $=$ Verkettung von zwei beliebigen Wörtern $x$ und $y$

    $
    x \circ y=x y:=\left(x_{1}, x_{2} \ldots x_{n}, y_{1}, y_{2} \ldots y_{m}\right)
    $
\end{definition}

\begin{definition}{Wortpotenzen} Sei $x$ ein Wort über einem Alphabet $\Sigma$\\
    \begin{minipage}{0.2\linewidth}
        \begin{itemize}
            \item $x^{0}=\varepsilon$
        \end{itemize}
    \end{minipage}
    \begin{minipage}{0.5\linewidth}
        \begin{itemize}
            \item $x^{n+1}=x^{n} \circ x=x^{n} x$
        \end{itemize}
    \end{minipage}
\end{definition}

\begin{definition}{Sprache}
    über Alphabet $\Sigma=$ Teilmenge $L \subseteq \Sigma^{*}$ von Wörtern
    \begin{itemize}
    \item $\Sigma_{1} \subseteq \Sigma_{2} \wedge L$ Sprache über $\Sigma_{1} \rightarrow L$ Sprache über $\Sigma_{2}$
    \item $\Sigma^{*}$ Sprache über jedem Alphabet $\Sigma$
    \item \{\}$=\emptyset$ ist die leere Sprache
    \end{itemize}

    Kleenesche Hülle $A^{*}$ von A: $\{\varepsilon\} \cup A \cup A A \cup A A A \cup \ldots$\\
    Konkatenation von A und B: $A B=\{u v \mid u \in A \text { und } v \in B\}$\\
    \textcolor{pink}{$w = uv \in \Sigma_A \cup \Sigma_B$} NUR wenn $u \in \Sigma_A$ und $v \in \Sigma_B$!!
\end{definition}