\graphicspath{{images/}}

\section*{Berechnungsmodelle}

\begin{definition}{Turing-berechenbar} = kann von Turing-Maschine gelöst werden
    
    Turing-berechenbare Funktion $T: \Sigma^* \rightarrow \delta^*$

    $T(\omega) = \begin{cases}
        u & \text{falls T auf } \omega \in \Sigma^* \text{ angesetzt, nach endlich vielen}\\ 
        &\text{Schritten mit u auf dem Band anhält}\\
        \uparrow & \text{falls T bei Input } \omega \in \Sigma^* \text{ nicht hält}
    \end{cases}$
\end{definition}

\begin{remark}
    {\footnotesize
    $\forall$ algorithmisch lösbare Problem ist turing-berechenbar $\Rightarrow$ Computer $\equiv$ TM
    }
\end{remark}

\begin{theorem}{Primitiv rekursive Grundfunktionen} $\forall n \in \mathbb{N}$, $\forall k \in \mathbb{N}$ {\footnotesize (k = Konstante)}
    
    \vspace{1mm}

    n-stellige konstante Funktion: $c_k^n = \mathbb{N}^n \rightarrow \mathbb{N} \text{ mit } c_k^n (x_1, ... , x_n) = k$

    \vspace{1mm}

    Nachfolgerfunktion: $\eta : \mathbb{N} \rightarrow \mathbb{N} \text{ mit } \eta (x) = x + 1$

    \vspace{1mm}
    
    n-stellige Projektion auf die k-te Komponente: 

    \vspace{1mm}

    $\quad \pi_k^n : \mathbb{N}^n \rightarrow \mathbb{N} \text{ mit } \pi_k^n (x_1, ... ,x_k,..., x_n) = x_k$ {\footnotesize $\quad$ ($1 < k < n$)}
    
    {\small n = Anzahl der Argumente, k = Position des Arguments}
\end{theorem}

\begin{formula}{Primitive Rekursion von 2 FUnktionen}
    HEKLP
\end{formula}

\begin{example2}{Primitiv rekursive Funktionen} 

    \begin{minipage}{0.4\linewidth}
        \begin{itemize}
            \item set zero: $c_0^1(x) = 0$
            \item add(0, x): $\pi_1^1(x) = x$
            \item add(1, x): $\eta(x) = x + 1$
            \item add(2, x): \\$\eta(\eta(x)) = x + 2$
        \end{itemize}
    \end{minipage}
    \begin{minipage}{0.6\linewidth}
        \begin{itemize}
            \item $c_5^4 : \N^4 \rightarrow \N$ mit $c_5^4(x_1, x_2, x_3, x_4) = 5$
            \item $\pi_1^3 : \N^3 \rightarrow \N$ mit $\pi_1^3(x_1, x_2, x_3) = x_1$
            \item $\pi_1^1 : \N \rightarrow \N$ mit $\pi_1^1(x) = x$
            \item $\pi_5^5 : \N^5 \rightarrow \N$ \\ mit $\pi_5^5(x_1, x_2, x_3, x_4, x_5) = x_5$	
        \end{itemize}
    \end{minipage}

    \vspace{3mm}

    Vorgängerfunktion: $p(x) =$ $0$ falls $x = 0$; $x - 1$ sonst

    \vspace{1mm}

    Addition: $add(x, y) =$ $y$ falls $x = 0$; $add(x - 1, y) + 1$ sonst

    add(0, y) = $\pi_2^2(y) = y$\\
    add(x + 1, y) = $\eta(\pi_1^3(add(x, y), x, y))$ = $add(x, y) + 1$

    \vspace{1mm}

    Subtraktion: $sub(x, y) =$ $0$ falls $y = 0$; $sub(x - 1, y - 1)$ sonst

    \vspace{1mm}

    Multiplikation: $mult(x, y) =$ $0$ falls $y = 0$; $mult(x, y - 1) + x$ sonst
     
    mul(0, y) = $c_0^1(y) = 0$\\
    mul(x + 1, y) = $add(mul(x, y), \pi_2^2(x,y))$ = $add(mul(x, y),y)$

    \vspace{1mm}

    Potenz: $exp(x, y) =$ $1$ falls $y = 0$; $mult(exp(x, y - 1), x)$ sonst 
\end{example2}

\begin{minipage}
    {0.62\linewidth}
\begin{concept}{LOOP{,} WHILE{,} GOTO}

        Zuweisung: $xi = xj + c$ oder $xi = xj - c$\\
        Return Wert $= x_0$\\
        Nicht gleichmächtig wie TM
\end{concept}
\end{minipage}
\begin{minipage}{0.34\linewidth}
    {\small
    Variablen: $x0, x1, x2, ...$\\
Konstanten: $0, 1, 2, ...$\\
Operationszeichen: $+, -$\\
Trennzeichen: $;$}
\end{minipage}

\begin{minipage}{0.5\linewidth}
    \begin{KR}{Loop (primitiv-rekursiv)}

            Schlüsselwörter: Loop, Do, End

            Sequenzen: $P$ und $Q \rightarrow P; Q$

            Schleifen: Loop $x$ do $P$ ... End
    \end{KR}
    \end{minipage}
    \begin{minipage}{0.5\linewidth}
    \begin{example} add(x1, x2) = x0

\begin{lstlisting}[style=Pseudocode, aboveskip=-0.05\baselineskip, belowskip=-0.5\baselineskip]
Loop x1 Do
    x2 = x2 + 1
End;
x0 = x2 + 0
\end{lstlisting}
    \end{example}
\end{minipage}

\begin{minipage}{0.5\linewidth}
    \begin{KR}{While (Turing vollständig)}

        Erweiterung der Sprache Loop

        While $xi > 0$ Do ... End
    \end{KR}
    \begin{example} mul(x1, x2) = x0 

\begin{lstlisting}[style=Pseudocode, aboveskip=-0.05\baselineskip, belowskip=-0.5\baselineskip]
While x1 > 0 Do
    x1 = x1 - 1;
    Loop x2 Do
        x0 = x0 + 1
    End
End
\end{lstlisting}
\end{example}
\end{minipage}
\begin{minipage}{0.5\linewidth}
\begin{KR}{GoTo (Turing vollständig)}\\
        Schlüsselwörter: Goto, If, Else
        
        Marker Mk: M1, M2, ...
        
        Sprunganweisung:\\ If $xi = c$ Then Goto $Mk$ 
\end{KR}
\begin{example} case distinction (x1, x2, x3) = x0
\begin{lstlisting}[style=Pseudocode, aboveskip=-0.05\baselineskip, belowskip=-0.5\baselineskip]
M1: x0 = x3 + 0;
M2: If x1 = 0 Then Goto M4;
M3: x0 = x2 + 0;
M4: Halt
\end{lstlisting}
\end{example}
\end{minipage}

\paragraph*{Grundfunktionen LOOP und WHILE}

\begin{minipage}{0.5\linewidth}
\begin{example2}{Maximum} max(x1, x2)
\begin{lstlisting}[style=Pseudocode, aboveskip=-0.05\baselineskip, belowskip=-0.5\baselineskip]
x0 = x2 + 0;
x3 = x1 - x2;
Loop x3 Do
    x0 = x1 + 0
End
\end{lstlisting}
\end{example2}

\begin{example2}{Addition} x1 + x2
\begin{lstlisting}[style=Pseudocode, aboveskip=-0.05\baselineskip, belowskip=-0.5\baselineskip]
While x1 > 0 Do
    x2 = x2 + 1;
    x1 = x1 - 1
End;
x0 = x2 + 0
\end{lstlisting}    
\end{example2}

\begin{example2}{Absolute Difference} |x1 - x2|
\begin{lstlisting}[style=Pseudocode, aboveskip=-0.05\baselineskip, belowskip=-0.5\baselineskip]
x4 = x2 + 0;
x3 = x1 - x2;
Loop x3 Do
    x4 = x1 + 0
End;
x5 = x2 + 0;
x3 = x2 - x1;
Loop x3 Do
    x5 = x1 + 0
End;
x0 = x4 - x5
\end{lstlisting}
\end{example2}
\end{minipage}
\begin{minipage}{0.5\linewidth}
\begin{example2}{Minimum} min(x1, x2)
\begin{lstlisting}[style=Pseudocode, aboveskip=-0.05\baselineskip, belowskip=-0.5\baselineskip]
x0 = x2 + 0;
x3 = x2 - x1;
Loop x3 Do
    x0 = x1 + 0
End
\end{lstlisting}
\end{example2} 
\begin{example2}{Division} x1 / x2 + remainder
\begin{lstlisting}[style=Pseudocode, aboveskip=-0.05\baselineskip, belowskip=-0.5\baselineskip]
x3 = x1 - x2;
x3 = x3 + 1;
While x3 > 0 Do
    x3 = x3 - x2;
    x0 = x0 + 1
End
\end{lstlisting}
\end{example2}

\begin{example2}{Fibonacci} fib(x1)
\begin{lstlisting}[style=Pseudocode, aboveskip=-0.05\baselineskip, belowskip=-0.5\baselineskip]
x2 = 0 + 0;
x0 = 1 + 0;
While x1 > 0 Do
    x3 = x0 + 0;
    x0 = x0 + x2;
    x2 = x3 + 0;
    x1 = x1 - 1
End
\end{lstlisting}
\end{example2}
\end{minipage}

\begin{concept}{Unvollständigkeit} 

    \begin{minipage}{0.5\linewidth}   
    Ackermannfunktion $a: \N^2 \rightarrow \N$ 
    
    a(0,m) = m + 1

    a(n+1,0) = a(n,1)

    a(n+1,m+1) = a(n,a(n+1,m))
    \end{minipage}
    \begin{minipage}{0.49\linewidth}
    {\footnotesize wächst schneller als jede primitiv rekursive Funktion\\
    total (überall definiert)\\
    $\Rightarrow$ nicht primitiv rekursiv\\
    aber: turing-berechenbar}
    \end{minipage}    
\end{concept}

\begin{example2}{Ackermannfunktion} berechne $a(2,1)$

    $
    \begin{aligned}
    a(2,1) & =a(1, a(2,0))=a(1, a(1,1))=a(1, a(0, a(1,0))) \\
    & =a(1, a(0, a(0,1)))=a(1, a(0,2))=a(1,3)=a(0, a(1,2)) \\
    & =a(0, a(0, a(1,1)))=a(0, a(0, a(0, a(1,0)))) \\
    & =a(0, a(0, a(0, a(0,1))))=a(0, a(0, a(0,2))) \\
    & =a(0, a(0,3))=a(0,4)=5
    \end{aligned}
    $
\end{example2}

\begin{example2}{Beweis dass Ackermannfunktion nicht primitiv rekursiv}

    Definiere zu jeder primitiv rekursiven Funktion $f$ \\eine Funktion $m(n, f)=\max \left\{f(\vec{x}) \mid \sum \vec{x} \leq n\right\}$
    
    Zeige, dass $\exists k \in \N$ für jede primitiv rekursive Funktion $f$, so dass $\forall n \geq k$  $m(n, f)<a(k, n)$ gilt
    
    Definiere $a_1(x)=a(x, x)$. Wäre $a_1$ primitiv rekursiv, dann gäbe es ein $k$ mit $m\left(n, a_1\right)<a(k, n)$ für $n \geq k$
    
    Insbesondere gilt also für $n=k$ :
    $$
    \max \left\{a_1(x) \mid x \leq k\right\}=m\left(k, a_1\right)<a(k, k)=a_1(k) .
    $$
    Also kann $a_1$ und somit $a$ nicht primitiv rekursiv sein. 
\end{example2}

\begin{concept}{LOOP-Interpreter} nicht LOOP-berechenbar, aber turing-berechenbar

    Sei x Code eines Programms P: für jeden Input y soll $I(x,y) = P(y)$ gelten {\footnotesize wenn P(y) definiert}
\end{concept}









