\subsection*{Partielle Integration}

\begin{example}
$$
\begin{aligned}
\int \ln (x) \cdot x^{2} \mathrm{~d} x & =\ln (x) \cdot \frac{x^{3}}{3}-\int \frac{1}{x} \cdot \frac{x^{3}}{3} \mathrm{~d} x \\
& =\ln (x) \cdot \frac{x^{3}}{3}-\int \frac{x^{2}}{3} \mathrm{~d} x \\
& =\ln (x) \cdot \frac{x^{3}}{3}-\frac{x^{3}}{9}+C \quad(C \in \mathbb{R})
\end{aligned}
$$

\end{example}

\begin{example}
    Mit einer ersten partiellen Integration erhält man

$$
\int x^{2} \cdot e^{-x} \mathrm{~d} x=x^{2} \cdot\left(-e^{-x}\right)-\int 2 x \cdot\left(-e^{-x}\right) \mathrm{d} x=-x^{2} \cdot e^{-x}+2 \int x \cdot e^{-x} \mathrm{~d} x
$$

Eine zweite partielle Integration ergibt

$$
\int x \cdot e^{-x} \mathrm{~d} x=x \cdot\left(-e^{-x}\right)-\int 1 \cdot\left(-e^{-x}\right) \mathrm{d} x=-x \cdot e^{-x}-e^{-x}+C
$$

Insgesamt ergibt sich

$$
\int x^{2} \cdot e^{-x} \mathrm{~d} x=-x^{2} \cdot e^{-x}-2 x \cdot e^{-x}-2 e^{-x}+C=-\left(x^{2}+2 x+2\right) \cdot e^{-x}+C
$$
\end{example}


\begin{example}
    Wir integrieren zweimal partiell und erhalten:
    $$
\begin{aligned}
\int e^{3 x} \cdot\left(x^{2}+7\right) \mathrm{d} x & =\frac{1}{3} e^{3 x}\left(x^{2}+7\right)-\int \frac{1}{3} e^{3 x} \cdot 2 x \mathrm{~d} x \\
& =\frac{1}{3} e^{3 x}\left(x^{2}+7\right)-\left(\frac{1}{9} e^{3 x} \cdot 2 x-\int \frac{1}{9} e^{3 x} \cdot 2 \mathrm{~d} x\right) \\
& =\frac{1}{3} e^{3 x}\left(x^{2}+7\right)-\frac{1}{9} e^{3 x} \cdot 2 x+\frac{2}{27} e^{3 x}+C \\
& =\frac{1}{3} e^{3 x}\left(x^{2}-\frac{2}{3} x+\frac{65}{9}\right)+C
\end{aligned}
$$
\end{example}

\begin{example}
    Eine erste partielle Integration ergibt

$$
\int e^{x} \cdot \cos (x) \mathrm{d} x=e^{x} \cdot \sin (x)-\int e^{x} \cdot \sin (x) \mathrm{d} x
$$

Eine weitere partielle Integration ergibt

$$
\int e^{x} \sin (x) \mathrm{d} x=e^{x} \cdot(-\cos (x))-\int e^{x} \cdot(-\cos (x)) \mathrm{d} x
$$

Damit erhalten wir insgesamt

$$
\int e^{x} \cdot \cos (x) \mathrm{d} x=e^{x} \cdot \sin (x)+e^{x} \cdot \cos (x)-\int e^{x} \cdot \cos (x) \mathrm{d} x
$$

Dies ist eine Gleichung, die nach dem gesuchten Integral aufgelöst werden kann, und man erhält

$$
\int e^{x} \cdot \cos (x) \mathrm{d} x=\frac{1}{2} e^{x} \cdot(\sin (x)+\cos (x))+C
$$

\end{example}

\begin{example}
    a. Wir erhalten mit der Formel $\sin ^{2}(x)=\frac{1}{2}(1-\cos (2 x))$
    $$
\int_{0}^{\pi} \sin ^{2}(x) \mathrm{d} x=\int_{0}^{\pi}\left(\frac{1}{2}(1-\cos (2 x))\right) \mathrm{d} x=\left.\left(\frac{x}{2}-\frac{1}{4} \sin (2 x)\right)\right|_{0} ^{\pi}=\frac{\pi}{2}
$$

b. Wir integrieren zuerst partiell:

$$
\int_{0}^{\pi} \sin ^{2}(x) \mathrm{d} x=-\left.\cos (x) \sin (x)\right|_{0} ^{\pi}+\int_{0}^{\pi} \cos ^{2}(x) \mathrm{d} x=\int_{0}^{\pi} \cos ^{2}(x) \mathrm{d} x
$$

Einsetzen von $\cos ^{2}(x)=1-\sin ^{2}(x)$ führt dann $z u$

$$
\int_{0}^{\pi} \sin ^{2}(x) \mathrm{d} x=\int_{0}^{\pi} \cos ^{2}(x) \mathrm{d} x=\int_{0}^{\pi}\left(1-\sin ^{2}(x)\right) \mathrm{d} x=\pi-\int_{0}^{\pi} \sin ^{2}(x) \mathrm{d} x
$$

Dies ist eine Gleichung, die nach dem gesuchten Integral aufgelöst werden kann, und man erhält

$$
\int_{0}^{\pi} \sin ^{2}(x) \mathrm{d} x=\frac{\pi}{2}
$$
\end{example}

\subsection*{Substitution}

\begin{example}

$$\int x^{2} \cdot \sqrt{1+x^{3}} \mathrm{~d} x$$

Substitution: $u(x)=1+x^{3}, \frac{\mathrm{d} u}{\mathrm{~d} x}=3 x^{2}, \mathrm{~d} x=\frac{\mathrm{d} u}{3 x^{2}}$. Berechnung:

$$
\begin{aligned}
\int x^{2} \cdot \sqrt{1+x^{3}} \mathrm{~d} x & =\int x^{2} \cdot \sqrt{u} \cdot \frac{\mathrm{d} u}{3 x^{2}}=\int \frac{1}{3} \cdot u^{\frac{1}{2}} \mathrm{~d} u=\frac{1}{3} \cdot \frac{u^{3 / 2}}{3 / 2}+C \\
& =\frac{2}{9} \cdot \sqrt{\left(1+x^{3}\right)^{3}}+C
\end{aligned}
$$

\end{example}

\begin{example}
    
$$\int \frac{1}{\sqrt[3]{1-t}} \mathrm{~d} t$$

Substitution: $u(t)=1-t, \frac{\mathrm{d} u}{\mathrm{~d} t}=-1, \mathrm{~d} t=-\mathrm{d} u$. Berechnung:

$$
\int \frac{1}{\sqrt[3]{1-t}} \mathrm{~d} t=\int\left(-u^{-1 / 3}\right) \mathrm{d} u=-\frac{3}{2} \cdot u^{2 / 3}+C=-\frac{3}{2} \cdot \sqrt[3]{(1-t)^{2}}+C
$$
\end{example}

\begin{example}
$$\int \frac{\mathrm{d} z}{z \cdot \ln (z)}$$

Substitution: $u(z)=\ln (z), \frac{\mathrm{d} u}{\mathrm{~d} z}=\frac{1}{z}, \mathrm{~d} z=z \cdot \mathrm{d} u$. Berechnung:

$$
\int \frac{\mathrm{d} z}{z \cdot \ln (z)}=\int \frac{z \cdot \mathrm{d} u}{z \cdot u}=\int \frac{\mathrm{d} u}{u}=\ln (|u|)+C=\ln (|\ln z|)+C
$$
\end{example}

\begin{example}
    
$$\int_{0}^{\pi} \cos ^{3}(x) \cdot \sin (x) \mathrm{d} x$$

Substitution: $u(x)=\cos (x), \frac{\mathrm{d} u}{\mathrm{~d} x}=-\sin (x)$. Berechnung:

$$
\int_{0}^{\pi} \cos ^{3}(x) \cdot \sin (x) \mathrm{d} x=\int_{1}^{-1}\left(-u^{3}\right) \mathrm{d} u=\int_{-1}^{1} u^{3} \mathrm{~d} u=\left[\frac{u^{4}}{4}\right]_{-1}^{1}=0
$$
\end{example}

\begin{example}
   
    $$\int_{0}^{1} \frac{\arctan (z)}{1+z^{2}} \mathrm{~d} z$$

Substitution: $u(z)=\arctan (z), \frac{\mathrm{d} u}{\mathrm{~d} z}=\frac{1}{1+z^{2}}$. Berechnung:

$$
\int_{0}^{1} \frac{\arctan (z)}{1+z^{2}} \mathrm{~d} z=\int_{0}^{\pi / 4} u \mathrm{~d} u=\left[\frac{u^{2}}{2}\right]_{0}^{\pi / 4}=\frac{(\pi / 4)^{2}}{2}=\frac{\pi^{2}}{32}
$$
\end{example}

\raggedcolumns

\subsection*{Partialbruchzerlegung}

\begin{example}

    Berechnen Sie das unbestimmte Integral

    $$
    \int \frac{2 x+4}{x^{2}+4 x-21} \mathrm{~d} x
    $$

    durch Partialbruchzerlegung und Substitution

    \tcblower

    Partialbruchzerlegung: Die Nullstellen von $x^{2}+4 x-21$ sind $x_{1}=-7$ und $x_{2}=3$, also haben wir den Ansatz

$$
\frac{2 x+4}{x^{2}+4 x-21}=\frac{A}{x+7}+\frac{B}{x-3}
$$

Daraus ergibt sich die Bedingung $A(x-3)+B(x+7)=2 x+4$, und durch Einsetzen von $x=3$ und $x=-7$ erhalten wir dann $A=B=1$. Die gesuchte Partialbruchzerlegung ist also

$$
\frac{2 x+4}{x^{2}+4 x-21}=\frac{1}{x+7}+\frac{1}{x-3}
$$

Wir können jetzt integrieren und erhalten

$$
\begin{aligned}
\int \frac{2 x+4}{x^{2}+4 x-21} \mathrm{~d} x & =\int \frac{1}{x+7} \mathrm{~d} x+\int \frac{1}{x-3} \mathrm{~d} x \\
& =\ln |x+7|+\ln |x-3|+C \\
& =\ln |(x+7)(x-3)|+C \\
& =\ln \left|x^{2}+4 x-21\right|+C
\end{aligned}
$$



Substitution $u=x^{2}+4 x-21$, mit $\mathrm{d} u=(2 x+4) \mathrm{d} x$ bzw. $\mathrm{d} x=\frac{\mathrm{d} u}{2 x+4}$ führt auf

$$
\int \frac{2 x+4}{x^{2}+4 x-21} \mathrm{~d} x=\int \frac{2 x+4}{u} \frac{\mathrm{d} u}{2 x+4}=\int \frac{\mathrm{d} u}{u}=\ln |u|+C
$$

Rücksubstitution ergibt

$$
\ln |u|+C=\ln \left|x^{2}+4 x-21\right|+C
$$

also insgesamt

$$
\int \frac{2 x+4}{x^{2}+4 x-21} \mathrm{~d} x=\ln \left|x^{2}+4 x-21\right|+C
$$

\end{example}

\begin{example}

$$\int \frac{5 x+11}{x^{2}+3 x-10} \mathrm{~d} x$$

Ansatz: $\frac{5 x+11}{x^{2}+3 x-10}=\frac{A}{x-2}+\frac{B}{x+5} \Rightarrow 5 x+11=A(x+5)+B(x-2)$

Bestimmung von $A$ und $B: x=2$ einsetzen $\Rightarrow A=3 ; \quad x=-5$ einsetzen $\Rightarrow B=2$ Berechnung des Integrals:

$$
\begin{aligned}
\int \frac{5 x+11}{x^{2}+3 x-10} \mathrm{~d} x & =\int \frac{3}{x-2}+\frac{2}{x+5} \mathrm{~d} x \\
& =3 \cdot \ln (|x-2|)+2 \cdot \ln (|x+5|)+C
\end{aligned}
$$

\end{example}

\begin{example}

    $$\int \frac{-9-y}{y^{2}-2 y-24} \mathrm{~d} y$$

    Ansatz: $\frac{-9-y}{y^{2}-2 y-24}=\frac{A}{y-6}+\frac{B}{y+4} \Rightarrow-9-y=A(y+4)+B(y-6)$

Bestimmung von $A$ und $B: y=6$ einsetzen $\Rightarrow A=-1.5 ; \quad y=-4$ einsetzen $\Rightarrow B=0.5$ Berechnung des Integrals:

$$
\begin{aligned}
\int \frac{-9-y}{y^{2}-2 y-24} \mathrm{~d} y & =\int \frac{-1.5}{y-6}+\frac{0.5}{y+4} \mathrm{~d} y \\
& =-1.5 \cdot \ln (|y-6|)+0.5 \cdot \ln (|y+4|)+C
\end{aligned}
$$
\end{example}


\raggedcolumns
\columnbreak