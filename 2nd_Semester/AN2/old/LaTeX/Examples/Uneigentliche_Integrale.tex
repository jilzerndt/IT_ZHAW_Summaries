\subsection*{Uneigentliche Integrale}


\begin{example}
    Berechnen Sie den Flächeninhalt, den die Kurven der drei Funktionen $y=e^{a x}, y=e^{-b x}$ und $y=0$ miteinander einschliessen $(a>0, b>0)$.

    Die gesuchte Fläche ist

$$
A=\int_{-\infty}^{0} e^{a x} \mathrm{~d} x+\int_{0}^{\infty} e^{-b x} \mathrm{~d} x=\frac{1}{a}+\frac{1}{b}
$$

\end{example}

\begin{example}
    Sei $a>0$ gegeben.

a. Für welches $c \in \mathbb{R}$ gilt

$$
\int_{c}^{\infty} e^{-a x} \mathrm{~d} x=1 ?
$$

b. Für welches $c \in \mathbb{R}$ gilt

$$
\int_{-\infty}^{c} e^{a x} \mathrm{~d} x=2 \quad ?
$$

\tcblower

a. Berechnung des uneigentlichen Integrals:

$$
\int_{c}^{\infty} e^{-a x} \mathrm{~d} x=\lim _{\lambda \rightarrow \infty}\left(\int_{c}^{\lambda} e^{-a x} \mathrm{~d} x\right)=\lim _{\lambda \rightarrow \infty}\left(\frac{1}{a}\left(-e^{-a \lambda}+e^{-a c}\right)\right)=\frac{1}{a} e^{-a c}
$$

Aus der Forderung $\int_{c}^{\infty} e^{-a x} \mathrm{~d} x=1$ ergibt sich also die Gleichung $\frac{1}{a} e^{-a c}=1$, aufgelöst nach $c$ erhalten wir die Lösung

$$
c=-\frac{\ln (a)}{a}
$$

b. Berechnung des uneigentlichen Integrals:

$$
\int_{-\infty}^{c} e^{-a x} \mathrm{~d} x=\lim _{\lambda \rightarrow-\infty}\left(\int_{\lambda}^{c} e^{a x} \mathrm{~d} x\right)=\lim _{\lambda \rightarrow-\infty}\left(\frac{1}{a}\left(e^{a c}-e^{a \lambda}\right)\right)=\frac{1}{a} e^{a c}
$$

Aus der Forderung $\int_{-\infty}^{c} e^{a x} \mathrm{~d} x=2$ ergibt sich also die Gleichung $\frac{1}{a} e^{a c}=2$, aufgelöst nach $c$ erhalten wir die Lösung

$$
c=\frac{\ln (2 a)}{a}
$$
\end{example}

\begin{example}
    Die Engelstrompete entsteht durch Rotation der Kurve von $f(x)=\frac{1}{x}$ um die $x$-Achse im Intervall $I=[1, \infty)$, d.h. es handelt sich um einen "uneigentlichen Rotationskörper".

a. Berechnen Sie das Volumen der Engelstrompete.

b. Stellen Sie die Mantelfläche der Engelstrompete als Integral dar.

\tcblower

a. Volumen des Rotationskörpers:

$$
V=\pi \int_{1}^{\infty}\left(\frac{1}{x}\right)^{2} \mathrm{~d} x=\pi \int_{1}^{\infty} \frac{1}{x^{2}} \mathrm{~d} x=\pi
$$

b. Mantelfäche des Rotationskörpers:

$$
M=2 \pi \int_{1}^{\infty} \frac{1}{x} \sqrt{1+\frac{1}{x^{4}}} \mathrm{~d} x
$$

Es kann durch einen Vergleich mit einem einfacheren Integral gezeigt werden, dass die Mantelfläche divergent ist:

$$
M=2 \pi \int_{1}^{\infty} \frac{1}{x} \underbrace{\sqrt{1+\frac{1}{x^{4}}}}_{>1} \mathrm{~d} x>2 \pi \int_{1}^{\infty} \frac{1}{x} \mathrm{~d} x=\infty
$$

Hier wird also ein endliches Volumen von einer unendlichen Fläche umschlossen!
\end{example}

\begin{example}
    Bestimmen Sie die gesamte Fläche, die die Kurve der Funktion

$$
y=\frac{2}{x(x+1)}
$$

mit der $x$-Achse über dem Intervall $[1, \infty)$ einschliesst.

\tcblower

Partialbruchzerlegung der gegebenen Funktion: $\frac{2}{x(x+1)}=\frac{2}{x}-\frac{2}{x+1}$. Berechnung der gesuchten Fläche:

$$
\begin{aligned}
A & =\int_{1}^{\infty}\left(\frac{2}{x}-\frac{2}{x+1}\right) \mathrm{d} x \\
& =\lim _{\lambda \rightarrow \infty}\left(\int_{1}^{\lambda}\left(\frac{2}{x}-\frac{2}{x+1}\right) \mathrm{d} x\right) \\
& =2 \cdot \lim _{\lambda \rightarrow \infty}\left(\left.\ln \left(\frac{x}{x+1}\right)\right|_{1} ^{\lambda}\right) \\
& =2 \cdot \lim _{\lambda \rightarrow \infty}\left(\ln \left(\frac{\lambda}{\lambda+1}\right)-\ln \left(\frac{1}{2}\right)\right) \\
& =2 \cdot \ln (2) \\
& \approx 1.38629
\end{aligned}
$$

Bemerkung: Die Fläche kann nicht als $A=\int_{1}^{\infty} \frac{2}{x} \mathrm{~d} x-\int_{1}^{\infty} \frac{2}{x+1} \mathrm{~d} x$ berechnet werden, da diese Teilintegrale beide divergent sind.

\end{example}



\begin{example}
    Bestimmen Sie die gesamte Fläche, die die Kurve der Funktion

$$
y=(x-1) \cdot e^{-x}
$$

mit der $x$-Achse über dem Intervall $[0, \infty)$ einschliesst. Hinweis: Es gilt $\lim _{\lambda \rightarrow \infty} \lambda e^{-\lambda}=0$.

\tcblower

Die Funktion $f(x)$ hat im Intervall $[0, \infty)$ bei $x=1$ eine Nullstelle. Deshalb zerfällt die gesuchte Fläche in zwei Teilflächen, welche separat berechnet werden müssen, nämlich

$$
A=\left|\int_{0}^{1}(x-1) e^{-x} \mathrm{~d} x\right|+\left|\int_{1}^{\infty}(x-1) e^{-x} \mathrm{~d} x\right|
$$

Das unbestimmte Integral von $f(x)$ ist (partielle Integration):

$$
\begin{aligned}
\int(x-1) e^{-x} \mathrm{~d} x & =-(x-1) e^{-x}-\int\left(-e^{-x}\right) \mathrm{d} x=-(x-1) e^{-x} \int e^{-x} \mathrm{~d} x \\
& =-(x-1) e^{-x}-e^{-x}+C=-x e^{-x}+C
\end{aligned}
$$

Berechnung der Teilintegrale:

$$
\begin{aligned}
& \int_{0}^{1}(x-1) e^{-x} \mathrm{~d} x=-\left.x e^{-x}\right|_{0} ^{1}=-\frac{1}{e} \\
& \int_{1}^{\infty}(x-1) e^{-x} \mathrm{~d} x=\lim _{\lambda \rightarrow \infty}\left(\int_{1}^{\lambda}(x-1) e^{-x} \mathrm{~d} x\right)=\lim _{\lambda \rightarrow \infty}\left(\left.\left(-x e^{-x}\right)\right|_{1} ^{\lambda}\right)=\frac{1}{e}-\lim _{\lambda \rightarrow \infty} \lambda e^{-\lambda}
\end{aligned}
$$

Es gilt $\lim _{\lambda \rightarrow \infty} \lambda e^{-\lambda}=0$ (vgl. Hinweis). Also folgt $\int_{1}^{\infty}(x-1) e^{-x} \mathrm{~d} x=\frac{1}{e}$. Insgesamt ist also die gesuchte Fläche

$$
A=\left|-\frac{1}{e}\right|+\left|\frac{1}{e}\right|=2 \cdot \frac{1}{e}=\frac{2}{e} \approx 0.7358
$$
    
\end{example}
