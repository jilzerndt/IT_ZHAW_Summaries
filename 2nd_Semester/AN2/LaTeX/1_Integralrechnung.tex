\section{Integralrechnen}
\subsection{Stammfunktionen}	
    \begin{lemma}{Integraltabelle}\\
    \resizebox*{\textwidth}{!}{	
	\def\arraystretch{1.5} 
	\begin{tabular}{|c|c|c|}
		\hline
		Funktion | f(x) & Ableitung | f'(x) & Integral | F(x)\\
		\hline
		\(1\) & \(0\) & \(x+C\) \\
		\hline
		\(x\) & \(1\) & \(\frac{1}{2}x^2+C\) \\
		\hline
		\(\frac{1}{x}\) & \(-\frac{1}{x^2}\) & \(\ln|x|+C\) \\
		\hline
		\(x^a \: with \: a \: \in \: \mathbb{R}\) & \(ax^{a-1}\) & \(\frac{x^{a+1}}{a+1}+C\) \\
		\hline
		\(\sin(x)\) & \(\cos(x)\) & \(-\cos(x)+C\) \\
		\hline
		\(\cos(x)\) & \(-\sin(x)\) & \(\sin(x)+C\) \\
		\hline
		\(\tan(x)\) & \(1+\tan^2{(x)=\frac{1}{\cos^2{(x)}}}\) & \(-\ln|\cos(x)|+C\) \\
		\hline
		\(\cot(x)\) & \(-1-\cot^2{(x)}=-\frac{1}{\sin^2{(x)}}\) & \(\ln(\sin(x))+C\) \\
		\hline
		\(e^x\) & \(e^x\) & \(e^x+C\) \\
		\hline
		\(a^x\) & \(\ln(a)\cdot a^x\) & \(\frac{a^x}{\ln(a)}+C\) \\
		\hline
		\(\ln(x)\) & \(\frac{1}{x}\) & \(x\ln(x)-x+C\) \\
		\hline
		\(\log_a(x)\) & \(\frac{1}{x\ln(a)}\) & \(x\log_a(x)-\frac{x}{\ln(a)}+C\) \\
		\hline
		\(\arcsin(x)\) & \(\frac{1}{\sqrt{1-x^2}}\) & \(x\arcsin(x)+\sqrt{1-x^2}+C\) \\
		\hline
		\(\arccos(x)\) & \(-\frac{1}{\sqrt{1-x^2}}\) & \(x\arccos(x)-\sqrt{1-x^2}+C\) \\
		\hline
		\(\arctan(x)\) & \(\frac{1}{1+x^2}\) & \(x\arctan(x)-\frac{1}{2}\ln(1+x^2)+C\) \\
		\hline
	\end{tabular}
    }
    \end{lemma}
    \begin{lemma}{Integrale von Linearkombinationen}\\
	Gegeben: 
	\[\int{f(x)dx} = F(x)+C, \quad  \int{g(x)dx} = G(x)+C\] 
    Das unbestimmte Integral der Linearkombination \(\lambda_1f(x) + \lambda_2g(x)\) ist:
    \[\int{(\lambda_1f(x)+\lambda_2g(x))} = \lambda_2F(x)+\lambda_2G(x)+C \quad (\lambda_1,\lambda_2 \in \mathbb{R} )\]
\end{lemma}
\begin{lemma}{Integral von verschobenen Funktionene}\\
	Gegeben:
	\[\int{f(x)dx} = F(x) + C \]
	Das unbestimte integral um Betrag k in x-Richtung verschoben ist:
	\[\int{f(x-k)dx}= F(x-k)+C \quad (k \in \mathbb{R}) \]
\end{lemma}
\begin{lemma}{Integrale von gestreckten Funktionen}\\
    Gegeben:
    \[\int{f(x)dx} = F(x)+C \]
    Das unbestimmte Integral um Faktor k in x-Richtung gestreckt ist:
    \[\int{f(k\cdot x)dx}= \frac{1}{k}F(k\cdot x)+C \quad (k\neq0 )\]
\end{lemma}
\begin{theorem}{Partielle Integration}\\
    \[\int{u'(x)v(x)dx} = u(x)\cdot v(x) - \int{u(x),v'(x)dx} \]	
\end{theorem}
\begin{theorem}{Partialbruchzerlegung}\\
    \begin{itemize}
	\item Bestimmung der Nullstellen \(x_1,x_2, \ldots ,x_n \) des Nennerpolynoms \(q(x)\) mit Vielfachheiten (einfache Nullstelle, doppelte usw)
	\item Zuordnen eines 
    \end{itemize}
\end{theorem}

















