\section{Integralrechnen}
\subsection{Stammfunktionen}
\begin{lemma}{Integraltabelle}\\
	\resizebox*{\textwidth}{!}{
		\def\arraystretch{1.5}
		\begin{tabular}{|c|c|c|}
			\hline
			Funktion | f(x)                           & Ableitung | f'(x)                         & Integral | F(x)                         \\
			\hline
			\(1\)                                     & \(0\)                                     & \(x+C\)                                 \\
			\hline
			\(x\)                                     & \(1\)                                     & \(\frac{1}{2}x^2+C\)                    \\
			\hline
			\(\frac{1}{x}\)                           & \(-\frac{1}{x^2}\)                        & \(\ln|x|+C\)                            \\
			\hline
			\(x^a \: with \: a \: \in \: \mathbb{R}\) & \(ax^{a-1}\)                              & \(\frac{x^{a+1}}{a+1}+C\)               \\
			\hline
			\(\sin(x)\)                               & \(\cos(x)\)                               & \(-\cos(x)+C\)                          \\
			\hline
			\(\cos(x)\)                               & \(-\sin(x)\)                              & \(\sin(x)+C\)                           \\
			\hline
			\(\tan(x)\)                               & \(1+\tan^2{(x)=\frac{1}{\cos^2{(x)}}}\)   & \(-\ln|\cos(x)|+C\)                     \\
			\hline
			\(\cot(x)\)                               & \(-1-\cot^2{(x)}=-\frac{1}{\sin^2{(x)}}\) & \(\ln(\sin(x))+C\)                      \\
			\hline
			\(e^x\)                                   & \(e^x\)                                   & \(e^x+C\)                               \\
			\hline
			\(a^x\)                                   & \(\ln(a)\cdot a^x\)                       & \(\frac{a^x}{\ln(a)}+C\)                \\
			\hline
			\(\ln(x)\)                                & \(\frac{1}{x}\)                           & \(x\ln(x)-x+C\)                         \\
			\hline
			\(\log_a(x)\)                             & \(\frac{1}{x\ln(a)}\)                     & \(x\log_a(x)-\frac{x}{\ln(a)}+C\)       \\
			\hline
			\(\arcsin(x)\)                            & \(\frac{1}{\sqrt{1-x^2}}\)                & \(x\arcsin(x)+\sqrt{1-x^2}+C\)          \\
			\hline
			\(\arccos(x)\)                            & \(-\frac{1}{\sqrt{1-x^2}}\)               & \(x\arccos(x)-\sqrt{1-x^2}+C\)          \\
			\hline
			\(\arctan(x)\)                            & \(\frac{1}{1+x^2}\)                       & \(x\arctan(x)-\frac{1}{2}\ln(1+x^2)+C\) \\
			\hline
		\end{tabular}
	}
\end{lemma}
\begin{lemma}{Integrale von Linearkombinationen}\\
	Gegeben:
	\[\int{f(x)\mathrm{d}x} = F(x)+C, \quad  \int{g(x)\mathrm{d}x} = G(x)+C\]
	Das unbestimmte Integral der Linearkombination \(\lambda_1f(x) + \lambda_2g(x)\) ist:
	\[\int{(\lambda_1f(x)+\lambda_2g(x))} = \lambda_2F(x)+\lambda_2G(x)+C \quad (\lambda_1,\lambda_2 \in \mathbb{R} )\]
\end{lemma}
\begin{lemma}{Integral von verschobenen Funktionene}\\
	Gegeben:
	\[\int{f(x)\mathrm{d}x} = F(x) + C \]
	Das unbestimte integral um Betrag k in x-Richtung verschoben ist:
	\[\int{f(x-k)\mathrm{d}x}= F(x-k)+C \quad (k \in \mathbb{R}) \]
\end{lemma}
\begin{lemma}{Integrale von gestreckten Funktionen}\\
	Gegeben:
	\[\int{f(x)\mathrm{d}x} = F(x)+C \]
	Das unbestimmte Integral um Faktor k in x-Richtung gestreckt ist:
	\[\int{f(k\cdot x)\mathrm{d}x}= \frac{1}{k}F(k\cdot x)+C \quad (k\neq0 )\]
\end{lemma}
\begin{theorem}{Partielle Integration}\\
	\[\int{u'(x)v(x)\mathrm{d}x} = u(x)\cdot v(x) - \int{u(x),v'(x)\mathrm{d}x} \]
\end{theorem}
\begin{theorem}{Partialbruchzerlegung}\\
	\begin{itemize}
		\item Bestimmung der Nullstellen \(x_1,x_2, \ldots ,x_n \) des Nennerpolynoms \(q(x)\) mit Vielfachheiten
		      (einfache Nullstelle, doppelte usw)
		      \[Beispiel \: Integral: \int{\frac{1}{x^2-1}\mathrm{d}x} \]
		\item Zuordnen der Nullstellen \(x_k\)vom \(q(x)\) zu einem Partialbruch mit unbekannten Koeffizienten
		      \(A,B_1,B_2,\ldots\), \(1\le k\le n\):
		      \[f(x)=\underbrace{ \frac{A}{x-x_1}}_{einfache \: Nullstelle \: x_1} +\underbrace
			      {\frac{B_1}{x-x_2}+\frac{B_2}{(x-x_2)^2}}_{doppelte \: Nullstelle \: x_2}+\ldots  \]
		      \[Beispiel:\quad \frac{1}{x^2-1} = \frac{A}{x-1}+\frac{B}{x+1} \]
		\item Bestimmung der Koeffizienten: alles auf den Hauptnenner bringen, geignete x-Werte einsetzen
		      \[Beispiel: \frac{1}{x^2-1}=\frac{A(x+1)+B(x-1)}{x^2-1} \]
		      \[Beispiel: 1 = A(x+1)+B(x-1) \quad x=1\: bzw. \: x=-1 \]
		      \[B = -\frac{1}{2} \quad A=\frac{1}{2} \]
		\item Werte in Partialbruch einsetzen
		      \[\frac{1}{2}\cdot \frac{1}{x-1}-\frac{1}{2}\cdot \frac{1}{x+1} \]
		\item Integral der Partialbrüche berechnen
		      \[\int{\frac{1}{x^2-1}\mathrm{d}x}= \frac{1}{2}\cdot \int{\frac{1}{x-1}\mathrm{d}x}-\frac{1}{2}\cdot
			      \int{\frac{1}{x+1}\mathrm{d}x} \]
		      \[\int{\frac{1}{x^2-1}\mathrm{d}x}=\frac{1}{2}\cdot\ln{\abs{x-1}}-\frac{1}{2}\cdot\ln{\abs{x+1}}
			  +C=\frac{1}{2} \cdot\ln{\abs{\frac{x-1}{x+1}}}+C\]
	\end{itemize}
\end{theorem}
\begin{remark}{Bemerkung}\\
    Falls die rationale Funktion \( f(x)=\frac{r(x)}{s(x)} \) unecht gebrochen-rational ist, d.h. \(\rightarrow\)
    \( deg(r(x))\ge deg(s(x)) \) gilt: %TODO do this shit
\end{remark}
\begin{theorem}{Substitution unbestimmtes Integral}\\
    \begin{itemize}
	\item Aufstellen und Ableiten der Substitutionsglichungen:
	    \[u=g(x),\quad \frac{\mathrm{d}u}{\mathrm{d}x}=g'(x),\quad \mathrm{d}x = \frac{\mathrm{d}u}{g'(x)} \]
	\item Durchführen der Substitution \(u=g(x) \)	 und \(\mathrm{d}x=\frac{\mathrm{d}u}{g'(x)} \) in \\das  
	    integral \(\displaystyle\int{f(x)\mathrm{d}x}\):
	    \[\int{f(x)\mathrm{d}x}=\int{r(u)}{\mathrm{d}u} \]
	\item Berechnen des Integrals mit Variable u:
	    \[\int{r(u)\mathrm{d}u}=r(u)+C \]
	\item Rücksubstitution:
	    \[r(u)+C=r(g(x))+C \]
    \end{itemize}	
\end{theorem}

\begin{theorem}{Substitution bestimmtes Integral}\\
    \begin{itemize}
	\item Aufstellen und Ableiten der Substitutionsglichungen:
	    \[u=g(x),\quad \frac{\mathrm{d}u}{\mathrm{d}x}=g'(x),\quad \mathrm{d}x = \frac{\mathrm{d}u}{g'(x)} \]
	\item Durchführen der Substitution \(u=g(x) \)	 und \(\mathrm{d}x=\frac{\mathrm{d}u}{g'(x)} \) in \\das  
	    integral \(\displaystyle\int{f(x)\mathrm{d}x}\):
	    \[\int_a^b{f(x)\mathrm{d}x}=\int_{g(a)}^{g(b)}{r(u)}{\mathrm{d}u} \]
	\item Berechnen des Integrals mit Variable u:
	    \[\int_{g(a)}^{g(b)}{r(u)\mathrm{d}u}=r(u)\Big|_{g(a)}^{g(b)} \]

    \end{itemize}	
\end{theorem}













