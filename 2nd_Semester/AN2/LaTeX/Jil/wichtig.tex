\subsubsection*{Wichtige Ableitungen, Stammfunktionen und Grenzwerte}

\begin{formula}{Integraltabelle}\\
    \resizebox*{1.02\linewidth}{!}{
	\def\arraystretch{1.7}
	\begin{tabular}{c|c|c}
	    Ableitung | f'(x)                          & Funktion | f(x)                          & Integral | F(x)                       \\
        \hline
	    \(0\)                                      & \(C\)                                    & \(x+C\)                               \\
        \hline
        \(1\)                                      & \(x\)                                    & \(\frac{1}{2}x^2+C\)                  \\
        \hline
        \(-\frac{1}{x^2}\)                         & \(\frac{1}{x}\)                          & \(\ln|x|+C\)                          \\
        \hline
        \(ax^{a-1}\)                               & \(x^a \: with \: a \: \in \: \mathbb{R}\) & \(\frac{x^{a+1}}{a+1}+C\)             \\
        \hline
        $x^{x} \cdot(1+\ln x) \quad x>0$          & $x^{x}$                                  &   \\
        \hline
        $\left(x^{x}\right)^{x}(x+2 x \ln (x)) \quad x>0$ & $x^{\left(x^{x}\right)}$         &  \\
        \hline  
        
        $\frac{1}{2\sqrt{x}}$                     & $\sqrt{x}$                               &  $\frac{2}{3}x^{\frac{3}{2}}$         \\
        \hline
        $\frac{1}{n} x^{\frac{1}{n}-1}$           & $\sqrt[n]{x}$                            & $\frac{n}{n+1} x^{\frac{1}{n}+1}$     \\
        \hline

        \(\ln(a)\cdot a^x\)                       & \(a^x\)                                  & \(\frac{a^x}{\ln(a)}+C\)              \\
        \hline
        $a^{b x} \cdot c \ln a$                   & $a^{b x}$                                & $\frac{1}{b \ln a} a^{b x}$             \\
        \hline
        \(e^x\)                                   & \(e^x\)                                  & \(e^x+C\)                             \\
        \hline
        \(\frac{1}{x}\)                           & \(\ln(x)\)                               & \(x\ln(x)-x+C\)                       \\
        \hline
        \(\frac{1}{x\ln(a)}\)                     & \(\log_a(x)\)                            & \(x\log_a(x)-\frac{x}{\ln(a)}+C\)     \\
        \hline
        \(\cos(x)\)                                & \(\sin(x)\)                              & \(-\cos(x)+C\)                        \\
        \hline
        \(-\sin(x)\)                               & \(\cos(x)\)                              & \(\sin(x)+C\)                         \\
        \hline
        \(1+\tan^2{(x)=\frac{1}{\cos^2{(x)}}}\)    & \(\tan(x)\)                              & \(-\ln|\cos(x)|+C\)                   \\
        \hline
        \(-1-\cot^2{(x)}=-\frac{1}{\sin^2{(x)}}\)  & \(\cot(x)\)                              & \(\ln(\sin(x))+C\)                    \\
        \hline
        \(\frac{1}{\sqrt{1-x^2}}\)                & \(\arcsin(x)\)                           & \(x\arcsin(x)+\sqrt{1-x^2}+C\)        \\
        \hline
        \(-\frac{1}{\sqrt{1-x^2}}\)               & \(\arccos(x)\)                           & \(x\arccos(x)-\sqrt{1-x^2}+C\)        \\
        \hline
        \(\frac{1}{1+x^2}\)                       & \(\arctan(x)\)                           & \(x\arctan(x)-\frac{1}{2}\ln(1+x^2)+C\)\\
        %add more here
        \hline
        $\sin ^{2}(x)$                            & $\frac{1}{2}(x-\sin (x) \cos (x))$       & $\sin (x) \cos (x) + C$                   \\
        \hline
        $\cos ^{2}(x)$                            & $\frac{1}{2}(x+\sin (x) \cos (x))$       & $\cos (x) \sin (x) + C$                   \\ 
        \hline
        $\tan ^{2}(x)$                            & $\tan (x)-x$                             & $\tan (x) + C$                            \\
        \hline
        $\cot ^{2}(x)$                            & $-\cot (x)-x$                            & $\cot (x) + C$                            \\
        \hline
        $\frac{f'(x)}{f(x)}$                      & $\ln |f(x)|$                             & $x \cdot(\ln |x|-1) + C$                 \\
        \hline
        $\frac{1}{x}(\ln x)^{n}$                  & $\frac{1}{n+1}(\ln x)^{n+1} \quad n \neq-1$ & $\frac{1}{2 n}\left(\ln x^{n}\right)^{2} \quad n \neq 0 + C$ \\
        \hline
        $\frac{1}{x \ln x}$                       & $\ln |\ln x| \quad x>0, x \neq 1$        & $\frac{1}{b \ln a} a^{b x} + C$           \\
        \hline
        $x \cdot e^{c x}$                         & $\frac{c x-1}{c^{2}} \cdot e^{c x}$      & $\frac{x^{n+1}}{n+1}\left(\ln x-\frac{1}{n+1}\right) \quad n \neq-1 + C$ \\
        \hline
        $x^{n} \ln x$                             & $\ln (\cosh (x))$                        & $\ln |f(x)| + C$                          \\
        \hline
        $\sin (x) \cos (x)$                       & $\frac{\sin^2(x)}{2} $                &\\
        \hline
        $\frac{-f^{\prime}(x)}{(f(x))^{2}}$       & $\frac{1}{f(x)}$                         & \\
        \hline
        $(a x+b)^{n}$                             & $\frac{1}{a \cdot(n+1)}(a x+b)^{n+1}$   & \\
        \hline
	\end{tabular}
    }
\end{formula}

\begin{KR}{Trick Gerade/Ungerade bei Integralen}\\
    Für ungerade Funktionen gilt $\int_{-a}^{+a} f(x) \dif x = 0$.
    \begin{itemize}
        \item Summe/Komposition: ungerade und ungerade $\rightarrow$ ungerade
        \item Produkt/Quotient: ungerade und gerade $\rightarrow$ ungerade
        \item Ableitung: gerade $\longrightarrow$ ungerade
    \end{itemize}
    Bsp ungerade: $f(x)$ = $-x$, $x$, $sin(x)$, $tan(x)$, Polynomfunktionen mit ungeradem Exponent\\
    gerade: $1$, $x^2$, $cos(x)$, $sec(x)$, Polynomf. mit geradem Exponent\\
    gerade und ungerade!! $f(x) = 0$
\end{KR}

\raggedcolumns
\columnbreak

    \begin{concept}{Ableitungsregeln}
        \begin{itemize}
        \item Summenregel
            \[f(x)=g(x)+h(x) \rightarrow f'(x)=g'(x)+h'(x) \]
        \item Differenzregel
            \[f(x)= g(x) - h(x) \rightarrow f'(x) = g'(x) - h'(x) \]
        \item Faktorregel 
            \[f(x)=a\cdot g(x) \rightarrow f'(x)=a \cdot g'(x) \]
        \item Produktregel
            \[f(x)=g(x)\cdot h(x) \rightarrow f'(x)=g'(x)\cdot h(x) + g(x) \cdot h'(x) \]
        \item Quotientenregel 
            \[f(x)=\frac{g(x)}{h(x)} \rightarrow f'(x)=\frac{g'(x)\cdot h(x)-g(x)\cdot
            h'(x)}{h^2(x)}\]
        \item Kettenregel
            \[f(x)=g(h(x)) \rightarrow f'(x)=g'(h(x))\cdot h'\]
        \item Potenz/Logarithmus 
            $$(a^{f(x)})' = ln(a) \cdot a^{f(x)} \cdot f'(x)$$
            \resizebox{\linewidth}{!}{
            $(f(x)^{g(x)})' = f(x)^{g(x)} \cdot (ln(f(x)) \cdot g(x))' =  f(x)^{g(x)} \cdot (ln(f(x)) \cdot g(x) \cdot \frac{f'(x)}{f(x)})$
            }
        \end{itemize}
    \end{concept}


    \begin{KR}{Differentialrechnung Tricks}
        \begin{itemize}
      \item Überall differenzierbar: Einheitliche Tangente (Ableitung 0 setzen) und dh: Grenzwerte müssen denselben Wert ergeben
      \item Zwei Funktionskurven berühren sich (aww): bedeutet dass sie an einer Stelle x0 den gleichen Funktionswert und die gleiche Ableitung haben
      \item Tangente bestimmen (Linearisierung): $f\left(x_{0}\right)+f^{\prime}\left(x_{0}\right)\left(x-x_{0}\right)$
    \end{itemize}
    \end{KR}


\begin{concept}{Integralregeln}
    \begin{itemize}
      \item Addition/Subtraktion:
      $$\int f(x-k) d x=F(x-k)+C$$
      \item Multiplikation:
      $$\int f(x \cdot k) d x=\frac{1}{k} F(x \cdot k)+C$$
      \item Skalarmultiplikation:
      $$\int \lambda_{1} f(x)+\lambda_{2} g(x) d x=\lambda_{1} F(x)+\lambda_{2} G(x)+C$$
    \end{itemize}
\end{concept}



\columnbreak



\begin{KR}{Grenzwert Berechnen Tricks}
    \begin{itemize}
      \item " $\frac{\infty}{\infty} "$ Trick: Erweitern mit $\frac{1}{n^{k}}$ (k: grösster Exponent)
    \end{itemize}
    
    $\lim _{n \rightarrow \infty} \frac{2 n^{6}-n^{3}}{7 n^{6}+n^{5}-3} \cdot \frac{\frac{1}{n^{6}}}{\frac{1}{n^{6}}}=\frac{2}{7}$
    
    \begin{itemize}
      \item " $\frac{\infty}{\infty} "$ Trick: Erweitern mit $\frac{1}{a^{k}}$ (a: grösste Basis, k: kleinster Exponent) $\lim _{n \rightarrow \infty} \frac{7^{n-1}+2^{n+1}}{7^{n}+5} \cdot \frac{\frac{1}{7^{n-1}}}{\frac{1}{7^{n-1}}}=\frac{1}{7}$
      \item " $\infty-\infty$ " Trick: Erweitern mit $\sqrt{\cdots}+\sqrt{\cdots}$
    \end{itemize}
    
    $\lim _{n \rightarrow \infty}\left(\sqrt{n^{2}+n}-\sqrt{n^{2}+1}\right)=1 / 2$
    
    \begin{itemize}
      \item e-like...: Trick: umformen zu $\left(\left(1+\frac{1}{x}\right)^{x}\right)^{a} \Rightarrow e^{a}$
    \end{itemize}
    \end{KR}

    \begin{concept}{Rechnen mit Grenzwerten von Funktionen}
        \begin{itemize}
            \item $\lim_{x \to x_0} (f + g)(x) = \lim_{x \to x_0} f(x) + \lim_{x \to x_0} g(x)$
            \item $\lim_{x \to x_0} (f \cdot g)(x) = \lim_{x \to x_0} f(x) \cdot \lim_{x \to x_0} g(x)$
            \item Sei $f \leq g$, so ist $\lim_{x \to x_0} f(x) \leq \lim_{x \to x_0} g(x)$
            \item Falls $g_1 \leq f \leq g_2$ und $\lim_{x \to x_0} g_1(x) = \lim_{x \to x_0} g_2(x)$, so existiert $\lim_{x \to x_0} f(x) = \lim_{x \to x_0} g_1(x)$
        \end{itemize}
    \end{concept}

    \begin{formula}{Spezielle Grenzwerte}
        \textcolor{pink}{\[n\rightarrow \infty\]}
        \resizebox*{0.9\linewidth}{!}{
        %\def\arraystretch{1.2}
        \begin{tabular}{c|c|c}
            \(\frac{1}{n}\rightarrow 0\)& \(e^n\rightarrow \infty\)&\(\frac{1}{n^k}\rightarrow 0\ \ \forall k\in
            \mathbb{R}^+\)\\
            \hline
            \(c+\frac{1}{n}\rightarrow c\)&\(e^{-n}\rightarrow 0\)&\((1+n)^{\frac{1}{n}}\rightarrow 1\)\\
            \hline
            \(\frac{c\cdot n}{c^n}\rightarrow 0\)&\(\frac{e^n}{n^c}\rightarrow \infty\)&\(\left(1+\frac{1}{n}\right)^c
            \rightarrow 1\)\\
            \hline
        \(\sqrt[n]{n}=n^{\frac{1}{n}}\rightarrow 1\)&\(\frac{\sin{n}}{n}\rightarrow 0\)&\(\left(1+\frac{1}{n}\right)^n
        \rightarrow e\)\\
        \hline
        \(\sqrt[n]{n!}\rightarrow \infty\)&\(\arctan{n}\rightarrow \frac{\pi}{2}\)&\(\left(1+\frac{c}{n}\right)^n
        \rightarrow e^c\)\\
        \hline
        \(\frac{1}{n}\sqrt[n]{n!} \rightarrow \frac{1}{e}\)&\(\ln{n}\rightarrow \infty\)&\(\left(1-\frac{1}{n} \right)^n
        \rightarrow \frac{1}{e}\)\\
        \hline
        \(\frac{c^n}{n!} \rightarrow 0\)&\(\frac{\ln{n}}{n}\rightarrow 0\)&\(\left(\frac{n}{n+c}\right)^n\rightarrow
        e^{-c}\)\\
        \hline
        \(\frac{n^n}{n!}\rightarrow \infty\)&\(\frac{\log{n}}{n-1}\rightarrow 1\)& \\
        \end{tabular}
        }
        \[n^c\cdot q^n \rightarrow 0 \quad \forall c \in \mathbb{Z},0\leq q \leq 1\]
        \[n(\sqrt[n]{c}-1)\rightarrow \ln{c}\quad \forall c > 0\]
        \textcolor{pink}{\[n\rightarrow 0\]}
        \resizebox*{0.9\linewidth}{!}{
        %\def\arraystretch{1.2}
        \begin{tabular}{c|c|c}
            \(\ln{n}\rightarrow -\infty\)&\(\frac{\sin{n}}{n}\rightarrow 1\)&\(\frac{1}{\arctan{n}}\rightarrow 1\)\\
            \hline
            \(n\log{n}\rightarrow 0\)&\(\frac{\cos{(n)}-1}{n}\rightarrow 0\)&\(\frac{e^n-1}{n}\rightarrow 1\)\\
            \hline
            \(\frac{\log{1}-n}{n}\rightarrow -1\)&\(\frac{1}{\cos{n}}\rightarrow1\)&\(\frac{e^cn-1}{n}\rightarrow c\)\\
            \hline 
            \(\frac{c^n-1}{n}\rightarrow\ln{c},\forall c>0\)&\(\frac{1-\cos{n}}{n^2}\rightarrow
            \frac{1}{2}\)&\((1+n)^{\frac{1}{n}}\rightarrow e\)\\
        \end{tabular}
        }
    \end{formula}

    \begin{definition}{Reihen - Funktionen}\\
        $
        \begin{array}{ll}
            \sum^{n}_{k=1} k = \frac{n \cdot (n+1)}{2} & \sum^{n}_{k=1} (2k - 1)^2 = \frac{n \cdot (4n^2-1)}{3}\\
            \sum^{n}_{k=1} 2k-1 = n^2 & \sum^{n}_{k=1} k^3 = \left( \frac{n \cdot (n+1)}{2}\right)^2\\
            \sum^{n}_{k=1} 2k = n(n+1) & \sum^{n}_{k=1} \frac{1}{k(k+1)} = \frac{n}{n+1}\\
            \sum^{n}_{k=1} k^2 = \frac{n \cdot (n+1) \cdot (2n+2)}{6} &
        \end{array}
        $
    \end{definition}




