\section{Taylorrreihen}
\begin{definition}{Definition Potenzreihen}\\
  \begin{itemize}
    \item Eine Potenzreihe ist eine undendliche Reihe vom Typ:
  \[p(x)=a_0+a_1x+a_2x^2+ \cdots = \sum_{k=0}^{\infty}{a_kx^k} \]
Die reellen Zahlen \(a_0,a_1, \cdots\) sind die Koeffizientend der Potenzreihe
  \item Allgemein können Potenzreihen mit einer verschiebung von \(x_0\) beschrieben werden, somit ist es eine
    Potenzreihe mit Zentrum \(x_0\):
    \[p(x)=a_0+a_1(x-x_0)+a_2(x-x_0)^2+\cdots = \sum_{k=0}^{\infty}{a_k(x-x_0)^k}\]
\end{itemize}
\end{definition}
\begin{definition}{Definition Taylorreihe}\\
  \begin{itemize}
    \item Die Taylorreihe oder Taylorentwicklung einer Funktion \(y=f(x)\) and der Stelle \(x_0\) ist die Potenzreihe:
      \[t_f(x)=\sum_{k=0}^{\infty}{a_k(x-x_0)^k}\]
      welche die gleiche Ableitung an der Stelle \(x_0 \text{ für alle }k\in \mathbb{N}\) hat wie die Funktion \(f(x)\)
  \end{itemize}
\end{definition}
\begin{definition}{Definition Taylorpolynom}\\
    \begin{itemize}
      \item Ein Taylorpolynom ist eine Taylorreihe \(t_f(x)\) welche nach \(n\text{-ter}\) Ordnung abgebrochen wird.
        Somit erhällt man das Taylorpolynom \(n\text{-ter}\) Ordnung von \(f(x)\) an der Stelle \(x_0\):
        \[p_n(x)=\sum_{k=0}^n{a_k(x-x_0)^k}\]
      \item Bemerkung: Die Tangente der Funktionskurve \(y=f(x) \) an der Stelle \(x_0\) ist exakt das Taylorpolynom 1.
        Ordnung von \(f(x)\) an der Stelle \(x_0\)
    \end{itemize}
\end{definition}
\begin{KR}{Vorgehen Berechnen Taylorreihe}
  \begin{itemize}
    \item Die Taylorreihe einer beliebig oft differenzierbaren Funktion \(t(x)\) an der Stelle \(x_0\) ist:
      \[t_f(x) = \sum_{k=0}^{\infty}{\frac{f^{k}(x_0)}{k!}\cdot(x-x_0)^k}\]
  \end{itemize}
\end{KR}
\begin{formula}{Formel für Taylorkoeffizienten}\\
  \begin{itemize}
    \item Formel für \(k\)-ten Taylorkoeffizientn der Taylorreihe \(t_f(x)\) von \(f(x)\) an der Stelle
      \(x_0\in\mathbb{R}\):
      \[a_k=\frac{f^{(k)}(x_0)}{k!}\quad (k\in\mathbb{N})\]
  \end{itemize}
\end{formula}
\subsection{Symetrie von Potenzreihen und Taylorreihen}
\begin{lemma}{Symetrie von Funktionen Repetition}\\
  \begin{itemize}
    \item Gerade Funktion: Funktion für die gilt: \(f(-x)=f(x)\) für alle \(x\in\mathbb{R}\rightarrow\) Funktion ist
      achsensymetrisch bzgl. \(y\)-Achse
    \item Ungerade Funktion: Funktion für die gilt: \(f(-x)=-f(x)\) für alle \(x\in\mathbb{R}\rightarrow\) Funktion ist
      punktsymetrisch bzgl. des Ursprungs
  \end{itemize}
\end{lemma}
\begin{lemma}{Symetrie von Potenzreihen}\\
  \begin{itemize}
    \item Eine Potenzreihe
      \[y=\sum_{k=0}^{\infty}{a_kx^k}=a_0+a_1x+a_2x^2+\cdots\]
    ist eine gerade bzw. ungerade Funktion, falls sie nur gerade bzw. nur ungerade Potenzen enthält.
  \end{itemize}
\end{lemma}
\begin{lemma}{Symetrie von Taylorreihen}\\
  \begin{itemize}
    \item  Falls die Funktion eine gerade Funktion ist, enthällt die Taylorreihe von \(f(x)\) an der Stelle \(x_0 = 0\)
      nur Potenzen mit geraden Exponenten,d.h. se gilt \(a_{2k+1}=0\) für alle \(k\in\mathbb{N}\)
    \item  Falls die Funktion eine ungerade Funktion ist, enthällt die Taylorreihe von \(f(x)\) an der Stelle \(x_0 = 0\)
      nur Potenzen mit ungeraden Exponenten,d.h. se gilt \(a_{2k}=0\) für alle \(k\in\mathbb{N}\)
  \end{itemize}
\end{lemma}
\begin{lemma}{Binomialkoeffizienten}\\
  \begin{itemize}
    \item Zeil: Taylorreihe von Potenzen mit beliebigen (nicht-natürlichen) Exponenten bestimmen, d.h. Funktionen vom
      Typ \(f(x=x^{\alpha})\) mit \(\alpha \in \mathbb{R}\)
    \item Untersuchen der Funktion bei \(f(x)=(1+x)^{\alpha}\) an der Stelle \(x_0=0\)
    \item Falls \(\alpha\in\mathbb{N}\) ist \(f(x)=(1+x)^{\alpha}\) ein Polynom (binomische Formel):
      \[(1+x)^n=\sum_{k=0}^n{\begin{pmatrix}n\\k\end{pmatrix}x^k},\quad
      \begin{pmatrix}n\\k\end{pmatrix}=\frac{n!}{k!(n-k)!}\quad (0\le k \le n)\]
    \item In diesem fall ist die binomische Formel auch die Taylorreihe, es gilt:
      \[a_k=\frac{f^{(k)}(0)}{k!}=\begin{pmatrix}n\\k\end{pmatrix}\]
    \item Falls \(\alpha\in\mathbb{R}\):
    \subitem Taylorkoeffizienten:
    \[\frac{f^{(k)}(0)}{k!}=\frac{\alpha\cdot(\alpha -1)\cdot\cdots\cdot(\alpha-k+1)}{k!}= 
      \begin{pmatrix}\alpha\\k\end{pmatrix}\quad (\alpha\in\mathbb{R},k\in\mathbb{N})\]
    \subitem Taylorreihe:
    \[t_f(x)=1+\begin{pmatrix}\alpha\\1\end{pmatrix}x+\begin{pmatrix}\alpha\\2\end{pmatrix}x^2+
    \begin{pmatrix}\alpha\\3\end{pmatrix}x^3+\cdots =\sum_{k=0}^{\infty}{\begin{pmatrix}\alpha\\k\end{pmatrix}x^k}\]
    Auch bekannt als Binomialreihe
  \end{itemize}
\end{lemma}
\begin{definition}{Regel von Bernoulli- de l’Hospital}\\
  \begin{itemize}
    \item Wenn die Funktionen \(f(x)\text{ und }g(x)\) an der Stelle \(x_0\) stetig differenzierbar sind aber der
      Grenzwert auf die Form \(\frac{0}{0}\text{ oder }\frac{\infty}{\infty}\) führt, kann der Limes der Ableitung
      beider Funktionen ausgewerted werden:
      \[\underset{x\rightarrow x_0}\lim\frac{f(x)}{g(x)}=\underset{x\rightarrow x_0}{\lim}\frac{f'(x)}{g'(x)}\]
    \item Dies kann beliebig oft Wiederholt werden, es gibt jedoch Fälle wo die Regel versagt, dann müssen andere
      Methoden verwendet werden.
  \end{itemize}
\end{definition}
\begin{definition}{Varianten von l'Hospital}\\
  \begin{itemize}
    \item Wenn ein Grenzwert \(\underset{x\rightarrow x_0}{\lim}f(x)\cdot g(x)\) von der Form \(0\cdot \infty\) ist,
      schreiben wir:
      \[f(x)\cdot g(x)=\frac{f(x)}{\frac{1}{g(x)}}\]
      und wenden die Regel an.
    \item Wenn ein Grenzwert \(\underset{x\rightarrow x_0}{\lim}(f(x)-g(x))\) von der Form \(\infty - \infty\) ist,
      schreiben wir:
      \[f(x)-g(x)=\frac{\frac{1}{g(x)}-\frac{1}{f(x)}}{\frac{1}{f(x)\cdot g(x)}}\]
      und wenden die Regel an.
  \end{itemize}
\end{definition}
\begin{definition}{Genauigkeit der Approximation}\\
  Nicht prüfungsrelevant\\
  \begin{itemize}
    \item Die Approximation ist im allgemeinen nicht Perfekt, d.h. \(p_n(x)\neq f(x)\text{ für }x\neq x_0\). Für die
      Abschätzung des Fehlers bzw. Restglieds \(R_n(x)=f(x)-p_n(x)\) gilt:
    \item Ist die Funtion f:\(\mathbb{R}\rightarrow\mathbb{R}\) mindestens \((n+1)\)-mal stetig differenzierbar, und ist 
    \(p_n(x)\) das Taylorpolynom \(n\)-ten Grades von \(f(x)\) an der Stelle \(x_0\).
    Dann gibt es ein \(\xi\) zwischen \(x_0\) und \(x\) so dass für das Restglied \(R_n(x)\) gilt:
    \[|R_n(x)|\leq \left|\frac{f^{n+1}(\xi)}{(n+1)!}(x-x_0)^{n+1}\right|\]
  \end{itemize}
\end{definition}
\subsection{Konvergenz von Potenzreihen}
\begin{definition}{Konvergenzradius}
  \begin{itemize}
    \item Der Konvergenzradius \(\rho\) einer Potenzreih\\ \(p(x)=\sum_{k=0}^{\infty}{a_k(x-x_0)^k}\) ist
      eine Zahl mit Folgenden Eigenschaften:
      \subitem - Für alle \(x\in\mathbb{R}\) mit \(|x-x_0|<\rho\) konvergiert die Reihe \(p(x)\)
      \subitem - Für alle \(x\in\mathbb{R}\) mit \(|x-x_0|>\rho\) divergiert die Reihe \(p(x)\)
    \item Es existieren folgende Extremfälle:
      \subitem - Konvergenzradius \(\rho = 0\): Dann konvergiert die Reihe \(p(x)\) nur für \(x=x_0\).
      \subitem - Konvergenzradius \(\rho = \infty\): Dann konvergiert die Reihe \(p(x)\) für alle \(x\in\mathbb{R}\).
  \end{itemize}
\end{definition}
\begin{formula}{Konvergenzradius Formel}\\
  Für die Potenzreihe \(\displaystyle p(x)=\sum_{k=0}^{\infty}{a_k(x-x_0)^k} \) ist der Konvergenzradius:
  \[\rho = \underset{k \rightarrow \infty}{\lim}\left| \frac{a_k}{a_{k+1}}\right| \quad \text{oder} \quad
  \rho=\underset{k \rightarrow \infty}{\lim} \frac{1}{\sqrt[k]{|a_k|} }\]
\end{formula}
\begin{formula}{Konvergenzbereich Formel}\\
  Der Konvergenzbereich in dem die Approximation der Funktion gilt ist definiert durch:
  \[(x_0 - \rho , x_0 + \rho) \]
\end{formula}
