\subsection*{Taylorreihen}

\begin{example}
    Bestimmen Sie das Taylorpolynom 4. Ordnung $p_{4}(x)$ der Funktion

    $$
    f(x)=\frac{1}{\sqrt{x}}
    $$
    
    um das Entwicklungszentrum $x_{0}=1$

    \tcblower

    Die Ableitungen von $f(x)$ bis zur Ordnung 4 sind

$$
f(x)=x^{-1 / 2}, f^{\prime}(x)=-\frac{1}{2} x^{-3 / 2}, f^{\prime \prime}(x)=\frac{3}{4} x^{-5 / 2}, $$$$f^{(3)}(x)=-\frac{15}{8} x^{-7 / 2}, f^{(4)}(x)=\frac{105}{16} x^{-9 / 2}
$$

Ausgewertet an der Stelle $x_{0}=1$ :

$$
f(1)=1, f^{\prime}(1)=-\frac{1}{2}, f^{\prime \prime}(1)=\frac{3}{4}, f^{(3)}(1)=-\frac{15}{8}, f^{(4)}=\frac{105}{16}
$$

Also ist das gesuchte Taylor-Polynom $p_{4}(x)$ :
\resizebox{\textwidth}{!}{
$
\begin{aligned}
p_{4}(x) & =\frac{1}{0!}+\frac{-1 / 2}{1!}(x-1)+\frac{3 / 4}{2!}(x-1)^{2}+\frac{-15 / 8}{3!}(x-1)^{3}+\frac{105 / 16}{4!}(x-1)^{4} \\
& =1-\frac{1}{2}(x-1)+\frac{3}{8}(x-1)^{2}-\frac{5}{16}(x-1)^{3}+\frac{35}{128}(x-1)^{4}
\end{aligned}
$}

\end{example}

\begin{example}
    Bestimmen Sie das Taylorpolynom 2. Ordnung

$$
f(x)=x \cdot \ln (x)
$$

um das Entwicklungszentrum $x_{0}=e$.
\tcblower

Die Ableitungen von $f(x)$ bis zur Ordnung 2 sind

$$
f(x)=x \cdot \ln (x), f^{\prime}(x)=\ln (x)+1, f^{\prime \prime}(x)=\frac{1}{x}
$$

Ausgewertet an der Stelle $x_{0}=e$ :

$$
f(e)=e, f^{\prime}(e)=2, f^{\prime \prime}(e)=\frac{1}{e}
$$

Also ist das gesuchte Taylor-Polynom:

$$
p_{2}(x)=\frac{e}{0!}+\frac{2}{1!}(x-e)+\frac{1 / e}{2!}(x-e)^{2}=e+2(x-e)+\frac{1}{2 e}(x-e)^{2}
$$

\end{example}

\begin{example}


$$
f(x)=\frac{1}{1-\sin (x)}
$$

soll in der Umgebung von $x_{0}=0$ durch eine Parabel (d.h. ein Polynom 2. Ordnung) ersetzt werden. Berechnen Sie dieses Näherungspolynom $p_{2}(x)$ und vergleichen Sie die Werte von $f(x)$ und $p_{2}(x)$ an der Stelle $x=0.2$.
\tcblower

Die Ableitungen von $f(x)$ bis zur Ordnung 2 sind

$$
f(x)=\frac{1}{1-\sin (x)}, f^{\prime}(x)=\frac{\cos (x)}{(1-\sin (x))^{2}}, $$$$f^{\prime \prime}(x)=\frac{-\sin (x)(1-\sin (x))+2 \cos ^{2}(x)}{(1-\sin (x))^{3}}
$$

Ausgewertet an der Stelle $x_{0}=0$ :

$$
f(0)=1, f^{\prime}(0)=1, f^{\prime \prime}(0)=2 \text {. }
$$

Also ist das gesuchte Taylor-Polynom:

$$
p_{2}(x)=\frac{1}{0!}+\frac{1}{1!} x+\frac{2}{2!} x^{2}=1+x+x^{2}
$$

Vergleich der Funktionswerte: $f(0.2) \approx 1.2479, p_{2}(0.2)=1.24$.
\end{example}

\begin{example}
    Taylorreihe $t_{f}(x)$ von $f(x)=-\ln (1-x)$ an der Stelle $x_{0}=0$.
    
    $$
t_{f}(x)=x+\frac{x^{2}}{2}+\frac{x^{3}}{3}+\ldots=\sum_{k=1}^{\infty} \frac{x^{k}}{k}
$$
\end{example}
\begin{example}
Taylorreihe $t_{g}(x)$ von $g(x)=\frac{1}{1-x}$ an der Stelle $x_{0}=0$.


$$
t_{g}(x)=1+x+x^{2}+\ldots=\sum_{k=0}^{\infty} x^{k}
$$

\end{example}

\begin{example}
    Bestimmen Sie das Taylorpolynom 4. Ordnung an der Stelle $x_{0}=0$ der Funktion $y=\cos ^{2}(x)$

a. Formel für die Taylorkoeffizienten 

b. Taylorreihe für $\cos (x)$ als Ausgangspunkt nehmen und quadrieren, d.h. von dem Produkt

$$
\cos ^{2}(x)=\left(1-\frac{x^{2}}{2}+\frac{x^{4}}{24} \mp \ldots\right)\left(1-\frac{x^{2}}{2}+\frac{x^{4}}{24} \mp \ldots\right)
$$

genügend viele Terme ausmultiplizieren.
\tcblower
a. Die Ableitungen von $y=\cos ^{2}(x)$ sind $y^{\prime}=-2 \cos (x) \sin (x)=-\sin (2 x), \quad y^{\prime \prime}=-2 \cos (2 x), \quad y^{(3)}=4 \sin (2 x), \quad y^{(4)}=8 \cos (2 x)$

Also

$$
a_{0}=1, \quad a_{1}=0, \quad a_{2}=\frac{-2}{2!}=-1, \quad a_{3}=0, \quad a_{4}=\frac{8}{4!}=\frac{1}{3}
$$

Das gesuchte Taylorpolynom ist also 
$
p_{4}(x)=1-x^{2}+\frac{x^{4}}{3}
$

b. Ausmultiplizieren liefert

$$
\left(1-\frac{x^{2}}{2}+\frac{x^{4}}{24} \mp \ldots\right)\left(1-\frac{x^{2}}{2}+\frac{x^{4}}{24} \mp \ldots\right)$$ $$=1+\frac{x^{4}}{4}+\ldots-2 \cdot \frac{x^{2}}{2}+2 \cdot \frac{x^{4}}{24} \pm \ldots=1-x^{2}+\frac{x^{4}}{3} \mp
$$

Daraus folgt ebenfalls
$
p_{4}(x)=1-x^{2}+\frac{x^{4}}{3}
$
\end{example}

\begin{example}
    Bestimmen Sie das Taylorpolynom 4. Ordnung an der Stelle $x_{0}=0$ der Funktion $y=\cos \left(x^{2}\right)$

a. Formel für die Taylorkoeffizienten 

b. Taylorreihe für $f(u)=\cos (u)$ als Ausgangspunkt, die Substitution $u=x^{2}$ 
\tcblower

a. Die Ableitungen von $y=\cos \left(x^{2}\right)$ sind

$$
\begin{gathered}
y^{\prime}=-2 x \sin \left(x^{2}\right), \quad y^{\prime \prime}=-2 \sin \left(x^{2}\right)-4 x^{2} \cos \left(x^{2}\right), \\ y^{(3)}=-12 x \cos \left(x^{2}\right)+8 x^{3} \sin \left(x^{2}\right) \\
y^{(4)}=-12 \cos \left(x^{2}\right)+48 x^{2} \sin \left(x^{2}\right)+16 x^{4} \cos \left(x^{2}\right)
\end{gathered}
$$

Also$a_{0}=1, \quad a_{1}=0, \quad a_{2}=0, \quad a_{3}=0, \quad a_{4}=\frac{-12}{4!}=-\frac{1}{2}$

Das gesuchte Taylorpolynom ist also $p_{4}(x)=1-\frac{x^{4}}{2}
$

b. Die (bekannte) Taylorreihe von $f(u)=\cos (u)$ ist

$$
t_{\cos }(u)=1-\frac{u^{2}}{2}+\frac{u^{4}}{4!} \mp \ldots
$$

Einsetzen von $u=x^{2}$ und Abbrechen nach dem Term 4. Ordnung liefert
$
p_{4}(x)=1-\frac{x^{4}}{2}
$
\end{example}

\begin{example}
    Wir betrachten die Funktion

$
f(x)=\frac{1}{1-2 x}
$

a. Bestimmen Sie die Taylorreihe von $f(x)$ um $x_{0}=0$, indem Sie die Formel für die TaylorKoeffizienten verwenden.

b. Bestätigen Sie das Resultat von a., indem Sie die Summenformel der unendlichen geometrischen Reihe auf den Funktionsausdruck anwenden.
\tcblower

a. Die Ableitungen von $f(x)=\frac{1}{1-2 x}$ sind

$$
f(x)=\frac{1}{1-2 x}, f^{\prime}(x)=\frac{2}{(1-2 x)^{2}}, f^{\prime \prime}(x)=\frac{8}{(1-2 x)^{3}}, $$ $$\ldots, f^{(k)}(x)=\frac{k!2^{k}}{(1-2 x)^{k+1}}
$$

An der Stelle $x_{0}=0: f^{(k)}(0)=k!2^{k}$, also $a_{k}=\frac{f^{(k)}(0)}{k!}=2^{k}$. Die Taylorreihe von $f(x)$ an der Stelle $x_{0}=0$ ist also

$$
t_{f}(x)=1+2 x+4 x^{2}+8 x^{3}+\ldots=\sum_{k=0}^{\infty} 2^{k} x^{k}
$$

b. Aus der Summenformel der unendlichen geometrischen Reihe, nämlich

$$
\sum_{k=0}^{\infty} q^{k}=\frac{1}{1-q} \quad \text { für }|q|<1
$$

folgt, angewendet für $q=2 x$, dieselbe Reihe wie bei a.
\end{example}

\begin{example}
    Bestimmen Sie die positive Lösung der Gleichung $\cos (x)=2 x^{2}$ näherungsweise durch Approximation von $\cos (x)$ durch

a. das Taylorpolynom 2. Ordnung an der Stelle $x_{0}=0$,

b. das Taylorpolynom 4. Ordnung an der Stelle $x_{0}=0$.
\tcblower
a. Taylorpolynom 2. Ordnung von $f(x)=\cos (x)$ an der Stelle $x_{0}=0$ : $p_{2}(x)=1-\frac{x^{2}}{2}$. Also erhalten wir die Gleichung

$$
1-\frac{x^{2}}{2}=2 x^{2}
$$

Positive Lösung dieser Gleichung:

$$
x=\sqrt{\frac{2}{5}} \approx 0.6325
$$

b. Taylorpolynom 4. Ordnung von $f(x)=\cos (x)$ an der Stelle $x_{0}=0$ : $p_{2}(x)=1-\frac{x^{2}}{2}+\frac{x^{4}}{24}$. Also erhalten wir die Gleichung

$$
1-\frac{x^{2}}{2}+\frac{x^{4}}{24}=2 x^{2}
$$

bzw.

$$
x^{4}-60 x^{2}+24=0
$$

Positive Lösungen dieser Gleichung (biquadratische Gleichung; mit Substitution $u=x^{2}$ lösen): $x_{1,2}=\sqrt{30 \pm \sqrt{876}}$; wir brauchen hier $x_{2}$, d.h.

$$
x=\sqrt{30-\sqrt{876}} \approx 0.6345
$$
\end{example}

\begin{example}

$$
f(x)=\left(1+e^{x}\right)^{2}
$$

a. Bestimmen Sie das Taylorpolynom 3. Ordnung $p_{3}(x)$ der Funktion $f(x)$ um das Entwicklungszentrum $x_{0}=0$.

b. Welchen Näherungswert erhält man mit $p_{3}(x)$ für den Funktionswert an der Stelle $x=0.2$ ? Bestimmen Sie auch die Abweichung vom exakten Funktionswert.
\tcblower
a. Mit Hilfe der Kettenregel berechnet man
- $f(0)=4$
- $f^{\prime}(0)=\left.2\left(e^{x}+e^{2 x}\right)\right|_{x=0}=4$
- $f^{\prime \prime}(0)=\left.2\left(e^{x}+2 e^{2 x}\right)\right|_{x=0}=6$
- $f^{\prime \prime \prime}(0)=\left.2\left(e^{x}+4 e^{2 x}\right)\right|_{x=0}=10$

Daraus folgt

$$
p_{3}(x)=4+4 x+3 x^{2}+\frac{5}{3} x^{3}
$$

b. Man berechnet

$$
f(0.2)-p_{3}(0.2) \approx 0.0013
$$
\end{example}

\begin{example}
    Lösen Sie die Gleichung

$$
\frac{1}{2}\left(e^{x}+e^{-x}\right)=4-x^{2}
$$

näherungsweise durch Entwicklung von $\frac{1}{2}\left(e^{x}+e^{-x}\right)$ in ein Taylorpolynom 4. Ordnung bei $x_{0}=0$.

\tcblower
Mit Hilfe der Taylorreihe

$$
e^{x}=\sum_{k=0}^{\infty} \frac{x^{k}}{k!}
$$

um 0 erhält man $\frac{1}{2}\left(e^{x}+e^{-x}\right)$

$$
=\frac{1}{2}\left(1+1+x-x+\frac{1}{2} x^{2}+\frac{1}{2} x^{2}+\frac{1}{6} x^{3}-\frac{1}{6} x^{3}+\frac{1}{24} x^{4}+\frac{1}{24} x^{4}+\ldots\right)
$$

und somit ist das Taylorpolynom 4. Ordnung $p_{4}$ der Funktion $\frac{1}{2}\left(e^{x}+e^{-x}\right)$ gegeben durch

$$
p_{4}(x)=1+\frac{1}{2} x^{2}+\frac{1}{24} x^{4}
$$

Die Gleichung $p_{4}(x)=4-x^{2}$ ist also die biquadratische Gleichung

$$
\frac{1}{24} x^{4}+\frac{3}{2} x^{2}-3=0 \quad \Leftrightarrow \quad x^{4}+36 x^{2}-72=0
$$

Mit der Substitution $u=x^{2}$ erhält man mit Hilfe der Auflösungsformel für quadratische Gleichungen

$$
u_{ \pm}=\frac{-36 \pm \sqrt{1584}}{2}=-18 \pm 6 \sqrt{11}
$$

Da nur $u_{+}>0$, folgt für die Näherungslösung $x_{ \pm}$der Gleichung

$$
x_{ \pm}= \pm \sqrt{6 \sqrt{11}-18}
$$
\end{example}


\begin{example}
    Berechnen Sie das Integral

$$
\int_{0}^{0.3} \sqrt{1+x^{2}} \mathrm{~d} x
$$

durch Entwicklung des Integranden in ein Taylorpolynom 6. Ordnung bei $x_{0}=0$ und gliedweise Integration.

Hinweis: Finden Sie zuerst ein geeignetes Taylorpolynom von $\sqrt{1+x}$ und ersetzen Sie dann $x$ durch $x^{2}$.

\tcblower
Das Taylorpolynom 3. Ordnung $p_{3}$ der Funktion $\sqrt{1+x}$ um 0 ist gegeben durch

$$
p_{3}(x)=1+\frac{1}{2} x-\frac{1}{8} x^{2}+\frac{1}{16} x^{3}
$$

Diese Formel findet man beispielsweise in einer Formelsammlung. Somit ist das Taylorpolynom 6. Ordnung der Funktion $\sqrt{1+x^{2}}$ gegeben durch

$$
p_{3}\left(x^{2}\right)=1+\frac{1}{2} x^{2}-\frac{1}{8} x^{4}+\frac{1}{16} x^{6}
$$
Wir berechnen den Näherungswert des gegebenen Integrals

$$
\begin{aligned}
\int_{0}^{0.3} \sqrt{1+x^{2}} d x & \approx \int_{0}^{0.3} d x+\frac{1}{2} \int_{0}^{0.3} x^{2} d x-\frac{1}{8} \int_{0}^{0.3} x^{4} d x+\frac{1}{16} \int_{0}^{0.3} x^{6} d x \\
& =0.3+\frac{1}{6} 0.3^{3}-\frac{1}{40} 0.3^{5}+\frac{1}{112} 0.3^{7} \\
& \approx 0.304441
\end{aligned}
$$
\end{example}


\begin{example}
    Berechnen Sie das Integral

$$
\int_{0}^{0.5} \frac{1}{2}\left(e^{\sqrt{x}}+e^{-\sqrt{x}}\right) \mathrm{d} x
$$

durch Entwicklung des Integranden in ein Taylorpolynom 3. Ordnung bei $x_{0}=0$ und gliedweise Integration.

Hinweis: Finden Sie zuerst ein geeignetes Taylorpolynom von $\frac{1}{2}\left(e^{x}+e^{-x}\right)$ und ersetzen Sie dann $x$ durch $\sqrt{x}$.

\tcblower
Mit Hilfe der Aufgabe 2 ergibt für sich das Taylorpolynom 6. Ordnung $p_{6}$ der Funktion $\frac{1}{2}\left(e^{x}+e^{-x}\right)$ um 0

$$
p_{6}(x)=1+\frac{1}{2} x^{2}+\frac{1}{24} x^{4}+\frac{1}{720} x^{6}
$$

Somit ist das Taylorpolynom 3. Ordnung der Funktion $\frac{1}{2}\left(e^{\sqrt{x}}+e^{-\sqrt{x}}\right)$ um 0 gegeben durch

$$
p_{6}(\sqrt{x})=1+\frac{1}{2} x+\frac{1}{24} x^{2}+\frac{1}{720} x^{3}
$$

Wir erhalten schliesslich

\resizebox{\textwidth}{!}{
$
\int_{0}^{0.5} \frac{1}{2}\left(e^{\sqrt{x}}+e^{-\sqrt{x}}\right) d x \approx \int_{0}^{0.5}\left(1+\frac{1}{2} x+\frac{1}{24} x^{2}+\frac{1}{720} x^{3}\right) d x \approx 0.56426
$
}
\end{example}
    
    






\subsection{Potenzreihen}

\begin{example}

$$
p(x)=x-\frac{x^{3}}{3}+\frac{x^{5}}{5}-\frac{x^{7}}{7}+\frac{x^{9}}{9}-\ldots
$$

a. Bestimmen Sie die Ableitung $p^{\prime}(x)$ von $p(x)$, indem Sie Term für Term ableiten.

b. Schreiben Sie $p^{\prime}(x)$ in geschlossener Form (geometrische Reihe!).

c. Integrieren Sie das bei b. erhaltene Resultat (mit $p(0)=0$ ), um einen geschlossenen Ausdruck für $p(x)$ zu erhalten.
\tcblower

a. Die Ableitung von $p(x)$ ist

$$
p^{\prime}(x)=1-x^{2}+x^{4}-x^{6} \pm \ldots
$$

b. Die bei a. erhaltene Reihendarstellung für $p^{\prime}(x)$ ist eine unendliche geometrische Reihe mit Summe

$$
p^{\prime}(x)=\frac{1}{1+x^{2}} \quad(\text { für }|x| \leq 1)
$$

c. Wir integrieren das Resultat von b. unbestimmt und erhalten

$$
p(x)=\int \frac{1}{1+x^{2}} \mathrm{~d} x=\arctan (x)+C
$$

Einsetzen von $p(0)=0$ liefert $C=0$, also $p(x)=\arctan (x)$.
\end{example}




\begin{example}
    Konvergenzbereich von $p_{1}(x)=1+2 x+3 x^{2}+4 x^{3}+\ldots$
\tcblower
 Es gilt $a_{k}=k+1$, also ist der Konvergenzradius:

$$
\rho=\lim _{k \rightarrow \infty}\left|\frac{a_{k}}{a_{k+1}}\right|=\lim _{k \rightarrow \infty}\left|\frac{k+1}{k+2}\right|=\lim _{k \rightarrow \infty} \frac{1+\frac{1}{k}}{1+\frac{2}{k}}=1
$$

(Berechnung auch mit der Regel von Bernoulli-de l'Hospital möglich.)

Verhalten am Rand des Konvergenzbereichs:

$$
\begin{aligned}
x=\rho=1: & 1+2+3+4+5+\ldots: \text { divergent } \\
x=-\rho=-1: & 1-2+3-4+5 \mp \ldots: \text { divergent }
\end{aligned}
$$

Die Potenzreihe konvergiert also für

$$
-1<x<1
$$
\end{example}


\begin{example}
    Konvergenzbereich von $p_{2}(x)=1+\frac{x}{2}+\frac{x^{2}}{4}+\frac{x^{3}}{8}+\ldots$
\tcblower
Es gilt $a_{k}=\frac{1}{2^{k}}$, also ist der Konvergenzradius

$$
\rho=\lim _{k \rightarrow \infty}\left|\frac{a_{k}}{a_{k+1}}\right|=\lim _{k \rightarrow \infty}\left|\frac{\frac{1}{2^{k}}}{\frac{1}{2^{k+1}}}\right|=\lim _{k \rightarrow \infty} \frac{2^{k+1}}{2^{k}}=\lim _{k \rightarrow \infty}(2)=2
$$

Verhalten am Rand des Konvergenzbereichs:

$$
\begin{aligned}
x=\rho=2: & 1+1+1+1+1+\ldots: \text { divergent } \\
x=-\rho=-2: & 1-1+1-1+1 \mp \ldots: \text { divergent }
\end{aligned}
$$

Die Potenzreihe konvergiert also für

$$
-2<x<2
$$
\end{example}

\begin{example}
    Konvergenz von $p_{3}(x)=1+\frac{x}{4 \cdot 2}+\frac{x^{2}}{4^{2} \cdot 3}+\frac{x^{3}}{4^{3} \cdot 4}+\frac{x^{4}}{4^{4} \cdot 5}+\ldots$
\tcblower
Die Koeffizienten der gegebenen Potenzreihe sind $a_{n}=\frac{1}{4^{n} \cdot(n+1)}$. Der Konvergenzradius ist also

\resizebox{\textwidth}{!}{
$
\rho=\lim _{n \rightarrow \infty}\left|\frac{a_{n}}{a_{n+1}}\right|=\lim _{n \rightarrow \infty} \frac{4^{n+1} \cdot(n+2)}{4^{n} \cdot(n+1)}=\lim _{n \rightarrow \infty} 4 \cdot\left(1+\frac{1}{n+1}\right)=4 \cdot 1=4
$
}

Verhalten am Rand des Konvergenzbereichs:

$$
\begin{array}{cl}
x=\rho=4: & 1+\frac{1}{2}+\frac{1}{3}+\frac{1}{4}+\ldots: \text { divergent } \\
x=-\rho=-4: & 1-\frac{1}{2}+\frac{1}{3}-\frac{1}{4} \pm \ldots: \text { konvergent }
\end{array}
$$

Die Potenzreihe konvergiert also für
$
-4 \leq x<4
$
\end{example}

\begin{example}
    Wir betrachten die Binomialreihe für $\alpha \notin \mathbb{N}$ beliebig, d.h. die Taylorreihe der Funktion

    $$
    f(x)=(1+x)^{\alpha}
    $$
    
    für ein beliebiges $\alpha \notin \mathbb{N}$, und den Konvergenzradius $\rho$ dieser Reihe.
    
    a. Empirisch sieht man, dass $\rho=1$ gelten muss. Bestätigen Sie dieses Ergebnis analytisch, indem Sie in die Formel für den Konvergenzradius einsetzen.
    
    b. Warum ist die bei a. durchgeführte Rechnung nicht gültig für den Fall $\alpha \in \mathbb{N}$ ?
\tcblower
a. Einsetzen von $a_{n}=\binom{\alpha}{n}=\frac{\alpha(\alpha-1) \ldots(\alpha-n+1)}{n!}$ in die Formel $\rho=\lim _{n \rightarrow \infty}\left|\frac{a_{n}}{a_{n+1}}\right|$ :

$$
\begin{aligned}
\rho & =\lim _{n \rightarrow \infty}\left|\frac{a_{n}}{a_{n+1}}\right| \\
& =\lim _{n \rightarrow \infty}\left|\frac{\alpha(\alpha-1) \ldots(\alpha-n+1) \cdot(n+1)!}{n!\cdot \alpha(\alpha-1) \ldots(\alpha-n)}\right| \\
& =\lim _{n \rightarrow \infty}\left|\frac{n+1}{\alpha-n}\right| \\
& =1
\end{aligned}
$$

(die letzte Tatsache kann man mit der Regel von Bernoulli-de l'Hospital oder anderen Überlegungen sehen).

b. Im Fall $\alpha \in \mathbb{N}$ ist $\binom{\alpha}{n}=0$ für $\alpha>n$. Deshalb sind die in der Formel $\rho=\lim _{n \rightarrow \infty}\left|\frac{a_{n}}{a_{n+1}}\right|$ auftretenden Quotienten für grosse $n$ alle von der Form $\frac{0}{0}$ und damit undefiniert.
\end{example}


\begin{example}
    Wir betrachten die Funktion
    $
    f(x)=\frac{1}{1-2 x}
    $
    
    Die Taylorreihe von $f(x)$ um $x_{0}=0$ ist
    
    $$
    t_{f}(x)=1+2 x+4 x^{2}+8 x^{3}+\ldots=\sum_{k=0}^{\infty} 2^{k} x^{k}
    $$
    
    
    a. Bestimmen Sie den Konvergenzradius $\rho$ dieser Reihe.
    
    b. Überlegen Sie sich, ob die Reihe auf dem Rand des Konvergenzbereichs konvergiert oder nicht (d.h. für $x=\rho$ und $x=-\rho$ ).

\tcblower
 a. Berechnung des Konvergenzradius:

$$
\rho=\lim _{k \rightarrow \infty}\left|\frac{a_{k}}{a_{k+1}}\right|=\lim _{k \rightarrow \infty} \frac{2^{k}}{2^{k+1}}=\lim _{k \rightarrow \infty} \frac{1}{2}=\frac{1}{2}
$$

b. Untersuchung des Verhaltens auf dem Rand des Konvergenzbereichs:

$$
\begin{array}{rr}
x=\rho=\frac{1}{2}: & 1+1+1+1+1+\ldots: \text { divergent } \\
x=-\rho=-\frac{1}{2}: & 1-1+1-1+1 \mp \ldots: \text { divergent }
\end{array}
$$
\end{example}


