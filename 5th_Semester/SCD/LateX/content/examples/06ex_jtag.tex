\section{Exercise: JTAG}

\begin{remark}
\textbf{Quiz Format:} Mixed (diagram matching, multiple choice, sequence ordering)
\textbf{Score:} 1.00 von 1.00 (100\%)
\textbf{Topics Covered:} JTAG boundary scan, test modes, infrastructure components, TAP controller, timing sequences
\end{remark}

\subsection{JTAG Boundary Scan Cell}

\begin{example2}{Diagram Matching - Boundary Scan Cell Components}\\
\textbf{Source:} Exercise JTAG, Question 1 (Pages 1-2)

\textbf{Topics:} Boundary scan cell, JTAG architecture, bidirectional I/O

\textbf{Question:}

% TODO: Add image from 06_JTAG.pdf, Page 1
% Description: Complete JTAG boundary scan cell schematic showing SDO, SDI, SHIFT, CLOCK, UPDATE, HIGHZ, MODE signals, with components labeled A through G (INPUT cells, OE cell, OUTPUT cells, multiplexers, output buffer). Shows data flow for bidirectional pin with PIN_IN, PIN_OE, and PIN_OUT signals.
% Priority: CRITICAL
% Suggested filename: exercise06_jtag_boundary_scan_cell.png
\\
% \includegraphics[width=\linewidth]{exercise06_jtag_boundary_scan_cell.png}
\\
% WHEN YOU ADD IMAGE: Uncomment line above (remove \%)

The above schematic shows a boundary scan cell for a bidirectional input/output pin of an FPGA. For each labelled part for the schematic, choose the matching description.

\textbf{A:} Reflects the logic state of an FPGA input pin, bypassing the JTAG facilities

\textbf{D:} Part of the scan chain

\textbf{F:} Allows the JTAG controller to set an output pin to high or low

\textbf{G:} Output or input pin

\textbf{E:} Allows the JTAG controller to set an output pin to high-impedance

\textbf{B:} Allows the customers logic implementation to set an output pin to high-impedance

\textbf{C:} Allows the customers logic implementation to set an output pin to high or low

\tcblower

\textbf{Correct Answers:}

\textbf{A →} Reflects the logic state of an FPGA input pin, bypassing the JTAG facilities

\textbf{D →} Part of the scan chain

\textbf{F →} Allows the JTAG controller to set an output pin to high or low

\textbf{G →} Output or input pin

\textbf{E →} Allows the JTAG controller to set an output pin to high-impedance

\textbf{B →} Allows the customers logic implementation to set an output pin to high-impedance

\textbf{C →} Allows the customers logic implementation to set an output pin to high or low

\textbf{Detailed Explanation:}

\textbf{JTAG Boundary Scan Cell Architecture:}

The boundary scan cell acts as an interface between the FPGA internal logic and external pins, allowing both normal operation and test modes.

\textbf{Component Functions:}

\textbf{A - Input Path (Bypass):}
\begin{itemize}
    \item Direct connection from PIN\_IN to internal logic
    \item Used during normal functional mode
    \item Reflects actual pin state without JTAG intervention
    \item Allows real-time monitoring of input signals
\end{itemize}

\textbf{D - Scan Chain Cells:}
\begin{itemize}
    \item Flip-flops connected in series (SDI → SDO)
    \item Three cells shown: INPUT, OE, OUTPUT
    \item Shifted by CLOCK signal when SHIFT is active
    \item Updated to output by UPDATE signal
    \item Form continuous chain through all I/O pins
\end{itemize}

\textbf{F - OUTPUT Test Cell:}
\begin{itemize}
    \item Multiplexer controlled by MODE signal
    \item In test mode: JTAG controller provides output value
    \item Allows external test equipment to drive pins
    \item Used for board-level interconnect testing
\end{itemize}

\textbf{G - Physical Pin:}
\begin{itemize}
    \item Actual I/O pad on the FPGA package
    \item Can be input, output, or bidirectional
    \item Connected through output buffer and input buffer
    \item Interface to external circuitry
\end{itemize}

\textbf{E - OE (Output Enable) Test Cell:}
\begin{itemize}
    \item Controls output buffer tri-state
    \item In test mode: JTAG controls high-Z state
    \item Allows isolation of outputs during testing
    \item Prevents bus conflicts in boundary scan
\end{itemize}

\textbf{B - OE from Logic:}
\begin{itemize}
    \item Normal mode: user logic controls output enable
    \item Multiplexed with JTAG control (E)
    \item Implements bidirectional I/O functionality
    \item Allows tri-state control by FPGA fabric
\end{itemize}

\textbf{C - OUTPUT from Logic:}
\begin{itemize}
    \item Normal mode: user logic provides output value
    \item Multiplexed with JTAG test value (F)
    \item Direct connection to output buffer
    \item Standard output data path
\end{itemize}

\textbf{Operation Modes:}

\textbf{Normal Mode (MODE = 0):}
\begin{itemize}
    \item Path A: PIN\_IN → Internal logic directly
    \item Path B: Logic controls OE (tri-state)
    \item Path C: Logic controls output value
    \item JTAG scan chain transparent
\end{itemize}

\textbf{Test Mode (MODE = 1):}
\begin{itemize}
    \item Path A: Still monitoring input
    \item Path E: JTAG controls OE from scan chain
    \item Path F: JTAG controls output from scan chain
    \item Full control for boundary scan testing
\end{itemize}

\textbf{JTAG Signals:}
\begin{itemize}
    \item \textbf{SDI:} Serial Data In (from previous cell)
    \item \textbf{SDO:} Serial Data Out (to next cell)
    \item \textbf{SHIFT:} Enable scan chain shifting
    \item \textbf{CLOCK:} Clock for scan flip-flops
    \item \textbf{UPDATE:} Transfer scan data to output
    \item \textbf{HIGHZ:} Global high-impedance control
    \item \textbf{MODE:} Select normal vs. test operation
\end{itemize}

\important{Problem type:} Analysis - Understanding JTAG boundary scan cell architecture and signal flow
\end{example2}

\subsection{JTAG Test Modes}

\begin{example2}{Matching - JTAG Testing Types}\\
\textbf{Source:} Exercise JTAG, Question 2 (Page 2)

\textbf{Topics:} JTAG test modes, structural test, functional test, boundary scan, BIST

\textbf{Question:}

Choose the matching test mode for each description about JTAG testing.

\textbf{1.} The test program tests the component without being aware of its function

\textbf{2.} Memory structures are tested with hardware dedicated for testing in the IC

\textbf{3.} Testing the connections of components on a printed circuit board

\textbf{4.} The component is tested for its behavior in normal operation

Options: Structural Test, Built In Selftest, Boundary Scan, Functional Test

\tcblower

\textbf{Correct Answers:}

\textbf{1. The test program tests the component without being aware of its function} → \textbf{Structural Test}

\textbf{2. Memory structures are tested with hardware dedicated for testing in the IC} → \textbf{Built In Selftest}

\textbf{3. Testing the connections of components on a printed circuit board} → \textbf{Boundary Scan}

\textbf{4. The component is tested for its behavior in normal operation} → \textbf{Functional Test}

\textbf{Detailed Explanation:}

\textbf{1. Structural Test:}
\begin{itemize}
    \item Tests physical structure without knowing functionality
    \item Focuses on manufacturing defects
    \item Examples: stuck-at faults, shorts, opens
    \item Uses automatic test pattern generation (ATPG)
    \item Does not verify correct logical operation
    \item Fault coverage metrics (e.g., 95\% stuck-at coverage)
\end{itemize}

\textbf{Typical structural tests:}
\begin{itemize}
    \item Stuck-at-0 and stuck-at-1 faults
    \item Bridging faults (shorts between nets)
    \item Open circuit faults
    \item Transition delay faults
    \item Does not require functional specification
\end{itemize}

\textbf{2. Built-In Self-Test (BIST):}
\begin{itemize}
    \item On-chip hardware generates test patterns
    \item Common for memory structures (MBIST)
    \item Reduces external test equipment requirements
    \item Tests run at-speed (actual operating frequency)
    \item Results compared with expected values on-chip
\end{itemize}

\textbf{BIST components:}
\begin{itemize}
    \item Pattern generator (LFSR, counter, etc.)
    \item Response analyzer (signature register)
    \item Test controller (FSM)
    \item Typically tests RAMs, ROMs, register files
\end{itemize}

\textbf{3. Boundary Scan:}
\begin{itemize}
    \item IEEE 1149.1 standard (JTAG)
    \item Tests board-level interconnections
    \item Detects PCB manufacturing defects
    \item No physical test probes required
    \item Tests shorts, opens between components
\end{itemize}

\textbf{Boundary scan tests:}
\begin{itemize}
    \item EXTEST: Drive outputs, check inputs on adjacent chips
    \item INTEST: Test internal logic through boundary scan
    \item Sample pin values during normal operation
    \item Verify solder joints, PCB traces
\end{itemize}

\textbf{4. Functional Test:}
\begin{itemize}
    \item Verifies correct logical behavior
    \item Tests actual specifications and requirements
    \item Uses real application scenarios
    \item Detects design errors, not just manufacturing faults
    \item Typically performed at system level
\end{itemize}

\textbf{Functional test approaches:}
\begin{itemize}
    \item Black-box testing (specification-based)
    \item Use case scenarios
    \item Regression testing
    \item Performance verification
    \item May not achieve high fault coverage
\end{itemize}

\textbf{Comparison:}

\begin{center}
\begin{tabular}{|l|l|l|l|}
\hline
\textbf{Test Type} & \textbf{Focus} & \textbf{Knowledge Needed} & \textbf{Level} \\
\hline
Structural & Faults & Circuit structure & Die \\
BIST & Memory & On-chip logic & Die \\
Boundary Scan & Connections & Pin locations & Board \\
Functional & Behavior & Specifications & System \\
\hline
\end{tabular}
\end{center}

\important{Problem type:} Definition - Understanding different testing methodologies and their applications
\end{example2}

\subsection{JTAG Infrastructure Components}

\begin{example2}{Matching - JTAG Component Functions}\\
\textbf{Source:} Exercise JTAG, Question 3 (Pages 3-4)

\textbf{Topics:} TAP controller, JTAG signals, registers, state machine

\textbf{Question:}

\textbf{Disclaimer:} This exercise covers material we did not discuss in detail in the lecture. A potential problem in the exam will not be as detailed as this exercise.

% TODO: Add image from 06_JTAG.pdf, Page 3
% Description: JTAG infrastructure block diagram showing TAP (Test Access Port) with TDI, TRST, TMS, TCK inputs, Instruction Register with SEL\_IR, SEL\_SCAN, SEL\_MBIST, SEL\_BYPASS controls, Device ID Register, MBIST Register, Bypass Register, and TDO output with multiplexer. Shows control signals CLOCKIR, UPDATEIR, SHIFTIR, SELECT, RESET, SHIFTDR, UPDATEDR, CLOCKDR connecting to various blocks.
% Priority: CRITICAL
% Suggested filename: exercise06_jtag_infrastructure.png
\\
% \includegraphics[width=0.9\linewidth]{exercise06_jtag_infrastructure.png}
\\
% WHEN YOU ADD IMAGE: Uncomment line above (remove \%)

Describe the Components of the JTAG infrastructure

\textbf{1.} Which JTAG Signal advances the states in the JTAG controller state machine

\textbf{2.} What TAP controller signal will set the \textbf{Device ID Register} to allow its contents to be readout via the scan chain?

\textbf{3.} Which JTAG signal controls the branching of the states in the JTAG controller state machine

\textbf{4.} With which JTAG signal, IC test equipment receives the contents of the scan chains

\textbf{5.} Which JTAG Register transfers data directly from TDI to TDO

\textbf{6.} Which JTAG register controls and checks the IC self tests

\textbf{7.} Which JTAG Register determines which scan chain is currently active

\textbf{8.} Which JTAG signal disables the entire JTAG infrastructure

Options: TCK, SHIFTDR, TMS, TDO, Bypass Register, MBIST Register, Instruction Register, TRST

\tcblower

\textbf{Correct Answers:}

\textbf{1. Which JTAG Signal advances the states in the JTAG controller state machine} → \textbf{TCK}

\textbf{2. What TAP controller signal will set the Device ID Register to allow its contents to be readout via the scan chain?} → \textbf{SHIFTDR}

\textbf{3. Which JTAG signal controls the branching of the states in the JTAG controller state machine} → \textbf{TMS}

\textbf{4. With which JTAG signal, IC test equipment receives the contents of the scan chains} → \textbf{TDO}

\textbf{5. Which JTAG Register transfers data directly from TDI to TDO} → \textbf{Bypass Register}

\textbf{6. Which JTAG register controls and checks the IC self tests} → \textbf{MBIST Register}

\textbf{7. Which JTAG Register determines which scan chain is currently active} → \textbf{Instruction Register}

\textbf{8. Which JTAG signal disables the entire JTAG infrastructure} → \textbf{TRST}

\textbf{Detailed Explanation:}

\textbf{JTAG Signals (TAP - Test Access Port):}

\textbf{1. TCK (Test Clock):}
\begin{itemize}
    \item Clock signal for TAP controller
    \item Advances state machine on rising edge
    \item Synchronizes all JTAG operations
    \item Independent of system clock
    \item Typically 1-10 MHz (slower than system clock)
\end{itemize}

\textbf{2. SHIFTDR (Shift Data Register):}
\begin{itemize}
    \item TAP controller output signal
    \item Enables shifting in selected data register
    \item When active: data shifts through Device ID, MBIST, or Bypass register
    \item Controlled by TAP state machine
    \item Used for both reading and writing register contents
\end{itemize}

\textbf{3. TMS (Test Mode Select):}
\begin{itemize}
    \item Input that controls state machine branching
    \item Sampled on rising edge of TCK
    \item Sequence of TMS values navigates through states
    \item Determines whether to go to IR or DR path
    \item Five consecutive '1's resets to Test-Logic-Reset state
\end{itemize}

\textbf{4. TDO (Test Data Out):}
\begin{itemize}
    \item Serial output from JTAG chain
    \item Connects to test equipment
    \item Multiplexed from active register
    \item Changes on falling edge of TCK
    \item High-impedance when not shifting data
\end{itemize}

\textbf{JTAG Registers:}

\textbf{5. Bypass Register:}
\begin{itemize}
    \item Single-bit register
    \item Provides shortest path from TDI to TDO
    \item Used to bypass devices in multi-device chain
    \item Minimizes scan chain length when not testing this device
    \item Mandatory in all JTAG devices
\end{itemize}

\textbf{6. MBIST (Memory Built-In Self-Test) Register:}
\begin{itemize}
    \item Controls on-chip memory testing
    \item Contains test commands and status
    \item Initiates memory test sequences
    \item Reports pass/fail results
    \item Device-specific implementation
\end{itemize}

\textbf{7. Instruction Register (IR):}
\begin{itemize}
    \item Determines which data register is active
    \item Selects test operation mode
    \item Instructions include: BYPASS, EXTEST, INTEST, SAMPLE, IDCODE
    \item Decodes to control signals (SEL\_IR, SEL\_SCAN, SEL\_MBIST, SEL\_BYPASS)
    \item Minimum 2 bits, typically 4-10 bits
\end{itemize}

\textbf{Example IR instructions:}
\begin{itemize}
    \item BYPASS: Select Bypass Register
    \item IDCODE: Select Device ID Register
    \item EXTEST: External test (boundary scan)
    \item INTEST: Internal test
    \item SAMPLE/PRELOAD: Sample pin values
    \item MBIST: Select MBIST Register
\end{itemize}

\textbf{8. TRST (Test Reset) - Optional:}
\begin{itemize}
    \item Asynchronous reset for TAP controller
    \item Active low (typically)
    \item Forces TAP to Test-Logic-Reset state
    \item Independent of TCK/TMS
    \item Optional signal (can use TMS sequence instead)
    \item Disables entire JTAG infrastructure when asserted
\end{itemize}

\textbf{Other JTAG Components:}

\textbf{Device ID Register:}
\begin{itemize}
    \item Contains manufacturer ID, part number, version
    \item 32-bit register (IEEE 1149.1 standard)
    \item Read-only
    \item Format: Version(4) | Part Number(16) | Manufacturer(11) | 1
    \item Used for automatic device recognition
\end{itemize}

\textbf{TAP Controller State Machine:}
\begin{itemize}
    \item 16-state finite state machine
    \item Two main paths: Instruction Register and Data Register
    \item Reset state: Test-Logic-Reset
    \item Idle state: Run-Test/Idle
    \item Controlled by TMS input
\end{itemize}

\textbf{Control Signals from TAP:}
\begin{itemize}
    \item CLOCKIR, SHIFTIR, UPDATEIR: Instruction Register control
    \item CLOCKDR, SHIFTDR, UPDATEDR: Data Register control
    \item SELECT: Choose between IR and DR path
    \item RESET: Reset TAP controller
\end{itemize}

\important{Problem type:} Application - Understanding JTAG architecture and signal functions
\end{example2}

\subsection{JTAG Operation Sequence}

\begin{example2}{Sequence Ordering - JTAG Timing Diagram}\\
\textbf{Source:} Exercise JTAG, Question 4 (Page 4)

\textbf{Topics:} JTAG timing, TAP states, data capture and shift

\textbf{Question:}

% TODO: Add image from 06_JTAG.pdf, Page 4
% Description: JTAG timing diagram showing TCK, TMS, TDI, TDO, SHIFT\_IR/SHIFT\_DR signals and TAP\_STATE progression. Shows 4 numbered phases with operations like EXIT1\_IR, SELECT\_DR, UPDATE\_IR, CAPTURE\_DR. Annotations show "Instruction Code", "EXIT1\_IR", "SELECT\_DR", "UPDATE\_IR", "CAPTURE\_DR" labels and numbered circles 1-4 marking sequence phases.
% Priority: CRITICAL
% Suggested filename: exercise06_jtag_sequence.png
\\
% \includegraphics[width=\linewidth]{exercise06_jtag_sequence.png}
\\
% WHEN YOU ADD IMAGE: Uncomment line above (remove \%)

Match the descriptions to the numbered phases in the JTAG timing diagram:

\textbf{1.} Pin input signals are captured

\textbf{2.} Use instruction code to select which scan chain is to be used

\textbf{3.} Captured data are shifted out

\textbf{4.} Return to idle state

Options: 1, 2, 3, 4

\tcblower

\textbf{Correct Answers:}

\textbf{Pin input signals are captured} → \textbf{2}

\textbf{Use instruction code to select which scan chain is to be used} → \textbf{1}

\textbf{Captured data are shifted out} → \textbf{3}

\textbf{Return to idle state} → \textbf{4}

\textbf{Detailed Explanation:}

\textbf{JTAG Operation Sequence:}

The timing diagram shows a complete JTAG test sequence with four distinct phases:

\textbf{Phase 1 - Instruction Loading:}
\begin{itemize}
    \item State: SHIFT\_IR (Shift Instruction Register)
    \item TDI shifts in instruction code
    \item Instruction determines which data register to use (EXTEST, INTEST, etc.)
    \item Multiple TCK cycles shift entire instruction
    \item Transitions through EXIT1\_IR → UPDATE\_IR
    \item UPDATE\_IR latches new instruction
\end{itemize}

\textbf{Example instructions:}
\begin{itemize}
    \item EXTEST: Select boundary scan register
    \item SAMPLE: Sample pin values
    \item BYPASS: Select bypass register
    \item IDCODE: Select device ID register
\end{itemize}

\textbf{Phase 2 - Data Capture:}
\begin{itemize}
    \item State: CAPTURE\_DR (Capture Data Register)
    \item Pin input signals are captured into scan cells
    \item One TCK cycle in CAPTURE\_DR state
    \item Parallel load from pins to boundary scan register
    \item Prepares data for shifting out
    \item Transitions through SELECT\_DR to get here
\end{itemize}

\textbf{What gets captured:}
\begin{itemize}
    \item Current logic levels on input pins
    \item Stored in input boundary scan cells
    \item Output values also captured
    \item Forms snapshot of pin states
\end{itemize}

\textbf{Phase 3 - Data Shift:}
\begin{itemize}
    \item State: SHIFT\_DR (Shift Data Register)
    \item Captured data shifts out via TDO
    \item New test data can shift in via TDI simultaneously
    \item Multiple TCK cycles (one per bit in scan chain)
    \item Test equipment receives pin states
    \item Can load new values for output pins
\end{itemize}

\textbf{During shift:}
\begin{itemize}
    \item TDO shows previous cell contents
    \item TDI loads next test pattern
    \item Entire scan chain shifts each TCK
    \item Takes N cycles for N-bit chain
\end{itemize}

\textbf{Phase 4 - Return to Idle:}
\begin{itemize}
    \item State: EXIT1\_DR → UPDATE\_DR → Run-Test/Idle
    \item UPDATE\_DR: New values latched to output pins
    \item Run-Test/Idle: TAP controller idle state
    \item System ready for next operation
    \item Can stay idle or start new sequence
\end{itemize}

\textbf{State transitions:}
\begin{itemize}
    \item Controlled by TMS signal
    \item TMS = 1: Move toward exit/update states
    \item TMS = 0: Stay in current state (shift) or move to idle
\end{itemize}

\textbf{Complete TAP State Sequence:}

\begin{enumerate}
    \item Test-Logic-Reset (power-on state)
    \item Run-Test/Idle (idle between operations)
    \item \textbf{Phase 1:} Select-IR → Capture-IR → Shift-IR → Exit1-IR → Update-IR
    \item \textbf{Phase 2:} Select-DR → Capture-DR
    \item \textbf{Phase 3:} Shift-DR → Exit1-DR
    \item \textbf{Phase 4:} Update-DR → Run-Test/Idle
\end{enumerate}

\textbf{Timing Relationships:}
\begin{itemize}
    \item TMS sampled on rising edge of TCK
    \item TDI sampled on rising edge of TCK
    \item TDO changes on falling edge of TCK
    \item Setup/hold times must be met
\end{itemize}

\textbf{Typical Use Case - Boundary Scan Test:}
\begin{enumerate}
    \item Load EXTEST instruction (Phase 1)
    \item Capture input pin states (Phase 2)
    \item Shift out captured values, shift in new output values (Phase 3)
    \item Update outputs, return to idle (Phase 4)
    \item Repeat to test board interconnects
\end{enumerate}

\important{Problem type:} Application - Understanding JTAG operational sequence and timing
\end{example2}

\raggedcolumns
\columnbreak

% ===== IMAGE SUMMARY =====
% Images needed for this exercise:
% 1. exercise06_jtag_boundary_scan_cell.png (Page 1) - CRITICAL - Complete boundary scan cell schematic
% 2. exercise06_jtag_infrastructure.png (Page 3) - CRITICAL - JTAG infrastructure block diagram
% 3. exercise06_jtag_sequence.png (Page 4) - CRITICAL - JTAG timing diagram with numbered phases
% =====================