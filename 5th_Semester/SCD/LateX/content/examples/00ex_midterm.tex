\section{Midterm Exam HS24}

\textit{Note: This is a comprehensive midterm exam covering multiple System on Chip Design topics. The exam was taken on Tuesday, 5 November 2024, with a time limit of 45 minutes. Grade: 41.19 out of 52.00 (79.21\%).}

\subsection{Boot Process}

\begin{example2}{Boot Process Configuration}

\textbf{Question 1:} The boot process must enable different sections of the processor in a defined sequence. Choose which step in the boot process enables and configures the following parts of the chip:

\begin{itemize}
    \item HPS GPIO pins: \underline{Preloader} \textcolor{frog}{$\surd$}
    \item SD-Controller: \underline{Boot ROM} \textcolor{frog}{$\surd$}
    \item SDRAM-Controller: \underline{Preloader} \textcolor{frog}{$\surd$}
    \item Watchdog: \underline{Boot ROM} \textcolor{frog}{$\surd$}
\end{itemize}

\tcblower

\textbf{Explanation:}

All answers correct.

The boot sequence in SoC FPGA:
\begin{itemize}
    \item \textbf{Boot ROM}: Initializes basic hardware (SD-Controller, Watchdog) and loads the preloader
    \item \textbf{Preloader}: Configures SDRAM controller, GPIO pins, and prepares system for bootloader
    \item \textbf{Bootloader (U-Boot)}: Loads operating system
\end{itemize}

\important{Problem type:} Understanding boot sequence and component initialization

\textbf{Source:} Midterm Exam HS24, Question 1
\end{example2}

\subsection{GPIO Architecture}

\begin{example2}{GPIO Register Circuit Labeling}

\textbf{Question 2:} To store the desired output value, or to keep the input value, registers are added to the GPIO port.

\includegraphics[width=\linewidth]{ex_exam_gpio_registers.png}

On the following image, place the given labels on the corresponding register. Use the labels that were left in the schematic to determine the function of the registers.

The possible spots are marked with a large red X, make sure the label's crosshair is positioned over the X to allow for correct automatic marking.

\textit{Image shows complex GPIO circuit with multiple registers, flip-flops, and control logic including:}
\begin{itemize}
    \item Output Enable Register
    \item Output Register
    \item Input Register
    \item Control signals for read/write from core
    \item PRN (Preset) signals
    \item DQS Logic Block and Dynamic OCT Control
\end{itemize}

\tcblower

\textbf{Explanation:}

Correct placement of all register labels based on GPIO circuit architecture.

\important{Problem type:} Circuit analysis - GPIO register identification

\textbf{Source:} Midterm Exam HS24, Question 2
\end{example2}

\subsection{Block Memory Purpose}

\begin{example2}{FPGA Block RAM Purpose}

\textbf{Question 3:} What is the purpose of the dedicated FPGA Block RAM (M4k, M10k or M20k, not MLABs or LUTs in logic cells):

(Multiple answers are possible)

Select one or more:
\begin{itemize}
    \item Implementation of combinatorial logic \textbf{\textcolor{red}{X}}
    \item For permanent storage of the FPGA configuration \textbf{\textcolor{red}{X}}
    \item To implement shift-registers \textbf{\textcolor{red}{X}}
    \item For implementation of ROMs in FPGA \textcolor{frog}{$\surd$}
    \item For implementation of FIFOs in FPGA \textcolor{frog}{$\surd$}
    \item To implement triple ported RAMs \textcolor{frog}{$\surd$}
\end{itemize}

\tcblower

\textbf{Explanation:}

Correct answers: For implementation of ROMs in FPGA, For implementation of FIFOs in FPGA, To implement triple ported RAMs

Block RAM in FPGAs is used for:
\begin{itemize}
    \item RAM, ROM, and FIFO implementations
    \item Dual-port and triple-port memory structures
    \item Large data storage within FPGA fabric
\end{itemize}

Block RAM cannot store FPGA configuration (requires non-volatile memory) and is inefficient for combinatorial logic.

\important{Problem type:} Understanding Block RAM applications

\textbf{Source:} Midterm Exam HS24, Question 3
\end{example2}

\subsection{Block Memory Timing}

\begin{example2}{Memory Timing Diagram Analysis}

\textbf{Question 4:} Select the correct description for the times as shown in the diagram:

\includegraphics[width=\linewidth]{ex_exam_block_memory_timing.png}

\textit{Timing diagram showing asynchronous memory signals with various timing parameters}

\tcblower

\textbf{Explanation:}

Timing parameters for asynchronous memory operations including setup times, hold times, and access delays.

\important{Problem type:} Timing diagram interpretation

\textbf{Source:} Midterm Exam HS24, Question 4
\end{example2}

\subsection{Block Memory Properties}

\begin{example2}{Block Memory Circuit Analysis}

\textbf{Question 5:} \includegraphics[width=0.8\linewidth]{ex_exam_block_memory_circuit.png}

Which answers are correct for the circuit shown above (several correct answers are possible)?

Select one or more:
\begin{itemize}
    \item a. There is one read and one write address bus, they can be different \textcolor{frog}{$\surd$}
    \item b. Writing with a clock frequency of more than 100MHz is not possible in this design \textbf{\textcolor{red}{X}}
    \item c. It is possible to read from and write to this RAM in the same clock cycle. \textcolor{frog}{$\surd$}
    \item d. The memory is 512 bytes deep (total amount of storage) \textcolor{frog}{$\surd$}
    \item e. 32-bit values can be read at the same time \textbf{\textcolor{red}{X}}
\end{itemize}

\tcblower

\textbf{Explanation:}

Correct answers: a, c, d

Analysis of the memory circuit shows:
\begin{itemize}
    \item Separate read and write address buses (dual-port capability)
    \item Simultaneous read and write operations possible
    \item Memory depth calculation based on address width
\end{itemize}

\important{Problem type:} Block memory architecture analysis

\textbf{Source:} Midterm Exam HS24, Question 5
\end{example2}

\subsection{Block Memory Architecture with Flip-Flops}

\begin{example2}{M4k Block Memory with Output Register}

\textbf{Question 6:} \includegraphics[width=0.7\linewidth]{ex_exam_m4k_block.png}

With the signal \textit{sync\_mem}, it can be selected whether the memory output is directly led to the signal \textit{data\_out}, or whether the data is registered in Flip-Flops before.

Which answers are correct for the block memory (M4k) shown (several correct answers are possible)?

\textbf{With the Flip-Flops present in the output path ...}

Select one or more:
\begin{itemize}
    \item a. ... the maximal clock frequency is higher than without the Flip-Flops present in the path \textcolor{frog}{$\surd$}
    \item b. ... the maximal clock frequency is lower than without the Flip-Flops present in the path \textbf{\textcolor{red}{X}}
    \item c. ... data becomes accessible at the output earlier than without the Flip-Flops present in the path \textbf{\textcolor{red}{X}}
    \item d. ... data becomes accessible at the output later than without the Flip-Flops present in the path \textcolor{frog}{$\surd$}
\end{itemize}

\tcblower

\textbf{Explanation:}

Correct answers: a, d

The output register (flip-flop):
\begin{itemize}
    \item \textbf{Increases maximum clock frequency}: By breaking the combinatorial path from memory to output
    \item \textbf{Increases latency}: Data appears one clock cycle later
    \item Trade-off: Higher throughput (f\_max) vs. longer latency
\end{itemize}

\important{Problem type:} Understanding memory timing trade-offs

\textbf{Source:} Midterm Exam HS24, Question 6
\end{example2}

\subsection{Additional Exam Questions}

\textit{Note: The midterm exam contains additional questions covering topics including:}
\begin{itemize}
    \item SDRAM architecture and timing
    \item Device Tree configuration
    \item JTAG boundary scan
    \item Timing analysis and constraints
    \item PLL configuration
    \item Multi-Gigabit Transceivers
    \item In-Memory Compute concepts
\end{itemize}

\textit{Due to the extensive length of the exam (15 sections, 52 points total), only the first 6 questions are fully detailed here. The complete exam covers all major SCD topics studied during the semester.}

\raggedcolumns
\columnbreak