\section{KR and Exercises: In-Memory Compute (Part A)}

\subsection{In-Memory Compute Concept}

\begin{example2}{IMC Technology Identification}

\textbf{Question:} The following drawing shows a new technological approach for processors.

%\includegraphics[width=0.6\linewidth]{ex_imc_architecture.png}

a) This approach is an example of: \underline{\hspace{4cm}} .

b) This approach is \underline{\hspace{4cm}} for high-precision calculations.

\tcblower

\textbf{Explanation:}

Correct answers:

a) This approach is an example of: \textbf{in-memory compute}

b) This approach is \textbf{not suitable} for high-precision calculations.

\important{Problem type:} Identifying in-memory compute architecture and its limitations

\textbf{Source:} Moodle Quiz In-Memory Compute, Question 1
\end{example2}

\subsection{Performance Bottlenecks}

\begin{example2}{CPU Memory Bottlenecks}

\textbf{Question:} The standard approach for a CPU using an ALU and memory has several bottlenecks. From the below list, select all items that present a performance bottleneck for data-heavy algorithms.

Incorrect answers deduct points. Points can not go negative.

Wählen Sie eine oder mehrere Antworten:
\begin{itemize}
    \item Number of instruction pipeline stages \textbf{\textcolor{red}{X}}
    \item Energy requirements of data transfer \textcolor{frog}{$\surd$}
    \item Memory access delay \textcolor{frog}{$\surd$}
    \item Cache size \textcolor{frog}{$\surd$}
\end{itemize}

\tcblower

\textbf{Explanation:}

Die Antwort ist richtig.

Die richtigen Antworten sind:
Cache size,
Memory access delay,
Energy requirements of data transfer

\important{Problem type:} Understanding memory bottlenecks in conventional CPU architectures

\textbf{Source:} Moodle Quiz In-Memory Compute, Question 2
\end{example2}

\raggedcolumns
\columnbreak

\section{KR and Exercises: In-Memory Compute (Part B)}

\subsection{Workload Characteristics}

\begin{example2}{Suitable Workloads for IMC}

\textbf{Question:} Which workload characteristic makes an application particularly suitable for in-memory or near-memory compute acceleration?

Wählen Sie eine Antwort:
\begin{itemize}
    \item It requires strict single-threaded execution with frequent synchronization. \textbf{\textcolor{red}{X}}
    \item It fits entirely into on-chip caches with no main memory accesses. \textbf{\textcolor{red}{X}}
    \item It is dominated by control flow and complex branching with little data reuse. \textbf{\textcolor{red}{X}}
    \item It is highly data-intensive with simple operations applied repeatedly to large data sets. \textcolor{frog}{$\surd$}
\end{itemize}

\tcblower

\textbf{Explanation:}

Die Antwort ist richtig.

Die richtige Antwort ist: It is highly data-intensive with simple operations applied repeatedly to large data sets.

\important{Problem type:} Understanding workload characteristics suitable for IMC acceleration

\textbf{Source:} Moodle Quiz In-Memory Compute (New Problems), Question 1
\end{example2}

\subsection{Analog Computation in Memory}

\begin{example2}{Analog IMC Concept}

\textbf{Question:} In the context of In-Memory Compute, what does the term 'analog computation in memory' typically refer to?

Wählen Sie eine Antwort:
\begin{itemize}
    \item Implementing quantum bits inside DRAM cells for superposition-based computation. \textbf{\textcolor{red}{X}}
    \item Performing digital logic gates strictly using CMOS standard cells outside the memory array. \textbf{\textcolor{red}{X}}
    \item Exploiting the continuous electrical properties of memory cells and bitlines to perform operations like vector-matrix multiplication. \textcolor{frog}{$\surd$}
    \item Using precise floating-point ALUs inside memory arrays. \textbf{\textcolor{red}{X}}
\end{itemize}

\tcblower

\textbf{Explanation:}

Die Antwort ist richtig.

Die richtige Antwort ist: Exploiting the continuous electrical properties of memory cells and bitlines to perform operations like vector-matrix multiplication.

\important{Problem type:} Understanding analog computation in memory concept

\textbf{Source:} Moodle Quiz In-Memory Compute (New Problems), Question 2
\end{example2}

\subsection{IMC vs NMC Differences}

\begin{example2}{In-Memory vs Near-Memory Compute}

\textbf{Question:} Which of the following best captures a key difference in typical use-cases for In-Memory Compute versus Near-Memory Compute?

Wählen Sie eine Antwort:
\begin{itemize}
    \item IMC commonly accelerates operations like bitwise logic or analog vector-matrix multiplies inside arrays, whereas NMC often accelerates more general kernels near memory using programmable or specialized logic. \textcolor{frog}{$\surd$}
    \item IMC and NMC both only accelerate floating-point matrix multiplication workloads. \textbf{\textcolor{red}{X}}
    \item IMC is ideal for fine-grained, irregular control tasks, while NMC targets only bulk bitwise operations. \textbf{\textcolor{red}{X}}
    \item IMC is used solely for security applications, while NMC is used solely for graphics. \textbf{\textcolor{red}{X}}
\end{itemize}

\tcblower

\textbf{Explanation:}

Die Antwort ist richtig.

Die richtige Antwort ist: IMC commonly accelerates operations like bitwise logic or analog vector-matrix multiplies inside arrays, whereas NMC often accelerates more general kernels near memory using programmable or specialized logic.

\important{Problem type:} Understanding differences between IMC and NMC

\textbf{Source:} Moodle Quiz In-Memory Compute (New Problems), Question 3
\end{example2}

\raggedcolumns
\columnbreak