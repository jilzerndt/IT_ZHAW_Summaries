\section{KR and Exercises: Multi-Gigabit Transceiver}

\subsection{8b10b Encoding}

\begin{example2}{Word Synchronization in 8b10b}

\textbf{Question:} A continuous, serial transmission needs a means of detecting the start of a word. Describe in your own words how 8b10b encoding solves this challenge.

\tcblower

\textbf{Explanation:}

8b/10b encoding detects the start of a word by periodically transmitting special control characters (commas) with unique, illegal-for-data bit patterns. The receiver searches the serial stream for these patterns and, once found, aligns its word boundaries accordingly.

Key points:
\begin{itemize}
    \item A comma has a distinctive pattern with five consecutive identical bits (11111 or 00000), which cannot occur in normal data symbols
    \item Receiver continuously scans the incoming bitstream for this pattern
    \item When the comma is detected, the receiver knows exactly where a 10-bit symbol boundary is
    \item From that point on, all following bits can be grouped correctly into 10-bit words → and therefore into the original 8-bit bytes
\end{itemize}

A special comma symbol with an unambiguous pattern allows the detection of the word border. The line code must guarantee that no combination of two code words leads to a shifted version of the comma pattern.

\important{Problem type:} Explanation - Understanding word synchronization in 8b10b encoding

\textbf{Source:} Moodle Quiz MGT, Question 1
\end{example2}

\subsection{Line Code Properties}

\begin{example2}{8b10b Encoding Properties}

\textbf{Question:} A line code has several properties that help the successful transmission of data via a serial connection. Below, four possible properties are shown.

Mark for each property whether 8b10b encoding provides this property.

\begin{center}
\begin{tabular}{|l|c|c|}
\hline
\textbf{Property} & \textbf{True} & \textbf{False} \\
\hline
Maximum number of transitions & & \textcolor{frog}{$\surd$} \\
\hline
Word synchronization & \textcolor{frog}{$\surd$} & \\
\hline
Clock recovery & \textcolor{frog}{$\surd$} & \\
\hline
DC balance & \textcolor{frog}{$\surd$} & \\
\hline
\end{tabular}
\end{center}

\tcblower

\textbf{Explanation:}

Correct answers:

\begin{itemize}
    \item Maximum number of transitions: False (8b10b does not limit maximum transitions, but ensures minimum transitions)
    \item Word synchronization: True (provided through comma characters)
    \item Clock recovery: True (sufficient transitions for CDR)
    \item DC balance: True (running disparity control ensures DC balance)
\end{itemize}

\important{Problem type:} Understanding 8b10b line code properties

\textbf{Source:} Moodle Quiz MGT, Question 2
\end{example2}

\subsection{MGT Transmission Steps}

\begin{example2}{MGT TX Processing Steps}

\textbf{Question:} A Multi-Gigabit Transceiver must perform multiple steps when transmitting (TX) or receiving (RX) data. Below, a few steps are given for TX. Drag the steps into the correct order with the first step at the top.

\textbf{Correct order:}
\begin{enumerate}
    \item Parallel Input
    \item Line Encoding
    \item Serialization
    \item Output
\end{enumerate}

\tcblower

\textbf{Explanation:}

Die Antwort ist richtig.

The MGT transmitter processes data in this sequence:
\begin{enumerate}
    \item \textbf{Parallel Input}: Data arrives in parallel form from FPGA fabric
    \item \textbf{Line Encoding}: Data is encoded (e.g., 8b10b) for transmission
    \item \textbf{Serialization}: Parallel encoded data is converted to serial bitstream
    \item \textbf{Output}: Serial data is transmitted via high-speed differential lines
\end{enumerate}

\important{Problem type:} Understanding MGT transmit data flow

\textbf{Source:} Moodle Quiz MGT, Question 3
\end{example2}

\raggedcolumns
\columnbreak