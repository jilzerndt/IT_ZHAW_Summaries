\section{Exercise: Booting and Device Tree}

\begin{remark}
\textbf{Quiz Format:} Multiple choice
\textbf{Duration:} 5 minutes 49 seconds
\textbf{Score:} 2.00 von 10.00 (20\%)
\textbf{Topics Covered:} Boot process, Boot ROM, Device Tree, SD card access, DDR SDRAM configuration
\end{remark}

\subsection{Boot Process Sequence}

\begin{example2}{Multiple Choice - Boot ROM Functions}\\
\textbf{Source:} Exercise Booting and Device Tree, Question 1

\textbf{Topics:} Boot process, Boot ROM, initialization, SD card

\textbf{Question:}

During the boot process, a sequence of several steps is followed. One of them is executed from the Boot ROM.

Of the following statements, check all that are true.

Select one or more answers:
\begin{itemize}
    \item The boot ROM loads the first user-definable program into on-chip RAM
    \item The Boot ROM sets up access to the SD card.
    \item The Boot ROM is the first step that can be reprogrammed by the user.
    \item The Boot ROM configures the external DDR SDRAM device.
\end{itemize}

\tcblower

\textbf{Correct Answers:}
\begin{itemize}
    \item The Boot ROM sets up access to the SD card.
    \item The boot ROM loads the first user-definable program into on-chip RAM
\end{itemize}

\textbf{Explanation:}

Your answer is incorrect.

Die richtigen Antworten sind: (The correct answers are:)
\begin{itemize}
    \item The Boot ROM sets up access to the SD card.
    \item The boot ROM loads the first user-definable program into on-chip RAM
\end{itemize}

The Boot ROM is executed immediately after power-on or reset. It is a fixed program stored in on-chip ROM that cannot be modified by the user. The Boot ROM performs the following functions:
\begin{enumerate}
    \item Initializes basic system components
    \item Sets up access to boot devices (SD card, QSPI flash, NAND flash, etc.)
    \item Loads the first user-definable program (typically the preloader or SPL) from the boot device into on-chip RAM
    \item Transfers control to the loaded program
\end{enumerate}

The Boot ROM does NOT configure the external DDR SDRAM - this is done by the preloader/SPL that is loaded by the Boot ROM.

The Boot ROM itself cannot be reprogrammed as it is fixed in silicon.

\important{Problem type:} Explanation - Understanding boot sequence and Boot ROM functionality
\end{example2}

\subsection{Device Tree Purpose}

\begin{example2}{Multiple Choice - Device Tree Functionality}\\
\textbf{Source:} Exercise Booting and Device Tree, Question 2

\textbf{Topics:} Device Tree, hardware description, Linux kernel, peripherals

\textbf{Question:}

What is the primary purpose of the Device Tree in embedded Linux systems?

Select one answer:
\begin{itemize}
    \item To store user applications and data
    \item To describe hardware configuration and peripherals to the kernel
    \item To provide a graphical user interface
    \item To manage network connections
\end{itemize}

\tcblower

\textbf{Correct Answer:} To describe hardware configuration and peripherals to the kernel

\textbf{Explanation:}

The Device Tree is a data structure that describes the hardware configuration of the system to the operating system kernel. It provides information about:
\begin{itemize}
    \item Available hardware devices and peripherals
    \item Memory-mapped register addresses
    \item Interrupt lines
    \item Clock connections
    \item Pin configurations
    \item Device-specific properties
\end{itemize}

The Device Tree allows the same kernel binary to run on different hardware configurations by providing the hardware-specific information at boot time. This separates hardware description from the kernel code, making the kernel more portable and easier to maintain.

The Device Tree does not store user applications, provide GUI functionality, or manage network connections - these are handled by other parts of the system.

\important{Problem type:} Definition - Understanding Device Tree purpose and functionality
\end{example2}

\raggedcolumns
\columnbreak

% ===== IMAGE SUMMARY =====
% No images needed for this exercise (text-only multiple choice questions)
% =====================