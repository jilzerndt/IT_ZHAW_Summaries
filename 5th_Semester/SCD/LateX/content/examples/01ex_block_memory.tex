\section{KR and Exercises: Block Memory}

\subsection{Implementations in SRAM-based FPGAs}

\begin{example2}{FIFO Implementation}

\textbf{Question:} How can FIFOs be implemented in SRAM based FPGAs? (Several Answers may be correct)
\begin{itemize}
    \item FPGA memory blocks can be used for FIFOs \textcolor{frog}{$\surd$}
    \item The D-Flip-Flops in the Logic Elements may be used for FIFOs \textcolor{frog}{$\surd$}
    \item The Cyclone V SoC FPGA has hard macros exclusively made for FIFOs \textbf{\textcolor{red}{X}}
    \item The external confuration flash may be used for FIFOs \textbf{\textcolor{red}{X}}
\end{itemize}

\tcblower

\textbf{Explanation:}

A FIFO can be implemented using flip flops if you only need a small FIFO. As soon as you need more than a few hundred bits of storage, using block memories is more efficient.

The external configuration flash memory is not suitable for FIFO usage as it is slow and has a limited number of write cycles.

There are no hard macros (silicon) that can only be used as a FIFO, since a block memory with some added logic is just as efficient and can also be used for other purposes. This makes more versatile use of the silicon area.

\important{Problem type:} Application - Understanding FPGA memory resources and FIFO implementation options

\textbf{Source:} Moodle Quiz Block Memory, Question 1
\end{example2}

\raggedcolumns

\subsection{Block Memory Architecture}

\begin{example2}{Block Memory Circuit Analysis}

\includegraphics[width=0.7\linewidth]{ex_block_memory_circuit_analysis.png}

\textbf{Question:} Which answers are correct for the circuit shown (several correct answers are possible)?

\begin{itemize}
    \item In all cases, all 32 bits of the 'data' signal are written to the memory at once. \textbf{\textcolor{red}{X}}
    \item Write data, wraddress, rdaddress, wren, q, byteen\_a are in the same clock domain. \textcolor{frog}{$\surd$}
    \item The 'data' signal may be in a different clock domain than the output 'q'. \textbf{\textcolor{red}{X}}
    \item It is possible to write individual 8-bit bytes to this 32-bit wide memory. \textcolor{frog}{$\surd$}
\end{itemize}

\tcblower

\textbf{Explanation:}

The byte enable signals allow the individual writing of 8-bit bytes to this 32-bit wide memory.

In the schematic shown, all control, address and data ('data' and 'q') signals are registered in flip flops that are connected to the same clock signal.

\important{Problem type:} Analysis - Circuit interpretation and signal timing analysis

\textbf{Source:} Moodle Quiz Block Memory, Question 2
\end{example2}

\begin{example2}{M4k Block Memory Architecture}

\textbf{Description:} M4k Block diagram showing DATA\_IN, ADRESSEN, WREN, CLK signals connected to a Speicher Zelle (memory cell) with flip-flops (D Q blocks) and sync\_mem, outputting to data\_out

\includegraphics[width=0.8\linewidth]{ex_block_memory_architecture_m4k.png}

\textbf{Question:}

Which answers are correct for the block memory (M4k) shown (several correct answers are possible)?

Select one or more answers:
\begin{itemize}
    \item If the Flip-flop in the data out path is bypassed, the processor may require less wait states \textcolor{frog}{$\surd$}
    \item Engaging the flip-flop of the data out path allows higher clock frequencies compared to when the Flip Flop is bypassed. \textcolor{frog}{$\surd$}
    \item Engaging the Flip-flop of the data out path increases in all cases the performance of the connected processor \textbf{\textcolor{red}{X}}
    \item The Flip-flop in the data out path reduces memory access bandwidth \textbf{\textcolor{red}{X}}
\end{itemize}

\important{Problem type:} Analysis - Memory architecture trade-offs between latency and clock frequency

\textbf{Source:} Moodle Quiz Block Memory, Question 3
\end{example2}

\begin{example2}{Block Memory Width and Addressing}

\includegraphics[width=0.6\linewidth]{ex_block_mem_sizes.png}

\textbf{Question:}

Which answers are correct for the graphic above (several correct answers are possible)?

\begin{itemize}
    \item The memory is 64 bytes deep (total amount of storage). \textbf{\textcolor{red}{X}}
    \item 32-bit values can be read at the same time \textcolor{frog}{$\surd$}
    \item 8-bit values can be written at the same time \textcolor{frog}{$\surd$}
    \item The memory is 256 bytes deep (total amount of storage). \textbf{\textcolor{red}{X}}
\end{itemize}

\tcblower

\textbf{Explanation:}

The 'data' signal shows the width of writing to the memory, the 'q' signal shows reading width.

The writing address width allows to determine the number of words in the memory: $2^5 = 32$ words. With its writing width of 8-bit, the memory can store 32 bytes.

\important{Problem type:} Analysis - Circuit interpretation and Memory Width

\textbf{Source:} Moodle Quiz Block Memory, Question 4
\end{example2}

\subsection{Block Memory Functionality}


\begin{example2}{Purpose of FPGA Block RAM}

\textbf{Question:}

What is the purpose of FPGA Block RAM (not LUT in logic cells):

\begin{itemize}
    \item Implementation of combinatorial logic \textbf{\textcolor{red}{X}}
    \item For permanent storage of the FPGA configuration \textbf{\textcolor{red}{X}}
    \item For implementation of FIFO in FPGA \textcolor{frog}{$\surd$}
    \item For implementation of ROM in FPGA \textcolor{frog}{$\surd$}
\end{itemize}

\tcblower

\textbf{Explanation:}

Since the block memory is in the FPGA fabric, it can not be accessed before the fabric is configured. This prevents the use of block memory as storage for the FPGA bitstream (the configuration). Also, block memories use SRAM cells: they lose their memory content as soon as power supply is interrupted.

Block memories can be used as RAM, ROM or FIFO. The use of block memories as register banks is no different from the use as RAM: acccessing a register by an address corresponds to accessing RAM.

The use as combinatorial logic cells (analogous to the use of LUT to implement combinatorial logic) is feasible but inefficient. Block memories are slower than LUT and too large for common combinatorial logic functions.

\important{Problem type:} Purpose and functionality of block memory

\textbf{Source:} Moodle Quiz Block Memory, Question 5
\end{example2}

\begin{example2}{Byte Enable Functionality}

\textbf{Question:} What is the purpose of byte enables in block memories?

\begin{itemize}
    \item Byte enables allow the possibility writing misaligned words in a byte processor system \textcolor{frog}{$\surd$}
    \item Reading memory in burst mode \textbf{\textcolor{red}{X}}
    \item If you want to connect an 8-bit page memory with 32-bit soft processor core \textbf{\textcolor{red}{X}}
    \item If individual bytes of a 32-bit wide memory should be written individually \textcolor{frog}{$\surd$}
\end{itemize}

\tcblower

\textbf{Explanation:}

The byte\_enable signal allows to write individual bytes to a memory with larger word size. For instance, a 32-bit wide memory would have four byte\_enable signals to select which of the four bytes on the bus need to be stored and which don't.

This allows writing smaller data types to the memory.

\important{Problem type:} Explanation - Understanding byte enable functionality in memory systems

\textbf{Source:} Moodle Quiz Block Memory, Question 6
\end{example2}

\raggedcolumns
\pagebreak


\subsection{Memory Timing}

\begin{example2}{Asynchronous Memory Write Cycle Timing Parameters}

\includegraphics[width=\linewidth]{ex_block_memory_asynchronous_memory.png}

\textbf{Question:}

For the given timing parameters for a write cycle, select the correct explanation:

\begin{itemize}
    \item tPWE: The minimum duration the signal must be active to insure data is written properly
    \item tHA: The time address must be kept after writing the memory
    \item tDH: The time data must be kept after writing the memory
    \item tSA: The time address must be stable before writing the memory
    \item tSD: The time data must be stable before writing the memory
\end{itemize}

\tcblower

\textbf{Explanation:}

\begin{itemize}
    \item tSA: address setup time, the address signals must be valid and stable for a certain time before the writeenable signal may be asserted.
    \item tHA: address hold time, the address signals must be kept (held) stable for a certain time after the writeenable signal is deasserted.
    \item tDS: data setup time, the data signals must be valid and stable for a certain time before the writeenable signal may be asserted.
    \item tDH: data hold time, the data signals must be kept (held) stable for a certain time after the writeenable signal is deasserted. Data will be stored at the rising edge of the writeenable signal
\end{itemize}
\important{Problem type:} Explanation - Memory timing parameter definitions and constraints

\textbf{Source:} Moodle Quiz Block Memory, Question 7
\end{example2}

\begin{example2}{Asynchronous Memory Access}

\includegraphics[width=\linewidth]{ex_block_memory_async_timing2.png}

\textbf{Question:}

For the three given timing parameters, select the correct explanation:

\textbf{Correct Answers:}
\begin{itemize}
    \item tOHA: If subsequent read cycles follow in too short distances, data outputs are never stable.
    \item tAA: If processor clocks data in too early after address is valid, data may be wrong.
    \item tRC: If processor removes address too early, read data already changed.
\end{itemize}

\tcblower

\textbf{Explanation:}
\begin{itemize}
    \item tOHA 'time output hold address': defines how long the output data is still valid after the change of the address signals.
    \item tAA 'address to access time': defines how long the memory block needs to provide valid output data. If data is read too early, it may not be valid.
    \item tRC 'read cycle': defines the minimum time for a read cycle.
\end{itemize}

\important{Problem type:} Explanation - Memory timing parameter definitions and constraints

\textbf{Source:} Moodle Quiz Block Memory, Question 8
\end{example2}

\important{TODO:} 
\begin{itemize}
    \item add KR on Memory Timing
    \item add KR that lists and explains any and all abbreviations plus an explanation that might occur in an exercise like this (see lectures and script to find the complete list)
\end{itemize}

\raggedcolumns
\columnbreak
