\section{Exercise: HPS and SoC}

\begin{remark}
\textbf{Quiz Format:} Multiple choice
\textbf{Duration:} 9 minutes 23 seconds
\textbf{Score:} 6.00 von 8.00 (75\%)
\textbf{Topics Covered:} HPS architecture, SoC FPGA, custom peripherals, software execution
\end{remark}

\subsection{Custom Peripheral Integration}

\begin{example2}{Multiple Choice - Custom Peripheral Workflow}\\
\textbf{Source:} Exercise HPS and SoC, Question 1

\textbf{Topics:} HPS, FPGA fabric, custom peripherals, design workflow

\textbf{Question:}

Which step comes first when adding a custom peripheral to the FPGA fabric for HPS access?

Select one answer:
\begin{itemize}
    \item Configuring the HPS in Platform Designer
    \item Programming the device with Quartus
    \item Writing a Linux device driver
    \item Designing the peripheral in VHDL or Verilog
\end{itemize}

\tcblower

\textbf{Correct Answer:} Designing the peripheral in VHDL or Verilog

\textbf{Explanation:}

Die Antwort ist falsch. (The answer given was false.)

Die richtige Antwort ist: Designing the peripheral in VHDL or Verilog

The workflow for adding a custom peripheral to the FPGA fabric that can be accessed by the HPS is:
\begin{enumerate}
    \item First, design the peripheral in VHDL or Verilog
    \item Then configure the HPS and peripheral connections in Platform Designer
    \item Program the device with Quartus
    \item Finally, write a Linux device driver to interface with the peripheral
\end{enumerate}

\important{Problem type:} Design - Understanding HPS peripheral integration workflow
\end{example2}

\subsection{Software Execution Component}

\begin{example2}{Multiple Choice - SoC FPGA Components}\\
\textbf{Source:} Exercise HPS and SoC, Question 2

\textbf{Topics:} SoC FPGA, HPS, operating systems, software applications

\textbf{Question:}

In a SoC FPGA, which component is responsible for running operating systems and complex software applications?

Select one answer:
\begin{itemize}
    \item Hard processor system (HPS)
    \item FPGA fabric
    \item DMA controller
    \item Bus bridge
\end{itemize}

\tcblower

\textbf{Correct Answer:} Hard processor system (HPS)

\textbf{Explanation:}

Die Antwort ist richtig. (The answer is correct.)

Die richtige Antwort ist: Hard processor system (HPS)

The hard processor system (HPS) is a dedicated ARM-based processor integrated into the SoC FPGA. It is specifically designed to run operating systems like Linux and complex software applications. The FPGA fabric is used for custom hardware logic, the DMA controller handles data transfers, and the bus bridge provides connectivity between components.

\important{Problem type:} Explanation - Understanding SoC FPGA architecture components
\end{example2}

\subsection{HPS Advantages in SoC}

\begin{example2}{Multiple Choice - HPS Benefits}\\
\textbf{Source:} Exercise HPS and SoC, Question 3

\textbf{Topics:} HPS, performance, power consumption, softcore processors

\textbf{Question:}

What is the main advantage of using a hard processor system (HPS) in an SoC FPGA?

Select one answer:
\begin{itemize}
    \item It allows for reconfiguration of the processor at runtime
    \item It enables direct analog signal processing
    \item It eliminates the need for any memory controllers
    \item It provides higher performance and lower power consumption compared to softcore processors
\end{itemize}

\tcblower

\textbf{Correct Answer:} It provides higher performance and lower power consumption compared to softcore processors

\textbf{Explanation:}

Die Antwort ist richtig. (The answer is correct.)

Die richtige Antwort ist: It provides higher performance and lower power consumption compared to softcore processors

Hard processor systems (HPS) are implemented as dedicated silicon (ASIC) rather than using FPGA fabric resources. This provides several advantages:
\begin{itemize}
    \item Higher performance - dedicated processor cores run faster than softcore implementations
    \item Lower power consumption - ASIC implementation is more power-efficient than FPGA logic
    \item Lower resource usage - frees up FPGA fabric for other custom logic
\end{itemize}

The HPS cannot be reconfigured at runtime (it's fixed silicon), does not directly process analog signals (requires ADC/DAC), and does not eliminate the need for memory controllers (though it has integrated controllers).

\important{Problem type:} Comparison - HPS vs. softcore processor trade-offs
\end{example2}

\subsection{SoC FPGA Architecture}

\begin{example2}{Multiple Choice - SoC Architecture Definition}\\
\textbf{Source:} Exercise HPS and SoC, Question 4

\textbf{Topics:} SoC FPGA, architecture, CPU, integration

\textbf{Question:}

Which of the following best describes a System-on-Chip (SoC) FPGA architecture?

Select one answer:
\begin{itemize}
    \item Multiple CPUs with no programmable logic
    \item An FPGA with only softcore processors
    \item A CPU and FPGA fabric integrated on a single chip
    \item A CPU and FPGA on separate chips connected by a PCB
\end{itemize}

\tcblower

\textbf{Correct Answer:} A CPU and FPGA fabric integrated on a single chip

\textbf{Explanation:}

Die Antwort ist richtig. (The answer is correct.)

Die richtige Antwort ist: A CPU and FPGA fabric integrated on a single chip

A System-on-Chip (SoC) FPGA combines:
\begin{itemize}
    \item A hard processor system (CPU) - typically ARM-based
    \item FPGA programmable logic fabric
    \item Memory controllers and peripherals
    \item High-speed interconnects
\end{itemize}

All integrated on a single silicon die. This integration provides benefits including:
\begin{itemize}
    \item Lower latency communication between CPU and FPGA
    \item Reduced board space and power consumption
    \item Shared memory interfaces
    \item Better system performance
\end{itemize}

This is different from separate CPU and FPGA chips connected via PCB, or FPGAs with only softcore processors implemented in fabric.

\important{Problem type:} Definition - Understanding SoC FPGA architecture fundamentals
\end{example2}

\raggedcolumns
\columnbreak

% ===== IMAGE SUMMARY =====
% No images needed for this exercise (text-only multiple choice questions)
% =====================