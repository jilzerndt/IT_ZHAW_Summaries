\section{General Purpose Input/Output (GPIO)}

\subsection{Learning Objectives}

\begin{concept}{Course Goals}\\
After this lecture and lab exercise, students will be able to:
\begin{itemize}
    \item Sketch the logic circuit of GPIO pins
    \item Explain the function of a configurable GPIO pin
    \item Configure GPIO pins in an SoC design
\end{itemize}
\end{concept}

\subsection{Agenda}

\begin{remark}
This lecture covers three main topics:
\begin{itemize}
    \item Applications and different modes of GPIO pins
    \item GPIO Internals: schematic and capabilities
    \item GPIO in a Platform Design
\end{itemize}
\end{remark}

\subsection{Typical Applications for GPIO}

% TODO: Add image from SCD_5a_GPIO.pdf, Page 5
% Description: Typical GPIO applications and use cases
% Priority: CRITICAL
% Suggested filename: lecture05A_gpio_applications.png
% \includegraphics[width=\linewidth]{lecture05A_gpio_applications.png}
% WHEN YOU ADD IMAGE: Uncomment line above

\begin{concept}{GPIO Applications}\\
GPIO (General Purpose Input/Output) pins are versatile digital interfaces used for:
\begin{itemize}
    \item LED control and status indicators
    \item Button and switch inputs
    \item Sensor interfacing
    \item Simple protocol communication
    \item External device control
    \item Interrupt generation
\end{itemize}
\end{concept}

\subsection{GPIO Operating Modes}

% TODO: Add image from SCD_5a_GPIO.pdf, Page 6
% Description: GPIO modes diagram showing output and bidirectional configurations
% Priority: CRITICAL
% Suggested filename: lecture05A_gpio_modes.png
% \includegraphics[width=\linewidth]{lecture05A_gpio_modes.png}
% WHEN YOU ADD IMAGE: Uncomment line above

\begin{definition}{GPIO Modes}
\begin{itemize}
    \item \textbf{Output Mode:} Pin drives signal to external circuit
    \item \textbf{Input Mode:} Pin reads signal from external circuit
    \item \textbf{Bidirectional (Tristate):} Pin can switch between input and output modes
\end{itemize}
\end{definition}

\subsection{GPIO Internal Architecture}

% TODO: Add image from SCD_5a_GPIO.pdf, Page 7
% Description: Detailed GPIO internal schematic showing registers, buffers, and control logic
% Priority: CRITICAL
% Suggested filename: lecture05A_gpio_internals.png
% \includegraphics[width=\linewidth]{lecture05A_gpio_internals.png}
% WHEN YOU ADD IMAGE: Uncomment line above

\begin{concept}{GPIO Hardware Components}\\
The GPIO pin structure includes several configurable elements:

\textbf{Core Components:}
\begin{itemize}
    \item \textbf{Input Buffer:} Conditions incoming signals for the FPGA core
    \item \textbf{Output Buffer:} Drives signals to external pins with programmable strength
    \item \textbf{Input Register:} Synchronizes and stores input data
    \item \textbf{Output Register:} Holds output data to be driven
    \item \textbf{Output Enable (OE) Register:} Controls direction (input/output)
\end{itemize}

\textbf{Programmable Features:}
\begin{itemize}
    \item Programmable driver strength
    \item Programmable slew rate control
    \item On-chip termination
    \item Internal pull-up resistor
    \item Bus-hold circuit
    \item Delay control
    \item Open-drain output option
\end{itemize}
\end{concept}

\begin{definition}{Output Enable Logic}\\
The Output Enable (OE) signal controls pin direction:
\begin{itemize}
    \item \texttt{OE = 1}: Output mode - pin drives signal
    \item \texttt{OE = 0}: Input mode - pin is high-impedance (tristate)
\end{itemize}
\end{definition}

\begin{remark}
All GPIO features are configurable through the Quartus II/Prime software during FPGA compilation. These settings optimize signal integrity and power consumption for specific applications.
\end{remark}

\raggedcolumns
\columnbreak

\subsection{GPIO Hardware Features}

\begin{definition}{Key Hardware Capabilities}
\begin{itemize}
    \item \textbf{Input and Output Buffers:} Provide electrical interface with triggers and drivers
    \item \textbf{Registers:} Synchronize data transfer (Input, Output, and Output Enable registers)
    \item \textbf{Driver Strength Control:} Adjustable output current capability
    \item \textbf{Slew Rate Control:} Controls signal transition speed to reduce EMI
    \item \textbf{On-chip Termination:} Provides impedance matching without external components
    \item \textbf{Pull-Up Resistor:} Internal weak pull-up for floating inputs
    \item \textbf{Bus-Hold Circuit:} Maintains last logic level when input is floating
\end{itemize}
\end{definition}

\subsection{FPGA vs HPS GPIO Architecture}

% TODO: Add image from SCD_5a_GPIO.pdf, Page 9
% Description: SoC diagram showing FPGA and HPS I/O column and row placement
% Priority: CRITICAL
% Suggested filename: lecture05A_fpga_vs_hps_io.png
% \includegraphics[width=\linewidth]{lecture05A_fpga_vs_hps_io.png}
% WHEN YOU ADD IMAGE: Uncomment line above

\begin{concept}{I/O Pin Assignment in SoC Devices}\\
Pins on an SoC device are dedicated to either FPGA fabric or Hard Processor System (HPS):

\textbf{Physical Organization:}
\begin{itemize}
    \item \textbf{FPGA I/O:} Located around FPGA fabric perimeter
    \item \textbf{HPS Column I/O:} Dedicated vertical columns for HPS
    \item \textbf{HPS Row I/O:} Dedicated horizontal rows for HPS
\end{itemize}

\textbf{Boot Requirements:}
\begin{itemize}
    \item CPU requires certain pins during boot process $\rightarrow$ fixed assignment
    \item Some peripheral functions need specific circuitry (e.g., USB only on specific pins)
    \item HPS peripherals are directly attached to system bus
\end{itemize}
\end{concept}

% TODO: Add image from SCD_5a_GPIO.pdf, Page 10
% Description: FPGA vs HPS I/O capabilities and control mechanisms
% Priority: CRITICAL
% Suggested filename: lecture05A_io_control.png
% \includegraphics[width=\linewidth]{lecture05A_io_control.png}
% WHEN YOU ADD IMAGE: Uncomment line above

\mult{2}

\begin{definition}{FPGA I/O Characteristics}
\begin{itemize}
    \item Attached to FPGA fabric
    \item Controlled after fabric configuration
    \item Used as GPIO or peripheral pins from FPGA
    \item Can be controlled by HPS via bridges
\end{itemize}
\end{definition}

\begin{definition}{HPS I/O Characteristics}
\begin{itemize}
    \item Available before fabric configuration
    \item Used for HPS peripherals (Ethernet, USB, SD, SPI, CAN)
    \item Used for GPIO from HPS
    \item Can be used from FPGA as "Loan I/O"
\end{itemize}
\end{definition}

\multend

% TODO: Add image from SCD_5a_GPIO.pdf, Page 11
% Description: HPS I/O pin configuration tool showing peripheral selection
% Priority: IMPORTANT
% Suggested filename: lecture05A_hps_pin_config.png
% \includegraphics[width=\linewidth]{lecture05A_hps_pin_config.png}
% WHEN YOU ADD IMAGE: Uncomment line above

\begin{concept}{HPS Pin Function Selection}\\
The Platform Designer configuration tool allows selection of HPS pin functions:
\begin{itemize}
    \item \textbf{FPGA Interfaces:} Bridges and connections to fabric
    \item \textbf{Peripheral Pins:} Ethernet, USB, SD, UART, I2C, etc.
    \item \textbf{HPS Clocks:} System and peripheral clock inputs
    \item \textbf{SDRAM:} DDR memory interface pins
\end{itemize}

\important{Note:} Dedicated pin locations limit configuration options. Some functions can only be assigned to specific pins due to internal circuitry requirements.
\end{concept}

\raggedcolumns
\columnbreak

\subsection{GPIO in Platform Design}

% TODO: Add images from SCD_5a_GPIO.pdf, Pages 12-30
% Description: Multiple slides covering GPIO configuration in Platform Designer
% Priority: CRITICAL for pages 12-20, IMPORTANT for pages 21-30
% Suggested filenames: lecture05A_platform_XX.png (where XX is page number)
% Images showing:
% - Platform Designer interface
% - GPIO IP core configuration
% - Address mapping
% - Interrupt configuration
% - Software driver interface
% - GPIO register access
% - Example configurations

\begin{KR}{Configuring GPIO in Platform Designer}\\
\textbf{Step 1: Add GPIO IP Core}
\begin{enumerate}
    \item Open Platform Designer in Quartus
    \item Add "PIO (Parallel I/O)" IP core from library
    \item Configure data width (1-32 bits)
    \item Set direction (input, output, or bidirectional)
\end{enumerate}

\textbf{Step 2: Configure Parameters}
\begin{itemize}
    \item Set data width to match application requirements
    \item Choose input/output/bidirectional mode
    \item Enable/disable interrupt generation
    \item Set edge capture capabilities if needed
    \item Configure reset value for output pins
\end{itemize}

\textbf{Step 3: Connect to System}
\begin{enumerate}
    \item Connect to system clock and reset
    \item Connect Avalon-MM slave interface to system interconnect
    \item Export external connections to top-level
    \item Assign base address for register access
\end{enumerate}

\textbf{Step 4: Generate and Integrate}
\begin{enumerate}
    \item Generate HDL from Platform Designer
    \item Instantiate system in top-level VHDL/Verilog
    \item Assign external pins in Pin Planner
    \item Compile design
\end{enumerate}
\end{KR}

\begin{definition}{GPIO Register Interface}\\
GPIO cores typically provide these registers:
\begin{itemize}
    \item \textbf{Data Register:} Read input values or write output values
    \item \textbf{Direction Register:} Set pin direction (bidirectional mode only)
    \item \textbf{Interrupt Mask:} Enable/disable interrupts per pin
    \item \textbf{Edge Capture:} Record edge events on input pins
\end{itemize}
\end{definition}

\begin{example2}{GPIO Software Access}\\
\textbf{Task:} Control an LED and read a button using GPIO

\textbf{C Code Example:}
\begin{lstlisting}[language=C, style=base]
// Define GPIO base addresses
#define LED_BASE 0xFF200000
#define BUTTON_BASE 0xFF200010

// LED control
volatile unsigned int *led_ptr = (unsigned int *)LED_BASE;
*led_ptr = 0x01;  // Turn on LED 0

// Button reading
volatile unsigned int *button_ptr = (unsigned int *)BUTTON_BASE;
unsigned int button_state = *button_ptr & 0x01;  // Read button 0
\end{lstlisting}

\tcblower

\textbf{Explanation:}
\begin{itemize}
    \item Use volatile pointers to prevent compiler optimization
    \item Access GPIO registers through memory-mapped addresses
    \item Bitwise operations allow individual pin control
    \item Base addresses defined in Platform Designer
\end{itemize}
\end{example2}

\begin{remark}
GPIO provides a simple but powerful interface for connecting external devices to an SoC. Proper configuration in Platform Designer and Pin Planner ensures reliable operation and optimal signal integrity.
\end{remark}

% ===== IMAGE SUMMARY =====
% Total images needed: 30
% CRITICAL priority: 12
% IMPORTANT priority: 15
% SUPPLEMENTARY priority: 3
%
% Quick extraction checklist:
% [ ] [SCD_5a_GPIO.pdf, Page 1] - Title slide (SUPPLEMENTARY)
% [ ] [SCD_5a_GPIO.pdf, Page 2] - Learning objectives (SUPPLEMENTARY)
% [ ] [SCD_5a_GPIO.pdf, Page 3] - Agenda (SUPPLEMENTARY)
% [ ] [SCD_5a_GPIO.pdf, Page 4] - Section divider (SUPPLEMENTARY)
% [ ] [SCD_5a_GPIO.pdf, Page 5] - GPIO applications (CRITICAL)
% [ ] [SCD_5a_GPIO.pdf, Page 6] - GPIO modes (CRITICAL)
% [ ] [SCD_5a_GPIO.pdf, Page 7] - GPIO internal schematic (CRITICAL)
% [ ] [SCD_5a_GPIO.pdf, Page 8] - GPIO hardware features (IMPORTANT)
% [ ] [SCD_5a_GPIO.pdf, Page 9] - FPGA vs HPS I/O physical layout (CRITICAL)
% [ ] [SCD_5a_GPIO.pdf, Page 10] - FPGA vs HPS I/O control (CRITICAL)
% [ ] [SCD_5a_GPIO.pdf, Page 11] - HPS pin configuration tool (IMPORTANT)
% [ ] [SCD_5a_GPIO.pdf, Pages 12-30] - Platform Designer and GPIO configuration (CRITICAL for 12-20, IMPORTANT for 21-30)
% =====================