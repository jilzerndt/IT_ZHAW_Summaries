\section{Multi-Gigabit Transceivers (MGT)}

\subsection{Introductory Exercises}

\begin{example2}{Exercise 1: Minimum Bitrate Calculation}\\
\textbf{Given:}
\begin{itemize}
    \item System records 24-bit samples
    \item Sampling rate: $150$ MSa/s (mega samples per second)
    \item Transfer via serial connection
\end{itemize}

\textbf{Question:} What is the minimum bitrate of the serial connection to keep up with the sampled data?

\tcblower

\textbf{Solution:}

Raw data rate calculation:
$$\text{Bitrate} = \text{Sample Rate} \times \text{Bits per Sample}$$
$$\text{Bitrate} = 150 \times 10^6 \text{ Sa/s} \times 24 \text{ bit/Sa}$$
$$\text{Bitrate} = 3.6 \times 10^9 \text{ bit/s} = 3.6 \text{ Gbit/s}$$

\important{Answer:} Minimum bitrate = $3.6$ Gbit/s (without line coding overhead)
\end{example2}

\begin{example2}{Exercise 2: Physical Bit Length}\\
\textbf{Given:}
\begin{itemize}
    \item Copper wire propagation speed: $c = 200{,}000$ km/s
    \item Serial transmission bitrate: $1$ Gbit/s ($1.00 \times 10^9$ bit/s)
\end{itemize}

\textbf{Question:} What is the length of a bit in centimeters in this medium?

\tcblower

\textbf{Solution:}

The physical length of one bit is the distance the signal travels during one bit period:
$$\text{Bit Length} = \frac{\text{Propagation Speed}}{\text{Bitrate}}$$
$$\text{Bit Length} = \frac{200{,}000 \times 10^3 \text{ m/s}}{1.00 \times 10^9 \text{ bit/s}}$$
$$\text{Bit Length} = 0.2 \text{ m} = 20 \text{ cm}$$

\important{Answer:} Each bit occupies $20$ cm in the transmission medium.
\end{example2}

\begin{example2}{Exercise 3: Clock Recovery Overhead}\\
\textbf{Given:}
\begin{itemize}
    \item Serial transmission at $1$ Gbit/s
    \item Samples from AD converter transmitted end-to-end
\end{itemize}

\textbf{Question:} By how many percent must you increase the bitrate to allow clock recovery from the serial transmission?

\tcblower

\textbf{Solution:}

This depends on the transmitted data characteristics:
\begin{itemize}
    \item If there are enough bit transitions between '0' and '1', the bitrate does not need to be increased
    \item Additional sync bits can be introduced to ensure a minimum number of transitions
    \item Adding overhead requires a faster bitrate
\end{itemize}

\textbf{Line coding examples:}
\begin{itemize}
    \item 8b/10b encoding: $+25\%$ overhead
    \item 64b/66b encoding: $+3.125\%$ overhead
\end{itemize}

\important{Note:} The answer depends on the chosen line coding scheme.
\end{example2}

\begin{example2}{Exercise 4: Frame Synchronization}\\
\textbf{Given:}
\begin{itemize}
    \item System records 24-bit samples at $150$ MHz
    \item Short interruption of service in receiver
    \item System resumes receiving after unknown time
\end{itemize}

\textbf{Question:} Design a way to find the start of a 24-bit sample in an ongoing serial transmission.

\tcblower

\textbf{Solution:}

Use a line code with synchronization patterns:
\begin{itemize}
    \item \textbf{Comma symbols} (8b/10b): Special control words with unique bit patterns
    \item \textbf{Sync field} (64b/66b): Header bits that mark frame boundaries
\end{itemize}

\textbf{Procedure:}
\begin{enumerate}
    \item Transmitter periodically sends sync patterns
    \item Receiver scans bitstream for known patterns
    \item When pattern is found, frame alignment is established
    \item Subsequent frames can be processed correctly
\end{enumerate}

\important{Answer:} Use a line code with Commas or a sync field for frame alignment.
\end{example2}

\raggedcolumns
\columnbreak

\subsection{Motivation and Goals}

\begin{concept}{Motivation for MGT Study}\\
\textbf{Key Questions:}
\begin{itemize}
    \item How does a multigigabit transceiver work?
    \item How can we achieve such fast transmission rates in an FPGA with limited clock speeds?
    \item Sixteen GTH Transceivers can yield $260.8$ Gbit of data per second. How to handle this amount of data?
\end{itemize}
\end{concept}

\begin{definition}{Learning Goals}\\
After this lesson, students should be able to:
\begin{enumerate}
    \item Explain the challenges of fast serial transmission
    \item Sketch the basic structure of an MGT and name its parts and their functions
    \item Understand the necessity for a line code
    \item Explain the properties of different line codes
    \item Select a line code according to their system's needs
    \item Calculate the overhead of their line code and the required bitrate
\end{enumerate}
\end{definition}

\subsection{Challenges of Serial Transmissions}

\begin{definition}{Serial Transmission Challenges}\\
High-speed serial transmission faces several technical challenges:

\paragraph{Sampling and Clock Recovery}
\begin{itemize}
    \item Clean sampling of the signal at the correct time
    \item Clock recovery from data stream without separate clock line
\end{itemize}

\paragraph{Word Alignment}
\begin{itemize}
    \item Synchronization to detect the start of words/frames
    \item Maintaining alignment throughout transmission
\end{itemize}

\paragraph{Control vs. Data Transmission}
\begin{itemize}
    \item Control signals: START, STOP, IDLE, ACK, NACK, SYNC
    \item Data payload
    \item Distinguishing between control and data
\end{itemize}

\paragraph{DC Balance}
\begin{itemize}
    \item Transmission medium must not accumulate DC charge
    \item Equal number of ones and zeros required
\end{itemize}

\paragraph{Error Detection and Protection}
\begin{itemize}
    \item Detecting bit errors in transmission
    \item Optional error correction capabilities
\end{itemize}
\end{definition}

\subsection{Receiver Architecture}

\subsubsection{RX Equalizer}

\begin{concept}{Adaptive Equalization}\\
\textbf{Bit Error Ratio (BER)} is a function of:
\begin{itemize}
    \item The transmitter quality
    \item The channel medium (signal attenuation, limited bandwidth)
    \item The receiver performance
\end{itemize}

The transmission media (channel) is bandwidth-limited, and the signal traveling through it is subjected to attenuation and distortion.
\end{concept}

\begin{definition}{Adaptive Filtering Modes}\\
Two types of adaptive filtering are available for GTH receivers:

\textbf{Low-Power Mode (LPM):}
\begin{itemize}
    \item Optimized for power efficiency with lower channel loss
    \item Applies a fixed gain before parallelization (SIPO)
    \item Suitable for shorter, higher-quality connections
\end{itemize}

\textbf{Decision Feedback Equalizer (DFE):}
\begin{itemize}
    \item For equalizing lossier channels
    \item Provides closer adjustment of filter parameters
    \item Discrete-time adaptive high-pass filter
    \item TAP values are coefficients set by adaptive algorithm
    \item Cannot remove pre-cursor, only compensates for post-cursors
\end{itemize}

\important{Note:} A linear equalizer allows both pre-cursor and post-cursor gain, but DFE provides better compensation for high-loss channels.
\end{definition}

% TODO: Add image from SCD_11_mgt.pdf, Slide 9
% Description: RX Equalizer block diagram showing LPM mode with SIPO, Fixed Gain, AGC
% Priority: CRITICAL
% Suggested filename: lecture11_rx_equalizer_lpm.png
\\
% \includegraphics[width=\linewidth]{lecture11_rx_equalizer_lpm.png}
\\
% WHEN YOU ADD IMAGE: Uncomment line above (remove %)

\subsubsection{Clock Data Recovery (CDR)}

\begin{definition}{Clock Recovery Process}\\
\textbf{Goal:} Find the perfect sampling point for serial data bits.

\textbf{CDR Components and Steps:}
\begin{enumerate}
    \item \textbf{Edge Sampler:} Detects edges in the incoming data stream
    \item \textbf{CDR FSM:} Finite State Machine detects phase relation to PLL-based clock
    \item \textbf{Phase Interpolator (PI):} Adjusts the PLL-based clock to align with data
    \item \textbf{Data Sampling:} Samples data at optimal point
\end{enumerate}

The RX clock data recovery (CDR) circuit in each GTH transceiver extracts the recovered clock and data from an incoming data stream.
\end{definition}

% TODO: Add image from SCD_11_mgt.pdf, Slide 10
% Description: Detailed CDR architecture showing Linear EQ, DFE, Edge Sampler, CDR FSM, Phase Interpolators, PLL, DEMUX
% Priority: CRITICAL
% Suggested filename: lecture11_cdr_architecture.png
\\
% \includegraphics[width=\linewidth]{lecture11_cdr_architecture.png}
\\
% WHEN YOU ADD IMAGE: Uncomment line above (remove %)

\begin{remark}
\textbf{DFE:} Decision Feedback Equalizer - compensates for signal distortion by adjusting for previously transmitted bits.
\end{remark}

\subsubsection{Word Alignment}

\begin{definition}{Word Alignment Process}\\
\textbf{Goal:} Find the start of transmission frames in the continuous bitstream.

\textbf{Procedure:}
\begin{enumerate}
    \item Scan bitstream for known patterns (comma symbols or sync fields)
    \item If the pattern is found, subsequent frames can be processed
    \item If pattern is not found, shift window by one bit and retry
    \item Repeat until correct alignment is achieved
\end{enumerate}
\end{definition}

\begin{concept}{Comma Alignment}\\
The alignment process searches for special bit patterns called "commas":
\begin{itemize}
    \item Commas are unique bit sequences that cannot appear in normal data
    \item In 8b/10b encoding: sequences with five consecutive '1's or '0's
    \item Once found, all subsequent data is aligned to correct byte boundaries
\end{itemize}
\end{concept}

% TODO: Add image from SCD_11_mgt.pdf, Slide 11
% Description: Conceptual view of comma alignment showing unaligned bits, comma detection, and aligned output
% Priority: CRITICAL
% Suggested filename: lecture11_word_alignment.png
\\
% \includegraphics[width=\linewidth]{lecture11_word_alignment.png}
\\
% WHEN YOU ADD IMAGE: Uncomment line above (remove %)

\raggedcolumns
\columnbreak

\subsubsection{Complete Receiver Block Diagram}

\begin{concept}{GTH Transceiver RX Overview}\\
The complete receiver contains many functional blocks:
\begin{itemize}
    \item Physical Medium Attachment (PMA): analog front-end
    \item Physical Coding Sublayer (PCS): digital processing
    \item Equalization (DFE/Linear EQ)
    \item Clock recovery (CDR)
    \item Serial-to-Parallel conversion (SIPO)
    \item Comma detection and alignment
    \item Elastic buffer for clock domain crossing
    \item 8B/10B or 128B/130B decoder
    \item Gearbox for width conversion
    \item PRBS checker for testing
    \item Various loopback modes
\end{itemize}

\important{Note:} Many additional blocks exist for decoding, error detection, synchronization, and protocol interpretation. Complete understanding is not required for exam.
\end{concept}

% TODO: Add image from SCD_11_mgt.pdf, Slide 12
% Description: Complete GTH transceiver RX block diagram showing all components from analog input to digital output
% Priority: IMPORTANT
% Suggested filename: lecture11_rx_block_diagram.png
\\
% \includegraphics[width=\linewidth]{lecture11_rx_block_diagram.png}
\\
% WHEN YOU ADD IMAGE: Uncomment line above (remove %)

\subsection{Line Codes}

\subsubsection{Line Code Requirements}

\begin{definition}{Line Code Goals}\\
A line code encodes data for transmission over a serial link. The main goals are:

\begin{enumerate}
    \item \textbf{Clock Recovery:} Guarantee a minimal number of transitions for the receiver to extract clock
    \item \textbf{DC Balance:} Transmit an equal number of ones and zeros (transmission medium must not be charged)
    \item \textbf{Error Detection:} Detect bit errors in transmission
    \item \textbf{Control Words:} Provide a means to distinguish between control and data words
\end{enumerate}
\end{definition}

\subsection{8b/10b Encoding}

\subsubsection{Basic Principle}

\begin{definition}{8b/10b Encoding Concept}\\
\textbf{Idea:} A DC-balanced code that allows clock recovery and provides control words.

\textbf{Solution:} Add 2 bits to every byte ($8$ bits $\rightarrow$ $10$ bits)

\textbf{Combinatorics:}
\begin{itemize}
    \item $8$ bits: $2^8 = 256$ combinations (data words needed)
    \item $10$ bits: $2^{10} = 1024$ combinations (available)
\end{itemize}

\textbf{DC Balance Constraint:} Minimum 4, maximum 6 'ones' per codeword
$$\binom{10}{4} + \binom{10}{5} + \binom{10}{6} = 210 + 252 + 210 = 672 \text{ combinations}$$

\textbf{Examples:}
\begin{itemize}
    \item Allowed: $0011110011$ or $1101101000$
    \item Not allowed: $1111100111$ (more than 5 consecutive ones)
\end{itemize}

\textbf{Clock Recovery:} No more than five consecutive '1's or '0's $\rightarrow$ exclude some words

\important{Result:} Still more than 256 combinations available after constraints.
\end{definition}

\subsubsection{Running Disparity}

\begin{concept}{Running Disparity (RD)}\\
\textbf{Idea:} Keep a balanced number of '1's on the transmission line over time.

\textbf{Mechanism:}
\begin{itemize}
    \item Keep track of the running disparity (RD)
    \item RD increases by $+2$ when sending a codeword with 6 '1's
    \item RD remains unchanged ($0$) when sending a codeword with 5 '1's
    \item RD decreases by $-2$ when sending a codeword with 4 '1's
\end{itemize}

\textbf{Selection Rules:}
\begin{itemize}
    \item If $RD = +1$: choose codeword with disparity $0$ or $-2$ (5 or 4 '1's)
    \item If $RD = -1$: choose codeword with disparity $0$ or $+2$ (5 or 6 '1's)
\end{itemize}

\important{Guarantee:} Running disparity will never exceed the range $[-1, +1]$.
\end{concept}

\begin{example2}{Running Disparity Example}\\
\textbf{Initial state:} $RD = -1$

\textbf{Transmission sequence:}
\begin{enumerate}
    \item Send codeword with 6 '1's: $RD = -1 + 2 = +1$
    \item Send codeword with 4 '1's: $RD = +1 - 2 = -1$
    \item Send codeword with 5 '1's: $RD = -1 + 0 = -1$
    \item Send codeword with 6 '1's: $RD = -1 + 2 = +1$
\end{enumerate}

The encoder automatically selects appropriate codewords to maintain balance.
\end{example2}

\subsubsection{Error Protection}

\begin{definition}{8b/10b Error Detection}\\
\textbf{Idea:} Use the line code to detect bit errors.

\textbf{Detection Mechanisms:}
\begin{enumerate}
    \item \textbf{Invalid Codeword:} Received codeword not in valid set $\rightarrow$ indicates bit error
    \item \textbf{Invalid Disparity:} Running disparity exceeds $[-1, +1]$ $\rightarrow$ indicates two-bit errors
\end{enumerate}

\textbf{Error Flags:}
\begin{itemize}
    \item "Out-of-table": Invalid codeword detected
    \item "Invalid disparity": Running disparity violation detected
\end{itemize}

\important{Limitation:} Error detection only; no error correction capability.
\end{definition}

\subsubsection{Control Words}

\begin{definition}{8b/10b Control Words}\\
More than 256 possible combinations allow distinction between data and control words.

\textbf{Structure:}
\begin{itemize}
    \item An 8-bit byte is split into 3-bit and 5-bit parts
    \item Each part is assigned either:
    \begin{itemize}
        \item A single, balanced code (6/4 bits)
        \item Or a pair of unbalanced codes (6/4 bits)
    \end{itemize}
    \item Extra combinations are used as control words
\end{itemize}

\textbf{Naming Convention:}
\begin{itemize}
    \item Data words: D.x.y
    \item Control words: K.x.y
\end{itemize}
\end{definition}

\begin{example2}{8b/10b Encoding Examples}\\
\textbf{Data Word D.00.3:}
\begin{itemize}
    \item 8-bit representation: \texttt{01100000}
    \item 10-bit representation: \texttt{1001110011} or \texttt{0110001100} (depending on RD)
\end{itemize}

\textbf{Control Word K.28.7:}
\begin{itemize}
    \item 8-bit representation: \texttt{11111100}
    \item 10-bit representation: \texttt{0011111000} or \texttt{1100000111} (depending on RD)
\end{itemize}
\end{example2}

\raggedcolumns
\columnbreak

\subsubsection{Comma Symbols}

\begin{definition}{Comma Symbols for Word Alignment}\\
8b/10b encoding defines \textbf{12 control words}. The function of most control words is not defined by the standard, except for special synchronization symbols.

\textbf{Comma Symbols:} K.28.1, K.28.5, K.28.7
\begin{itemize}
    \item Used for word border detection
    \item Contain sequences of five consecutive '1's or '0's
    \item This pattern cannot occur in normal data or other control words
\end{itemize}

\textbf{Usage:}
\begin{enumerate}
    \item Send commas at the beginning to allow synchronization
    \item Send commas intermittently to ensure/verify continued synchronization
    \item Receiver scans for comma pattern to find word boundaries
\end{enumerate}
\end{definition}

\begin{example2}{Word Alignment Exercise}\\
\textbf{Task:} In the following bit stream, draw the word borders:

\texttt{10010011010111001001111100101010010011101101000100001011\\ 10101001001}

\tcblower

\textbf{Solution approach:}
\begin{enumerate}
    \item Look for sequences of five consecutive '1's or '0's
    \item These mark comma positions
    \item Word boundaries are aligned to commas
    \item Each word is 10 bits
\end{enumerate}

Comma pattern appears: \texttt{...11111...} at certain positions.
\end{example2}

\subsubsection{8b/10b Overhead}

\begin{concept}{Bandwidth Overhead}\\
8b/10b encoding introduces significant overhead:

\textbf{Calculation:}
\begin{itemize}
    \item 2 extra bits added for every 8 data bits
    \item Overhead percentage: $\frac{2}{8} = 25\%$
    \item Effective data rate: $\frac{8}{10} = 80\%$ of line rate
\end{itemize}

\textbf{Example:}
A $6.4$ Gbit/s data transfer requires an $8$ Gbit/s line rate:
$$\text{Line Rate} = \frac{\text{Data Rate}}{0.8} = \frac{6.4 \text{ Gbit/s}}{0.8} = 8 \text{ Gbit/s}$$

\important{Trade-off:} 8b/10b encoding is simple to encode and decode, but comes at significant cost in data rate (25\% overhead).
\end{concept}

\subsection{64b/66b Encoding}

\subsubsection{Basic Principle}

\begin{definition}{64b/66b Encoding Concept}\\
64b/66b encoding adds 2 bits for every 64 data bits.

\textbf{Goals:}
\begin{itemize}
    \item Clock recovery: guaranteed transitions
    \item Word border detection via sync field
    \item Control data transmission
\end{itemize}

\textbf{Not Provided:}
\begin{itemize}
    \item No explicit error detection
    \item No guaranteed DC balance (relies on scrambling)
\end{itemize}

\textbf{Sync Field:}
\begin{itemize}
    \item 2-bit header: can be either '01' or '10'
    \item '01': payload contains pure data (no control)
    \item '10': payload contains control data mixed with data
\end{itemize}

\textbf{Scrambling:}
\begin{itemize}
    \item Payload is scrambled to ensure more transitions
    \item Evens out disparity over time
    \item Does not guarantee DC balance for every codeword
\end{itemize}
\end{definition}

\subsubsection{Control Data Transmission}

\begin{definition}{64b/66b Control Data Format}\\
When sync field = '10', payload contains control data:

\textbf{Structure:}
\begin{itemize}
    \item \textbf{Byte 0:} Type field defines whether each payload byte is control or data
    \item \textbf{Bytes 1-7:} Mix of control and data bytes as indicated by type field
\end{itemize}

\textbf{Control Byte Content:}
\begin{itemize}
    \item Sync markers
    \item Commands: start, stop, idle, interrupt
    \item Application-specific control information
    \item User-defined control codes
\end{itemize}

\important{Flexibility:} The protocol can define its own control byte meanings.
\end{definition}

\subsubsection{Word Border Detection}

\begin{definition}{64b/66b Synchronization}\\
Sync field ('01' or '10') guarantees a transition every 66 bits.

\textbf{Gearbox Alignment Procedure:}
\begin{enumerate}
    \item Choose a pair of bits as potential sync field
    \item Loop:
    \begin{itemize}
        \item If '01' or '10': increment "good" counter
        \item If '00' or '11': increment "bad" counter
    \end{itemize}
    \item If "bad" counter overruns: shift by one bit, retry from step 1
    \item If "good" counter overruns: sync field found, alignment established
    \item Gearbox outputs parallel words in configurable width
\end{enumerate}

\important{Robustness:} Statistical approach ensures synchronization even with occasional errors.
\end{definition}

\begin{concept}{64b/66b Overhead}\\
\textbf{Calculation:}
\begin{itemize}
    \item 2 extra bits added for every 64 data bits
    \item Overhead percentage: $\frac{2}{64} = 3.125\%$
    \item Effective data rate: $\frac{64}{66} \approx 96.97\%$ of line rate
\end{itemize}

\textbf{Example:}
A $3.6$ Gbit/s data transfer requires:
$$\text{Line Rate} = 3.6 \times \frac{66}{64} = 3.7125 \text{ Gbit/s}$$

\important{Advantage:} Much lower overhead than 8b/10b (3.125\% vs. 25\%).
\end{concept}

\raggedcolumns
\columnbreak

\subsection{General Transceiver Structure}

\begin{concept}{Transceiver Functions Overview}\\
Modern multi-gigabit transceivers address all serial transmission challenges:
\begin{itemize}
    \item Clock recovery: CDR with phase interpolation
    \item DC balance: Line coding (8b/10b or 64b/66b)
    \item Error detection: Line code validation
    \item Control words: Embedded control channels
    \item Word alignment: Comma detection or sync fields
    \item Equalization: Adaptive filtering for signal quality
\end{itemize}
\end{concept}

% TODO: Add image from SCD_11_mgt.pdf, Slide 24
% Description: General transceiver structure overview diagram
% Priority: IMPORTANT
% Suggested filename: lecture11_transceiver_overview.png
\\
% \includegraphics[width=\linewidth]{lecture11_transceiver_overview.png}
\\
% WHEN YOU ADD IMAGE: Uncomment line above (remove %)

\subsubsection{GTH Channel Primitive}

\begin{definition}{GTHE3/4\_CHANNEL Structure}\\
Each GTH channel primitive consists of three main components:

\paragraph{Transmitter (TX):}
\begin{itemize}
    \item TX Interface: parallel data input
    \item 8B/10B or 128B/130B Encoder
    \item TX Sync/Async Gearbox: width conversion
    \item Pattern Generator: for testing
    \item Phase Adjust FIFO: clock domain crossing
    \item TX Phase Interpolator: timing adjustment
    \item PISO: Parallel-In Serial-Out converter
    \item TX Driver: high-speed analog output
    \item Pre/Post Emphasis: signal conditioning
\end{itemize}

\paragraph{Receiver (RX):}
\begin{itemize}
    \item Equalizer (DFE or Linear EQ)
    \item Clock Data Recovery (CDR)
    \item SIPO: Serial-In Parallel-Out converter
    \item Comma Detect and Align
    \item 8B/10B or 128B/130B Decoder
    \item RX Sync/Async Gearbox
    \item RX Elastic Buffer: clock domain crossing
    \item PRBS Checker: for testing
    \item RX Interface: parallel data output
\end{itemize}

\paragraph{Channel PLL:}
\begin{itemize}
    \item Provides high-speed clocks for serialization
    \item Independent from core FPGA clocking
\end{itemize}
\end{definition}

% TODO: Add image from SCD_11_mgt.pdf, Slide 25
% Description: Detailed GTHE3/4_CHANNEL primitive topology showing all TX and RX blocks
% Priority: CRITICAL
% Suggested filename: lecture11_gth_channel_topology.png
\\
% \includegraphics[width=\linewidth]{lecture11_gth_channel_topology.png}
\\
% WHEN YOU ADD IMAGE: Uncomment line above (remove %)

\subsubsection{Serialization and Parallelization}

\begin{example2}{Data Width Conversion Example}\\
\textbf{Scenario:} Converting between parallel FPGA logic and high-speed serial link.

\textbf{Transmitter Path:}
\begin{itemize}
    \item FPGA logic: 32-bit data at 100 MHz
    \item After 8b/10b encoding: 40-bit data at 100 MHz
    \item Serializer (PISO): Serial data at 4000 MHz (4 GHz)
\end{itemize}

\textbf{Receiver Path:}
\begin{itemize}
    \item Serial data input: 4000 MHz (4 GHz)
    \item After SIPO: 40-bit data at 100 MHz
    \item After 8b/10b decoding: 32-bit data at 100 MHz
    \item To FPGA logic: 32-bit data at 100 MHz
\end{itemize}

\important{Key Insight:} Serialization allows $40\times$ increase in clock frequency compared to FPGA fabric.
\end{example2}

% TODO: Add image from SCD_11_mgt.pdf, Slide 26
% Description: Data path example showing parallel widths and clock frequencies through transceiver
% Priority: CRITICAL
% Suggested filename: lecture11_data_width_example.png
\\
% \includegraphics[width=\linewidth]{lecture11_data_width_example.png}
\\
% WHEN YOU ADD IMAGE: Uncomment line above (remove %)

\subsection{Transceiver Clocking}

\begin{concept}{High-Speed Clock Generation}\\
Serializers need high-speed clocks in the GHz range, far exceeding typical FPGA fabric capabilities.

\textbf{Solution:} Dedicated PLLs in transceiver blocks provide GHz clocking:
\begin{itemize}
    \item \textbf{QPLL (Quad PLL):} Shared among four channels in a quad
    \item \textbf{CPLL (Channel PLL):} Dedicated to individual channel
\end{itemize}

\important{Hard Macro:} Transceivers are hard silicon macros, not FPGA fabric, enabling GHz timing to be met.
\end{concept}

\begin{definition}{GTH Quad Configuration}\\
Four GTHE3/4\_CHANNEL primitives plus one GTHE3/4\_COMMON primitive form a Quad:

\textbf{GTHE3/4\_COMMON contains:}
\begin{itemize}
    \item QPLL0: First quad PLL
    \item QPLL1: Second quad PLL
    \item Reference clock distribution
\end{itemize}

\textbf{Each GTHE3/4\_CHANNEL contains:}
\begin{itemize}
    \item CPLL: Channel-specific PLL
    \item Transmitter
    \item Receiver
\end{itemize}

\textbf{Clock Distribution:}
\begin{itemize}
    \item Reference clocks from IBUFDS\_GTE3/4
    \item Recovered clock routed directly from RX PMA
    \item Clock sharing between channels possible
\end{itemize}
\end{definition}

% TODO: Add image from SCD_11_mgt.pdf, Slide 27
% Description: GTH transceiver quad configuration showing QPLL, CPLL, and four channels
% Priority: CRITICAL
% Suggested filename: lecture11_gth_quad_config.png
\\
% \includegraphics[width=\linewidth]{lecture11_gth_quad_config.png}
\\
% WHEN YOU ADD IMAGE: Uncomment line above (remove %)

\begin{remark}
\textbf{Note:} CPLL and QPLL have different clock rate ranges. Finding a working clocking scheme for specific data rates can be challenging. Consult device datasheets for supported frequency ranges.
\end{remark}

% TODO: Add image from SCD_11_mgt.pdf, Slide 28
% Description: Detailed CPLL and QPLL frequency range specifications table
% Priority: IMPORTANT
% Suggested filename: lecture11_pll_frequency_ranges.png
\\
% \includegraphics[width=\linewidth]{lecture11_pll_frequency_ranges.png}
\\
% WHEN YOU ADD IMAGE: Uncomment line above (remove %)

\raggedcolumns
\columnbreak

\subsection{Application Exercises with Solutions}

\begin{example2}{Exercise: Line Code Overhead Calculation}\\
\textbf{Given:}
\begin{itemize}
    \item System records 24-bit samples
    \item Sampling rate: 150 MHz
    \item Serial connection for data transfer
\end{itemize}

\textbf{Questions:}
\begin{enumerate}
    \item What is minimum bitrate (raw data, no line code)?
    \item Calculate minimum bitrate with 8b/10b encoding
    \item Calculate minimum bitrate with 64b/66b encoding
\end{enumerate}

\tcblower

\textbf{Solutions:}

\paragraph{1. Raw Data Rate:}
$$\text{Bitrate} = 150 \times 10^6 \text{ Sa/s} \times 24 \text{ bit/Sa} = 3{,}600{,}000{,}000 \text{ bit/s}$$
$$\text{Bitrate} = 3.6 \text{ Gbit/s}$$

\paragraph{2. With 8b/10b Encoding:}
8b/10b adds 25\% overhead:
$$\text{Bitrate}_{8b/10b} = 3.6 \text{ Gbit/s} \times 1.25 = 4.5 \text{ Gbit/s}$$

\paragraph{3. With 64b/66b Encoding:}
64b/66b adds $\frac{66}{64}$ factor:
$$\text{Bitrate}_{64b/66b} = 3.6 \text{ Gbit/s} \times \frac{66}{64} = 3.7125 \text{ Gbit/s}$$

\important{Summary:}
\begin{center}
\begin{tabular}{|l|c|}
\hline
\textbf{Encoding} & \textbf{Required Line Rate} \\
\hline
No encoding & 3.6 Gbit/s \\
8b/10b & 4.5 Gbit/s \\
64b/66b & 3.7125 Gbit/s \\
\hline
\end{tabular}
\end{center}
\end{example2}

\subsection{Future Hardware Trends}

\subsubsection{Current and Future Transceiver Speeds}

\begin{concept}{Bandwidth Evolution}\\
Bandwidth requirements continue to grow exponentially:

\textbf{Current State:}
\begin{itemize}
    \item Data centers use optical connections with transceivers
    \item Link speeds: 100 to 800 Gbps per link
    \item FPGA transceivers: up to 58 or 112 Gbps (manufacturer dependent)
\end{itemize}

\textbf{Future Direction:}
\begin{itemize}
    \item Manufacturers aim to place optical transceivers directly on die
    \item Technology: Co-Packaged Optics (CPO)
    \item Brings optical fiber directly into chip package
\end{itemize}

\important{Trend:} Bandwidth is never enough; continuous push for higher speeds.
\end{concept}

\subsubsection{Co-Packaged Optics (CPO)}

\begin{definition}{Co-Packaged Optics Technology}\\
\textbf{Concept:} Optical transceivers (light modulators) are packaged onto the die, bringing the optical fiber directly into the chip package.

\textbf{Architecture:}
\begin{itemize}
    \item \textbf{Lasers:} Stay outside chip package (for now)
    \begin{itemize}
        \item Difficult to bond to silicon (but not impossible)
        \item Use non-silicon materials requiring specialized tools
    \end{itemize}
    \item \textbf{Modulators:} Integrated on silicon die
    \begin{itemize}
        \item Multiple modulators can use a single laser
        \item Wavelength division multiplexing possible
    \end{itemize}
\end{itemize}

\textbf{Performance:}
\begin{itemize}
    \item CPOs reach 100 Gb/s per lane
    \item Small size enables up to 1 Tbps per millimeter of die edge
    \item Market leaders claim speeds of 32 Tbps on single chiplet using multiple lanes
\end{itemize}
\end{definition}

\begin{definition}{Photonic Modulator Technologies}\\
Two main modulator types are used in CPO:

\paragraph{Microring Modulator:}
\begin{itemize}
    \item Uses ring-shaped waveguide
    \item Filters out specific wavelengths that form standing wave in ring
    \item Electronic manipulation of refractive index changes filtered wavelength
    \item Photonic bits formed by filtering or transmitting laser
    \item More compact design
\end{itemize}

\paragraph{Mach-Zehnder Modulator:}
\begin{itemize}
    \item Splits laser into two lanes
    \item One lane gets phase-shifted by electric signal
    \item At junction: destructive or constructive interference occurs
    \item Creates photonic bits through interference patterns
    \item Well-established technology
\end{itemize}

\textbf{Components:}
\begin{itemize}
    \item Lasers (external)
    \item Optical circuits (on-die)
    \item Digital signal processors (on-die)
    \item Electronic control circuits
\end{itemize}
\end{definition}

% TODO: Add image from SCD_11_mgt.pdf, Slide 37
% Description: Co-packaged optics architecture diagram
% Priority: IMPORTANT
% Suggested filename: lecture11_cpo_architecture.png
\\
% \includegraphics[width=\linewidth]{lecture11_cpo_architecture.png}
\\
% WHEN YOU ADD IMAGE: Uncomment line above (remove %)

% TODO: Add image from SCD_11_mgt.pdf, Slide 38
% Description: Photonic modulator types (Microring and Mach-Zehnder) with operation diagrams
% Priority: IMPORTANT
% Suggested filename: lecture11_photonic_modulators.png
\\
% \includegraphics[width=\linewidth]{lecture11_photonic_modulators.png}
\\
% WHEN YOU ADD IMAGE: Uncomment line above (remove %)

\begin{remark}
\textbf{Industry Trend:} As bandwidth demands continue to increase, the industry is moving toward:
\begin{itemize}
    \item Direct optical integration on chip packages
    \item Eventually, full photonic integration including lasers
    \item Eliminates electrical bottlenecks at chip I/O
    \item Enables unprecedented aggregate bandwidths ($>$ 10 Tbps)
\end{itemize}
\end{remark}

% ===== IMAGE SUMMARY =====
% Total images needed: 15
% CRITICAL priority: 6
% IMPORTANT priority: 6
% SUPPLEMENTARY priority: 3
%
% Quick extraction checklist:
% [X] [SCD_11_mgt.pdf, Slide 1] - Title slide (SUPPLEMENTARY)
% [X] [SCD_11_mgt.pdf, Slides 2-5] - Exercise slides (included in text)
% [X] [SCD_11_mgt.pdf, Slide 6] - Motivation text (included in text)
% [X] [SCD_11_mgt.pdf, Slide 7] - Goals text (included in text)
% [X] [SCD_11_mgt.pdf, Slide 8] - Challenges text (included in text)
% [ ] [SCD_11_mgt.pdf, Slide 9] - RX Equalizer LPM mode diagram (CRITICAL)
% [ ] [SCD_11_mgt.pdf, Slide 10] - CDR architecture detail (CRITICAL)
% [ ] [SCD_11_mgt.pdf, Slide 11] - Word alignment diagram (CRITICAL)
% [ ] [SCD_11_mgt.pdf, Slide 12] - Complete RX block diagram (IMPORTANT)
% [X] [SCD_11_mgt.pdf, Slide 13] - Line code goals (included in text)
% [X] [SCD_11_mgt.pdf, Slides 14-20] - 8b/10b encoding details (included in text)
% [X] [SCD_11_mgt.pdf, Slides 21-23] - 64b/66b encoding details (included in text)
% [ ] [SCD_11_mgt.pdf, Slide 24] - Transceiver overview diagram (IMPORTANT)
% [ ] [SCD_11_mgt.pdf, Slide 25] - GTH channel topology (CRITICAL)
% [ ] [SCD_11_mgt.pdf, Slide 26] - Data width example (CRITICAL)
% [ ] [SCD_11_mgt.pdf, Slide 27] - GTH quad configuration (CRITICAL)
% [ ] [SCD_11_mgt.pdf, Slide 28] - PLL frequency ranges (IMPORTANT)
% [X] [SCD_11_mgt.pdf, Slides 29-35] - Exercises with solutions (included in text)
% [X] [SCD_11_mgt.pdf, Slide 36] - Future hardware intro (included in text)
% [ ] [SCD_11_mgt.pdf, Slide 37] - CPO architecture (IMPORTANT)
% [ ] [SCD_11_mgt.pdf, Slide 38] - Photonic modulators (IMPORTANT)
% =====================