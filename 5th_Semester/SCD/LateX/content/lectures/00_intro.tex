\section{Introduction}

\subsection{Course Information}

\begin{remark}
\textbf{Course:} System on Chip Design (SCD)

\textbf{Institution:} Zürcher Hochschule für Angewandte Wissenschaften (ZHAW)

\textbf{Instructors:}
\begin{itemize}
    \item Tobias Welti (welo@zhaw.ch, +41 58 934 67 30)
    \item Dominique Cachin (cacd@zhaw.ch, +41 79 45559 01)
\end{itemize}
\end{remark}

\begin{definition}{Course Materials and Schedule}
\begin{itemize}
    \item \textbf{Platform:} Moodle - https://moodle.zhaw.ch/course/view.php?id=25948
    \item \textbf{Script:} Available on Moodle (scd\_script.pdf)
    \item \textbf{Lab Instructions:} https://github.zhaw.ch/pages/hpmm/scd-labs/index.html
\end{itemize}
\end{definition}

% TODO: Add image from SCD_1a_what_is_SoC.pdf, Page 4
% Description: Course schedule table showing lecture topics and lab assignments
% Priority: IMPORTANT
% Suggested filename: lecture01a_course_schedule.png
% \includegraphics[width=\linewidth]{lecture01a_course_schedule.png}

\subsection{Assessment and Grading}

\begin{definition}{Grading Components}
\begin{itemize}
    \item \textbf{Electronic Quiz:} 15\% (November 11, 2025, Moodle test)
    \item \textbf{Lab Exercises:} 15\% (6 labs during semester, graded by lecturer)
    \item \textbf{Written Exam:} 70\% (January 2026, Moodle test)
\end{itemize}
\end{definition}

\begin{definition}{Lab Grading System}\\
Seven labs (four lessons each) contribute to the lab grade.

\textbf{Credits per Lab:}
\begin{itemize}
    \item Not done: 0 points
    \item Required tasks done with small errors: 1 point
    \item Required tasks done without errors: 2 points
\end{itemize}

\textbf{Lab Grade Formula:}
$$\text{Lab Grade} = \frac{\text{Sum of Points}}{12} \times 5 + 1$$
\end{definition}

\begin{remark}
\textbf{Exam Guidelines:}
\begin{itemize}
    \item Open book: lecture and lab notes, personal notes, books allowed
    \item No generative AI such as ChatGPT
    \item Calculators allowed
\end{itemize}
\end{remark}

\subsection{Course Objectives}

\begin{concept}{Target Audience}\\
This course is designed for engineers who want to:
\begin{itemize}
    \item Design high-performance digital circuits with SoC-FPGAs, beyond writing VHDL code
    \item Gain in-depth background knowledge of SoC and FPGA (for software engineers)
    \item Design systems with Linux on SoC-FPGA
    \item Obtain introduction and basic knowledge of Integrated Circuit design
    \item Design general high-speed digital systems with complex peripherals (DDRAM)
\end{itemize}
\end{concept}

\begin{concept}{Learning Goals}\\
By the end of this course, students will be able to:
\begin{itemize}
    \item Work with FPGA block memory
    \item Configure a FPGA-SoC with ARM hardcore processor
    \item Configure the I/O and computer peripherals (DRAM) of a FPGA
    \item Port Yocto Linux to SoC-FPGA
    \item Configure and analyse timing to drive synthesis
    \item Configure clock generators in FPGA and route clocks on PCB
    \item Check signal integrity of clock and data lines on PCB
    \item Explain differences between different signaling standards
    \item Connect high-speed FPGA peripherals with differential signals
    \item Realize a project with video and audio output (Pacman game)
\end{itemize}
\end{concept}

\subsection{Lab Setup}

\begin{definition}{Laboratory Environment}\\
\textbf{Location:} Lab TE 519

\textbf{Equipment:}
\begin{itemize}
    \item Lab PCs with required software setup running on Linux
    \item DE1-SoC Development Board with Intel Cyclone V SoC FPGA
    \item Hardware only available in lab (not distributed to students)
\end{itemize}

\textbf{Work Organization:}
\begin{itemize}
    \item Students work in teams of two
    \item Lab instructions available on GitHub
\end{itemize}
\end{definition}

\raggedcolumns
\columnbreak