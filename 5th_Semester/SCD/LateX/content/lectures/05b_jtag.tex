\section{JTAG - Joint Test Action Group}

\subsection{Overview of IC Testing}

\begin{concept}{Agenda}\\
This lecture covers:
\begin{itemize}
    \item Overview of IC testing approaches
    \item Functional vs Structural testing
    \item Introduction to JTAG standard
    \item Core Scan methodology
    \item Boundary Scan technique
    \item Boundary Scan Register architecture
    \item TAP (Test Access Port) Controller
\end{itemize}
\end{concept}

\subsection{General Test Procedure}

\begin{definition}{Standard Test Methodology}\\
Testing integrated circuits follows three fundamental steps:
\begin{enumerate}
    \item Set hardware to a defined state
    \item Execute one or multiple clock cycles
    \item Read out the resulting state and verify
\end{enumerate}

\textbf{Critical Questions:}
\begin{itemize}
    \item How do we set the hardware to a defined state?
    \item How can we read out the current state without modifying it?
\end{itemize}
\end{definition}

% TODO: Add image from SCD_5b_jtag.pdf, Page 3-4
% Description: General test procedure diagrams
% Priority: CRITICAL
% Suggested filename: lecture05B_test_procedure.png

\subsection{Functional Testing Challenges}

\mult{2}

\begin{concept}{Simple Logic IC Testing}\\
\textbf{Characteristics:}
\begin{itemize}
    \item Limited number of flip-flops
    \item Limited number of states
    \item All states can be tested
    \item Functional test is feasible
\end{itemize}
\end{concept}

\begin{concept}{Complex IC Testing}\\
\textbf{Challenges:}
\begin{itemize}
    \item Many test cases needed
    \item Time-consuming execution
    \item Time-consuming development
    \item Test coverage problematic
    \item Debugging support lacking
\end{itemize}
\end{concept}

\multend

% TODO: Add image from SCD_5b_jtag.pdf, Page 7
% Description: Functional test development time trends (1977-1990)
% Priority: IMPORTANT
% Suggested filename: lecture05B_test_time_trend.png

\begin{remark}
As IC complexity increased, functional test development time grew from 3-6 months (1977-1980) to 12-24 months (1987-1990). This trend necessitated alternative testing approaches.
\end{remark}

\subsection{Shift Register Fundamentals}

% TODO: Add images from SCD_5b_jtag.pdf, Pages 8-9
% Description: Shift register operation diagrams
% Priority: IMPORTANT
% Suggested filename: lecture05B_shift_register.png

\begin{concept}{Shift Register Operation}\\
A shift register chains flip-flops together, allowing serial data input and parallel state observation:
\begin{itemize}
    \item Data shifts through on each clock edge
    \item Enables serial loading of test patterns
    \item Allows serial readout of circuit state
    \item Forms basis for scan chain testing
\end{itemize}
\end{concept}

\raggedcolumns
\columnbreak

\subsection{JTAG Standard Overview}

\begin{definition}{JTAG - Joint Test Action Group}\\
\textbf{Standardization:}
\begin{itemize}
    \item Standardized in 1990
    \item IEEE 1149.1-1990 standard
    \item Purpose: Testing of parts and components
    \item Enables boundary scan testing
    \item Provides debug access
\end{itemize}

\textbf{Key Features:}
\begin{itemize}
    \item Serial test interface (4-5 pins)
    \item Standardized test access port
    \item Boundary scan capability
    \item Device identification
    \item Program/debug support
\end{itemize}
\end{definition}

% TODO: Add images from SCD_5b_jtag.pdf, Pages 10-15
% Description: JTAG standard details and architecture
% Priority: CRITICAL
% Suggested filename: lecture05B_jtag_standard_XX.png

\subsection{Core Scan Methodology}

% TODO: Add images from SCD_5b_jtag.pdf, Pages 16-20
% Description: Core scan chain diagrams
% Priority: CRITICAL
% Suggested filename: lecture05B_core_scan_XX.png

\begin{concept}{Core Scan Technique}\\
Core scan converts internal flip-flops into a shift register for testing:

\textbf{Normal Mode:}
\begin{itemize}
    \item Flip-flops operate in functional mode
    \item Data flows through combinational logic
\end{itemize}

\textbf{Test Mode:}
\begin{itemize}
    \item Flip-flops form scan chain
    \item Test patterns shifted in serially
    \item Results shifted out serially
    \item Provides observability and controllability
\end{itemize}
\end{concept}

\subsection{Boundary Scan}

% TODO: Add images from SCD_5b_jtag.pdf, Pages 21-28
% Description: Boundary scan architecture and operation
% Priority: CRITICAL
% Suggested filename: lecture05B_boundary_scan_XX.png

\begin{definition}{Boundary Scan Architecture}\\
Boundary scan places test cells between core logic and I/O pins:

\textbf{Capabilities:}
\begin{itemize}
    \item Test interconnections between ICs
    \item Program internal devices (FPGA configuration)
    \item Debug embedded systems
    \item Observe/control I/O pins
\end{itemize}

\textbf{Boundary Scan Register:}
\begin{itemize}
    \item Shift register around IC perimeter
    \item One cell per I/O pin
    \item Can capture, shift, and update data
    \item Provides pin-level access
\end{itemize}
\end{definition}

\subsection{TAP Controller}

% TODO: Add images from SCD_5b_jtag.pdf, Pages 29-38
% Description: TAP controller state machine and signals
% Priority: CRITICAL
% Suggested filename: lecture05B_tap_controller_XX.png

\begin{definition}{Test Access Port (TAP)}\\
\textbf{TAP Signals:}
\begin{itemize}
    \item \textbf{TCK:} Test Clock - synchronizes test operations
    \item \textbf{TMS:} Test Mode Select - controls TAP state machine
    \item \textbf{TDI:} Test Data In - serial data input
    \item \textbf{TDO:} Test Data Out - serial data output
    \item \textbf{TRST:} Test Reset (optional) - asynchronous reset
\end{itemize}
\end{definition}

\begin{KR}{JTAG Testing Procedure}\\
\textbf{Step 1: Access JTAG Interface}
\begin{itemize}
    \item Connect to TCK, TMS, TDI, TDO pins
    \item Initialize TAP controller
\end{itemize}

\textbf{Step 2: Select Test Operation}
\begin{enumerate}
    \item Navigate TAP state machine using TMS
    \item Select instruction register or data register
    \item Load instruction (e.g., EXTEST, SAMPLE, BYPASS)
\end{enumerate}

\textbf{Step 3: Execute Test}
\begin{enumerate}
    \item Shift test data through TDI
    \item Capture internal/boundary scan data
    \item Shift results out through TDO
    \item Verify expected vs actual results
\end{enumerate}
\end{KR}

\begin{example2}{JTAG Applications}\\
\textbf{Common Uses:}
\begin{itemize}
    \item \textbf{Manufacturing Test:} Verify PCB assembly and interconnects
    \item \textbf{FPGA Configuration:} Program configuration memory
    \item \textbf{Firmware Programming:} Flash memory programming
    \item \textbf{Debug Access:} Access processor debug features
    \item \textbf{Boundary Scan:} Test IC-to-IC connections
\end{itemize}
\end{example2}

\begin{remark}
JTAG has become the standard interface for device programming, debugging, and testing in modern embedded systems. Its serial nature minimizes pin count while providing comprehensive test access.
\end{remark}

% ===== IMAGE SUMMARY =====
% Total images needed: 38
% CRITICAL priority: 25
% IMPORTANT priority: 10
% SUPPLEMENTARY priority: 3
%
% Quick extraction checklist:
% [ ] [SCD_5b_jtag.pdf, Pages 1-38] - Complete JTAG lecture content (CRITICAL for technical diagrams)
% =====================