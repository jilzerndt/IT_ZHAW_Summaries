\section{FPGA Block Memory}

\subsection{Introduction to FPGA Block Memory}

\begin{definition}{FPGA Block Memory}\\
FPGA block memory consists of small blocks of random access memory integrated into the FPGA fabric.

\textbf{Key Characteristics:}
\begin{itemize}
    \item Used instead of thousands of flip-flops for data storage
    \item Multiple applications in digital design
    \item Dual-port capability for simultaneous access
    \item Configurable data widths and addressing modes
\end{itemize}
\end{definition}

% TODO: Add image from SCD_1b_Memory_1.pdf, Page 2
% Description: FPGA block memory symbol showing ports and parameters
% Priority: CRITICAL
% Suggested filename: lecture01b_block_memory_symbol.png
% \includegraphics[width=\linewidth]{lecture01b_block_memory_symbol.png}

\begin{remark}
FPGA fabric contains not only logic cells but also dedicated SRAM blocks and DSP blocks for efficient implementation of memory and arithmetic operations.
\end{remark}

% TODO: Add image from SCD_1b_Memory_1.pdf, Page 3
% Description: FPGA fabric diagram showing SRAM and DSP blocks
% Priority: IMPORTANT
% Suggested filename: lecture01b_fpga_fabric_blocks.png
% \includegraphics[width=\linewidth]{lecture01b_fpga_fabric_blocks.png}

\subsection{Learning Objectives}

\begin{concept}{Course Learning Goals}\\
After this lecture and lab exercise, students will be able to:
\begin{itemize}
    \item Understand the architecture of FPGA block memory
    \item Explain the difference between registered and unregistered output from a memory block
    \item Understand the consequences of either configuration choice
    \item Sketch and interpret timing diagrams of memory accesses
    \item List at least four applications of FPGA block memory
\end{itemize}
\end{concept}

\subsection{Applications of FPGA Block Memory}

\begin{example}
\textbf{Common Applications:}
\begin{itemize}
    \item \textbf{Look-Up Tables (LUT):} e.g., Synthesizer sine wave generation
    \item \textbf{ROM/RAM/Cache:} In SoC processors or softcore processors
    \item \textbf{Configuration/Status Register Bank:}
    \begin{itemize}
        \item Port A: Processor access
        \item Port B: Logic access
    \end{itemize}
    \item \textbf{Cache for Flash Memory:} Speed up access to slow external flash
    \item \textbf{FIFOs:}
    \begin{itemize}
        \item Receive/transmit buffers
        \item Clock crossing bridges
    \end{itemize}
\end{itemize}
\end{example}

\subsubsection{Direct Digital Synthesis (DDS)}

\begin{concept}{LUT in Direct Digital Synthesis}\\
In DDS, a lookup table stores pre-calculated values of a sine wave period.

\textbf{Process:}
\begin{enumerate}
    \item Phase accumulator generates address (phase index)
    \item LUT outputs corresponding sine value
    \item Low-pass filter smooths the output
    \item Result: Clean sinusoidal waveform
\end{enumerate}

\textbf{Advantages:}
\begin{itemize}
    \item Fast generation of complex waveforms
    \item Precise frequency control
    \item Low resource usage
\end{itemize}
\end{concept}

% TODO: Add image from SCD_1b_Memory_1.pdf, Page 8
% Description: DDS block diagram with LUT, accumulator, and low-pass filter
% Priority: IMPORTANT
% Suggested filename: lecture01b_dds_diagram.png
% \includegraphics[width=\linewidth]{lecture01b_dds_diagram.png}

\subsubsection{FIFO Buffers}

\begin{definition}{FIFO (First-In-First-Out) Buffer}\\
A FIFO is a data structure where the first element added is the first one to be removed.

\textbf{Key Operations:}
\begin{itemize}
    \item \textbf{Enqueue:} Add element to back of queue
    \item \textbf{Dequeue:} Remove element from front of queue
\end{itemize}

\textbf{Use Cases:}
\begin{itemize}
    \item Buffering data between different clock domains
    \item Rate matching between fast and slow interfaces
    \item Packet buffering in communication systems
\end{itemize}
\end{definition}

% TODO: Add image from SCD_1b_Memory_1.pdf, Page 9
% Description: FIFO buffer illustration showing enqueue and dequeue operations
% Priority: IMPORTANT
% Suggested filename: lecture01b_fifo_buffer.png
% \includegraphics[width=\linewidth]{lecture01b_fifo_buffer.png}

\subsubsection{Clock Crossing Bridge}

\begin{example2}{Using Block Memory for Clock Domain Crossing}\\
\textbf{Application:} Decoupling clock frequencies in audio systems

\textbf{Example Configuration:}
\begin{itemize}
    \item Write Clock (WR\_CLK): 12.288 MHz (I2S audio clock)
    \item Read Clock (RD\_CLK): 50 MHz (system clock)
    \item FIFO depth: 256 words
    \item Serial audio data input via I2S protocol
\end{itemize}

\tcblower

\textbf{Benefits:}
\begin{itemize}
    \item Safely transfers data between clock domains
    \item Prevents metastability issues
    \item Allows for rate buffering
    \item Provides elastic buffer for timing variations
\end{itemize}
\end{example2}

% TODO: Add image from SCD_1b_Memory_1.pdf, Page 10
% Description: Clock crossing bridge using FIFO with different clock domains
% Priority: CRITICAL
% Suggested filename: lecture01b_clock_crossing_fifo.png
% \includegraphics[width=\linewidth]{lecture01b_clock_crossing_fifo.png}

\raggedcolumns
\columnbreak

\subsubsection{Cache for Flash Memory}

\begin{concept}{Memory Hierarchy}\\
Memory systems follow a general principle: \textbf{larger = slower = cheaper}

\textbf{Hierarchy (fast to slow):}
\begin{enumerate}
    \item Registers (fastest, smallest, most expensive per bit)
    \item Cache (FPGA Block RAM)
    \item Main Memory (external DRAM)
    \item Flash Memory (non-volatile, slowest)
\end{enumerate}

Block RAM can serve as cache to speed up access to slow external flash memory.
\end{concept}

% TODO: Add image from SCD_1b_Memory_1.pdf, Page 11
% Description: Memory hierarchy diagram showing flash and block RAM
% Priority: IMPORTANT
% Suggested filename: lecture01b_memory_hierarchy.png
% \includegraphics[width=\linewidth]{lecture01b_memory_hierarchy.png}

\subsubsection{Character-Based Graphics Controller}

\begin{example2}{Video Controller with Character Memory}\\
\textbf{System Overview:}
\begin{itemize}
    \item \textbf{Character Set:} 64 unique characters
    \item \textbf{Screen Buffer:} 3072 character positions
    \item Uses block RAM to store character and color data
\end{itemize}

\tcblower

\textbf{Architecture Components:}
\begin{itemize}
    \item \textbf{Character RAM:} Stores which character at each position
    \item \textbf{Color RAM:} Stores color attributes for each position
    \item \textbf{Graphic RAM:} Stores pixel patterns for characters
    \item \textbf{Video Counter:} Generates timing signals
    \item \textbf{Color Multiplexer:} Selects output colors
\end{itemize}
\end{example2}

% TODO: Add image from SCD_1b_Memory_1.pdf, Page 12
% Description: Character-based graphics display example
% Priority: SUPPLEMENTARY
% Suggested filename: lecture01b_character_graphics.png
% \includegraphics[width=\linewidth]{lecture01b_character_graphics.png}

% TODO: Add image from SCD_1b_Memory_1.pdf, Page 13
% Description: Block diagram of video controller showing all RAM blocks and data paths
% Priority: CRITICAL
% Suggested filename: lecture01b_video_controller_diagram.png
% \includegraphics[width=\linewidth]{lecture01b_video_controller_diagram.png}

\subsection{SRAM Cell Architecture}

\begin{definition}{Volatile CMOS Memory Cell}\\
SRAM (Static Random Access Memory) uses a cross-coupled inverter pair to store one bit.

\textbf{Components:}
\begin{itemize}
    \item \textbf{Two Cross-Coupled Inverters:} Form a bistable latch
    \item \textbf{Two Access Transistors:} Connect cell to bit lines
    \item \textbf{Word Line:} Activates access transistors for read/write
    \item \textbf{Bit Lines (true and complement):} Transfer data in/out
\end{itemize}

\textbf{Characteristics:}
\begin{itemize}
    \item Volatile: Data lost when power removed
    \item Static: No refresh needed (unlike DRAM)
    \item Fast: Single-cycle access possible
    \item Large: 6 transistors per bit
\end{itemize}
\end{definition}

% TODO: Add image from SCD_1b_Memory_1.pdf, Page 15
% Description: CMOS SRAM cell circuit diagram
% Priority: CRITICAL
% Suggested filename: lecture01b_sram_cell_circuit.png
% \includegraphics[width=\linewidth]{lecture01b_sram_cell_circuit.png}

\begin{concept}{Organization of Memory Cells}\\
In a discrete SRAM chip, memory cells are organized in a matrix:
\begin{itemize}
    \item \textbf{Word Lines:} Select rows of cells
    \item \textbf{Bit Lines:} Transfer data from/to selected cells
    \item \textbf{Address Decoder:} Activates appropriate word line
    \item \textbf{Sense Amplifiers:} Detect small voltage changes during read
    \item \textbf{Write Drivers:} Drive bit lines during write operations
    \item \textbf{RD/WR\# Control:} Selects read or write operation
\end{itemize}
\end{concept}

% TODO: Add image from SCD_1b_Memory_1.pdf, Page 16
% Description: SRAM organization showing word lines, bit lines, and decoder
% Priority: CRITICAL
% Suggested filename: lecture01b_sram_organization.png
% \includegraphics[width=\linewidth]{lecture01b_sram_organization.png}

% TODO: Add image from SCD_1b_Memory_1.pdf, Page 17
% Description: Die photo of SRAM chip showing physical layout
% Priority: SUPPLEMENTARY
% Suggested filename: lecture01b_sram_die_photo.png
% \includegraphics[width=\linewidth]{lecture01b_sram_die_photo.png}

\raggedcolumns
\columnbreak

\subsection{FPGA RAM Block Architecture}

\begin{definition}{Cyclone V Memory Blocks}\\
Cyclone V FPGAs contain two types of RAM blocks:

\begin{itemize}
    \item \textbf{M10K Block:} 10 Kbit capacity
    \begin{itemize}
        \item Larger memory blocks for bulk storage
        \item Can be configured in various width/depth combinations
    \end{itemize}
    \item \textbf{MLAB (Memory Logic Array Block):} 640 bit capacity
    \begin{itemize}
        \item Smaller, distributed memory
        \item Can be used as small FIFOs or shift registers
    \end{itemize}
\end{itemize}

\textbf{Key Feature:} Address decoder and control logic are integrated, enabling fast clock speeds.
\end{definition}

% TODO: Add image from SCD_1b_Memory_1.pdf, Page 18
% Description: FPGA RAM block structure
% Priority: IMPORTANT
% Suggested filename: lecture01b_fpga_ram_block.png
% \includegraphics[width=\linewidth]{lecture01b_fpga_ram_block.png}

\begin{concept}{Multiport RAM Architecture}\\
FPGA RAM blocks feature dual-port architecture:

\textbf{Characteristics:}
\begin{itemize}
    \item \textbf{Two Sides:} Port A and Port B
    \item \textbf{Independent Clocks:} Each port can have different clock
    \item \textbf{Different Data Widths:} Port A and B can have different widths
    \item \textbf{Mixed-Port Data Flow:} Write on one port, read on other
    \item \textbf{Same-Port Data Flow:} Read and write on same port
\end{itemize}

\textbf{Example Application:}
\begin{itemize}
    \item Port A: HPS (Hard Processor System) access
    \item Port B: FPGA logic access
    \item Result: Clock-independent shared memory between CPU and FPGA
\end{itemize}
\end{concept}

% TODO: Add image from SCD_1b_Memory_1.pdf, Page 19
% Description: Dual-port RAM block diagram showing Port A and Port B
% Priority: CRITICAL
% Suggested filename: lecture01b_dual_port_ram.png
% \includegraphics[width=\linewidth]{lecture01b_dual_port_ram.png}

\subsection{Memory Timing}

\subsubsection{SRAM Read Cycle}

\begin{definition}{SRAM Read Timing Parameters}\\
\textbf{Key Timing Parameters:}
\begin{itemize}
    \item \textbf{$t_{RC}$ (Read Cycle Time):} Minimum time the address must be applied (typical: 10 ns)
    \item \textbf{$t_{AA}$ (Address Access Time):} Time from address stable until data valid (typical: 10 ns)
    \item \textbf{$t_{OHA}$ (Output Hold Time):} Time data remains valid after address change (typical: 2 ns)
\end{itemize}

\textbf{Read Sequence:}
\begin{enumerate}
    \item Apply stable address
    \item Wait for $t_{AA}$
    \item Data becomes valid at output
    \item Data remains valid for $t_{OHA}$ after address changes
\end{enumerate}
\end{definition}

% TODO: Add image from SCD_1b_Memory_1.pdf, Page 20
% Description: SRAM read cycle timing diagram
% Priority: CRITICAL
% Suggested filename: lecture01b_sram_read_timing.png
% \includegraphics[width=\linewidth]{lecture01b_sram_read_timing.png}

\subsubsection{SRAM Write Cycle}

\begin{definition}{SRAM Write Timing Parameters}\\
\textbf{Key Timing Parameters:}
\begin{itemize}
    \item \textbf{$t_{WC}$ (Write Cycle):} Minimum time an address must be valid
    \item \textbf{$t_{SA}$ (Address Setup Time):} Minimum time address must be applied before WE active (typical: 0 ns)
    \item \textbf{$t_{PWE}$ (Write Pulse Width):} Minimal width of write pulse (typical: 8 ns)
    \item \textbf{$t_{HA}$ (Address Hold):} Minimum time address may go unstable after rising edge of WE (typical: 0 ns)
    \item \textbf{$t_{SD}$ (Data Setup Time):} Minimum time data must be stable before rising edge of WE (typical: 7 ns)
    \item \textbf{$t_{HD}$ (Data Hold Time):} Minimum time data must remain stable after rising edge of WE (typical: 0 ns)
\end{itemize}

\textbf{Write Sequence:}
\begin{enumerate}
    \item Apply address ($t_{SA}$ before WE)
    \item Apply data ($t_{SD}$ before WE rising edge)
    \item Assert WE (write enable) for at least $t_{PWE}$
    \item Hold address for $t_{HA}$ after WE
    \item Hold data for $t_{HD}$ after WE
\end{enumerate}
\end{definition}

% TODO: Add image from SCD_1b_Memory_1.pdf, Page 21
% Description: SRAM write cycle timing diagram
% Priority: CRITICAL
% Suggested filename: lecture01b_sram_write_timing.png
% \includegraphics[width=\linewidth]{lecture01b_sram_write_timing.png}

\raggedcolumns
\columnbreak

\subsection{FPGA Memory Read and Write Operations}

\subsubsection{Write Operation}

\begin{definition}{FPGA Memory Write Timing}\\
Writing to FPGA block memory is synchronous (clock-driven).

\textbf{Signals:}
\begin{itemize}
    \item \textbf{CLK:} Clock signal
    \item \textbf{WREN:} Write enable (active high)
    \item \textbf{ADDRESS:} Memory address
    \item \textbf{DATA\_IN:} Data to be written
\end{itemize}

\textbf{Write Process:}
\begin{enumerate}
    \item On rising edge of CLK:
    \begin{itemize}
        \item If WREN = 1: DATA\_IN is written to ADDRESS
        \item If WREN = 0: No write occurs
    \end{itemize}
    \item Data is stored in memory cell
    \item Write completes in one clock cycle
\end{enumerate}
\end{definition}

% TODO: Add image from SCD_1b_Memory_1.pdf, Page 22
% Description: FPGA memory write block diagram and timing
% Priority: CRITICAL
% Suggested filename: lecture01b_fpga_write_timing.png
% \includegraphics[width=\linewidth]{lecture01b_fpga_write_timing.png}

\begin{example}
\textbf{Sequential Writing:}
Writing to consecutive memory addresses:
\begin{itemize}
    \item Clock cycle 1: Write to address a0
    \item Clock cycle 2: Write to address a1
    \item Clock cycle 3: Write to address a2
    \item Each write occurs on the rising edge of CLK when WREN is high
\end{itemize}
\end{example}

% TODO: Add image from SCD_1b_Memory_1.pdf, Page 23
% Description: Sequential write timing diagram
% Priority: IMPORTANT
% Suggested filename: lecture01b_sequential_write.png
% \includegraphics[width=\linewidth]{lecture01b_sequential_write.png}

\subsubsection{Read Operation - Asynchronous Mode}

\begin{definition}{Asynchronous Read Mode}\\
In asynchronous (unregistered) read mode, data appears at the output after internal memory access delay.

\textbf{Characteristics:}
\begin{itemize}
    \item \textbf{No output register:} Data comes directly from memory array
    \item \textbf{Faster initial access:} Data available within same clock cycle
    \item \textbf{Combinatorial path:} Creates longer critical path
    \item \textbf{Lower $f_{max}$:} Reduces maximum achievable clock frequency
\end{itemize}

\textbf{Read Enable:} Optional signal (rden) can gate the output
\end{definition}

% TODO: Add image from SCD_1b_Memory_1.pdf, Page 24
% Description: FPGA memory read block diagram (asynchronous)
% Priority: IMPORTANT
% Suggested filename: lecture01b_fpga_read_async.png
% \includegraphics[width=\linewidth]{lecture01b_fpga_read_async.png}

\subsubsection{Read Operation - Synchronous Mode}

\begin{definition}{Synchronous Read Mode}\\
In synchronous (registered) read mode, data is latched in an output register.

\textbf{Characteristics:}
\begin{itemize}
    \item \textbf{Output register:} Data latched on clock edge
    \item \textbf{One cycle latency:} Data available one clock cycle after address
    \item \textbf{Registered path:} Breaks combinatorial path
    \item \textbf{Higher $f_{max}$:} Enables higher clock frequencies
\end{itemize}

\textbf{Timing:}
\begin{itemize}
    \item Clock cycle $n$: Address applied
    \item Clock cycle $n+1$: Data available at output
\end{itemize}
\end{definition}

% TODO: Add image from SCD_1b_Memory_1.pdf, Page 25
% Description: FPGA memory read block diagram (synchronous)
% Priority: CRITICAL
% Suggested filename: lecture01b_fpga_read_sync.png
% \includegraphics[width=\linewidth]{lecture01b_fpga_read_sync.png}

% TODO: Add image from SCD_1b_Memory_1.pdf, Page 26
% Description: Synchronous read timing diagram showing one-cycle delay
% Priority: CRITICAL
% Suggested filename: lecture01b_sync_read_timing.png
% \includegraphics[width=\linewidth]{lecture01b_sync_read_timing.png}

\subsection{Comparison of Read Modes}

\begin{highlight}{Asynchronous vs Synchronous Read}\\
\textbf{Path Delay Analysis:}

\textbf{Asynchronous Mode:}
$$\text{Path delay} = t_{clk\rightarrow q} + t_{mem} + t_{rout} + t_{su} = 0.5 + 2.5 + 1.5 + 0.5 = 5.0 \text{ ns}$$
$$f_{max} = 200 \text{ MHz}$$

\textbf{Synchronous Mode:}
\begin{itemize}
    \item Memory to register: $t_{mem} = 2.5$ ns
    \item Register to logic cell: Standard registered path
    \item $f_{max} = 286$ MHz (from memory)
    \item $f_{max} = 400$ MHz (M10K block capability)
\end{itemize}

\important{Synchronous mode enables higher clock frequencies by breaking the combinatorial path, at the cost of one cycle latency.}
\end{highlight}

% TODO: Add image from SCD_1b_Memory_1.pdf, Page 27
% Description: Path delay comparison diagram for async vs sync modes
% Priority: CRITICAL
% Suggested filename: lecture01b_read_mode_comparison.png
% \includegraphics[width=\linewidth]{lecture01b_read_mode_comparison.png}

% TODO: Add image from SCD_1b_Memory_1.pdf, Page 28
% Description: Timing diagram comparing async and sync read
% Priority: IMPORTANT
% Suggested filename: lecture01b_async_vs_sync_timing.png
% \includegraphics[width=\linewidth]{lecture01b_async_vs_sync_timing.png}

\begin{remark}
\textbf{Trade-offs:}

\textbf{Asynchronous Mode (no output FF):}
\begin{itemize}
    \item[+] Data available after one clock cycle
    \item[-] Combinatorial path is longer (slower clock possible)
\end{itemize}

\textbf{Synchronous Mode (with output FF):}
\begin{itemize}
    \item[+] Higher maximum clock frequency
    \item[+] Better timing closure
    \item[-] One additional cycle latency
\end{itemize}
\end{remark}

\raggedcolumns
\columnbreak

\subsection{Performance Specifications}

\begin{highlight}{Speed of Embedded Memory in Cyclone V}\\
\textbf{Performance by Configuration:}

\begin{center}
\begin{tabular}{|l|c|c|c|c|}
\hline
\textbf{Memory Mode} & \textbf{ALUTs} & \textbf{Memory} & \textbf{Speed} & \textbf{Unit} \\
\hline
\multicolumn{5}{|c|}{\textbf{Asynchronous Read (No Output Register)}} \\
\hline
Single port & 0 & 1 & 450/380/330 & MHz \\
Simple dual-port & 0 & 1 & 450/380/330 & MHz \\
ROM & 0 & 1 & 450/380/330 & MHz \\
\hline
\multicolumn{5}{|c|}{\textbf{Synchronous Read (With Output Register)}} \\
\hline
Single port & 0 & 1 & 315/275/240 & MHz \\
Simple dual-port & 0 & 1 & 315/275/240 & MHz \\
Simple dual-port (old data) & 0 & 1 & 275/240/180 & MHz \\
True dual-port & 0 & 1 & 315/275/240 & MHz \\
ROM & 0 & 1 & 315/275/240 & MHz \\
\hline
\multicolumn{5}{|c|}{\textbf{Clock Timing Constraints}} \\
\hline
Min pulse width (high) & - & - & 1450/1550/1650 & ps \\
Min pulse width (low) & - & - & 1000/1200/1350 & ps \\
\hline
\end{tabular}
\end{center}

\textbf{Notes:}
\begin{itemize}
    \item Speed grades shown as: C6 / C7 / C8 (fastest to slowest)
    \item To achieve maximum performance, use global clock routing from on-chip PLL
    \item Set PLL to 50\% output duty cycle
    \item No $f_{max}$ degradation when using CRC error detection
\end{itemize}
\end{highlight}

% TODO: Add image from SCD_1b_Memory_1.pdf, Page 29
% Description: Full performance table from datasheet
% Priority: IMPORTANT
% Suggested filename: lecture01b_performance_table.png
% \includegraphics[width=\linewidth]{lecture01b_performance_table.png}

\subsection{Memory Block Port Configuration}

\begin{definition}{Dual-Port Memory Interface}\\
FPGA memory blocks provide two independent ports (A and B) with the following signals:

\textbf{Port A Signals:}
\begin{itemize}
    \item \textbf{DATA\_IN\_A:} Input data
    \item \textbf{ADRESSEN\_A:} Address bus
    \item \textbf{WREN\_A:} Write enable
    \item \textbf{RDEN\_A:} Read enable (optional)
    \item \textbf{BYTEEN\_A:} Byte enable for partial writes
    \item \textbf{DATA\_OUT\_A:} Output data
\end{itemize}

\textbf{Port B Signals:}
\begin{itemize}
    \item \textbf{DATA\_IN\_B:} Input data
    \item \textbf{ADRESSEN\_B:} Address bus
    \item \textbf{WREN\_B:} Write enable
    \item \textbf{RDEN\_B:} Read enable (optional)
    \item \textbf{BYTEEN\_B:} Byte enable for partial writes
    \item \textbf{DATA\_OUT\_B:} Output data
\end{itemize}

Both ports can operate independently with different clocks, addresses, and data widths.
\end{definition}

% TODO: Add image from SCD_1b_Memory_1.pdf, Page 30
% Description: Conceptual drawing of Altera memory block with dual ports
% Priority: CRITICAL
% Suggested filename: lecture01b_dual_port_concept.png
% \includegraphics[width=\linewidth]{lecture01b_dual_port_concept.png}

% TODO: Add image from SCD_1b_Memory_1.pdf, Page 32
% Description: Physical layout of M10K rows in FPGA
% Priority: SUPPLEMENTARY
% Suggested filename: lecture01b_m10k_layout.png
% \includegraphics[width=\linewidth]{lecture01b_m10k_layout.png}

\subsection{Memory Resources Summary}

\begin{highlight}{Memory Resources in Cyclone V}\\
\textbf{Cyclone V Device Family Resources:}

\begin{center}
\begin{tabular}{|l|c|c|c|c|}
\hline
\textbf{Resource} & \textbf{5CEA2} & \textbf{5CEA4} & \textbf{5CEA5} & \textbf{5CEA9} \\
\hline
Logic Elements & 25K & 49K & 85K & 150K \\
M10K Blocks & 125 & 200 & 300 & 500 \\
Memory (Kbit) & 1,250 & 2,000 & 3,000 & 5,000 \\
\hline
Variable-precision DSP & & & & \\
18×19 Multiplier & 36 & 84 & 144 & 234 \\
\hline
Hard Floating-point & Yes & Yes & Yes & Yes \\
Fractional Synthesis & 2 & 4 & 4 & 6 \\
\hline
12.5 Gbps Transceiver & - & - & 4 & 6 \\
\hline
\end{tabular}
\end{center}

\textbf{Key Points:}
\begin{itemize}
    \item Larger devices have proportionally more memory blocks
    \item M10K blocks provide flexible memory implementation
    \item DSP blocks complement memory for signal processing
    \item High-speed transceivers available in larger devices
\end{itemize}
\end{highlight}

% TODO: Add image from SCD_1b_Memory_1.pdf, Page 33
% Description: Memory resources table for Cyclone V family
% Priority: IMPORTANT
% Suggested filename: lecture01b_cyclone_resources.png
% \includegraphics[width=\linewidth]{lecture01b_cyclone_resources.png}

% ===== IMAGE SUMMARY =====
% Total images needed: 23
% CRITICAL priority: 11
% IMPORTANT priority: 9
% SUPPLEMENTARY priority: 3
%
% Quick extraction checklist:
% [ ] [SCD_1b_Memory_1.pdf, Page 2] - Block memory symbol with ports (CRITICAL)
% [ ] [SCD_1b_Memory_1.pdf, Page 3] - FPGA fabric with SRAM and DSP blocks (IMPORTANT)
% [ ] [SCD_1b_Memory_1.pdf, Page 8] - DDS block diagram (IMPORTANT)
% [ ] [SCD_1b_Memory_1.pdf, Page 9] - FIFO buffer illustration (IMPORTANT)
% [ ] [SCD_1b_Memory_1.pdf, Page 10] - Clock crossing FIFO diagram (CRITICAL)
% [ ] [SCD_1b_Memory_1.pdf, Page 11] - Memory hierarchy diagram (IMPORTANT)
% [ ] [SCD_1b_Memory_1.pdf, Page 12] - Character graphics example (SUPPLEMENTARY)
% [ ] [SCD_1b_Memory_1.pdf, Page 13] - Video controller block diagram (CRITICAL)
% [ ] [SCD_1b_Memory_1.pdf, Page 15] - SRAM cell circuit (CRITICAL)
% [ ] [SCD_1b_Memory_1.pdf, Page 16] - SRAM organization (CRITICAL)
% [ ] [SCD_1b_Memory_1.pdf, Page 17] - SRAM die photo (SUPPLEMENTARY)
% [ ] [SCD_1b_Memory_1.pdf, Page 18] - FPGA RAM block structure (IMPORTANT)
% [ ] [SCD_1b_Memory_1.pdf, Page 19] - Dual-port RAM diagram (CRITICAL)
% [ ] [SCD_1b_Memory_1.pdf, Page 20] - SRAM read timing diagram (CRITICAL)
% [ ] [SCD_1b_Memory_1.pdf, Page 21] - SRAM write timing diagram (CRITICAL)
% [ ] [SCD_1b_Memory_1.pdf, Page 22] - FPGA write timing (CRITICAL)
% [ ] [SCD_1b_Memory_1.pdf, Page 23] - Sequential write timing (IMPORTANT)
% [ ] [SCD_1b_Memory_1.pdf, Page 24] - FPGA read async diagram (IMPORTANT)
% [ ] [SCD_1b_Memory_1.pdf, Page 25] - FPGA read sync diagram (CRITICAL)
% [ ] [SCD_1b_Memory_1.pdf, Page 26] - Sync read timing diagram (CRITICAL)
% [ ] [SCD_1b_Memory_1.pdf, Page 27] - Read mode comparison (CRITICAL)
% [ ] [SCD_1b_Memory_1.pdf, Page 28] - Async vs sync timing (IMPORTANT)
% [ ] [SCD_1b_Memory_1.pdf, Page 29] - Performance table (IMPORTANT)
% [ ] [SCD_1b_Memory_1.pdf, Page 30] - Dual-port concept drawing (CRITICAL)
% [ ] [SCD_1b_Memory_1.pdf, Page 32] - M10K physical layout (SUPPLEMENTARY)
% [ ] [SCD_1b_Memory_1.pdf, Page 33] - Cyclone V resources table (IMPORTANT)
% =====================