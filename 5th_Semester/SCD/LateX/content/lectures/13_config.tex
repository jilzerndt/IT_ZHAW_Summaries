\section{FPGA Configuration}

\subsection{Motivation and Learning Objectives}

\begin{concept}{Motivation}\\
Understanding FPGA configuration is essential for:
\begin{itemize}
    \item System bring-up and debugging
    \item Selecting appropriate configuration strategy
    \item Designing robust embedded systems
    \item Optimizing boot time and reliability
\end{itemize}
\end{concept}

% TODO: Add image from SCD_13_config.pdf, Slide 2
% Description: Motivation diagram showing FPGA with question marks on various configuration aspects
% Priority: SUPPLEMENTARY
% Suggested filename: lecture13_motivation.png
\\
% \includegraphics[width=0.8\linewidth]{lecture13_motivation.png}
\\
% WHEN YOU ADD IMAGE: Uncomment line above (remove %)

\begin{definition}{Learning Objectives}\\
After this lesson, you should be able to:
\begin{enumerate}
    \item Explain the operation principle of volatile and non-volatile FPGA technologies
    \item Identify configuration modes from a schematic of an FPGA and configuration device
    \item Calculate the necessary size of a configuration device
    \item Estimate the configuration time for a given FPGA device
    \item Choose between active and passive configuration according to requirements
\end{enumerate}
\end{definition}

\subsection{Configuration Overview}

\subsubsection{Logic Block Configuration}

\begin{definition}{Logic Cell Configuration Parameters}\\
Each logic cell in the FPGA requires configuration for multiple aspects:

\textbf{LUT Configuration:}
\begin{itemize}
    \item LUT truth table contents (function to implement)
    \item Determines combinational logic behavior
\end{itemize}

\textbf{Flip-Flop Configuration:}
\begin{itemize}
    \item Bypass flip-flop (combinational only)
    \item Use flip-flop (register output)
\end{itemize}

\textbf{Input Multiplexer:}
\begin{itemize}
    \item Select input sources
    \item Route signals from interconnect
\end{itemize}

\textbf{Clock Selection:}
\begin{itemize}
    \item Choose clock source
    \item Clock enable signals
\end{itemize}
\end{definition}

% TODO: Add image from SCD_13_config.pdf, Slide 6
% Description: Logic block diagram showing configurable LUT, FF, input MUX, and clock select
% Priority: CRITICAL
% Suggested filename: lecture13_logic_block_config.png
\\
% \includegraphics[width=\linewidth]{lecture13_logic_block_config.png}
\\
% WHEN YOU ADD IMAGE: Uncomment line above (remove %)

\subsubsection{Signal Routing Configuration}

\begin{definition}{Routing Configuration}\\
The programmable interconnect requires configuration for:

\textbf{Signal Routing:}
\begin{itemize}
    \item Connection matrices between logic blocks
    \item Routing multiplexers
    \item Wire segment selection
\end{itemize}

\textbf{LAB Connections:}
\begin{itemize}
    \item Connections to adjacent LABs (Logic Array Blocks)
    \item Register chains between LABs
    \item Carry chains for arithmetic
\end{itemize}

\textbf{Clock Region Crossings:}
\begin{itemize}
    \item Clock domain boundaries
    \item Global vs. regional clock routing
    \item Clock buffer insertion
\end{itemize}

\important{Complexity:} Routing configuration typically consumes the majority of configuration bits.
\end{definition}

% TODO: Add image from SCD_13_config.pdf, Slide 7
% Description: Signal routing diagram showing LAB connections and routing configuration
% Priority: CRITICAL
% Suggested filename: lecture13_routing_config.png
\\
% \includegraphics[width=\linewidth]{lecture13_routing_config.png}
\\
% WHEN YOU ADD IMAGE: Uncomment line above (remove %)

\subsubsection{I/O Configuration}

\begin{definition}{I/O Pin Configuration}\\
Each I/O pin requires extensive configuration:

\textbf{I/O Routing:}
\begin{itemize}
    \item Connection to FPGA fabric
    \item Direct input/output paths
\end{itemize}

\textbf{Timing Configuration:}
\begin{itemize}
    \item I/O clock selection
    \item Input and output delays
    \item Setup and hold adjustments
\end{itemize}

\textbf{Registers:}
\begin{itemize}
    \item Input register enable
    \item Output register enable
    \item Output Enable (OE) register
\end{itemize}

\textbf{Electrical Properties:}
\begin{itemize}
    \item Voltage standard (VCCIO)
    \item Pull-up resistor enable
    \item Bus hold enable
    \item Signal polarity (inverting/non-inverting)
    \item Drive strength (current capability)
\end{itemize}
\end{definition}

% TODO: Add image from SCD_13_config.pdf, Slide 8
% Description: I/O configuration diagram showing registers, delays, and electrical settings
% Priority: CRITICAL
% Suggested filename: lecture13_io_config.png
\\
% \includegraphics[width=\linewidth]{lecture13_io_config.png}
\\
% WHEN YOU ADD IMAGE: Uncomment line above (remove %)

\raggedcolumns
\columnbreak

\subsubsection{PLL Configuration}

\begin{definition}{Phase-Locked Loop Configuration}\\
PLLs require configuration for clock generation and distribution:

\textbf{Multiplexer Configuration:}
\begin{itemize}
    \item Input clock source selection
    \item Output routing selection
\end{itemize}

\textbf{Frequency Control:}
\begin{itemize}
    \item Divider ratios (input and output)
    \item Multiplication factors
    \item Feedback path configuration
\end{itemize}

\textbf{Phase and Timing:}
\begin{itemize}
    \item Phase delays for each output
    \item Duty cycle adjustment
\end{itemize}

\textbf{Control Signals:}
\begin{itemize}
    \item Clock source selection
    \item Output enable for each clock
    \item Power-down mode
\end{itemize}
\end{definition}

% TODO: Add image from SCD_13_config.pdf, Slide 9
% Description: PLL configuration diagram showing dividers, MUX, and clock source selection
% Priority: IMPORTANT
% Suggested filename: lecture13_pll_config.png
\\
% \includegraphics[width=\linewidth]{lecture13_pll_config.png}
\\
% WHEN YOU ADD IMAGE: Uncomment line above (remove %)

\subsubsection{Block Memory Configuration}

\begin{definition}{Block RAM Configuration}\\
Embedded memory blocks require configuration for:

\textbf{Output Path:}
\begin{itemize}
    \item Output registers enable
    \item Bypass registers for lower latency
\end{itemize}

\textbf{Port Configuration:}
\begin{itemize}
    \item Port widths (data and address)
    \item Read/write enable signals
    \item Byte enable configuration
\end{itemize}

\textbf{Operating Modes:}
\begin{itemize}
    \item Single-port mode
    \item Dual-port mode (true dual-port or simple dual-port)
    \item FIFO mode
    \item Independent clocks for each port
\end{itemize}

\textbf{Memory Initialization:}
\begin{itemize}
    \item Initial memory contents
    \item ROM functionality
    \item Lookup tables
\end{itemize}

\important{Size Impact:} Memory initialization data can significantly increase bitstream size.
\end{definition}

% TODO: Add image from SCD_13_config.pdf, Slide 10
% Description: Block memory configuration showing output registers, port widths, and modes
% Priority: IMPORTANT
% Suggested filename: lecture13_memory_config.png
\\
% \includegraphics[width=\linewidth]{lecture13_memory_config.png}
\\
% WHEN YOU ADD IMAGE: Uncomment line above (remove %)

\subsection{FPGA Technologies}

\subsubsection{SRAM-Based FPGAs}

\begin{definition}{SRAM Configuration Technology}\\
\textbf{Manufacturers:} Xilinx, Intel (Altera), Lattice

\textbf{Storage Mechanism:}
\begin{itemize}
    \item Uses SRAM cells for configuration storage
    \item Each configuration bit stored in one SRAM cell
    \item Standard 6-transistor SRAM cell design
\end{itemize}

\textbf{SRAM Cell Structure:}
\begin{itemize}
    \item Two cross-coupled inverters (bistable latch)
    \item Two access transistors (word line control)
    \item Connected to bit line and word line
    \item Drives configuration item directly
\end{itemize}

\textbf{Characteristics:}
\begin{itemize}
    \item \textbf{Volatile storage:} Configuration lost when power removed
    \item Must be configured every time after power-on
    \item Fast reconfiguration possible
    \item Unlimited rewrite cycles
    \item Requires external non-volatile memory for boot
\end{itemize}

\textbf{Advantages:}
\begin{itemize}
    \item Mature technology
    \item High density
    \item Fast configuration
    \item Easy partial reconfiguration
\end{itemize}

\textbf{Disadvantages:}
\begin{itemize}
    \item Volatile (requires boot memory)
    \item Susceptible to radiation-induced upsets (SEU)
    \item More complex boot process
\end{itemize}
\end{definition}

% TODO: Add image from SCD_13_config.pdf, Slide 12
% Description: SRAM cell diagram showing transistors, bit line, word line, and connection to config item
% Priority: CRITICAL
% Suggested filename: lecture13_sram_cell.png
\\
% \includegraphics[width=0.8\linewidth]{lecture13_sram_cell.png}
\\
% WHEN YOU ADD IMAGE: Uncomment line above (remove %)

\raggedcolumns
\columnbreak

\subsubsection{Antifuse Technology}

\begin{definition}{Antifuse Configuration Technology}\\
\textbf{Manufacturers:} QuickLogic, Microsemi (older products)

\textbf{Operating Principle:}
\begin{itemize}
    \item Antifuse: opposite of a fuse
    \item \textbf{Unprogrammed state:} Open circuit (insulation intact)
    \item \textbf{Programmed state:} Closed circuit (insulation blown)
\end{itemize}

\textbf{Structure:}
\begin{itemize}
    \item Thin insulation layer between two conductors
    \item High voltage applied to blow insulation
    \item Creates permanent low-resistance connection
    \item Transfer gates control programming voltage
\end{itemize}

\textbf{Programming Process:}
\begin{enumerate}
    \item Apply high voltage ($V_{CC} + $ programming voltage)
    \item Current breaks down insulation layer
    \item Forms conductive channel
    \item Leave insulation intact for open connections
\end{enumerate}

\textbf{Characteristics:}
\begin{itemize}
    \item \textbf{One-time programmable (OTP):} Process not reversible
    \item Non-volatile: retains configuration without power
    \item Very small cell size (high density)
    \item Immune to radiation upsets
    \item No boot time required
\end{itemize}

\textbf{Advantages:}
\begin{itemize}
    \item Non-volatile (instant-on)
    \item Small die size
    \item High security (difficult to reverse-engineer)
    \item Radiation-hard
    \item Low power
\end{itemize}

\textbf{Disadvantages:}
\begin{itemize}
    \item Cannot be reconfigured
    \item Higher cost per unit
    \item Limited applications
\end{itemize}
\end{definition}

% TODO: Add image from SCD_13_config.pdf, Slide 13
% Description: Antifuse structure showing transfer gate
% Priority: IMPORTANT
% Suggested filename: lecture13_antifuse_structure.png
\\
% \includegraphics[width=0.7\linewidth]{lecture13_antifuse_structure.png}
\\
% WHEN YOU ADD IMAGE: Uncomment line above (remove %)

% TODO: Add image from SCD_13_config.pdf, Slide 14
% Description: Antifuse cross-section showing VCC, insulation layer, and programming
% Priority: IMPORTANT
% Suggested filename: lecture13_antifuse_detail.png
\\
% \includegraphics[width=0.8\linewidth]{lecture13_antifuse_detail.png}
\\
% WHEN YOU ADD IMAGE: Uncomment line above (remove %)

\subsubsection{Flash-Based FPGAs}

\begin{definition}{Flash Configuration Technology}\\
\textbf{Manufacturers:} Microsemi (Actel), Lattice

\textbf{Storage Mechanism:}
\begin{itemize}
    \item Uses flash memory cells
    \item Floating-gate transistor technology
    \item Same principle as standard flash memory
\end{itemize}

\textbf{Flash Cell Structure:}
\begin{itemize}
    \item Floating gate isolated by silicon dioxide (SiO$_2$)
    \item Control gate above floating gate
    \item Source, drain, and substrate
\end{itemize}

\textbf{Programming and Erasing:}

\paragraph{Write cell to '0' (ON state):}
\begin{enumerate}
    \item Apply high voltage ($U_P$) to control gate
    \item Electrons tunnel through SiO$_2$ to floating gate
    \item Charge deposited on floating gate
    \item Transistor conducts when control gate = '1'
\end{enumerate}

\paragraph{Erase cell to '1' (OFF state):}
\begin{enumerate}
    \item Apply negative voltage ($-U_P$) to control gate
    \item Electrons tunnel out of floating gate
    \item Floating gate discharged
    \item Transistor blocks regardless of control gate value
\end{enumerate}

\textbf{Characteristics:}
\begin{itemize}
    \item \textbf{Non-volatile:} Retains configuration without power
    \item \textbf{Rewritable:} Can be reprogrammed multiple times
    \item Limited write cycles ($\sim 10^5$ to $10^6$)
    \item Silicon dioxide (SiO$_2$) is excellent insulator
    \item Instant-on capability
\end{itemize}

\textbf{Advantages:}
\begin{itemize}
    \item Non-volatile (no boot memory needed)
    \item Reprogrammable (unlike antifuse)
    \item Single-chip solution
    \item Low power consumption
    \item Radiation tolerant
    \item Secure (difficult to read out)
\end{itemize}

\textbf{Disadvantages:}
\begin{itemize}
    \item Lower density than SRAM
    \item Limited write cycles
    \item Slower programming time
    \item Higher cost
\end{itemize}
\end{definition}

% TODO: Add image from SCD_13_config.pdf, Slide 15
% Description: Flash cell diagram showing floating gate, control gate, and ON/OFF states
% Priority: CRITICAL
% Suggested filename: lecture13_flash_cell.png
\\
% \includegraphics[width=\linewidth]{lecture13_flash_cell.png}
\\
% WHEN YOU ADD IMAGE: Uncomment line above (remove %)

\begin{remark}
\textbf{Technology Comparison Summary:}
\begin{center}
\begin{tabular}{|l|c|c|c|}
\hline
\textbf{Property} & \textbf{SRAM} & \textbf{Antifuse} & \textbf{Flash} \\
\hline
Volatile & Yes & No & No \\
Reprogrammable & Yes & No & Yes \\
Density & High & Highest & Medium \\
Boot time & Required & None & None \\
Write cycles & Unlimited & 1 & $10^5$-$10^6$ \\
Cost & Medium & High & High \\
Security & Medium & High & High \\
\hline
\end{tabular}
\end{center}
\end{remark}

\raggedcolumns
\columnbreak

\subsection{Active Serial Configuration}

\subsubsection{Active Configuration Concept}

\begin{definition}{Active Serial Configuration Mode}\\
\textbf{Principle:} FPGA takes active role in configuration process.

\textbf{FPGA Responsibilities:}
\begin{itemize}
    \item Drives DCLK (Data Clock) using internal oscillator
    \item Controls configuration memory via control lines
    \item Generates addresses for memory reads
    \item Manages configuration sequence
\end{itemize}

\textbf{Configuration Memory Role:}
\begin{itemize}
    \item Passive device (SPI flash memory)
    \item Responds to FPGA commands
    \item Outputs data on request
\end{itemize}

\textbf{Mode Selection:}
\begin{itemize}
    \item Mode select pins determine configuration mode
    \item Set during board design (pull-up/pull-down)
    \item Read by FPGA at power-up
\end{itemize}
\end{definition}

% TODO: Add image from SCD_13_config.pdf, Slide 17
% Description: Active serial configuration block diagram showing FPGA, config memory, and mode select
% Priority: CRITICAL
% Suggested filename: lecture13_active_serial.png
\\
% \includegraphics[width=\linewidth]{lecture13_active_serial.png}
\\
% WHEN YOU ADD IMAGE: Uncomment line above (remove %)

\subsubsection{Configuration Controller}

\begin{definition}{FPGA Configuration Controller}\\
The configuration controller is a dedicated hardware block inside the FPGA:

\textbf{Functions:}
\begin{enumerate}
    \item \textbf{Power Detection:}
    \begin{itemize}
        \item Detects stable power supply
        \item Begins configuration sequence when stable
    \end{itemize}
    \item \textbf{Configuration Trigger:}
    \begin{itemize}
        \item Responds to nConfig input
        \item Initiates configuration on request
    \end{itemize}
    \item \textbf{Power Monitoring:}
    \begin{itemize}
        \item Monitors supply voltage
        \item Flags power interruptions
        \item Restarts configuration if needed
    \end{itemize}
    \item \textbf{Status Indication:}
    \begin{itemize}
        \item Indicates configuration complete
        \item Signals errors if detected
    \end{itemize}
    \item \textbf{Clock Generation:}
    \begin{itemize}
        \item Generates configuration clock (DCLK)
        \item Typical frequency: 40 MHz
        \item Drives external flash memory
    \end{itemize}
\end{enumerate}
\end{definition}

% TODO: Add image from SCD_13_config.pdf, Slide 18
% Description: Configuration controller block diagram showing mode select, power detection, and control signals
% Priority: CRITICAL
% Suggested filename: lecture13_config_controller.png
\\
% \includegraphics[width=\linewidth]{lecture13_config_controller.png}
\\
% WHEN YOU ADD IMAGE: Uncomment line above (remove %)

\subsubsection{Configuration Timing}

\begin{definition}{Active Serial Configuration Signals}\\
\textbf{Key Signals and Their Functions:}

\paragraph{nConfig (Input):}
\begin{itemize}
    \item Active-low configuration trigger
    \item Assertion starts configuration
    \item Can be driven externally or pulled high
\end{itemize}

\paragraph{nStatus (Output):}
\begin{itemize}
    \item Indicates configuration in progress
    \item Low during configuration
    \item High when idle or complete
\end{itemize}

\paragraph{nCSO (Output):}
\begin{itemize}
    \item Chip Select Output (active-low)
    \item Selects external configuration flash
    \item Controls flash memory enable
\end{itemize}

\paragraph{DCLK (Output):}
\begin{itemize}
    \item Data Clock for configuration
    \item Drives external flash memory clock
    \item Typical frequency: 40 MHz
\end{itemize}

\paragraph{ASDO (Output):}
\begin{itemize}
    \item Active Serial Data Output
    \item Serial data from FPGA to flash memory
    \item Transfers read address to flash memory
\end{itemize}

\paragraph{DATA0 (Input):}
\begin{itemize}
    \item Serial data from flash memory to FPGA
    \item Transfers configuration bitstream
    \item Clocked by DCLK
\end{itemize}

\paragraph{CONF\_DONE (Output):}
\begin{itemize}
    \item Indicates configuration finished
    \item Goes high when complete
    \item 200 DCLK cycles later, FPGA switches to user mode
\end{itemize}

\paragraph{INIT\_DONE (Output):}
\begin{itemize}
    \item Signals user mode active
    \item High when FPGA in functional mode
    \item FPGA design begins execution
\end{itemize}
\end{definition}

% TODO: Add image from SCD_13_config.pdf, Slide 19
% Description: Configuration timing diagram showing all signals during configuration sequence
% Priority: CRITICAL
% Suggested filename: lecture13_config_timing.png
\\
% \includegraphics[width=\linewidth]{lecture13_config_timing.png}
\\
% WHEN YOU ADD IMAGE: Uncomment line above (remove %)

\begin{KR}{Active Serial Configuration Sequence}\\
\textbf{Step-by-step process:}

\paragraph{1. Power-Up and Initialization:}
\begin{itemize}
    \item Power supplies stabilize
    \item Configuration controller initializes
    \item Mode select pins sampled
\end{itemize}

\paragraph{2. Configuration Start:}
\begin{itemize}
    \item nConfig asserted or power-up detected
    \item nStatus goes low (configuration in progress)
    \item nCSO asserted (selects flash memory)
\end{itemize}

\paragraph{3. Data Transfer:}
\begin{itemize}
    \item DCLK begins toggling
    \item ASDO sends read address to flash
    \item DATA0 receives configuration data
    \item FPGA loads configuration bits sequentially
\end{itemize}

\paragraph{4. Configuration Complete:}
\begin{itemize}
    \item All configuration bits loaded
    \item CONF\_DONE goes high
    \item Wait 200 DCLK cycles
\end{itemize}

\paragraph{5. User Mode:}
\begin{itemize}
    \item FPGA switches to user mode
    \item INIT\_DONE goes high
    \item User design begins operation
\end{itemize}
\end{KR}

% TODO: Add image from SCD_13_config.pdf, Slide 20
% Description: Detailed timing diagram with annotations explaining each signal transition
% Priority: IMPORTANT
% Suggested filename: lecture13_timing_detail.png
\\
% \includegraphics[width=\linewidth]{lecture13_timing_detail.png}
\\
% WHEN YOU ADD IMAGE: Uncomment line above (remove %)

\raggedcolumns
\columnbreak

\subsection{Configuration Data Size}

\begin{concept}{Bitstream Size Determination}\\
\textbf{Size Factors:}
\begin{itemize}
    \item Number of configurable items depends on FPGA fabric size
    \item Logic elements (LUTs, FFs)
    \item Routing resources
    \item Block RAM initialization
    \item I/O configuration
    \item PLL settings
\end{itemize}

\textbf{Finding Bitstream Size:}
\begin{itemize}
    \item Datasheet states bitstream size for each device variant
    \item Varies significantly between device families
    \item Must select configuration memory accordingly
\end{itemize}

\textbf{Multiple FPGA Configuration:}
\begin{itemize}
    \item Multiple FPGAs can be configured from one memory
    \item Daisy-chain configuration
    \item Sum of individual bitstream sizes
\end{itemize}
\end{concept}

\begin{definition}{Configuration Memory Selection}\\
\textbf{Intel/Altera Configuration Devices:}
\begin{itemize}
    \item Dedicated configuration memory devices
    \item Various capacities: EPCS1, EPCS4, EPCS16, EPCS64, EPCS128
    \item Number indicates capacity in Megabits
\end{itemize}

\textbf{Selection Criteria:}
\begin{itemize}
    \item Select smallest device that can store all necessary data
    \item Consider:
    \begin{itemize}
        \item Bitstream size
        \item Optional: Bootloader code
        \item Optional: Operating system
        \item Optional: Application software
    \end{itemize}
\end{itemize}

\textbf{Rule of Thumb:}
$$\text{Bitstream Size} \approx \text{Number of LUTs} \times 0.2 \text{ Mbit/1k LUTs}$$

\important{Note:} Block memories also influence bitstream size. Always consult datasheet to avoid surprises.
\end{definition}

\begin{example2}{Configuration Memory Size Calculation}\\
\textbf{Given:} System with multiple FPGAs:
\begin{itemize}
    \item Two Cyclone II EP2C20 FPGAs (one master, one slave, different content)
    \item One Cyclone II EP2C35 FPGA (slave)
\end{itemize}

\textbf{Task:} Choose minimal EPCS configuration device.

\tcblower

\textbf{Solution:}

From datasheet, bitstream sizes:
\begin{itemize}
    \item EP2C20: 3{,}892{,}496 bits per device
    \item EP2C35: 6{,}848{,}608 bits per device
\end{itemize}

Total bitstream size:
$$\text{Total} = 2 \times 3{,}892{,}496 + 6{,}848{,}608 = 14{,}633{,}600 \text{ bits}$$

Convert to Megabits:
$$14{,}633{,}600 \text{ bits} = 14.63 \text{ Mbit}$$

\important{Answer:} EPCS16 (16 Mbit) is the smallest sufficient device.
\end{example2}

\subsection{SPI Flash Memory Interface}

\begin{definition}{SPI Flash Configuration Memory}\\
\textbf{Memory Type:}
\begin{itemize}
    \item Configuration data stored in non-volatile memory
    \item Usually SPI (Serial Peripheral Interface) flash
    \item Also: QSPI (Quad SPI) flash for higher speed
\end{itemize}

\textbf{Standard Interface:}
\begin{itemize}
    \item Standard SPI protocol
    \item Compatible with many flash devices
    \item Common vendors: Micron, Winbond, Macronix, Cypress
\end{itemize}

\textbf{Signal Mapping:}
\begin{center}
\begin{tabular}{|l|l|l|}
\hline
\textbf{FPGA Signal} & \textbf{SPI Signal} & \textbf{Function} \\
\hline
DATA0 & SO (Serial Out) & Data from flash to FPGA \\
ASDO & SI (Serial In) & Address/commands to flash \\
DCLK & SCK (Serial Clock) & Clock signal \\
nCSO & CS (Chip Select) & Flash enable (active-low) \\
\hline
\end{tabular}
\end{center}
\end{definition}

% TODO: Add image from SCD_13_config.pdf, Slide 23
% Description: FPGA to SPI flash connection diagram showing signal mapping
% Priority: CRITICAL
% Suggested filename: lecture13_spi_flash_connection.png
\\
% \includegraphics[width=\linewidth]{lecture13_spi_flash_connection.png}
\\
% WHEN YOU ADD IMAGE: Uncomment line above (remove %)

\begin{definition}{SPI Flash Read Protocol}\\
\textbf{Read Cycle Structure:}
\begin{enumerate}
    \item \textbf{8-bit Command Word:}
    \begin{itemize}
        \item READ command (typically 0x03)
        \item Fast READ command (0x0B)
        \item Other commands for status, ID, etc.
    \end{itemize}
    \item \textbf{24-bit Address:}
    \begin{itemize}
        \item Specifies memory location
        \item Allows up to 16 MB addressing
        \item Sent MSB first
    \end{itemize}
    \item \textbf{8-bit Data Words:}
    \begin{itemize}
        \item Configuration data
        \item Sequential reads possible
        \item Address auto-increments
    \end{itemize}
\end{enumerate}

\textbf{Timing:}
\begin{itemize}
    \item SCK driven by FPGA DCLK
    \item SI (ASDO) transmits command and address
    \item SO (DATA0) returns data bytes
    \item CS must be low during transaction
\end{itemize}
\end{definition}

% TODO: Add image from SCD_13_config.pdf, Slide 24
% Description: SPI flash read cycle timing diagram showing command, address, and data phases
% Priority: CRITICAL
% Suggested filename: lecture13_spi_read_cycle.png
\\
% \includegraphics[width=\linewidth]{lecture13_spi_read_cycle.png}
\\
% WHEN YOU ADD IMAGE: Uncomment line above (remove %)

\raggedcolumns
\columnbreak

\subsection{Passive Serial Configuration}

\begin{definition}{Passive Configuration Mode}\\
\textbf{Principle:} FPGA has passive role in configuration.

\textbf{External Controller:}
\begin{itemize}
    \item External device drives clock
    \item External device drives control lines
    \item External device provides data
    \item Could be: Microcontroller, CPLD, another FPGA
\end{itemize}

\textbf{FPGA Role:}
\begin{itemize}
    \item Configuration controller inactive
    \item Accepts data passively
    \item Monitors configuration progress
\end{itemize}

\textbf{Use Cases:}
\begin{itemize}
    \item Reconfiguration during runtime
    \item Custom boot sequences
    \item Supervised configuration
    \item Partial reconfiguration applications
\end{itemize}

\important{Flexibility:} Allows complete control over configuration process by external logic.
\end{definition}

% TODO: Add image from SCD_13_config.pdf, Slide 26
% Description: Passive serial configuration diagram showing external controller driving FPGA
% Priority: IMPORTANT
% Suggested filename: lecture13_passive_serial.png
\\
% \includegraphics[width=\linewidth]{lecture13_passive_serial.png}
\\
% WHEN YOU ADD IMAGE: Uncomment line above (remove %)

\subsubsection{Configuration Chains}

\begin{definition}{Multi-FPGA Configuration Chain}\\
\textbf{Chain Structure:}
\begin{itemize}
    \item One Master device (FPGA or external config device)
    \item One or multiple Slave devices (passive mode)
    \item Single configuration memory for several FPGAs
\end{itemize}

\textbf{Operation:}
\begin{enumerate}
    \item Master initiates configuration
    \item Master receives data from memory
    \item After Master configured, data passes through to first Slave
    \item Each Slave takes its bitstream from data stream
    \item Chain continues until all FPGAs configured
\end{enumerate}

\textbf{Signal Connections:}
\begin{itemize}
    \item Master FPGA:
    \begin{itemize}
        \item Active mode configuration
        \item nCE always active (tied low or controlled)
    \end{itemize}
    \item Slave FPGAs:
    \begin{itemize}
        \item Passive mode configuration
        \item nCE controlled by previous device in chain
    \end{itemize}
    \item Last device:
    \begin{itemize}
        \item nCE output left open (no more devices)
    \end{itemize}
\end{itemize}

\textbf{Advantages:}
\begin{itemize}
    \item Single configuration memory
    \item Reduced board complexity
    \item Lower cost
\end{itemize}

\textbf{Disadvantages:}
\begin{itemize}
    \item Sequential configuration (slower)
    \item Failure in one device affects all
    \item More complex debugging
\end{itemize}
\end{definition}

% TODO: Add image from SCD_13_config.pdf, Slide 27
% Description: Configuration chain diagram showing master and slave FPGAs with nCE connections
% Priority: CRITICAL
% Suggested filename: lecture13_config_chain.png
\\
% \includegraphics[width=\linewidth]{lecture13_config_chain.png}
\\
% WHEN YOU ADD IMAGE: Uncomment line above (remove %)

\subsection{Parallel Configuration}

\begin{concept}{Parallel Configuration Mode}\\
\textbf{Principle:} Transfer multiple bits simultaneously.

\textbf{Characteristics:}
\begin{itemize}
    \item Faster configuration than serial modes
    \item Requires more pins (data bus width: 8, 16, or 32 bits)
    \item Suitable for large devices or fast boot requirements
\end{itemize}

\textbf{Trade-offs:}
\begin{itemize}
    \item \textbf{Advantage:} Significantly faster configuration time
    \item \textbf{Disadvantage:} Requires many more I/O pins
    \item \textbf{Disadvantage:} More complex PCB routing
\end{itemize}

\textbf{Applications:}
\begin{itemize}
    \item Large FPGAs with long serial configuration times
    \item Systems requiring fast boot
    \item When I/O pins are available
\end{itemize}
\end{concept}

% TODO: Add image from SCD_13_config.pdf, Slide 28
% Description: Parallel configuration interface diagram
% Priority: SUPPLEMENTARY
% Suggested filename: lecture13_parallel_config.png
\\
% \includegraphics[width=0.8\linewidth]{lecture13_parallel_config.png}
\\
% WHEN YOU ADD IMAGE: Uncomment line above (remove %)

\raggedcolumns
\columnbreak

\subsection{JTAG Configuration}

\subsubsection{Direct JTAG Configuration}

\begin{definition}{JTAG Configuration Mode}\\
\textbf{Advantages:}
\begin{itemize}
    \item Configuration directly through JTAG pins
    \item No additional dedicated configuration pins required
    \item Uses existing boundary scan infrastructure
    \item Standard interface (IEEE 1149.1)
\end{itemize}

\textbf{JTAG Signals:}
\begin{itemize}
    \item TDI (Test Data In)
    \item TDO (Test Data Out)
    \item TCK (Test Clock)
    \item TMS (Test Mode Select)
\end{itemize}

\textbf{Configuration Process:}
\begin{enumerate}
    \item Connect JTAG programmer to FPGA
    \item Programmer sends bitstream through TDI
    \item FPGA loads configuration via TAP controller
    \item Configuration complete, FPGA enters user mode
\end{enumerate}

\textbf{Use Cases:}
\begin{itemize}
    \item Development and debugging
    \item Prototyping
    \item Lab testing
    \item Field updates
\end{itemize}

\textbf{Limitations:}
\begin{itemize}
    \item Volatile (SRAM-based FPGAs only)
    \item Requires JTAG connection
    \item Not suitable for production deployment
    \item Slower than dedicated configuration modes
\end{itemize}
\end{definition}

% TODO: Add image from SCD_13_config.pdf, Slide 30
% Description: JTAG configuration showing FPGA fabric, JTAG TAP controller, and connector
% Priority: IMPORTANT
% Suggested filename: lecture13_jtag_config.png
\\
% \includegraphics[width=\linewidth]{lecture13_jtag_config.png}
\\
% WHEN YOU ADD IMAGE: Uncomment line above (remove %)

\subsubsection{Serial Flash Loader with JTAG}

\begin{definition}{Serial Flash Loader (SFL)}\\
\textbf{Purpose:} Program configuration flash memory through JTAG.

\textbf{Requirements:}
\begin{itemize}
    \item Serial Flash Loader (SFL) IP block in design
    \item JTAG connection
    \item Configuration flash memory
\end{itemize}

\textbf{Two-Step Programming Process:}

\paragraph{Step 1: Load SFL via JTAG:}
\begin{enumerate}
    \item Connect JTAG programmer
    \item Configure FPGA with design including SFL IP
    \item FPGA enters user mode with SFL active
\end{enumerate}

\paragraph{Step 2: SFL Programs Flash:}
\begin{enumerate}
    \item SFL receives bitstream via JTAG
    \item SFL writes bitstream to configuration flash
    \item Flash memory now contains permanent configuration
\end{enumerate}

\textbf{Subsequent Power-Up:}
\begin{enumerate}
    \item FPGA reads configuration from flash (active serial mode)
    \item No JTAG connection needed
    \item Permanent configuration
\end{enumerate}

\textbf{Advantages:}
\begin{itemize}
    \item No need for dedicated flash programmer
    \item Can program flash through JTAG
    \item Useful for field updates
    \item Single connection for development and production
\end{itemize}
\end{definition}

% TODO: Add image from SCD_13_config.pdf, Slide 31
% Description: Serial Flash Loader diagram showing two-step process with JTAG and flash
% Priority: CRITICAL
% Suggested filename: lecture13_sfl_jtag.png
\\
% \includegraphics[width=\linewidth]{lecture13_sfl_jtag.png}
\\
% WHEN YOU ADD IMAGE: Uncomment line above (remove %)

\subsubsection{EPCS Serial Flash Controller}

\begin{definition}{EPCS Controller IP Peripheral}\\
\textbf{Purpose:} Access external flash memory from embedded processor.

\textbf{Architecture:}
\begin{itemize}
    \item Peripheral block in Platform Designer (Qsys)
    \item Connected to processor bus (e.g., Nios V, ARM)
    \item Provides SPI master interface to flash
\end{itemize}

\textbf{Use Cases:}
\begin{itemize}
    \item Store software programs in flash
    \item Store operating system
    \item Store data files
    \item Remote system updates
\end{itemize}

\textbf{Question:} How can we use external flash memory to store software for embedded processor (e.g., Nios softcore)?

\textbf{Answer:} Remote Upgrade capability through EPCS Controller.
\end{definition}

% TODO: Add image from SCD_13_config.pdf, Slide 32
% Description: EPCS controller architecture showing Nios V CPU, JTAG, and flash connections
% Priority: IMPORTANT
% Suggested filename: lecture13_epcs_controller.png
\\
% \includegraphics[width=\linewidth]{lecture13_epcs_controller.png}
\\
% WHEN YOU ADD IMAGE: Uncomment line above (remove %)

\begin{KR}{EPCS Development Mechanism}\\
\textbf{Programming Configuration Flash via EPCS Controller:}

\paragraph{Step 1: Configure FPGA via JTAG}
\begin{itemize}
    \item Load FPGA configuration including:
    \begin{itemize}
        \item EPCS Controller IP
        \item Nios V Softcore processor
        \item Memory interfaces
        \item Other peripherals
    \end{itemize}
\end{itemize}

\paragraph{Step 2: Load Data into Flash}
\begin{itemize}
    \item Use JTAG interface
    \item EPCS Controller IP acts as SPI master
    \item Write bitstream and software to flash memory
\end{itemize}

\textbf{FPGA Power-Up Sequence (After Programming):}

\paragraph{1. FPGA Configuration from Flash:}
\begin{itemize}
    \item FPGA actively configures from flash memory
    \item Hardware loads bitstream
\end{itemize}

\paragraph{2. Processor Boot:}
\begin{itemize}
    \item Nios V processor starts execution
    \item Begins from bootloader in ROM
\end{itemize}

\paragraph{3. Software Loading:}
\begin{itemize}
    \item Bootloader copies program/OS from flash
    \item Destination: Working memory (SDRAM)
\end{itemize}

\paragraph{4. Exception Vector Setup:}
\begin{itemize}
    \item Exception vectors reprogrammed
    \item Point to SDRAM locations
\end{itemize}

\paragraph{5. Execution Start:}
\begin{itemize}
    \item Reset processor
    \item Execute from SDRAM
    \item System operational
\end{itemize}
\end{KR}

% TODO: Add image from SCD_13_config.pdf, Slide 33
% Description: EPCS development mechanism flowchart showing two-step programming and boot sequence
% Priority: CRITICAL
% Suggested filename: lecture13_epcs_development.png
\\
% \includegraphics[width=\linewidth]{lecture13_epcs_development.png}
\\
% WHEN YOU ADD IMAGE: Uncomment line above (remove %)

\raggedcolumns
\columnbreak

\subsection{Development Phase Configuration}

\subsubsection{Volatile JTAG Configuration}

\begin{definition}{Development Workflow}\\
\textbf{JTAG Configuration for Development:}
\begin{itemize}
    \item Program .sof file (SRAM Object File) via JTAG port
    \item Fast iteration cycle
    \item No need to handle SD cards or reprogram flash
    \item Immediate design verification
\end{itemize}

\textbf{Characteristics:}
\begin{itemize}
    \item \textbf{Volatile:} Configuration lost on power cycle
    \item \textbf{Fast:} Quick programming time
    \item \textbf{Convenient:} No external programmer needed
\end{itemize}

\textbf{Typical Development Cycle:}
\begin{enumerate}
    \item Modify HDL code
    \item Compile design (generate .sof)
    \item Program FPGA via JTAG
    \item Test functionality
    \item Iterate
\end{enumerate}

\important{Note:} .sof files are for volatile JTAG configuration. .pof files (Programmer Object File) are for non-volatile flash programming.
\end{definition}

% TODO: Add image from SCD_13_config.pdf, Slide 35
% Description: Volatile JTAG configuration workflow diagram
% Priority: SUPPLEMENTARY
% Suggested filename: lecture13_volatile_jtag.png
\\
% \includegraphics[width=0.8\linewidth]{lecture13_volatile_jtag.png}
\\
% WHEN YOU ADD IMAGE: Uncomment line above (remove %)

\subsubsection{Development Board USB Configuration}

\begin{definition}{USB-Based Configuration on Dev Boards}\\
Many development boards include on-board programming capability:

\textbf{Architecture:}
\begin{itemize}
    \item External CPLD acts as configuration device
    \item USB-to-serial converter (e.g., FTDI chip) interfaces to PC
    \item Run/Program switch selects configuration mode
\end{itemize}

\textbf{Program Mode (Serial Interface):}
\begin{itemize}
    \item USB exposes serial programming interface
    \item Program flash memory via USB and CPLD
    \item Configuration stored on board permanently
    \item No JTAG adapter needed
\end{itemize}

\textbf{Run Mode (JTAG Interface):}
\begin{itemize}
    \item FPGA configured in active mode from flash
    \item USB exposes JTAG interface
    \item PC can reconfigure FPGA via JTAG
    \item Useful for development iterations
\end{itemize}

\textbf{Components:}
\begin{itemize}
    \item FTDI USB chip (USB-to-serial/JTAG converter)
    \item Configuration CPLD (manages modes)
    \item Run/Program switch (mode selection)
    \item Configuration flash memory
    \item FPGA device
\end{itemize}

\important{Convenience:} Single USB connection serves both development (JTAG) and production (flash programming).
\end{definition}

% TODO: Add image from SCD_13_config.pdf, Slide 36
% Description: Development board USB configuration diagram showing CPLD, FTDI, and mode switching
% Priority: IMPORTANT
% Suggested filename: lecture13_devboard_usb.png
\\
% \includegraphics[width=\linewidth]{lecture13_devboard_usb.png}
\\
% WHEN YOU ADD IMAGE: Uncomment line above (remove %)

\subsection{Flash Memory Programming}

\subsubsection{Byte Blaster Connection}

\begin{definition}{Direct Flash Programming}\\
\textbf{Byte Blaster Setup:}
\begin{itemize}
    \item Direct connection to configuration flash
    \item FPGA must be disabled during programming
    \item nCE = 1: disables FPGA configuration interface
\end{itemize}

\textbf{Procedure:}
\begin{enumerate}
    \item Connect Byte Blaster to flash memory
    \item Ensure FPGA nCE = 1 (disabled)
    \item Load .pof file into programmer software
    \item Program flash memory
    \item Remove programmer
    \item Power cycle or toggle nCE to configure FPGA
\end{enumerate}

\textbf{File Type:}
\begin{itemize}
    \item Use .pof files (Programmer Object File)
    \item Contains full flash memory image
    \item Includes bitstream and any additional data
\end{itemize}
\end{definition}

% TODO: Add image from SCD_13_config.pdf, Slide 37
% Description: Byte Blaster connection diagram showing flash, nCE control, and connector
% Priority: IMPORTANT
% Suggested filename: lecture13_byte_blaster.png
\\
% \includegraphics[width=\linewidth]{lecture13_byte_blaster.png}
\\
% WHEN YOU ADD IMAGE: Uncomment line above (remove %)

\begin{concept}{SPI Flash Programming Protocol}\\
\textbf{Write Operation:}
\begin{enumerate}
    \item Send write command
    \item Specify starting address (24-bit)
    \item Send subsequent data bytes
\end{enumerate}

\textbf{Address Auto-Increment:}
\begin{itemize}
    \item When writing to continuous ranges
    \item Address increments automatically
    \item Efficient for large transfers
\end{itemize}

\textbf{Page Programming:}
\begin{itemize}
    \item Flash memories have page structure
    \item Typical page size: 256 bytes
    \item Must respect page boundaries
\end{itemize}

\textbf{Erase Operations:}
\begin{itemize}
    \item Sector erase (typically 4 KB, 32 KB, or 64 KB)
    \item Chip erase (entire memory)
    \item Must erase before programming
\end{itemize}
\end{concept}

% TODO: Add image from SCD_13_config.pdf, Slide 38
% Description: SPI flash programming protocol timing diagram
% Priority: SUPPLEMENTARY
% Suggested filename: lecture13_spi_programming.png
\\
% \includegraphics[width=\linewidth]{lecture13_spi_programming.png}
\\
% WHEN YOU ADD IMAGE: Uncomment line above (remove %)

\raggedcolumns
\columnbreak

\subsection{Exercises}

\begin{example2}{Exercise 1: Configuration Memory Size}\\
\textbf{Given:} Your board will have:
\begin{itemize}
    \item Two Cyclone II EP2C20 FPGAs (one as Master, one as Slave, different content)
    \item One Cyclone II EP2C35 FPGA (as Slave)
\end{itemize}

\textbf{Task:} Choose the minimal FPGA configuration device (EPCS).
\begin{itemize}
    \item Calculate required configuration memory
    \item Select matching EPCS device (storage coded into name in Mbit)
\end{itemize}

\textbf{Rule of Thumb:} 1k LUT requires 0.2 Mbit in bitstream file (plus block memories).

\tcblower

\textbf{Solution:}

From Cyclone II datasheet, bitstream sizes:
\begin{itemize}
    \item EP2C20: 3{,}892{,}496 bits per device
    \item EP2C35: 6{,}848{,}608 bits per device
\end{itemize}

\textbf{Calculation:}
\begin{align*}
\text{Total} &= 2 \times \text{EP2C20} + 1 \times \text{EP2C35} \\
&= 2 \times 3{,}892{,}496 + 6{,}848{,}608 \\
&= 7{,}784{,}992 + 6{,}848{,}608 \\
&= 14{,}633{,}600 \text{ bits}
\end{align*}

Convert to Megabits:
$$14{,}633{,}600 \text{ bits} = 14.63 \text{ Mbit}$$

\textbf{EPCS Device Selection:}
\begin{itemize}
    \item EPCS4: 4 Mbit (too small)
    \item EPCS16: 16 Mbit (\checkmark sufficient)
    \item EPCS64: 64 Mbit (unnecessary)
\end{itemize}

\important{Answer:} EPCS16 is the minimal sufficient device.
\end{example2}

\begin{example2}{Exercise 2: Configuration Time}\\
\textbf{Given:}
\begin{itemize}
    \item Two EP2C20 FPGA devices in daisy chain
    \item One EP2C20 has bitstream of 3.9 Mbit
    \item Configuration clock (DCLK): 20 MHz
\end{itemize}

\textbf{Task:} Calculate minimum configuration time for the system.

\tcblower

\textbf{Solution:}

\textbf{Total bitstream size:}
$$\text{Total} = 2 \times 3.9 \text{ Mbit} = 7.8 \text{ Mbit}$$

\textbf{DCLK period:}
$$T_{DCLK} = \frac{1}{f_{DCLK}} = \frac{1}{20 \times 10^6 \text{ Hz}} = 50 \text{ ns}$$

\textbf{Configuration time formula:}
$$t_{config} = \text{Total bitstream size} \times \frac{T_{DCLK}}{1 \text{ bit}}$$

\textbf{Calculation:}
\begin{align*}
t_{config} &= 7.8 \times 10^6 \text{ bits} \times 50 \times 10^{-9} \text{ s/bit} \\
&= 390 \times 10^{-3} \text{ s} \\
&= 390 \text{ ms}
\end{align*}

\important{Answer:} Minimum configuration time is \textbf{390 milliseconds}.

\textbf{Additional considerations:}
\begin{itemize}
    \item This is theoretical minimum (ideal conditions)
    \item Actual time includes overhead:
    \begin{itemize}
        \item Power-up stabilization
        \item Flash memory access time
        \item Internal FPGA initialization
        \item Transition to user mode
    \end{itemize}
    \item Typical actual time: 400-450 ms
\end{itemize}
\end{example2}

% ===== IMAGE SUMMARY =====
% Total images needed: 28
% CRITICAL priority: 15
% IMPORTANT priority: 10
% SUPPLEMENTARY priority: 3
%
% Quick extraction checklist:
% [X] [SCD_13_config.pdf, Slide 2] - Motivation diagram (SUPPLEMENTARY)
% [ ] [SCD_13_config.pdf, Slide 6] - Logic block configuration (CRITICAL)
% [ ] [SCD_13_config.pdf, Slide 7] - Routing configuration (CRITICAL)
% [ ] [SCD_13_config.pdf, Slide 8] - I/O configuration (CRITICAL)
% [ ] [SCD_13_config.pdf, Slide 9] - PLL configuration (IMPORTANT)
% [ ] [SCD_13_config.pdf, Slide 10] - Memory configuration (IMPORTANT)
% [ ] [SCD_13_config.pdf, Slide 12] - SRAM cell diagram (CRITICAL)
% [ ] [SCD_13_config.pdf, Slide 13] - Antifuse structure (IMPORTANT)
% [ ] [SCD_13_config.pdf, Slide 14] - Antifuse detail (IMPORTANT)
% [ ] [SCD_13_config.pdf, Slide 15] - Flash cell (CRITICAL)
% [ ] [SCD_13_config.pdf, Slide 17] - Active serial overview (CRITICAL)
% [ ] [SCD_13_config.pdf, Slide 18] - Configuration controller (CRITICAL)
% [ ] [SCD_13_config.pdf, Slide 19] - Configuration timing (CRITICAL)
% [ ] [SCD_13_config.pdf, Slide 20] - Timing detail (IMPORTANT)
% [ ] [SCD_13_config.pdf, Slide 23] - SPI flash connection (CRITICAL)
% [ ] [SCD_13_config.pdf, Slide 24] - SPI read cycle (CRITICAL)
% [ ] [SCD_13_config.pdf, Slide 26] - Passive serial (IMPORTANT)
% [ ] [SCD_13_config.pdf, Slide 27] - Configuration chain (CRITICAL)
% [ ] [SCD_13_config.pdf, Slide 28] - Parallel configuration (SUPPLEMENTARY)
% [ ] [SCD_13_config.pdf, Slide 30] - JTAG configuration (IMPORTANT)
% [ ] [SCD_13_config.pdf, Slide 31] - Serial Flash Loader (CRITICAL)
% [ ] [SCD_13_config.pdf, Slide 32] - EPCS controller (IMPORTANT)
% [ ] [SCD_13_config.pdf, Slide 33] - EPCS development (CRITICAL)
% [ ] [SCD_13_config.pdf, Slide 35] - Volatile JTAG (SUPPLEMENTARY)
% [ ] [SCD_13_config.pdf, Slide 36] - Dev board USB (IMPORTANT)
% [ ] [SCD_13_config.pdf, Slide 37] - Byte Blaster (IMPORTANT)
% [ ] [SCD_13_config.pdf, Slide 38] - SPI programming (SUPPLEMENTARY)
% [X] [SCD_13_config.pdf, Slides 40-43] - Exercises (included in text)
% =====================