\section{PacMan Project}

\subsection{Introduction to the PacMan Game Architecture}

\begin{concept}{Project Overview}\\
The PacMan project demonstrates the integration of a processor (HPS) with programmable logic (FPGA) to create a classic arcade game. The project focuses on:
\begin{itemize}
    \item Goal definition and system architecture
    \item Partitioning tasks between processor and FPGA
    \item Understanding VGA graphics principles
    \item Character-based graphic generation in FPGA
\end{itemize}
\end{concept}

% TODO: Add image from SCD_4b_pacman_project.pdf, Page 1
% Description: PacMan Project title slide with introduction topics
% Priority: SUPPLEMENTARY
% Suggested filename: lecture04B_intro_title.png
\\
% \includegraphics[width=\linewidth]{lecture04B_intro_title.png}
\\
% WHEN YOU ADD IMAGE: Uncomment line above

\subsection{System Overview}

% TODO: Add image from SCD_4b_pacman_project.pdf, Page 2
% Description: Complete system architecture diagram showing HPS-FPGA integration
% Priority: CRITICAL
% Suggested filename: lecture04B_system_overview.png
\\
% \includegraphics[width=\linewidth]{lecture04B_system_overview.png}
\\
% WHEN YOU ADD IMAGE: Uncomment line above

\begin{definition}{System Architecture}\\
The PacMan system consists of several interconnected components:
\begin{itemize}
    \item Hard Processor System (HPS) running the game logic
    \item FPGA fabric handling graphics generation
    \item Video memory for frame buffer storage
    \item VGA controller for display output
\end{itemize}
\end{definition}

\subsection{Software and Programmable Logic Partitioning}

% TODO: Add image from SCD_4b_pacman_project.pdf, Page 3
% Description: Diagram showing pacman_top.vhd and graphic_controller.vhd architecture
% Priority: CRITICAL
% Suggested filename: lecture04B_sw_hw_partition.png
\\
% \includegraphics[width=\linewidth]{lecture04B_sw_hw_partition.png}
\\
% WHEN YOU ADD IMAGE: Uncomment line above

\begin{concept}{Hardware-Software Partitioning}\\
The system is divided into distinct hardware and software components:

\textbf{FPGA Components:}
\begin{itemize}
    \item \texttt{pacman\_top.vhd} - Top-level entity
    \item \texttt{graphic\_controller.vhd} - VGA timing and display control
    \item Video Memory - Frame buffer storage
    \item Clock management (50 MHz input clock)
\end{itemize}

\textbf{Software Components:}
\begin{itemize}
    \item Game logic running on HPS
    \item Character position management
    \item Collision detection
    \item Score tracking
\end{itemize}
\end{concept}

\raggedcolumns
\columnbreak

\subsection{VGA Graphics Principles}

\begin{definition}{VGA Resolution and Timing}\\
VGA (Video Graphics Array) is a display standard that defines how images are rendered on a screen through synchronized horizontal and vertical scanning.

\textbf{Key Parameters:}
\begin{itemize}
    \item Horizontal pixels per line: H pixels/line
    \item Vertical lines per frame: V lines/frame
    \item Each pixel is composed of Red, Green, and Blue (RGB) components
    \item Scanning occurs left-to-right, top-to-bottom
\end{itemize}
\end{definition}

% TODO: Add image from SCD_4b_pacman_project.pdf, Page 4
% Description: VGA graphics scanning pattern diagram showing H pixels and V lines
% Priority: CRITICAL
% Suggested filename: lecture04B_vga_graphics.png
\\
% \includegraphics[width=\linewidth]{lecture04B_vga_graphics.png}
\\
% WHEN YOU ADD IMAGE: Uncomment line above

\subsubsection{CRT Display Technology}

% TODO: Add image from SCD_4b_pacman_project.pdf, Page 5
% Description: TV tube cross-section showing cathode, deflection coil, shadow mask, and phosphor screen
% Priority: IMPORTANT
% Suggested filename: lecture04B_crt_tube.png
\\
% \includegraphics[width=\linewidth]{lecture04B_crt_tube.png}
\\
% WHEN YOU ADD IMAGE: Uncomment line above

\begin{concept}{CRT (Cathode Ray Tube) Operation}\\
Traditional CRT displays use electron beams to create images:

\textbf{Components:}
\begin{itemize}
    \item \textbf{Cathode:} Separate electron beams for Red, Green, and Blue
    \item \textbf{Deflection Coil (Yoke):} Magnetically steers beam in left-to-right, top-to-bottom pattern (separate H and V coils)
    \item \textbf{Shadow Mask:} Ensures R beam only illuminates R pixels, etc.
    \item \textbf{Phosphor Screen:} Emits light when excited by electron beam
    \item \textbf{Anode:} Accelerates electrons toward the screen
\end{itemize}

\textbf{Operation:}
The intensity of the electron beam determines the brightness of each pixel. The beam is continuously scanned across the screen in a raster pattern.
\end{concept}

\begin{remark}
Although modern displays use LCD or LED technology, understanding CRT operation helps explain VGA timing requirements that are still used today.
\end{remark}

\subsubsection{VGA Scanning Pattern}

% TODO: Add image from SCD_4b_pacman_project.pdf, Page 6
% Description: VGA scanning pattern showing 1024x768 pixels with horizontal and vertical return
% Priority: CRITICAL
% Suggested filename: lecture04B_vga_scan_pattern.png
\\
% \includegraphics[width=\linewidth]{lecture04B_vga_scan_pattern.png}
\\
% WHEN YOU ADD IMAGE: Uncomment line above

\begin{definition}{Raster Scanning}\\
VGA displays use a raster scanning method:
\begin{itemize}
    \item \textbf{Horizontal Scan:} Beam moves left-to-right across 1024 pixels
    \item \textbf{Horizontal Return:} Beam returns to left edge (blanking period)
    \item \textbf{Vertical Scan:} After 768 lines, beam returns to top (vertical return)
    \item \textbf{Frame Rate:} Complete cycle repeats continuously (typically 60 Hz)
\end{itemize}
\end{definition}

\raggedcolumns
\columnbreak

\subsection{VGA Timing Specifications}

\subsubsection{Horizontal Video Timing for 1024 $\times$ 768 Resolution}

% TODO: Add image from SCD_4b_pacman_project.pdf, Page 7
% Description: Horizontal timing diagram showing HSYNC, back porch, visible video, front porch
% Priority: CRITICAL
% Suggested filename: lecture04B_h_timing.png
\\
% \includegraphics[width=\linewidth]{lecture04B_h_timing.png}
\\
% WHEN YOU ADD IMAGE: Uncomment line above

\begin{highlight}{Horizontal Timing Parameters}\\
\begin{center}
\begin{tabular}{|l|c|c|}
\hline
\textbf{Parameter} & \textbf{Time} & \textbf{Pixel Clocks} \\
\hline
Pixel Clock & 15.39 ns & 65 MHz \\
HSYNC Pulse Width ($T_{pw}$) & 2.092 $\mu$s & 136 pixclk \\
Back Porch ($T_{bp}$) & 369.2 ns & 24 pixclk \\
Visible Video ($T_{disp}$) & 15.754 $\mu$s & 1024 pixclk \\
Front Porch ($T_{fp}$) & 2.462 $\mu$s & 160 pixclk \\
\hline
\textbf{Total Line Time} & \textbf{20.66 $\mu$s} & \textbf{1344 pixclk} \\
\hline
\end{tabular}
\end{center}
\end{highlight}

\begin{concept}{Horizontal Sync Signal}\\
The HSYNC signal controls horizontal beam positioning:
\begin{itemize}
    \item \textbf{HSYNC Pulse:} Triggers horizontal retrace (136 clocks)
    \item \textbf{Back Porch:} Blanking after HSYNC before visible video (24 clocks)
    \item \textbf{Visible Video:} Active pixel data display (1024 clocks)
    \item \textbf{Front Porch:} Blanking before next HSYNC (160 clocks)
\end{itemize}
\end{concept}

\subsubsection{Vertical Video Timing for 1024 $\times$ 768 Resolution}

% TODO: Add image from SCD_4b_pacman_project.pdf, Page 8
% Description: Vertical timing diagram showing VSYNC and line counts
% Priority: CRITICAL
% Suggested filename: lecture04B_v_timing.png
\\
% \includegraphics[width=\linewidth]{lecture04B_v_timing.png}
\\
% WHEN YOU ADD IMAGE: Uncomment line above

\begin{highlight}{Vertical Timing Parameters}\\
\begin{center}
\begin{tabular}{|l|c|c|}
\hline
\textbf{Parameter} & \textbf{Time} & \textbf{Lines} \\
\hline
One Line Period & 20.66 $\mu$s & 1 HSYNC \\
VSYNC Pulse Width & 124.0 $\mu$s & 6 lines \\
Back Porch & 599.4 $\mu$s & 29 lines \\
Visible Video & 15.86 ms & 768 lines \\
Front Porch & 61.97 $\mu$s & 3 lines \\
\hline
\textbf{Total Frame Time} & \textbf{16.66 ms} & \textbf{806 lines} \\
\hline
\end{tabular}
\end{center}
\end{highlight}

\begin{concept}{Vertical Sync Signal}\\
The VSYNC signal controls vertical beam positioning:
\begin{itemize}
    \item \textbf{VSYNC Pulse:} Triggers vertical retrace (6 lines)
    \item \textbf{Back Porch:} Blanking after VSYNC (29 lines)
    \item \textbf{Visible Video:} Active lines containing pixel data (768 lines)
    \item \textbf{Front Porch:} Blanking before next VSYNC (3 lines)
    \item \textbf{Frame Rate:} $\frac{1}{16.66 \text{ ms}} \approx 60$ Hz
\end{itemize}
\end{concept}

\raggedcolumns
\columnbreak

\subsection{Character-Based Graphics in FPGA}

% TODO: Add image from SCD_4b_pacman_project.pdf, Page 9
% Description: Character-based graphics architecture diagram
% Priority: CRITICAL
% Suggested filename: lecture04B_char_graphics.png
\\
% \includegraphics[width=\linewidth]{lecture04B_char_graphics.png}
\\
% WHEN YOU ADD IMAGE: Uncomment line above

\begin{concept}{Character-Based Display System}\\
The PacMan project uses character-based graphics for efficient FPGA implementation:

\textbf{Advantages:}
\begin{itemize}
    \item Reduced memory requirements
    \item Simplified graphics controller logic
    \item Easy character animation
    \item Efficient for tile-based games
\end{itemize}

\textbf{System Components:}
\begin{itemize}
    \item Character ROM - stores character patterns
    \item Video Memory - stores which character appears at each position
    \item Display Controller - reads memory and generates video signal
\end{itemize}
\end{concept}

% TODO: Add image from SCD_4b_pacman_project.pdf, Page 10
% Description: Character memory organization and addressing
% Priority: CRITICAL
% Suggested filename: lecture04B_char_memory.png
\\
% \includegraphics[width=\linewidth]{lecture04B_char_memory.png}
\\
% WHEN YOU ADD IMAGE: Uncomment line above

% TODO: Add image from SCD_4b_pacman_project.pdf, Page 11
% Description: Character pattern ROM structure
% Priority: IMPORTANT
% Suggested filename: lecture04B_char_rom.png
\\
% \includegraphics[width=\linewidth]{lecture04B_char_rom.png}
\\
% WHEN YOU ADD IMAGE: Uncomment line above

% TODO: Add image from SCD_4b_pacman_project.pdf, Page 12
% Description: VGA controller block diagram
% Priority: CRITICAL
% Suggested filename: lecture04B_vga_controller.png
\\
% \includegraphics[width=\linewidth]{lecture04B_vga_controller.png}
\\
% WHEN YOU ADD IMAGE: Uncomment line above

% TODO: Add image from SCD_4b_pacman_project.pdf, Page 13
% Description: Timing generation and synchronization
% Priority: IMPORTANT
% Suggested filename: lecture04B_timing_gen.png
\\
% \includegraphics[width=\linewidth]{lecture04B_timing_gen.png}
\\
% WHEN YOU ADD IMAGE: Uncomment line above

% TODO: Add image from SCD_4b_pacman_project.pdf, Page 14
% Description: Graphics pipeline dataflow
% Priority: IMPORTANT
% Suggested filename: lecture04B_graphics_pipeline.png
\\
% \includegraphics[width=\linewidth]{lecture04B_graphics_pipeline.png}
\\
% WHEN YOU ADD IMAGE: Uncomment line above

% TODO: Add image from SCD_4b_pacman_project.pdf, Page 15
% Description: Memory interface and addressing
% Priority: IMPORTANT
% Suggested filename: lecture04B_memory_interface.png
\\
% \includegraphics[width=\linewidth]{lecture04B_memory_interface.png}
\\
% WHEN YOU ADD IMAGE: Uncomment line above

% TODO: Add image from SCD_4b_pacman_project.pdf, Page 16
% Description: Color palette and RGB generation
% Priority: IMPORTANT
% Suggested filename: lecture04B_color_palette.png
\\
% \includegraphics[width=\linewidth]{lecture04B_color_palette.png}
\\
% WHEN YOU ADD IMAGE: Uncomment line above

% TODO: Add image from SCD_4b_pacman_project.pdf, Page 17
% Description: Character animation and updates
% Priority: IMPORTANT
% Suggested filename: lecture04B_char_animation.png
\\
% \includegraphics[width=\linewidth]{lecture04B_char_animation.png}
\\
% WHEN YOU ADD IMAGE: Uncomment line above

% TODO: Add image from SCD_4b_pacman_project.pdf, Page 18
% Description: Game logic integration with graphics
% Priority: IMPORTANT
% Suggested filename: lecture04B_game_logic.png
\\
% \includegraphics[width=\linewidth]{lecture04B_game_logic.png}
\\
% WHEN YOU ADD IMAGE: Uncomment line above

% TODO: Add image from SCD_4b_pacman_project.pdf, Page 19
% Description: HPS-FPGA communication interface
% Priority: IMPORTANT
% Suggested filename: lecture04B_hps_fpga_comm.png
\\
% \includegraphics[width=\linewidth]{lecture04B_hps_fpga_comm.png}
\\
% WHEN YOU ADD IMAGE: Uncomment line above

% TODO: Add image from SCD_4b_pacman_project.pdf, Page 20
% Description: Project implementation details
% Priority: SUPPLEMENTARY
% Suggested filename: lecture04B_implementation.png
\\
% \includegraphics[width=\linewidth]{lecture04B_implementation.png}
\\
% WHEN YOU ADD IMAGE: Uncomment line above

% TODO: Add image from SCD_4b_pacman_project.pdf, Page 21
% Description: Summary and conclusions
% Priority: SUPPLEMENTARY
% Suggested filename: lecture04B_summary.png
\\
% \includegraphics[width=\linewidth]{lecture04B_summary.png}
\\
% WHEN YOU ADD IMAGE: Uncomment line above

\begin{remark}
The PacMan project demonstrates practical application of VGA timing principles, character-based graphics, and hardware-software partitioning in an FPGA-based system. The modular architecture allows for efficient implementation and easy modification of game elements.
\end{remark}

% ===== IMAGE SUMMARY =====
% Total images needed: 21
% CRITICAL priority: 10
% IMPORTANT priority: 9
% SUPPLEMENTARY priority: 2
%
% Quick extraction checklist:
% [ ] [SCD_4b_pacman_project.pdf, Page 1] - PacMan Project title slide (SUPPLEMENTARY)
% [ ] [SCD_4b_pacman_project.pdf, Page 2] - System overview diagram (CRITICAL)
% [ ] [SCD_4b_pacman_project.pdf, Page 3] - SW/HW partitioning diagram (CRITICAL)
% [ ] [SCD_4b_pacman_project.pdf, Page 4] - VGA graphics principle (CRITICAL)
% [ ] [SCD_4b_pacman_project.pdf, Page 5] - CRT tube structure (IMPORTANT)
% [ ] [SCD_4b_pacman_project.pdf, Page 6] - VGA scanning pattern (CRITICAL)
% [ ] [SCD_4b_pacman_project.pdf, Page 7] - Horizontal timing diagram (CRITICAL)
% [ ] [SCD_4b_pacman_project.pdf, Page 8] - Vertical timing diagram (CRITICAL)
% [ ] [SCD_4b_pacman_project.pdf, Page 9] - Character-based graphics (CRITICAL)
% [ ] [SCD_4b_pacman_project.pdf, Page 10] - Character memory organization (CRITICAL)
% [ ] [SCD_4b_pacman_project.pdf, Page 11] - Character pattern ROM (IMPORTANT)
% [ ] [SCD_4b_pacman_project.pdf, Page 12] - VGA controller block diagram (CRITICAL)
% [ ] [SCD_4b_pacman_project.pdf, Page 13] - Timing generation (IMPORTANT)
% [ ] [SCD_4b_pacman_project.pdf, Page 14] - Graphics pipeline (IMPORTANT)
% [ ] [SCD_4b_pacman_project.pdf, Page 15] - Memory interface (IMPORTANT)
% [ ] [SCD_4b_pacman_project.pdf, Page 16] - Color palette (IMPORTANT)
% [ ] [SCD_4b_pacman_project.pdf, Page 17] - Character animation (IMPORTANT)
% [ ] [SCD_4b_pacman_project.pdf, Page 18] - Game logic integration (IMPORTANT)
% [ ] [SCD_4b_pacman_project.pdf, Page 19] - HPS-FPGA communication (IMPORTANT)
% [ ] [SCD_4b_pacman_project.pdf, Page 20] - Implementation details (SUPPLEMENTARY)
% [ ] [SCD_4b_pacman_project.pdf, Page 21] - Summary (SUPPLEMENTARY)
% =====================