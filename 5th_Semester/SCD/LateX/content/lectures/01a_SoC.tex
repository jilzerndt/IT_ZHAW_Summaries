\section{Introduction to System on Chip Design}

\subsection{Course Information}

\begin{remark}
\textbf{Course:} System on Chip Design (SCD)

\textbf{Institution:} Zürcher Hochschule für Angewandte Wissenschaften (ZHAW)

\textbf{Instructors:}
\begin{itemize}
    \item Tobias Welti (welo@zhaw.ch, +41 58 934 67 30)
    \item Dominique Cachin (cacd@zhaw.ch, +41 79 45559 01)
\end{itemize}
\end{remark}

\begin{definition}{Course Materials and Schedule}
\begin{itemize}
    \item \textbf{Platform:} Moodle - https://moodle.zhaw.ch/course/view.php?id=25948
    \item \textbf{Script:} Available on Moodle (scd\_script.pdf)
    \item \textbf{Lab Instructions:} https://github.zhaw.ch/pages/hpmm/scd-labs/index.html
\end{itemize}
\end{definition}

% TODO: Add image from SCD_1a_what_is_SoC.pdf, Page 4
% Description: Course schedule table showing lecture topics and lab assignments
% Priority: IMPORTANT
% Suggested filename: lecture01a_course_schedule.png
\\
% \includegraphics[width=\linewidth]{lecture01a_course_schedule.png}
\\

\subsection{Assessment and Grading}

\begin{definition}{Grading Components}
\begin{itemize}
    \item \textbf{Electronic Quiz:} 15\% (November 11, 2025, Moodle test)
    \item \textbf{Lab Exercises:} 15\% (6 labs during semester, graded by lecturer)
    \item \textbf{Written Exam:} 70\% (January 2026, Moodle test)
\end{itemize}
\end{definition}

\begin{definition}{Lab Grading System}\\
Seven labs (four lessons each) contribute to the lab grade.

\textbf{Credits per Lab:}
\begin{itemize}
    \item Not done: 0 points
    \item Required tasks done with small errors: 1 point
    \item Required tasks done without errors: 2 points
\end{itemize}

\textbf{Lab Grade Formula:}
$$\text{Lab Grade} = \frac{\text{Sum of Points}}{12} \times 5 + 1$$
\end{definition}

\begin{remark}
\textbf{Exam Guidelines:}
\begin{itemize}
    \item Open book: lecture and lab notes, personal notes, books allowed
    \item No generative AI such as ChatGPT
    \item Calculators allowed
\end{itemize}
\end{remark}

\subsection{Course Objectives}

\begin{concept}{Target Audience}\\
This course is designed for engineers who want to:
\begin{itemize}
    \item Design high-performance digital circuits with SoC-FPGAs, beyond writing VHDL code
    \item Gain in-depth background knowledge of SoC and FPGA (for software engineers)
    \item Design systems with Linux on SoC-FPGA
    \item Obtain introduction and basic knowledge of Integrated Circuit design
    \item Design general high-speed digital systems with complex peripherals (DDRAM)
\end{itemize}
\end{concept}

\begin{concept}{Learning Goals}\\
By the end of this course, students will be able to:
\begin{itemize}
    \item Work with FPGA block memory
    \item Configure a FPGA-SoC with ARM hardcore processor
    \item Configure the I/O and computer peripherals (DRAM) of a FPGA
    \item Port Yocto Linux to SoC-FPGA
    \item Configure and analyse timing to drive synthesis
    \item Configure clock generators in FPGA and route clocks on PCB
    \item Check signal integrity of clock and data lines on PCB
    \item Explain differences between different signaling standards
    \item Connect high-speed FPGA peripherals with differential signals
    \item Realize a project with video and audio output (Pacman game)
\end{itemize}
\end{concept}

\subsection{Lab Setup}

\begin{definition}{Laboratory Environment}\\
\textbf{Location:} Lab TE 519

\textbf{Equipment:}
\begin{itemize}
    \item Lab PCs with required software setup running on Linux
    \item DE1-SoC Development Board with Intel Cyclone V SoC FPGA
    \item Hardware only available in lab (not distributed to students)
\end{itemize}

\textbf{Work Organization:}
\begin{itemize}
    \item Students work in teams of two
    \item Lab instructions available on GitHub
\end{itemize}
\end{definition}

\raggedcolumns
\columnbreak

\subsection{Fundamentals of Sequential Digital Circuits}

\begin{concept}{General Representation of Sequential Circuits}\\
Sequential circuits consist of two main components:
\begin{itemize}
    \item \textbf{D-Flip-Flops:} Storage elements that hold state
    \item \textbf{Combinatorial Logic:} Logic circuits that compute next state and outputs
    \item \textbf{Common Clock:} Synchronizes all flip-flops
\end{itemize}
\end{concept}

% TODO: Add image from SCD_1a_what_is_SoC.pdf, Page 10
% Description: Block diagram showing sequential circuit with D-flip-flops and combinatorial logic
% Priority: CRITICAL
% Suggested filename: lecture01a_sequential_circuit_diagram.png
\\
% \includegraphics[width=\linewidth]{lecture01a_sequential_circuit_diagram.png}
\\

\subsection{FPGA Architecture}

\begin{definition}{SRAM-based FPGAs}\\
SRAM-based FPGAs use \textbf{Lookup Tables (LUTs)} to generate logic functions.

\textbf{Key Concept:} A LUT is essentially a small RAM (e.g., 4x1 RAM) that can implement any Boolean function of its inputs.
\end{definition}

% TODO: Add image from SCD_1a_what_is_SoC.pdf, Page 11
% Description: Diagram showing 4x1 RAM as lookup table
% Priority: CRITICAL
% Suggested filename: lecture01a_lut_structure.png
\\
% \includegraphics[width=\linewidth]{lecture01a_lut_structure.png}
\\

\begin{definition}{Logic Cells (ALM)}\\
\textbf{ALM:} Adaptive Logic Module (Altera/Intel terminology)

Different FPGA manufacturers use different names:
\begin{itemize}
    \item \textbf{Altera (Intel):} Adaptive Logic Module (ALM)
    \item \textbf{Xilinx:} Configurable Logic Block (CLB) or Slice
\end{itemize}

Each logic cell typically contains:
\begin{itemize}
    \item Lookup tables (LUTs)
    \item Flip-flops
    \item Multiplexers
    \item Carry logic
\end{itemize}
\end{definition}

% TODO: Add image from SCD_1a_what_is_SoC.pdf, Page 12
% Description: Comparison of logic cell structures from Altera and Xilinx
% Priority: IMPORTANT
% Suggested filename: lecture01a_logic_cells_comparison.png
\\
% \includegraphics[width=\linewidth]{lecture01a_logic_cells_comparison.png}
\\

\begin{concept}{Classic FPGA Architecture}\\
A traditional FPGA consists of:
\begin{itemize}
    \item \textbf{FPGA Fabric:} Array of configurable logic cells
    \item \textbf{Interconnection Network:} Programmable routing between logic cells
    \item \textbf{Logic Cells:} Containing lookup tables and flip-flops
\end{itemize}
\end{concept}

% TODO: Add image from SCD_1a_what_is_SoC.pdf, Page 13
% Description: Classic FPGA fabric architecture showing interconnections
% Priority: CRITICAL
% Suggested filename: lecture01a_classic_fpga_architecture.png
\\
% \includegraphics[width=\linewidth]{lecture01a_classic_fpga_architecture.png}
\\

\begin{definition}{Additional FPGA Resources}\\
Modern FPGA fabric may also contain:
\begin{itemize}
    \item \textbf{SRAM Blocks:} Embedded memory blocks for data storage
    \item \textbf{DSP Blocks:} Dedicated hardware for digital signal processing
    \item \textbf{Hard IP Blocks:} Pre-designed functional blocks (e.g., PCIe, memory controllers)
\end{itemize}
\end{definition}

% TODO: Add image from SCD_1a_what_is_SoC.pdf, Page 14
% Description: FPGA fabric showing SRAM and DSP blocks
% Priority: IMPORTANT
% Suggested filename: lecture01a_fpga_additional_blocks.png
\\
% \includegraphics[width=\linewidth]{lecture01a_fpga_additional_blocks.png}
\\

\raggedcolumns
\columnbreak

\subsection{FPGA Applications}

\begin{example}
\textbf{Home Electronics:} FPGAs are used in consumer devices for video processing, display controllers, and smart home automation.
\end{example}

% TODO: Add image from SCD_1a_what_is_SoC.pdf, Page 16
% Description: Home electronics applications
% Priority: SUPPLEMENTARY
% Suggested filename: lecture01a_home_electronics.png
\\
% \includegraphics[width=\linewidth]{lecture01a_home_electronics.png}
\\

\begin{example}
\textbf{Medical Diagnostics:} FPGAs provide real-time signal processing for medical imaging equipment and diagnostic devices.
\end{example}

% TODO: Add image from SCD_1a_what_is_SoC.pdf, Page 17
% Description: Medical diagnostics equipment
% Priority: SUPPLEMENTARY
% Suggested filename: lecture01a_medical_diagnostics.png
\\
% \includegraphics[width=\linewidth]{lecture01a_medical_diagnostics.png}
\\

\begin{example}
\textbf{Industrial Automation:} FPGAs enable precise control and monitoring in industrial systems.
\end{example}

% TODO: Add image from SCD_1a_what_is_SoC.pdf, Page 18
% Description: Industrial automation applications
% Priority: SUPPLEMENTARY
% Suggested filename: lecture01a_automation.png
\\
% \includegraphics[width=\linewidth]{lecture01a_automation.png}
\\

\begin{example}
\textbf{Studio Equipment:} Professional audio and video equipment uses FPGAs for real-time processing and effects.
\end{example}

% TODO: Add image from SCD_1a_what_is_SoC.pdf, Page 19
% Description: Professional studio equipment
% Priority: SUPPLEMENTARY
% Suggested filename: lecture01a_studio_equipment.png
\\
% \includegraphics[width=\linewidth]{lecture01a_studio_equipment.png}
\\

\begin{remark}
\textbf{Products where FPGAs are NOT ideal:}
\begin{itemize}
    \item Battery-operated mobile devices (high power consumption)
    \item Very high-volume products like mobile phones (cost per unit too high)
    \item Trivial control applications (washing machine, bike computer - overkill)
    \item Automotive applications (reliability and certification requirements)
    \item Some aerospace applications (radiation hardness requirements)
\end{itemize}
\end{remark}

\subsection{Evolution from FPGA to SoC}

\begin{concept}{System on Chip (SoC) FPGA}\\
Modern SoC FPGAs combine:
\begin{itemize}
    \item \textbf{Hard Processor System (HPS):} Fixed CPU subsystem
    \item \textbf{FPGA Fabric:} Programmable logic
    \item \textbf{Bridges:} Connections between processor and FPGA
\end{itemize}

This integration enables software and hardware co-design on a single chip.
\end{concept}

% TODO: Add image from SCD_1a_what_is_SoC.pdf, Page 22
% Description: SoC FPGA architecture showing CPU portion and FPGA portion with Chip Planner view
% Priority: CRITICAL
% Suggested filename: lecture01a_soc_fpga_architecture.png
\\
% \includegraphics[width=\linewidth]{lecture01a_soc_fpga_architecture.png}
\\

\begin{definition}{SoC FPGA Components}\\
\textbf{CPU Portion (Hard Processor System):}
\begin{itemize}
    \item Cortex-A9 CPU Subsystem
    \item Flash Controllers
    \item SDRAM Controller Subsystem
    \item On-Chip Memories
    \item Support Peripherals
    \item PLLs (Phase-Locked Loops)
    \item Debug Interface
    \item Peripherals Control Block
\end{itemize}

\textbf{FPGA Portion:}
\begin{itemize}
    \item User I/O
    \item FPGA Fabric (LUTs, RAMs, Multipliers \& Routing)
    \item HSSI Transceivers (High-Speed Serial Interface)
    \item PLLs
    \item Hard PCIe
    \item Hard Memory Controllers
\end{itemize}

\textbf{Processor-FPGA Bridges:} Enable communication between CPU and FPGA fabric
\end{definition}

\subsection{Topics Covered in SCD Course}

\begin{highlight}{Course Topics Overview}\\
The following topics will be discussed throughout the course:
\begin{enumerate}
    \item Bringing Linux to SoC
    \item FPGA System Builder (Platform Designer)
    \item Bootloader for Linux
    \item RAM and FPGA on-chip Memory
    \item DRAM (Dynamic Random Access Memory)
    \item JTAG (Joint Test Action Group)
    \item Timing Analysis and Synthesis
    \item Clock Distribution and PLL
    \item High-Speed I/O Interfaces
    \item FPGA Configuration
\end{enumerate}
\end{highlight}

% TODO: Add image from SCD_1a_what_is_SoC.pdf, Page 21
% Description: Overview slide showing all subjects discussed in SCD
% Priority: IMPORTANT
% Suggested filename: lecture01a_subjects_overview.png
\\
% \includegraphics[width=\linewidth]{lecture01a_subjects_overview.png}
\\

\raggedcolumns
\columnbreak

\subsection{Bringing Linux to SoC-FPGA}

\begin{concept}{Linux on SoC-FPGA}\\
Running Linux on SoC-FPGA enables:
\begin{itemize}
    \item High-level software development using familiar tools
    \item Access to vast ecosystem of Linux software
    \item Network connectivity and modern protocols
    \item File systems and standard I/O
    \item Dynamic hardware reconfiguration from software
\end{itemize}
\end{concept}

% TODO: Add image from SCD_1a_what_is_SoC.pdf, Page 23
% Description: Linux on SoC concept illustration
% Priority: IMPORTANT
% Suggested filename: lecture01a_linux_on_soc.png
\\
% \includegraphics[width=\linewidth]{lecture01a_linux_on_soc.png}
\\

\subsection{FPGA System Builder}

\begin{definition}{Platform Designer}\\
\textbf{Platform Designer} (formerly known as Qsys) is Intel's system integration tool for SoC FPGAs.

\textbf{Functions:}
\begin{itemize}
    \item Graphical system design interface
    \item Integration of IP cores
    \item Automatic interconnect generation
    \item System-level simulation
    \item Export to Quartus for compilation
\end{itemize}
\end{definition}

% TODO: Add image from SCD_1a_what_is_SoC.pdf, Page 24
% Description: Platform Designer system description and Chip Planner in Quartus
% Priority: CRITICAL
% Suggested filename: lecture01a_platform_designer.png
\\
% \includegraphics[width=\linewidth]{lecture01a_platform_designer.png}
\\

\subsection{Bootloader for Linux}

\begin{concept}{Boot Process}\\
The bootloader is responsible for:
\begin{itemize}
    \item Initializing hardware (CPU, memory, peripherals)
    \item Loading the Linux kernel into memory
    \item Passing boot parameters to the kernel
    \item Transferring control to the operating system
\end{itemize}

\textbf{Common bootloaders:} U-Boot, GRUB
\end{concept}

% TODO: Add image from SCD_1a_what_is_SoC.pdf, Page 25
% Description: Bootloader process diagram
% Priority: IMPORTANT
% Suggested filename: lecture01a_bootloader.png
\\
% \includegraphics[width=\linewidth]{lecture01a_bootloader.png}
\\

\subsection{Memory Systems}

\begin{definition}{RAM and FPGA On-Chip Memory}\\
\textbf{On-Chip Memory Types:}
\begin{itemize}
    \item \textbf{Block RAM (BRAM):} Embedded memory blocks in FPGA fabric
    \item \textbf{Distributed RAM:} Memory implemented using LUTs
    \item \textbf{On-Chip RAM in HPS:} Fast memory integrated in processor system
\end{itemize}

\textbf{Characteristics:}
\begin{itemize}
    \item Very fast access (single cycle or few cycles)
    \item Limited size (kilobytes to few megabytes)
    \item No refresh needed (SRAM-based)
    \item Higher cost per bit than external memory
\end{itemize}
\end{definition}

% TODO: Add image from SCD_1a_what_is_SoC.pdf, Page 26
% Description: RAM and FPGA on-chip memory architecture
% Priority: CRITICAL
% Suggested filename: lecture01a_onchip_memory.png
\\
% \includegraphics[width=\linewidth]{lecture01a_onchip_memory.png}
\\

\begin{definition}{Dynamic RAM (DRAM)}\\
\textbf{Inventor:} Robert H. Dennard (invented at IBM in 1967)

\textbf{Key Characteristics:}
\begin{itemize}
    \item \textbf{Storage Mechanism:} Stores data as charge in capacitors
    \item \textbf{Refresh Required:} Must be periodically refreshed (capacitors leak)
    \item \textbf{High Density:} Much higher storage density than SRAM
    \item \textbf{Lower Cost:} Cheaper per bit than SRAM
    \item \textbf{Slower Access:} Requires complex timing protocols
\end{itemize}

\textbf{Common Types:}
\begin{itemize}
    \item DDR3, DDR4, DDR5: Double Data Rate SDRAM
    \item LPDDR: Low Power DDR for mobile devices
\end{itemize}
\end{definition}

% TODO: Add image from SCD_1a_what_is_SoC.pdf, Page 27
% Description: DRAM technology and Robert H. Dennard
% Priority: SUPPLEMENTARY
% Suggested filename: lecture01a_dram.png
\\
% \includegraphics[width=\linewidth]{lecture01a_dram.png}
\\

\raggedcolumns
\columnbreak

\subsection{JTAG (Joint Test Action Group)}

\begin{definition}{JTAG Interface}\\
\textbf{JTAG} is a standard for testing and debugging electronic systems.

\textbf{Primary Uses:}
\begin{itemize}
    \item Boundary scan testing of PCBs
    \item Programming FPGAs and flash memory
    \item Debugging processors (via debug interface)
    \item In-system programming
\end{itemize}

\textbf{Signal Lines:}
\begin{itemize}
    \item \textbf{TDI:} Test Data In
    \item \textbf{TDO:} Test Data Out
    \item \textbf{TCK:} Test Clock
    \item \textbf{TMS:} Test Mode Select
    \item \textbf{TRST:} Test Reset (optional)
\end{itemize}
\end{definition}

\begin{definition}{JTAG State Machine}\\
The JTAG interface operates through a state machine with the following main states:
\begin{itemize}
    \item \textbf{Test-Logic-Reset:} Initial state
    \item \textbf{Run-Test/Idle:} Idle state
    \item \textbf{Select-DR-Scan:} Select data register scan
    \item \textbf{Select-IR-Scan:} Select instruction register scan
    \item \textbf{Capture-DR/IR:} Capture data or instruction
    \item \textbf{Shift-DR/IR:} Shift data or instruction
    \item \textbf{Update-DR/IR:} Update data or instruction register
\end{itemize}
\end{definition}

% TODO: Add image from SCD_1a_what_is_SoC.pdf, Page 28
% Description: JTAG state machine diagram
% Priority: CRITICAL
% Suggested filename: lecture01a_jtag_state_machine.png
\\
% \includegraphics[width=\linewidth]{lecture01a_jtag_state_machine.png}
\\

\subsection{Timing Analysis and Synthesis}

\begin{concept}{Static Timing Analysis}\\
Static timing analysis verifies that a digital circuit meets all timing constraints without requiring simulation of the circuit's operation.

\textbf{Key Concepts:}
\begin{itemize}
    \item \textbf{Setup Time ($t_{su}$):} Minimum time data must be stable before clock edge
    \item \textbf{Hold Time ($t_h$):} Minimum time data must remain stable after clock edge
    \item \textbf{Clock-to-Q Delay ($t_{co}$):} Time for data to appear at flip-flop output after clock
    \item \textbf{Propagation Delay:} Time for signals to travel through logic and routing
\end{itemize}
\end{concept}

\begin{definition}{Timing Paths and Slack}\\
\textbf{Data Arrival Time} is composed of:
\begin{enumerate}
    \item Clock network delay from clock node to launch D-FF: 3.00 ns
    \item Clock-to-Q delay from launch D-FF: 2.00 ns
    \item Logic delay (cells, I/O pins, routing): 5.00 ns
\end{enumerate}
\textbf{Total Data Arrival Time:} 10.00 ns

\textbf{Data Required Time} is determined by:
\begin{enumerate}
    \item Clock period: 40.00 ns
    \item Clock network delay to latch D-FF: 2.00 ns
    \item Setup time of latch D-FF: -0.50 ns
\end{enumerate}
\textbf{Total Required Time:} 41.50 ns

\textbf{Setup Slack} = Required Time - Arrival Time = 41.50 ns - 10.00 ns = 31.50 ns

\important{Positive slack indicates timing is met. Negative slack indicates timing violation.}
\end{definition}

% TODO: Add image from SCD_1a_what_is_SoC.pdf, Page 29
% Description: Timing analysis diagram showing arrival and required times
% Priority: CRITICAL
% Suggested filename: lecture01a_timing_analysis.png
\\
% \includegraphics[width=\linewidth]{lecture01a_timing_analysis.png}
\\

\subsection{Clock Distribution and PLL}

\begin{definition}{Phase-Locked Loop (PLL)}\\
A \textbf{PLL} is a control system that generates an output signal whose phase is related to the phase of an input signal.

\textbf{Key Functions:}
\begin{itemize}
    \item \textbf{Frequency Multiplication:} Generate higher frequency clocks
    \item \textbf{Frequency Division:} Generate lower frequency clocks
    \item \textbf{Phase Alignment:} Align clock phases
    \item \textbf{Jitter Reduction:} Clean up noisy clock signals
    \item \textbf{Clock Deskew:} Compensate for clock distribution delays
\end{itemize}
\end{definition}

\begin{concept}{Clock Distribution Network}\\
FPGAs have dedicated clock distribution networks:
\begin{itemize}
    \item \textbf{Global Clock Networks (GCLK):} Low-skew distribution across entire device
    \item \textbf{Regional Clock Networks:} Lower skew within regions
    \item \textbf{Dual-Purpose Clock Networks (DPCLK):} Flexible routing
    \item \textbf{Clock Control Blocks (CDPCLK):} Gating and multiplexing
\end{itemize}

\important{Proper clock distribution is critical for achieving timing closure in high-speed designs.}
\end{concept}

% TODO: Add image from SCD_1a_what_is_SoC.pdf, Page 30
% Description: Clock distribution network diagram showing GCLK, DPCLK, and CDPCLK
% Priority: CRITICAL
% Suggested filename: lecture01a_clock_distribution.png
\\
% \includegraphics[width=\linewidth]{lecture01a_clock_distribution.png}
\\

\subsection{High-Speed I/O Interfaces}

\begin{concept}{Differential Signaling}\\
High-speed interfaces use \textbf{differential signaling} where information is transmitted using two complementary signals.

\textbf{Advantages:}
\begin{itemize}
    \item Better noise immunity (common-mode noise rejection)
    \item Higher data rates possible
    \item Lower EMI emissions
    \item Better signal integrity over long distances
\end{itemize}

\textbf{Common Standards:}
\begin{itemize}
    \item LVDS (Low-Voltage Differential Signaling)
    \item TMDS (Transition-Minimized Differential Signaling)
    \item SerDes (Serializer/Deserializer)
\end{itemize}
\end{concept}

% TODO: Add image from SCD_1a_what_is_SoC.pdf, Page 31
% Description: High-speed I/O interface with output buffer
% Priority: IMPORTANT
% Suggested filename: lecture01a_highspeed_io.png
\\
% \includegraphics[width=\linewidth]{lecture01a_highspeed_io.png}
\\

\subsection{FPGA Configuration}

\begin{definition}{FPGA Configuration Process}\\
FPGAs are volatile devices that must be configured at power-up.

\textbf{Configuration Sources:}
\begin{itemize}
    \item \textbf{Configuration Flash:} Non-volatile memory storing bitstream
    \item \textbf{JTAG:} Programming via debugger (Byte Blaster)
    \item \textbf{Serial Configuration:} Via SPI or similar interface
    \item \textbf{Parallel Configuration:} Faster configuration method
\end{itemize}

\textbf{Configuration Modes:}
\begin{itemize}
    \item \textbf{Active Serial (AS):} FPGA controls configuration process
    \item \textbf{Passive Serial (PS):} External controller drives configuration
    \item \textbf{JTAG Mode:} For development and debugging
\end{itemize}
\end{definition}

% TODO: Add image from SCD_1a_what_is_SoC.pdf, Page 32
% Description: FPGA configuration circuit with flash memory and JTAG
% Priority: CRITICAL
% Suggested filename: lecture01a_fpga_configuration.png
\\
% \includegraphics[width=\linewidth]{lecture01a_fpga_configuration.png}
\\

\begin{remark}
\textbf{Configuration Pins:}
\begin{itemize}
    \item \textbf{nCE:} Active-low chip enable
    \item \textbf{nCONFIG:} Configuration control
    \item \textbf{CONF\_DONE:} Configuration complete signal
    \item \textbf{DATA:} Configuration data lines
\end{itemize}

Typical pull-up/pull-down resistors: 10k$\Omega$
\end{remark}

% ===== IMAGE SUMMARY =====
% Total images needed: 21
% CRITICAL priority: 10
% IMPORTANT priority: 7
% SUPPLEMENTARY priority: 4
%
% Quick extraction checklist:
% [ ] [SCD_1a_what_is_SoC.pdf, Page 4] - Course schedule table (IMPORTANT)
% [ ] [SCD_1a_what_is_SoC.pdf, Page 10] - Sequential circuit block diagram (CRITICAL)
% [ ] [SCD_1a_what_is_SoC.pdf, Page 11] - LUT structure with 4x1 RAM (CRITICAL)
% [ ] [SCD_1a_what_is_SoC.pdf, Page 12] - Logic cells comparison Altera vs Xilinx (IMPORTANT)
% [ ] [SCD_1a_what_is_SoC.pdf, Page 13] - Classic FPGA architecture (CRITICAL)
% [ ] [SCD_1a_what_is_SoC.pdf, Page 14] - FPGA fabric with SRAM and DSP blocks (IMPORTANT)
% [ ] [SCD_1a_what_is_SoC.pdf, Page 16] - Home electronics applications (SUPPLEMENTARY)
% [ ] [SCD_1a_what_is_SoC.pdf, Page 17] - Medical diagnostics equipment (SUPPLEMENTARY)
% [ ] [SCD_1a_what_is_SoC.pdf, Page 18] - Industrial automation (SUPPLEMENTARY)
% [ ] [SCD_1a_what_is_SoC.pdf, Page 19] - Studio equipment (SUPPLEMENTARY)
% [ ] [SCD_1a_what_is_SoC.pdf, Page 21] - Subjects discussed in SCD overview (IMPORTANT)
% [ ] [SCD_1a_what_is_SoC.pdf, Page 22] - SoC FPGA architecture diagram (CRITICAL)
% [ ] [SCD_1a_what_is_SoC.pdf, Page 23] - Linux on SoC illustration (IMPORTANT)
% [ ] [SCD_1a_what_is_SoC.pdf, Page 24] - Platform Designer interface (CRITICAL)
% [ ] [SCD_1a_what_is_SoC.pdf, Page 25] - Bootloader process diagram (IMPORTANT)
% [ ] [SCD_1a_what_is_SoC.pdf, Page 26] - On-chip memory architecture (CRITICAL)
% [ ] [SCD_1a_what_is_SoC.pdf, Page 27] - DRAM and Robert Dennard (SUPPLEMENTARY)
% [ ] [SCD_1a_what_is_SoC.pdf, Page 28] - JTAG state machine (CRITICAL)
% [ ] [SCD_1a_what_is_SoC.pdf, Page 29] - Timing analysis diagram (CRITICAL)
% [ ] [SCD_1a_what_is_SoC.pdf, Page 30] - Clock distribution network (CRITICAL)
% [ ] [SCD_1a_what_is_SoC.pdf, Page 31] - High-speed I/O interface (IMPORTANT)
% [ ] [SCD_1a_what_is_SoC.pdf, Page 32] - FPGA configuration circuit (CRITICAL)
% =====================