\section{SDRAM - Synchronous Dynamic Random Access Memory}

\subsection{Introduction and Motivation}

% TODO: Add image from SCD_4a_ddr_sdram.pdf, Slide 2
% Description: Motivation image showing computer storage hierarchy context
% Priority: SUPPLEMENTARY
% Suggested filename: lecture04a_motivation.png
% \includegraphics[width=\linewidth]{lecture04a_motivation.png}
% WHEN YOU ADD IMAGE: Uncomment line above (remove \%)

\begin{definition}{Learning Objectives}
\begin{itemize}
    \item List the pros and cons of SDRAM
    \item Describe the principle of operation of an SDRAM cell
    \item Calculate the number of banks, words, and bits of an SDRAM device
    \item Determine the refresh frequency of an SDRAM device
    \item Extract timing properties from an SDRAM datasheet
    \item Interpret the read and write accesses in a timing diagram and identify the corresponding timing parameters
    \item Explain the internal operation of a DDR SDRAM device
\end{itemize}
\end{definition}

\subsection{Memory Hierarchy and Storage Types}

\begin{concept}{Computer Storage Hierarchy}\\
Memory systems are organized hierarchically based on speed, cost, and capacity.

% TODO: Add image from SCD_4a_ddr_sdram.pdf, Slide 4
% Description: Storage hierarchy pyramid showing Primary/Secondary/Tertiary memory
% Priority: CRITICAL
% Suggested filename: lecture04a_storage_hierarchy.png
% \includegraphics[width=0.8\linewidth]{lecture04a_storage_hierarchy.png}
% WHEN YOU ADD IMAGE: Uncomment line above (remove \%)

\textbf{Hierarchy levels:}
\begin{itemize}
    \item \textbf{Fast and expensive:} Primary Memory (SRAM, DRAM)
    \item \textbf{Medium:} Secondary Memory (Disk)
    \item \textbf{Slow and cheap:} Tertiary Memory (Tape)
\end{itemize}

\textbf{Trade-offs:}
\begin{itemize}
    \item High capacity $\Rightarrow$ Slower access time
    \item Fast access $\Rightarrow$ High cost per bit
\end{itemize}
\end{concept}

\begin{example2}{Typical Memory System Configuration}\\
\textbf{System components and timing:}

% TODO: Add image from SCD_4a_ddr_sdram.pdf, Slide 5
% Description: Detailed memory hierarchy diagram with processor, cache levels, and timing
% Priority: CRITICAL
% Suggested filename: lecture04a_memory_system.png
% \includegraphics[width=\linewidth]{lecture04a_memory_system.png}
% WHEN YOU ADD IMAGE: Uncomment line above (remove \%)

\begin{itemize}
    \item \textbf{1st Level Cache:} Integrated in processor
    \item \textbf{2nd Level Cache:} SRAM, typical 5 ns access time
    \item \textbf{Main Memory:} DRAM, typical 50 ns access time (10x slower than cache)
    \item \textbf{Disk:} Typical 5 ms access time (100,000x slower than DRAM)
    \item \textbf{Tape:} Typical 5-50 s access time (1,000x slower than disk)
\end{itemize}

\tcblower

\textbf{Example calculation:}

For a 2 GHz processor (0.5 ns per clock cycle):
\begin{itemize}
    \item Disk access time: 5 ms = 10,000,000 clock cycles
    \item Transfer rate: Typically 1-4 GByte/s between cache levels
    \item I/O Bus: Typically 100-200 MByte/s
\end{itemize}

\important{The large speed differences motivate the use of caching and hierarchical memory structures.}
\end{example2}

\begin{definition}{Types of Semiconductor Memory}
\begin{itemize}
    \item \textbf{Volatile Memory:} Data lost when power is removed
    \begin{itemize}
        \item SRAM - Static Random Access Memory
        \item SDRAM - Synchronous Dynamic Random Access Memory (SDR, DDR)
    \end{itemize}
    \item \textbf{Non-volatile Memory:} Data retained without power
    \begin{itemize}
        \item Mask-programmed ROM
        \item PROM - Programmable Read Only Memory (OTP)
        \item EEPROM - Electrically Erasable PROM
        \item Flash EEPROM (NOR, NAND)
        \item nvSRAM, FRAM - Non-volatile RAM technologies
    \end{itemize}
\end{itemize}
\end{definition}

% TODO: Add image from SCD_4a_ddr_sdram.pdf, Slide 6
% Description: Memory types classification tree diagram
% Priority: IMPORTANT
% Suggested filename: lecture04a_memory_types.png
% \includegraphics[width=\linewidth]{lecture04a_memory_types.png}
% WHEN YOU ADD IMAGE: Uncomment line above (remove \%)

\subsection{SRAM vs DRAM Comparison}

\mult{2}

\begin{concept}{Static RAM (SRAM)}\\
\textbf{Cell structure:}
\begin{itemize}
    \item 6 transistors per bit (flip-flop/latch)
    \item Large cell area
\end{itemize}

% TODO: Add image from SCD_4a_ddr_sdram.pdf, Slide 7
% Description: SRAM cell circuit diagram with 6 transistors
% Priority: CRITICAL
% Suggested filename: lecture04a_sram_cell.png
% \includegraphics[width=\linewidth]{lecture04a_sram_cell.png}
% WHEN YOU ADD IMAGE: Uncomment line above (remove \%)

\textbf{Characteristics:}
\begin{itemize}
    \item Low density, high cost
    \item Up to 512 Mb per device
    \item Almost no static power consumption
    \item Asynchronous interface (no clock)
    \item Simple bus connection
    \item All accesses take similar time (approximately 5 ns, 200 MHz)
    \item Suitable for distributed accesses
\end{itemize}
\end{concept}

\begin{concept}{Dynamic RAM (DRAM)}\\
\textbf{Cell structure:}
\begin{itemize}
    \item 1 transistor + 1 capacitor per bit
    \item Small cell area
    \item Reduced complexity
    \item Reduced cost
\end{itemize}

% TODO: Add image from SCD_4a_ddr_sdram.pdf, Slide 9-10
% Description: DRAM cell circuit diagram showing transistor and capacitor
% Priority: CRITICAL
% Suggested filename: lecture04a_dram_cell.png
% \includegraphics[width=\linewidth]{lecture04a_dram_cell.png}
% WHEN YOU ADD IMAGE: Uncomment line above (remove \%)

\textbf{Characteristics:}
\begin{itemize}
    \item High density, low cost
    \item Up to 64 Gb per device
    \item Leakage currents require periodic refresh
    \item Synchronous interface (clocked)
    \item Requires dedicated SDRAM controller
    \item Long latency for first access
    \item Fast access for burst data
    \item Large overhead for single byte access
\end{itemize}
\end{concept}

\multend

\begin{definition}{DRAM Cell Operation}\\
\textbf{Storage mechanism:}

Information is stored as electrical charge in a capacitor. The charge represents a binary value (0 or 1).

% TODO: Add image from SCD_4a_ddr_sdram.pdf, Slide 11
% Description: DRAM cell structure with voltage decay graph showing capacitor discharge
% Priority: CRITICAL
% Suggested filename: lecture04a_dram_operation.png
% \includegraphics[width=\linewidth]{lecture04a_dram_operation.png}
% WHEN YOU ADD IMAGE: Uncomment line above (remove \%)

\textbf{Key features:}
\begin{itemize}
    \item Information stored as charge in capacitor
    \item High integration density allows large memories at low cost
    \item Leakage current causes loss of charge
    \item Capacitor holds charge only for a few milliseconds
    \item Charge must be refreshed periodically $\Rightarrow$ \textit{dynamic}
    \item Refresh logic usually located on SDRAM device
\end{itemize}

\textbf{Cell components:}
\begin{itemize}
    \item Bit line: used for reading/writing data
    \item Word line: activates transistor for access
    \item FET (Field Effect Transistor): controls access to capacitor
    \item Capacitor (C): stores the charge
\end{itemize}
\end{definition}

\begin{remark}
The reduced cell size (1T+1C vs 6T) allows DRAM to achieve much higher densities than SRAM, making it the preferred technology for main memory despite the need for refresh circuitry.
\end{remark}

\begin{example2}{Historical Context}\\
\textbf{Inventor:} Robert H. Dennard

% TODO: Add image from SCD_4a_ddr_sdram.pdf, Slide 18
% Description: Photo of Robert H. Dennard
% Priority: SUPPLEMENTARY
% Suggested filename: lecture04a_dennard.png
% \includegraphics[width=0.4\linewidth]{lecture04a_dennard.png}
% WHEN YOU ADD IMAGE: Uncomment line above (remove \%)

\begin{itemize}
    \item Born in 1932
    \item Invented DRAM at IBM in 1968
    \item Revolutionary impact on computer memory technology
\end{itemize}

\tcblower

\textbf{1 Mbit DRAM (1985):}

% TODO: Add image from SCD_4a_ddr_sdram.pdf, Slide 19
% Description: Photo of 1Mbit DRAM chip showing die size
% Priority: SUPPLEMENTARY
% Suggested filename: lecture04a_1mbit_dram.png
% \includegraphics[width=0.6\linewidth]{lecture04a_1mbit_dram.png}
% WHEN YOU ADD IMAGE: Uncomment line above (remove \%)

\begin{itemize}
    \item Die size: 8.66 mm $\times$ 3.96 mm
    \item First megabit-scale DRAM
\end{itemize}
\end{example2}

\subsection{Memory Architecture and Organization}

\begin{definition}{Basic Memory Architecture}\\
A memory is organized as an $n \times m$ array:
\begin{itemize}
    \item $n$ = number of words
    \item $m$ = number of data bits per word
    \item Bit cell stores a single bit ('0' or '1')
\end{itemize}

% TODO: Add image from SCD_4a_ddr_sdram.pdf, Slide 12
% Description: Generic memory array diagram showing address decoder, word lines, and data lines
% Priority: CRITICAL
% Suggested filename: lecture04a_memory_array.png
% \includegraphics[width=\linewidth]{lecture04a_memory_array.png}
% WHEN YOU ADD IMAGE: Uncomment line above (remove \%)

\textbf{Components:}
\begin{itemize}
    \item $k$ address lines where $n = 2^k$
    \item Address decoder selects one out of $n$ word lines
    \item Word lines: one per memory word
    \item Data lines: $m$ lines for word width
    \item Address range: 0 (lowest) to $n-1$ (highest)
\end{itemize}
\end{definition}

\begin{concept}{SDRAM Memory Devices}\\
DRAM cells use less area on silicon, allowing more bits per device.

\textbf{Capacity comparison:}
\begin{itemize}
    \item \textbf{SRAM:} Up to 512 Mb $\Rightarrow$ $2^{29}$ bits $\Rightarrow$ 26 address lines (byte addressing)
    \item \textbf{DRAM:} Up to 512 Gb $\Rightarrow$ $2^{39}$ bits $\Rightarrow$ 36 address lines
    \item \textbf{Largest devices:} Up to 512 Gb organized as 32G$\times$16b or 128G$\times$4b
    \item 128G words require 38 address lines
\end{itemize}

\important{Problem:} A very large address decoder with 38 address pins is impractical.

\textbf{Solution:} Address multiplexing
\begin{itemize}
    \item Use wider rows with multiple words per row
    \item Give row address first, then column address
    \item Only half the number of address pins required
\end{itemize}
\end{concept}

\begin{definition}{Address Multiplexing}\\
Addresses are multiplexed to reduce the number of required pins.

% TODO: Add image from SCD_4a_ddr_sdram.pdf, Slide 14
% Description: Address multiplexing diagram showing row and column address timing
% Priority: CRITICAL
% Suggested filename: lecture04a_address_mux.png
% \includegraphics[width=\linewidth]{lecture04a_address_mux.png}
% WHEN YOU ADD IMAGE: Uncomment line above (remove \%)

\textbf{Example: 1 Mbit DRAM}
\begin{itemize}
    \item Total: 1024 rows $\times$ 1024 columns
    \item Row address: 10 bits
    \item Column address: 10 bits
    \item Address select signal determines whether row or column address is present
    \item Only 10 address pins needed instead of 20
\end{itemize}

\textbf{Process:}
\begin{enumerate}
    \item Present row address on address lines
    \item Latch row address with RAS (Row Address Strobe)
    \item Present column address on same address lines
    \item Latch column address with CAS (Column Address Strobe)
\end{enumerate}
\end{definition}

\begin{concept}{DRAM Architecture}\\
DRAM is organized in a rectangular structure for optimization.

% TODO: Add image from SCD_4a_ddr_sdram.pdf, Slide 15
% Description: DRAM internal architecture showing row/column organization
% Priority: CRITICAL
% Suggested filename: lecture04a_dram_architecture.png
% \includegraphics[width=\linewidth]{lecture04a_dram_architecture.png}
% WHEN YOU ADD IMAGE: Uncomment line above (remove \%)

\textbf{Organization:}
\begin{itemize}
    \item \textbf{Rows:} Activated by row address
    \item \textbf{Columns:} Selected by column address
    \item \textbf{Controller logic:} Required for multiplexed addresses and refresh cycles
    \item \textbf{Amplifier per column:} One sense amplifier for each column
\end{itemize}

\textbf{Example configuration:}
\begin{itemize}
    \item Multiple DRAM chips to build a memory system
    \item One bit per chip (bit-sliced organization)
    \item 4 bits per row $\Rightarrow$ 2 column address lines
\end{itemize}
\end{concept}

\begin{concept}{Refresh Cycles and Amplifiers}\\
The non-square array structure optimizes refresh and power.

% TODO: Add image from SCD_4a_ddr_sdram.pdf, Slide 16
% Description: Diagram showing row/column ratio optimization
% Priority: IMPORTANT
% Suggested filename: lecture04a_refresh_optimization.png
% \includegraphics[width=\linewidth]{lecture04a_refresh_optimization.png}
% WHEN YOU ADD IMAGE: Uncomment line above (remove \%)

\textbf{Key relationships:}
\begin{itemize}
    \item A DRAM has as many refresh cycles as rows
    \item The number of refresh cycles determines the time between two refreshes
    \item A DRAM has as many amplifiers as columns
    \item The number of amplifiers essentially determines the power dissipation
    \item Optimization is achieved with a non-square array structure
\end{itemize}

\textbf{Example: 1 Mbit DRAM}
\begin{itemize}
    \item 512 rows $\times$ 2048 columns
    \item 11-bit addressing (row and column)
    \item Fewer refresh cycles (512 vs 1024)
    \item More columns (higher power but acceptable)
\end{itemize}
\end{concept}

\subsection{DRAM Timing and Access Cycles}

\begin{definition}{Single Inline Memory Module (SIMM)}\\
Early memory module standard with asynchronous interface.

% TODO: Add image from SCD_4a_ddr_sdram.pdf, Slide 20
% Description: 30-pin SIMM module pinout diagram
% Priority: IMPORTANT
% Suggested filename: lecture04a_simm.png
% \includegraphics[width=\linewidth]{lecture04a_simm.png}
% WHEN YOU ADD IMAGE: Uncomment line above (remove \%)

\textbf{Characteristics:}
\begin{itemize}
    \item 30-Pin SIMM configuration
    \item Asynchronous operation (no clock)
    \item Word size and number of words determined by module configuration
\end{itemize}
\end{definition}

\begin{concept}{Read Access Operation}\\
Reading data from DRAM involves multiple steps with specific timing requirements.

% TODO: Add image from SCD_4a_ddr_sdram.pdf, Slide 22
% Description: Detailed DRAM read access diagram showing internal operation
% Priority: CRITICAL
% Suggested filename: lecture04a_read_access.png
% \includegraphics[width=\linewidth]{lecture04a_read_access.png}
% WHEN YOU ADD IMAGE: Uncomment line above (remove \%)

\textbf{Read sequence:}
\begin{enumerate}
    \item Address placed on address bus
    \item RAS (Row Address Strobe) activated
    \item Row address latched and row activated
    \item CAS (Column Address Strobe) activated
    \item Column address latched
    \item Data amplified and output
    \item Data available on data bus
\end{enumerate}

\textbf{Internal operations:}
\begin{itemize}
    \item Sense amplifiers detect small voltage differences
    \item Row activation precharges bit lines
    \item Column multiplexer selects specific word
    \item Output buffers drive data bus
\end{itemize}
\end{concept}

\begin{concept}{Refresh Operation}\\
DRAM requires periodic refresh to maintain data integrity.

% TODO: Add image from SCD_4a_ddr_sdram.pdf, Slide 23
% Description: DRAM refresh cycle diagram
% Priority: CRITICAL
% Suggested filename: lecture04a_refresh.png
% \includegraphics[width=\linewidth]{lecture04a_refresh.png}
% WHEN YOU ADD IMAGE: Uncomment line above (remove \%)

\textbf{Refresh process:}
\begin{itemize}
    \item One row refreshed per refresh cycle
    \item Addressing the row automatically refreshes it
    \item Occurs before potential read/write operations
    \item All rows must be refreshed within the retention time (typically 64 ms)
    \item Refresh can be initiated by external controller or internal logic
\end{itemize}

\important{Refresh operations consume time and power, reducing effective memory bandwidth.}
\end{concept}

\begin{definition}{Asynchronous Read Cycle Timing}\\
Key timing parameters for asynchronous DRAM.

% TODO: Add image from SCD_4a_ddr_sdram.pdf, Slide 24
% Description: Timing diagram for asynchronous DRAM read cycle
% Priority: CRITICAL
% Suggested filename: lecture04a_async_read_timing.png
% \includegraphics[width=\linewidth]{lecture04a_async_read_timing.png}
% WHEN YOU ADD IMAGE: Uncomment line above (remove \%)

\textbf{Timing parameters:}
\begin{itemize}
    \item $t_{RAC}$: RAS Access Time (typical 60 ns) - time from RAS to valid data
    \item $t_{CAC}$: CAS Access Time (typical 30 ns) - time from CAS to valid data
    \item $t_{RCD}$: RAS to CAS Delay - minimum time between RAS and CAS
\end{itemize}

The CAS access time is shorter because the row is already activated.
\end{definition}

\begin{concept}{Page Mode Operation}\\
Page mode allows faster access to multiple words in the same row.

% TODO: Add image from SCD_4a_ddr_sdram.pdf, Slide 25
% Description: Page mode timing diagram showing multiple CAS cycles
% Priority: IMPORTANT
% Suggested filename: lecture04a_page_mode.png
% \includegraphics[width=\linewidth]{lecture04a_page_mode.png}
% WHEN YOU ADD IMAGE: Uncomment line above (remove \%)

\textbf{Operation:}
\begin{itemize}
    \item Row address given once with RAS
    \item Multiple column addresses given with CAS
    \item Each CAS cycle reads one word from the active row
    \item Significantly faster than full RAS/CAS cycles
    \item Useful for sequential or burst accesses
\end{itemize}

\textbf{Advantages:}
\begin{itemize}
    \item Reduced latency for subsequent accesses
    \item Better bandwidth utilization
    \item Lower power consumption per access
\end{itemize}
\end{concept}

\begin{definition}{Asynchronous Write Cycle Timing}\\
Write operations follow a similar but distinct timing pattern.

% TODO: Add image from SCD_4a_ddr_sdram.pdf, Slide 26
% Description: Timing diagram for asynchronous DRAM write cycle
% Priority: IMPORTANT
% Suggested filename: lecture04a_async_write_timing.png
% \includegraphics[width=\linewidth]{lecture04a_async_write_timing.png}
% WHEN YOU ADD IMAGE: Uncomment line above (remove \%)

\textbf{Write sequence:}
\begin{itemize}
    \item RAS activated with row address
    \item CAS activated with column address
    \item WE (Write Enable) signal activated
    \item Data placed on data bus
    \item Data written to selected cell
    \item Data bus may go to high impedance after write
\end{itemize}
\end{definition}

\begin{definition}{Refresh Cycle Timing}\\
Dedicated refresh cycles maintain data integrity.

% TODO: Add image from SCD_4a_ddr_sdram.pdf, Slide 27
% Description: Timing diagram showing refresh cycles
% Priority: IMPORTANT
% Suggested filename: lecture04a_refresh_timing.png
% \includegraphics[width=\linewidth]{lecture04a_refresh_timing.png}
% WHEN YOU ADD IMAGE: Uncomment line above (remove \%)

\textbf{Refresh process:}
\begin{itemize}
    \item RAS activated with row address
    \item CAS activated simultaneously (CAS before RAS refresh)
    \item One row refreshed per cycle
    \item No data transfer occurs
    \item Internal refresh counter can increment automatically
\end{itemize}
\end{definition}

\begin{concept}{Fast Page Mode DRAM}\\
Fast Page Mode (FPM) improves on basic page mode operation.

% TODO: Add image from SCD_4a_ddr_sdram.pdf, Slide 28
% Description: Fast Page Mode timing diagram
% Priority: IMPORTANT
% Suggested filename: lecture04a_fpm_timing.png
% \includegraphics[width=\linewidth]{lecture04a_fpm_timing.png}
% WHEN YOU ADD IMAGE: Uncomment line above (remove \%)

\textbf{Improvements:}
\begin{itemize}
    \item Faster CAS cycle times
    \item Pipelined access within a page
    \item Reduced setup and hold times
    \item Better suited for burst transfers
\end{itemize}
\end{concept}

\begin{example2}{Datasheet Exercise: IS41LV16100D}\\
\textbf{Task:} Extract timing parameters from a 16 Mb DRAM (1M $\times$ 16) datasheet.

\textbf{Find the following parameters:}
\begin{itemize}
    \item $t_{RCD}$ = RAS to CAS Delay Time
    \item $t_{RAC}$ = RAS Access Time
    \item $t_{CAC}$ = CAS Access Time
    \item $t_{PC}$ = Page Cycle Time
\end{itemize}

\tcblower

\textbf{Typical values from datasheet:}
\begin{itemize}
    \item These values depend on the speed grade of the device
    \item Faster speed grades have shorter timing parameters
    \item Trade-off between speed and cost
\end{itemize}

\important{Always consult the actual datasheet for precise timing values for your specific device and speed grade.}
\end{example2}

\begin{concept}{Access Time vs Cycle Time}\\
Important distinction for memory performance analysis.

% TODO: Add image from SCD_4a_ddr_sdram.pdf, Slide 30
% Description: Timing diagram showing access time and cycle time with precharge
% Priority: CRITICAL
% Suggested filename: lecture04a_access_vs_cycle.png
% \includegraphics[width=\linewidth]{lecture04a_access_vs_cycle.png}
% WHEN YOU ADD IMAGE: Uncomment line above (remove \%)

\textbf{Key timing parameters:}
\begin{itemize}
    \item $t_{RAS}$: RAS Active Time - how long RAS must remain active
    \item $t_{RC}$: Row Cycle Time - minimum time between successive row accesses
    \item $t_{RP}$: RAS Precharge Time - time needed to precharge before next RAS
    \item $t_{RCD}$: RAS to CAS Delay - minimum delay between RAS and CAS
\end{itemize}

\textbf{Problem:} Long access time for rows because rows, bit lines, and sense amplifiers need to be precharged.

\textbf{Question:} How could we improve the access speed?

\important{The precharge time $t_{RP}$ shown before the next RAS limits performance. This motivates the use of banking and interleaving.}
\end{concept}

\subsection{Banking and Interleaving}

\begin{concept}{Memory Banks and Interleaved Operation}\\
Banks allow overlapping of operations to hide latency.

% TODO: Add image from SCD_4a_ddr_sdram.pdf, Slide 31
% Description: Bank interleaving diagram showing overlapped operations
% Priority: CRITICAL
% Suggested filename: lecture04a_banking.png
% \includegraphics[width=\linewidth]{lecture04a_banking.png}
% WHEN YOU ADD IMAGE: Uncomment line above (remove \%)

\textbf{Concept:}
\begin{itemize}
    \item Split memory into multiple independent banks
    \item Read from one bank while precharging another
    \item Interleaved operation hides precharge time
    \item Bank address (BA) selects bank for current command
\end{itemize}

\textbf{Benefits:}
\begin{itemize}
    \item Increased effective bandwidth
    \item Reduced perceived latency
    \item Better utilization of memory bus
\end{itemize}

\textbf{Trade-off:}
\begin{itemize}
    \item Controller becomes more complex
    \item Requires careful scheduling of bank accesses
    \item Optimal when access pattern has good locality
\end{itemize}
\end{concept}

\subsection{DRAM Controller Design}

\begin{concept}{Principle of a DRAM Controller}\\
The DRAM controller bridges the CPU and DRAM devices.

% TODO: Add image from SCD_4a_ddr_sdram.pdf, Slide 32
% Description: DRAM controller block diagram
% Priority: CRITICAL
% Suggested filename: lecture04a_controller.png
% \includegraphics[width=\linewidth]{lecture04a_controller.png}
% WHEN YOU ADD IMAGE: Uncomment line above (remove \%)

\textbf{Controller functions:}
\begin{itemize}
    \item Generate DRAM control signals (RAS, CAS, WE)
    \item Multiplex addresses (row and column)
    \item Manage refresh cycles
    \item Arbitrate between processor access and refresh
    \item Convert internal bus signals to DRAM protocol
\end{itemize}

\textbf{Components:}
\begin{itemize}
    \item \textbf{Refresh Logic:} Generates periodic refresh requests
    \item \textbf{Refresh Counter:} Tracks which row to refresh next
    \item \textbf{Control Signal Generator:} Creates RAS, CAS, WE timing
    \item \textbf{Address Multiplexer:} Switches between row and column addresses
    \item \textbf{Access/Refresh Arbitration:} Prioritizes CPU and refresh requests
\end{itemize}

\important{The CPU requires a corresponding controller to convert internal bus signals into DRAM-specific signals.}
\end{concept}

\begin{remark}
Commercial SDRAM controller IP cores (like Altera SDRAM Controller IP) provide ready-made solutions that handle all timing and control signal generation automatically.
\end{remark}

\subsection{Synchronous DRAM (SDRAM)}

\begin{concept}{Synchronous Memory Interface}\\
SDRAM adds a clock signal to synchronize all operations.

% TODO: Add image from SCD_4a_ddr_sdram.pdf, Slide 35
% Description: SDRAM timing diagram with clock
% Priority: CRITICAL
% Suggested filename: lecture04a_sdram_timing.png
% \includegraphics[width=\linewidth]{lecture04a_sdram_timing.png}
% WHEN YOU ADD IMAGE: Uncomment line above (remove \%)

\textbf{Key characteristics:}
\begin{itemize}
    \item All signals synchronized to a common clock
    \item Address and control signals sampled at clock rising edge
    \item Predictable timing - all operations take fixed number of clock cycles
    \item Burst access mode for sequential data
\end{itemize}

\textbf{Read cycle timing:}
\begin{itemize}
    \item Long latency for first data item in a row
    \item Short access time for subsequent items in same row
    \item Row address given first, followed by column addresses
\end{itemize}
\end{concept}

\begin{definition}{SDRAM Timing Parameters}\\
Key timing values are specified in clock cycles.

% TODO: Add image from SCD_4a_ddr_sdram.pdf, Slide 36
% Description: Detailed SDRAM read cycle timing with parameter labels
% Priority: CRITICAL
% Suggested filename: lecture04a_sdram_parameters.png
% \includegraphics[width=\linewidth]{lecture04a_sdram_parameters.png}
% WHEN YOU ADD IMAGE: Uncomment line above (remove \%)

\textbf{Critical parameters:}
\begin{itemize}
    \item $t_{RCD}$ = RAS to CAS Delay (in clock cycles)
    \item $t_{CL}$ = CAS Latency (in clock cycles)
    \item $t_{RP}$ = RAS Precharge Time (in clock cycles)
\end{itemize}

\textbf{Typical values:}
\begin{itemize}
    \item Access time: 20 to 50 ns
    \item Clock cycle time: 10 ns at 100 MHz bus (PC100)
    \item Parameters expressed as number of clock cycles
\end{itemize}
\end{definition}

\begin{concept}{Advantages of Synchronous Operation}\\
Synchronous interface provides several benefits over asynchronous.

\textbf{Benefits:}
\begin{itemize}
    \item \textbf{Simplified timing:} Address and control signals sampled at rising clock edge
    \item \textbf{Glitch immunity:} Glitches on address/control lines do not cause failures
    \item \textbf{Predictable delays:} RAS-CAS delays are fixed number of clock cycles
    \item \textbf{Fixed access times:} All operations take deterministic time
    \item \textbf{Fixed precharge:} RAS precharge time is fixed number of cycles
\end{itemize}

\important{The DRAM controller must provide signals synchronous to a common clock, but this simplifies overall system design.}
\end{concept}

\subsection{DDR SDRAM Technology}

\begin{concept}{Double Data Rate (DDR) SDRAM}\\
DDR transfers data on both clock edges, doubling effective bandwidth.

% TODO: Add image from SCD_4a_ddr_sdram.pdf, Slide 38
% Description: DDR SDRAM timing showing data on both clock edges
% Priority: CRITICAL
% Suggested filename: lecture04a_ddr_timing.png
% \includegraphics[width=\linewidth]{lecture04a_ddr_timing.png}
% WHEN YOU ADD IMAGE: Uncomment line above (remove \%)

\textbf{Key innovation:}
\begin{itemize}
    \item Data transferred on rising AND falling clock edges
    \item Doubles data rate without increasing clock frequency
    \item Same $t_{RP}$, $t_{RCD}$, $t_{CL}$ as SDR SDRAM
    \item Requires differential data strobe signals (DQS)
\end{itemize}

\textbf{Advantages:}
\begin{itemize}
    \item Double bandwidth compared to SDR
    \item Lower power per bit transferred
    \item Compatible with existing memory architecture
\end{itemize}
\end{concept}

\begin{definition}{DDR SDRAM Designation}\\
DDR memory modules are designated by bandwidth and timing parameters.

\textbf{Naming convention:}
\begin{itemize}
    \item PC xxxx a-b-c or DDR yyy a-b-c
    \item xxxx = Bandwidth in MB/s
    \item yyy = Transfer rate in MT/s (Mega Transfers per second)
    \item a = CAS Latency ($t_{CL}$)
    \item b = RAS-CAS Delay ($t_{RCD}$)
    \item c = RAS Precharge ($t_{RP}$)
\end{itemize}
\end{definition}

\begin{example2}{DDR Memory Designation}\\
\textbf{Example:} PC 1600 2-3-3 or DDR200 2-3-3

\tcblower

\textbf{Interpretation:}
\begin{itemize}
    \item 200 MHz transfer rate (DDR200)
    \item 1600 MB/s bandwidth (for 64-bit wide DIMM: 200 MT/s $\times$ 8 bytes)
    \item Access time: 50 ns for first access, then 8 bytes every 5 ns
    \item Timing: 2-3-3 means $t_{CL}$=2, $t_{RCD}$=3, $t_{RP}$=3 clock cycles
\end{itemize}

\important{An additional fourth value may specify Row-Active-Time ($t_{RAS}$).}
\end{example2}

\begin{highlight}{DDR Performance Comparison}\\
\begin{center}
\begin{tabular}{|l|c|c|c|c|}
\hline
\textbf{Type} & \textbf{Bus Clock} & \textbf{Transfer Rate} & \textbf{Bandwidth} & \textbf{Timing} \\
\hline
PC100 & 100 MHz & 100 MT/s & 800 MB/s & SDR \\
\hline
DDR-400 & 200 MHz & 400 MT/s & 3200 MB/s & CL2-2-2-5 \\
\hline
DDR2-800 & 400 MHz & 800 MT/s & 6400 MB/s & CL4-4-4-10 \\
\hline
DDR3-1600 & 800 MHz & 1600 MT/s & 12800 MB/s & CL9-9-9-21 \\
\hline
\end{tabular}
\end{center}

\textbf{Notes:}
\begin{itemize}
    \item Bandwidth values are for 64-bit wide DIMM modules
    \item Transfer rate is double the bus clock (DDR)
    \item Higher frequencies require longer latencies in clock cycles
\end{itemize}
\end{highlight}

\begin{definition}{Timing Parameter Details}\\
Understanding what the timing numbers mean:

\textbf{Parameter definitions:}
\begin{itemize}
    \item \textbf{CAS Latency ($t_{CL}$):} Number of clock cycles waiting time for addressing a non-directly following memory cell within the same row
    \item \textbf{RAS to CAS Delay ($t_{RCD}$):} Number of clock cycles waiting time between row addressing and column addressing
    \item \textbf{RAS Precharge Delay ($t_{RP}$):} Number of clock cycles waiting time between two consecutive row addressings
    \item \textbf{Row-Active-Time ($t_{RAS}$):} Number of clock cycles waiting time between two consecutive row addressings of the same row
\end{itemize}
\end{definition}

\begin{example2}{DDR Timing Examples}\\
\textbf{Different DDR generations with same absolute timing:}

\begin{itemize}
    \item \textbf{DDR-400 CL2-2-2-5:} 200 MHz bus clock
    \begin{itemize}
        \item $t_{CL}$ = 10 ns, $t_{RCD}$ = 10 ns, $t_{RP}$ = 10 ns, $t_{RAS}$ = 25 ns
    \end{itemize}
    \item \textbf{DDR2-800 CL4-4-4-10:} 400 MHz bus clock
    \begin{itemize}
        \item $t_{CL}$ = 10 ns, $t_{RCD}$ = 10 ns, $t_{RP}$ = 10 ns, $t_{RAS}$ = 25 ns
    \end{itemize}
    \item \textbf{DDR3-1600 CL9-9-9-21:} 800 MHz bus clock
    \begin{itemize}
        \item $t_{CL}$ = 11.25 ns, $t_{RCD}$ = 11.25 ns, $t_{RP}$ = 11.25 ns, $t_{RAS}$ = 26.25 ns
    \end{itemize}
\end{itemize}

\tcblower

\textbf{Observations:}
\begin{itemize}
    \item Higher clock frequencies require more clock cycles for same absolute time
    \item Bandwidth doubles with each generation despite similar absolute timings
    \item DDR3 shows slightly longer absolute latencies due to technology constraints
\end{itemize}
\end{example2}

\begin{example2}{Actual Timing Examples}\\
\textbf{Different speed grades have different absolute timings:}

\begin{itemize}
    \item Fast timing: $t_{CL}$ = 10.0 ns
    \item Medium timing: $t_{CL}$ = 13.5 ns
    \item Slower timing: $t_{CL}$ = 13.75 ns
\end{itemize}

\tcblower

\important{Faster timing grades cost more but provide better performance. Always match memory speed to motherboard capabilities.}
\end{example2}

\begin{concept}{DDR2 SDRAM}\\
DDR2 doubles the data rate again through internal prefetch.

% TODO: Add image from SCD_4a_ddr_sdram.pdf, Slide 42
% Description: DDR2 architecture diagram showing 4x prefetch
% Priority: IMPORTANT
% Suggested filename: lecture04a_ddr2_architecture.png
% \includegraphics[width=\linewidth]{lecture04a_ddr2_architecture.png}
% WHEN YOU ADD IMAGE: Uncomment line above (remove \%)

\textbf{Key features:}
\begin{itemize}
    \item 4-bit prefetch (vs 2-bit in DDR)
    \item Internal clock runs at half the I/O rate
    \item Example: 100 MHz internal, 200 MHz bus, 400 MBit/s transfer
    \item Lower operating voltage: 1.8V (vs 2.5V for DDR)
    \item Improved signal integrity with on-die termination
\end{itemize}
\end{concept}

\begin{concept}{DDR3 SDRAM}\\
DDR3 continues the evolution with even higher speeds.

% TODO: Add image from SCD_4a_ddr_sdram.pdf, Slide 43
% Description: DDR3 architecture diagram showing 8x prefetch
% Priority: IMPORTANT
% Suggested filename: lecture04a_ddr3_architecture.png
% \includegraphics[width=\linewidth]{lecture04a_ddr3_architecture.png}
% WHEN YOU ADD IMAGE: Uncomment line above (remove \%)

\textbf{Key features:}
\begin{itemize}
    \item 8-bit prefetch (vs 4-bit in DDR2)
    \item Even lower operating voltage: 1.5V (vs 1.8V for DDR2)
    \item Higher clock frequencies (400 MHz bus for DDR3-1600)
    \item Improved power efficiency
    \item Better signal integrity
\end{itemize}

\textbf{Common speeds:}
\begin{itemize}
    \item DDR3-800 (PC3-6400): 100 MHz core, 400 MHz I/O, 800 MT/s
    \item DDR3-1066 (PC3-8500): 133 MHz core, 533 MHz I/O, 1066 MT/s
    \item DDR3-1333 (PC3-10600): 166 MHz core, 666 MHz I/O, 1333 MT/s
    \item DDR3-1600 (PC3-12800): 200 MHz core, 800 MHz I/O, 1600 MT/s
\end{itemize}
\end{concept}

\begin{example2}{DDR3 on DE1-SoC Board}\\
\textbf{IS42R16320D SDRAM chip:}

% TODO: Add image from SCD_4a_ddr_sdram.pdf, Slide 44
% Description: DE1-SoC board with SDRAM chip highlighted
% Priority: SUPPLEMENTARY
% Suggested filename: lecture04a_de1soc_sdram.png
% \includegraphics[width=0.7\linewidth]{lecture04a_de1soc_sdram.png}
% WHEN YOU ADD IMAGE: Uncomment line above (remove \%)

\textbf{Determine the following:}
\begin{itemize}
    \item Number of words: ?
    \item Word size: ?
    \item Number of bits: ?
    \item Number of address lines: ?
    \item Number of banks: ?
\end{itemize}

\tcblower

\textbf{Organization analysis:}
\begin{itemize}
    \item Number of rows: ?
    \item Number of columns: ?
\end{itemize}

\important{Exercise: Consult the IS42R16320D datasheet to answer these questions.}
\end{example2}

\begin{example2}{SDRAM Block Diagram Analysis}\\
\textbf{Analyze the internal block diagram:}

% TODO: Add image from SCD_4a_ddr_sdram.pdf, Slide 46
% Description: Internal SDRAM block diagram showing banks, row/column decoders
% Priority: CRITICAL
% Suggested filename: lecture04a_sdram_block_diagram.png
% \includegraphics[width=\linewidth]{lecture04a_sdram_block_diagram.png}
% WHEN YOU ADD IMAGE: Uncomment line above (remove \%)

\textbf{Determine from the diagram:}
\begin{itemize}
    \item Number of banks
    \item Row address width
    \item Column address width
    \item Word width
    \item Total number of words
\end{itemize}

\tcblower

\textbf{Analysis approach:}
\begin{enumerate}
    \item Count the number of independent memory arrays (banks)
    \item Determine address bus widths from decoder inputs
    \item Identify data bus width
    \item Calculate total capacity from organization
\end{enumerate}
\end{example2}

\begin{concept}{Non-volatile SRAM (nvSRAM)}\\
Combines SRAM speed with non-volatile storage.

% TODO: Add image from SCD_4a_ddr_sdram.pdf, Slide 8
% Description: nvSRAM architecture diagram
% Priority: SUPPLEMENTARY
% Suggested filename: lecture04a_nvsram.png
% \includegraphics[width=0.7\linewidth]{lecture04a_nvsram.png}
% WHEN YOU ADD IMAGE: Uncomment line above (remove \%)

\textbf{Key features:}
\begin{itemize}
    \item Speed comparable to SRAM
    \item Data retained without power
    \item Each SRAM cell has a Flash or other NV cell alongside
    \item NV contents updated regularly during operation
    \item Upon power loss, modified SRAM cells stored in NV cells
    \item Backup powered by capacitor during power-down
\end{itemize}

\textbf{Applications:}
\begin{itemize}
    \item Critical data storage
    \item Fast access with data persistence
    \item Replacing battery-backed SRAM
\end{itemize}
\end{concept}

% ===== IMAGE SUMMARY =====
% Total images needed: 33
% CRITICAL priority: 21
% IMPORTANT priority: 9
% SUPPLEMENTARY priority: 3
%
% Quick extraction checklist:
% [X] [SCD_4a_ddr_sdram.pdf, Slide 2] - Motivation image (SUPPLEMENTARY)
% [X] [SCD_4a_ddr_sdram.pdf, Slide 4] - Storage hierarchy pyramid (CRITICAL)
% [X] [SCD_4a_ddr_sdram.pdf, Slide 5] - Memory system with timing (CRITICAL)
% [X] [SCD_4a_ddr_sdram.pdf, Slide 6] - Memory types tree (IMPORTANT)
% [X] [SCD_4a_ddr_sdram.pdf, Slide 7] - SRAM cell circuit (CRITICAL)
% [X] [SCD_4a_ddr_sdram.pdf, Slide 9-10] - DRAM cell circuit (CRITICAL)
% [X] [SCD_4a_ddr_sdram.pdf, Slide 11] - DRAM operation with decay (CRITICAL)
% [X] [SCD_4a_ddr_sdram.pdf, Slide 18] - Robert Dennard photo (SUPPLEMENTARY)
% [X] [SCD_4a_ddr_sdram.pdf, Slide 19] - 1Mbit DRAM chip (SUPPLEMENTARY)
% [X] [SCD_4a_ddr_sdram.pdf, Slide 12] - Memory array diagram (CRITICAL)
% [X] [SCD_4a_ddr_sdram.pdf, Slide 14] - Address multiplexing (CRITICAL)
% [X] [SCD_4a_ddr_sdram.pdf, Slide 15] - DRAM architecture (CRITICAL)
% [X] [SCD_4a_ddr_sdram.pdf, Slide 16] - Refresh optimization (IMPORTANT)
% [X] [SCD_4a_ddr_sdram.pdf, Slide 20] - SIMM pinout (IMPORTANT)
% [X] [SCD_4a_ddr_sdram.pdf, Slide 22] - Read access diagram (CRITICAL)
% [X] [SCD_4a_ddr_sdram.pdf, Slide 23] - Refresh diagram (CRITICAL)
% [X] [SCD_4a_ddr_sdram.pdf, Slide 24] - Async read timing (CRITICAL)
% [X] [SCD_4a_ddr_sdram.pdf, Slide 25] - Page mode timing (IMPORTANT)
% [X] [SCD_4a_ddr_sdram.pdf, Slide 26] - Async write timing (IMPORTANT)
% [X] [SCD_4a_ddr_sdram.pdf, Slide 27] - Refresh timing (IMPORTANT)
% [X] [SCD_4a_ddr_sdram.pdf, Slide 28] - Fast page mode (IMPORTANT)
% [X] [SCD_4a_ddr_sdram.pdf, Slide 30] - Access vs cycle time (CRITICAL)
% [X] [SCD_4a_ddr_sdram.pdf, Slide 31] - Banking diagram (CRITICAL)
% [X] [SCD_4a_ddr_sdram.pdf, Slide 32] - Controller block diagram (CRITICAL)
% [X] [SCD_4a_ddr_sdram.pdf, Slide 35] - SDRAM timing (CRITICAL)
% [X] [SCD_4a_ddr_sdram.pdf, Slide 36] - SDRAM parameters (CRITICAL)
% [X] [SCD_4a_ddr_sdram.pdf, Slide 38] - DDR timing (CRITICAL)
% [X] [SCD_4a_ddr_sdram.pdf, Slide 42] - DDR2 architecture (IMPORTANT)
% [X] [SCD_4a_ddr_sdram.pdf, Slide 43] - DDR3 architecture (IMPORTANT)
% [X] [SCD_4a_ddr_sdram.pdf, Slide 44] - DE1-SoC SDRAM (SUPPLEMENTARY)
% [X] [SCD_4a_ddr_sdram.pdf, Slide 46] - SDRAM block diagram (CRITICAL)
% [X] [SCD_4a_ddr_sdram.pdf, Slide 8] - nvSRAM architecture (SUPPLEMENTARY)
% =====================