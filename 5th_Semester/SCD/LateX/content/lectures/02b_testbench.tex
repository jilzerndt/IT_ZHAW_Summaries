\section{VHDL Testbench and Simulation}

\subsection{Learning Objectives}

\begin{concept}{Course Goals}\\
Students will be able to:
\begin{itemize}
    \item Name at least four advantages of simulation versus hardware testing
    \item Describe the operation of a testbench
    \item Implement a procedure-based testbench for their designs
\end{itemize}
\end{concept}

\subsection{Testbench Architecture}

\begin{definition}{Testbench Components}\\
A testbench is a VHDL construct that provides automated testing for a design.

\textbf{Key Components:}
\begin{itemize}
    \item \textbf{Device Under Test (DUT):} The design being tested
    \item \textbf{Stimulus Generator:} Creates input signals for DUT
    \item \textbf{Expected Results:} Known correct outputs
    \item \textbf{Comparison Logic:} Compares actual vs expected outputs
    \item \textbf{Infrastructure:} Clock generation, file I/O, reporting
\end{itemize}

\textbf{Important Characteristic:}
\begin{itemize}
    \item Testbench entity has \textbf{no ports} (no external I/O)
    \item All signals are internal to simulation
\end{itemize}
\end{definition}

% TODO: Add image from SCD_2b_Testbench.pdf, Page 3
% Description: Testbench architecture block diagram
% Priority: CRITICAL
% Suggested filename: lecture02b_testbench_architecture.png
% \includegraphics[width=\linewidth]{lecture02b_testbench_architecture.png}

\begin{code}{Testbench Entity Structure}\\
\begin{lstlisting}[language=VHDL, style=base]
-- Testbench file: block_top_tb.vhd

ENTITY block_top_tb IS
  -- Entity has no ports since no I/Os
END block_top_tb;

ARCHITECTURE behavior OF block_top_tb IS
  -- Component declarations
  -- Signal declarations
  -- Constants
BEGIN
  -- DUT instantiation
  -- Stimulus generation
  -- Response checking
END behavior;
\end{lstlisting}

\important{The testbench has no ports because it is the top-level entity in simulation.}
\end{code}

% TODO: Add image from SCD_2b_Testbench.pdf, Page 4
% Description: Testbench structure showing entity without ports
% Priority: CRITICAL
% Suggested filename: lecture02b_testbench_structure.png
% \includegraphics[width=\linewidth]{lecture02b_testbench_structure.png}

\raggedcolumns
\columnbreak

\subsection{Purpose of Simulation}

\begin{concept}{Why Simulate?}\\
Simulation provides significant advantages over hardware-only testing:

\textbf{Accessibility Benefits:}
\begin{itemize}
    \item Testing and debugging in SoC is difficult:
    \begin{itemize}
        \item Pins buried under device
        \item Internal signals not accessible from outside
        \item Many parallel processes occurring simultaneously
    \end{itemize}
    \item Simulation allows tracking of \textbf{every signal at every point in time}
    \item All internal states visible and traceable
\end{itemize}

\textbf{Verification Benefits:}
\begin{itemize}
    \item \textbf{Automated verification:} Comparison of output with expected results
    \item \textbf{Regression testing:} Test every block automatically
    \item \textbf{Better test coverage:} Easily provoke edge cases and corner conditions
\end{itemize}

\textbf{Development Efficiency:}
\begin{itemize}
    \item \textbf{Faster iterations:} DUT can be recompiled in seconds vs synthesized in minutes
    \item \textbf{Reduced risk:} ASIC fabrication takes months and costs \$100,000+ depending on process
    \item \textbf{Lower testing effort:} Simulation can reduce overall testing time
    \item \textbf{Automation:} Can be automated even for large designs
\end{itemize}
\end{concept}

\subsection{Questasim/ModelSim Tool Flow}

\begin{concept}{VHDL Compilation and Simulation Flow}\\
The simulation process involves two main phases:

\textbf{Phase 1: Compilation}
\begin{enumerate}
    \item User VHDL files are written (design + testbench)
    \item Questasim compiler processes all VHDL files
    \item Compiled objects stored in libraries (typically \texttt{work})
    \item Standard libraries automatically included
\end{enumerate}

\textbf{Phase 2: Simulation}
\begin{enumerate}
    \item Simulator loads compiled libraries
    \item Testbench procedures generate stimuli
    \item DUT responds to stimuli
    \item Results captured as waveforms and text output
\end{enumerate}
\end{concept}

\subsection{Procedure-Based Testbench}

\begin{concept}{Why Procedure-Based?}\\
Advantages of using procedures:
\begin{itemize}
    \item Simple writing and expansion of test cases
    \item Supports scripting approach
    \item Easy to maintain and modify
    \item Input format: \texttt{command [arg\_0 arg\_1 ... arg\_n]}
\end{itemize}
\end{concept}

\begin{definition}{Test Script Structure}\\
Format: \texttt{command arg1 arg2 arg3 arg4}

Each argument is one byte (2 hex digits), MSB = arg1, LSB = arg4
\end{definition}

% ===== IMAGE SUMMARY =====
% Total images needed: 16
% CRITICAL priority: 9
% IMPORTANT priority: 6
% SUPPLEMENTARY priority: 1
% =====================