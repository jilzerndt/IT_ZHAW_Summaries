 \section{FPGA Memory Configurations and Modes}

\subsection{Learning Objectives}

\begin{concept}{Course Goals}\\
Students will be able to:
\begin{itemize}
    \item List and identify several modes of operation of block memories in FPGA
    \item Explain how to use block memories for clock domain crossing
    \item Calculate signal widths for memories with different data bus widths
    \item Understand how a block memory can be defined in VHDL code
    \item Configure a block memory using the toolchain
\end{itemize}
\end{concept}

\subsection{Recap: FPGA Memory Basics}

\begin{remark}
\textbf{Key Topics from Previous Lecture:}
\begin{itemize}
    \item Applications of FPGA memory
    \item Architecture of SRAM cells
    \item Memory cells organization
    \item Timing diagrams of read/write accesses
    \item Registered vs unregistered output modes
\end{itemize}
\end{remark}

\subsection{Memory Block Configurations}

\begin{definition}{4K-Bit Memory Block Configurations}\\
Hard macros (silicon) of RAM blocks support various word sizes from 1 to 32 bits.

\textbf{Key Characteristics:}
\begin{itemize}
    \item Toolchain combines RAM blocks to build desired word size and memory capacity
    \item Even larger than 32 bits possible through block combination
    \item Maximum performance: ~250 MHz
    \item Total RAM bits per M4K block (including parity): 4,608 bits
\end{itemize}

\textbf{Supported Configurations:}
\begin{itemize}
    \item 4K $\times$ 1 bit
    \item 2K $\times$ 2 bits
    \item 1K $\times$ 4 bits
    \item 512 $\times$ 8 bits
    \item 512 $\times$ 9 bits (with parity)
    \item 256 $\times$ 16 bits
    \item 256 $\times$ 18 bits (with parity)
    \item 128 $\times$ 32 bits (not available in true dual-port mode)
    \item 128 $\times$ 36 bits (not available in true dual-port mode, with parity)
\end{itemize}
\end{definition}

% TODO: Add image from SCD_2a_Memory_2.pdf, Page 5
% Description: Table of memory configurations
% Priority: IMPORTANT
% Suggested filename: lecture02a_memory_configurations.png
% \includegraphics[width=\linewidth]{lecture02a_memory_configurations.png}

\raggedcolumns
\columnbreak

\subsection{Memory Operation Modes}

\subsubsection{ROM Mode}

\begin{definition}{Read-Only Memory (ROM) in FPGA}\\
ROM blocks provide non-volatile storage initialized at configuration time.

\textbf{Characteristics:}
\begin{itemize}
    \item Memory contents loaded from external HEX file to configuration file
    \item Contents programmed during FPGA configuration
    \item Memory cannot be modified at runtime
    \item Block type: AUTO (toolchain selects optimal implementation)
\end{itemize}

\textbf{Interface:}
\begin{itemize}
    \item \textbf{address:} Read address input
    \item \textbf{read\_data:} Data output
    \item \textbf{clock:} Synchronous clock (for registered output)
\end{itemize}
\end{definition}

% TODO: Add image from SCD_2a_Memory_2.pdf, Page 6
% Description: ROM block diagram with configuration interface
% Priority: IMPORTANT
% Suggested filename: lecture02a_rom_mode.png
% \includegraphics[width=\linewidth]{lecture02a_rom_mode.png}

\begin{definition}{Intel HEX Format}\\
Intel HEX is a file format for representing binary data in ASCII text format.

\textbf{Record Structure:}
$$\text{:LLAAAATTDD...DDCC}$$

Where:
\begin{itemize}
    \item \textbf{::} Start code (1 character)
    \item \textbf{LL:} Byte count (2 hex digits) - number of data bytes
    \item \textbf{AAAA:} Address (4 hex digits) - 16-bit address
    \item \textbf{TT:} Record type (2 hex digits):
    \begin{itemize}
        \item 00 = Data record
        \item 01 = End of file record
        \item 02 = Extended segment address
        \item 04 = Extended linear address
    \end{itemize}
    \item \textbf{DD...DD:} Data bytes (2n hex digits)
    \item \textbf{CC:} Checksum (2 hex digits)
\end{itemize}

\textbf{Example Records:}
\begin{verbatim}
:040000100FFFF0000C   (Address 0x0000: 0xFF, 0xFF, 0x00, 0x00)
:0400020000FFFF002F   (Address 0x0002: 0x00, 0xFF, 0xFF, 0x00)
\end{verbatim}
\end{definition}

% TODO: Add image from SCD_2a_Memory_2.pdf, Page 7
% Description: Intel HEX format structure diagram
% Priority: IMPORTANT
% Suggested filename: lecture02a_intel_hex_format.png
% \includegraphics[width=\linewidth]{lecture02a_intel_hex_format.png}

\subsubsection{Single Port RAM Mode}

\begin{definition}{Single Port RAM}\\
Single port RAM provides one port for both read and write operations.

\textbf{Interface Signals:}
\begin{itemize}
    \item \textbf{write\_data:} Input data bus
    \item \textbf{read\_data:} Output data bus
    \item \textbf{write\_enable:} Write control signal
    \item \textbf{address:} Memory address
    \item \textbf{clock:} Synchronous clock
\end{itemize}

\textbf{Features:}
\begin{itemize}
    \item Memory contents may be preset using a HEX file
    \item Contents initialized during configuration
    \item Random access during operation
    \item Read and write through same port (not simultaneous)
\end{itemize}

\textbf{Example Configuration:}
\begin{itemize}
    \item Data width: 8 bits
    \item Address width: 5 bits
    \item Capacity: $2^5 = 32$ words
\end{itemize}
\end{definition}

% TODO: Add image from SCD_2a_Memory_2.pdf, Page 8
% Description: Single port RAM block diagram
% Priority: CRITICAL
% Suggested filename: lecture02a_single_port_ram.png
% \includegraphics[width=\linewidth]{lecture02a_single_port_ram.png}

% TODO: Add image from SCD_2a_Memory_2.pdf, Page 9
% Description: Chip Planner view of single port RAM
% Priority: SUPPLEMENTARY
% Suggested filename: lecture02a_chip_planner_single_port.png
% \includegraphics[width=\linewidth]{lecture02a_chip_planner_single_port.png}

\subsubsection{Simple Dual Port Mode}

\begin{definition}{Simple Dual Port RAM}\\
Simple dual port mode provides separate ports for reading and writing.

\textbf{Port Characteristics:}
\begin{itemize}
    \item \textbf{Port A:} Write-only
    \item \textbf{Port B:} Read-only
\end{itemize}

\textbf{Interface Signals:}
\begin{itemize}
    \item \textbf{Write\_data:} Input data for writing
    \item \textbf{Write\_address:} Address for write operations
    \item \textbf{Write\_enable:} Write control signal
    \item \textbf{Read\_data:} Output data from reading
    \item \textbf{Read\_address:} Address for read operations
    \item \textbf{clock:} Single clock for both ports
\end{itemize}

\textbf{Important Limitations:}
\begin{itemize}
    \item Port B does not know if there is any data written
    \item Port A does not know if the data has been read
    \item No handshaking mechanism between ports
\end{itemize}

\textbf{Configuration Options:}
\begin{itemize}
    \item Simple Dual Port
    \item Synchronous inputs/outputs
    \item One clock or dual clock operation
\end{itemize}
\end{definition}

% TODO: Add image from SCD_2a_Memory_2.pdf, Page 10
% Description: Simple dual port RAM block diagram
% Priority: CRITICAL
% Suggested filename: lecture02a_simple_dual_port.png
% \includegraphics[width=\linewidth]{lecture02a_simple_dual_port.png}

\raggedcolumns
\columnbreak

\subsubsection{True Dual Port Mode}

\begin{definition}{True Dual Port RAM}\\
True dual port mode allows both ports to read and write independently.

\textbf{Port Capabilities:}
\begin{itemize}
    \item \textbf{Both Port A and Port B:} Can read AND write
    \item Independent operation on both ports
    \item Simultaneous access possible
\end{itemize}

\textbf{Interface Signals (per port):}
\begin{itemize}
    \item \textbf{address:} Memory address
    \item \textbf{write\_data:} Input data for writing
    \item \textbf{write\_enable:} Write control signal
    \item \textbf{read\_data:} Output data from reading
    \item \textbf{clock:} Synchronous clock (can be different for each port)
\end{itemize}

\textbf{Configuration:}
\begin{itemize}
    \item True Dual Port mode
    \item Synchronous inputs/outputs
    \item One clock or dual clock operation
    \item Capacity: e.g., 32 words RAM
\end{itemize}
\end{definition}

% TODO: Add image from SCD_2a_Memory_2.pdf, Page 11
% Description: True dual port RAM block diagram
% Priority: CRITICAL
% Suggested filename: lecture02a_true_dual_port.png
% \includegraphics[width=\linewidth]{lecture02a_true_dual_port.png}

\begin{remark}
\textbf{Potential Issues in True Dual Port Mode:}

What could go wrong?
\begin{itemize}
    \item \textbf{Write Conflict:} Both ports writing to same address simultaneously
    \item \textbf{Read-During-Write:} Port B reading from address Port A is writing to
    \item \textbf{No Port Coordination:} No port knows what the other is doing
    \item \textbf{No Data Validation:} No checking mechanisms to ensure data being read has been written
\end{itemize}

\important{Careful design and handshaking protocols are required to avoid data corruption in true dual port configurations.}
\end{remark}

% TODO: Add image from SCD_2a_Memory_2.pdf, Page 12
% Description: True dual port potential problems diagram
% Priority: IMPORTANT
% Suggested filename: lecture02a_dual_port_problems.png
% \includegraphics[width=\linewidth]{lecture02a_dual_port_problems.png}

% TODO: Add image from SCD_2a_Memory_2.pdf, Page 13
% Description: True dual port problem scenarios explained
% Priority: IMPORTANT
% Suggested filename: lecture02a_dual_port_scenarios.png
% \includegraphics[width=\linewidth]{lecture02a_dual_port_scenarios.png}

\subsubsection{Triple Port Mode}

\begin{concept}{Triple Port Memory}\\
Triple port memory provides one write port and two independent read ports.

\textbf{Port Configuration:}
\begin{itemize}
    \item \textbf{Write Port:} Single write address and data input
    \item \textbf{Read Port A:} Independent read address and data output
    \item \textbf{Read Port B:} Independent read address and data output
\end{itemize}

\textbf{Implementation:}
The synthesis tool implements triple port memory using two dual-port RAM blocks:
\begin{itemize}
    \item Both blocks receive same write data
    \item Each read port connects to separate RAM block
    \item Ensures data consistency across read ports
\end{itemize}

\textbf{Synthesis Result:}
\begin{itemize}
    \item \texttt{altdpram:altdpram\_component1} - First dual-port RAM
    \item \texttt{altdpram:altdpram\_component2} - Second dual-port RAM
    \item Shared write interface
    \item Independent read interfaces
\end{itemize}
\end{concept}

% TODO: Add image from SCD_2a_Memory_2.pdf, Page 14
% Description: Triple port memory block diagram and synthesis result
% Priority: IMPORTANT
% Suggested filename: lecture02a_triple_port.png
% \includegraphics[width=\linewidth]{lecture02a_triple_port.png}

\subsection{Different Bus Widths}

\begin{concept}{Utilizing Different Bus Widths}\\
FPGA memory blocks support different data widths on read and write ports.

\textbf{Key Principle:}
Both sides must cover the full memory space:
$$\text{Number of words} \times \text{Word size} = \text{Capacity}$$

\textbf{Relationship:}
\begin{itemize}
    \item \textbf{Larger word size} $\Rightarrow$ Smaller address width
    \item \textbf{Smaller word size} $\Rightarrow$ Larger address width
\end{itemize}

\textbf{Example Configuration:}
\begin{itemize}
    \item \textbf{Write Port:} 8-bit data width
    \item \textbf{Read Port:} 32-bit data width
    \item Ratio: 1:4 (read width is 4× write width)
    \item Read address is 2 bits narrower than write address
\end{itemize}

\textbf{Use Case:}
Useful when FPGA logic needs word size larger than bus width of processor.
\end{concept}

% TODO: Add image from SCD_2a_Memory_2.pdf, Page 16
% Description: Different bus width configuration diagram
% Priority: CRITICAL
% Suggested filename: lecture02a_different_bus_widths.png
% \includegraphics[width=\linewidth]{lecture02a_different_bus_widths.png}

\begin{example2}{Exercise: Different Word Widths}\\
\textbf{Given:} Simple dual-port memory with different input and output word widths.

\textbf{Task:} Calculate the width of both read and write address signals.

\textbf{Specifications:}
\begin{itemize}
    \item \textbf{Capacity:} 16,384 bits = 16 Kbit
    \item \textbf{Write data width:} 16 bits
    \item \textbf{Read data width:} 256 bits
\end{itemize}

\tcblower

\textbf{Solution:}

\textbf{Write Side:}
\begin{align*}
\text{Number of words} &= \frac{16,384 \text{ bits}}{16 \text{ bits/word}} = 1,024 \text{ words} \\
\text{Address bits} &= \log_2(1,024) = 10 \text{ bits}
\end{align*}

\textbf{Read Side:}
\begin{align*}
\text{Number of words} &= \frac{16,384 \text{ bits}}{256 \text{ bits/word}} = 64 \text{ words} \\
\text{Address bits} &= \log_2(64) = 6 \text{ bits}
\end{align*}

\textbf{Answer:}
\begin{itemize}
    \item Write address: \texttt{address[9..0]} (10 bits)
    \item Read address: \texttt{rdaddress[5..0]} (6 bits)
\end{itemize}
\end{example2}

% TODO: Add image from SCD_2a_Memory_2.pdf, Page 21
% Description: Exercise problem diagram
% Priority: SUPPLEMENTARY
% Suggested filename: lecture02a_exercise_problem.png
% \includegraphics[width=\linewidth]{lecture02a_exercise_problem.png}

% TODO: Add image from SCD_2a_Memory_2.pdf, Page 22
% Description: Exercise solution diagram
% Priority: IMPORTANT
% Suggested filename: lecture02a_exercise_solution.png
% \includegraphics[width=\linewidth]{lecture02a_exercise_solution.png}

\raggedcolumns
\columnbreak

\subsection{Clock Domain Crossing}

\begin{concept}{Separated Clock Domains}\\
Memory blocks can operate with different clock domains on input and output.

\textbf{Configuration Options:}

\textbf{Option 1: Input/Output Clock Separation}
\begin{itemize}
    \item Inputs controlled by \texttt{inclock}
    \item Outputs controlled by \texttt{outclock}
    \item Data written in one domain, consumed in another
    \item Both clocks can be related or independent
\end{itemize}

\textbf{Option 2: Write/Read Clock Separation}
\begin{itemize}
    \item Writing controlled by \texttt{wrclock}
    \item Reading controlled by \texttt{rdclock}
    \item Completely independent writer and reader
    \item Asynchronous operation between domains
\end{itemize}

\important{Clock domain crossing requires careful design to avoid metastability and data corruption.}
\end{concept}

% TODO: Add image from SCD_2a_Memory_2.pdf, Page 17
% Description: Clock domain separation configurations
% Priority: CRITICAL
% Suggested filename: lecture02a_clock_domains.png
% \includegraphics[width=\linewidth]{lecture02a_clock_domains.png}

\begin{example2}{Clock Crossing Bridge Using FIFO}\\
\textbf{Application:} Audio data buffering with clock domain crossing

\textbf{Configuration:}
\begin{itemize}
    \item \textbf{Write Clock (WR\_CLK):} 12.288 MHz (I2S audio clock)
    \item \textbf{Read Clock (RD\_CLK):} 50 MHz (system clock)
    \item \textbf{FIFO Depth:} 256 words
    \item \textbf{Write Enable:} Controlled by I2S protocol
    \item \textbf{Data Path:} Serial audio data input via I2S
\end{itemize}

\tcblower

\textbf{Operation:}
\begin{itemize}
    \item Input on fast clock, but sparse writing events (every 256 cycles)
    \item Output on slow clock, consuming data every cycle
    \item Control logic provided by FIFO soft IP block or custom developer logic
    \item Some delay introduced by FIFO (typically a few clock cycles)
\end{itemize}

\textbf{Benefits:}
\begin{itemize}
    \item Safe data transfer between clock domains
    \item Rate buffering and flow control
    \item Prevents data loss due to clock frequency mismatch
\end{itemize}
\end{example2}

% TODO: Add image from SCD_2a_Memory_2.pdf, Page 18
% Description: FIFO clock crossing bridge diagram with I2S
% Priority: CRITICAL
% Suggested filename: lecture02a_fifo_clock_crossing.png
% \includegraphics[width=\linewidth]{lecture02a_fifo_clock_crossing.png}

\subsection{Byte Enable Feature}

\begin{definition}{Byte Enable}\\
Byte enable allows writing individual bytes of wider data inputs.

\textbf{Signal:} \texttt{byteena[3..0]} for 32-bit data

\textbf{Functionality:}
\begin{itemize}
    \item Each bit in \texttt{byteena} controls one byte of data
    \item \texttt{byteena[0] = '1'} enables writing of data[7..0]
    \item \texttt{byteena[1] = '1'} enables writing of data[15..8]
    \item \texttt{byteena[2] = '1'} enables writing of data[23..16]
    \item \texttt{byteena[3] = '1'} enables writing of data[31..24]
\end{itemize}

\textbf{Use Cases:}
\begin{itemize}
    \item Partial word updates
    \item Byte-addressable memory emulation
    \item Efficient memory usage in mixed-width systems
\end{itemize}

\textbf{Alternative:} Multiple write operations with address manipulation
\end{definition}

% TODO: Add image from SCD_2a_Memory_2.pdf, Page 19
% Description: Byte enable feature question
% Priority: SUPPLEMENTARY
% Suggested filename: lecture02a_byte_enable_intro.png
% \includegraphics[width=\linewidth]{lecture02a_byte_enable_intro.png}

% TODO: Add image from SCD_2a_Memory_2.pdf, Page 20
% Description: Byte enable configuration in IP
% Priority: IMPORTANT
% Suggested filename: lecture02a_byte_enable_config.png
% \includegraphics[width=\linewidth]{lecture02a_byte_enable_config.png}

\raggedcolumns
\columnbreak

\subsection{VHDL Implementation}

\begin{concept}{Why Use VHDL for RAM Blocks?}\\
Advantages of describing RAM blocks in VHDL:
\begin{itemize}
    \item \textbf{Platform Independence:} Code portable across FPGA families
    \item \textbf{Easier Customization:} Modify parameters without GUI
    \item \textbf{Reusability:} Parameterized designs for multiple uses
    \item \textbf{Scriptable:} Automate generation and configuration
    \item \textbf{Simulation:} Easier to simulate and verify behavior
\end{itemize}
\end{concept}

\begin{definition}{VHDL Array Type}\\
Arrays in VHDL allow defining memory structures.

\textbf{Syntax:}
\begin{lstlisting}[language=VHDL, style=base]
type <type_name> is array (<low> to <high>) of <data_type>;
\end{lstlisting}

\textbf{Example:}
\begin{lstlisting}[language=VHDL, style=base]
type 64b_ram is array(0 to 7) of std_logic_vector(7 downto 0);
\end{lstlisting}

This defines an array of 8 elements, each 8 bits wide, creating a 64-bit RAM structure.

\textbf{Memory Layout:}
\begin{itemize}
    \item Index 0: \texttt{00000000}
    \item Index 1: \texttt{00010001}
    \item Index 2: \texttt{00100010}
    \item Index 3: \texttt{00110011}
    \item Index 4: \texttt{01000100}
    \item Index 5: \texttt{01010101}
    \item Index 6: \texttt{01100110}
    \item Index 7: \texttt{01110111}
\end{itemize}
\end{definition}

% TODO: Add image from SCD_2a_Memory_2.pdf, Page 25
% Description: VHDL array type memory layout visualization
% Priority: IMPORTANT
% Suggested filename: lecture02a_vhdl_array.png
% \includegraphics[width=\linewidth]{lecture02a_vhdl_array.png}

\begin{code}{Generic RAM in VHDL}\\
Complete VHDL code for a parameterized RAM block:

\begin{lstlisting}[language=VHDL, style=basesmol]
library ieee;
use ieee.std_logic_1164.all;
use ieee.numeric_std.all;
use work.all;

entity ram_generic is
  generic (
    RAM_WIDTH : natural := 10;  -- Port width
    RAM_DEPTH : natural := 12); -- Address width
  port (
    clock   : in  std_logic;
    address : in  std_logic_vector(RAM_DEPTH-1 downto 0);
    wren    : in  std_logic;
    data    : in  std_logic_vector(RAM_WIDTH-1 downto 0);
    q       : out std_logic_vector(RAM_WIDTH-1 downto 0));
end ram_generic;

architecture rtl of ram_generic is
  -- Define memory type: array of words
  type mem is array(0 to (2**RAM_DEPTH)-1) of 
    std_logic_vector(RAM_WIDTH-1 downto 0);
  signal ram_block : mem;
begin
  process(all)
  begin
    if rising_edge(clock) then
      if (wren = '1') then
        -- Write operation
        ram_block(to_integer(unsigned(address))) <= data;
      end if;
      -- Read operation (synchronous)
      q <= ram_block(to_integer(unsigned(address)));
    end if;
  end process;
end rtl;
\end{lstlisting}

\textbf{Key Features:}
\begin{itemize}
    \item Generic parameters for width and depth
    \item Synchronous read and write
    \item Type conversion: \texttt{to\_integer(unsigned(address))}
    \item Infers block RAM in synthesis
\end{itemize}
\end{code}

% TODO: Add image from SCD_2a_Memory_2.pdf, Page 26
% Description: Complete VHDL RAM code
% Priority: CRITICAL
% Suggested filename: lecture02a_vhdl_ram_code.png
% \includegraphics[width=\linewidth]{lecture02a_vhdl_ram_code.png}

\begin{example}
\textbf{Synthesis Result in Cyclone IV:}

The VHDL code synthesizes to:
\begin{itemize}
    \item \textbf{SYNC\_RAM block:} Inferred synchronous RAM
    \item \textbf{Output registers:} \texttt{q[0]\textasciitilde reg[9..0]}
    \item \textbf{Inputs:} clock, data[9..0], address[11..0], wren
\end{itemize}

The synthesizer recognizes the RAM pattern and maps it to dedicated hardware blocks.
\end{example}

% TODO: Add image from SCD_2a_Memory_2.pdf, Page 27
% Description: Synthesis result schematic
% Priority: IMPORTANT
% Suggested filename: lecture02a_synthesis_result.png
% \includegraphics[width=\linewidth]{lecture02a_synthesis_result.png}

\raggedcolumns
\columnbreak

\subsection{IP Catalog Configuration}

\begin{KR}{Configuring Memory Using IP Catalog}\\
\textbf{Step 1: Open IP Catalog}
\begin{itemize}
    \item In Quartus, navigate to Tools $\rightarrow$ IP Catalog
    \item Select device family: Cyclone V (E/GX/GT/SX/SE/ST)
\end{itemize}

\textbf{Step 2: Select Memory Type}
\begin{itemize}
    \item Navigate to: Library $\rightarrow$ Basic Functions $\rightarrow$ On Chip Memory
    \item Choose from available options:
    \begin{itemize}
        \item FIFO
        \item RAM Initializer
        \item RAM: 1-PORT
        \item RAM: 2-PORT
        \item ROM: 1-PORT
        \item ROM: 2-PORT
        \item Shift register (RAM-based)
    \end{itemize}
\end{itemize}

\textbf{Step 3: Configure Memory Parameters}
See detailed configuration options below.
\end{KR}

% TODO: Add image from SCD_2a_Memory_2.pdf, Page 29
% Description: IP Catalog navigation screenshot
% Priority: IMPORTANT
% Suggested filename: lecture02a_ip_catalog.png
% \includegraphics[width=\linewidth]{lecture02a_ip_catalog.png}

\subsubsection{Data Bus Width and Depth Configuration}

\begin{KR}{Configuring Width and Depth}\\
\textbf{Configuration Parameters:}

\textbf{Width Configuration:}
\begin{itemize}
    \item Question: "How wide should the 'q' output bus be?"
    \item Example: 16 bits
    \item Arbitrary values allowed
\end{itemize}

\textbf{Depth Configuration:}
\begin{itemize}
    \item Question: "How many 16-bit words of memory?"
    \item Example: 65,536 words
    \item Results in: 96 bits + 128 M10K + 3 registers
\end{itemize}

\textbf{Memory Block Type:}
\begin{itemize}
    \item \textbf{Auto:} Let toolchain decide optimal implementation
    \item \textbf{MLAB:} Force use of Memory LAB (small distributed RAM)
    \item \textbf{M10K:} Force use of M10K blocks
\end{itemize}

\textbf{Maximum Block Depth:}
\begin{itemize}
    \item Set to Auto for optimal packing
    \item Can specify manual limit if needed
\end{itemize}

\textbf{Clocking Method:}
\begin{itemize}
    \item \textbf{Single clock:} One clock for all operations
    \item \textbf{Dual clock:} Separate input and output clocks
\end{itemize}
\end{KR}

% TODO: Add image from SCD_2a_Memory_2.pdf, Page 30
% Description: Width and depth configuration screen
% Priority: CRITICAL
% Suggested filename: lecture02a_width_depth_config.png
% \includegraphics[width=\linewidth]{lecture02a_width_depth_config.png}

\subsubsection{Port and Enable Configuration}

\begin{KR}{Configuring Ports and Enables}\\
\textbf{Registered Ports:}
\begin{itemize}
    \item $\square$ 'data' and 'wren' input ports
    \item $\square$ 'address' input port
    \item $\square$ 'q' output port (recommended for high performance)
\end{itemize}

\textbf{Clock Enable:}
\begin{itemize}
    \item $\square$ Create one clock enable signal for each clock
    \item All registered ports controlled by enable signal
\end{itemize}

\textbf{Byte Enable:}
\begin{itemize}
    \item $\square$ Create byte enable for port A
    \item Specify byte width (typically 8 bits)
\end{itemize}

\textbf{Asynchronous Clear:}
\begin{itemize}
    \item $\square$ Create 'aclr' asynchronous clear for registered ports
\end{itemize}

\textbf{Read Enable:}
\begin{itemize}
    \item $\square$ Create 'rden' read enable signal (optional)
\end{itemize}

\important{Registering the 'q' output improves timing performance but adds one cycle latency.}
\end{KR}

% TODO: Add image from SCD_2a_Memory_2.pdf, Page 31
% Description: Port configuration screen
% Priority: CRITICAL
% Suggested filename: lecture02a_port_config.png
% \includegraphics[width=\linewidth]{lecture02a_port_config.png}

\subsubsection{Read-During-Write Configuration}

\begin{definition}{Read-During-Write Behavior}\\
Defines output behavior when reading from a location being written to.

\textbf{Options:}
\begin{itemize}
    \item \textbf{New Data:} Output shows newly written data
    \item \textbf{Old Data:} Output shows data before write
    \item \textbf{Don't Care:} Output is undefined (X in simulation)
\end{itemize}

\textbf{Byte Enable Interaction:}
\begin{itemize}
    \item $\square$ Get X's for write-masked bytes instead of old data
    \item Applies when byte enable is used
    \item Helps identify unintended read-during-write conditions
\end{itemize}

\important{Read-during-write behavior is crucial for avoiding data hazards in memory access patterns.}
\end{definition}

% TODO: Add image from SCD_2a_Memory_2.pdf, Page 32
% Description: Read-during-write configuration
% Priority: IMPORTANT
% Suggested filename: lecture02a_read_during_write.png
% \includegraphics[width=\linewidth]{lecture02a_read_during_write.png}

\subsubsection{Memory Initialization}

\begin{KR}{Initializing Memory Contents}\\
\textbf{Question:} "Do you want to specify the initial content of the memory?"

\textbf{Options:}

\textbf{Option 1: Blank Memory}
\begin{itemize}
    \item $\circ$ No, leave it blank
    \item Memory contains undefined values at power-up
\end{itemize}

\textbf{Option 2: Simulation-Only Init}
\begin{itemize}
    \item $\circ$ Yes, but leave it blank in simulation
    \item Initialize for hardware, but not in simulation
\end{itemize}

\textbf{Option 3: File-Based Init}
\begin{itemize}
    \item $\circ$ Yes, use this file for memory content data
    \item File formats:
    \begin{itemize}
        \item Hexadecimal (Intel-format) File [.hex]
        \item Memory Initialization File [.mif]
    \end{itemize}
    \item Specify filename and path
    \item Select which port's dimensions to match (Port A or Port B)
\end{itemize}

\textbf{Advanced Feature:}
\begin{itemize}
    \item $\square$ Use Memory Content Editor
    \item Capture and update content independently of system clock
    \item Requires Instance ID assignment
\end{itemize}
\end{KR}

% TODO: Add image from SCD_2a_Memory_2.pdf, Page 33
% Description: Memory initialization configuration
% Priority: IMPORTANT
% Suggested filename: lecture02a_memory_init.png
% \includegraphics[width=\linewidth]{lecture02a_memory_init.png}

\subsubsection{Synthesis and Simulation Support}

\begin{remark}
\textbf{Simulation Libraries:}

To properly simulate the generated design, the following simulation model files are needed:
\begin{itemize}
    \item \textbf{altera\_mf:} Altera megafunction simulation library
\end{itemize}

\textbf{Timing and Resource Estimation:}
\begin{itemize}
    \item $\square$ Generate netlist for timing and resource estimation
    \item Enables better design optimization in third-party synthesis tools
    \item Check with tool vendor for support information
    \item Note: Netlist generation can be time-intensive
    \item Process time depends on design size and system speed
\end{itemize}

\important{Not all third-party synthesis tools support netlist-based estimation. Verify compatibility before enabling this feature.}
\end{remark}

% TODO: Add image from SCD_2a_Memory_2.pdf, Page 34
% Description: Synthesis and simulation configuration
% Priority: SUPPLEMENTARY
% Suggested filename: lecture02a_synthesis_sim.png
% \includegraphics[width=\linewidth]{lecture02a_synthesis_sim.png}

% ===== IMAGE SUMMARY =====
% Total images needed: 24
% CRITICAL priority: 8
% IMPORTANT priority: 12
% SUPPLEMENTARY priority: 4
%
% Quick extraction checklist:
% [ ] [SCD_2a_Memory_2.pdf, Page 5] - Memory configurations table (IMPORTANT)
% [ ] [SCD_2a_Memory_2.pdf, Page 6] - ROM mode block diagram (IMPORTANT)
% [ ] [SCD_2a_Memory_2.pdf, Page 7] - Intel HEX format structure (IMPORTANT)
% [ ] [SCD_2a_Memory_2.pdf, Page 8] - Single port RAM diagram (CRITICAL)
% [ ] [SCD_2a_Memory_2.pdf, Page 9] - Chip Planner single port view (SUPPLEMENTARY)
% [ ] [SCD_2a_Memory_2.pdf, Page 10] - Simple dual port diagram (CRITICAL)
% [ ] [SCD_2a_Memory_2.pdf, Page 11] - True dual port diagram (CRITICAL)
% [ ] [SCD_2a_Memory_2.pdf, Page 12] - Dual port problems (IMPORTANT)
% [ ] [SCD_2a_Memory_2.pdf, Page 13] - Dual port scenarios (IMPORTANT)
% [ ] [SCD_2a_Memory_2.pdf, Page 14] - Triple port memory (IMPORTANT)
% [ ] [SCD_2a_Memory_2.pdf, Page 16] - Different bus widths (CRITICAL)
% [ ] [SCD_2a_Memory_2.pdf, Page 17] - Clock domain separation (CRITICAL)
% [ ] [SCD_2a_Memory_2.pdf, Page 18] - FIFO clock crossing (CRITICAL)
% [ ] [SCD_2a_Memory_2.pdf, Page 19] - Byte enable intro (SUPPLEMENTARY)
% [ ] [SCD_2a_Memory_2.pdf, Page 20] - Byte enable config (IMPORTANT)
% [ ] [SCD_2a_Memory_2.pdf, Page 21] - Exercise problem (SUPPLEMENTARY)
% [ ] [SCD_2a_Memory_2.pdf, Page 22] - Exercise solution (IMPORTANT)
% [ ] [SCD_2a_Memory_2.pdf, Page 25] - VHDL array visualization (IMPORTANT)
% [ ] [SCD_2a_Memory_2.pdf, Page 26] - VHDL RAM code (CRITICAL)
% [ ] [SCD_2a_Memory_2.pdf, Page 27] - Synthesis result (IMPORTANT)
% [ ] [SCD_2a_Memory_2.pdf, Page 29] - IP Catalog navigation (IMPORTANT)
% [ ] [SCD_2a_Memory_2.pdf, Page 30] - Width/depth config (CRITICAL)
% [ ] [SCD_2a_Memory_2.pdf, Page 31] - Port configuration (CRITICAL)
% [ ] [SCD_2a_Memory_2.pdf, Page 32] - Read-during-write (IMPORTANT)
% [ ] [SCD_2a_Memory_2.pdf, Page 33] - Memory initialization (IMPORTANT)
% [ ] [SCD_2a_Memory_2.pdf, Page 34] - Synthesis/simulation (SUPPLEMENTARY)
% =====================