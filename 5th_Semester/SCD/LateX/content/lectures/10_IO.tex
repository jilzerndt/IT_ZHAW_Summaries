\section{Inputs and Outputs (IO)}

\subsection{Connecting FPGA/SoC to Peripherals}

\begin{remark}
Key questions when connecting peripherals to FPGAs:
\begin{itemize}
    \item How to connect a peripheral (e.g., SDRAM) to FPGA?
    \item What are the voltage levels and currents?
    \item Are there standards for interfacing?
\end{itemize}
\end{remark}

\subsection{FPGA I/O Overview}

\begin{definition}{FPGA I/O Topics}
\begin{itemize}
    \item Single Ended I/Os
    \item Logic Levels and standards on I/Os
    \item FPGA I/O Banks
    \item Differential I/Os
    \item LVDS (Low Voltage Differential Signaling)
    \item Open Drain Outputs (I2C)
    \item Bus Hold
    \item Configuration of I/Os in FPGA
\end{itemize}
\end{definition}

\subsection{Simplified Single Ended CMOS Output/Input Stage}

\begin{concept}{Push-Pull Configuration}\\
Single-ended CMOS I/O uses a push-pull configuration with two complementary transistors.
\end{concept}

% TODO: Add image from SCD_10_io_1.pdf, Slide 4
% Description: Simplified CMOS push-pull output/input stage diagram from CT2
% Priority: CRITICAL
% Suggested filename: lecture10_pushpull_basic.png
\\
% \includegraphics[width=0.8\linewidth]{lecture10_pushpull_basic.png}
\\
% WHEN YOU ADD IMAGE: Uncomment line above (remove %)

\mult{2}

\begin{definition}{Input Stage}\\
\textbf{Impedance:} Very high impedance ($> 1$ M$\Omega$)

The input stage presents minimal load to the driving circuit.
\end{definition}

\begin{definition}{Output Stage}\\
\textbf{Impedance:} Very low impedance ($< 20$ $\Omega$)

The output stage can source or sink substantial current.
\end{definition}

\multend

% TODO: Add image from SCD_10_io_1.pdf, Slide 5
% Description: CMOS input/output stage with impedance specifications and transistor details
% Priority: CRITICAL
% Suggested filename: lecture10_cmos_io_stage.png
\\
% \includegraphics[width=\linewidth]{lecture10_cmos_io_stage.png}
\\
% WHEN YOU ADD IMAGE: Uncomment line above (remove %)

\subsection{MOS-FET Transistor Principles}

\paragraph{P-Channel MOS-FET}

\begin{definition}{P-Channel MOS-FET Transistor}\\
\textbf{MOS-FET:} Metal Oxide Semiconductor Field Effect Transistor

\textbf{Operation:}
\begin{itemize}
    \item Gate-to-Source Voltage ($V_{GS}$) controls Drain-to-Source Current ($I_{DS}$)
    \item Threshold voltage: $-V_{TP}$ (negative for P-channel)
    \item Conducts when $V_{GS} < -V_{TP}$
\end{itemize}
\end{definition}

% TODO: Add image from SCD_10_io_1.pdf, Slide 6
% Description: P-Channel MOS-FET transistor diagram with voltage/current characteristics
% Priority: CRITICAL
% Suggested filename: lecture10_pchannel_mosfet.png
\\
% \includegraphics[width=\linewidth]{lecture10_pchannel_mosfet.png}
\\
% WHEN YOU ADD IMAGE: Uncomment line above (remove %)

\paragraph{N-Channel MOS-FET}

\begin{definition}{N-Channel MOS-FET Transistor}\\
\textbf{Operation:}
\begin{itemize}
    \item Gate-to-Source Voltage ($V_{GS}$) controls Drain-to-Source Current ($I_{DS}$)
    \item Threshold voltage: $+V_{TN}$ (positive for N-channel)
    \item Conducts when $V_{GS} > +V_{TN}$
\end{itemize}
\end{definition}

% TODO: Add image from SCD_10_io_1.pdf, Slide 7
% Description: N-Channel MOS-FET transistor diagram with voltage/current characteristics
% Priority: CRITICAL
% Suggested filename: lecture10_nchannel_mosfet.png
\\
% \includegraphics[width=\linewidth]{lecture10_nchannel_mosfet.png}
\\
% WHEN YOU ADD IMAGE: Uncomment line above (remove %)

\begin{remark}
\textbf{Voltage Nomenclature in MOS-FET Circuits:}
\begin{itemize}
    \item $V_{DD}$: Supply voltage at the Drain
    \item $V_{SS}$: Ground voltage at the Source (sometimes referred to as GND)
\end{itemize}
\end{remark}

\subsection{Output Circuit Operation}

\paragraph{N-Channel Transistor Operation}

\begin{concept}{N-Channel Switching Behavior}\\
When the input exceeds the N-threshold ($V_{TN}$), the N-channel transistor turns ON, pulling the output to GND ($0$V).

\textbf{States:}
\begin{enumerate}
    \item Input $< V_{TN}$ $\rightarrow$ N-channel OFF $\rightarrow$ Output HIGH
    \item Input $> V_{TN}$ $\rightarrow$ N-channel ON $\rightarrow$ Output LOW (GND)
\end{enumerate}
\end{concept}

% TODO: Add image from SCD_10_io_1.pdf, Slide 9
% Description: Output circuit showing N-channel transistor operation with timing diagrams
% Priority: CRITICAL
% Suggested filename: lecture10_nchannel_operation.png
\\
% \includegraphics[width=\linewidth]{lecture10_nchannel_operation.png}
\\
% WHEN YOU ADD IMAGE: Uncomment line above (remove %)

\paragraph{P-Channel Transistor Operation}

\begin{concept}{P-Channel Switching Behavior}\\
When the input falls below the P-threshold ($V_{DD} - V_{TP}$), the P-channel transistor turns ON, pulling the output to $V_{DD}$.

\textbf{States:}
\begin{enumerate}
    \item Input $> V_{DD} - V_{TP}$ $\rightarrow$ P-channel OFF $\rightarrow$ Output LOW
    \item Input $< V_{DD} - V_{TP}$ $\rightarrow$ P-channel ON $\rightarrow$ Output HIGH ($V_{DD}$)
\end{enumerate}
\end{concept}

% TODO: Add image from SCD_10_io_1.pdf, Slide 10
% Description: Output circuit showing P-channel transistor operation with timing diagrams
% Priority: CRITICAL
% Suggested filename: lecture10_pchannel_operation.png
\\
% \includegraphics[width=\linewidth]{lecture10_pchannel_operation.png}
\\
% WHEN YOU ADD IMAGE: Uncomment line above (remove %)

\begin{concept}{Combined Push-Pull Operation}\\
Both transistors work complementarily:
\begin{itemize}
    \item Low input: P-channel ON, N-channel OFF $\rightarrow$ Output HIGH
    \item High input: P-channel OFF, N-channel ON $\rightarrow$ Output LOW
    \item Only one transistor is ON at any time (ideally no short circuit)
\end{itemize}
\end{concept}

% TODO: Add image from SCD_10_io_1.pdf, Slide 11
% Description: Complete push-pull operation showing both transistors with timing diagrams
% Priority: CRITICAL
% Suggested filename: lecture10_pushpull_operation.png
\\
% \includegraphics[width=\linewidth]{lecture10_pushpull_operation.png}
\\
% WHEN YOU ADD IMAGE: Uncomment line above (remove %)

\subsection{High Performance vs. Low Power Transistors}

\begin{definition}{Semiconductor Process Differences}\\
\textbf{High Performance Transistor:}
\begin{itemize}
    \item Lower threshold voltage ($V_T$)
    \item Faster switching (steeper edge transitions)
    \item Higher current consumption
    \item Thresholds closer together
\end{itemize}

\textbf{Low Power Transistor:}
\begin{itemize}
    \item Higher threshold voltage ($V_T$)
    \item Slower switching (less steep edge transitions)
    \item Lower current consumption
    \item Thresholds further apart
\end{itemize}
\end{definition}

% TODO: Add image from SCD_10_io_1.pdf, Slide 12
% Description: Comparison diagram showing threshold voltages for high performance vs low power transistors
% Priority: IMPORTANT
% Suggested filename: lecture10_hp_vs_lp_transistors.png
\\
% \includegraphics[width=\linewidth]{lecture10_hp_vs_lp_transistors.png}
\\
% WHEN YOU ADD IMAGE: Uncomment line above (remove %)

\raggedcolumns
\columnbreak

\subsection{Technology Scaling and Voltage Levels}

\begin{concept}{Transistor Miniaturization}\\
As transistors become smaller, the threshold voltage decreases. The threshold must be approximately $\frac{1}{2}$ of the supply voltage ($V_{DD}$).

\textbf{Relationship:}
$$V_T \approx \frac{V_{DD}}{2}$$

Example gate size: $90$ nm
\end{concept}

% TODO: Add image from SCD_10_io_1.pdf, Slide 13
% Description: Cross-section of CMOS circuit showing transistor dimensions
% Priority: IMPORTANT
% Suggested filename: lecture10_cmos_cross_section.png
\\
% \includegraphics[width=\linewidth]{lecture10_cmos_cross_section.png}
\\
% WHEN YOU ADD IMAGE: Uncomment line above (remove %)

\begin{concept}{Scaling Trend}\\
With each generation of CMOS process, the supply voltage ($V_{DD}$) decreases because the threshold needs to be approximately half of the supply voltage.

\textbf{Example scaling:}
\begin{itemize}
    \item Threshold level $\sim 1.65$ V $\rightarrow$ $V_{DD} = 3.3$ V
    \item Threshold level $\sim 0.9$ V $\rightarrow$ $V_{CC} = 1.8$ V
\end{itemize}
\end{concept}

% TODO: Add image from SCD_10_io_1.pdf, Slide 14
% Description: Diagram showing relationship between threshold voltage and supply voltage for different process nodes
% Priority: CRITICAL
% Suggested filename: lecture10_threshold_scaling.png
\\
% \includegraphics[width=\linewidth]{lecture10_threshold_scaling.png}
\\
% WHEN YOU ADD IMAGE: Uncomment line above (remove %)

\begin{highlight}{Technology Trend Table}\\
\begin{center}
\begin{tabular}{|c|c|c|}
\hline
\textbf{Market Introduction} & \textbf{Process (Gate Size)} & \textbf{$V_{DD}$ (Core Voltage)} \\
\hline
1994 & 0.50 $\mu$m & 3.3 V \\
1998 & 0.25 $\mu$m & 2.5 V \\
2002 & 0.18 $\mu$m & 1.8 V \\
2004 & 90 nm & 1.2 V \\
2006 & 65 nm & 1.2 V \\
2013 & 28 nm & 1.1 V \\
2015 & 14 nm & 0.9 V \\
2019 & 7 nm & 0.8 V \\
2021 & 5 nm & 0.75 V \\
2022 & 3 nm & 0.7 V \\
\hline
\end{tabular}
\end{center}

\important{Note:} As process nodes shrink, both gate size and supply voltage decrease continuously.
\end{highlight}

\subsection{Level Shifters for I/O Pads}

\begin{concept}{Voltage Domain Separation}\\
Modern FPGAs have different voltage domains:
\begin{itemize}
    \item Core logic operates at low voltage (e.g., $V_{DDINT} = 0.7$ V)
    \item I/O pads operate at higher voltages (e.g., $V_{DDIO} = 1.2, 1.5, 1.8, 2.5, 3.3$ V)
\end{itemize}

\textbf{Challenge:} How can we drive an I/O with $V_{DD} = 3.3$ V with a core voltage of $0.7$ V?

\textbf{Solution:} Level shifters are required between internal and external signals.
\end{concept}

% TODO: Add image from SCD_10_io_1.pdf, Slide 16
% Description: FPGA block diagram showing core logic, level shifter, and I/O pads with voltage levels
% Priority: CRITICAL
% Suggested filename: lecture10_level_shifter_overview.png
\\
% \includegraphics[width=\linewidth]{lecture10_level_shifter_overview.png}
\\
% WHEN YOU ADD IMAGE: Uncomment line above (remove %)

% TODO: Add image from SCD_10_io_1.pdf, Slide 17
% Description: Detailed level shifter implementation between FPGA core and I/O
% Priority: IMPORTANT
% Suggested filename: lecture10_level_shifter_detail.png
\\
% \includegraphics[width=\linewidth]{lecture10_level_shifter_detail.png}
\\
% WHEN YOU ADD IMAGE: Uncomment line above (remove %)

\raggedcolumns
\columnbreak

\subsection{Transmission Standards and Applications}

\begin{highlight}{Common I/O Standards}\\
\begin{center}
\begin{adjustbox}{max width=\linewidth}
\begin{tabular}{|l|l|l|l|}
\hline
\textbf{Standard} & \textbf{Comment} & \textbf{Application} & \textbf{Comment} \\
\hline
LVTTL, LVCMOS & Low Voltage TTL/CMOS & General Purpose & Inter-FPGA connection \\
\hline
LVDS, Mini-LVDS & Low Voltage Differential Signaling & SGMII, SFI, SPI & Ethernet MAC to PHY \\
LVPECL & Low Voltage Pseudo ECL & & \\
\hline
LVDS & Low Voltage Differential Signaling & LCD Panels & Display interface \\
\hline
SSTL-18 & Stub Series Terminated Logic (1.8V) & DDR2 & DDRAM2 \\
\hline
SSTL-15 & Stub Series Terminated Logic (1.5V) & DDR3 & DDRAM3 \\
\hline
POD-12 & Pseudo Open Drain (1.2V) & DDR4 & DDRAM4 \\
\hline
HSUL-12 & High Speed Unterminated Logic & LPDDR2 & Low Power DDR2 \\
\hline
PECL & Pseudo ECL & SDI & Serial Digital Interface (Video) \\
\hline
HSTL & High Speed Transceiver Logic & DDR2 & High-speed memory \\
\hline
RSDS & Reduced Swing Differential Signaling & SGMII, SFI & Ethernet, Serial Fiber \\
\hline
\end{tabular}
\end{adjustbox}
\end{center}
\end{highlight}

% TODO: Add image from SCD_10_io_1.pdf, Slide 19
% Description: Comprehensive table of I/O standards supported by FPGAs
% Priority: IMPORTANT
% Suggested filename: lecture10_io_standards_table1.png
\\
% \includegraphics[width=\linewidth]{lecture10_io_standards_table1.png}
\\
% WHEN YOU ADD IMAGE: Uncomment line above (remove %)

% TODO: Add image from SCD_10_io_1.pdf, Slide 20
% Description: Additional I/O standards table showing voltage levels and specifications
% Priority: IMPORTANT
% Suggested filename: lecture10_io_standards_table2.png
\\
% \includegraphics[width=\linewidth]{lecture10_io_standards_table2.png}
\\
% WHEN YOU ADD IMAGE: Uncomment line above (remove %)

\subsection{LVCMOS Specifications}

\paragraph{LVCMOS Outputs}

\begin{definition}{LVCMOS 3.3V Output Levels}\\
For $V_{DD} = 3.3$ V:
\begin{itemize}
    \item \textbf{Output Low:} $< 0.2$ V
    \item \textbf{Output High:} $> 3.1$ V
\end{itemize}

The output does not go below a minimum voltage at HIGH and does not go above a maximum voltage at LOW.
\end{definition}

% TODO: Add image from SCD_10_io_1.pdf, Slide 21
% Description: LVCMOS 3.3V output voltage specifications diagram
% Priority: CRITICAL
% Suggested filename: lecture10_lvcmos33_output.png
\\
% \includegraphics[width=\linewidth]{lecture10_lvcmos33_output.png}
\\
% WHEN YOU ADD IMAGE: Uncomment line above (remove %)

\begin{definition}{LVCMOS 1.8V Output Levels}\\
For $V_{DD} = 1.8$ V:
\begin{itemize}
    \item \textbf{Output Low:} $< 0.45$ V
    \item \textbf{Output High:} $> 1.35$ V
\end{itemize}
\end{definition}

% TODO: Add image from SCD_10_io_1.pdf, Slide 22
% Description: LVCMOS 1.8V output voltage specifications diagram
% Priority: CRITICAL
% Suggested filename: lecture10_lvcmos18_output.png
\\
% \includegraphics[width=\linewidth]{lecture10_lvcmos18_output.png}
\\
% WHEN YOU ADD IMAGE: Uncomment line above (remove %)

\paragraph{LVCMOS Inputs}

\begin{definition}{LVCMOS Input Levels}\\
\textbf{For 3.3V LVCMOS:}
\begin{itemize}
    \item \textbf{Input Low:} $< 0.8$ V
    \item \textbf{Input High:} $> 1.7$ V
\end{itemize}

\textbf{For 1.8V LVCMOS:}
\begin{itemize}
    \item \textbf{Input Low:} $< 0.63$ V
    \item \textbf{Input High:} $> 1.17$ V
\end{itemize}

Input must be below a maximum voltage to be accepted as LOW signal. Input must be above a minimum voltage to be accepted as HIGH signal.
\end{definition}

% TODO: Add image from SCD_10_io_1.pdf, Slide 23
% Description: LVCMOS input voltage specifications for 3.3V and 1.8V
% Priority: CRITICAL
% Suggested filename: lecture10_lvcmos_input_levels.png
\\
% \includegraphics[width=\linewidth]{lecture10_lvcmos_input_levels.png}
\\
% WHEN YOU ADD IMAGE: Uncomment line above (remove %)

\raggedcolumns
\columnbreak

\subsection{Signal Margins and Noise Immunity}

\begin{highlight}{LVCMOS Margin Comparison}\\
\textbf{Margin} means voltages in this range are accepted as low respectively high level (acceptable tolerance).

\begin{center}
\begin{tabular}{|c|c|c|c|c|c|c|}
\hline
\textbf{$V_{DD}$/V} & \textbf{Output Low/V} & \textbf{Input Low/V} & \textbf{Margin Low/V} & \textbf{Output High/V} & \textbf{Input High/V} & \textbf{Margin High/V} \\
\hline
3.3 & $< 0.2$ & $< 0.8$ & 0.6 & $> 3.1$ & $> 1.7$ & 1.4 \\
\hline
1.8 & $< 0.45$ & $< 0.63$ & 0.18 & $> 1.35$ & $> 1.17$ & 0.18 \\
\hline
\end{tabular}
\end{center}

\important{Observation:} 3.3V LVCMOS provides much larger noise margins than 1.8V LVCMOS, resulting in better noise immunity.
\end{highlight}

% TODO: Add image from SCD_10_io_1.pdf, Slide 24
% Description: Visual comparison of noise margins for 3.3V vs 1.8V LVCMOS
% Priority: CRITICAL
% Suggested filename: lecture10_lvcmos_margins.png
\\
% \includegraphics[width=\linewidth]{lecture10_lvcmos_margins.png}
\\
% WHEN YOU ADD IMAGE: Uncomment line above (remove %)

\subsection{LVCMOS vs. LVTTL}

\begin{concept}{Comparison: LVCMOS vs. LVTTL}\\
\textbf{CMOS:} Complementary Metal-Oxide-Semiconductor (MOS-FET Transistors)

\textbf{TTL:} Transistor-Transistor-Logic (Bipolar Transistors)

\textbf{3.3V LVCMOS:}
\begin{itemize}
    \item Higher margins (Low: $0.6$ V, High: $1.4$ V)
    \item Lower current consumption
    \item Higher ESD sensitivity
\end{itemize}

\textbf{3.3V LVTTL:}
\begin{itemize}
    \item Lower margins (Low: $0.35$ V, High: $0.70$ V)
    \item High current consumption
    \item Lower ESD sensitivity
\end{itemize}
\end{concept}

% TODO: Add image from SCD_10_io_1.pdf, Slide 25
% Description: Side-by-side comparison of LVCMOS and LVTTL voltage levels and characteristics
% Priority: IMPORTANT
% Suggested filename: lecture10_lvcmos_vs_lvttl.png
\\
% \includegraphics[width=\linewidth]{lecture10_lvcmos_vs_lvttl.png}
\\
% WHEN YOU ADD IMAGE: Uncomment line above (remove %)

\subsection{I/O Banks and Supply Pins}

\begin{concept}{Multiple Voltage Domains}\\
\textbf{Question:} How to provide different supply voltages to different outputs?

\textbf{Answer:} FPGAs organize I/O pads into banks, where each bank can have an independent I/O voltage ($V_{CCIO}$).
\end{concept}

% TODO: Add image from SCD_10_io_1.pdf, Slide 26
% Description: FPGA diagram showing level shifter and question about different supply voltages
% Priority: IMPORTANT
% Suggested filename: lecture10_voltage_domains_question.png
\\
% \includegraphics[width=\linewidth]{lecture10_voltage_domains_question.png}
\\
% WHEN YOU ADD IMAGE: Uncomment line above (remove %)

\begin{definition}{I/O Bank Structure}\\
FPGA I/O pads are organized into banks with separate power supplies:
\begin{itemize}
    \item \textbf{$V_{CCINT}$:} Core voltage for internal logic
    \item \textbf{$V_{CCA\_PLL}$, $V_{CCD\_PLL}$:} PLL analog and digital supplies
    \item \textbf{$V_{CCIO1}$, $V_{CCIO2}$, etc.:} I/O bank voltages (can be different for each bank)
    \item \textbf{$V_{REFB5N0}$:} Reference voltages for certain I/O standards
    \item \textbf{$GND\_PLL$:} Ground pins for PLLs
\end{itemize}

Each bank can support a different I/O standard and voltage level.
\end{definition}

% TODO: Add image from SCD_10_io_1.pdf, Slide 27
% Description: Detailed FPGA I/O bank structure with power supply pins
% Priority: CRITICAL
% Suggested filename: lecture10_io_banks.png
\\
% \includegraphics[width=\linewidth]{lecture10_io_banks.png}
\\
% WHEN YOU ADD IMAGE: Uncomment line above (remove %)

\raggedcolumns
\columnbreak

\subsection{Differential I/O}

\subsubsection{Differential Signaling Principles}

\begin{definition}{Differential Signaling}\\
Differential signaling uses a pair of signal lines for each transmission:
\begin{itemize}
    \item Two signals A and B always assume opposite polarity
    \item The receiver sees $(A - B) =$ double the voltage swing
    \item Higher voltage swing improves noise margin
\end{itemize}

\textbf{Key advantages:}
\begin{enumerate}
    \item Doubled signal amplitude at receiver
    \item Excellent noise immunity
    \item Low electromagnetic interference (EMI)
    \item High data rates possible
\end{enumerate}
\end{definition}

% TODO: Add image from SCD_10_io_1.pdf, Slide 29
% Description: Differential signaling principle with opposite polarity signals
% Priority: CRITICAL
% Suggested filename: lecture10_differential_principle.png
\\
% \includegraphics[width=\linewidth]{lecture10_differential_principle.png}
\\
% WHEN YOU ADD IMAGE: Uncomment line above (remove %)

\begin{concept}{Common Mode Noise Rejection}\\
Noise induced on the two lines (provided the lines are routed in parallel) is compensated by the differential calculation $(A - B)$.

Since both lines experience similar noise, the differential signal remains clean:
\begin{itemize}
    \item Common mode noise affects both lines equally
    \item Differential receiver subtracts the signals: noise cancels out
    \item Only the intended signal difference is detected
\end{itemize}
\end{concept}

% TODO: Add image from SCD_10_io_1.pdf, Slide 30
% Description: Illustration of common mode noise rejection in differential signaling
% Priority: CRITICAL
% Suggested filename: lecture10_common_mode_rejection.png
\\
% \includegraphics[width=\linewidth]{lecture10_common_mode_rejection.png}
\\
% WHEN YOU ADD IMAGE: Uncomment line above (remove %)

\begin{definition}{Differential Voltage Specifications}\\
The voltage swing on differential lines is specified very small (e.g., $100$ mV). The common voltage ($V_{COMM}$) has a DC offset (e.g., $2$ V).

\textbf{Formulas:}
$$V_1 = V_{COMM} - \frac{V_{DIFF}}{2}$$
$$V_2 = V_{COMM} + \frac{V_{DIFF}}{2}$$

where $V_{DIFF}$ is the differential voltage amplitude.
\end{definition}

% TODO: Add image from SCD_10_io_1.pdf, Slide 31
% Description: Differential signaling voltage levels with common mode and differential voltages
% Priority: IMPORTANT
% Suggested filename: lecture10_differential_voltages.png
\\
% \includegraphics[width=\linewidth]{lecture10_differential_voltages.png}
\\
% WHEN YOU ADD IMAGE: Uncomment line above (remove %)

\begin{concept}{Current Flow Direction}\\
Current flow direction changes between logic high and logic low. The direction depends on the standard.

\textbf{Example:}
\begin{itemize}
    \item \textbf{Logic Low:} $A = 1.95$ V, $B = 2.05$ V $\rightarrow$ $A - B = -0.10$ V
    \item \textbf{Logic High:} $A = 2.05$ V, $B = 1.95$ V $\rightarrow$ $A - B = +0.10$ V
\end{itemize}
\end{concept}

% TODO: Add image from SCD_10_io_1.pdf, Slide 32
% Description: Current flow direction for differential signaling showing logic levels
% Priority: IMPORTANT
% Suggested filename: lecture10_differential_current.png
\\
% \includegraphics[width=\linewidth]{lecture10_differential_current.png}
\\
% WHEN YOU ADD IMAGE: Uncomment line above (remove %)

\subsubsection{Signal Quality Measurement}

\begin{concept}{Eye Diagram}\\
Signal transmission quality is measured using an eye diagram:
\begin{itemize}
    \item Overlays many bit periods
    \item Opening in the center indicates good signal quality
    \item Closure indicates signal degradation, jitter, or noise
    \item Larger opening = better signal integrity
\end{itemize}
\end{concept}

% TODO: Add image from SCD_10_io_1.pdf, Slide 33
% Description: Example eye diagram for signal quality measurement
% Priority: IMPORTANT
% Suggested filename: lecture10_eye_diagram1.png
\\
% \includegraphics[width=0.8\linewidth]{lecture10_eye_diagram1.png}
\\
% WHEN YOU ADD IMAGE: Uncomment line above (remove %)

% TODO: Add image from SCD_10_io_1.pdf, Slide 34
% Description: Additional eye diagram example showing signal quality
% Priority: SUPPLEMENTARY
% Suggested filename: lecture10_eye_diagram2.png
\\
% \includegraphics[width=0.8\linewidth]{lecture10_eye_diagram2.png}
\\
% WHEN YOU ADD IMAGE: Uncomment line above (remove %)

\raggedcolumns
\columnbreak

\subsection{Low Voltage Differential Signaling (LVDS)}

\begin{definition}{LVDS Standard}\\
\textbf{Introduced:} By National Semiconductor and TI in 1994

\textbf{Principle:} Instead of a defined voltage level, a controlled current ($3.5$ mA) is transmitted.

\textbf{Operation:}
\begin{itemize}
    \item Current direction depends on signal state (high or low)
    \item A $100$ $\Omega$ resistor serves as termination
    \item Voltage drop: $\pm 350$ mV across termination resistor
    \item Very small voltage fluctuations between low and high ($700$ mV total)
    \item Polarity detection: $+350$ mV = high, $-350$ mV = low
\end{itemize}
\end{definition}

% TODO: Add image from SCD_10_io_1.pdf, Slide 35
% Description: LVDS logo or introductory diagram
% Priority: SUPPLEMENTARY
% Suggested filename: lecture10_lvds_intro.png
\\
% \includegraphics[width=0.6\linewidth]{lecture10_lvds_intro.png}
\\
% WHEN YOU ADD IMAGE: Uncomment line above (remove %)

% TODO: Add image from SCD_10_io_1.pdf, Slide 36
% Description: Detailed LVDS circuit diagram showing current sources and termination
% Priority: CRITICAL
% Suggested filename: lecture10_lvds_circuit.png
\\
% \includegraphics[width=\linewidth]{lecture10_lvds_circuit.png}
\\
% WHEN YOU ADD IMAGE: Uncomment line above (remove %)

\begin{highlight}{LVDS Characteristics}\\
\textbf{Advantages:}
\begin{itemize}
    \item Differential signal transmission for PCB and backplane buses
    \item High noise immunity due to compensation on differential lines
    \item Current controlled, limited to $3.5$ mA:
    \begin{itemize}
        \item Low energy consumption
        \item Robust against short-circuits (current is limited)
    \end{itemize}
    \item Low electromagnetic radiation
    \item Very high data transfer rates: up to $1.923$ Gbit/s
    \item Precision timing (low jitter)
\end{itemize}

\textbf{Applications:}
\begin{itemize}
    \item High resolution displays ($2048 \times 1536$ pixels)
    \item Physical Ethernet interfaces: 1000BASE-X and SGMII
    \item Hard drives: Serial ATA, SCSI, Firewire
\end{itemize}
\end{highlight}

\begin{remark}
LVDS requires \textbf{two pins per signal} (Plus and Minus), doubling the pin count compared to single-ended signaling.
\end{remark}

% TODO: Add image from SCD_10_io_1.pdf, Slide 38
% Description: Typical LVDS application showing pin requirements
% Priority: IMPORTANT
% Suggested filename: lecture10_lvds_application.png
\\
% \includegraphics[width=\linewidth]{lecture10_lvds_application.png}
\\
% WHEN YOU ADD IMAGE: Uncomment line above (remove %)

\begin{definition}{Special Differential I/Os on FPGA}\\
\textbf{LVDSCLKp, LVDSCLKn:}
\begin{itemize}
    \item Differential clock inputs for high clock speeds and low clock jitter
    \item Number of I/Os depends on the FPGA
\end{itemize}

\textbf{PLL\_OUTp, PLL\_OUTn:}
\begin{itemize}
    \item Differential clock outputs from the PLL
    \item May only be used differentially
\end{itemize}
\end{definition}

\raggedcolumns
\columnbreak

\subsection{Open Drain Outputs}

\subsubsection{Multi-Device Bus Problem}

\begin{concept}{Problem: Several Devices on the Same Bus}\\
When multiple devices are connected to the same bus using push-pull outputs, contention can occur:
\begin{itemize}
    \item One device drives HIGH
    \item Another device drives LOW
    \item Result: Short circuit through both output stages
    \item High current, potential device damage
\end{itemize}
\end{concept}

% TODO: Add image from SCD_10_io_1.pdf, Slide 41
% Description: Problem diagram showing push-pull outputs creating conflicts
% Priority: CRITICAL
% Suggested filename: lecture10_pushpull_problem.png
\\
% \includegraphics[width=\linewidth]{lecture10_pushpull_problem.png}
\\
% WHEN YOU ADD IMAGE: Uncomment line above (remove %)

\mult{2}

\begin{definition}{Push-Pull Output}\\
Output stage can drive output HIGH or LOW using both P-channel and N-channel transistors.

\textbf{Issue:} Cannot be safely used by multiple drivers on the same net.
\end{definition}

\begin{definition}{Open-Drain Output}\\
Output stage can only drive output LOW. The output is pulled HIGH by an external resistor.

\textbf{Advantage:} Multiple devices can share the same bus safely.
\end{definition}

\multend

% TODO: Add image from SCD_10_io_1.pdf, Slide 42
% Description: Comparison diagram of push-pull vs open-drain output stages
% Priority: CRITICAL
% Suggested filename: lecture10_pushpull_vs_opendrain.png
\\
% \includegraphics[width=\linewidth]{lecture10_pushpull_vs_opendrain.png}
\\
% WHEN YOU ADD IMAGE: Uncomment line above (remove %)

\subsubsection{Open Drain Solution}

\begin{concept}{Wired-AND Configuration}\\
Several devices with open drain outputs can be connected together to one bus:
\begin{itemize}
    \item External pull-up resistor pulls bus to $V_{DD}$ (HIGH)
    \item Any device can pull bus to GND (LOW) by turning on N-MOS transistor
    \item If no device drives LOW, bus remains HIGH
    \item Multiple devices can drive LOW simultaneously without conflict
    \item Bus is LOW if \textbf{any} device drives LOW (Wired-AND logic)
\end{itemize}
\end{concept}

% TODO: Add image from SCD_10_io_1.pdf, Slide 43
% Description: Open drain circuit with multiple devices and pull-up resistor
% Priority: CRITICAL
% Suggested filename: lecture10_opendrain_circuit.png
\\
% \includegraphics[width=\linewidth]{lecture10_opendrain_circuit.png}
\\
% WHEN YOU ADD IMAGE: Uncomment line above (remove %)

\begin{example2}{I2C Interface}\\
\textbf{Application:} Open drain outputs are used in the I2C (Inter-Integrated Circuit) bus.

\textbf{Characteristics:}
\begin{itemize}
    \item Two-wire interface: SDA (data) and SCL (clock)
    \item Both lines use open drain with pull-up resistors
    \item Multiple devices can share the bus
    \item Bus arbitration prevents conflicts
\end{itemize}
\end{example2}

% TODO: Add image from SCD_10_io_1.pdf, Slide 44
% Description: I2C interface example showing open drain usage
% Priority: IMPORTANT
% Suggested filename: lecture10_i2c_example.png
\\
% \includegraphics[width=\linewidth]{lecture10_i2c_example.png}
\\
% WHEN YOU ADD IMAGE: Uncomment line above (remove %)

\begin{remark}
\textbf{Disadvantage of Open Drain:} When the transistor is switched on, high current flows through the pull-up resistor, resulting in power consumption.

$$I = \frac{V_{CCIO}}{R_{pullup}}$$

Larger pull-up resistor reduces current but slows down rise time.
\end{remark}

% TODO: Add image from SCD_10_io_1.pdf, Slide 45
% Description: Open drain output with pull-up resistor showing current flow
% Priority: IMPORTANT
% Suggested filename: lecture10_opendrain_current.png
\\
% \includegraphics[width=0.8\linewidth]{lecture10_opendrain_current.png}
\\
% WHEN YOU ADD IMAGE: Uncomment line above (remove %)

\begin{definition}{Bidirectional Open Drain Bus}\\
Several open drain outputs connected together form a bidirectional bus (principle of I2C):
\begin{itemize}
    \item Bus operates in two directions
    \item The actual bus driver controls the CMOS transistor
    \item No short circuit even if multiple CMOS transistors are turned on
    \item Any device can read the bus state at any time
\end{itemize}
\end{definition}

% TODO: Add image from SCD_10_io_1.pdf, Slide 46
% Description: Bidirectional open drain bus configuration
% Priority: IMPORTANT
% Suggested filename: lecture10_opendrain_bidirectional.png
\\
% \includegraphics[width=\linewidth]{lecture10_opendrain_bidirectional.png}
\\
% WHEN YOU ADD IMAGE: Uncomment line above (remove %)

\begin{remark}
FPGAs often provide programmable internal pull-up transistors, eliminating the need for external pull-up resistors in some applications.
\end{remark}

% TODO: Add image from SCD_10_io_1.pdf, Slide 47
% Description: Programmable pull-up configuration in FPGA
% Priority: SUPPLEMENTARY
% Suggested filename: lecture10_programmable_pullup.png
\\
% \includegraphics[width=0.8\linewidth]{lecture10_programmable_pullup.png}
\\
% WHEN YOU ADD IMAGE: Uncomment line above (remove %)

\raggedcolumns
\columnbreak

\subsection{Bus Hold}

\subsubsection{Tristate Bus Challenge}

\begin{concept}{Tristate Output}\\
A push-pull output with an additional ENABLE input:
\begin{itemize}
    \item ENABLE = $1$: Output drives HIGH or LOW (normal operation)
    \item ENABLE = $0$: Output is high-impedance (floating)
\end{itemize}

\textbf{Question:} What is the signal level when ENABLE = $0$?

\textbf{Problem:} When all drivers are disabled, the bus line floats at an undefined voltage level, potentially causing:
\begin{itemize}
    \item Increased power consumption (inputs in undefined state)
    \item Unreliable operation
    \item Noise sensitivity
\end{itemize}
\end{concept}

% TODO: Add image from SCD_10_io_1.pdf, Slide 49
% Description: Tristate bus diagram showing the floating state problem
% Priority: IMPORTANT
% Suggested filename: lecture10_tristate_problem.png
\\
% \includegraphics[width=\linewidth]{lecture10_tristate_problem.png}
\\
% WHEN YOU ADD IMAGE: Uncomment line above (remove %)

\subsubsection{Bus Hold Solution}

\begin{definition}{Bus Hold Circuit}\\
A bus hold circuit keeps the last defined value on the line when all drivers are disabled.

\textbf{Operation:}
\begin{itemize}
    \item Weak feedback circuit maintains previous state
    \item Can be overridden by any active driver
    \item Prevents floating inputs
    \item Similar principle to SRAM cell (weak bistable latch)
\end{itemize}

\textbf{Question:} Where do you know this circuit from?

\textbf{Answer:} It is similar to an SRAM memory cell!
\end{definition}

% TODO: Add image from SCD_10_io_1.pdf, Slide 50
% Description: Bus hold circuit diagram showing weak latch
% Priority: IMPORTANT
% Suggested filename: lecture10_bushold_circuit.png
\\
% \includegraphics[width=\linewidth]{lecture10_bushold_circuit.png}
\\
% WHEN YOU ADD IMAGE: Uncomment line above (remove %)

% TODO: Add image from SCD_10_io_1.pdf, Slide 51
% Description: Bus hold implementation details
% Priority: SUPPLEMENTARY
% Suggested filename: lecture10_bushold_detail.png
\\
% \includegraphics[width=\linewidth]{lecture10_bushold_detail.png}
\\
% WHEN YOU ADD IMAGE: Uncomment line above (remove %)

\subsection{Configuration of I/Os in FPGA}

\subsubsection{Pin Assignment and I/O Standards}

\begin{KR}{Setting I/O Standards in Quartus}\\
\textbf{Location:} Assignments $\rightarrow$ Pins

\textbf{Procedure:}
\begin{enumerate}
    \item Open Quartus project
    \item Navigate to Assignments menu
    \item Select Pin Planner or Pins
    \item For each pin/signal:
    \begin{itemize}
        \item Assign physical pin location
        \item Set I/O standard (e.g., LVCMOS 3.3V, LVDS, etc.)
        \item Configure direction (input, output, bidirectional)
    \end{itemize}
    \item Save and recompile design
\end{enumerate}
\end{KR}

% TODO: Add image from SCD_10_io_1.pdf, Slide 52
% Description: Pinout data sheet or FPGA pin configuration table
% Priority: IMPORTANT
% Suggested filename: lecture10_pinout_datasheet.png
\\
% \includegraphics[width=\linewidth]{lecture10_pinout_datasheet.png}
\\
% WHEN YOU ADD IMAGE: Uncomment line above (remove %)

% TODO: Add image from SCD_10_io_1.pdf, Slide 53
% Description: Quartus Pin Planner screenshot showing I/O standard assignment
% Priority: CRITICAL
% Suggested filename: lecture10_quartus_pin_assignment.png
\\
% \includegraphics[width=\linewidth]{lecture10_quartus_pin_assignment.png}
\\
% WHEN YOU ADD IMAGE: Uncomment line above (remove %)

\subsubsection{Advanced I/O Features}

\begin{KR}{Setting Drive Strength, Pull-Up, and Bus Hold}\\
\textbf{Location:} Assignments $\rightarrow$ Assignment Editor $\rightarrow$ I/O Features

\textbf{Configurable parameters:}
\begin{itemize}
    \item \textbf{Drive Strength:} Controls output current capability
    \begin{itemize}
        \item Higher drive strength: faster edges, more EMI
        \item Lower drive strength: slower edges, less EMI
    \end{itemize}
    \item \textbf{Pull-Up/Pull-Down:} Internal weak resistors
    \begin{itemize}
        \item Pull-up: brings floating input to HIGH
        \item Pull-down: brings floating input to LOW
    \end{itemize}
    \item \textbf{Bus Hold:} Maintains last driven state
    \item \textbf{Slew Rate:} Controls edge transition speed
    \item \textbf{Open Drain:} Enables open drain mode
\end{itemize}

\textbf{Procedure:}
\begin{enumerate}
    \item Open Assignment Editor
    \item Select I/O Features category
    \item Choose pin or group of pins
    \item Configure desired features
    \item Apply and recompile
\end{enumerate}
\end{KR}

% TODO: Add image from SCD_10_io_1.pdf, Slide 54
% Description: Quartus Assignment Editor showing I/O feature configuration
% Priority: CRITICAL
% Suggested filename: lecture10_io_features_config.png
\\
% \includegraphics[width=\linewidth]{lecture10_io_features_config.png}
\\
% WHEN YOU ADD IMAGE: Uncomment line above (remove %)

\subsubsection{Verification}

\begin{KR}{Checking if Assignments Were Applied}\\
After configuration and compilation, verify that assignments were correctly applied:

\textbf{Methods:}
\begin{enumerate}
    \item Check compilation report:
    \begin{itemize}
        \item Look for I/O assignment section
        \item Verify pin locations and standards
    \end{itemize}
    \item Review Pin Planner:
    \begin{itemize}
        \item Confirm visual pin placement
        \item Check I/O standard indicators
    \end{itemize}
    \item Examine Assignment Report:
    \begin{itemize}
        \item Lists all pin assignments
        \item Shows actual vs. requested settings
    \end{itemize}
    \item Check for warnings/errors:
    \begin{itemize}
        \item Incompatible I/O standards in same bank
        \item Missing reference voltages
        \item Conflicting drive strengths
    \end{itemize}
\end{enumerate}
\end{KR}

% TODO: Add image from SCD_10_io_1.pdf, Slide 55
% Description: Quartus compilation report or verification screen
% Priority: IMPORTANT
% Suggested filename: lecture10_assignment_verification.png
\\
% \includegraphics[width=\linewidth]{lecture10_assignment_verification.png}
\\
% WHEN YOU ADD IMAGE: Uncomment line above (remove %)

% ===== IMAGE SUMMARY =====
% Total images needed: 55
% CRITICAL priority: 28
% IMPORTANT priority: 19
% SUPPLEMENTARY priority: 8
%
% Quick extraction checklist:
% [X] [SCD_10_io_1.pdf, Slide 1] - Title slide (SUPPLEMENTARY)
% [X] [SCD_10_io_1.pdf, Slide 2] - Overview questions (SUPPLEMENTARY)
% [X] [SCD_10_io_1.pdf, Slide 3] - I/O topics list (SUPPLEMENTARY)
% [ ] [SCD_10_io_1.pdf, Slide 4] - Simplified CMOS push-pull (CRITICAL)
% [ ] [SCD_10_io_1.pdf, Slide 5] - CMOS I/O impedances (CRITICAL)
% [ ] [SCD_10_io_1.pdf, Slide 6] - P-Channel MOS-FET (CRITICAL)
% [ ] [SCD_10_io_1.pdf, Slide 7] - N-Channel MOS-FET (CRITICAL)
% [X] [SCD_10_io_1.pdf, Slide 8] - Voltage nomenclature (SUPPLEMENTARY)
% [ ] [SCD_10_io_1.pdf, Slide 9] - N-Channel operation (CRITICAL)
% [ ] [SCD_10_io_1.pdf, Slide 10] - P-Channel operation (CRITICAL)
% [ ] [SCD_10_io_1.pdf, Slide 11] - Combined operation (CRITICAL)
% [ ] [SCD_10_io_1.pdf, Slide 12] - HP vs LP transistors (IMPORTANT)
% [ ] [SCD_10_io_1.pdf, Slide 13] - CMOS cross-section (IMPORTANT)
% [ ] [SCD_10_io_1.pdf, Slide 14] - Threshold scaling (CRITICAL)
% [X] [SCD_10_io_1.pdf, Slide 15] - Technology trend table (included in text)
% [ ] [SCD_10_io_1.pdf, Slide 16] - Level shifter overview (CRITICAL)
% [ ] [SCD_10_io_1.pdf, Slide 17] - Level shifter detail (IMPORTANT)
% [X] [SCD_10_io_1.pdf, Slide 18] - I/O standards table (included in text)
% [ ] [SCD_10_io_1.pdf, Slide 19] - I/O standards supported (IMPORTANT)
% [ ] [SCD_10_io_1.pdf, Slide 20] - Additional I/O standards (IMPORTANT)
% [ ] [SCD_10_io_1.pdf, Slide 21] - LVCMOS 3.3V output (CRITICAL)
% [ ] [SCD_10_io_1.pdf, Slide 22] - LVCMOS 1.8V output (CRITICAL)
% [ ] [SCD_10_io_1.pdf, Slide 23] - LVCMOS input levels (CRITICAL)
% [ ] [SCD_10_io_1.pdf, Slide 24] - LVCMOS margins (CRITICAL)
% [ ] [SCD_10_io_1.pdf, Slide 25] - LVCMOS vs LVTTL (IMPORTANT)
% [ ] [SCD_10_io_1.pdf, Slide 26] - Voltage domains question (IMPORTANT)
% [ ] [SCD_10_io_1.pdf, Slide 27] - I/O banks structure (CRITICAL)
% [X] [SCD_10_io_1.pdf, Slide 28] - Section title (SUPPLEMENTARY)
% [ ] [SCD_10_io_1.pdf, Slide 29] - Differential principle (CRITICAL)
% [ ] [SCD_10_io_1.pdf, Slide 30] - Common mode rejection (CRITICAL)
% [ ] [SCD_10_io_1.pdf, Slide 31] - Differential voltages (IMPORTANT)
% [ ] [SCD_10_io_1.pdf, Slide 32] - Current direction (IMPORTANT)
% [ ] [SCD_10_io_1.pdf, Slide 33] - Eye diagram 1 (IMPORTANT)
% [ ] [SCD_10_io_1.pdf, Slide 34] - Eye diagram 2 (SUPPLEMENTARY)
% [ ] [SCD_10_io_1.pdf, Slide 35] - LVDS intro (SUPPLEMENTARY)
% [ ] [SCD_10_io_1.pdf, Slide 36] - LVDS circuit (CRITICAL)
% [X] [SCD_10_io_1.pdf, Slide 37] - LVDS characteristics (included in text)
% [ ] [SCD_10_io_1.pdf, Slide 38] - LVDS application (IMPORTANT)
% [X] [SCD_10_io_1.pdf, Slide 39] - Special diff I/Os (included in text)
% [X] [SCD_10_io_1.pdf, Slide 40] - Section title (SUPPLEMENTARY)
% [ ] [SCD_10_io_1.pdf, Slide 41] - Push-pull problem (CRITICAL)
% [ ] [SCD_10_io_1.pdf, Slide 42] - Push-pull vs open-drain (CRITICAL)
% [ ] [SCD_10_io_1.pdf, Slide 43] - Open drain circuit (CRITICAL)
% [ ] [SCD_10_io_1.pdf, Slide 44] - I2C example (IMPORTANT)
% [ ] [SCD_10_io_1.pdf, Slide 45] - Open drain current (IMPORTANT)
% [ ] [SCD_10_io_1.pdf, Slide 46] - Bidirectional bus (IMPORTANT)
% [ ] [SCD_10_io_1.pdf, Slide 47] - Programmable pull-up (SUPPLEMENTARY)
% [X] [SCD_10_io_1.pdf, Slide 48] - Section title (SUPPLEMENTARY)
% [ ] [SCD_10_io_1.pdf, Slide 49] - Tristate problem (IMPORTANT)
% [ ] [SCD_10_io_1.pdf, Slide 50] - Bus hold circuit (IMPORTANT)
% [ ] [SCD_10_io_1.pdf, Slide 51] - Bus hold detail (SUPPLEMENTARY)
% [ ] [SCD_10_io_1.pdf, Slide 52] - Pinout datasheet (IMPORTANT)
% [ ] [SCD_10_io_1.pdf, Slide 53] - Quartus pin assignment (CRITICAL)
% [ ] [SCD_10_io_1.pdf, Slide 54] - I/O features config (CRITICAL)
% [ ] [SCD_10_io_1.pdf, Slide 55] - Assignment verification (IMPORTANT)
% =====================