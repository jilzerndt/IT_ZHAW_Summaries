\section{Booting and Device Tree}

\subsection{Learning Objectives}

\begin{concept}{Course Goals}\\
After this lecture and lab exercise, students will be able to:
\begin{itemize}
    \item Sketch the resources that need to be configured during boot
    \item Explain the boot sequence of an SoC
    \item Configure and build a bootloader program for their SoC
    \item Understand the basic concept of a device tree
    \item Modify the device tree
\end{itemize}
\end{concept}

\subsection{SoC FPGA Architecture Overview}

\begin{definition}{SoC FPGA Boot Components}\\
The Cyclone V SoC FPGA architecture includes:

\textbf{Hard Processor System (HPS):}
\begin{itemize}
    \item Dual ARM Cortex-A9 cores with NEON/FPU
    \item L1 Cache (instruction and data)
    \item L2 Cache (shared)
    \item Dedicated MPU pins
    \item Dedicated DDR pins for external SDRAM
\end{itemize}

\textbf{Interconnect:}
\begin{itemize}
    \item L3 Interconnect fabric
    \item SDRAM Controller Subsystem
    \item Bridges between HPS and FPGA
\end{itemize}

\textbf{FPGA Fabric:}
\begin{itemize}
    \item Programmable logic
    \item FPGA pins for custom I/O
    \item Connection to HPS via bridges
\end{itemize}
\end{definition}

% TODO: Add image from SCD_3b_Booting_and_Device_Tree.pdf, Page 5
% Description: SoC FPGA architecture block diagram
% Priority: CRITICAL
% Suggested filename: lecture03b_soc_architecture.png
% \includegraphics[width=\linewidth]{lecture03b_soc_architecture.png}

% TODO: Add image from SCD_3b_Booting_and_Device_Tree.pdf, Page 6
% Description: Dedicated HPS I/O pins and DDR connection
% Priority: IMPORTANT
% Suggested filename: lecture03b_hps_io_pins.png
% \includegraphics[width=\linewidth]{lecture03b_hps_io_pins.png}

\raggedcolumns
\columnbreak

\subsection{Boot Sequence Overview}

\begin{concept}{Boot Process}\\
Booting a processor involves bringing it from reset to a running application through multiple steps.

\textbf{Boot Stages:}
$$\text{Reset} \rightarrow \text{Boot ROM} \rightarrow \text{Preloader (SPL)} \rightarrow \text{Bootloader (U-Boot)} \rightarrow \text{Operating System} \rightarrow \text{Application}$$

\textbf{Key Characteristics:}
\begin{itemize}
    \item Each step executed by hard or software
    \item Not all steps always required
    \item User software (everything after Boot ROM) can be modified
    \item Progressive initialization of system resources
\end{itemize}
\end{concept}

% TODO: Add image from SCD_3b_Booting_and_Device_Tree.pdf, Page 8
% Description: Complete boot sequence diagram
% Priority: CRITICAL
% Suggested filename: lecture03b_boot_sequence.png
% \includegraphics[width=\linewidth]{lecture03b_boot_sequence.png}

\subsubsection{Stage 1: Reset}

\begin{definition}{Reset Stage}\\
Reset brings the HPS to a defined default state.

\textbf{When Reset Occurs:}
\begin{itemize}
    \item At power-on
    \item On user request (button or input signal on dedicated pin)
    \item Triggered by debugger
    \item Triggered by watchdog timer
    \item Triggered from FPGA fabric
\end{itemize}

\textbf{Reset Process:}
\begin{itemize}
    \item Controlled by hardware reset controller block
    \item Controls sequence of resetting modules
    \item Resets clocks
    \item Resets CPU cores
    \item Reaches defined default state for critical resources
\end{itemize}
\end{definition}

% TODO: Add image from SCD_3b_Booting_and_Device_Tree.pdf, Page 9
% Description: Reset stage in boot sequence
% Priority: IMPORTANT
% Suggested filename: lecture03b_reset_stage.png
% \includegraphics[width=\linewidth]{lecture03b_reset_stage.png}

% TODO: Add image from SCD_3b_Booting_and_Device_Tree.pdf, Page 10
% Description: Reset stage details (2)
% Priority: SUPPLEMENTARY
% Suggested filename: lecture03b_reset_details.png
% \includegraphics[width=\linewidth]{lecture03b_reset_details.png}

\subsubsection{Stage 2: Boot ROM}

\begin{definition}{Boot ROM Stage}\\
The Boot ROM is the first code executed after reset.

\textbf{Characteristics:}
\begin{itemize}
    \item 64 KB of code in on-chip ROM
    \item Located at reset exception address
    \item Cannot be modified (hard-coded in silicon)
\end{itemize}

\textbf{Boot ROM Functions:}
\begin{enumerate}
    \item Determine boot source (read BSEL pins)
    \item Initialize HPS core systems:
    \begin{itemize}
        \item Set up watchdog timer
        \item Configure clocking (PLL)
        \item Initialize flash controller
    \end{itemize}
    \item Copy preloader program from selected boot source to on-chip RAM
    \item Jump execution to preloader program
\end{enumerate}
\end{definition}

% TODO: Add image from SCD_3b_Booting_and_Device_Tree.pdf, Page 11
% Description: Boot ROM stage diagram
% Priority: CRITICAL
% Suggested filename: lecture03b_boot_rom_stage.png
% \includegraphics[width=\linewidth]{lecture03b_boot_rom_stage.png}

% TODO: Add image from SCD_3b_Booting_and_Device_Tree.pdf, Page 12
% Description: Boot ROM stage details (2)
% Priority: SUPPLEMENTARY
% Suggested filename: lecture03b_boot_rom_details.png
% \includegraphics[width=\linewidth]{lecture03b_boot_rom_details.png}

\begin{definition}{Boot Source Selection (BSEL)}\\
BSEL is a group of input pins that tell Boot ROM which device to boot from.

\textbf{BSEL Pin Configuration:}

\begin{center}
\begin{tabular}{|c|l|}
\hline
\textbf{BSEL[2..0]} & \textbf{Flash Device} \\
\hline
0x0 & Reserved \\
0x1 & FPGA (HPS-to-FPGA bridge) \\
0x2 & 1.8V NAND flash memory \\
0x3 & 3.3V NAND flash memory \\
0x4 & 1.8V SD/MMC flash memory \\
0x5 & 3.3V SD/MMC flash memory \\
0x6 & 1.8V quad SPI flash memory \\
0x7 & 3.3V quad SPI flash memory \\
\hline
\end{tabular}
\end{center}

\textbf{Hardware Configuration:}
\begin{itemize}
    \item BSEL pins set by DIP switches or resistors on PCB
    \item Configuration must match physical boot device
    \item Determines which flash memory Boot ROM will access
\end{itemize}
\end{definition}

% TODO: Add image from SCD_3b_Booting_and_Device_Tree.pdf, Page 13
% Description: Boot source selection table and configuration
% Priority: CRITICAL
% Suggested filename: lecture03b_bsel_configuration.png
% \includegraphics[width=\linewidth]{lecture03b_bsel_configuration.png}

\begin{example2}{SD Card Boot Structure}\\
\textbf{Boot from SD Card:}

When BSEL is configured for SD card boot, the Boot ROM looks for the preloader on the SD card.

\textbf{SD Card Partition Structure:}
\begin{itemize}
    \item \textbf{Master Boot Record (MBR):} 512 bytes
    \begin{itemize}
        \item Stores partition table
        \item Located at beginning of SD card
    \end{itemize}
    \item \textbf{RAW Partition (Type A2):} 64 KB × 4 = 256 KB
    \begin{itemize}
        \item Contains four copies of preloader image
        \item Redundancy for reliability
        \item No filesystem (raw binary data)
    \end{itemize}
\end{itemize}

\tcblower

\textbf{Boot Process:}
\begin{enumerate}
    \item Boot ROM reads MBR to find partition table
    \item Locates RAW partition (type A2)
    \item Copies preloader image to on-chip memory
    \item Jumps to preloader entry point
\end{enumerate}

\important{The preloader partition has no filesystem - it contains raw executable code.}
\end{example2}

% TODO: Add image from SCD_3b_Booting_and_Device_Tree.pdf, Page 14
% Description: SD card partition structure for booting
% Priority: CRITICAL
% Suggested filename: lecture03b_sd_card_structure.png
% \includegraphics[width=\linewidth]{lecture03b_sd_card_structure.png}

\raggedcolumns
\columnbreak

\subsubsection{Stage 3: Preloader (SPL)}

\begin{definition}{Preloader/SPL (Second Program Loader)}\\
The preloader is user-defined code that initializes critical hardware.

\textbf{Key Responsibilities:}
\begin{itemize}
    \item \textbf{Clock Reconfiguration:} Set up system clocks for optimal performance
    \item \textbf{SDRAM Initialization:} Initialize SDRAM interface and check RAM size
    \item \textbf{I/O Pin Configuration:} Configure I/O pins of HPS
    \item \textbf{Peripheral Initialization:} Initialize peripheral interfaces needed to load U-Boot
    \item \textbf{Load U-Boot:} Copy U-Boot bootloader to SDRAM
    \item \textbf{Jump to U-Boot:} Transfer execution to bootloader
\end{itemize}

\textbf{Characteristics:}
\begin{itemize}
    \item Executes from on-chip RAM (fast, limited size)
    \item User-defined and customizable
    \item Must be small to fit in on-chip RAM
    \item Critical for system initialization
\end{itemize}
\end{definition}

% TODO: Add image from SCD_3b_Booting_and_Device_Tree.pdf, Page 15
% Description: Preloader stage diagram
% Priority: CRITICAL
% Suggested filename: lecture03b_preloader_stage.png
% \includegraphics[width=\linewidth]{lecture03b_preloader_stage.png}

% TODO: Add image from SCD_3b_Booting_and_Device_Tree.pdf, Page 16
% Description: Preloader stage details (2)
% Priority: SUPPLEMENTARY
% Suggested filename: lecture03b_preloader_details.png
% \includegraphics[width=\linewidth]{lecture03b_preloader_details.png}

\subsubsection{Stage 4: Bootloader (U-Boot)}

\begin{definition}{U-Boot Bootloader}\\
U-Boot is a versatile, feature-rich bootloader program.

\textbf{Primary Functions:}
\begin{itemize}
    \item \textbf{Hardware Tests:} Verify system functionality
    \item \textbf{FPGA Configuration:} Load bitstream to FPGA fabric
    \item \textbf{Load Linux Kernel:} Copy kernel image to memory
    \item \textbf{Load Device Tree:} Provide hardware description to kernel
\end{itemize}

\textbf{Open Source:}
\begin{itemize}
    \item Available at: https://docs.u-boot.org/en/latest/
    \item Highly configurable
    \item Active community support
    \item Board-specific configurations available
\end{itemize}
\end{definition}

% TODO: Add image from SCD_3b_Booting_and_Device_Tree.pdf, Page 17
% Description: Bootloader stage diagram
% Priority: CRITICAL
% Suggested filename: lecture03b_bootloader_stage.png
% \includegraphics[width=\linewidth]{lecture03b_bootloader_stage.png}

\begin{concept}{U-Boot Capabilities}\\
U-Boot is the earliest point of developer interaction during boot.

\textbf{Testing and Access:}
\begin{itemize}
    \item Access and test SDRAM
    \item Test FPGA fabric
    \item Test on-chip peripherals
    \item Access flash memory
    \item Access USB devices
    \item Access network interfaces
\end{itemize}

\textbf{User Interface:}
\begin{itemize}
    \item Interactive shell for user commands
    \item Scripting support
    \item Environment variable configuration
    \item Boot menu system
\end{itemize}

\textbf{System Configuration:}
\begin{itemize}
    \item Configure FPGA fabric
    \item Set boot parameters
    \item Load and execute operating system kernel
\end{itemize}
\end{concept}

% TODO: Add image from SCD_3b_Booting_and_Device_Tree.pdf, Page 18
% Description: U-Boot capabilities and interface
% Priority: IMPORTANT
% Suggested filename: lecture03b_uboot_capabilities.png
% \includegraphics[width=\linewidth]{lecture03b_uboot_capabilities.png}

\subsubsection{Stage 5: Operating System}

\begin{definition}{Operating System Boot}\\
The OS is loaded and initialized by the bootloader.

\textbf{OS Choices:}
\begin{itemize}
    \item \textbf{RTOS:} Real-Time Operating System
    \item \textbf{Embedded Linux:} Lightweight Linux distribution
    \item \textbf{Desktop Linux:} Full-featured Linux distribution
\end{itemize}

\textbf{Hardware Description Requirement:}
The OS needs a device tree to know what hardware is present:
\begin{itemize}
    \item In the HPS (Hard Processor System)
    \item In FPGA fabric (custom IP blocks)
    \item On the PCB (external peripherals)
\end{itemize}

\important{Without a device tree, Linux cannot discover and utilize the hardware automatically.}
\end{definition}

% TODO: Add image from SCD_3b_Booting_and_Device_Tree.pdf, Page 19
% Description: Operating system stage diagram
% Priority: IMPORTANT
% Suggested filename: lecture03b_os_stage.png
% \includegraphics[width=\linewidth]{lecture03b_os_stage.png}

\subsubsection{Stage 6: Application}

\begin{definition}{User Application}\\
The final stage is the user application running under the OS.

\textbf{Characteristics:}
\begin{itemize}
    \item Started in its own process context by the OS
    \item OS provides resource management
    \item OS controls access to FPGA and peripherals
    \item OS provides device drivers
    \item Application runs in user space (protected mode)
\end{itemize}

\textbf{Resource Access:}
\begin{itemize}
    \item Access to hardware through OS drivers
    \item Memory mapping for FPGA peripherals
    \item Standard Linux APIs available
    \item Multi-process support
\end{itemize}
\end{definition}

% TODO: Add image from SCD_3b_Booting_and_Device_Tree.pdf, Page 20
% Description: Application stage diagram
% Priority: IMPORTANT
% Suggested filename: lecture03b_application_stage.png
% \includegraphics[width=\linewidth]{lecture03b_application_stage.png}

\raggedcolumns
\columnbreak

\subsection{Boot Sequence Options}

\begin{concept}{Customization Possibilities}\\
Everything marked "User Software" can be modified by the developer:

\textbf{Modifiable Components:}
\begin{itemize}
    \item Preloader (SPL)
    \item Bootloader (U-Boot)
    \item Operating System
    \item Applications
\end{itemize}

\textbf{Developer Actions:}
\begin{itemize}
    \item Optimize performance
    \item Add capabilities
    \item Remove unnecessary features
    \item Contribute to open source projects
\end{itemize}
\end{concept}

\begin{remark}
\textbf{Alternative Boot Path:}

An RTOS can be loaded directly by the Boot ROM by replacing the Preloader image.

\textbf{Questions to Consider:}
\begin{itemize}
    \item Why would you do this?
    \item What are the advantages?
    \item What problems might arise?
\end{itemize}

\textbf{Analysis:}
\begin{itemize}
    \item \textbf{Advantage:} Faster boot time (fewer stages)
    \item \textbf{Advantage:} Simpler system (less software)
    \item \textbf{Problem:} Limited initialization (Boot ROM constraints)
    \item \textbf{Problem:} Size constraints (must fit in space allocated for preloader)
\end{itemize}
\end{remark}

% TODO: Add image from SCD_3b_Booting_and_Device_Tree.pdf, Page 21
% Description: Boot sequence options and customization
% Priority: IMPORTANT
% Suggested filename: lecture03b_boot_options.png
% \includegraphics[width=\linewidth]{lecture03b_boot_options.png}

\subsection{Device Tree}

\begin{definition}{Device Tree Concept}\\
A device tree is a data structure that describes hardware to the operating system.

\textbf{Why Needed:}
\begin{itemize}
    \item Embedded Linux cannot automatically discover hardware
    \item Hardware configuration varies between designs
    \item FPGA fabric contains custom IP blocks
    \item PCB layout differs between boards
\end{itemize}

\textbf{Loading Process:}
\begin{itemize}
    \item Bootloader must load Device Tree Blob (DTB) to memory
    \item DTB loaded before kernel boot
    \item Kernel reads DTB to discover hardware
\end{itemize}

\textbf{Information Provided:}
\begin{itemize}
    \item Device address ranges
    \item Type of device
    \item Compatible drivers
    \item Register maps
    \item Clock configurations
    \item Interrupt assignments
    \item Other device properties
\end{itemize}
\end{definition}

% TODO: Add image from SCD_3b_Booting_and_Device_Tree.pdf, Page 24
% Description: Device tree concept and purpose
% Priority: CRITICAL
% Suggested filename: lecture03b_device_tree_concept.png
% \includegraphics[width=\linewidth]{lecture03b_device_tree_concept.png}

\begin{concept}{Device Tree Functions}\\
The device tree serves multiple purposes:

\textbf{Memory Map Description:}
\begin{itemize}
    \item Describes complete system memory map
    \item Translates between memory maps of subregions
    \item Maps HPS memory space
    \item Maps FPGA peripheral space
    \item Maps external peripheral space
\end{itemize}

\textbf{Peripheral Details:}
\begin{itemize}
    \item Peripheral base addresses
    \item Address ranges and sizes
    \item Interrupt connections
    \item Clock sources
    \item Driver compatibility information
\end{itemize}

\textbf{Bridge Configuration:}
\begin{itemize}
    \item HPS-FPGA bridge specifications
    \item Data bus width
    \item Address translation
    \item Memory-mapped block types
    \item Associated drivers
\end{itemize}
\end{concept}

% TODO: Add image from SCD_3b_Booting_and_Device_Tree.pdf, Page 25
% Description: Device tree memory map translation
% Priority: CRITICAL
% Suggested filename: lecture03b_device_tree_memory_map.png
% \includegraphics[width=\linewidth]{lecture03b_device_tree_memory_map.png}

\begin{concept}{Device Tree Exploration in Linux}\\
The device tree can be explored in the running Linux file system:

\textbf{File System Locations:}
\begin{itemize}
    \item \texttt{/sys/class} - Device classes
    \item \texttt{/sys/devices/platform/sopc@0} - Platform devices
\end{itemize}

\textbf{Description Contents:}
\begin{itemize}
    \item Hierarchy of hardware components
    \item Peripheral addresses and clocks
    \item Compatible driver information
    \item Device status (enabled/disabled)
    \item HPS-FPGA bridge configuration
    \item Memory map details
\end{itemize}
\end{concept}

% TODO: Add image from SCD_3b_Booting_and_Device_Tree.pdf, Page 26
% Description: Device tree in Linux file system
% Priority: IMPORTANT
% Suggested filename: lecture03b_device_tree_filesystem.png
% \includegraphics[width=\linewidth]{lecture03b_device_tree_filesystem.png}

\raggedcolumns
\columnbreak

\subsection{Device Tree Structure}

\begin{definition}{Device Tree Hierarchy}\\
The device tree is organized as a hierarchical structure:

\textbf{Root Node: /}
\begin{itemize}
    \item \texttt{\#size-cells}: Number of cells for size values
    \item \texttt{\#address-cells}: Number of cells for address values
    \item \texttt{compatible}: Device compatibility string
\end{itemize}

\textbf{Major Child Nodes:}
\begin{itemize}
    \item \texttt{cpus\{\}}: CPU configuration
    \begin{itemize}
        \item Bus frequency
        \item CPU cores and their properties
    \end{itemize}
    \item \texttt{memory\{\}}: Physical memory description
    \item \texttt{clocks\{\}}: Clock sources and configurations
    \item \texttt{sopc@0}: SoC peripheral controller
    \begin{itemize}
        \item \texttt{bridge@0xc0000000}: HPS-FPGA bridge
        \item Memory map definitions
        \item \texttt{jtag\_uart\_0: serial@0x...}: JTAG UART peripheral
        \item Additional peripheral nodes
    \end{itemize}
\end{itemize}
\end{definition}

% TODO: Add image from SCD_3b_Booting_and_Device_Tree.pdf, Page 27
% Description: Device tree hierarchy structure
% Priority: CRITICAL
% Suggested filename: lecture03b_device_tree_hierarchy.png
% \includegraphics[width=\linewidth]{lecture03b_device_tree_hierarchy.png}

\begin{definition}{Device Tree Nodes}\\
Devices are described as nodes in the device tree.

\textbf{Node Structure:}
$$\texttt{label: name@unit\_address}$$

\textbf{Node Contents:}
\begin{itemize}
    \item \textbf{Properties:} Key-value pairs describing the device
    \item \textbf{Child Nodes:} Nested devices or sub-components
\end{itemize}

\textbf{Example Node:}
\begin{lstlisting}[language=C, style=base]
cpu@0x1 {
    device_type = "cpu";
    compatible = "arm,cortex-a9";
    reg = <0x00000001>;
    next-level-cache = <&hps_0_L2>;
    enable-method = "psci";
};
\end{lstlisting}

\textbf{Node Components:}
\begin{itemize}
    \item \texttt{cpu@0x1}: Node name with unit address
    \item \texttt{device\_type}: Type of device
    \item \texttt{compatible}: Compatible driver string
    \item \texttt{reg}: Register address
    \item \texttt{next-level-cache}: Reference to cache (phandle)
\end{itemize}
\end{definition}

% TODO: Add image from SCD_3b_Booting_and_Device_Tree.pdf, Page 28
% Description: Device tree node structure
% Priority: CRITICAL
% Suggested filename: lecture03b_node_structure.png
% \includegraphics[width=\linewidth]{lecture03b_node_structure.png}

\begin{definition}{Node Structure Components}\\
A typical peripheral node contains several key properties:

\textbf{Example: JTAG UART Node}
\begin{lstlisting}[language=C, style=base]
serial@0x100020000 {
    compatible = "altr,juart-16.0", "altr,juart-1.0";
    reg = <0x00000001 0x00020000 0x00000008>;
    interrupt-parent = <&hps_0_arm_gic_0>;
    interrupts = <0 42 4>;
    clocks = <&clk_0>;
};
\end{lstlisting}

\textbf{Property Descriptions:}
\begin{itemize}
    \item \textbf{Label:} \texttt{serial@0x100020000}
    \item \textbf{Properties:} Device characteristics and configuration
    \item \textbf{Values:} Specific settings for this instance
\end{itemize}
\end{definition}

% TODO: Add image from SCD_3b_Booting_and_Device_Tree.pdf, Page 29
% Description: Complete node structure example
% Priority: CRITICAL
% Suggested filename: lecture03b_complete_node.png
% \includegraphics[width=\linewidth]{lecture03b_complete_node.png}

\subsection{Node Properties}

\begin{definition}{Common Node Properties}\\
Standard properties found in device tree nodes:

\textbf{compatible}
\begin{itemize}
    \item Defines programming model
    \item Identifies compatible driver
    \item Can list multiple compatible strings (fallback options)
    \item Example: \texttt{"altr,juart-16.0", "altr,juart-1.0"}
\end{itemize}

\textbf{reg}
\begin{itemize}
    \item Address and length of memory region
    \item Format: \texttt{<base\_address length>}
    \item Can use multiple address cells for extended addressing
    \item Example: \texttt{<0x00000001 0x00020000 0x00000008>}
    \begin{itemize}
        \item Address: 0x1'0002'0000 (35-bit ARM physical address)
        \item Length: 0x08 bytes
    \end{itemize}
\end{itemize}

\textbf{interrupt-parent}
\begin{itemize}
    \item Phandle to interrupt controller
    \item Specifies which controller handles interrupts
    \item Example: \texttt{<\&hps\_0\_arm\_gic\_0>}
\end{itemize}

\textbf{interrupts}
\begin{itemize}
    \item Interrupt number(s) and configuration
    \item Format depends on interrupt controller
    \item Example: \texttt{<0 42 4>}
\end{itemize}

\textbf{clocks}
\begin{itemize}
    \item Clock reference (phandle)
    \item Links to clock provider node
    \item Example: \texttt{<\&clk\_0>}
\end{itemize}
\end{definition}

\begin{remark}
\textbf{35-bit ARM Physical Address:}

The ARM Cortex-A9 in Cyclone V uses a 35-bit physical address space.

\textbf{Address Ranges:}
\begin{itemize}
    \item HPS internal: 0x0'0000'0000 - 0x0'FFFF'FFFF
    \item LW (Lightweight) Bridge: 0x1'0000'0000 - 0x1'FFFF'FFFF
    \item HPS-to-FPGA Bridge: Higher addresses
\end{itemize}

Everything on the Lightweight bridge uses addresses starting at 0x1'0000'0000.
\end{remark}

% TODO: Add image from SCD_3b_Booting_and_Device_Tree.pdf, Page 30
% Description: Node properties detailed explanation
% Priority: CRITICAL
% Suggested filename: lecture03b_node_properties.png
% \includegraphics[width=\linewidth]{lecture03b_node_properties.png}

\raggedcolumns
\columnbreak

\subsection{Bus Properties}

\begin{definition}{Bus Node Properties}\\
Bus nodes define address spaces and translation between parent and child buses.

\textbf{Example: HPS-FPGA Bridge}
\begin{lstlisting}[language=C, style=base]
bridge@0xc0000000 {
    compatible = "altr,bridge-16.0", "simple-bus";
    reg = <0xc0000000 0x20000000>,  // HPS-FPGA bridge
          <0xff200000 0x00200000>;  // Lightweight bridge
    reg-names = "axi_h2f", "axi_h2f_lw";
    clocks = <&clk_0 &clk_0>;
    clock-names = "h2f_axi_clock", "h2f_lw_axi_clock";
    #address-cells = <2>;
    #size-cells = <1>;
    ranges = <0x00000000 0x00000000 0xc0000000 0x00010000>,
             <0x00000001 0x00021000 0xff221000 0x00000020>,
             ...
};
\end{lstlisting}

\textbf{Key Properties:}
\begin{itemize}
    \item \textbf{reg:} Base address and range of bridge regions
    \item \textbf{\#address-cells:} Number of 32-bit cells for base address (2 = 64-bit)
    \item \textbf{\#size-cells:} Number of 32-bit cells for size (1 = 32-bit)
    \item \textbf{ranges:} Address translation table
\end{itemize}
\end{definition}

% TODO: Add image from SCD_3b_Booting_and_Device_Tree.pdf, Page 31
% Description: Bus properties and address translation
% Priority: CRITICAL
% Suggested filename: lecture03b_bus_properties.png
% \includegraphics[width=\linewidth]{lecture03b_bus_properties.png}

\begin{definition}{Memory Ranges and Address Translation}\\
The \texttt{ranges} property defines address translation between buses.

\textbf{Format:}
$$\texttt{<child\_address parent\_address size>}$$

\textbf{Example:}
\begin{lstlisting}[language=C, style=base]
ranges = <0x00000000 0x00000000 0xc0000000 0x00010000>,
         <0x00000001 0x00021000 0xff221000 0x00000020>,
         <0x00000001 0x00021020 0xff221020 0x00000020>,
         <0x00000001 0x00020000 0xff220000 0x00000008>;
\end{lstlisting}

\textbf{Translation Components:}
\begin{itemize}
    \item \textbf{Child address:} Address on the child bus (from device perspective)
    \item \textbf{Parent address:} Address on parent bus (from CPU/Linux perspective)
    \item \textbf{Size:} Size of the mapped region
\end{itemize}

\textbf{Usage:}
\begin{itemize}
    \item Linux programs use parent address
    \item Hardware uses child address
    \item Device tree provides translation
\end{itemize}

\textbf{Example Device Reference:}
\begin{lstlisting}[language=C, style=base]
jtag_uart: serial@0x100020000 {
    reg = <0x00000001 0x00020000 0x00000008>;
    ...
};
\end{lstlisting}

Child address 0x1'0002'0000 translates to parent address 0xff220000 for Linux access.
\end{definition}

% TODO: Add image from SCD_3b_Booting_and_Device_Tree.pdf, Page 32
% Description: Memory ranges and address translation
% Priority: CRITICAL
% Suggested filename: lecture03b_memory_ranges.png
% \includegraphics[width=\linewidth]{lecture03b_memory_ranges.png}

\subsection{Building the Device Tree}

\begin{KR}{Device Tree Build Process}\\
\textbf{Step 1: Extract System Information}

The system description in \texttt{*.sopcinfo} contains the memory map in XML format.

\begin{lstlisting}[language=bash, style=base]
# Convert .sopcinfo into human-readable header file
sopc-create-header-files pacman_soc.sopcinfo
# Generates: system.h
\end{lstlisting}

\textbf{Step 2: Add Device Nodes}

Add custom nodes to device tree template using vendor documentation.

\textbf{Example: GPIO Node}
\begin{lstlisting}[language=C, style=base]
gpio_altr: gpio@ff200000 {
    compatible = "altr,pio-1.0";
    reg = <0xff200000 0x10>;
    interrupts = <0 45 4>;
    altr,ngpio = <32>;
    altr,interrupt-type = <IRQ_TYPE_EDGE_RISING>;
    #gpio-cells = <2>;
    gpio-controller;
    #interrupt-cells = <2>;
    interrupt-controller;
};
\end{lstlisting}

\textbf{Step 3: Compile Device Tree}

Use device tree compiler (\texttt{dtc}) to compile source to binary blob.

\important{Refer to vendor documentation for specific property requirements and values.}
\end{KR}

% TODO: Add image from SCD_3b_Booting_and_Device_Tree.pdf, Page 33
% Description: Device tree building process example
% Priority: CRITICAL
% Suggested filename: lecture03b_device_tree_build.png
% \includegraphics[width=\linewidth]{lecture03b_device_tree_build.png}

\begin{concept}{Device Tree Compilation Flow}\\
The complete compilation flow:

\textbf{Input Files:}
\begin{enumerate}
    \item \textbf{.sopcinfo:} Platform Designer system description
    \item \textbf{system.h:} Generated header with memory map
    \item \textbf{*.dtsi:} Device tree include file with custom nodes
    \item \textbf{*.dts:} Main device tree source file
\end{enumerate}

\textbf{Tools:}
\begin{enumerate}
    \item \texttt{sopc-create-header-files}: Convert .sopcinfo to .h
    \item \texttt{dtc} (Device Tree Compiler): Compile .dts to .dtb
\end{enumerate}

\textbf{Output:}
\begin{itemize}
    \item \textbf{.dtb:} Device Tree Blob (binary format)
    \item Loaded by bootloader
    \item Passed to Linux kernel
\end{itemize}

\textbf{File Locations:}
\begin{itemize}
    \item Base template: \texttt{socfpga\_cyclone5\_de1soc.dtsi}
    \item Custom additions: Your \texttt{*.dtsi} file
    \item Final blob: \texttt{socfpga.dtb}
\end{itemize}
\end{concept}

% TODO: Add image from SCD_3b_Booting_and_Device_Tree.pdf, Page 34
% Description: Complete device tree compilation flow diagram
% Priority: CRITICAL
% Suggested filename: lecture03b_dt_compilation_flow.png
% \includegraphics[width=\linewidth]{lecture03b_dt_compilation_flow.png}

\raggedcolumns
\columnbreak

\subsection{Booting the Kernel}

\begin{definition}{U-Boot Configuration File}\\
U-Boot uses a configuration file to locate kernel and device tree.

\textbf{File: extlinux.conf}
\begin{itemize}
    \item Located in BOOT partition of SD card
    \item Text file with boot configuration
\end{itemize}

\textbf{Configuration Parameters:}
\begin{lstlisting}[language=bash, style=base]
KERNEL: "zImage"         # Kernel image filename
FDT: "socfpga.dtb"      # Device Tree Blob filename
APPEND: root=/dev/mmcblk0p2  # Boot options (root filesystem)
\end{lstlisting}

\textbf{Boot Process:}
\begin{enumerate}
    \item U-Boot reads \texttt{extlinux.conf}
    \item Locates kernel image file
    \item Locates device tree blob file
    \item Loads both to memory
    \item Passes control to kernel with device tree pointer
\end{enumerate}
\end{definition}

% TODO: Add image from SCD_3b_Booting_and_Device_Tree.pdf, Page 35
% Description: extlinux.conf configuration
% Priority: IMPORTANT
% Suggested filename: lecture03b_extlinux_config.png
% \includegraphics[width=\linewidth]{lecture03b_extlinux_config.png}

\begin{example}
\textbf{U-Boot Loader Output During Boot:}

\begin{enumerate}
    \item Read \texttt{extlinux.conf} configuration
    \item Load kernel image into memory
    \item Load Device Tree Blob into memory
    \item Start kernel execution
\end{enumerate}

The console output shows each step with memory addresses and file sizes, providing visibility into the boot process.
\end{example}

% TODO: Add image from SCD_3b_Booting_and_Device_Tree.pdf, Page 36
% Description: U-Boot console output during kernel boot
% Priority: IMPORTANT
% Suggested filename: lecture03b_uboot_output.png
% \includegraphics[width=\linewidth]{lecture03b_uboot_output.png}

\subsection{U-Boot Build Process}

\begin{concept}{U-Boot Build Flow}\\
Building U-Boot requires cross-compilation and configuration.

\textbf{Prerequisites:}
\begin{itemize}
    \item Cross-compiler toolchain for ARM
    \item U-Boot source code
    \item Board-specific configuration
    \item Generated handoff files from Quartus
\end{itemize}

\textbf{Build Flow:}
\begin{enumerate}
    \item Configure HPS in Platform Designer
    \item Compile design in Quartus
    \item Generate handoff folder
    \item Build preloader (SPL)
    \item Configure U-Boot for board
    \item Compile U-Boot
    \item Combine SPL and U-Boot into single image
    \item Copy to SD card
\end{enumerate}
\end{concept}

% TODO: Add image from SCD_3b_Booting_and_Device_Tree.pdf, Page 38
% Description: U-Boot build flow diagram
% Priority: CRITICAL
% Suggested filename: lecture03b_uboot_build_flow.png
% \includegraphics[width=\linewidth]{lecture03b_uboot_build_flow.png}

\subsection{Platform Designer Configuration}

\begin{KR}{Building SoC System in Platform Designer}\\
\textbf{Step 1: Configure HPS Block}
\begin{itemize}
    \item Configure HPS block settings
    \item Set up clocks and PLLs
    \item Configure SDRAM controller
    \item Enable peripherals
    \item Configure FPGA-HPS bridges
\end{itemize}

\textbf{Step 2: Add FPGA IP Blocks}
\begin{itemize}
    \item Add custom IP blocks to FPGA fabric
    \item Connect via bridges to HPS
    \item Configure IP block parameters
\end{itemize}

\textbf{Step 3: Assign Addresses}
\begin{itemize}
    \item Assign base addresses to all peripherals
    \item Create memory map
    \item Verify no address conflicts
    \item Document address assignments
\end{itemize}

\textbf{Example System Components:}
\begin{itemize}
    \item \textbf{hps\_0:} Hard Processor System block
    \item \textbf{sysid\_qsys\_0:} System ID peripheral
    \item \textbf{jtag\_uart\_0:} JTAG UART for debug
    \item \textbf{pio\_0:} Parallel I/O block
    \item Clock and reset connections
\end{itemize}
\end{KR}

% TODO: Add image from SCD_3b_Booting_and_Device_Tree.pdf, Page 39
% Description: Platform Designer system configuration
% Priority: CRITICAL
% Suggested filename: lecture03b_platform_designer.png
% \includegraphics[width=\linewidth]{lecture03b_platform_designer.png}

\subsection{SOPCINFO File}

\begin{definition}{.sopcinfo File}\\
The \texttt{.sopcinfo} file is generated by Platform Designer and contains the complete system description.

\textbf{Contents:}
\begin{itemize}
    \item Description of HPS, including all configuration settings
    \item Description of IP blocks and their parameters
    \item Interconnect topology (connections between blocks)
    \item Memory addresses and address maps
    \item Clock and reset connections
    \item Interrupt assignments
\end{itemize}

\textbf{Format:}
\begin{itemize}
    \item XML-based format
    \item Machine-readable
    \item Contains embedded C macros for software
\end{itemize}

\textbf{Example Content:}
\begin{lstlisting}[language=XML, style=base]
<kind>embeddedsw.CMacro.BIT_CLEARING_EDGE_REGISTER</kind>
<version>0</version>
<name>embeddedsw.CMacro.BIT_MODIFYING_OUTPUT_REGISTER</name>
<name>embeddedsw.CMacro.CAPTURE</name>
\end{lstlisting}

\textbf{Usage:}
\begin{itemize}
    \item Input to \texttt{sopc-create-header-files}
    \item Generates \texttt{system.h} header file
    \item Used for building bootloader
    \item Used for building device tree
\end{itemize}
\end{definition}

% TODO: Add image from SCD_3b_Booting_and_Device_Tree.pdf, Page 40
% Description: .sopcinfo file structure and content
% Priority: IMPORTANT
% Suggested filename: lecture03b_sopcinfo_file.png
% \includegraphics[width=\linewidth]{lecture03b_sopcinfo_file.png}

\raggedcolumns
\columnbreak

\subsection{Quartus Handoff Folder}

\begin{definition}{Handoff Folder}\\
Quartus compiler creates a handoff folder containing configuration information required to build the preloader and bootloader.

\textbf{Generated Files:}
\begin{itemize}
    \item \texttt{alt\_types.h} - Type definitions
    \item \texttt{emif.xml} - EMIF configuration
    \item \texttt{hps.xml} - HPS configuration
    \item \texttt{pacman\_soc\_hps\_0.hiof} - HPS I/O configuration
    \item \texttt{sdram\_io.h} - SDRAM I/O settings
    \item \texttt{sequencer.c/.h} - Memory sequencer code
    \item \texttt{sequencer\_auto*.c/.h} - Auto-generated sequencer files
    \item \texttt{sequencer\_defines.h} - Sequencer definitions
    \item \texttt{system.h} - System memory map
    \item \texttt{tclrpt.c/.h} - TCL report files
\end{itemize}

\textbf{Configuration Information:}
\begin{itemize}
    \item Pin configuration (pinmux)
    \item Enabled peripherals
    \item Clock configuration
    \item SDRAM timing and configuration
\end{itemize}
\end{definition}

% TODO: Add image from SCD_3b_Booting_and_Device_Tree.pdf, Page 41
% Description: Handoff folder contents
% Priority: IMPORTANT
% Suggested filename: lecture03b_handoff_folder.png
% \includegraphics[width=\linewidth]{lecture03b_handoff_folder.png}

\begin{definition}{hps.xml Configuration File}\\
The \texttt{hps.xml} file contains HPS peripheral and clock configuration.

\textbf{Example Contents:}
\begin{lstlisting}[language=XML, style=base]
<name>EMAC</name> <choice>value</choice>
<name>USB</name> <choice>value</choice>
<name>SPI</name> <choice>value</choice>
<name>I2C</name> <choice>value</choice>
<name>SDMMC</name> <choice>value</choice>
<name>UART</name> <choice>value</choice>
<name>CAN</name> <choice>value</choice>
<name>PLL_CLK</name> <value>...</value>
<name>SDRAM_CLK</name> <value>...</value>
\end{lstlisting}

\textbf{Configuration Categories:}
\begin{itemize}
    \item Peripheral enable/disable settings
    \item Clock frequency settings
    \item Pin multiplexing choices
    \item I/O standard selections
\end{itemize}

\important{The handoff folder contains minimal but essential configuration information.}
\end{definition}

% TODO: Add image from SCD_3b_Booting_and_Device_Tree.pdf, Page 42
% Description: hps.xml file structure
% Priority: IMPORTANT
% Suggested filename: lecture03b_hps_xml.png
% \includegraphics[width=\linewidth]{lecture03b_hps_xml.png}

\subsection{BSP Generator}

\begin{KR}{Creating Bootloader BSP (Board Support Package)}\\
\textbf{Tool: bsp-create-settings}

The BSP generator converts the handoff folder to header files for bootloader compilation.

\textbf{Command:}
\begin{lstlisting}[language=bash, style=base]
cd $PROJECT/
mkdir -p software/bootloader

bsp-create-settings \
  --type spl \
  --bsp-dir software/bootloader \
  --preloader-settings-dir "hps_isw_handoff/pacman_soc_hps_0" \
  --settings software/bootloader/settings.bsp
\end{lstlisting}

\textbf{Generated Files:}
\begin{itemize}
    \item Header files in \texttt{generated/} directory
    \item Pin multiplexing configuration
    \item PLL configuration
    \item Reset manager configuration
    \item SDRAM initialization sequences
\end{itemize}

\textbf{Output Directory:}
\begin{itemize}
    \item \texttt{software/bootloader/generated/}
    \item Contains all necessary headers
    \item Ready for preloader compilation
\end{itemize}
\end{KR}

% TODO: Add image from SCD_3b_Booting_and_Device_Tree.pdf, Page 43
% Description: BSP generator command and process
% Priority: CRITICAL
% Suggested filename: lecture03b_bsp_generator.png
% \includegraphics[width=\linewidth]{lecture03b_bsp_generator.png}

\subsection{U-Boot Compilation}

\begin{KR}{Cross-Compiling U-Boot}\\
\textbf{Prerequisites:}
\begin{itemize}
    \item ARM cross-compiler toolchain installed
    \item U-Boot source code (e.g., socfpga\_v2020.04 branch)
    \item Modified source code for your board (if needed)
\end{itemize}

\textbf{Step 1: Configure for Target Board}
\begin{lstlisting}[language=bash, style=base]
cd ~/de1soc/
# Select default configuration for Cyclone V
make socfpga_cyclone5_defconfig
# Output: configuration written to .config
\end{lstlisting}

\textbf{Step 2: Compile}
\begin{lstlisting}[language=bash, style=base]
make
# Compiles all components:
# - scripts/basic/fixdep
# - scripts/kconfig/conf.o
# - SPL (preloader)
# - U-Boot proper
# Output: u-boot-with-spl.sfp
\end{lstlisting}

\textbf{Build Output Files:}
\begin{itemize}
    \item \texttt{u-boot-with-spl.sfp} - Combined preloader + U-Boot image
    \item Ready to copy to SD card
\end{itemize}

\textbf{Customization:}
\begin{itemize}
    \item Modify source code as needed for custom boards
    \item Adjust board-specific settings
    \item Add custom commands or features
\end{itemize}
\end{KR}

% TODO: Add image from SCD_3b_Booting_and_Device_Tree.pdf, Page 44
% Description: U-Boot compilation process and output
% Priority: CRITICAL
% Suggested filename: lecture03b_uboot_compilation.png
% \includegraphics[width=\linewidth]{lecture03b_uboot_compilation.png}

\subsection{Copying to SD Card}

\begin{KR}{Writing Bootloader to SD Card}\\
\textbf{Important Note:} The preloader partition has no filesystem - it contains raw binary data.

\textbf{Command:}
\begin{lstlisting}[language=bash, style=base]
# Copy bootloader image to raw partition (type A2)
dd if=u-boot-with-spl.sfp of=/dev/sdx3 bs=1M
sync
\end{lstlisting}

\textbf{Parameters:}
\begin{itemize}
    \item \texttt{if=}: Input file (bootloader image)
    \item \texttt{of=}: Output device (\texttt{/dev/sdx3} where x is SD card device letter)
    \item \texttt{bs=1M}: Block size of 1 megabyte
    \item \texttt{sync}: Ensure data is written to disk
\end{itemize}

\textbf{Important:}
\begin{itemize}
    \item The bootloader image contains 4 copies of the preloader
    \item Correctly formatted for raw partition
    \item No filesystem needed
    \item Direct binary copy to partition
\end{itemize}

\important{Always verify the correct device name (sdx) before running dd command to avoid overwriting wrong disk!}
\end{KR}

% TODO: Add image from SCD_3b_Booting_and_Device_Tree.pdf, Page 45
% Description: SD card write command
% Priority: IMPORTANT
% Suggested filename: lecture03b_sd_card_write.png
% \includegraphics[width=\linewidth]{lecture03b_sd_card_write.png}

% ===== IMAGE SUMMARY =====
% Total images needed: 37
% CRITICAL priority: 20
% IMPORTANT priority: 14
% SUPPLEMENTARY priority: 3
%
% Quick extraction checklist:
% [ ] [SCD_3b_Booting_and_Device_Tree.pdf, Page 5] - SoC architecture diagram (CRITICAL)
% [ ] [SCD_3b_Booting_and_Device_Tree.pdf, Page 6] - HPS I/O pins (IMPORTANT)
% [ ] [SCD_3b_Booting_and_Device_Tree.pdf, Page 8] - Boot sequence overview (CRITICAL)
% [ ] [SCD_3b_Booting_and_Device_Tree.pdf, Page 9] - Reset stage (IMPORTANT)
% [ ] [SCD_3b_Booting_and_Device_Tree.pdf, Page 10] - Reset details (SUPPLEMENTARY)
% [ ] [SCD_3b_Booting_and_Device_Tree.pdf, Page 11] - Boot ROM stage (CRITICAL)
% [ ] [SCD_3b_Booting_and_Device_Tree.pdf, Page 12] - Boot ROM details (SUPPLEMENTARY)
% [ ] [SCD_3b_Booting_and_Device_Tree.pdf, Page 13] - BSEL configuration (CRITICAL)
% [ ] [SCD_3b_Booting_and_Device_Tree.pdf, Page 14] - SD card structure (CRITICAL)
% [ ] [SCD_3b_Booting_and_Device_Tree.pdf, Page 15] - Preloader stage (CRITICAL)
% [ ] [SCD_3b_Booting_and_Device_Tree.pdf, Page 16] - Preloader details (SUPPLEMENTARY)
% [ ] [SCD_3b_Booting_and_Device_Tree.pdf, Page 17] - Bootloader stage (CRITICAL)
% [ ] [SCD_3b_Booting_and_Device_Tree.pdf, Page 18] - U-Boot capabilities (IMPORTANT)
% [ ] [SCD_3b_Booting_and_Device_Tree.pdf, Page 19] - OS stage (IMPORTANT)
% [ ] [SCD_3b_Booting_and_Device_Tree.pdf, Page 20] - Application stage (IMPORTANT)
% [ ] [SCD_3b_Booting_and_Device_Tree.pdf, Page 21] - Boot options (IMPORTANT)
% [ ] [SCD_3b_Booting_and_Device_Tree.pdf, Page 24] - Device tree concept (CRITICAL)
% [ ] [SCD_3b_Booting_and_Device_Tree.pdf, Page 25] - Device tree memory map (CRITICAL)
% [ ] [SCD_3b_Booting_and_Device_Tree.pdf, Page 26] - Device tree in filesystem (IMPORTANT)
% [ ] [SCD_3b_Booting_and_Device_Tree.pdf, Page 27] - Device tree hierarchy (CRITICAL)
% [ ] [SCD_3b_Booting_and_Device_Tree.pdf, Page 28] - Node structure (CRITICAL)
% [ ] [SCD_3b_Booting_and_Device_Tree.pdf, Page 29] - Complete node example (CRITICAL)
% [ ] [SCD_3b_Booting_and_Device_Tree.pdf, Page 30] - Node properties (CRITICAL)
% [ ] [SCD_3b_Booting_and_Device_Tree.pdf, Page 31] - Bus properties (CRITICAL)
% [ ] [SCD_3b_Booting_and_Device_Tree.pdf, Page 32] - Memory ranges (CRITICAL)
% [ ] [SCD_3b_Booting_and_Device_Tree.pdf, Page 33] - Device tree build (CRITICAL)
% [ ] [SCD_3b_Booting_and_Device_Tree.pdf, Page 34] - DT compilation flow (CRITICAL)
% [ ] [SCD_3b_Booting_and_Device_Tree.pdf, Page 35] - extlinux config (IMPORTANT)
% [ ] [SCD_3b_Booting_and_Device_Tree.pdf, Page 36] - U-Boot output (IMPORTANT)
% [ ] [SCD_3b_Booting_and_Device_Tree.pdf, Page 38] - U-Boot build flow (CRITICAL)
% [ ] [SCD_3b_Booting_and_Device_Tree.pdf, Page 39] - Platform Designer (CRITICAL)
% [ ] [SCD_3b_Booting_and_Device_Tree.pdf, Page 40] - sopcinfo file (IMPORTANT)
% [ ] [SCD_3b_Booting_and_Device_Tree.pdf, Page 41] - Handoff folder (IMPORTANT)
% [ ] [SCD_3b_Booting_and_Device_Tree.pdf, Page 42] - hps.xml file (IMPORTANT)
% [ ] [SCD_3b_Booting_and_Device_Tree.pdf, Page 43] - BSP generator (CRITICAL)
% [ ] [SCD_3b_Booting_and_Device_Tree.pdf, Page 44] - U-Boot compilation (CRITICAL)
% [ ] [SCD_3b_Booting_and_Device_Tree.pdf, Page 45] - SD card write (IMPORTANT)
% =====================