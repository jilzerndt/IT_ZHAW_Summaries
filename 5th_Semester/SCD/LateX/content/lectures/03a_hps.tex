\section{Hard Processor System (HPS) in SoC FPGA}

\subsection{Learning Objectives}

\begin{concept}{Course Goals}\\
After this lecture and lab exercise, students will be able to:
\begin{itemize}
    \item Explain why it is often required to execute software in a SoC
    \item Explain the difference between a softcore and a hard processor system
    \item Name advantages for both softcore and hard processor options
    \item Configure a HPS and/or a softcore processor
    \item Apply the principle of bus bridges to access components in different chip parts
\end{itemize}
\end{concept}

\subsection{FPGA Applications and Software Requirements}

\begin{concept}{Typical FPGA Applications}\\
FPGAs excel at tasks requiring deterministic, real-time processing:

\textbf{Data Processing:}
\begin{itemize}
    \item Filtering and triggering
    \item Control loops with precise timing
    \item Real-time signal processing
\end{itemize}

\textbf{Video Processing:}
\begin{itemize}
    \item Color space transformation
    \item Compression and decompression
    \item Filters and effects
    \item Real-time video manipulation
\end{itemize}

\textbf{Key Advantage:} Deterministic latency - guaranteed response times
\end{concept}

% TODO: Add image from SCD_3a_HPS.pdf, Page 4
% Description: Typical FPGA applications
% Priority: IMPORTANT
% Suggested filename: lecture03a_fpga_applications.png
% \includegraphics[width=\linewidth]{lecture03a_fpga_applications.png}

\begin{concept}{Tasks Better Solved in Software}\\
Many system functions are more efficiently implemented in software:

\textbf{Complex Algorithms:}
\begin{itemize}
    \item Video processing filters requiring coefficient calculation over multiple frames
    \item Complex decision-making algorithms
    \item Machine learning inference
\end{itemize}

\textbf{Networking and Communication:}
\begin{itemize}
    \item Web server with configuration controls
    \item Network stack for data transfer via internet
    \item Protocol implementations (TCP/IP, HTTP, etc.)
\end{itemize}

\textbf{System Management:}
\begin{itemize}
    \item API for easier control of FPGA application
    \item User interface and configuration management
    \item File system and data logging
\end{itemize}

\important{Combining FPGA hardware acceleration with CPU software flexibility provides optimal system performance.}
\end{concept}

% TODO: Add image from SCD_3a_HPS.pdf, Page 5
% Description: Software tasks in SoC systems
% Priority: IMPORTANT
% Suggested filename: lecture03a_software_tasks.png
% \includegraphics[width=\linewidth]{lecture03a_software_tasks.png}

\raggedcolumns
\columnbreak

\subsection{Evolution from Discrete Systems to SoC}

\begin{concept}{Classic FPGA \& CPU Architecture}\\
Traditional systems use separate FPGA and CPU chips.

\textbf{System Before (Discrete Architecture):}
\begin{itemize}
    \item \textbf{Multiple Devices:} Separate FPGA and CPU chips
    \item \textbf{Multiple Memories:} Different memory for each device
    \item \textbf{Limited Bandwidth:} Chip-to-chip communication bottleneck
    \item \textbf{Complex PCB:} More layers, more routing
    \item \textbf{Higher Costs:} Multiple chips, complex board
    \item \textbf{Power Consumption:} Data transfer between chips costs energy
    \item \textbf{Speed:} Inter-chip communication is slow
\end{itemize}

\textbf{Layout:}
One PCB contains discrete FPGA \& CPU as separate components.
\end{concept}

% TODO: Add image from SCD_3a_HPS.pdf, Page 6
% Description: Classic discrete FPGA and CPU architecture
% Priority: CRITICAL
% Suggested filename: lecture03a_discrete_architecture.png
% \includegraphics[width=\linewidth]{lecture03a_discrete_architecture.png}

\begin{definition}{SoC FPGA Integrated Architecture}\\
Modern SoC FPGAs integrate CPU and FPGA on a single die.

\textbf{System After (SoC FPGA):}
\begin{itemize}
    \item \textbf{Single Device:} CPU and FPGA in one chip
    \item \textbf{Single Memory:} Shared memory system
    \item \textbf{High Bandwidth:} Internal connections between FPGA and CPU
    \item \textbf{Simpler PCB:} Fewer layers required
    \item \textbf{Lower Costs:} Single chip solution
    \item \textbf{Less Power:} No inter-chip communication overhead
    \item \textbf{Better Performance:} Fast internal data paths
\end{itemize}

\important{SoC FPGA architecture provides significant advantages in bandwidth, power, cost, and PCB complexity.}
\end{definition}

% TODO: Add image from SCD_3a_HPS.pdf, Page 10
% Description: SoC FPGA integrated architecture
% Priority: CRITICAL
% Suggested filename: lecture03a_soc_architecture.png
% \includegraphics[width=\linewidth]{lecture03a_soc_architecture.png}

\subsection{Evolution Stages of SoC FPGA}

\begin{definition}{Stage 1: Pure FPGA}\\
Original FPGA architecture with basic programmable logic.

\textbf{Components:}
\begin{itemize}
    \item \textbf{Programmable Logic:} LUTs and flip-flops
    \item \textbf{GPIO:} General-purpose I/O pins
    \item \textbf{On-Chip Memory:} SRAM blocks
    \item \textbf{Clock Generation:} PLLs for clock management
    \item \textbf{Multipliers:} Dedicated DSP blocks
    \item \textbf{Fast Serial I/O:} Gigabit transceivers (SERDES)
\end{itemize}

\textbf{Capabilities:}
\begin{itemize}
    \item Hardware acceleration
    \item Parallel processing
    \item Real-time deterministic operation
\end{itemize}
\end{definition}

% TODO: Add image from SCD_3a_HPS.pdf, Page 7
% Description: Pure FPGA architecture
% Priority: IMPORTANT
% Suggested filename: lecture03a_pure_fpga.png
% \includegraphics[width=\linewidth]{lecture03a_pure_fpga.png}

\begin{definition}{Stage 2: FPGA with Softcore Processor}\\
Add programmable CPU implemented in FPGA fabric.

\textbf{Softcore Processor Characteristics:}
\begin{itemize}
    \item \textbf{Implementation:} IP core in FPGA fabric (e.g., NIOS V, RISC-V)
    \item \textbf{Software Execution:} Can run software inside FPGA
    \item \textbf{Clock Speed:} Relatively slow CPU clock (limited by fabric)
    \item \textbf{Memory:} Small on-chip memory or external memory via I/O pins
    \item \textbf{Resource Usage:} Consumes FPGA logic resources
    \item \textbf{Flexibility:} Configurable architecture (instruction set, peripherals)
\end{itemize}

\textbf{Advantages:}
\begin{itemize}
    \item No external CPU required
    \item Customizable processor architecture
    \item Tight integration with FPGA logic
\end{itemize}

\textbf{Disadvantages:}
\begin{itemize}
    \item Uses valuable FPGA resources
    \item Lower performance than hard CPU
    \item Limited memory bandwidth
\end{itemize}
\end{definition}

% TODO: Add image from SCD_3a_HPS.pdf, Page 8
% Description: FPGA with softcore processor
% Priority: CRITICAL
% Suggested filename: lecture03a_softcore_processor.png
% \includegraphics[width=\linewidth]{lecture03a_softcore_processor.png}

\begin{definition}{Stage 3: SoC with Hard Processor Core}\\
Integrate dedicated hardware CPU with FPGA fabric.

\textbf{Hard Processor System (HPS) Characteristics:}
\begin{itemize}
    \item \textbf{Full Integrated Processor:} Dedicated silicon, not programmable fabric
    \item \textbf{Fast CPU Clock:} Up to 1.0 GHz or higher
    \item \textbf{Dedicated Peripherals:} Hardware I/O controllers
    \item \textbf{Dedicated Pins:} Separate pins for CPU peripherals
    \item \textbf{Memory Controller:} Dedicated SDRAM controller to external memory
    \item \textbf{High Bandwidth:} Fast data transfers between FPGA and CPU
    \item \textbf{No FPGA Resources:} Doesn't consume programmable logic
\end{itemize}

\textbf{Advantages:}
\begin{itemize}
    \item High CPU performance
    \item Full operating system support
    \item Preserves all FPGA resources
    \item Professional peripherals (Ethernet, USB, PCIe)
\end{itemize}
\end{definition}

% TODO: Add image from SCD_3a_HPS.pdf, Page 9
% Description: SoC with hard processor core
% Priority: CRITICAL
% Suggested filename: lecture03a_hard_processor.png
% \includegraphics[width=\linewidth]{lecture03a_hard_processor.png}

\raggedcolumns
\columnbreak

\subsection{System-Level Benefits of SoC FPGA}

\begin{highlight}{Performance Advantages}\\
SoC FPGAs provide exceptional computational capabilities:

\textbf{Processing Power:}
\begin{itemize}
    \item \textbf{1,600 GMACS:} Multiply-accumulate operations in FPGA
    \item \textbf{320 GFLOPS:} Floating-point operations in DSP multipliers
    \item \textbf{125 Gbps:} Interconnect bandwidth between CPU and FPGA
\end{itemize}

\textbf{Memory Architecture:}
\begin{itemize}
    \item \textbf{Cache Coherent Access:} FPGA can access CPU caches
    \item \textbf{Shared Memory:} Unified memory space
    \item \textbf{DMA Support:} Direct memory access for high-speed transfers
\end{itemize}

\textbf{CPU Integration:}
\begin{itemize}
    \item \textbf{Custom Instructions:} Hardware accelerators appear as CPU instructions
    \item \textbf{Tight Coupling:} Minimal latency between CPU and FPGA
    \item \textbf{Coherent Fabric:} Automatic cache management
\end{itemize}

\important{The combination of high-speed interconnects and cache coherency enables unprecedented CPU-FPGA cooperation.}
\end{highlight}

% TODO: Add image from SCD_3a_HPS.pdf, Page 11
% Description: System-level performance benefits
% Priority: CRITICAL
% Suggested filename: lecture03a_system_benefits.png
% \includegraphics[width=\linewidth]{lecture03a_system_benefits.png}

\begin{concept}{Processor-Level Benefits}\\
The HPS provides full-featured processor capabilities:

\textbf{CPU Features:}
\begin{itemize}
    \item \textbf{Fast Multicore:} Up to 1 GHz or higher clock speed
    \item \textbf{Full OS Support:} Can run Linux, FreeRTOS, etc.
    \item \textbf{MMU:} Memory Management Unit for virtual memory
    \item \textbf{Cache Hierarchy:} L1 and L2 caches for performance
\end{itemize}

\textbf{Peripheral Expansion:}
\begin{itemize}
    \item Ability to expand CPU peripherals by placing IP blocks in FPGA
    \item Add controllers: SD, I$^2$C, CAN, UART, Ethernet, USB
    \item Custom interface protocols
    \item Flexible I/O configuration
\end{itemize}

\textbf{Hardware Acceleration:}
\begin{itemize}
    \item \textbf{Custom Accelerators:} Complex calculations offloaded to FPGA
    \item \textbf{Parallel Execution:} CPU continues other tasks while FPGA computes
    \item \textbf{Direct Memory Access:} DMA for efficient data transfer
    \item \textbf{Cache Coherency:} Automatic synchronization between CPU and FPGA
\end{itemize}
\end{concept}

% TODO: Add image from SCD_3a_HPS.pdf, Page 12
% Description: Processor-level benefits
% Priority: IMPORTANT
% Suggested filename: lecture03a_processor_benefits.png
% \includegraphics[width=\linewidth]{lecture03a_processor_benefits.png}

\subsection{SoC FPGA Block Architecture}

\begin{definition}{SoC FPGA Main Components}\\
A SoC FPGA consists of three main sections:

\textbf{1. CPU Portion (Hard Processor System):}
\begin{itemize}
    \item Cortex-A9 CPU Subsystem (dual-core)
    \item Flash Controllers
    \item SDRAM Controller Subsystem
    \item On-Chip Memories
    \item Support Peripherals
    \item PLLs (Phase-Locked Loops)
    \item Debug Interface
    \item Peripheral Control Block
\end{itemize}

\textbf{2. FPGA Portion:}
\begin{itemize}
    \item User I/O pins
    \item FPGA Fabric (LUTs, RAMs, Multipliers, Routing)
    \item HSSI Transceivers (High-Speed Serial Interface)
    \item PLLs
    \item Hard PCIe controller
    \item Hard Memory Controllers
\end{itemize}

\textbf{3. Processor-FPGA Bridges:}
\begin{itemize}
    \item High-bandwidth connections
    \item Multiple bridge types for different use cases
    \item Enable communication between CPU and FPGA
\end{itemize}
\end{definition}

% TODO: Add image from SCD_3a_HPS.pdf, Page 14
% Description: SoC FPGA block diagram
% Priority: CRITICAL
% Suggested filename: lecture03a_soc_blocks.png
% \includegraphics[width=\linewidth]{lecture03a_soc_blocks.png}

\begin{concept}{SoC FPGA Architecture Details}\\
\textbf{Key Characteristics:}

\textbf{Three Main Parts:}
\begin{itemize}
    \item FPGA fabric
    \item CPU (Hard Processor System)
    \item Bridges connecting them
\end{itemize}

\textbf{FPGA Independence:}
\begin{itemize}
    \item FPGA fabric is same as in FPGA-only devices
    \item FPGA may be used without CPU (some manufacturers)
    \item CPU can be powered down if not needed
\end{itemize}

\textbf{CPU Options:}
\begin{itemize}
    \item ARM Cortex-A9, -A53, -A78 (dual- to octa-core)
    \item ARM Cortex-R52 (real-time)
    \item RISC-V processors also available
\end{itemize}

\textbf{Interconnect Features:}
\begin{itemize}
    \item Multiple bridges provide high bandwidth
    \item Multiport SDRAM controller handles concurrent accesses
    \item DMA controller for efficient data movement
\end{itemize}
\end{concept}

% TODO: Add image from SCD_3a_HPS.pdf, Page 15
% Description: SoC FPGA architecture details
% Priority: IMPORTANT
% Suggested filename: lecture03a_architecture_details.png
% \includegraphics[width=\linewidth]{lecture03a_architecture_details.png}

\raggedcolumns
\columnbreak

\subsection{Detailed SoC FPGA Architecture}

\begin{definition}{ARM Cortex-A9 CPU Subsystem}\\
The HPS is built around ARM Cortex-A9 processor cores.

\textbf{CPU Features:}
\begin{itemize}
    \item \textbf{Dual-Core:} Two ARM Cortex-A9 processors
    \item \textbf{L1 Cache:} Per-core Level 1 cache (instruction \& data)
    \item \textbf{L2 Cache:} Shared Level 2 cache for both cores
    \item \textbf{FPU:} Floating-Point Unit per core
    \item \textbf{NEON:} SIMD (Single Instruction Multiple Data) instruction set
\end{itemize}

\textbf{Pin Configuration:}
\begin{itemize}
    \item \textbf{Dedicated MPU Pins:} For CPU peripherals
    \item \textbf{Dedicated DDR Pins:} For external SDRAM
    \item \textbf{FPGA Pins:} Shared or dedicated for FPGA I/O
\end{itemize}

\textbf{Interconnect:}
\begin{itemize}
    \item System bus: Pipelined network between components
    \item High-bandwidth internal connections
    \item Multiport DDR controller for concurrent access
\end{itemize}
\end{definition}

% TODO: Add image from SCD_3a_HPS.pdf, Page 16
% Description: Detailed SoC FPGA architecture diagram
% Priority: CRITICAL
% Suggested filename: lecture03a_detailed_architecture.png
% \includegraphics[width=\linewidth]{lecture03a_detailed_architecture.png}

% TODO: Add image from SCD_3a_HPS.pdf, Page 17
% Description: CPU subsystem architecture with caches
% Priority: IMPORTANT
% Suggested filename: lecture03a_cpu_subsystem.png
% \includegraphics[width=\linewidth]{lecture03a_cpu_subsystem.png}

\begin{definition}{HPS Peripherals and Controllers}\\
The HPS includes comprehensive peripheral support:

\textbf{Standard Microcontroller Peripherals:}
\begin{itemize}
    \item Timers (watchdog, general-purpose)
    \item SPI (Serial Peripheral Interface)
    \item I$^2$C (Inter-Integrated Circuit)
    \item UART (Universal Asynchronous Receiver-Transmitter)
    \item GPIO (General-Purpose I/O)
\end{itemize}

\textbf{High-Speed Interfaces:}
\begin{itemize}
    \item \textbf{Gigabit Ethernet:} To external PHY
    \item \textbf{PCIe:} PCI Express controller
    \item \textbf{USB:} USB 2.0 OTG controller
\end{itemize}

\textbf{Memory Interfaces:}
\begin{itemize}
    \item \textbf{Static Memory Controller:} For QSPI NAND/NOR Flash, SD-card
    \item \textbf{Multiport DDR Controller:} For external SDRAM
\end{itemize}

\textbf{System Components:}
\begin{itemize}
    \item \textbf{DMA:} Direct Memory Access controller
    \item \textbf{Scratch RAM:} Used in boot process, sufficient for FreeRTOS
    \item \textbf{Boot ROM:} Contains boot code
\end{itemize}
\end{definition}

% TODO: Add image from SCD_3a_HPS.pdf, Page 18
% Description: HPS peripherals and interfaces
% Priority: IMPORTANT
% Suggested filename: lecture03a_hps_peripherals.png
% \includegraphics[width=\linewidth]{lecture03a_hps_peripherals.png}

\subsection{FPGA-HPS Bridges}

\begin{definition}{Altera FPGA-HPS Bridge Architecture}\\
Three main communication paths connect FPGA and HPS:

\textbf{1. General Purpose AXI Bridges:}
\begin{itemize}
    \item \textbf{H2F (HPS-to-FPGA):} CPU accesses FPGA components
    \item \textbf{F2H (FPGA-to-HPS):} FPGA accesses HPS memory/peripherals
    \item \textbf{Width:} Up to 128-bit wide
    \item \textbf{Modes:} Streaming and memory-mapped
    \item \textbf{Use:} High-bandwidth data transfers
\end{itemize}

\textbf{2. Lightweight AXI Bridge:}
\begin{itemize}
    \item \textbf{Width:} 32-bit
    \item \textbf{Mode:} Memory-mapped access only
    \item \textbf{Use:} Control registers, configuration, status
    \item \textbf{Latency:} Lower latency than general purpose bridges
\end{itemize}

\textbf{3. Multiport DDR Controller:}
\begin{itemize}
    \item \textbf{Width:} 256-bit
    \item \textbf{Function:} FPGA direct access to external SDRAM
    \item \textbf{Feature:} Cache coherent with CPU
    \item \textbf{Use:} Shared memory between CPU and FPGA
\end{itemize}

\important{Different bridges are optimized for different use cases: lightweight for control, general purpose for data, DDR for shared memory.}
\end{definition}

% TODO: Add image from SCD_3a_HPS.pdf, Page 19
% Description: FPGA-HPS bridge architecture diagram
% Priority: CRITICAL
% Suggested filename: lecture03a_bridges.png
% \includegraphics[width=\linewidth]{lecture03a_bridges.png}

\begin{example2}{Address Mapping Example}\\
\textbf{Memory Map:}

\begin{center}
\begin{tabular}{|l|l|l|}
\hline
\textbf{Region Name} & \textbf{Base Address} & \textbf{Size} \\
\hline
Peripherals & 0xFC00\_0000 & 62 MB \\
Lightweight FPGA slaves & 0xFF20\_0000 & 2 MB \\
\hline
\end{tabular}
\end{center}

\tcblower

\textbf{Example: Control Register in FPGA}

\textbf{CPU View (Full Address):}
$$\text{0xFF22\_0030}$$

This address selects the lightweight bridge (starts with 0xFF20\_xxxx).

\textbf{FPGA View (Local Address):}
$$\text{0x0002\_0030}$$

The bridge translates the address:
\begin{itemize}
    \item Remove base address: 0xFF20\_0000
    \item Remaining offset: 0x0002\_0030
    \item This is the address seen by FPGA component
\end{itemize}

\important{Address translation by bridges allows FPGA components to use simple local addresses while appearing in CPU memory map.}
\end{example2}

% TODO: Add image from SCD_3a_HPS.pdf, Page 20
% Description: Address mapping example with bridge translation
% Priority: CRITICAL
% Suggested filename: lecture03a_address_mapping.png
% \includegraphics[width=\linewidth]{lecture03a_address_mapping.png}

\raggedcolumns
\columnbreak

\subsection{SoC FPGA Configuration}

\begin{concept}{Three Types of Components in SoC FPGA}\\
SoC FPGAs contain different types of configurable components:

\textbf{1. CPU Hard Macros (Silicon):}
\begin{itemize}
    \item CPU cores, SDRAM controller
    \item Processor-FPGA bridges
    \item Fixed hardware, not reconfigurable
    \item \textbf{Configuration Method:} Platform Designer
\end{itemize}

\textbf{2. FPGA Hard Macros (Silicon):}
\begin{itemize}
    \item Gigabit transceivers (HSSI)
    \item PLLs (Phase-Locked Loops)
    \item PCIe controller
    \item I/O blocks, multipliers, memory blocks
    \item \textbf{Configuration Method:} Wizards (generate VHDL or config files)
\end{itemize}

\textbf{3. Soft Macros (Programmable):}
\begin{itemize}
    \item I/O interfaces, PLLs
    \item Custom logic in FPGA fabric
    \item IP blocks from library
    \item \textbf{Configuration Method:} 
    \begin{itemize}
        \item Soft IP blocks in Platform Designer
        \item Custom VHDL code
    \end{itemize}
\end{itemize}
\end{concept}

% TODO: Add image from SCD_3a_HPS.pdf, Page 22
% Description: Three types of SoC FPGA components
% Priority: CRITICAL
% Suggested filename: lecture03a_component_types.png
% \includegraphics[width=\linewidth]{lecture03a_component_types.png}

\subsection{Platform Designer HPS Configuration}

\begin{definition}{Platform Designer HPS Block}\\
Platform Designer provides a graphical interface for configuring the HPS.

\textbf{Available Interfaces:}
\begin{itemize}
    \item \textbf{memory:} Memory interface conduit
    \item \textbf{hps\_io:} HPS I/O conduit
    \item \textbf{h2f\_reset:} Reset output from HPS
    \item \textbf{h2f\_axi\_master:} HPS-to-FPGA AXI master
    \item \textbf{f2h\_axi\_slave:} FPGA-to-HPS AXI slave
    \item \textbf{h2f\_lw\_axi\_master:} Lightweight HPS-to-FPGA AXI master
    \item \textbf{f2h\_irq0/1:} Interrupt receivers (32 interrupts each)
\end{itemize}

\textbf{Clock Inputs:}
\begin{itemize}
    \item Separate clocks for each AXI interface
    \item Allows different clock domains
\end{itemize}
\end{definition}

% TODO: Add image from SCD_3a_HPS.pdf, Page 23
% Description: Platform Designer HPS block interface
% Priority: CRITICAL
% Suggested filename: lecture03a_platform_designer_hps.png
% \includegraphics[width=\linewidth]{lecture03a_platform_designer_hps.png}

\begin{KR}{Configuring HPS in Platform Designer}\\
\textbf{Configuration Tabs:}

\textbf{1. FPGA Interfaces:}
\begin{itemize}
    \item $\square$ Enable MPU standby and event signals
    \item $\square$ Enable general purpose signals
    \item $\square$ Enable Debug APB interface
    \item $\square$ Enable System Trace Macrocell hardware events
    \item $\square$ Enable FPGA Cross Trigger interface
    \item $\square$ Enable FPGA Trace Port Interface Unit
    \item $\square$ Enable boot from FPGA signals
    \item $\square$ Enable HLGPI Interface
\end{itemize}

\textbf{2. AXI Bridges:}
\begin{itemize}
    \item \textbf{FPGA-to-HPS width:} 64-bit, 128-bit, or 256-bit
    \item \textbf{HPS-to-FPGA width:} 32-bit, 64-bit, or 128-bit
    \item \textbf{Lightweight width:} 32-bit (fixed)
\end{itemize}

\textbf{3. Peripheral Pins:}
\begin{itemize}
    \item Select pin-set for each peripheral
    \item Enable or disable peripherals
    \item Configure pin multiplexing
\end{itemize}

\textbf{4. SDRAM Configuration:}
\begin{itemize}
    \item DDR3/DDR2 parameters
    \item Memory timing configuration
    \item Port assignments
\end{itemize}

\textbf{5. HPS Clocks:}
\begin{itemize}
    \item Clock source selection
    \item PLL configuration
    \item Clock dividers
\end{itemize}
\end{KR}

\subsection{Hardware Development Flow}

\begin{concept}{Platform Designer Role}\\
Platform Designer is Intel's system builder for SoC designs.

\textbf{Configuration Scope:}
\begin{itemize}
    \item \textbf{CPU and Peripherals:} Select and configure HPS features
    \item \textbf{IP Blocks:} Add and configure FPGA IP components
    \item \textbf{Interconnects:} Automatic generation of bus connections
\end{itemize}

\textbf{Output Products:}
\begin{itemize}
    \item \textbf{Hardware Description:} Memory map, device types
    \item \textbf{For Software:} Header files with addresses and registers
    \item \textbf{For FPGA:} VHDL files for interconnects and peripherals
    \item \textbf{System Configuration:} Stored project settings
\end{itemize}

\textbf{IP Implementations:}
\begin{itemize}
    \item peripheral\_1.vhdl
    \item peripheral\_n.vhdl
    \item Interconnect fabric
    \item Address decoders
\end{itemize}
\end{concept}

% TODO: Add image from SCD_3a_HPS.pdf, Page 24
% Description: Hardware development flow with Platform Designer
% Priority: CRITICAL
% Suggested filename: lecture03a_dev_flow.png
% \includegraphics[width=\linewidth]{lecture03a_dev_flow.png}

\begin{KR}{Complete Hardware Development Flow}\\
\textbf{Step 1: System Definition (Platform Designer)}
\begin{itemize}
    \item Configure HPS and peripherals
    \item Add and configure IP blocks
    \item Define interconnections
    \item Generate system
\end{itemize}

\textbf{Step 2: Quartus Compilation}
\begin{itemize}
    \item \textbf{Synthesis:} Convert VHDL to gate-level netlist
    \item \textbf{Optimization:} Minimize logic usage
    \item \textbf{Mapping:} Map to FPGA resources
    \item \textbf{Placement:} Place components on chip
    \item \textbf{Routing:} Route connections between components
    \item \textbf{Retiming:} Optimize for timing
    \item \textbf{Fitting:} Final placement optimization
\end{itemize}

\textbf{Step 3: Timing Verification}
\begin{itemize}
    \item Static timing analysis
    \item Check all timing constraints met
    \item Identify critical paths
\end{itemize}

\textbf{Step 4: Bitstream Generation}
\begin{itemize}
    \item Generate configuration bitstream
    \item Include HPS parameters
    \item Program into device
\end{itemize}
\end{KR}

% TODO: Add image from SCD_3a_HPS.pdf, Page 25
% Description: Quartus compilation flow
% Priority: IMPORTANT
% Suggested filename: lecture03a_quartus_flow.png
% \includegraphics[width=\linewidth]{lecture03a_quartus_flow.png}

% ===== IMAGE SUMMARY =====
% Total images needed: 17
% CRITICAL priority: 10
% IMPORTANT priority: 7
% SUPPLEMENTARY priority: 0
%
% Quick extraction checklist:
% [ ] [SCD_3a_HPS.pdf, Page 4] - Typical FPGA applications (IMPORTANT)
% [ ] [SCD_3a_HPS.pdf, Page 5] - Software tasks in SoC (IMPORTANT)
% [ ] [SCD_3a_HPS.pdf, Page 6] - Discrete architecture (CRITICAL)
% [ ] [SCD_3a_HPS.pdf, Page 7] - Pure FPGA stage (IMPORTANT)
% [ ] [SCD_3a_HPS.pdf, Page 8] - Softcore processor stage (CRITICAL)
% [ ] [SCD_3a_HPS.pdf, Page 9] - Hard processor stage (CRITICAL)
% [ ] [SCD_3a_HPS.pdf, Page 10] - SoC integrated architecture (CRITICAL)
% [ ] [SCD_3a_HPS.pdf, Page 11] - System-level benefits (CRITICAL)
% [ ] [SCD_3a_HPS.pdf, Page 12] - Processor-level benefits (IMPORTANT)
% [ ] [SCD_3a_HPS.pdf, Page 14] - SoC FPGA blocks (CRITICAL)
% [ ] [SCD_3a_HPS.pdf, Page 15] - Architecture details (IMPORTANT)
% [ ] [SCD_3a_HPS.pdf, Page 16] - Detailed architecture diagram (CRITICAL)
% [ ] [SCD_3a_HPS.pdf, Page 17] - CPU subsystem with caches (IMPORTANT)
% [ ] [SCD_3a_HPS.pdf, Page 18] - HPS peripherals (IMPORTANT)
% [ ] [SCD_3a_HPS.pdf, Page 19] - Bridge architecture (CRITICAL)
% [ ] [SCD_3a_HPS.pdf, Page 20] - Address mapping example (CRITICAL)
% [ ] [SCD_3a_HPS.pdf, Page 22] - Component types (CRITICAL)
% [ ] [SCD_3a_HPS.pdf, Page 23] - Platform Designer HPS (CRITICAL)
% [ ] [SCD_3a_HPS.pdf, Page 24] - Development flow (CRITICAL)
% [ ] [SCD_3a_HPS.pdf, Page 25] - Quartus compilation (IMPORTANT)
% =====================