\section{Timing Analysis Part 1}

\subsection{Introduction to Timing Analysis}

\begin{concept}{Learning Objectives}\\
After this lecture, students will be able to:
\begin{itemize}
    \item Understand timing constraints in digital systems
    \item Analyze setup and hold time requirements
    \item Calculate maximum clock frequency
    \item Identify timing violations
    \item Apply timing constraints in FPGA designs
\end{itemize}
\end{concept}

% TODO: Add images from SCD_6_timing1.pdf, Pages 1-5
% Description: Introduction and overview slides
% Priority: IMPORT ANT
% Suggested filename: lecture06_intro_XX.png

\subsection{Flip-Flop Timing Parameters}

\begin{definition}{Critical Timing Parameters}\\
Every flip-flop has characteristic timing requirements:

\textbf{Setup Time ($t_{su}$):}
\begin{itemize}
    \item Minimum time data must be stable BEFORE clock edge
    \item Ensures proper data capture
    \item Typically 50-200 ps for modern FPGAs
\end{itemize}

\textbf{Hold Time ($t_h$):}
\begin{itemize}
    \item Minimum time data must remain stable AFTER clock edge
    \item Prevents metastability
    \item Typically 0-100 ps for modern FPGAs
\end{itemize}

\textbf{Clock-to-Q Delay ($t_{co}$):}
\begin{itemize}
    \item Time from clock edge to output change
    \item Represents flip-flop propagation delay
    \item Typically 100-400 ps for modern FPGAs
\end{itemize}
\end{definition}

% TODO: Add images from SCD_6_timing1.pdf, Pages 6-12
% Description: Flip-flop timing diagrams showing setup, hold, and clock-to-Q
% Priority: CRITICAL
% Suggested filename: lecture06_ff_timing_XX.png

\begin{iequation}
T_{clk} \geq t_{co} + t_{logic} + t_{routing} + t_{su}
\end{iequation}

where:
\begin{itemize}
    \item $T_{clk}$ = Clock period
    \item $t_{co}$ = Clock-to-output delay
    \item $t_{logic}$ = Combinational logic delay
    \item $t_{routing}$ = Routing delay
    \item $t_{su}$ = Setup time
\end{itemize}

\subsection{Setup Time Analysis}

\begin{definition}{Setup Time Constraint}\\
The setup time constraint ensures data arrives at the destination flip-flop with sufficient margin before the clock edge.

\textbf{Setup Equation:}
$$t_{arrival} \leq T_{clk} - t_{su}$$

where $t_{arrival}$ is the time when data reaches the flip-flop input.
\end{definition}

% TODO: Add images from SCD_6_timing1.pdf, Pages 13-20
% Description: Setup time analysis diagrams and examples
% Priority: CRITICAL
% Suggested filename: lecture06_setup_analysis_XX.png

\begin{example2}{Setup Time Calculation}\\
\textbf{Given:}
\begin{itemize}
    \item Clock frequency: 100 MHz ($T_{clk} = 10$ ns)
    \item $t_{co} = 0.5$ ns
    \item $t_{logic} = 4$ ns
    \item $t_{routing} = 2$ ns
    \item $t_{su} = 0.3$ ns
\end{itemize}

\tcblower

\textbf{Solution:}

Data arrival time:
$$t_{arrival} = t_{co} + t_{logic} + t_{routing} = 0.5 + 4 + 2 = 6.5 \text{ ns}$$

Required time (setup constraint):
$$t_{required} = T_{clk} - t_{su} = 10 - 0.3 = 9.7 \text{ ns}$$

Setup slack:
$$\text{Slack} = t_{required} - t_{arrival} = 9.7 - 6.5 = 3.2 \text{ ns}$$

\important{Result:} Setup constraint is MET with 3.2 ns positive slack.
\end{example2}

\subsection{Hold Time Analysis}

\begin{definition}{Hold Time Constraint}\\
The hold time constraint ensures data remains stable at the flip-flop input for sufficient time after the clock edge.

\textbf{Hold Equation:}
$$t_{arrival} \geq t_h$$

\textbf{Critical:} Hold violations cannot be fixed by changing clock frequency!
\end{definition}

% TODO: Add images from SCD_6_timing1.pdf, Pages 21-28
% Description: Hold time analysis diagrams and examples
% Priority: CRITICAL
% Suggested filename: lecture06_hold_analysis_XX.png

\begin{remark}
Hold time violations typically occur when:
\begin{itemize}
    \item Clock skew is too large
    \item Fast data path between flip-flops
    \item Insufficient routing delay
    \item Clock and data paths are unbalanced
\end{itemize}
\end{remark}

\raggedcolumns
\columnbreak

\subsection{Clock Skew}

\begin{definition}{Clock Skew}\\
Clock skew is the difference in arrival time of the clock signal at different flip-flops.

\textbf{Types:}
\begin{itemize}
    \item \textbf{Positive Skew:} Clock arrives later at destination FF
    \item \textbf{Negative Skew:} Clock arrives earlier at destination FF
\end{itemize}

\textbf{Effects:}
\begin{itemize}
    \item Positive skew: Helps setup, hurts hold
    \item Negative skew: Hurts setup, helps hold
\end{itemize}
\end{definition}

% TODO: Add images from SCD_6_timing1.pdf, Pages 29-35
% Description: Clock skew diagrams and impact analysis
% Priority: CRITICAL
% Suggested filename: lecture06_clock_skew_XX.png

\begin{iequation}
T_{clk,min} = t_{co} + t_{logic} + t_{routing} + t_{su} - \text{skew}
\end{iequation}

\subsection{Maximum Clock Frequency}

\begin{KR}{Determining Maximum Clock Frequency}\\
\textbf{Step 1: Identify Critical Path}
\begin{itemize}
    \item Find path with longest delay
    \item Sum all delays: $t_{co} + t_{logic} + t_{routing}$
\end{itemize}

\textbf{Step 2: Apply Setup Constraint}
$$T_{clk,min} = t_{critical\_path} + t_{su}$$

\textbf{Step 3: Calculate Maximum Frequency}
$$f_{max} = \frac{1}{T_{clk,min}}$$

\textbf{Step 4: Verify Hold Constraints}
\begin{itemize}
    \item Check shortest path
    \item Ensure $t_{co,min} + t_{logic,min} + t_{routing,min} \geq t_h$
\end{itemize}
\end{KR}

% TODO: Add images from SCD_6_timing1.pdf, Pages 36-41
% Description: Maximum frequency calculation examples and timing reports
% Priority: CRITICAL
% Suggested filename: lecture06_fmax_calc_XX.png

\begin{highlight}{Key Timing Concepts}\\
\begin{center}
\begin{tabular}{|l|l|}
\hline
\textbf{Concept} & \textbf{Impact} \\
\hline
Setup Time & Limits maximum frequency \\
Hold Time & Independent of frequency \\
Clock Skew & Affects both setup and hold \\
Critical Path & Determines $f_{max}$ \\
Slack & Timing margin (pos/neg) \\
\hline
\end{tabular}
\end{center}
\end{highlight}

\begin{remark}
Modern FPGA tools perform static timing analysis (STA) to automatically identify timing violations and report slack for all paths. Always verify timing reports before deploying designs.
\end{remark}

% ===== IMAGE SUMMARY =====
% Total images needed: 41
% CRITICAL priority: 30
% IMPORTANT priority: 8
% SUPPLEMENTARY priority: 3
%
% Quick extraction checklist:
% [ ] [SCD_6_timing1.pdf, Pages 1-41] - Complete timing analysis part 1 content
% =====================