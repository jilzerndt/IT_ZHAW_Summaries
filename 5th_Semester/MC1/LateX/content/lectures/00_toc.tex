\section{Table of Contents}

\begin{concept}{Course Structure Overview}\\
This summary covers embedded systems programming across 12 main chapters:
\begin{enumerate}
    \item Understanding Memory and I/O
    \item Power Management
    \item Direct Memory Access (DMA)
    \item From Hardware Timer to Scheduler
    \item Structuring Embedded Software - Part 1
    \item Finite State Machines (FSMs)
    \item Real-Time Operating Systems - Basics
    \item Real-Time Operating Systems - Advanced
    \item Structuring Embedded Software - Part 2
    \item Verification
    \item Watchdog and System Monitoring
    \item Security for Embedded Systems
\end{enumerate}

\textbf{Total KRs:} [TBD - fill in after all chapters complete]\\
\textbf{Total Examples:} [TBD - fill in after all chapters complete]
\end{concept}

\subsection{1. Understanding Memory and I/O}

\mult{2}

\begin{definition}{Chapter 1: Understanding Memory and I/O - The Basics of Embedded Software}\\
\textbf{Topics Covered:}
\begin{itemize}
    \item Memory sections and allocation
    \item Scope and lifetime of variables
    \item I/O operations in embedded systems
    \item Hardware-software interaction
\end{itemize}
\end{definition}

\begin{concept}{Key Concepts and KR} 
\begin{itemize}
    \item \important{TODO:} Fill in key concepts and knowledge requirements after chapter completion
\end{itemize}
\end{concept}

\begin{theorem}{LateX Files}
    \begin{itemize}
        \item \texttt{lecture01A\_memory.tex} - Memory concepts
        \item \texttt{lecture01B\_io.tex} - I/O concepts
    \end{itemize}
\end{theorem}

\begin{corollary}{Relevant Course Materials}
    \begin{itemize}
        \item \textbf{L01:} Understanding Memory and I/O
        \item \textbf{P01:} Edge Detection and Debouncing
    \end{itemize}
\end{corollary}

\multend

\subsection{2. Power Management}

\important{TODO:} Complete all chapters analogue to first one like above.

%TODO: Update this table of contents after Phase 1.5 analysis to reflect the optimal exam-focused structure