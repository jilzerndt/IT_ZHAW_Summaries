\section{Table of Contents}

\begin{concept}{Course Structure Overview}\\
This summary covers embedded systems programming across 12 main chapters:
\begin{enumerate}
    \item Understanding Memory and I/O
    \item Power Management
    \item Direct Memory Access (DMA)
    \item From Hardware Timer to Scheduler
    \item Structuring Embedded Software - Part 1
    \item Finite State Machines (FSMs)
    \item Real-Time Operating Systems - Basics
    \item Real-Time Operating Systems - Advanced
    \item Structuring Embedded Software - Part 2
    \item Verification
    \item Watchdog and System Monitoring
    \item Security for Embedded Systems
\end{enumerate}

\textbf{Total KRs:} [TBD - fill in after all chapters complete]\\
\textbf{Total Examples:} [TBD - fill in after all chapters complete]
\end{concept}

\subsection{1. Understanding Memory and I/O}

\mult{2}

\begin{definition}{Chapter 1: Understanding Memory and I/O - The Basics of Embedded Software}\\
\textbf{Topics Covered:}
\begin{itemize}
    \item Memory sections and allocation
    \item Scope and lifetime of variables
    \item I/O operations in embedded systems
    \item Hardware-software interaction
\end{itemize}
\end{definition}

\begin{concept}{Key Concepts and KR} 
\begin{itemize}
    \item ... TODO: Fill in key concepts and knowledge requirements after chapter completion
\end{itemize}
\end{concept}

\begin{theorem}{LateX Files}
    \begin{itemize}
        \item \texttt{lecture01A\_memory.tex} - Memory concepts
        \item \texttt{lecture01B\_io.tex} - I/O concepts
    \end{itemize}
\end{theorem}

\begin{corollary}{Relevant Course Materials}\\
\textbf{Lecture Coverage:} Lecture T01 (SW1, 19.09)
\textbf{Related Labs:} P\_01 (Edge detection/debouncing)
\end{corollary}

\multend

\subsection{2. Power Management}

\begin{definition}{Chapter 2: Power Management and Low Power Applications}\\
\textbf{File:} \texttt{02\_power\_management.tex}

\textbf{Topics Covered:}
\begin{itemize}
    \item Power consumption basics
    \item Low power modes (Sleep, Stop)
    \item Power optimization techniques
    \item Wake-up mechanisms (WFI, WFE)
\end{itemize}

\textbf{Key Concepts:} Energy efficiency, sleep modes, power management strategies

\textbf{Lecture Coverage:} Lecture T02 (Week 2, 26.09)

\textbf{Related Labs:} P\_04 (Power modes)
\end{definition}

\subsection{3. Direct Memory Access (DMA)}

\begin{definition}{Chapter 3: Digital Sensors - Direct Memory Access}\\
\textbf{File:} \texttt{03\_dma.tex}

\textbf{Topics Covered:}
\begin{itemize}
    \item DMA basics and configuration
    \item DMA channels and priorities
    \item Peripheral-to-memory transfers
    \item DMA for sensor data acquisition
\end{itemize}

\textbf{Key Concepts:} Direct memory access, efficient data transfer, sensor interfacing

\textbf{Lecture Coverage:} Lecture T03 (Week 4, 10.10)

\textbf{Related Labs:} P\_06 (Motion Sensor II - DMA \& Power)
\end{definition}

\subsection{4. From Hardware Timer to Scheduler}

\begin{definition}{Chapter 4: Scheduling with Hardware Timers}\\
\textbf{File:} \texttt{04\_timer\_to\_scheduler.tex}

\textbf{Topics Covered:}
\begin{itemize}
    \item Hardware timer basics
    \item Interrupt-driven scheduling
    \item Task scheduling with timers
    \item Multi-task scheduling approaches
\end{itemize}

\textbf{Key Concepts:} Timer-based scheduling, interrupt handling, task management

\textbf{Lecture Coverage:} Lecture T04 (Week 5, 17.10)

\textbf{Related Labs:} P\_05 (Motion Sensor I - Basic access)
\end{definition}

\subsection{5. Structuring Embedded Software - Part 1}

\begin{definition}{Chapter 5: Software Architecture Fundamentals}\\
\textbf{File:} \texttt{05\_sw\_structure\_part1.tex}

\textbf{Topics Covered:}
\begin{itemize}
    \item Module structure and interfaces
    \item Design patterns in C
    \item Header file organization
    \item Encapsulation and abstraction
\end{itemize}

\textbf{Key Concepts:} Modular design, clean architecture, software patterns

\textbf{Lecture Coverage:} Lecture T05 (Week 6, 24.10)

\textbf{Related Labs:} P\_06 (Motion Sensor II), P\_02 (Matrix keyboard)
\end{definition}

\subsection{6. Finite State Machines (FSMs)}

\begin{definition}{Chapter 6: Partitioning Reactive Systems with FSMs}\\
\textbf{File:} \texttt{06\_fsm\_basics.tex}

\textbf{Topics Covered:}
\begin{itemize}
    \item FSM fundamentals
    \item Event-driven programming
    \item Cooperating FSMs
    \item FSM implementation in C
    \item Event queues
\end{itemize}

\textbf{Key Concepts:} State machines, event handling, reactive systems

\textbf{Lecture Coverage:} Lecture T06 (Week 7, 31.10)

\textbf{Related Labs:} P\_08\_09 (Cooperating FSMs)
\end{definition}

\subsection{7. Real-Time Operating Systems - Basics}

\begin{definition}{Chapter 7: RTOS Fundamentals}\\
\textbf{File:} \texttt{07\_rtos\_basics.tex}

\textbf{Topics Covered:}
\begin{itemize}
    \item RTOS concepts
    \item Tasks and threads
    \item Scheduling algorithms
    \item Inter-task communication
\end{itemize}

\textbf{Key Concepts:} Real-time scheduling, multitasking, RTOS architecture

\textbf{Lecture Coverage:} Lecture T07 (Week 8, 07.11)

\textbf{Related Labs:} P\_10, P\_11 (Zephyr RTOS)
\end{definition}

\subsection{8. Real-Time Operating Systems - Advanced}

\begin{definition}{Chapter 8: Advanced RTOS Topics}\\
\textbf{File:} \texttt{08\_rtos\_advanced.tex}

\textbf{Topics Covered:}
\begin{itemize}
    \item Synchronization primitives
    \item Priority inversion
    \item Resource management
    \item Real-time constraints
\end{itemize}

\textbf{Key Concepts:} Concurrency, synchronization, real-time guarantees

\textbf{Lecture Coverage:} Lecture T08 (Week 9, 14.11)

\textbf{Related Labs:} P\_11 (Zephyr - Real-time operating system)
\end{definition}

\subsection{9. Structuring Embedded Software - Part 2}

\begin{definition}{Chapter 9: Advanced Software Architecture}\\
\textbf{File:} \texttt{09\_sw\_structure\_part2.tex}

\textbf{Topics Covered:}
\begin{itemize}
    \item Advanced design patterns
    \item Software layering
    \item Driver architecture
    \item System integration
\end{itemize}

\textbf{Key Concepts:} Layered architecture, driver development, system design

\textbf{Lecture Coverage:} Lecture T09 (Week 10, 21.11)

\textbf{Related Labs:} P\_07 (Motion Sensor III - File system)
\end{definition}

\subsection{10. Verification}

\begin{definition}{Chapter 10: Software Verification Techniques}\\
\textbf{File:} \texttt{10\_verification.tex}

\textbf{Topics Covered:}
\begin{itemize}
    \item Testing strategies
    \item Debugging techniques
    \item Code verification
    \item Quality assurance
\end{itemize}

\textbf{Key Concepts:} Testing, debugging, verification methods

\textbf{Lecture Coverage:} Lecture T10 (Week 11, 28.11)

\textbf{Related Labs:} P\_12 (Zephyr - Unit Testing)
\end{definition}

\subsection{11. Watchdog and System Monitoring}

\begin{definition}{Chapter 11: Watchdog Timers and System Reliability}\\
\textbf{File:} \texttt{11\_watchdog.tex}

\textbf{Topics Covered:}
\begin{itemize}
    \item Watchdog timer basics
    \item System monitoring
    \item Fault detection
    \item Recovery mechanisms
\end{itemize}

\textbf{Key Concepts:} System reliability, watchdog implementation, fault tolerance

\textbf{Lecture Coverage:} Lecture T11 (Week 12, 05.12)

\textbf{Related Labs:} P\_12 (Unit Testing)
\end{definition}

\subsection{12. Security for Embedded Systems}

\begin{definition}{Chapter 12: Embedded Systems Security}\\
\textbf{File:} \texttt{12\_security.tex}

\textbf{Topics Covered:}
\begin{itemize}
    \item Security fundamentals
    \item Cryptography basics
    \item Secure boot
    \item Threat mitigation
\end{itemize}

\textbf{Key Concepts:} Security principles, cryptography, secure design

\textbf{Lecture Coverage:} Lecture T12 (Week 13, 12.12)

\textbf{Related Labs:} P\_13 (Zephyr - Security)
\end{definition}

%TODO: Update this table of contents after Phase 1.5 analysis to reflect the optimal exam-focused structure