\section{Umweltmanagement, Führung, Entwicklungsmodi}


\begin{concept}{Kulturmodell nach Schein}\\
Edgar Schein unterscheidet drei Ebenen der Unternehmenskultur:
\begin{itemize}
    \item \textbf{Symbole und Zeichen (Artefakte)}: Sichtbarer und direkt beobachtbarer Teil einer Organisationskultur
    \begin{itemize}
        \item Feiern und Riten (Aufnahmeriten, Bekräftigungsriten, Integrationsriten)
        \item Art der Kommunikation
        \item Umgangsformen
        \item Architektur der Unternehmensgebäude
    \end{itemize}
    \item \textbf{Normen und Standards}: Weniger sichtbarer Teil der Organisationskultur
    \begin{itemize}
        \item Handlungsmaximen, Verhaltensrichtlinien und implizite Verbote
        \item Sichtbarer Teil: Code of Conduct, Führungsrichtlinien, Mission Statement
        \item Orientierungsmuster für Bereiche ohne formell geregelte Angelegenheiten
    \end{itemize}
    \item \textbf{Basisannahmen}: Grundlegende Orientierungs- und Vorstellungsmuster
    \begin{itemize}
        \item Annahmen über die Umwelt
        \item Annahmen über Wahrheit und Zeit
        \item Annahmen über die Natur des Menschen
        \item Annahmen über das menschliche Handeln
        \item Annahmen über die Natur sozialer Beziehungen
    \end{itemize}
\end{itemize}
\end{concept}

\subsubsection{Führung}

\begin{concept}{Handlungsebenen des Managements}
\begin{itemize}
    \item \textbf{Normatives Management}: Beschäftigt sich mit konfligierenden Anliegen und Interessen; Aufbau unternehmerischer Legitimations- und Verständigungspotentiale
    \item \textbf{Strategisches Management}: Bewältigt Komplexität und Ungewissheit der Marktbedingungen; Aufbau nachhaltiger Wettbewerbsvorteile
    \item \textbf{Operatives Management}: Befasst sich mit der Knappheit der Produktionsfaktoren; Gewährleistung effizienter Abläufe und Problemlösungsroutinen
\end{itemize}
\end{concept}

\begin{concept}{Managementkreislauf} (auch PDCA-Zyklus oder Deming-Zyklus genannt) 
\begin{itemize}
    \item \textbf{Plan (Planen)}: Ziele setzen, Strategien entwickeln, Massnahmen planen
    \item \textbf{Do (Umsetzen)}: Ausführen der geplanten Massnahmen
    \item \textbf{Check (Überprüfen)}: Kontrolle der Ergebnisse, Vergleich mit den Zielen
    \item \textbf{Act (Handeln)}: Korrekturmassnahmen einleiten, Standards setzen
\end{itemize}

Der Fokus liegt dabei auf Effektivität ("Die richtigen Dinge tun") und Effizienz ("Die Dinge richtig tun").
\end{concept}

\begin{definition}{Anforderungen an eine Führungsperson}
\begin{itemize}
    \item \textbf{Fachkompetenzen}: Fachwissen, Methodenkenntnisse, analytische Fähigkeiten
    \item \textbf{Führungskompetenzen}: Zielorientierung, Entscheidungsfähigkeit, Durchsetzungsvermögen
    \item \textbf{Soziale Kompetenzen}: Kommunikationsfähigkeit, Teamfähigkeit, Empathie
    \item \textbf{Persönliche Kompetenzen}: Belastbarkeit, Selbstreflexion, ethische Grundhaltung
\end{itemize}
\end{definition}

\begin{concept}{Menschenbilder}

    \textbf{Historische Menschenbilder}:
    \begin{itemize}
        \item Nach Machiavelli: Menschen sind heuchlerisch, hinterhältig und unzuverlässig
        \item Nach Adam Smith: Jeder Mensch verfolgt seinen persönlichen Nutzen
    \end{itemize}
     \textbf{Menschenbilder nach McGregor}:
    \begin{itemize}
        \item Theorie X: Mitarbeiter arbeitet nur unter Zwang und muss kontrolliert werden
        \item Theorie Y: Mitarbeiter sind motiviert und streben nach Verantwortung
    \end{itemize}
     \textbf{Menschenbilder nach Schein}:
    \begin{itemize}
        \item rational \& economic man: Mitarbeiter verfolgt eigene Ziele und muss angetrieben werden
        \item social man: Mitarbeiter erwartet Anerkennung, Loyalität und Verständnis
        \item self-actualizing man: Mitarbeiter ist leistungsmotiviert und will gefördert werden
        \item complex man: Mitarbeiter wollen flexibel geführt werden, da sie wandlungs- und lernfähig sind
    \end{itemize}

\end{concept}

\important{see Personalmanagement}

\subsubsection{Umweltmanagement}

\begin{concept}{Dreifache Unternehmensverantwortung}\\
Die dreifache Unternehmensverantwortung (Triple Bottom Line) umfasst:
\begin{itemize}
    \item \textbf{Ökonomische Verantwortung}: Erzielen eines angemessenen Gewinns, Sicherung der Wettbewerbsfähigkeit, effiziente Ressourcennutzung
    \item \textbf{Ökologische Verantwortung}: Schonung der natürlichen Ressourcen, Verringerung von Umweltbelastungen, umweltgerechte Produktion und Produkte
    \item \textbf{Soziale Verantwortung}: Fairer Umgang mit Mitarbeitenden und Geschäftspartnern, Engagement für gesellschaftliche Belange, Unterstützung sozialer Projekte
\end{itemize}
\end{concept}

\begin{definition}{Corporate Social Responsibility (CSR)}Auswirkungen der unternehmerischen Tätigkeit auf Gesellschaft und Umwelt. CSR umfasst ein breites Spektrum von Themen, die bei der Unternehmensführung zu beachten sind, darunter:
\begin{itemize}
    \item Arbeitsbedingungen (inkl. Gesundheitsschutz)
    \item Menschenrechte
    \item Umweltschutz
    \item Korruptionsprävention
    \item Fairer Wettbewerb
    \item Verbraucherinteressen
    \item Steuern und Transparenz
\end{itemize}
\end{definition}

\begin{concept}{Ökonomisches und ökologisches System}\\
Zwischen dem ökonomischen und dem ökologischen System bestehen komplexe Wechselwirkungen:
\begin{itemize}
    \item Das ökonomische System ist ein Subsystem des ökologischen Systems
    \item Die Wirtschaft benötigt natürliche Ressourcen (Input) und erzeugt Abfälle und Emissionen (Output)
    \item Neben der Wertschöpfung findet auch eine "Schadschöpfung" statt
    \item Herausforderung für Unternehmen: Wertschöpfung maximieren, Schadschöpfung minimieren
\end{itemize}
\end{concept}

\begin{concept}{Verantwortungsbereiche von Unternehmen}\\
Die Verantwortung eines Unternehmens kann in verschiedene Scopes unterteilt werden:
\begin{itemize}
    \item \textbf{Scope 1}: Direkte Emissionen aus eigenen oder kontrollierten Quellen (z.B. eigene Produktion)
    \item \textbf{Scope 2}: Indirekte Emissionen aus eingekaufter Energie (z.B. Strom, Wärme)
    \item \textbf{Scope 3}: Alle anderen indirekten Emissionen in der Wertschöpfungskette (z.B. Lieferkette, Transporte, Nutzung der verkauften Produkte)
\end{itemize}

Zunehmend wird von Unternehmen erwartet, dass sie auch für Scope 3-Emissionen Verantwortung übernehmen.
\end{concept}

\begin{concept}{Nachhaltigkeitsziele der UN (SDGs)}\\
Die Agenda 2030 der Vereinten Nationen umfasst 17 Ziele für nachhaltige Entwicklung (Sustainable Development Goals, SDGs), die als Orientierungsrahmen für Unternehmen dienen können:
\begin{itemize}
    \item Keine Armut
    \item Kein Hunger
    \item Gesundheit und Wohlbefinden
    \item Hochwertige Bildung
    \item Geschlechtergleichheit
    \item Sauberes Wasser und sanitäre Einrichtungen
    \item Bezahlbare und saubere Energie
    \item Menschenwürdige Arbeit und Wirtschaftswachstum
    \item Industrie, Innovation und Infrastruktur
    \item Weniger Ungleichheiten
    \item Nachhaltige Städte und Gemeinden
    \item Verantwortungsvoller Konsum und Produktion
    \item Massnahmen zum Klimaschutz
    \item Leben unter Wasser
    \item Leben an Land
    \item Frieden, Gerechtigkeit und starke Institutionen
    \item Partnerschaften zur Erreichung der Ziele
\end{itemize}
\end{concept}

\begin{definition}{Ökobilanz}\\
Die Ökobilanz ist ein Instrument zur Erfassung und Bewertung von Umweltauswirkungen über den gesamten Produktlebenszyklus (cradle-to-grave). Sie dient:
\begin{itemize}
    \item Der Erfassung und Bewertung der Umweltauswirkungen über den gesamten Produktlebenszyklus
    \item Dem Produktvergleich anhand des erbrachten Nutzens
    \item Der Identifikation von Schwachstellen und Prioritäten des Handelns
    \item Der Bewertung von Entscheidungsalternativen
    \item Der Ableitung und Überwachung von Verbesserungsmassnahmen
\end{itemize}

Positive Effekte einer Ökobilanz:
\begin{itemize}
    \item Geringerer Ressourcenverbrauch senkt Kosten
    \item Umweltfreundliches Image motiviert Mitarbeiter
    \item Zunehmende Nachfrage nach umweltfreundlichen Produkten
    \item Öffentliche Hand wendet vermehrt ökologische Auswahlkriterien bei Auftragsvergabe an
\end{itemize}
\end{definition}


\subsubsection{Entwicklungsmodi}

\begin{definition}{Entwicklungsmodi}\\
Entwicklungsmodi bezeichnen die verschiedenen Arten der Weiterentwicklung einer Unternehmung:
\begin{itemize}
    \item \textbf{Optimierung}: Die kontinuierliche, ständig ablaufende Verbesserung des Bestehenden (inkrementelle Veränderung)
    \begin{itemize}
        \item Schrittweise Verbesserung bestehender Produkte, Dienstleistungen und Prozesse
        \item Innerhalb bestehender Strukturen und Systeme
        \item Geringe Risiken, begrenzte Verbesserungspotenziale
    \end{itemize}
    \item \textbf{Erneuerung}: Die diskontinuierliche, nur sprunghaft stattfindende Schaffung von völlig Neuem (radikale Veränderung)
    \begin{itemize}
        \item Entwicklung neuer Produkte, Dienstleistungen und Geschäftsmodelle
        \item Oft verbunden mit organisatorischen Veränderungen
        \item Höhere Risiken, grössere Verbesserungspotenziale
    \end{itemize}
\end{itemize}

Die beiden Entwicklungsmodi ergänzen sich und sollten in einem ausgewogenen Verhältnis stehen: Optimierung sorgt für kontinuierliche Verbesserung, während Erneuerung disruptive Innovationen ermöglicht.
\end{definition}

\begin{concept}{Innovationstypen nach Grad der Veränderung}\\
Innovationen lassen sich nach dem Grad der Veränderung in verschiedene Typen einteilen:
\begin{itemize}
    \item \textbf{Inkrementelle Innovation}: Schrittweise Verbesserung bestehender Produkte, Prozesse oder Geschäftsmodelle
    \begin{itemize}
        \item Beispiel: Neue Version eines Smartphones mit verbesserter Kamera
        \item Geringes Risiko, begrenzte Rendite
        \item Kurzfristiger Zeithorizont
        \item Optimierung als Entwicklungsmodus
    \end{itemize}
    \item \textbf{Radikale Innovation}: Grundlegende Neuerung, die bestehende Lösungen vollständig ersetzt
    \begin{itemize}
        \item Beispiel: Einführung des ersten Smartphones
        \item Hohes Risiko, hohe potenzielle Rendite
        \item Langfristiger Zeithorizont
        \item Erneuerung als Entwicklungsmodus
    \end{itemize}
    \item \textbf{Disruptive Innovation}: Innovation, die bestehende Märkte und Geschäftsmodelle grundlegend verändert
    \begin{itemize}
        \item Beispiel: Streaming-Dienste vs. DVD-Verleih
        \item Sehr hohes Risiko, sehr hohe potenzielle Rendite
        \item Langfristiger Zeithorizont, oft unvorhersehbare Marktentwicklung
        \item Erneuerung als Entwicklungsmodus
    \end{itemize}
\end{itemize}
\end{concept}

\begin{concept}{Kontinuierlicher Verbesserungsprozess (KVP)}\\
Der Kontinuierliche Verbesserungsprozess (KVP) ist ein systematischer Ansatz zur Optimierung von Prozessen, Produkten und Dienstleistungen. Er basiert auf dem Grundgedanken, dass viele kleine Verbesserungsschritte in der Summe zu einer signifikanten Verbesserung führen.

Prinzipien des KVP:
\begin{itemize}
    \item \textbf{Kundenorientierung}: Fokus auf die Erfüllung von Kundenbedürfnissen
    \item \textbf{Prozessorientierung}: Betrachtung von Abläufen statt einzelner Funktionen
    \item \textbf{Mitarbeiterorientierung}: Einbeziehung aller Mitarbeiter
    \item \textbf{Kontinuität}: Ständige, nie endende Verbesserung
    \item \textbf{Kleine Schritte}: Fokus auf inkrementelle Verbesserungen
\end{itemize}

Methoden im Rahmen des KVP:
\begin{itemize}
    \item \textbf{PDCA-Zyklus} (Plan-Do-Check-Act): Systematisches Vorgehen bei Verbesserungen
    \item \textbf{5-Why-Methode}: Ursachenanalyse durch wiederholtes Fragen nach dem "Warum"
    \item \textbf{Pareto-Analyse}: Konzentration auf die wichtigsten Faktoren (80/20-Regel)
    \item \textbf{5S-Methode}: Ordnung und Sauberkeit am Arbeitsplatz
    \item \textbf{Ishikawa-Diagramm}: Strukturierte Ursachenanalyse
\end{itemize}
\end{concept}

\begin{concept}{Ambidextres Management}\\
Ambidextres Management (beidhändiges Management) beschreibt die Fähigkeit eines Unternehmens, sowohl Optimierung (Exploitation) als auch Erneuerung (Exploration) gleichzeitig zu betreiben:
\begin{itemize}
    \item \textbf{Exploitation (Ausnutzung)}: Optimierung bestehender Geschäftsmodelle, Effizienzsteigerung, Kostenreduktion
    \item \textbf{Exploration (Erkundung)}: Entwicklung neuer Geschäftsfelder, Experimentieren, Innovation
\end{itemize}

Organisatorische Ansätze für ambidextres Management:
\begin{itemize}
    \item \textbf{Strukturelle Ambidextrie}: Trennung von Exploration und Exploitation in verschiedene Organisationseinheiten
    \item \textbf{Kontextuelle Ambidextrie}: Gleichzeitige Verfolgung von Exploration und Exploitation innerhalb derselben Organisationseinheit
    \item \textbf{Temporale Ambidextrie}: Zeitliche Abwechslung zwischen Phasen der Exploration und Exploitation
    \item \textbf{Domänenspezifische Ambidextrie}: Exploration in bestimmten Bereichen, Exploitation in anderen
\end{itemize}
\end{concept}

\begin{KR}{Implementierung von Ambidextrem Management}\\
\textbf{Ausgangssituation analysieren}
\begin{itemize}
    \item Aktuelle Balance zwischen Exploitation und Exploration bewerten
    \item Stärken und Schwächen in beiden Bereichen identifizieren
    \item Innovationspipeline und -erfolge analysieren
    \item Unternehmenskultur und Führungsstil einschätzen
    \item Organisationsstruktur und -prozesse bewerten
\end{itemize}

\textbf{Strategischen Ansatz wählen}
\begin{itemize}
    \item Passenden Ansatz für ambidextres Management auswählen (strukturell, kontextuell, temporal oder domänenspezifisch)
    \item Ziele für Exploitation und Exploration definieren
    \item Ressourcenverteilung zwischen Exploitation und Exploration festlegen
    \item Zeithorizont für verschiedene Innovationsprojekte bestimmen
    \item Governance-Modell für Innovationsprojekte entwickeln
\end{itemize}

\textbf{Organisatorische Voraussetzungen schaffen}
\begin{itemize}
    \item Bei struktureller Ambidextrie: Separate Einheiten für Exploration einrichten
    \item Bei kontextueller Ambidextrie: Freiräume und Anreize für Exploration schaffen
    \item Führungskräfte für ambidextres Management sensibilisieren
    \item Kommunikation und Wissenstransfer zwischen Exploration und Exploitation sicherstellen
    \item Entscheidungsprozesse für Innovationsprojekte definieren
\end{itemize}

\textbf{Mitarbeiter befähigen}
\begin{itemize}
    \item Mitarbeiter für beide Modi qualifizieren
    \item Rollen und Verantwortlichkeiten klar definieren
    \item Anreizsysteme für beide Modi entwickeln
    \item Kreativität und unternehmerisches Denken fördern
    \item Fehlertoleranz und Experimentierfreude kultivieren
\end{itemize}

\textbf{Steuerung und Weiterentwicklung}
\begin{itemize}
    \item Kennzahlen für Exploitation und Exploration etablieren
    \item Regelmässige Überprüfung der Balance zwischen beiden Modi
    \item Erfolge in beiden Bereichen sichtbar machen
    \item Aus Erfahrungen lernen und Ansatz kontinuierlich verbessern
    \item Flexibel auf Veränderungen im Unternehmensumfeld reagieren
\end{itemize}
\end{KR}



\subsubsection{Nachhaltigkeit als Wettbewerbsfaktor}

\begin{concept}{Wirtschaftliche Vorteile der Nachhaltigkeit}\\
Nachhaltigkeit wird zunehmend zum Wettbewerbsfaktor für Unternehmen. Wirtschaftliche Vorteile einer nachhaltigen Unternehmensführung:
\begin{itemize}
    \item \textbf{Kostenreduktion}: Einsparung von Ressourcen, Energie und Abfallkosten
    \item \textbf{Risikominimierung}: Verringerung von ökologischen und sozialen Risiken
    \item \textbf{Innovationsförderung}: Entwicklung neuer, nachhaltiger Produkte und Geschäftsmodelle
    \item \textbf{Mitarbeitergewinnung und -bindung}: Attraktivität als Arbeitgeber
    \item \textbf{Kundenbindung}: Erfüllung steigender Kundenerwartungen hinsichtlich Nachhaltigkeit
    \item \textbf{Reputationsverbesserung}: Positives Image bei allen Stakeholdern
    \item \textbf{Zugang zu nachhaltigen Investitionen}: Bessere Bewertung durch ESG-Ratings
\end{itemize}
\end{concept}

\begin{concept}{Nachhaltige Geschäftsmodelle}\\
Nachhaltige Geschäftsmodelle integrieren ökonomische, ökologische und soziale Aspekte:
\begin{itemize}
    \item \textbf{Kreislaufwirtschaft}: Produkte und Materialien werden so lange wie möglich im Kreislauf gehalten
    \item \textbf{Sharing Economy}: Gemeinsame Nutzung von Ressourcen und Produkten
    \item \textbf{Product-as-a-Service}: Verkauf von Nutzung statt Produkten
    \item \textbf{Verlängerung der Produktlebensdauer}: Reparatur, Wiederaufbereitung, Upcycling
    \item \textbf{Social Business}: Unternehmen mit primär sozialer Zielsetzung
    \item \textbf{Nachhaltige Plattformmodelle}: Verbindung von Anbietern und Nachfragern nachhaltiger Produkte
\end{itemize}
\end{concept}

\subsubsection{Change Management}

\begin{definition}{Change Management}\\
Change Management umfasst alle Aktivitäten zur systematischen Planung, Steuerung und Begleitung von Veränderungsprozessen in Organisationen. Ziel ist es, geplante Veränderungen erfolgreich umzusetzen und Widerstände zu überwinden.
\end{definition}

\begin{concept}{Erfolgsfaktoren des Change Managements}\\
Für den Erfolg von Veränderungsprozessen sind verschiedene Faktoren entscheidend:
\begin{itemize}
    \item \textbf{Klare Vision und Strategie}: Gemeinsames Verständnis der Ziele und des Weges
    \item \textbf{Starkes Commitment der Führung}: Vorbildfunktion und aktive Unterstützung
    \item \textbf{Einbeziehung der Betroffenen}: Partizipation und Mitgestaltung
    \item \textbf{Offene Kommunikation}: Transparenz über Gründe, Ziele und Fortschritte
    \item \textbf{Berücksichtigung von Widerständen}: Konstruktiver Umgang mit Bedenken
    \item \textbf{Ressourcenbereitstellung}: Ausreichend Zeit, Budget und Personal
    \item \textbf{Kurzfristige Erfolge}: Sichtbare Fortschritte zur Motivation
    \item \textbf{Konsequente Umsetzung}: Durchhaltevermögen und Nachhaltigkeit
\end{itemize}
\end{concept}


\begin{KR}{Durchführung eines Change-Management-Prozesses}\\
\textbf{Veränderung vorbereiten}
\begin{itemize}
    \item Veränderungsbedarf analysieren und begründen
    \item Klare Vision und Ziele definieren
    \item Betroffene Personen und Gruppen identifizieren
    \item Stakeholder-Analyse durchführen
    \item Change-Team zusammenstellen
\end{itemize}

\textbf{Veränderungsstrategie entwickeln}
\begin{itemize}
    \item Veränderungsansatz wählen (revolutionär vs. evolutionär)
    \item Veränderungsschritte und Meilensteine planen
    \item Ressourcen zuweisen
    \item Kommunikationsstrategie entwickeln
    \item Qualifizierungsbedarf ermitteln
\end{itemize}

\textbf{Betroffene einbinden}
\begin{itemize}
    \item Frühzeitig und transparent informieren
    \item Beteiligungsmöglichkeiten schaffen
    \item Bedenken und Widerstände ernst nehmen
    \item Unterstützung und Qualifizierung anbieten
    \item Promotoren und Multiplikatoren gewinnen
\end{itemize}

\textbf{Veränderung umsetzen}
\begin{itemize}
    \item Schrittweise implementieren
    \item Quick Wins realisieren und kommunizieren
    \item Fortschritt kontinuierlich überprüfen
    \item Bei Bedarf nachsteuern
    \item Erfolge feiern
\end{itemize}

\textbf{Veränderung verankern}
\begin{itemize}
    \item Neue Prozesse und Strukturen formalisieren
    \item Anreizsysteme anpassen
    \item Erfolge messen und dokumentieren
    \item Lessons Learned festhalten
    \item Kontinuierlichen Verbesserungsprozess etablieren
\end{itemize}
\end{KR}

\begin{concept}{Widerstände gegen Veränderungen}\\
In Veränderungsprozessen treten häufig Widerstände auf, die verschiedene Ursachen haben können:
\begin{itemize}
    \item \textbf{Angst vor Unsicherheit}: Unklarheit über persönliche Konsequenzen
    \item \textbf{Verlust von Status und Einfluss}: Befürchtete Verschlechterung der Position
    \item \textbf{Gewohnheiten und Routinen}: Aufgabe vertrauter Arbeitsweisen
    \item \textbf{Mangelndes Verständnis}: Unzureichende Information über Gründe und Ziele
    \item \textbf{Fehlende Überzeugung}: Zweifel am Nutzen oder der Machbarkeit
    \item \textbf{Gruppeneinflüsse}: Soziale Normen und Gruppendruck
    \item \textbf{Timing und Kontext}: Ungünstiger Zeitpunkt oder Überlastung
\end{itemize}

Umgang mit Widerständen:
\begin{itemize}
    \item Frühzeitige Information und Kommunikation
    \item Einbeziehung der Betroffenen
    \item Qualifizierung und Unterstützung
    \item Bedenken ernst nehmen und adressieren
    \item Vorteile herausstellen und Erfolge sichtbar machen
\end{itemize}
\end{concept}


\begin{KR}{Entwicklung einer Innovationskultur}\\
\textbf{Ausgangssituation analysieren}
\begin{itemize}
    \item Bestehende Innovationskultur bewerten
    \item Innovationsbarrieren identifizieren
    \item Innovationsbereitschaft der Mitarbeiter einschätzen
    \item Innovationspotenziale erkennen
    \item Branchenentwicklung und Wettbewerbssituation analysieren
\end{itemize}

\textbf{Vision und Strategie entwickeln}
\begin{itemize}
    \item Innovationsvision formulieren
    \item Innovationsstrategie ableiten
    \item Innovationsziele definieren
    \item Innovationsfelder priorisieren
    \item Ressourcen für Innovation planen
\end{itemize}

\textbf{Strukturen und Prozesse gestalten}
\begin{itemize}
    \item Freiräume für Innovation schaffen
    \item Ideenmanagement einführen
    \item Innovationsteams bilden
    \item Innovationsprozesse definieren
    \item Schnittstellen zum Tagesgeschäft gestalten
\end{itemize}

\textbf{Führung und Mitarbeiter entwickeln}
\begin{itemize}
    \item Führungskräfte als Innovationstreiber etablieren
    \item Innovationskompetenzen aufbauen
    \item Kreativitätstechniken vermitteln
    \item Experimentierfreude fördern
    \item Fehlertoleranz und konstruktives Feedback etablieren
\end{itemize}

\textbf{Innovationskultur leben und weiterentwickeln}
\begin{itemize}
    \item Innovationserfolge sichtbar machen und feiern
    \item Innovationskommunikation gestalten
    \item Anreizsysteme für Innovation schaffen
    \item Innovationscontrolling etablieren
    \item Kontinuierliche Anpassung der Innovationsstrategie
\end{itemize}
\end{KR}