\section{Einführung SGMM und Umweltsphären}

\small

\begin{definition}{BWL und Betrieb} \small
Die Betriebswirtschaftslehre ist eine Teilwissenschaft der Wirtschaftswissenschaften und beschäftigt sich mit der Analyse und Gestaltung von betrieblichen Entscheidungen und Prozessen.
\end{definition}

\begin{definition}{Unternehmen} \small
Ein Unternehmen ist eine eigenständige, rechtliche und wirtschaftliche Einheit, die Sachgüter und Dienstleistungen anbietet. Der \textbf{Betrieb} ist die örtliche Produktionsstätte, während die \textbf{Firma} im rechtlichen Sinn der Name des Unternehmens ist.
\end{definition}

\mult{2}

\begin{concept}{Ökonomisches Prinzip}
Rationales wirtschaftliche Handeln
\begin{itemize}
    \item \textbf{Minimalprinzip}: Mit gegebenen Mitteln maximalen Ertrag erzielen
    \item \textbf{Maximalprinzip}: Mit minimalen Mitteln einen vorgegebenen Ertrag erzielen
    \item \textbf{Optimumprinzip}: Mit möglichst wenig Aufwand einen möglichst hohen Ertrag erzielen
\end{itemize}
\end{concept}

\begin{concept}{Bedürfnis\, Bedarf und Nachfrage}
\begin{itemize}
    \item \textbf{Bedürfnis}: Empfundener Mangel und der Wunsch, diesen zu beseitigen
    \item \textbf{Bedarf}: Ökonomisch relevantes, durch Kaufkraft gedecktes Bedürfnis
    \item \textbf{Nachfrage}: Durch Kaufbereitschaft zum Ausdruck gebrachter Bedarf
\end{itemize}
\end{concept}

\begin{definition}{Güterarten}
\begin{itemize}
    \item Nach Art: Sachgüter vs. Dienstleistungen
    \item Nach Verwendungszweck: Konsumgüter vs. Produktionsgüter
    \item Nach Knappheit: Freie Güter vs. wirtschaftliche Güter
    \item Nach Verfügbarkeit: Private Güter vs. öffentliche Güter
\end{itemize}
\end{definition}

\begin{theorem}{Unternehmensverantwortung} (Triple Bottom Line)
\begin{itemize}
    \item \textbf{Ökonomische Verantwortung}: Profitabilität, Erhalt der Wettbewerbsfähigkeit
    \item \textbf{Ökologische Verantwortung}: Umweltschutz, nachhaltige Ressourcennutzung
    \item \textbf{Soziale Verantwortung}: Verantwortung gegenüber Mitarbeitern, Gesellschaft
\end{itemize}
\end{theorem}

\multend

\begin{KR}{SGMM als Managementmodell}
Das St. Galler Management-Modell (SGMM) dient als Denkgrundlage und Orientierungshilfe für praktische Fragestellungen im Unternehmenskontext. Es bietet:
Komplexitätsreduktion, Ordnungsrahmen und Wirkungszusammenhänge, Strukturierung der Kommunikation, Gemeinsame Sprache für Managementfragen
\end{KR}

\begin{theorem}{Kategorien des SGMM}
\begin{itemize}
    \item \textbf{Umweltsphären}: Relevante Bezugsräume im Umfeld der Unternehmung
    \item \textbf{Anspruchsgruppen}: Gruppen und Individuen, die von der Unternehmenstätigkeit betroffen sind
    \item \textbf{Interaktionsthemen}: Gegenstände der Austauschbeziehungen zwischen Anspruchsgruppen und Unternehmen
    \item \textbf{Ordnungsmomente}: Strategie, Struktur, Kultur
    \item \textbf{Entwicklungsmodi}: Erneuerung und Optimierung
    \item \textbf{Prozesse}: Management-, Geschäfts- und Unterstützungsprozesse
\end{itemize}
\end{theorem}

\raggedcolumns

\begin{corollary}{Umweltsphären}
\begin{itemize}
    \item \textbf{Ökonomische Umwelt}: Wirtschaftliche Rahmenbedingungen, Arbeitsmarkt, Teuerung, internat. Wirtschaftsbeziehungen
    \item \textbf{Technologische Umwelt}: Technik/Naturwissens., Produktionsverfahren, Materialien, Transport-/Kommunikationsmittel
    \item \textbf{Soziale Umwelt}: Gesellschaftliche Trends, Werte und Bedürfnisse der Menschen
    \item \textbf{Ökologische Umwelt}: Gesamthaushalt der Natur, Rohstoffe, Energie, Klima, Abfälle
\end{itemize}
\end{corollary}

\begin{corollary}{Anspruchsgruppen (Stakeholder)}
    Kapitalgeber (Eigenkapital-/Fremdkapitalgeber), Kunden, Mitarbeitende, Öffentlichkeit/NGOs, Staat und Behörden, Lieferanten, Konkurrenz
\vspace{1mm}\\
Diese Anspruchsgruppen leiten aus ihrer Beziehung zum Unternehmen Erwartungen und Ansprüche ab, die sie selbst oder durch Interessenvertreter an das Unternehmen herantragen.
\end{corollary}


\begin{corollary}{Interaktionsthemen}
\begin{itemize}
    \item \textbf{Personen- und kulturgebundene Elemente}: Anliegen, Interessen, Normen und Werte
    \item \textbf{Objektgebundene Elemente}: Ressourcen (Kapital, Arbeit, Wissen, etc.)
\end{itemize}
\end{corollary}

\begin{minipage}{0.5\linewidth}
\begin{KR}{Interaktionsthemenanalyse}
Systematischen Untersuchung der Wechselbeziehungen zw. einem Unternehmen und Anspruchsgruppen.

\textbf{Beschreibung des Interaktionsthemas}
    \begin{itemize}
        \item Welche Ressource des Unternehmens ist betroffen?
        \item In welcher Umweltsphäre spielt sich der Sachverhalt ab?
    \end{itemize}
\textbf{Analyse der betroffenen Anspruchsgruppe}
    \begin{itemize}
        \item Welche Anliegen bringt die Anspruchsgruppe vor?
        \item Welche Interessen verfolgt die Anspruchsgruppe?
        \item Welche Normen und Werte stützen die Anliegen?
    \end{itemize}
\textbf{Bewertung aus Unternehmenssicht}
    \begin{itemize}
        \item Welche Gefahren ergeben sich für das Unternehmen?
        \item Welche Reaktionsmöglichkeiten hat das Unternehmen?
    \end{itemize}
\end{KR}
\end{minipage}
\begin{minipage}{0.5\linewidth}
\begin{example}
Beispiel Tesla:

\textbf{Interaktionsthema}: Elektromobilität

\textbf{Ressource}: Glaubwürdigkeit (Image) und Kapital

\textbf{Umweltsphäre}: Natur/Umwelt

\textbf{Anspruchsgruppe}: Kunden

\textbf{Anliegen}: Umweltfreundliches Auto

\textbf{Interessen}: CO2-Ausstoss, Steuererleichterungen

\textbf{Normen}: Umweltschutzgesetze, Subventionen

\textbf{Werte}: Umweltschutz, Nachhaltigkeit

\textbf{Gefahren aus Unternehmenssicht}: Verlust an Glaubwürdigkeit, Imageschaden, sinkende Verkaufszahlen durch Kürzung von Subventionszahlungen

\textbf{Reaktionsmöglichkeiten}: Investitionen in F\&E bei der Batterieproduktion, Lobbying in der Politik, Imagekampagne
\end{example}
\end{minipage}

\begin{minipage}{0.55\linewidth}
\begin{corollary}{Ordnungsmomente} des Unternehmens
\begin{itemize}
    \item \textbf{Strategie}: Langfristige Ausrichtung und Ziele 
    \item \textbf{Struktur}: Aufbau- und Ablauforganisation 
    \item \textbf{Kultur}: Gemeinsame Werte, Normen, Überzeugungen
\end{itemize}
\small
Zwischen Prozessen und Ordnungsmomenten besteht eine Wechselwirkung: Die Ordnungsmomente ergeben sich aus dem Alltagsgeschehen und strukturieren dieses wiederum.
\end{corollary}

\begin{corollary}{Entwicklungsmodi} (Veränderungsarten) 
\begin{itemize}
    \item \textbf{Optimierung}: kontinuierliche, ständig ablaufende Verbesserung des Bestehenden (inkrementelle V.)
    \item \textbf{Erneuerung}: diskontinuierliche, nur sprunghaft stattfindende Schaffung von völlig Neuem (radikale V.)
\end{itemize}
\end{corollary}
\end{minipage}
\begin{minipage}{0.45\linewidth}

\begin{corollary}{Prozesse}
\begin{itemize}
    \item \textbf{Managementprozesse}: Grundlegende Aufgaben: Gestaltung, Lenkung, Entwicklung der Unternehmung (normative Orientierungsprozesse, strategische Entwicklungsprozesse und operative Führungsprozesse)
    \item \textbf{Geschäftsprozesse}: Kernaktivitäten einer Unternehmung, auf den Kundennutzen ausgerichtet (Leistungserstellung, Vertrieb...)
    \item \textbf{Unterstützungsprozesse}: Interne Dienstleistungen für effektiven Vollzug der Geschäftsprozesse (Personal, EDV, Finanzen...)
\end{itemize}
\end{corollary}
\end{minipage}