\section{Einführung SGMM und Umweltsphären}

\subsection{BWL Grundbegriffe}

\begin{definition}{BWL und Betrieb}\\
Die Betriebswirtschaftslehre ist eine Teilwissenschaft der Wirtschaftswissenschaften und beschäftigt sich mit der Analyse und Gestaltung von betrieblichen Entscheidungen und Prozessen.

Ein Unternehmen ist eine eigenständige, rechtliche und wirtschaftliche Einheit, die Sachgüter und Dienstleistungen anbietet. Der \textbf{Betrieb} ist die örtliche Produktionsstätte, während die \textbf{Firma} im rechtlichen Sinn der Name des Unternehmens ist.
\end{definition}

\begin{definition}{Ökonomisches Prinzip}\\
Das ökonomische Prinzip beschreibt das rationale wirtschaftliche Handeln und kann in drei Ausprägungen vorkommen:
\begin{itemize}
    \item \textbf{Minimalprinzip}: Mit gegebenen Mitteln maximalen Ertrag erzielen
    \item \textbf{Maximalprinzip}: Mit minimalen Mitteln einen vorgegebenen Ertrag erzielen
    \item \textbf{Optimumprinzip}: Mit möglichst wenig Aufwand einen möglichst hohen Ertrag erzielen
\end{itemize}
\end{definition}

\begin{definition}{Bedürfnis\, Bedarf und Nachfrage}\\
\begin{itemize}
    \item \textbf{Bedürfnis}: Empfundener Mangel und der Wunsch, diesen zu beseitigen
    \item \textbf{Bedarf}: Ökonomisch relevantes, durch Kaufkraft gedecktes Bedürfnis
    \item \textbf{Nachfrage}: Durch Kaufbereitschaft zum Ausdruck gebrachter Bedarf
\end{itemize}
\end{definition}

\begin{definition}{Güterarten}\\
Man unterscheidet verschiedene Arten von Gütern:
\begin{itemize}
    \item Nach Art: Sachgüter vs. Dienstleistungen
    \item Nach Verwendungszweck: Konsumgüter vs. Produktionsgüter
    \item Nach Knappheit: Freie Güter vs. wirtschaftliche Güter
    \item Nach Verfügbarkeit: Private Güter vs. öffentliche Güter
\end{itemize}
\end{definition}

\begin{definition}{Unternehmensverantwortung}\\
Die Unternehmensverantwortung umfasst drei Dimensionen:
\begin{itemize}
    \item \textbf{Ökonomische Verantwortung}: Profitabilität, Erhalt der Wettbewerbsfähigkeit
    \item \textbf{Ökologische Verantwortung}: Umweltschutz, nachhaltige Ressourcennutzung
    \item \textbf{Soziale Verantwortung}: Verantwortung gegenüber Mitarbeitern, Gesellschaft
\end{itemize}
Diese drei Dimensionen bilden das Prinzip der dreifachen Unternehmensverantwortung (Triple Bottom Line).
\end{definition}

\subsection{St. Galler Management-Modell (SGMM)}

\begin{concept}{SGMM als Managementmodell}\\
Das St. Galler Management-Modell (SGMM) dient als Denkgrundlage und Orientierungshilfe für praktische Fragestellungen im Unternehmenskontext. Es bietet:
\begin{itemize}
    \item Komplexitätsreduktion
    \item Ordnungsrahmen und Wirkungszusammenhänge
    \item Strukturierung der Kommunikation
    \item Gemeinsame Sprache für Managementfragen
\end{itemize}
\end{concept}

\begin{concept}{Kategorien des SGMM}\\
Das SGMM besteht aus folgenden Begriffskategorien:
\begin{itemize}
    \item Umweltsphären: Relevante Bezugsräume im Umfeld der Unternehmung
    \item Anspruchsgruppen: Gruppen und Individuen, die von der Unternehmenstätigkeit betroffen sind
    \item Interaktionsthemen: Gegenstände der Austauschbeziehungen zwischen Anspruchsgruppen und Unternehmen
    \item Ordnungsmomente: Strategie, Struktur, Kultur
    \item Entwicklungsmodi: Erneuerung und Optimierung
    \item Prozesse: Management-, Geschäfts- und Unterstützungsprozesse
\end{itemize}
\end{concept}

\subsection{Umweltsphären}

\begin{definition}{Umweltsphären}\\
Umweltsphären bezeichnen relevante Bezugsräume im Umfeld eines Unternehmens und beeinflussen dessen Handlungsspielraum. Man unterscheidet:
\begin{itemize}
    \item \textbf{Ökonomische Umwelt}: Wirtschaftliche Rahmenbedingungen, Arbeitsmarkt, Teuerung, internationale Wirtschaftsbeziehungen
    \item \textbf{Technologische Umwelt}: Technik und Naturwissenschaften, Produktionsverfahren, Materialien, Transport- und Kommunikationsmittel
    \item \textbf{Soziale Umwelt}: Gesellschaftliche Trends, Werte und Bedürfnisse der Menschen
    \item \textbf{Ökologische Umwelt}: Gesamthaushalt der Natur, Rohstoffe, Energie, Klima, Abfälle
\end{itemize}
\end{definition}

\subsection{Anspruchsgruppen}

\begin{definition}{Anspruchsgruppen (Stakeholder)}\\
Anspruchsgruppen (Stakeholder) sind alle Gruppen und Individuen, die in irgendeiner Form von der Tätigkeit eines Unternehmens betroffen sind. Zu den wichtigsten Anspruchsgruppen zählen:
\begin{itemize}
    \item Kapitalgeber (Eigenkapital- und Fremdkapitalgeber)
    \item Kunden
    \item Mitarbeitende
    \item Öffentlichkeit/NGOs
    \item Staat und Behörden
    \item Lieferanten
    \item Konkurrenz
\end{itemize}
Diese Anspruchsgruppen leiten aus ihrer Beziehung zum Unternehmen Erwartungen und Ansprüche ab, die sie selbst oder durch Interessenvertreter an das Unternehmen herantragen.
\end{definition}

\subsection{Interaktionsthemen}

\begin{definition}{Interaktionsthemen}\\
Interaktionsthemen sind Gegenstände der Austauschbeziehungen zwischen Anspruchsgruppen und dem Unternehmen. Man unterscheidet:
\begin{itemize}
    \item \textbf{Personen- und kulturgebundene Elemente}: Anliegen, Interessen, Normen und Werte
    \item \textbf{Objektgebundene Elemente}: Ressourcen (Kapital, Arbeit, Wissen, etc.)
\end{itemize}
\end{definition}

\begin{concept}{Interaktionsthemenanalyse}\\
Die Interaktionsthemenanalyse ist ein Instrument zur systematischen Untersuchung der Wechselbeziehungen zwischen einem Unternehmen und seinen Anspruchsgruppen. Dabei wird folgender Ablauf verfolgt:
\begin{itemize}
    \item \textbf{Schritt 1}: Beschreibung des Interaktionsthemas
    \begin{itemize}
        \item Welche Ressource des Unternehmens ist betroffen?
        \item In welcher Umweltsphäre spielt sich der Sachverhalt ab?
    \end{itemize}
    \item \textbf{Schritt 2}: Analyse der betroffenen Anspruchsgruppe
    \begin{itemize}
        \item Welche Anliegen bringt die Anspruchsgruppe vor?
        \item Welche Interessen verfolgt die Anspruchsgruppe?
        \item Welche Normen und Werte stützen die Anliegen?
    \end{itemize}
    \item \textbf{Schritt 3}: Bewertung aus Unternehmenssicht
    \begin{itemize}
        \item Welche Gefahren ergeben sich für das Unternehmen?
        \item Welche Reaktionsmöglichkeiten hat das Unternehmen?
    \end{itemize}
\end{itemize}
\end{concept}

\begin{example}
Eine Interaktionsthemenanalyse am Beispiel Tesla könnte so aussehen:
\begin{itemize}
    \item \textbf{Interaktionsthema}: Elektromobilität
    \item \textbf{Ressource}: Glaubwürdigkeit (Image) und Kapital
    \item \textbf{Umweltsphäre}: Natur/Umwelt
    \item \textbf{Anspruchsgruppe}: Kunden (Käufer von Tesla Automobilen)
    \item \textbf{Anliegen}: Umweltfreundliches Auto, "saubere" Mobilität
    \item \textbf{Interessen}: Wenig CO2-Ausstoss, Steuererleichterungen
    \item \textbf{Normen}: Umweltschutzgesetz, Verordnung über CO2-Emissionen, Subventionen
    \item \textbf{Werte}: Umweltschutz, Nachhaltigkeit
    \item \textbf{Gefahren aus Unternehmenssicht}: Verlust an Glaubwürdigkeit, Imageschaden, sinkende Verkaufszahlen durch Kürzung von Subventionszahlungen
    \item \textbf{Reaktionsmöglichkeiten}: Investitionen in F\&E bei der Batterieproduktion, Lobbying in der Politik, Imagekampagne
\end{itemize}
\end{example}

\subsection{Ordnungsmomente}

\begin{definition}{Ordnungsmomente}\\
Ordnungsmomente geben dem Alltagsgeschehen einer Unternehmung, das in Form von Prozessen abläuft, eine gesamtheitliche Ausrichtung und Sinngebung. Die drei Teilbereiche sind:
\begin{itemize}
    \item \textbf{Strategie}: Langfristige Ausrichtung und Ziele des Unternehmens
    \item \textbf{Struktur}: Aufbau- und Ablauforganisation des Unternehmens
    \item \textbf{Kultur}: Gemeinsame Werte, Normen und Überzeugungen
\end{itemize}
Zwischen Prozessen und Ordnungsmomenten besteht eine Wechselwirkung: Die Ordnungsmomente ergeben sich aus dem Alltagsgeschehen und strukturieren dieses wiederum.
\end{definition}

\subsection{Entwicklungsmodi}

\begin{definition}{Entwicklungsmodi}\\
Entwicklungsmodi bezeichnen die verschiedenen Arten der Weiterentwicklung einer Unternehmung:
\begin{itemize}
    \item \textbf{Optimierung}: Die kontinuierliche, ständig ablaufende Verbesserung des Bestehenden (inkrementelle Veränderung)
    \item \textbf{Erneuerung}: Die diskontinuierliche, nur sprunghaft stattfindende Schaffung von völlig Neuem (radikale Veränderung)
\end{itemize}
\end{definition}

\subsection{Prozesse}

\begin{definition}{Prozesse}\\
Das SGMM begreift eine Unternehmung als ein System von Prozessen, d.h. routinemässigen Abläufen, welche den unternehmerischen Alltag prägen. Je besser diese Abläufe optimiert sind, desto grösser ist in der Regel der Erfolg der Unternehmung. Man unterscheidet:
\begin{itemize}
    \item \textbf{Managementprozesse}: Grundlegende Aufgaben, die mit der Gestaltung, Lenkung und Entwicklung der Unternehmung zu tun haben (normative Orientierungsprozesse, strategische Entwicklungsprozesse und operative Führungsprozesse)
    \item \textbf{Geschäftsprozesse}: Kernaktivitäten einer Unternehmung, die auf den Kundennutzen ausgerichtet sind (Leistungserstellung, Vertrieb etc.)
    \item \textbf{Unterstützungsprozesse}: Interne Dienstleistungen für einen effektiven Vollzug der Geschäftsprozesse (Personal, EDV, Finanzen etc.)
\end{itemize}
\end{definition}