\section{Realisation}

\subsection{Fertigungstypen}

\begin{definition}{Fertigungstypen}\\
Fertigungstypen unterscheiden sich nach der Häufigkeit der Wiederholung eines Auftrags oder Prozesses:
\begin{itemize}
    \item \textbf{Einzelfertigung}: Einmalige oder sehr seltene Herstellung eines Produkts (z.B. Schiffsbau, Spezialmaschinen)
    \item \textbf{Serienfertigung}: Wiederholte Herstellung einer begrenzten Anzahl gleicher Produkte (z.B. Möbel, Haushaltsgeräte)
    \item \textbf{Massenfertigung}: Kontinuierliche Herstellung einer unbegrenzten Anzahl gleicher Produkte (z.B. Schrauben, Getränke)
\end{itemize}
\end{definition}

\subsection{Fertigungsverfahren}

\begin{definition}{Fertigungsverfahren}\\
Fertigungsverfahren beschreiben die Art und Weise, wie die Abfolge der einzelnen Produktionsprozesse gestaltet wird:
\begin{itemize}
    \item \textbf{Werkstattfertigung}: Die Arbeitsschritte werden in spezialisierten Werkstätten durchgeführt, das Werkstück wandert von Werkstatt zu Werkstatt
    \begin{itemize}
        \item \textbf{Vorteile}: Hohe Flexibilität, gut für Einzel- und Kleinserienfertigung
        \item \textbf{Nachteile}: Lange Transportwege, hohe Rüstzeiten, komplexe Steuerung
    \end{itemize}
    \item \textbf{Fliessfertigung}: Die Arbeitsschritte folgen in einer festen Reihenfolge aufeinander, das Werkstück wird entlang einer Linie bearbeitet
    \begin{itemize}
        \item \textbf{Vorteile}: Kurze Durchlaufzeiten, einfache Steuerung, hohe Produktivität
        \item \textbf{Nachteile}: Geringe Flexibilität, anfällig für Störungen, hoher Kapitalbedarf
    \end{itemize}
    \item \textbf{Gruppenfertigung}: Kombination aus Werkstatt- und Fliessfertigung, Bildung von Fertigungsinseln für komplette Teilefamilien
    \begin{itemize}
        \item \textbf{Vorteile}: Mittlere Flexibilität, kompakte Materialflüsse, gute Übersichtlichkeit
        \item \textbf{Nachteile}: Mittlere Komplexität, Gefahr der Unterlastung einzelner Maschinen
    \end{itemize}
\end{itemize}
\end{definition}

\subsection{Kennzahlen der Produktion}

\begin{definition}{Produktionskennzahlen}\\
Zur Messung und Steuerung der Produktion werden verschiedene Kennzahlen verwendet:
\begin{itemize}
    \item \textbf{Rentabilität}: $\frac{\text{Ertrag} - \text{Aufwand}}{\text{Kapitaleinsatz}} = \frac{\text{Gewinn}}{\text{Kapitaleinsatz}}$
    \item \textbf{Produktivität}: $\frac{\text{Ausbringungsmenge}}{\text{Faktoreinsatzmenge}}$
    \item \textbf{Fehlerquote}: $\frac{\text{Fehlerhafte Produkte}}{\text{Total hergestellte Produkte}}$
    \item \textbf{Wirtschaftlichkeit}: $\frac{\text{Ertrag}}{\text{Aufwand}}$
\end{itemize}
\end{definition}

\begin{example}
Beispiel für verschiedene Fertigungstypen und -verfahren:

\textbf{Smartville (Hambach, Frankreich)}:
Die Produktion des Smart ist ein gutes Beispiel für moderne Fertigungskonzepte:
\begin{itemize}
    \item \textbf{Fertigungstiefe}: Mit etwa 10\% sehr gering, d.h. 90\% der Wertschöpfung wird von Zulieferern erbracht
    \item \textbf{Fertigungstyp}: Massenfertigung
    \item \textbf{Fertigungsverfahren}: Fliessfertigung mit Just-in-Sequence-Anlieferung
    \item \textbf{Besonderheit}: Zulieferer sind direkt auf dem Werksgelände angesiedelt (Industriepark-Konzept)
\end{itemize}

Vorteile:
\begin{itemize}
    \item Kurze Transportwege und -zeiten
    \item Minimale Lagerbestände
    \item Hohe Flexibilität bei Modellvarianten
    \item Konzentration auf Kernkompetenzen
\end{itemize}
\end{example}

\begin{KR}{Durchführung einer Make-or-Buy-Analyse}\\
\paragraph{Ausgangslage analysieren}
\begin{itemize}
    \item Produktanforderungen und -spezifikationen definieren
    \item Eigene Kernkompetenzen identifizieren
    \item Marktsituation und verfügbare Lieferanten analysieren
    \item Strategische Bedeutung des Produkts oder der Komponente bewerten
\end{itemize}

\paragraph{Kostenanalyse durchführen}
\begin{itemize}
    \item Kosten der Eigenfertigung ermitteln (Fixkosten, variable Kosten)
    \item Kosten des Fremdbezugs ermitteln (Einkaufspreis, Transaktionskosten)
    \item Break-Even-Punkt berechnen
    \item Kostenvergleich für verschiedene Szenarien erstellen
\end{itemize}

\paragraph{Qualitative Faktoren berücksichtigen}
\begin{itemize}
    \item Qualitätsanforderungen und -sicherung betrachten
    \item Know-how-Schutz und -Entwicklung bewerten
    \item Abhängigkeit von Lieferanten einschätzen
    \item Flexibilität bei Mengen- und Variantenschwankungen analysieren
    \item Steuerungs- und Kontrollmöglichkeiten vergleichen
\end{itemize}

\paragraph{Risikobewertung vornehmen}
\begin{itemize}
    \item Ausfallrisiken bewerten
    \item Qualitätsrisiken abschätzen
    \item Versorgungssicherheit einschätzen
    \item Preisänderungsrisiken berücksichtigen
\end{itemize}

\paragraph{Entscheidung treffen und umsetzen}
\begin{itemize}
    \item Gesamtbewertung von Kosten, qualitativen Faktoren und Risiken
    \item Entscheidung für Make oder Buy treffen
    \item Bei Buy-Entscheidung: Lieferanten auswählen und Vertrag gestalten
    \item Bei Make-Entscheidung: Produktionsprozess planen und implementieren
    \item Regelmässige Überprüfung der Entscheidung
\end{itemize}
\end{KR}

\begin{KR}{Terminplanung mit Rückwärtsterminierung}\\
\paragraph{Ausgangssituation analysieren}
\begin{itemize}
    \item Liefertermin des Produkts festlegen
    \item Produktionsschritte und deren Abhängigkeiten identifizieren
    \item Produktionszeiten für jeden Arbeitsvorgang ermitteln
    \item Rüst- und Transportzeiten berücksichtigen
    \item Verfügbare Kapazitäten prüfen
\end{itemize}

\paragraph{Rückwärtsterminierung durchführen}
\begin{itemize}
    \item Ausgehend vom Liefertermin den spätestmöglichen Endtermin des letzten Arbeitsvorgangs bestimmen
    \item Rückwärts die Endtermine aller vorgelagerten Arbeitsvorgänge berechnen
    \item Starttermine der Arbeitsvorgänge durch Abzug der Bearbeitungszeit vom Endtermin ermitteln
    \item Spätestmöglichen Produktionsstart bestimmen
\end{itemize}

\paragraph{Terminplan überprüfen}
\begin{itemize}
    \item Kapazitätskonflikte identifizieren
    \item Durchführbarkeit des Terminplans prüfen
    \item Bei Bedarf kritische Arbeitsvorgänge identifizieren und optimieren
    \item Mögliche Engpässe erkennen und Lösungen vorbereiten
\end{itemize}

\paragraph{Produktionssteuerung vorbereiten}
\begin{itemize}
    \item Produktionsaufträge erstellen
    \item Materialbereitstellung planen
    \item Ressourcen disponieren
    \item Kontrollmechanismen für die Einhaltung der Termine festlegen
\end{itemize}

\paragraph{Terminüberwachung einrichten}
\begin{itemize}
    \item Meilensteine definieren
    \item Soll-Ist-Vergleiche planen
    \item Eskalationsprozesse für Verzögerungen festlegen
    \item Nachsteuerungsmassnahmen vorbereiten
\end{itemize}
\end{KR}ection{Grundlagen der Leistungserstellung}

\begin{definition}{Leistungserstellungsprozess}\\
Der Leistungserstellungsprozess (auch Produktionsprozess) umfasst alle Aktivitäten, die zur Herstellung von Produkten oder zur Erbringung von Dienstleistungen erforderlich sind. Er ist ein zentraler Geschäftsprozess im St. Galler Management-Modell und bildet die Grundlage für die Wertschöpfung eines Unternehmens.
\end{definition}

\begin{definition}{Produktion}\\
Produktion bezeichnet die Umwandlung von Sachgütern und Dienstleistungen in andere Sachgüter und Dienstleistungen. Ziel ist es, durch den Einsatz der Produktionsfaktoren Arbeit, Betriebsmittel und Werkstoffe ein Produkt mit höherem Wert herzustellen.
\end{definition}

\begin{definition}{Produktionslogistik}\\
Die Produktionslogistik hat das Ziel, den Produktionsprozess art- und mengenmässig, räumlich und zeitlich abgestimmt mit den benötigten Produktionsfaktoren zu versorgen. Sie sorgt dafür, dass die
\begin{itemize}
    \item Richtige Menge der
    \item Richtigen Objekte am
    \item Richtigen Ort zum
    \item Richtigen Zeitpunkt in der
    \item Richtigen Qualität zu den
    \item Richtigen Kosten
\end{itemize}
zur Verfügung steht.
\end{definition}

\subsection{Evolution der Fertigung}

\begin{concept}{Industrie 1.0 bis 4.0}\\
Die Entwicklung der industriellen Produktion wird in vier Phasen eingeteilt:
\begin{itemize}
    \item \textbf{Industrie 1.0 (Manufaktur)}: Beginn der Industrialisierung durch Mechanisierung mit Wasser- und Dampfkraft (18. Jahrhundert)
    \item \textbf{Industrie 2.0 (Fliessband/Massenproduktion)}: Einführung arbeitsteiliger Massenproduktion mit Hilfe elektrischer Energie (Anfang 20. Jahrhundert, Ford/Taylor)
    \item \textbf{Industrie 3.0 (Automatisierung)}: Einsatz von Elektronik und IT zur Automatisierung der Produktion (ab 1970er Jahre)
    \item \textbf{Industrie 4.0 (Digitalisierung)}: Vernetzung der Produktion durch cyber-physische Systeme und Internet der Dinge, dezentrale Steuerung, individualisierte Produktion (Gegenwart)
\end{itemize}
\end{concept}

\subsection{Wertschöpfungsmanagement}

\begin{definition}{Wertschöpfungsmanagement}\\
Das Wertschöpfungsmanagement umfasst die Gestaltung und Steuerung aller Prozesse, die zur Schaffung von Mehrwert für den Kunden beitragen. Es lässt sich in strategisches und operatives Prozessmanagement unterteilen:
\begin{itemize}
    \item \textbf{Strategisches Prozessmanagement}: Ausgestaltung der Wertschöpfungsarchitektur mit In-/Outsourcing-Entscheiden zur effektiven Erfüllung der Kundenbedürfnisse
    \item \textbf{Operatives Prozessmanagement}: Optimierung der Geschäftsprozesse zur Steigerung von Effizienz und Effektivität
\end{itemize}
\end{definition}

\begin{concept}{Wertschöpfungsarchitektur}\\
Die Wertschöpfungsarchitektur beschreibt die strategische Ausgestaltung der Wertschöpfungskette eines Unternehmens. Grundtypen der Wertschöpfungsarchitektur sind:
\begin{itemize}
    \item \textbf{Komplettanbieter}: Abdeckung der gesamten Wertschöpfungskette
    \item \textbf{Spezialanbieter}: Fokussierung auf einen Teilbereich der Wertschöpfungskette
    \item \textbf{Lösungsanbieter}: Kombination von Produkten und Dienstleistungen zu Gesamtlösungen
\end{itemize}

Zur Analyse der Wertschöpfungsarchitektur dienen Instrumente wie die Prozesslandkarte, die Wertekette nach Porter und die Identifikation von Kernkompetenzen.
\end{concept}

\subsection{Make-or-Buy-Entscheidung}

\begin{definition}{Make-or-Buy-Entscheidung}\\
Die Make-or-Buy-Entscheidung ist die Abwägung zwischen Eigenfertigung (Make) und Fremdbezug (Buy) von Produkten, Komponenten oder Dienstleistungen.
\begin{itemize}
    \item \textbf{Make (Eigenfertigung)}: Das Unternehmen stellt das Produkt oder die Komponente selbst her
    \item \textbf{Buy (Fremdbezug)}: Das Unternehmen kauft das Produkt oder die Komponente von einem externen Lieferanten
\end{itemize}
\end{definition}

\begin{concept}{Vor- und Nachteile der Buy-Entscheidung}\\
Die Entscheidung für den Fremdbezug (Buy) hat folgende Vor- und Nachteile:
\begin{itemize}
    \item \textbf{Vorteile}:
    \begin{itemize}
        \item Konzentration auf Kernkompetenzen
        \item Nutzung des Know-hows spezialisierter Anbieter
        \item Flexibilität bei Nachfrageschwankungen
        \item Reduktion der Fixkosten
        \item Einsparung von Investitionen
    \end{itemize}
    \item \textbf{Nachteile}:
    \begin{itemize}
        \item Abhängigkeit von Lieferanten
        \item Verlust von Know-how
        \item Qualitätsrisiken
        \item Verlust der Kontrolle über Kosten und Prozesse
        \item Informationsverluste an Wettbewerber
    \end{itemize}
\end{itemize}
\end{concept}

\subsection{Produktionsprogrammplanung}

\begin{definition}{Produktionsprogramm}\\
Das Produktionsprogramm bestimmt Art, Menge und Zeitpunkt der zu produzierenden Produkte in einem Unternehmen.
\begin{itemize}
    \item \textbf{Produktionsprogrammbreite}: Anzahl der von einem Unternehmen hergestellten Produktarten
    \item \textbf{Produktionsprogrammtiefe}: Anzahl der Artikel und Typen, die innerhalb einer Produktart vom Unternehmen angeboten werden
\end{itemize}

Im Idealfall sind die Ressourcen (Mensch und Maschinen) optimal ausgelastet, d.h. weder unterbeschäftigt noch überbeansprucht.
\end{definition}

\begin{concept}{Fertigungsstrukturen}\\
Je nach Art der Produktion und des Produkts lassen sich verschiedene Fertigungsstrukturen unterscheiden:
\begin{itemize}
    \item \textbf{Nach räumlicher Anordnung}:
    \begin{itemize}
        \item Werkstattfertigung
        \item Fliessfertigung
        \item Gruppenfertigung
    \end{itemize}
    \item \textbf{Nach zeitlicher Kontinuität}:
    \begin{itemize}
        \item Kontinuierliche Fertigung
        \item Diskontinuierliche Fertigung
    \end{itemize}
    \item \textbf{Nach Auftragsbezug}:
    \begin{itemize}
        \item Auftragsfertigung
        \item Lagerfertigung
    \end{itemize}
\end{itemize}
\end{concept}

\subsection{Produktionsplanung und -steuerung (PPS)}

\begin{definition}{Produktionsplanung und -steuerung}\\
Die Produktionsplanung und -steuerung (PPS) umfasst die Planung, Steuerung und Überwachung des Produktionsprozesses. Sie besteht aus:
\begin{itemize}
    \item \textbf{Produktionsplanung}: Planung der Vorgänge 1-12 Monate im Voraus
    \item \textbf{Produktionssteuerung}: Freigabe und Steuerung der Aufträge 1-2 Wochen im Voraus
\end{itemize}

Ziele der PPS sind:
\begin{itemize}
    \item Kurze Lieferfristen garantieren
    \item Hohe Liefertreue sicherstellen
    \item Koordination mit Lieferanten
    \item Geringe Durchlaufzeit realisieren
\end{itemize}
\end{definition}

\begin{definition}{Terminplanung}\\
Die Terminplanung ist ein wichtiger Bestandteil der Produktionsplanung und legt fest, wann welche Produktionsschritte durchgeführt werden sollen. Es gibt zwei grundlegende Methoden der Terminplanung:
\begin{itemize}
    \item \textbf{Vorwärtsterminierung}: Planung vom frühestmöglichen Starttermin ausgehend nach vorne
    \begin{itemize}
        \item \textbf{Vorteile}: Geringerer Zeitdruck bei Produktion, hohe Terminsicherheit
        \item \textbf{Nachteile}: Längere Liegezeiten, höhere Kapitalbindung
    \end{itemize}
    \item \textbf{Rückwärtsterminierung}: Planung vom spätestmöglichen Fertigstellungstermin ausgehend nach hinten
    \begin{itemize}
        \item \textbf{Vorteile}: Vermeidung von langen Liegezeiten, geringere Kapitalbindung
        \item \textbf{Nachteile}: Hoher Termindruck, Störanfälligkeit, keine Zeitreserven
    \end{itemize}
\end{itemize}
\end{definition}

\begin{definition}{Kapazitätsplanung}\\
Die Kapazitätsplanung hat die Aufgabe, den Kapazitätsbedarf zu ermitteln und die verfügbaren Kapazitäten optimal zu nutzen. Sie umfasst:
\begin{itemize}
    \item \textbf{Kapazitätsbedarfsermittlung}: Feststellung des benötigten Kapazitätsbedarfs für jeden Auftrag
    \item \textbf{Maschinenbelegungsplanung}: Optimale Zuteilung der Aufträge zu den Maschinen
\end{itemize}

Ziele der Kapazitätsplanung sind die Minimierung der Durchlaufzeit und die Maximierung der Kapazitätsauslastung, was oft in einem Zielkonflikt steht.
\end{definition}

\begin{concept}{Engpassmanagement}\\
Trotz sorgfältiger Planung können Engpässe entstehen. Mögliche Massnahmen zur Bewältigung von Engpässen sind:
\begin{itemize}
    \item Rückweisung von Aufträgen
    \item Verschiebung von Aufträgen
    \item Kurzfristige Erhöhung der Produktionsfaktoren (z.B. Überstunden, Zusatzschichten)
    \item Langfristige Erhöhung der Kapazität (z.B. Investitionen in neue Maschinen)
\end{itemize}
\end{concept}

\important{not done!!!}

\subsection{Produktionsprogrammplanung}

\begin{definition}{Ziele der Produktionsprogrammplanung}\\
Die Produktionsprogrammplanung dient der optimalen Gestaltung des Produktangebots mit dem Ziel der Gewinnmaximierung. Die Festlegung des optimalen Produktionsprogramms hängt ab von:
\begin{itemize}
    \item Beschaffungssituation (Materialverfügbarkeit, Lieferzeiten)
    \item Kapazitätssituation (Maschinen, Personal, Flächen)
    \item Absatzsituation (Marktpotenzial, Nachfrage, Wettbewerb)
\end{itemize}
\end{definition}

\begin{concept}{Vor- und Nachteile einer breiten Produktpalette}\\
Die Breite des Produktionsprogramms hat verschiedene Auswirkungen:
\begin{itemize}
    \item \textbf{Vorteile einer breiten Produktpalette}:
    \begin{itemize}
        \item Bessere Abdeckung unterschiedlicher Kundenbedürfnisse
        \item Risikodiversifikation (Ausgleich von Nachfrageschwankungen)
        \item Cross-Selling-Potenziale
        \item Höhere Marktpräsenz
    \end{itemize}
    \item \textbf{Nachteile einer breiten Produktpalette}:
    \begin{itemize}
        \item Höhere Komplexitätskosten
        \item Geringere Stückzahlen pro Produkt (weniger Skaleneffekte)
        \item Höhere Lagerkosten
        \item Schwierigere Produktionsplanung und -steuerung
    \end{itemize}
\end{itemize}
\end{concept}

\subsection{Produktivitätssteigerung}

\begin{definition}{Massnahmen zur Produktivitätssteigerung}\\
Zur Steigerung der Produktivität können verschiedene Massnahmen ergriffen werden:
\begin{itemize}
    \item \textbf{Technische Massnahmen}:
    \begin{itemize}
        \item Automatisierung von Prozessen
        \item Einsatz neuer Technologien
        \item Optimierung des Materialflusses
        \item Verbesserung der Arbeitsplatzgestaltung
    \end{itemize}
    \item \textbf{Organisatorische Massnahmen}:
    \begin{itemize}
        \item Optimierung der Ablauforganisation
        \item Prozessanalyse und -optimierung
        \item Einführung von kontinuierlichen Verbesserungsprozessen
        \item Verbesserung der Qualitätssicherung
    \end{itemize}
    \item \textbf{Personalwirtschaftliche Massnahmen}:
    \begin{itemize}
        \item Qualifizierung und Weiterbildung der Mitarbeiter
        \item Verbesserung der Arbeitsbedingungen
        \item Einführung leistungsorientierter Vergütungssysteme
        \item Förderung der Mitarbeitermotivation
    \end{itemize}
\end{itemize}
\end{definition}

\begin{concept}{Lean Production}\\
Lean Production (schlanke Produktion) ist ein Konzept zur Produktivitätssteigerung, das auf die Vermeidung von Verschwendung abzielt. Grundprinzipien:
\begin{itemize}
    \item \textbf{Just-in-Time}: Produktion und Lieferung zum genau benötigten Zeitpunkt
    \item \textbf{Kaizen}: Kontinuierlicher Verbesserungsprozess
    \item \textbf{Pull-Prinzip}: Produktion wird durch tatsächlichen Bedarf gesteuert
    \item \textbf{Null-Fehler-Prinzip}: Fehlervermeidung statt Fehlerbeseitigung
    \item \textbf{Standardisierung}: Vereinheitlichung von Prozessen
    \item \textbf{Visualisierung}: Transparente Darstellung von Prozessen und Ergebnissen
\end{itemize}

Ziel ist die Vermeidung von sieben Arten der Verschwendung (Muda):
\begin{itemize}
    \item Überproduktion
    \item Wartezeit
    \item Transport
    \item Überbearbeitung
    \item Lagerbestände
    \item Bewegung
    \item Fehler und Nacharbeit
\end{itemize}
\end{concept}

\begin{example}
Beispiel für Lean Production in der Automobilindustrie:

Toyota hat mit dem Toyota-Produktionssystem (TPS) die Grundlage für Lean Production geschaffen:
\begin{itemize}
    \item Produktion nach dem Takt des Kundenbedarfs
    \item Verwendung von Kanban-Karten zur Steuerung des Materialflusses
    \item Jidoka: Automatisierung mit menschlicher Intelligenz (Maschinen stoppen bei Fehlern automatisch)
    \item Andon-System: Visualisierung von Problemen durch Signallampen
    \item Kontinuierliche Prozessverbesserung durch Kaizen-Teams
    \item Standardisierte Arbeitsabläufe
\end{itemize}

Ergebnis: Kürzere Durchlaufzeiten, geringere Lagerbestände, höhere Qualität, bessere Mitarbeiterproduktivität.
\end{example}

\begin{KR}{Implementierung von Lean Production}\\
\paragraph{Ist-Zustand analysieren}
\begin{itemize}
    \item Prozesse kartieren (Wertstromanalyse)
    \item Verschwendungen identifizieren
    \item Kennzahlen erheben (Durchlaufzeit, Bestände, Fehlerquoten)
    \item Mitarbeiter befragen
\end{itemize}

\paragraph{Ziele festlegen}
\begin{itemize}
    \item Konkrete Verbesserungsziele definieren
    \item Kennzahlen für die Zielerreichung bestimmen
    \item Prioritäten setzen
    \item Zeitrahmen festlegen
\end{itemize}

\paragraph{Mitarbeiter einbeziehen}
\begin{itemize}
    \item Lean-Prinzipien schulen
    \item Verantwortlichkeiten festlegen
    \item Verbesserungsteams bilden
    \item Führungskräfte als Vorbild etablieren
\end{itemize}

\paragraph{Werkzeuge einführen}
\begin{itemize}
    \item 5S-Methode (Sortieren, Systematisieren, Säubern, Standardisieren, Selbstdisziplin)
    \item Kanban-Systeme für die Materialsteuerung
    \item SMED (Single Minute Exchange of Die) für schnelles Umrüsten
    \item Poka Yoke (Fehlervermeidung durch technische Massnahmen)
    \item Total Productive Maintenance (TPM)
\end{itemize}

\paragraph{Kontinuierliche Verbesserung etablieren}
\begin{itemize}
    \item Regelmässige Verbesserungsworkshops
    \item Vorschlagswesen implementieren
    \item Erfolge messen und visualisieren
    \item Lerneffekte dokumentieren und weitergeben
    \item Nachhaltigkeit durch Standards sichern
\end{itemize}
\end{KR}