\section{Realisation}

\begin{definition}{Fertigungstypen}
\begin{itemize}
    \item \textbf{Einzelfertigung}: Einmalige oder sehr seltene Herstellung eines Produkts (z.B. Schiffsbau, Spezialmaschinen)
    \item \textbf{Serienfertigung}: Wiederholte Herstellung einer begrenzten Anzahl gleicher Produkte (z.B. Möbel, Haushaltsgeräte)
    \item \textbf{Massenfertigung}: Kontinuierliche Herstellung einer unbegrenzten Anzahl gleicher Produkte (z.B. Schrauben, Getränke)
\end{itemize}
\end{definition}



\begin{definition}{Fertigungsverfahren}
\begin{itemize}
    \item \textbf{Werkstattfertigung}: Die Arbeitsschritte werden in spezialisierten Werkstätten durchgeführt, das Werkstück wandert von Werkstatt zu Werkstatt
    \begin{itemize}
        \item \textbf{Vorteile}: Hohe Flexibilität, gut für Einzel- und Kleinserienfertigung
        \item \textbf{Nachteile}: Lange Transportwege, hohe Rüstzeiten, komplexe Steuerung
    \end{itemize}
    \item \textbf{Fliessfertigung}: Die Arbeitsschritte folgen in einer festen Reihenfolge aufeinander, das Werkstück wird entlang einer Linie bearbeitet
    \begin{itemize}
        \item \textbf{Vorteile}: Kurze Durchlaufzeiten, einfache Steuerung, hohe Produktivität
        \item \textbf{Nachteile}: Geringe Flexibilität, anfällig für Störungen, hoher Kapitalbedarf
    \end{itemize}
    \item \textbf{Gruppenfertigung}: Kombination aus Werkstatt- und Fliessfertigung, Bildung von Fertigungsinseln für komplette Teilefamilien
    \begin{itemize}
        \item \textbf{Vorteile}: Mittlere Flexibilität, kompakte Materialflüsse, gute Übersichtlichkeit
        \item \textbf{Nachteile}: Mittlere Komplexität, Gefahr der Unterlastung einzelner Maschinen
    \end{itemize}
\end{itemize}
\end{definition}



\begin{definition}{Produktionskennzahlen}
\begin{itemize}
    \item \textbf{Rentabilität}: $(\text{Ertrag} - \text{Aufwand})/(\text{Kapitaleinsatz}) = \text{Gewinn}/\text{Kapitaleinsatz}$
    \item \textbf{Produktivität}: $\frac{\text{Ausbringungsmenge}}{\text{Faktoreinsatzmenge}}$
    \item \textbf{Fehlerquote}: $\frac{\text{Fehlerhafte Produkte}}{\text{Total hergestellte Produkte}}$
    \item \textbf{Wirtschaftlichkeit}: $\frac{\text{Ertrag}}{\text{Aufwand}}$
\end{itemize}
\end{definition}

\begin{KR}{Durchführung einer Make-or-Buy-Analyse}\\
\textbf{Ausgangslage analysieren}
Produktanforderungen und -spezifikationen, eigene Kernkompetenzen, Marktsituation und Lieferanten analysieren, strategische Bedeutung des Produkts oder der Komponente bewerten.


\textbf{Kostenanalyse durchführen} Make or Buy Entscheidung


\textbf{Qualitative Faktoren berücksichtigen}
\begin{itemize}
    \item Qualitätsanforderungen und -sicherung betrachten
    \item Know-how-Schutz und -Entwicklung bewerten
    \item Abhängigkeit von Lieferanten einschätzen
    \item Flexibilität bei Mengen- und Variantenschwankungen analysieren
    \item Steuerungs- und Kontrollmöglichkeiten vergleichen
\end{itemize}

\textbf{Risikobewertung vornehmen}
Ausfallrisiken, Qualitätsrisiken, Versorgungssicherheit, Preisänderungsrisiken


\textbf{Entscheidung umsetzen}
\begin{itemize}
    \item Bei Buy-Entscheidung: Lieferanten auswählen und Vertrag gestalten
    \item Bei Make-Entscheidung: Produktionsprozess planen und implementieren
\end{itemize}
\end{KR}

\begin{KR}{Terminplanung mit Rückwärtsterminierung}\\
\textbf{Ausgangssituation analysieren}
Liefertermin, Produktionsschritte, Abhängigkeiten und Kapazitäten erfassen.

\textbf{Rückwärtsterminierung durchführen}
\begin{itemize}
    \item Ausgehend vom Liefertermin den spätestmöglichen Endtermin des letzten Arbeitsvorgangs bestimmen
    \item Rückwärts die Endtermine aller vorgelagerten Arbeitsvorgänge berechnen
    \item Starttermine der Arbeitsvorgänge durch Abzug der Bearbeitungszeit vom Endtermin ermitteln
    \item Spätestmöglichen Produktionsstart bestimmen
\end{itemize}

\textbf{Terminplan überprüfen}
Kapazitätsengpässe identifizieren, Pufferzeiten einplanen, kritische Pfade erkennen.

\textbf{Produktionssteuerung vorbereiten}
Produktionsaufträge erstellen, Materialbereitstellung planen, Ressourcen disponieren.

\textbf{Terminüberwachung einrichten}
Meilensteine definieren, Soll-Ist-Vergleiche planen, Eskalationsprozesse für Verzögerungen festlegen, Nachsteuerungsmassnahmen vorbereiten.

\end{KR}



\subsubsection{Evolution der Fertigung}

\begin{concept}{Industrie 1.0 bis 4.0}
\begin{itemize}
    \item \textbf{Industrie 1.0 (Manufaktur)}: Beginn der Industrialisierung durch Mechanisierung mit Wasser- und Dampfkraft (18. Jahrhundert)
    \item \textbf{Industrie 2.0 (Fliessband/Massenproduktion)}: Einführung arbeitsteiliger Massenproduktion mit Hilfe elektrischer Energie (Anfang 20. Jahrhundert, Ford/Taylor)
    \item \textbf{Industrie 3.0 (Automatisierung)}: Einsatz von Elektronik und IT zur Automatisierung der Produktion (ab 1970er Jahre)
    \item \textbf{Industrie 4.0 (Digitalisierung)}: Vernetzung der Produktion durch cyber-physische Systeme und Internet der Dinge, dezentrale Steuerung, individualisierte Produktion (Gegenwart)
\end{itemize}
\end{concept}

\subsubsection{Wertschöpfungsmanagement}

\begin{definition}{Wertschöpfungsmanagement}\\
Das Wertschöpfungsmanagement umfasst die Gestaltung und Steuerung aller Prozesse, die zur Schaffung von Mehrwert für den Kunden beitragen. Es lässt sich in strategisches und operatives Prozessmanagement unterteilen:
\begin{itemize}
    \item \textbf{Strategisches Prozessmanagement}: Ausgestaltung der Wertschöpfungsarchitektur mit In-/Outsourcing-Entscheiden zur effektiven Erfüllung der Kundenbedürfnisse
    \item \textbf{Operatives Prozessmanagement}: Optimierung der Geschäftsprozesse zur Steigerung von Effizienz und Effektivität
\end{itemize}
\end{definition}

\begin{concept}{Wertschöpfungsarchitektur}\\
Die Wertschöpfungsarchitektur beschreibt die strategische Ausgestaltung der Wertschöpfungskette eines Unternehmens. Grundtypen der Wertschöpfungsarchitektur sind:
\begin{itemize}
    \item \textbf{Komplettanbieter}: Abdeckung der gesamten Wertschöpfungskette
    \item \textbf{Spezialanbieter}: Fokussierung auf einen Teilbereich der Wertschöpfungskette
    \item \textbf{Lösungsanbieter}: Kombination von Produkten und Dienstleistungen zu Gesamtlösungen
\end{itemize}

Zur Analyse der Wertschöpfungsarchitektur dienen Instrumente wie die Prozesslandkarte, die Wertekette nach Porter und die Identifikation von Kernkompetenzen.
\end{concept}



\begin{KR}{Make-or-Buy-Entscheidung}
\begin{itemize}
    \item \textbf{Make (Eigenfertigung)}: Das Unternehmen stellt das Produkt oder die Komponente selbst her
    \item \textbf{Buy (Fremdbezug)}: Das Unternehmen kauft das Produkt oder die Komponente von einem externen Lieferanten
\end{itemize}
\end{KR}


\subsubsection{Produktionsprogrammplanung}

\begin{definition}{Produktionsprogramm}
\begin{itemize}
    \item \textbf{Produktionsprogrammbreite}: Anzahl der von einem Unternehmen hergestellten Produktarten
    \item \textbf{Produktionsprogrammtiefe}: Anzahl der Artikel und Typen, die innerhalb einer Produktart vom Unternehmen angeboten werden
\end{itemize}

Im Idealfall sind die Ressourcen (Mensch und Maschinen) optimal ausgelastet, d.h. weder unterbeschäftigt noch überbeansprucht.
\end{definition}

\begin{concept}{Fertigungsstrukturen}
\begin{itemize}
    \item \textbf{Nach räumlicher Anordnung}: Werkstattfertigung, Fliessfertigung, Gruppenfertigung
    \item \textbf{Nach zeitlicher Kontinuität}: Kontinuierliche/diskontinuierliche Fertigung
    \item \textbf{Nach Auftragsbezug}: Auftragsfertigung/Lagerfertigung
\end{itemize}
\end{concept}


\begin{definition}{Produktionsplanung und -steuerung}
Die Produktionsplanung und -steuerung (PPS) umfasst die Planung, Steuerung und Überwachung des Produktionsprozesses. Sie besteht aus:
\begin{itemize}
    \item \textbf{Produktionsplanung}: Planung der Vorgänge 1-12 Monate im Voraus
    \item \textbf{Produktionssteuerung}: Freigabe und Steuerung der Aufträge 1-2 Wochen im Voraus
\end{itemize}

Ziele der PPS sind: kurze Lieferzeiten, hohe Liefertreue, optimale Ressourcennutzung und Minimierung der Durchlaufzeiten.
\end{definition}


\subsubsection{Produktivitätssteigerung}

\begin{definition}{Massnahmen zur Produktivitätssteigerung}
\begin{itemize}
    \item \textbf{Technische Massnahmen}: Automatisierung, neue Technologien, Optimierung des Materialflusses, Verbesserung der Arbeitsplatzgestaltung
    \item \textbf{Organisatorische Massnahmen}: Optimierung der Ablauforganisation, Prozessanalyse und -optimierung, Einführung von kontinuierlichen Verbesserungsprozessen, Verbesserung der Qualitätssicherung
    \item \textbf{Personalwirtschaftliche Massnahmen}: Qualifizierung und Weiterbildung der Mitarbeiter, Verbesserung der Arbeitsbedingungen, Einführung leistungsorientierter Vergütungssysteme, Förderung der Mitarbeitermotivation
\end{itemize}
\end{definition}

\begin{concept}{Lean Production} (schlanke Produktion) ist ein Konzept zur Produktivitätssteigerung, das auf die Vermeidung von Verschwendung abzielt. Grundprinzipien:
\begin{itemize}
    \item \textbf{Just-in-Time}: Produktion und Lieferung zum genau benötigten Zeitpunkt
    \item \textbf{Kaizen}: Kontinuierlicher Verbesserungsprozess
    \item \textbf{Pull-Prinzip}: Produktion wird durch tatsächlichen Bedarf gesteuert
    \item \textbf{Null-Fehler-Prinzip}: Fehlervermeidung statt Fehlerbeseitigung
    \item \textbf{Standardisierung}: Vereinheitlichung von Prozessen
    \item \textbf{Visualisierung}: Transparente Darstellung von Prozessen und Ergebnissen
\end{itemize}

Ziel ist die Vermeidung von sieben Arten der Verschwendung (Muda):
Überproduktion, Wartezeiten, Transport, Überbearbeitung, Lagerbestände, Bewegung und Fehler/Nacharbeit.
\end{concept}
