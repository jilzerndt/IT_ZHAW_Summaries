\section{Absatz- und Produktionsplanung}

\subsection{Grundlagen der Materialwirtschaft}

\begin{definition}{Materialwirtschaft}\\
Die Materialwirtschaft ist für die Verwaltung, Planung und Steuerung der Materialbewegungen innerhalb eines Unternehmens und zwischen dem Unternehmen und seiner Umwelt zuständig. Sie erfüllt zwei Hauptaufgaben:
\begin{itemize}
    \item \textbf{Technische Aufgabe}: Sicherstellung der Materialverfügbarkeit zur richtigen Zeit in der richtigen Menge und Qualität am richtigen Ort
    \item \textbf{Wirtschaftliche Aufgabe}: Minimierung der Kosten für Beschaffung, Lagerung und Transport
\end{itemize}

Materialwirtschaft ist äusserst wichtig, denn fehlt im Leistungserstellungsprozess auch nur ein einziges Teilchen, kann dies die Produktion verzögern, lahmlegen und sehr kostspielig werden lassen.
\end{definition}

\begin{definition}{Bestandteile der Materialwirtschaft}\\
Die Materialwirtschaft umfasst verschiedene Teilbereiche:
\begin{itemize}
    \item \textbf{Materialdisposition}: Planung des Materialbedarfs und der Beschaffungszeitpunkte
    \item \textbf{Einkauf/Beschaffung}: Beschaffung der benötigten Materialien zu optimalen Konditionen
    \item \textbf{Warenannahme}: Kontrolle der eingehenden Materialien
    \item \textbf{Lagerhaltung}: Lagern und Verwalten der Materialien
    \item \textbf{Materialbereitstellung}: Transport der Materialien zum Einsatzort
    \item \textbf{Entsorgung}: Verwertung und Entsorgung von Abfällen und Rückständen
\end{itemize}
\end{definition}

\subsection{Beschaffungsprozesse}

\begin{definition}{Beschaffungsobjekte}\\
Je nach Verwendungszweck unterscheidet man verschiedene Beschaffungsobjekte:
\begin{itemize}
    \item \textbf{Rohstoffe}: Hauptbestandteile des Produkts, die verarbeitet werden
    \item \textbf{Hilfsstoffe}: Nebenbestandteile des Produkts
    \item \textbf{Betriebsstoffe}: Materialien, die bei der Herstellung verbraucht werden, aber nicht ins Produkt eingehen (z.B. Schmiermittel, Reinigungsmittel)
    \item \textbf{Montageteile}: Vorproduzierte Komponenten, die in das Produkt eingebaut werden
    \item \textbf{Handelswaren}: Nicht für den Produktionsprozess bestimmte Güter, die unverändert weiterverkauft werden
\end{itemize}
\end{definition}

\begin{definition}{Beschaffungskonzepte}\\
Es existieren verschiedene Beschaffungskonzepte in der Materialwirtschaft:
\begin{itemize}
    \item \textbf{Order-to-Stock}: Materialbeschaffung auf Vorrat (Lagerhaltung)
    \item \textbf{Order-to-Make}: Materialbeschaffung nach Eingang eines Kundenauftrags
    \item \textbf{Just-in-Time (JIT)}: Material wird genau zum Zeitpunkt des Bedarfs geliefert
    \item \textbf{Just-in-Sequence (JIS)}: Weiterentwicklung des JIT, Material wird nicht nur zum richtigen Zeitpunkt, sondern auch in der richtigen Reihenfolge geliefert
\end{itemize}
\end{definition}

\begin{definition}{Insourcing und Outsourcing}\\
Unternehmen stehen bei der Leistungserstellung vor der grundsätzlichen Entscheidung "Make or Buy":
\begin{itemize}
    \item \textbf{Insourcing}: Verlagerung von zuvor im Markt bezogenen Leistungen in die eigene Wertschöpfung
    \begin{itemize}
        \item \textbf{Vorteile}: Reduktion von Lieferzeiten, Unabhängigkeit von Lieferanten, Aufrechterhaltung von Qualitätsstandards, Auslastung eigener Fertigungskapazitäten
    \end{itemize}
    \item \textbf{Outsourcing}: Verlagerung von Teilen der Wertschöpfung auf externe Lieferanten (langfristig)
    \begin{itemize}
        \item \textbf{Vorteile}: Minimierung der Fixkosten, flexibel planbare Beschaffungsmenge und Zeitspanne, Minimierung der Lagerkosten, Ausweichmöglichkeit bei Kapazitätsengpässen
    \end{itemize}
\end{itemize}
\end{definition}

\begin{concept}{Make-or-Buy-Entscheidung}\\
Ziel der Make-or-Buy-Entscheidung ist die Minimierung der bei der Materialbereitstellung anfallenden Kosten.
\begin{itemize}
    \item \textbf{Kostenfunktion "make"}: $K = \text{Variable Kosten pro Stück} + \text{Fixkosten}$
    \item \textbf{Kostenfunktion "buy"}: $K = \text{Variable Kosten pro Stück}$
\end{itemize}

Der Break-Even-Punkt (Entscheidungsgrenze) wird erreicht, wenn beide Kostenfunktionen gleich sind:
\begin{itemize}
    \item $\text{Variable Kosten pro Stück (make)} \times x + \text{Fixkosten} = \text{Variable Kosten pro Stück (buy)} \times x$
    \item $x = \frac{\text{Fixkosten}}{\text{Variable Kosten pro Stück (buy)} - \text{Variable Kosten pro Stück (make)}}$
\end{itemize}

Wenn die Produktionsmenge grösser als $x$ ist, ist Eigenfertigung (make) günstiger, andernfalls Fremdbezug (buy).
\end{concept}

\subsection{Lagerung und Kosten}

\begin{definition}{Kostenanfall in der Materialwirtschaft}\\
Die Materialwirtschaft steht im Spannungsfeld verschiedener Kostenfaktoren:
\begin{itemize}
    \item \textbf{Beschaffungskosten}: Kosten für die Beschaffung (Bestellkosten, Transportkosten)
    \item \textbf{Lagerkosten}: Kosten für die Lagerung
    \begin{itemize}
        \item \textbf{Kapitalbindung}: Die eingelagerten Waren binden Geld, welches nicht für anderes (z.B. gewinnbringende Geldanlage) genutzt werden kann
        \item \textbf{Lagerunterhaltungskosten}: Kosten für Lagerplatz, Bewachung, Temperierung, Wertverlust, Verderben, Lagerschäden, Diebstahl, Versicherung, etc.
    \end{itemize}
    \item \textbf{Fehlmengenkosten}: Kosten, die entstehen, wenn Material nicht rechtzeitig oder in unzureichender Menge zur Verfügung steht (Produktionsausfälle, Konventionalstrafen, Umsatzeinbussen)
\end{itemize}
\end{definition}

\begin{concept}{Magisches Dreieck der Materialwirtschaft}\\
Das magische Dreieck der Materialwirtschaft besteht aus drei Komponenten, die gleichzeitig die Ziele der Materialwirtschaft darstellen und im Zielkonflikt zueinander stehen:
\begin{itemize}
    \item \textbf{Lieferbereitschaft}: Soll möglichst hoch sein (aber: hohe Bestände binden Kapital)
    \item \textbf{Beschaffungskosten}: Sollen möglichst tief sein (aber: grössere Bestellmengen erhöhen die Lagerbestände)
    \item \textbf{Kapitalbindung}: Soll so tief wie möglich gehalten sein (aber: geringe Bestände erhöhen das Risiko von Lieferengpässen)
\end{itemize}

Mögliche Zielkonflikte:
\begin{itemize}
    \item \textbf{Lieferbereitschaft vs. Kapitalbindung}: Hohe Lieferbereitschaft erfordert hohe Lagerbestände, was zu hoher Kapitalbindung führt
    \item \textbf{Beschaffungskosten vs. Kapitalbindung}: Niedrige Beschaffungskosten durch Mengenrabatte bei Grossbestellungen führen zu hohen Lagerbeständen und somit zu hoher Kapitalbindung
\end{itemize}
\end{concept}

\subsection{Materialanalyse und -klassifikation}

\begin{definition}{ABC-Analyse}\\
Die ABC-Analyse ist ein Verfahren zur Klassifizierung von Materialien nach ihrer wirtschaftlichen Bedeutung. Sie hilft bei der Identifikation der Beschaffungsobjekte, welche wertvoll sind und damit viel Kapital binden:
\begin{itemize}
    \item \textbf{A-Güter}: Hoher Wertanteil (ca. 70-80\% des Gesamtwerts), geringer Mengenanteil (ca. 10-20\% der Gesamtmenge)
    \item \textbf{B-Güter}: Mittlerer Wertanteil (ca. 15-20\% des Gesamtwerts), mittlerer Mengenanteil (ca. 30\% der Gesamtmenge)
    \item \textbf{C-Güter}: Geringer Wertanteil (ca. 5-10\% des Gesamtwerts), hoher Mengenanteil (ca. 50-60\% der Gesamtmenge)
\end{itemize}

Anwendungsmöglichkeiten:
\begin{itemize}
    \item Kostenarten im Verhältnis zu Kostenvolumen
    \item Optimierung der Lagerwirtschaft
    \item Umsatzanteil von Lieferanten-/Kundengruppen ("Key-Account-Management")
\end{itemize}
\end{definition}

\begin{definition}{XYZ-Analyse}\\
Die XYZ-Analyse ergänzt die ABC-Analyse und klassifiziert Materialien nach der Vorhersagegenauigkeit ihres Bedarfs:
\begin{itemize}
    \item \textbf{X-Güter}: Konstanter Verbrauch, hohe Vorhersagegenauigkeit
    \item \textbf{Y-Güter}: Schwankender Verbrauch, mittlere Vorhersagegenauigkeit
    \item \textbf{Z-Güter}: Unregelmässiger Verbrauch, geringe Vorhersagegenauigkeit
\end{itemize}

Die XYZ-Analyse hilft bei der Entscheidung über das geeignete Beschaffungskonzept:
\begin{itemize}
    \item \textbf{X-Güter}: Eignen sich für Just-in-Time-Beschaffung
    \item \textbf{Y- und Z-Güter}: Erfordern höhere Sicherheitsbestände
\end{itemize}
\end{definition}

\subsection{Lagerorganisation}

\begin{definition}{Arten von Lagern}\\
Je nach Position im Produktionsprozess unterscheidet man verschiedene Arten von Lagern:
\begin{itemize}
    \item \textbf{Eingangslager}: Lager vor der Produktion, versorgen die Produktion mit den nötigen Materialien
    \item \textbf{Zwischenlager}: Lager parallel zur Produktion, dienen als Puffer zwischen verschiedenen Produktionsstufen
    \item \textbf{Fertigwarenlager}: Lager nach der Produktion, speichern die fertigen Produkte bis zum Versand
\end{itemize}
\end{definition}

\begin{definition}{Lagerfunktionen}\\
Lager erfüllen verschiedene Funktionen:
\begin{itemize}
    \item \textbf{Zeitüberbrückungsfunktion}: Ausgleich zeitlicher Unterschiede zwischen Beschaffung, Produktion und Absatz
    \item \textbf{Sicherungsfunktion}: Absicherung gegen Lieferengpässe, Produktionsstörungen oder schwankende Nachfrage
    \item \textbf{Spekulationsfunktion}: Ausnutzung günstiger Einkaufskonditionen oder erwarteter Preissteigerungen
    \item \textbf{Transformationsfunktion}: Umwandlung von Liefereinheiten in Produktions- oder Verkaufseinheiten
    \item \textbf{Servicefunktion}: Sicherstellung einer hohen Lieferbereitschaft gegenüber Kunden
\end{itemize}
\end{definition}

\begin{definition}{Lagerkosten}\\
Lagerkosten setzen sich aus verschiedenen Komponenten zusammen:
\begin{itemize}
    \item \textbf{Lagerunterhalt}: Mietkosten, Energiekosten, Versicherung des Lagerguts, Instandhaltung, Kosten des Bestandesrisikos (z.B. Diebstahl, Feuer)
    \item \textbf{Kapitalbindung}: Entgangene Zinsen alternativer Anlagemöglichkeiten (Opportunitätskosten)
\end{itemize}
\end{definition}

\subsection{Lagerkennzahlen}

\begin{definition}{Lagerkennzahlen}\\
Zur Steuerung und Kontrolle der Lagerhaltung dienen verschiedene Kennzahlen:
\begin{itemize}
    \item \textbf{Durchschnittlicher Lagerbestand}: \\
    $\frac{\text{Anfangsbestand} + \text{Endbestand}}{2}$
    \item \textbf{Lagerumschlagshäufigkeit}: \\
    $\frac{\text{Jahresverbrauch}}{\text{Durchschnittlicher Lagerbestand}}$
    \item \textbf{Durchschnittliche Lagerdauer (in Tagen)}: \\
    $\frac{360}{\text{Umschlagshäufigkeit}}$
\end{itemize}

Eine hohe Lagerumschlagshäufigkeit (und damit eine geringe Lagerdauer) ist in der Regel anzustreben, da dies auf eine effiziente Lagerhaltung mit geringer Kapitalbindung hinweist.
\end{definition}

\begin{KR}{Durchführung einer ABC-Analyse}\\
\paragraph{Daten sammeln und aufbereiten}
\begin{itemize}
    \item Artikelnummern und Bezeichnungen aller Lagermaterialien erfassen
    \item Verbrauchsmengen der letzten Periode ermitteln
    \item Einstandspreise pro Einheit bestimmen
    \item Lagerwert jedes Artikels berechnen (Menge × Einstandspreis)
\end{itemize}

\paragraph{Daten sortieren und klassifizieren}
\begin{itemize}
    \item Artikel nach Lagerwert absteigend sortieren
    \item Kumulierten Lagerwert und kumulierten prozentualen Anteil berechnen
    \item A-Artikel: Erste 10-20\% der Artikel mit ca. 70-80\% des Gesamtwerts
    \item B-Artikel: Nächste 30\% der Artikel mit ca. 15-20\% des Gesamtwerts
    \item C-Artikel: Verbleibende 50-60\% der Artikel mit ca. 5-10\% des Gesamtwerts
\end{itemize}

\paragraph{Massnahmen ableiten}
\begin{itemize}
    \item A-Artikel: Intensives Bestandsmanagement, genaue Bedarfsermittlung, häufige Inventuren, niedrige Sicherheitsbestände
    \item B-Artikel: Mittlere Kontrollintensität, regelmässige Inventuren, mittlere Sicherheitsbestände
    \item C-Artikel: Geringe Kontrollintensität, vereinfachte Bestellverfahren, höhere Sicherheitsbestände
\end{itemize}

\paragraph{Ergebniskontrolle}
\begin{itemize}
    \item Grafische Darstellung der ABC-Verteilung (Lorenz-Kurve)
    \item Berechnung der Lagerkennzahlen vor und nach der Optimierung
    \item Berechnung der Kosteneinsparungen
\end{itemize}
\end{KR}

\begin{KR}{Make-or-Buy-Entscheidung}\\
\paragraph{Kostenanalyse der Eigenfertigung}
\begin{itemize}
    \item Fixkosten der Eigenfertigung ermitteln (Maschinen, Werkzeuge, Fixanteile des Personals)
    \item Variable Kosten pro Stück berechnen (Material, variable Personalkosten, Energie)
    \item Kostenfunktion "make" aufstellen: $K = \text{Variable Kosten pro Stück} \times x + \text{Fixkosten}$
\end{itemize}

\paragraph{Kostenanalyse des Fremdbezugs}
\begin{itemize}
    \item Angebotspreise verschiedener Lieferanten einholen
    \item Zusatzkosten berücksichtigen (Transport, Qualitätskontrolle, Beschaffungsaufwand)
    \item Kostenfunktion "buy" aufstellen: $K = \text{Variable Kosten pro Stück} \times x$
\end{itemize}

\paragraph{Break-Even-Analyse durchführen}
\begin{itemize}
    \item Break-Even-Menge berechnen: $x = \frac{\text{Fixkosten}}{\text{Variable Kosten pro Stück (buy)} - \text{Variable Kosten pro Stück (make)}}$
    \item Kosten für unterschiedliche Mengen vergleichen
    \item Sensitivitätsanalyse durchführen (Veränderung von Preisen, Kosten, Mengen)
\end{itemize}

\paragraph{Entscheidungsfindung}
\begin{itemize}
    \item Erwartete Produktionsmenge mit Break-Even-Menge vergleichen
    \item Qualitative Faktoren berücksichtigen (Know-how, Abhängigkeit, Flexibilität)
    \item Risiken analysieren (Lieferantenausfall, Kapazitätsengpässe)
    \item Entscheidung für Make oder Buy treffen
\end{itemize}
\end{KR}

\begin{example}
Beispiel zur ABC-Analyse:

\textbf{Ausgangsdaten:}

\begin{tabular}{|l|r|r|r|}
\hline
\textbf{Lagerartikel} & \textbf{Menge (Stück)} & \textbf{Einstandspreis pro Stück (CHF)} & \textbf{Lagerwert (CHF)} \\
\hline
Artikel 1 & 55'000 & 12.00 & 660'000 \\
\hline
Artikel 2 & 30'000 & 9.80 & 294'000 \\
\hline
Artikel 3 & 20'000 & 40.00 & 800'000 \\
\hline
Artikel 4 & 4'360 & 77.00 & 335'720 \\
\hline
Artikel 5 & 150'000 & 0.50 & 75'000 \\
\hline
Artikel 6 & 200'000 & 0.30 & 60'000 \\
\hline
Artikel 7 & 9'300 & 19.00 & 176'700 \\
\hline
\textbf{Summe} & & & \textbf{2'401'420} \\
\hline
\end{tabular}

\textbf{Sortierung nach Lagerwert und Klassifizierung:}

\begin{tabular}{|l|r|r|r|r|r|}
\hline
\textbf{Lagerartikel} & \textbf{Lagerwert (CHF)} & \textbf{\% Wert} & \textbf{Kum. \% Wert} & \textbf{\% Menge} & \textbf{Klasse} \\
\hline
Artikel 3 & 800'000 & 33.3\% & 33.3\% & 14.3\% & A \\
\hline
Artikel 1 & 660'000 & 27.5\% & 60.8\% & 14.3\% & A \\
\hline
Artikel 4 & 335'720 & 14.0\% & 74.8\% & 14.3\% & A \\
\hline
Artikel 2 & 294'000 & 12.2\% & 87.0\% & 14.3\% & B \\
\hline
Artikel 7 & 176'700 & 7.4\% & 94.4\% & 14.3\% & B \\
\hline
Artikel 5 & 75'000 & 3.1\% & 97.5\% & 14.3\% & C \\
\hline
Artikel 6 & 60'000 & 2.5\% & 100.0\% & 14.3\% & C \\
\hline
\textbf{Summe} & \textbf{2'401'420} & \textbf{100.0\%} & & \textbf{100.0\%} & \\
\hline
\end{tabular}

\textbf{Ergebnis:}
\begin{itemize}
    \item A-Artikel (3, 1, 4): 74.8\% des Werts, 42.9\% der Artikelmenge
    \item B-Artikel (2, 7): 19.6\% des Werts, 28.6\% der Artikelmenge
    \item C-Artikel (5, 6): 5.6\% des Werts, 28.6\% der Artikelmenge
\end{itemize}

\textbf{Empfehlung:}
Die A-Artikel (3, 1, 4) binden drei Viertel des Kapitals und sollten daher intensiv überwacht werden. Eine Optimierung der Bestellmengen und Lagerbestände bei diesen Artikeln bietet das grösste Einsparpotenzial.
\end{example}