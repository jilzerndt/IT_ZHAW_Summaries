\section{Absatz- und Produktionsplanung}

\begin{definition}{Materialwirtschaft}
     Verwaltung, Planung und Steuerung der Materialbewegungen innerhalb eines Unternehmens und zwischen dem Unternehmen und seiner Umwelt zuständig. 
\begin{itemize}
    \item \textbf{Technische Aufgabe}: Sicherstellung der Materialverfügbarkeit zur richtigen Zeit in der richtigen Menge und Qualität am richtigen Ort
    \item \textbf{Wirtschaftliche Aufgabe}: Minimierung der Kosten für Beschaffung, Lagerung und Transport
\end{itemize}

\end{definition}

\begin{definition}{Bestandteile der Materialwirtschaft}
\begin{itemize}
    \item \textbf{Materialdisposition}: Planung des Materialbedarfs und der Beschaffungszeitpunkte
    \item \textbf{Einkauf/Beschaffung}: Beschaffung der benötigten Materialien zu optimalen Konditionen
    \item \textbf{Warenannahme}: Kontrolle der eingehenden Materialien
    \item \textbf{Lagerhaltung}: Lagern und Verwalten der Materialien
    \item \textbf{Materialbereitstellung}: Transport der Materialien zum Einsatzort
    \item \textbf{Entsorgung}: Verwertung und Entsorgung von Abfällen und Rückständen
\end{itemize}
\end{definition}

\subsubsection{Beschaffungsprozesse}

\begin{definition}{Beschaffungsobjekte}
\begin{itemize}
    \item \textbf{Rohstoffe}: Hauptbestandteile des Produkts, die verarbeitet werden
    \item \textbf{Hilfsstoffe}: Nebenbestandteile des Produkts
    \item \textbf{Betriebsstoffe}: Materialien, die bei der Herstellung verbraucht werden, aber nicht ins Produkt eingehen (z.B. Schmiermittel, Reinigungsmittel)
    \item \textbf{Montageteile}: Vorproduzierte Komponenten, die in das Produkt eingebaut werden
    \item \textbf{Handelswaren}: Nicht für den Produktionsprozess bestimmte Güter, die unverändert weiterverkauft werden
\end{itemize}
\end{definition}

\begin{definition}{Beschaffungskonzepte}
\begin{itemize}
    \item \textbf{Order-to-Stock}: Materialbeschaffung auf Vorrat (Lagerhaltung)
    \item \textbf{Order-to-Make}: Materialbeschaffung nach Eingang eines Kundenauftrags
    \item \textbf{Just-in-Time (JIT)}: Material wird genau zum Zeitpunkt des Bedarfs geliefert
    \item \textbf{Just-in-Sequence (JIS)}: Weiterentwicklung des JIT, Material wird nicht nur zum richtigen Zeitpunkt, sondern auch in der richtigen Reihenfolge geliefert
\end{itemize}
\end{definition}

\begin{definition}{Insourcing und Outsourcing}
Unternehmen stehen bei der Leistungserstellung vor der grundsätzlichen Entscheidung "Make or Buy":

\textbf{Insourcing}: Verlagerung von zuvor im Markt bezogenen Leistungen in die eigene Wertschöpfung
    \begin{itemize}
        \item \textbf{Vorteile}: Reduktion von Lieferzeiten, Unabhängigkeit von Lieferanten, Aufrechterhaltung von Qualitätsstandards, Auslastung eigener Fertigungskapazitäten
    \end{itemize}
\textbf{Outsourcing}: Verlagerung von Teilen der Wertschöpfung auf externe Lieferanten (langfristig)
    \begin{itemize}
        \item \textbf{Vorteile}: Minimierung der Fixkosten, flexibel planbare Beschaffungsmenge und Zeitspanne, Minimierung der Lagerkosten, Ausweichmöglichkeit bei Kapazitätsengpässen
    \end{itemize}
\end{definition}

\begin{concept}{Make-or-Buy-Entscheidung} Minimierung der bei der Materialbereitstellung anfallenden Kosten.
\begin{itemize}
    \item \textbf{Kostenfunktion "make"}: $K = \text{Variable Kosten pro Stück} + \text{Fixkosten}$
    \item \textbf{Kostenfunktion "buy"}: $K = \text{Variable Kosten pro Stück}$
\end{itemize}

Der Break-Even-Punkt (Entscheidungsgrenze) wird erreicht, wenn beide Kostenfunktionen gleich sind:

$$\text{make} \times x + \text{Fixkosten} = \text{buy} \times x \text{ und } x = \frac{\text{Fixkosten}}{\text{buy} - \text{make}}$$

Wenn die Produktionsmenge grösser als $x$ ist, ist Eigenfertigung (make) günstiger, andernfalls Fremdbezug (buy).
\end{concept}

\subsubsection{Lagerung und Kosten}

\begin{definition}{Kostenanfall in der Materialwirtschaft}
\begin{itemize}
    \item \textbf{Beschaffungskosten}: Kosten für die Beschaffung (Bestell-/Transportkosten)
    \item \textbf{Lagerkosten}: Kosten für die Lagerung
    \begin{itemize}
        \item \textbf{Kapitalbindung}: Die eingelagerten Waren binden Geld, welches nicht für anderes (z.B. gewinnbringende Geldanlage) genutzt werden kann
        \item \textbf{Lagerunterhaltungskosten}: Kosten für Lagerplatz, Bewachung, Temperierung, Wertverlust, Verderben, Lagerschäden, Diebstahl, Versicherung, etc.
    \end{itemize}
    \item \textbf{Fehlmengenkosten}: Kosten, die entstehen, wenn Material nicht rechtzeitig oder in unzureichender Menge zur Verfügung steht (Produktionsausfälle, Konventionalstrafen, Umsatzeinbussen)
\end{itemize}
\end{definition}

\begin{concept}{Magisches Dreieck der Materialwirtschaft}
\begin{itemize}
    \item \textbf{Lieferbereitschaft}: Soll möglichst hoch sein (aber: hohe Bestände binden Kapital)
    \item \textbf{Beschaffungskosten}: Sollen möglichst tief sein (aber: grössere Bestellmengen erhöhen die Lagerbestände)
    \item \textbf{Kapitalbindung}: Soll so tief wie möglich gehalten sein (aber: geringe Bestände erhöhen das Risiko von Lieferengpässen)
\end{itemize}

Mögliche Zielkonflikte:
\begin{itemize}
    \item \textbf{Lieferbereitschaft vs. Kapitalbindung}: Hohe Lieferbereitschaft erfordert hohe Lagerbestände, was zu hoher Kapitalbindung führt
    \item \textbf{Beschaffungskosten vs. Kapitalbindung}: Niedrige Beschaffungskosten durch Mengenrabatte bei Grossbestellungen führen zu hohen Lagerbeständen und somit zu hoher Kapitalbindung
\end{itemize}
\end{concept}

\subsubsection{Materialanalyse und -klassifikation}


\begin{KR}{Durchführung einer ABC-Analyse}\\
\textbf{Daten sammeln und aufbereiten}
\begin{itemize}
    \item Artikelnummern und Bezeichnungen aller Lagermaterialien erfassen
    \item Verbrauchsmengen der letzten Periode ermitteln
    \item Einstandspreise pro Einheit bestimmen
    \item Lagerwert jedes Artikels berechnen (Menge × Einstandspreis)
\end{itemize}

\textbf{Daten sortieren und klassifizieren}
\begin{itemize}
    \item Artikel nach Lagerwert absteigend sortieren
    \item Kumulierten Lagerwert und kumulierten prozentualen Anteil berechnen
    \item A-Artikel: Erste 10-20\% der Artikel mit ca. 70-80\% des Gesamtwerts
    \item B-Artikel: Nächste 30\% der Artikel mit ca. 15-20\% des Gesamtwerts
    \item C-Artikel: Verbleibende 50-60\% der Artikel mit ca. 5-10\% des Gesamtwerts
\end{itemize}

\textbf{Massnahmen ableiten}
\begin{itemize}
    \item A-Artikel: Intensives Bestandsmanagement, genaue Bedarfsermittlung, häufige Inventuren, niedrige Sicherheitsbestände
    \item B-Artikel: Mittlere Kontrollintensität, regelmässige Inventuren, mittlere Sicherheitsbestände
    \item C-Artikel: Geringe Kontrollintensität, vereinfachte Bestellverfahren, höhere Sicherheitsbestände
\end{itemize}

\textbf{Ergebniskontrolle}
\begin{itemize}
    \item Grafische Darstellung der ABC-Verteilung (Lorenz-Kurve)
    \item Berechnung der Lagerkennzahlen vor und nach der Optimierung
    \item Berechnung der Kosteneinsparungen
\end{itemize}
\end{KR}

\begin{definition}{XYZ-Analyse}\\
Die XYZ-Analyse ergänzt die ABC-Analyse und klassifiziert Materialien nach der Vorhersagegenauigkeit ihres Bedarfs:
\begin{itemize}
    \item \textbf{X-Güter}: Konstanter Verbrauch, hohe Vorhersagegenauigkeit
    \item \textbf{Y-Güter}: Schwankender Verbrauch, mittlere Vorhersagegenauigkeit
    \item \textbf{Z-Güter}: Unregelmässiger Verbrauch, geringe Vorhersagegenauigkeit
\end{itemize}

Die XYZ-Analyse hilft bei Entscheidung über geeignetes Beschaffungskonzept:
\begin{itemize}
    \item \textbf{X-Güter}: Eignen sich für Just-in-Time-Beschaffung
    \item \textbf{Y- und Z-Güter}: Erfordern höhere Sicherheitsbestände
\end{itemize}
\end{definition}

\subsubsection{Lagerorganisation}

\begin{definition}{Arten von Lagern}
\begin{itemize}
    \item \textbf{Eingangslager}: Lager vor der Produktion, versorgen die Produktion mit den nötigen Materialien
    \item \textbf{Zwischenlager}: Lager parallel zur Produktion, dienen als Puffer zwischen verschiedenen Produktionsstufen
    \item \textbf{Fertigwarenlager}: Lager nach der Produktion, speichern die fertigen Produkte bis zum Versand
\end{itemize}
\end{definition}

\begin{definition}{Lagerfunktionen}
\begin{itemize}
    \item \textbf{Zeitüberbrückungsfunktion}: Ausgleich zeitlicher Unterschiede zwischen Beschaffung, Produktion und Absatz
    \item \textbf{Sicherungsfunktion}: Absicherung gegen Lieferengpässe, Produktionsstörungen oder schwankende Nachfrage
    \item \textbf{Spekulationsfunktion}: Ausnutzung günstiger Einkaufskonditionen oder erwarteter Preissteigerungen
    \item \textbf{Transformationsfunktion}: Umwandlung von Liefereinheiten in Produktions- oder Verkaufseinheiten
    \item \textbf{Servicefunktion}: Sicherstellung einer hohen Lieferbereitschaft gegenüber Kunden
\end{itemize}
\end{definition}

\begin{definition}{Lagerkosten}
\begin{itemize}
    \item \textbf{Lagerunterhalt}: Mietkosten, Energiekosten, Versicherung des Lagerguts, Instandhaltung, Kosten des Bestandesrisikos (z.B. Diebstahl, Feuer)
    \item \textbf{Kapitalbindung}: Entgangene Zinsen alternativer Anlagemöglichkeiten (Opportunitätskosten)
\end{itemize}
\end{definition}

\subsubsection{Lagerkennzahlen}

\begin{definition}{Lagerkennzahlen}
\begin{itemize}
    \item \textbf{Durchschnittlicher Lagerbestand}: 
    $\frac{\text{Anfangsbestand} + \text{Endbestand}}{2}$
    \item \textbf{Lagerumschlagshäufigkeit}: 
    $\frac{\text{Jahresverbrauch}}{\text{Durchschnittlicher Lagerbestand}}$
    \item \textbf{Durchschnittliche Lagerdauer (in Tagen)}: 
    $\frac{360}{\text{Umschlagshäufigkeit}}$
\end{itemize}

Eine hohe Lagerumschlagshäufigkeit (und damit eine geringe Lagerdauer) ist in der Regel anzustreben, da dies auf eine effiziente Lagerhaltung mit geringer Kapitalbindung hinweist.
\end{definition}



