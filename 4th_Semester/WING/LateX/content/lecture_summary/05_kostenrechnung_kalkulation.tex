
\section{Kalkulation}

\begin{definition}{Kalkulation}
     Berechnung der Kosten und Preise von Produkten oder Dienstleistungen. Sie dient der Preisfindung und Preisbeurteilung und ist eine wichtige Grundlage für die Offertenerstellung. Die Hauptaufgaben der Kalkulation sind:
    
    Ermittlung der Selbstkosten (Totalkosten), Preisfindung, Preisbeurteilung und Offertenerstellung. 
\end{definition}

\begin{concept}{Arten der Kalkulation}
\begin{itemize}
    \item Nach dem Zeitpunkt:
    \begin{itemize}
        \item \textbf{Vorkalkulation}: Vor der Produktion (für Angebote)
        \item \textbf{Nachkalkulation}: Nach der Produktion (zur Kontrolle)
    \end{itemize}
    \item Nach dem Umfang:
    \begin{itemize}
        \item \textbf{Vollkostenkalkulation}: Berücksichtigung aller Kosten (fixe und variable)
        \item \textbf{Teilkostenkalkulation}: Nur Berücksichtigung der variablen Kosten
    \end{itemize}
    \item Nach der Branche:
    \begin{itemize}
        \item \textbf{Industriekalkulation}: Für produzierende Unternehmen
        \item \textbf{Handelskalkulation}: Für Handelsunternehmen
        \item \textbf{Dienstleistungskalkulation}: Für Dienstleistungsunternehmen
    \end{itemize}
\end{itemize}
\end{concept}



\begin{formula}{Zuschlagskalkulation} aus BAB abgeleitet
\begin{itemize}
    \item \textbf{Materialgemeinkosten-Zuschlagssatz (MGK)}: 
    $$\frac{\text{Materialgemeinkosten}}{\text{Materialeinzelkosten}} \times 100\%$$
    \item \textbf{Fertigungsgemeinkosten-Zuschlagssatz (FGK)}: 
    $$\frac{\text{Fertigungsgemeinkosten}}{\text{Fertigungslöhne}} \times 100\%$$
    \item \textbf{Verwaltungs- und Vertriebsgemeinkosten-Zuschlagssatz (VVGK)}: 
    $$\frac{\text{Verwaltungs- und Vertriebsgemeinkosten}}{\text{Herstellkosten}} \times 100\%$$
\end{itemize}
\end{formula}

\subsubsection{Kalkulation im Industriebetrieb}

\begin{KR}{Kalkulationsschema im Industriebetrieb} 
    
Zuschlagskalkulation:
\begin{itemize}
    \item Einzelmaterial
    \item + Materialgemeinkosten (MGK)
    \item = Materialkosten
    \item + Einzellöhne
    \item + Fertigungsgemeinkosten (FGK)
    \item = Fertigungskosten
    \item = Herstellkosten (Materialkosten + Fertigungskosten)
    \item + Verwaltungs- und Vertriebsgemeinkosten (VVGK)
    \item = Selbstkosten
    \item + Gewinnzuschlag
    \item = Nettoverkaufspreis (exkl. MwSt.)
\end{itemize}

Bei Bedarf kann Nettoverkaufspreis um weitere Elemente erweitert werden:
\begin{itemize}
    \item + Verkaufssonderkosten
    \item = Nettobarverkaufspreis
    \item + Skonto
    \item = Nettokreditverkaufspreis
    \item + Rabatt
    \item = Bruttokreditverkaufspreis (= Offerten-Preis)
    \item + MwSt.
    \item = Bruttokreditverkaufspreis (inkl. MwSt.)
\end{itemize}
\end{KR}

\subsubsection{Kalkulation im Handelsbetrieb}

\begin{KR}{Handelskalkulation}
    Verfahren zur Preisbestimmung im Handelsunternehmen. Im Gegensatz zum Industriebetrieb findet hier keine Produktion statt, sondern es werden fertige Waren eingekauft und weiterverkauft. Die Handelskalkulation besteht aus drei Teilen:
\begin{itemize}
    \item \textbf{Einkaufskalkulation}: Berechnung des Einstandspreises
    \item \textbf{Betriebsinterne Kalkulation}: Berechnung des Nettoverkaufspreises
    \item \textbf{Verkaufskalkulation}: Berechnung des Bruttoverkaufspreises
\end{itemize}

Diese drei Teile zusammen werden als Gesamtkalkulation bezeichnet.
\end{KR}

\begin{definition}{Einkaufskalkulation}
    Ermittlung des Einstandspreises (Wareneinkaufspreis). Das Schema sieht wie folgt aus:
\begin{itemize}
    \item Listeneinkaufspreis (Katalogpreis des Lieferanten)
    \item - Lieferantenrabatt
    \item = Zieleinkaufspreis
    \item - Lieferantenskonto
    \item = Bareinkaufspreis
    \item + Bezugskosten (Fracht, Zoll, Versicherung)
    \item = Einstandspreis (Bezugspreis)
\end{itemize}
\end{definition}

\begin{definition}{Betriebsinterne Kalkulation}
    Ermittlung des Nettoverkaufspreises. Das Schema sieht wie folgt aus:
\begin{itemize}
    \item Einstandspreis
    \item + Handlungskosten (Personal, Miete, Verwaltung, etc.)
    \item + Gewinnzuschlag
    \item = Nettoverkaufspreis (Nettoerlös)
\end{itemize}
\end{definition}

\begin{definition}{Verkaufskalkulation}
     Ermittlung des Bruttoverkaufspreises (Ladenpreis). Das Schema sieht wie folgt aus:
\begin{itemize}
    \item Nettoverkaufspreis
    \item + Kundenskonto
    \item = Zielverkaufspreis
    \item + Kundenrabatt
    \item = Bruttoverkaufspreis (Ladenpreis, exkl. MwSt.)
    \item + MwSt.
    \item = Bruttoverkaufspreis (inkl. MwSt.)
\end{itemize}
\end{definition}
