\section{Finanzen II: Cashflow und Kennzahlen}

\subsection{Geldflussrechnung (Cashflow-Rechnung)}

\begin{definition}{Geldflussrechnung}\\
Die Geldflussrechnung (auch Kapitalflussrechnung oder Cashflow-Rechnung) vermittelt den Anspruchsgruppen ein Bild über die Fähigkeit eines Unternehmens, Zahlungsmittel zu erwirtschaften, und gibt Auskunft über den Zahlungsmittelbedarf eines Unternehmens. Sie ergänzt die Bilanz und Erfolgsrechnung und zeigt die realen Geldflüsse einer Periode.
\end{definition}

\begin{definition}{Bestandteile der Geldflussrechnung}\\
Die Geldflussrechnung setzt sich aus folgenden Positionen zusammen:
\begin{itemize}
    \item \textbf{Geldfluss (Cashflow) aus Betriebstätigkeit (Operating Cashflow, OCF)}: Zahlungsmittelzu- und -abflüsse aus der laufenden Geschäftstätigkeit (z.B. Zahlungen von Kunden, Zahlungen an Lieferanten und Mitarbeiter)
    \item \textbf{Geldfluss (Cashflow) aus Investitionstätigkeit (Investing Cashflow, ICF)}: Zahlungsmittelzu- und -abflüsse aus der Investitionstätigkeit (z.B. Kauf/Verkauf von Sachanlagen, Finanzanlagen)
    \item \textbf{Geldfluss (Cashflow) aus Finanzierungstätigkeit (Financing Cashflow, FCF)}: Zahlungsmittelzu- und -abflüsse aus der Finanzierungstätigkeit (z.B. Aufnahme/Rückzahlung von Krediten, Dividendenzahlungen)
\end{itemize}
\end{definition}

\begin{concept}{Cashflow-Schemata}\\
Verschiedene typische Cashflow-Schemata können die finanzielle Situation eines Unternehmens charakterisieren:
\begin{itemize}
    \item \textbf{Normalfall (gesunde Unternehmung)}: 
    \begin{itemize}
        \item OCF positiv: Das Unternehmen erwirtschaftet Geld aus seiner operativen Tätigkeit
        \item ICF negativ: Das Unternehmen investiert
        \item FCF negativ: Das Unternehmen zahlt Schulden zurück oder schüttet Dividenden aus
    \end{itemize}
    \item \textbf{Expandierende Firma (Wachstumsstrategie)}:
    \begin{itemize}
        \item OCF positiv: Das Unternehmen erwirtschaftet Geld aus seiner operativen Tätigkeit
        \item ICF stark negativ: Das Unternehmen tätigt umfangreiche Investitionen
        \item FCF positiv: Das Unternehmen nimmt zusätzliche Finanzierungsmittel auf, da der operative Cashflow für die Investitionen nicht ausreicht
    \end{itemize}
    \item \textbf{Erfolgreiche Firma mit wenig Investitionsmöglichkeiten}:
    \begin{itemize}
        \item OCF stark positiv: Das Unternehmen erwirtschaftet viel Geld aus seiner operativen Tätigkeit
        \item ICF leicht negativ oder neutral: Das Unternehmen tätigt nur Ersatzinvestitionen
        \item FCF stark negativ: Das Unternehmen zahlt hohe Dividenden und/oder kauft eigene Aktien zurück
    \end{itemize}
    \item \textbf{Startup oder Firma mit existenziellen Problemen}:
    \begin{itemize}
        \item OCF negativ: Das Unternehmen verbrennt Geld in seiner operativen Tätigkeit (Cash Loss)
        \item ICF negativ: Das Unternehmen muss trotzdem investieren
        \item FCF positiv: Das Unternehmen braucht zusätzliche Finanzierungsmittel, um zu überleben
    \end{itemize}
\end{itemize}
\end{concept}

\subsection{Unternehmensfinanzierung}

\begin{definition}{Hauptformen der Unternehmensfinanzierung}\\
Es gibt verschiedene Formen der Unternehmensfinanzierung:
\begin{itemize}
    \item \textbf{Innenfinanzierung}: Finanzierung aus dem Unternehmen selbst
    \begin{itemize}
        \item Selbstfinanzierung durch einbehaltene Gewinne
        \item Finanzierung aus Abschreibungen
        \item Finanzierung durch Umschichtung von Vermögensteilen
    \end{itemize}
    \item \textbf{Aussenfinanzierung}: Finanzierung von ausserhalb des Unternehmens
    \begin{itemize}
        \item Eigenfinanzierung (z.B. Kapitalerhöhung, Aufnahme neuer Gesellschafter)
        \item Fremdfinanzierung (z.B. Bankkredit, Anleihen, Lieferantenkredit)
    \end{itemize}
\end{itemize}
\end{definition}

\subsection{Finanzkennzahlen}

\begin{concept}{Rolle des CFO und Finanzziele}\\
Der Chief Financial Officer (CFO) ist für das Management des Finanzdreiecks verantwortlich:
\begin{itemize}
    \item \textbf{Liquidität}: Sicherstellung der Zahlungsfähigkeit
    \item \textbf{Rentabilität}: Erzielen einer angemessenen Rendite
    \item \textbf{Sicherheit}: Gewährleistung einer stabilen Finanzstruktur
\end{itemize}

Diese drei Bereiche stehen in einem Zielkonflikt zueinander. Der CFO ist dafür verantwortlich, die drei Bereiche in Einklang mit der Unternehmensstrategie zu bringen und so zu gestalten, dass die unternehmerischen Ziele erreicht werden können.
\end{concept}

\begin{definition}{Liquiditätskennzahlen}\\
Liquiditätskennzahlen zeigen die Fähigkeit eines Unternehmens, seinen kurzfristigen Zahlungsverpflichtungen nachzukommen:
\begin{itemize}
    \item \textbf{Liquiditätsgrad I (Cash Ratio)}: $\frac{\text{Flüssige Mittel}}{\text{Kurzfristiges Fremdkapital}} \times 100\%$
    \begin{itemize}
        \item Richtwert: $\geq 20-30\%$
    \end{itemize}
    \item \textbf{Liquiditätsgrad II (Quick Ratio)}: $\frac{\text{Flüssige Mittel + Forderungen}}{\text{Kurzfristiges Fremdkapital}} \times 100\%$
    \begin{itemize}
        \item Richtwert: $\geq 100-120\%$
    \end{itemize}
    \item \textbf{Liquiditätsgrad III (Current Ratio)}: $\frac{\text{Umlaufvermögen}}{\text{Kurzfristiges Fremdkapital}} \times 100\%$
    \begin{itemize}
        \item Richtwert: $\geq 150-200\%$
    \end{itemize}
\end{itemize}
\end{definition}

\begin{definition}{Sicherheitskennzahlen}\\
Sicherheitskennzahlen geben Auskunft über die Kapitalstruktur und die finanzielle Stabilität eines Unternehmens:
\begin{itemize}
    \item \textbf{Eigenkapitalquote}: $\frac{\text{Eigenkapital}}{\text{Gesamtkapital}} \times 100\%$
    \begin{itemize}
        \item Richtwert: $\geq 30\%$ (abhängig von Branche)
    \end{itemize}
    \item \textbf{Fremdkapitalquote}: $\frac{\text{Fremdkapital}}{\text{Gesamtkapital}} \times 100\%$
    \begin{itemize}
        \item Richtwert: $\leq 70\%$ (abhängig von Branche)
    \end{itemize}
    \item \textbf{Verschuldungsgrad}: $\frac{\text{Fremdkapital}}{\text{Eigenkapital}} \times 100\%$
    \begin{itemize}
        \item Richtwert: $\leq 200\%$ (abhängig von Branche)
    \end{itemize}
\end{itemize}
\end{definition}

\begin{definition}{Rentabilitätskennzahlen}\\
Rentabilitätskennzahlen messen die Profitabilität eines Unternehmens:
\begin{itemize}
    \item \textbf{Eigenkapitalrentabilität (Return on Equity, ROE)}: $\frac{\text{Jahresgewinn}}{\text{Eigenkapital}} \times 100\%$
    \begin{itemize}
        \item Richtwert: $> 8-10\%$ (abhängig von Branche und Risikoprämie)
    \end{itemize}
    \item \textbf{Gesamtkapitalrentabilität (Return on Assets, ROA)}: $\frac{\text{Jahresgewinn + Fremdkapitalzinsen}}{\text{Gesamtkapital}} \times 100\%$
    \begin{itemize}
        \item Richtwert: $> 6-8\%$ (abhängig von Branche)
    \end{itemize}
    \item \textbf{Umsatzrentabilität (Return on Sales, ROS)}: $\frac{\text{Jahresgewinn}}{\text{Umsatz}} \times 100\%$
    \begin{itemize}
        \item Richtwert: abhängig von Branche, typischerweise 2-5\% im Handel, 5-10\% in der Industrie
    \end{itemize}
\end{itemize}
\end{definition}

\begin{concept}{Leverage-Effekt}\\
Der Leverage-Effekt (Hebelwirkung) beschreibt die Auswirkung des Fremdkapitals auf die Eigenkapitalrentabilität:
\begin{itemize}
    \item Solange die Gesamtkapitalrentabilität höher ist als der Fremdkapitalzinssatz, erhöht Fremdkapital die Eigenkapitalrentabilität (positiver Leverage-Effekt).
    \item Wenn die Gesamtkapitalrentabilität niedriger ist als der Fremdkapitalzinssatz, senkt Fremdkapital die Eigenkapitalrentabilität (negativer Leverage-Effekt).
\end{itemize}

Formel: ROE = ROA + (ROA - i) $\times$ Verschuldungsgrad

Wobei:
\begin{itemize}
    \item ROE = Eigenkapitalrentabilität
    \item ROA = Gesamtkapitalrentabilität
    \item i = Fremdkapitalzinssatz
\end{itemize}
\end{concept}

\begin{KR}{Finanzanalyse durchführen}\\
\paragraph{Datengrundlage aufbereiten}
\begin{itemize}
    \item Bilanz, Erfolgsrechnung und Geldflussrechnung beschaffen
    \item Daten auf Vollständigkeit und Richtigkeit prüfen
    \item Bilanzpositionen ggf. bereinigen (z.B. stille Reserven auflösen)
    \item Zahlen in ein einheitliches Format bringen
\end{itemize}

\paragraph{Kennzahlen berechnen}
\begin{itemize}
    \item Liquiditätskennzahlen ermitteln (Liquiditätsgrade I, II und III)
    \item Sicherheitskennzahlen berechnen (Eigenkapitalquote, Fremdkapitalquote, Verschuldungsgrad)
    \item Rentabilitätskennzahlen bestimmen (ROE, ROA, ROS)
    \item Weitere branchenspezifische Kennzahlen berechnen
\end{itemize}

\paragraph{Analyse und Interpretation}
\begin{itemize}
    \item Kennzahlen mit Richtwerten vergleichen
    \item Zeitliche Entwicklung der Kennzahlen analysieren (Trend)
    \item Branchenvergleich durchführen (Benchmarking)
    \item Zusammenhänge zwischen den Kennzahlen erkennen
    \item Stärken und Schwächen identifizieren
\end{itemize}

\paragraph{Massnahmen ableiten}
\begin{itemize}
    \item Handlungsbedarf in den Bereichen Liquidität, Sicherheit und Rentabilität erkennen
    \item Konkrete Massnahmen zur Verbesserung der Finanzsituation formulieren
    \item Prioritäten setzen und Zeithorizont festlegen
    \item Verantwortlichkeiten für die Umsetzung definieren
\end{itemize}
\end{KR}

\begin{example}
Beispiel zur Berechnung und Interpretation von Finanzkennzahlen:

Ein Unternehmen weist folgende Bilanzpositionen auf (in CHF):
\begin{itemize}
    \item Flüssige Mittel: 100'000
    \item Forderungen: 150'000
    \item Vorräte: 200'000
    \item Kurzfristiges Fremdkapital: 300'000
    \item Eigenkapital: 500'000
    \item Gesamtkapital: 1'100'000
    \item Jahresgewinn: 55'000
    \item Fremdkapitalzinsen: 25'000
    \item Umsatz: 2'000'000
\end{itemize}

\textbf{Liquiditätskennzahlen:}
\begin{itemize}
    \item Liquiditätsgrad I: $\frac{100'000}{300'000} \times 100\% = 33,3\%$ (Richtwert erfüllt)
    \item Liquiditätsgrad II: $\frac{100'000 + 150'000}{300'000} \times 100\% = 83,3\%$ (unter Richtwert)
    \item Liquiditätsgrad III: $\frac{100'000 + 150'000 + 200'000}{300'000} \times 100\% = 150\%$ (Richtwert erfüllt)
\end{itemize}

\textbf{Sicherheitskennzahlen:}
\begin{itemize}
    \item Eigenkapitalquote: $\frac{500'000}{1'100'000} \times 100\% = 45,5\%$ (Richtwert erfüllt)
    \item Fremdkapitalquote: $\frac{600'000}{1'100'000} \times 100\% = 54,5\%$ (Richtwert erfüllt)
    \item Verschuldungsgrad: $\frac{600'000}{500'000} \times 100\% = 120\%$ (Richtwert erfüllt)
\end{itemize}

\textbf{Rentabilitätskennzahlen:}
\begin{itemize}
    \item Eigenkapitalrentabilität: $\frac{55'000}{500'000} \times 100\% = 11\%$ (Richtwert erfüllt)
    \item Gesamtkapitalrentabilität: $\frac{55'000 + 25'000}{1'100'000} \times 100\% = 7,3\%$ (Richtwert erfüllt)
    \item Umsatzrentabilität: $\frac{55'000}{2'000'000} \times 100\% = 2,75\%$ (branchenabhängig)
\end{itemize}

\textbf{Interpretation:} Das Unternehmen weist eine gute Eigenkapitalausstattung auf und erfüllt die meisten Richtwerte. Bei der Quick Ratio (Liquiditätsgrad II) besteht jedoch eine gewisse Liquiditätsanspannung, was auf ein erhöhtes Risiko bei kurzfristigen Zahlungsverpflichtungen hindeuten könnte. Die Rentabilitätskennzahlen sind solide, was auf eine gute Ertragskraft schließen lässt.
\end{example}

\raggedcolumns

\section{Kalkulation}

\subsection{Grundlagen der Kalkulation}

\begin{definition}{Kalkulation}\\
Die Kalkulation ist die Berechnung der Kosten und Preise von Produkten oder Dienstleistungen. Sie dient der Preisfindung und Preisbeurteilung und ist eine wichtige Grundlage für die Offertenerstellung. Die Hauptaufgaben der Kalkulation sind:
\begin{itemize}
    \item Ermittlung der Selbstkosten (Totalkosten)
    \item Preisfindung
    \item Preisbeurteilung
    \item Offertenerstellung
\end{itemize}
\end{definition}

\begin{concept}{Arten der Kalkulation}\\
Man unterscheidet verschiedene Arten der Kalkulation:
\begin{itemize}
    \item Nach dem Zeitpunkt:
    \begin{itemize}
        \item \textbf{Vorkalkulation}: Vor der Produktion (für Angebote)
        \item \textbf{Nachkalkulation}: Nach der Produktion (zur Kontrolle)
    \end{itemize}
    \item Nach dem Umfang:
    \begin{itemize}
        \item \textbf{Vollkostenkalkulation}: Berücksichtigung aller Kosten (fixe und variable)
        \item \textbf{Teilkostenkalkulation}: Nur Berücksichtigung der variablen Kosten
    \end{itemize}
    \item Nach der Branche:
    \begin{itemize}
        \item \textbf{Industriekalkulation}: Für produzierende Unternehmen
        \item \textbf{Handelskalkulation}: Für Handelsunternehmen
        \item \textbf{Dienstleistungskalkulation}: Für Dienstleistungsunternehmen
    \end{itemize}
\end{itemize}
\end{concept}

\subsection{Kalkulation im Industriebetrieb}

\begin{definition}{Zuschlagskalkulation}\\
Die Zuschlagskalkulation ist ein Verfahren der Vollkostenkalkulation, bei dem die Gemeinkosten mit Hilfe von Zuschlagssätzen auf die Kostenträger verrechnet werden. Die Zuschlagssätze werden aus dem Betriebsabrechnungsbogen (BAB) abgeleitet.

Die wichtigsten Zuschlagssätze sind:
\begin{itemize}
    \item \textbf{Materialgemeinkosten-Zuschlagssatz (MGK)}: \\
    $\frac{\text{Materialgemeinkosten}}{\text{Materialeinzelkosten}} \times 100\%$
    \item \textbf{Fertigungsgemeinkosten-Zuschlagssatz (FGK)}: \\
    $\frac{\text{Fertigungsgemeinkosten}}{\text{Fertigungslöhne}} \times 100\%$
    \item \textbf{Verwaltungs- und Vertriebsgemeinkosten-Zuschlagssatz (VVGK)}: \\
    $\frac{\text{Verwaltungs- und Vertriebsgemeinkosten}}{\text{Herstellkosten}} \times 100\%$
\end{itemize}
\end{definition}

\begin{definition}{Kalkulationsschema im Industriebetrieb}\\
Das Kalkulationsschema für die Zuschlagskalkulation im Industriebetrieb sieht wie folgt aus:
\begin{itemize}
    \item Einzelmaterial
    \item + Materialgemeinkosten (MGK)
    \item = Materialkosten
    \item + Einzellöhne
    \item + Fertigungsgemeinkosten (FGK)
    \item = Fertigungskosten
    \item = Herstellkosten (Materialkosten + Fertigungskosten)
    \item + Verwaltungs- und Vertriebsgemeinkosten (VVGK)
    \item = Selbstkosten
    \item + Gewinnzuschlag
    \item = Nettoverkaufspreis (exkl. MwSt.)
\end{itemize}

Bei Bedarf kann der Nettoverkaufspreis noch um weitere Elemente erweitert werden:
\begin{itemize}
    \item + Verkaufssonderkosten
    \item = Nettobarverkaufspreis
    \item + Skonto
    \item = Nettokreditverkaufspreis
    \item + Rabatt
    \item = Bruttokreditverkaufspreis (= Offerten-Preis)
    \item + MwSt.
    \item = Bruttokreditverkaufspreis (inkl. MwSt.)
\end{itemize}
\end{definition}

\begin{example}
Beispiel für eine Zuschlagskalkulation im Industriebetrieb:

\textbf{Ausgangsdaten aus dem BAB:}
\begin{itemize}
    \item Materialgemeinkosten: 30'000 CHF
    \item Materialeinzelkosten: 150'000 CHF
    \item Fertigungsgemeinkosten: 80'000 CHF
    \item Fertigungslöhne (Einzellöhne): 320'000 CHF
    \item Verwaltungs- und Vertriebsgemeinkosten: 165'000 CHF
    \item Herstellkosten: 580'000 CHF
\end{itemize}

\textbf{Berechnung der Zuschlagssätze:}
\begin{itemize}
    \item MGK-Zuschlagssatz: $\frac{30'000}{150'000} \times 100\% = 20\%$
    \item FGK-Zuschlagssatz: $\frac{80'000}{320'000} \times 100\% = 25\%$
    \item VVGK-Zuschlagssatz: $\frac{165'000}{580'000} \times 100\% = 28,4\%$
\end{itemize}

\textbf{Kalkulation eines Produkts:}
\begin{itemize}
    \item Einzelmaterial: 1,00 CHF
    \item + Materialgemeinkosten (20\%): 0,20 CHF
    \item = Materialkosten: 1,20 CHF
    \item + Einzellöhne: 2,00 CHF
    \item + Fertigungsgemeinkosten (25\%): 0,50 CHF
    \item = Fertigungskosten: 2,50 CHF
    \item = Herstellkosten: 3,70 CHF
    \item + Verwaltungs- und Vertriebsgemeinkosten (28,4\%): 1,05 CHF
    \item = Selbstkosten: 4,75 CHF
    \item + Gewinnzuschlag (10\%): 0,48 CHF
    \item = Nettoverkaufspreis: 5,23 CHF (exkl. MwSt.)
\end{itemize}
\end{example}

\subsection{Kalkulation im Handelsbetrieb}

\begin{definition}{Handelskalkulation}\\
Die Handelskalkulation ist ein Verfahren zur Preisbestimmung im Handelsunternehmen. Im Gegensatz zum Industriebetrieb findet hier keine Produktion statt, sondern es werden fertige Waren eingekauft und weiterverkauft. Die Handelskalkulation besteht aus drei Teilen:
\begin{itemize}
    \item \textbf{Einkaufskalkulation}: Berechnung des Einstandspreises
    \item \textbf{Betriebsinterne Kalkulation}: Berechnung des Nettoverkaufspreises
    \item \textbf{Verkaufskalkulation}: Berechnung des Bruttoverkaufspreises
\end{itemize}

Diese drei Teile zusammen werden als Gesamtkalkulation bezeichnet.
\end{definition}

\begin{definition}{Einkaufskalkulation}\\
Die Einkaufskalkulation im Handelsbetrieb dient der Ermittlung des Einstandspreises (Wareneinkaufspreis). Das Schema sieht wie folgt aus:
\begin{itemize}
    \item Listeneinkaufspreis (Katalogpreis des Lieferanten)
    \item - Lieferantenrabatt
    \item = Zieleinkaufspreis
    \item - Lieferantenskonto
    \item = Bareinkaufspreis
    \item + Bezugskosten (Fracht, Zoll, Versicherung)
    \item = Einstandspreis (Bezugspreis)
\end{itemize}
\end{definition}

\begin{definition}{Betriebsinterne Kalkulation}\\
Die betriebsinterne Kalkulation im Handelsbetrieb dient der Ermittlung des Nettoverkaufspreises. Das Schema sieht wie folgt aus:
\begin{itemize}
    \item Einstandspreis
    \item + Handlungskosten (Personal, Miete, Verwaltung, etc.)
    \item + Gewinnzuschlag
    \item = Nettoverkaufspreis (Nettoerlös)
\end{itemize}
\end{definition}

\begin{definition}{Verkaufskalkulation}\\
Die Verkaufskalkulation im Handelsbetrieb dient der Ermittlung des Bruttoverkaufspreises (Ladenpreis). Das Schema sieht wie folgt aus:
\begin{itemize}
    \item Nettoverkaufspreis
    \item + Kundenskonto
    \item = Zielverkaufspreis
    \item + Kundenrabatt
    \item = Bruttoverkaufspreis (Ladenpreis, exkl. MwSt.)
    \item + MwSt.
    \item = Bruttoverkaufspreis (inkl. MwSt.)
\end{itemize}
\end{definition}

\begin{example}
Beispiel für eine Handelskalkulation:

\textbf{Einkaufskalkulation:}
\begin{itemize}
    \item Listeneinkaufspreis: 10,00 CHF
    \item - Lieferantenrabatt (10\%): 1,00 CHF
    \item = Zieleinkaufspreis: 9,00 CHF
    \item - Lieferantenskonto (2\%): 0,18 CHF
    \item = Bareinkaufspreis: 8,82 CHF
    \item + Bezugskosten: 0,48 CHF
    \item = Einstandspreis: 9,30 CHF
\end{itemize}

\textbf{Betriebsinterne Kalkulation:}
\begin{itemize}
    \item Einstandspreis: 9,30 CHF
    \item + Handlungskosten (12\%): 1,12 CHF
    \item + Gewinnzuschlag (8\%): 0,83 CHF
    \item = Nettoverkaufspreis: 11,25 CHF
\end{itemize}

\textbf{Verkaufskalkulation:}
\begin{itemize}
    \item Nettoverkaufspreis: 11,25 CHF
    \item + Kundenskonto (2\%): 0,23 CHF
    \item = Zielverkaufspreis: 11,48 CHF
    \item + Kundenrabatt (5\%): 0,60 CHF
    \item = Bruttoverkaufspreis: 12,08 CHF (exkl. MwSt.)
    \item + MwSt. (7,7\%): 0,93 CHF
    \item = Bruttoverkaufspreis: 13,01 CHF (inkl. MwSt.)
\end{itemize}
\end{example}

\begin{KR}{Zuschlagskalkulation im Industriebetrieb durchführen}\\
\paragraph{Zuschlagssätze aus dem BAB ermitteln}
\begin{itemize}
    \item Materialgemeinkosten-Zuschlagssatz (MGK) berechnen
    \item Fertigungsgemeinkosten-Zuschlagssatz (FGK) berechnen
    \item Verwaltungs- und Vertriebsgemeinkosten-Zuschlagssatz (VVGK) berechnen
\end{itemize}

\paragraph{Produktkalkulation vornehmen}
\begin{itemize}
    \item Einzelmaterial des Produkts bestimmen
    \item Materialgemeinkosten mit MGK-Zuschlagssatz berechnen
    \item Materialkosten ermitteln (Einzelmaterial + Materialgemeinkosten)
    \item Einzellöhne des Produkts bestimmen
    \item Fertigungsgemeinkosten mit FGK-Zuschlagssatz berechnen
    \item Fertigungskosten ermitteln (Einzellöhne + Fertigungsgemeinkosten)
    \item Herstellkosten berechnen (Materialkosten + Fertigungskosten)
    \item Verwaltungs- und Vertriebsgemeinkosten mit VVGK-Zuschlagssatz berechnen
    \item Selbstkosten ermitteln (Herstellkosten + Verwaltungs- und Vertriebsgemeinkosten)
\end{itemize}

\paragraph{Verkaufspreis festlegen}
\begin{itemize}
    \item Gewinnzuschlag bestimmen
    \item Nettoverkaufspreis berechnen (Selbstkosten + Gewinnzuschlag)
    \item Bei Bedarf: Verkaufssonderkosten, Skonto, Rabatt berücksichtigen
    \item MwSt. hinzurechnen
\end{itemize}
\end{KR}

\begin{KR}{Handelskalkulation durchführen}\\
\paragraph{Einkaufskalkulation erstellen}
\begin{itemize}
    \item Listeneinkaufspreis (Katalogpreis des Lieferanten) ermitteln
    \item Lieferantenrabatt abziehen
    \item Zieleinkaufspreis berechnen
    \item Lieferantenskonto abziehen
    \item Bareinkaufspreis berechnen
    \item Bezugskosten hinzurechnen
    \item Einstandspreis ermitteln
\end{itemize}

\paragraph{Betriebsinterne Kalkulation vornehmen}
\begin{itemize}
    \item Handlungskosten berechnen
    \item Gewinnzuschlag bestimmen
    \item Nettoverkaufspreis ermitteln
\end{itemize}

\paragraph{Verkaufskalkulation erstellen}
\begin{itemize}
    \item Kundenskonto hinzurechnen
    \item Zielverkaufspreis berechnen
    \item Kundenrabatt hinzurechnen
    \item Bruttoverkaufspreis (exkl. MwSt.) ermitteln
    \item MwSt. hinzurechnen
    \item Bruttoverkaufspreis (inkl. MwSt.) berechnen
\end{itemize}
\end{KR}