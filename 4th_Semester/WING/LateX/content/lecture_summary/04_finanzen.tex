\small



\begin{definition}{Aufgaben der Kundenbindung}
\begin{itemize}
    \item \textbf{Systematische Betreuung}: Regelmäßige Kontaktpflege und Beratung
    \item \textbf{Beschwerdemanagement}: Professioneller Umgang mit Kundenbeschwerden
    \item \textbf{After-Sales-Services}: Kundenbetreuung nach dem Kauf (Wartung, Support)
    \item \textbf{Cross-Selling}: Angebot ergänzender Produkte/Dienstleistungen
    \item \textbf{Up-Selling}: Angebot höherwertiger Produkte/Dienstleistungen
    \item \textbf{Kundenclubs und -karten}: Schaffung exklusiver Kundenvorteile
\end{itemize}
\end{definition}

\begin{definition}{Instrumente des CRM}
Customer Relationship Managements 
\begin{itemize}
    \item \textbf{Kundendatenbank}: Systematische Erfassung und Analyse von Kundendaten
    \item \textbf{Kundensegmentierung}: Einteilung von Kunden nach relevanten Kriterien
    \item \textbf{Kundenlebenszyklus-Management}: Anpassung der Maßnahmen
    \item \textbf{Beschwerdemanagement}: Systematische Bearbeitung Kundenreklamationen
    \item \textbf{Kundenzufriedenheits-Messung}: Regelmäßige Erhebung Kundenzufriedenheit
    \item \textbf{Loyalitätsprogramme}: Belohnung treue Kund. (Bonusprogramme, Rabatte)
    \item \textbf{Personalisierte Kommunikation}: Individuell angepasste Kundenansprache
\end{itemize}
\end{definition}

\begin{concept}{Kundenpotenzial ausbauen} Ziel des CRM
\begin{itemize}
    \item \textbf{Ausbau der Beziehungsintensität}: Steigerung der Kauffrequenz
    \item \textbf{Ausbau des Share of Wallet}: Erhöhung des Anteils am Kundenbudget
    \item \textbf{Cross-Selling-Potenzial}: Verkauf zusätzlicher Produkte/Dienstleistungen
    \item \textbf{Up-Selling-Potenzial}: Verkauf höherwertiger Produkte/Dienstleistungen
    \item \textbf{Reduktion der Abwanderungsrate}: Senkung der Kundenabwanderung
    \item \textbf{Weiterempfehlungen}: Gewinnung von Neukunden durch Bestandskunden
\end{itemize}
\end{concept}



\begin{KR}{Entwicklung einer CRM-Strategie}

\begin{minipage}{0.5\linewidth}
\textbf{Ausgangssituation analysieren}
\begin{itemize}
    \item Bestandskundendaten auswerten
    \item Kundensegmente identifizieren
    \item Kundenlebenszyklus analysieren
    \item Customer Journey erfassen
\end{itemize}

\textbf{Ziele definieren}
\begin{itemize}
    \item Kundengewinnungsziele festlegen
    \item Kundenbindungsziele definieren
    \item KPIs für CRM-Massnahmen
    \item Wirtschaftlichkeitsziele setzen
\end{itemize}
\end{minipage}
\begin{minipage}{0.5\linewidth}
\textbf{Ressourcen planen}
\begin{itemize}
    \item CRM-System auswählen/anpassen
    \item Personalbedarf ermitteln
    \item Budget festlegen
    \item Prozesse definieren
\end{itemize}

\textbf{Umsetzung steuern}
\begin{itemize}
    \item Massnahmen implementieren
    \item Mitarbeiter schulen
    \item Controlling-System einrichten
    \item Kontinuierliche Optimierung sicherstellen
\end{itemize}
\end{minipage}

\textbf{Massnahmen entwickeln}: Massnahmen zur Kundengewinnung und Kundenbindung, Kommunikationskonzept, Personalisierungsstrategie
\end{KR}



\section{Finanzen I: Bilanz und Erfolgsrechnung}

\begin{definition}{Rechnungswesen}
    Systematische Erfassung, Überwachung und informatorische Auswertung aller quantifizierbaren Vorgänge eines Unternehmens.
\begin{itemize}
    \item \textbf{Externes Rechnungswesen (Finanzbuchhaltung)}: Für externe Adressaten wie Investoren, Banken, Staat
    \item \textbf{Internes Rechnungswesen (Betriebsbuchhaltung)}: Für interne Steuerung und Entscheidungsfindung
\end{itemize}
\end{definition}

\begin{concept}{Finanzbuchhaltung (FIBU)}
    Dokumentation aller Geschäftsvorfälle eines Unternehmens und bildet die Grundlage für den Jahresabschluss.
\begin{itemize}
    \item \textbf{Bilanz}: Überblick über Vermögen und Kapital zu einem bestimmten Stichtag
    \item \textbf{Erfolgsrechnung (GuV)}: Darstellung von Aufwand und Ertrag während einer Periode
    \item \textbf{Geldflussrechnung (Kapitalflussrechnung)}: Darstellung der Veränderung der liquiden Mittel während einer Periode
    \item \textbf{Anhang}: Ergänzende Informationen zum Jahresabschluss
\end{itemize}
Hauptzweck: Gesetzliche Anforderungen erfüllen, Gewinn/Verlust ermitteln, Basis für Besteuerung
\end{concept}

\begin{concept}{Betriebsbuchhaltung (BEBU)}
    (auch Kostenrechnung) dient der internen Steuerung und Kontrolle.
\begin{itemize}
    \item \textbf{Kostenartenrechnung}: Welche Kosten in welcher Höhe sind angefallen?
    \item \textbf{Kostenstellenrechnung}: Wo sind Kosten angefallen?
    \item \textbf{Kostenträgerrechnung}: Wofür sind Kosten angefallen?
    \item \textbf{Erfolgsrechnung}: Welche Erträge und Aufwendungen sind entstanden?
\end{itemize}
Hauptzweck: Interne Steuerung, Kalkulation, Preisgestaltung, Entscheidungsgrundlage für Management
\end{concept}

\subsubsection{Bilanz}

\begin{definition}{Bilanz}
    Gegenüberstellung von Aktiven (Vermögen/Mittelverwendung) und Passiven (Schulden/Mittelherkunft) zum Bilanzstichtag. Sie ist eine Momentaufnahme der Vermögens- und Kapitallage des Unternehmens.

Für die Bilanz gilt stets: Summe Aktiven = Summe Passiven
\end{definition}

\begin{definition}{Aktiven (Vermögen)}
Die Aktivseite der Bilanz zeigt, worin das Vermögen des Unternehmens gebunden ist. Sie wird nach dem Liquiditätsprinzip (Verfügbarkeit) gegliedert:
\begin{itemize}
    \item \textbf{Umlaufvermögen}: Vermögenswerte, die kurzfristig (innerhalb eines Jahres) umgeschlagen oder verbraucht werden
    \begin{itemize}
        \item Flüssige Mittel (Kasse, Bank, Post)
        \item Wertschriften (als Liquiditätsreserve)
        \item Forderungen aus Lieferungen und Leistungen (Debitoren)
        \item Vorräte (Waren, Rohstoffe, Fertigprodukte)
        \item Aktive Rechnungsabgrenzungen
    \end{itemize}
    \item \textbf{Anlagevermögen}: Vermögenswerte, die langfristig (über ein Jahr hinaus) im Unternehmen verbleiben
    \begin{itemize}
        \item Finanzanlagen (langfristige Wertschriften, Beteiligungen)
        \item Sachanlagen (Grundstücke, Gebäude, Maschinen, Einrichtungen)
        \item Immaterielle Anlagen (Patente, Lizenzen, Goodwill)
    \end{itemize}
\end{itemize}
\end{definition}

\begin{definition}{Passiven (Kapital)}
Die Passivseite der Bilanz zeigt, wie das Vermögen des Unternehmens finanziert ist. Sie wird nach dem Fälligkeitsprinzip (Fristigkeit) gegliedert:

\textbf{Fremdkapital}: Kapital, das Unternehmen von Dritten zur Verfügung gestellt wird
    \begin{itemize}
        \item Kurzfristiges Fremdkapital (Fälligkeit bis ein Jahr)
        \begin{itemize}
            \item Verbindlichkeiten aus Lieferungen und Leistungen (Kreditoren)
            \item Kurzfristige Finanzverbindlichkeiten (Bankkredite)
            \item Kundenanzahlungen
            \item Passive Rechnungsabgrenzungen
        \end{itemize}
        \item Langfristiges Fremdkapital (Fälligkeit über ein Jahr)
        \begin{itemize}
            \item Langfristige Finanzverbindlichkeiten (Darlehen, Anleihen)
            \item Hypotheken
            \item Rückstellungen
        \end{itemize}
    \end{itemize}
\textbf{Eigenkapital}: Vom Eigentümer eingebrachtes Kapital und thesaurierte Gewinne
    \begin{itemize}
        \item Grund-/Aktienkapital
        \item Reserven (gesetzliche und freie Reserven)
        \item Gewinnvortrag / Bilanzverlust
        \item Jahresgewinn / Jahresverlust
    \end{itemize}

Für das Eigenkapital gilt: Eigenkapital = Summe Aktiven - Fremdkapital
\end{definition}

\begin{concept}{Funktion der Bilanz}
\begin{itemize}
    \item \textbf{Dokumentationsfunktion}: Darstellung der finanziellen Lage des Unternehmens
    \item \textbf{Informationsfunktion}: Information der internen und externen Adressaten
    \item \textbf{Rechenschaftsfunktion}: Rechenschaftslegung gegenüber Eigentümern und Gläubigern
    \item \textbf{Grundlage für Erfolgsermittlung}: Basis für die Berechnung des periodischen Erfolgs
    \item \textbf{Grundlage für Finanzanalyse}: Basis für Berechnung von Finanzkennzahlen
\end{itemize}
\end{concept}

\subsubsection{Erfolgsrechnung}

\begin{definition}{Erfolgsrechnung}
 (auch Gewinn- und Verlustrechnung, GuV) stellt den während einer Periode angefallenen Aufwand eines Unternehmens den Erträgen gegenüber. Daraus resultiert der Gewinn oder Verlust einer bestimmten Periode.

Grundformel: Ertrag - Aufwand = Gewinn/Verlust
\end{definition}

\begin{definition}{Aufwand und Ertrag}

    \textbf{Aufwand}: Wertmässiger Verbrauch von Gütern und Dienstleistungen in einer Periode
    \begin{itemize}
        \item Betriebsaufwand (z.B. Materialaufwand, Personalaufwand, Abschreibungen)
        \item Betriebsfremder Aufwand (z.B. Finanzaufwand)
        \item Ausserordentlicher Aufwand
    \end{itemize}
\textbf{Ertrag}: Wertmässige Zunahme durch betriebliche Leistungen in einer Periode
    \begin{itemize}
        \item Betriebsertrag (z.B. Umsatzerlöse, Eigenleistungen)
        \item Betriebsfremder Ertrag (z.B. Finanzertrag)
        \item Ausserordentlicher Ertrag
    \end{itemize}
\end{definition}

\begin{definition}{Gewinngrössen}
In der Erfolgsrechnung:
\begin{itemize}
    \item \textbf{Bruttogewinn (Gross Profit)}: Umsatzerlöse - Warenaufwand/Herstellkosten
    \item \textbf{EBITDA} (Earnings Before Interest, Taxes, Depreciation and Amortization): Gewinn vor Zinsen, Steuern, Abschreibungen und Amortisationen
    \item \textbf{EBIT} (Earnings Before Interest and Taxes): Gewinn vor Zinsen und Steuern
    \item \textbf{EBT} (Earnings Before Taxes): Gewinn vor Steuern
    \item \textbf{Reingewinn/Nettogewinn/EAT} (Earnings After Taxes): Gewinn nach Steuern
\end{itemize}
\end{definition}

\begin{KR}{Bilanz erstellen}\\
\textbf{Vermögenswerte erfassen}
\begin{itemize}
    \item Flüssige Mittel (Kasse, Bank, Post) identifizieren
    \item Forderungen aus Lieferungen und Leistungen ermitteln
    \item Warenvorräte und andere Vorräte bewerten
    \item Sachanlagen (Maschinen, Fahrzeuge, Gebäude) erfassen
    \item Finanzanlagen und immaterielle Anlagen einbeziehen
\end{itemize}

\textbf{Kapitalstruktur ermitteln}
\begin{itemize}
    \item Kurzfristige Verbindlichkeiten (Kreditoren, kurzfristige Bankschulden) erfassen
    \item Langfristige Verbindlichkeiten (Darlehen, Hypotheken) identifizieren
    \item Eigenkapital (Grund-/Aktienkapital, Reserven) bestimmen
    \item Jahresgewinn/-verlust übertragen
\end{itemize}

\textbf{Bilanz strukturieren}
\begin{itemize}
    \item Aktivseite nach Liquiditätsprinzip gliedern (Umlauf- vor Anlagevermögen)
    \item Passivseite nach Fälligkeitsprinzip gliedern (kurz-/langfristiges FK, EK)
    \item Bilanzsumme berechnen und Bilanzgleichung prüfen (Aktiven = Passiven)
\end{itemize}
\end{KR}

\begin{KR}{Erfolgsrechnung erstellen}\\
\textbf{Erträge erfassen}
\begin{itemize}
    \item Umsatzerlöse aus Verkäufen von Produkten/Dienstleistungen ermitteln
    \item Andere betriebliche Erträge erfassen
    \item Finanzerträge (Zinsen, Dividenden) berücksichtigen
    \item Ausserordentliche Erträge einbeziehen
\end{itemize}

\textbf{Aufwendungen ermitteln}
\begin{itemize}
    \item Materialaufwand/Warenaufwand berechnen
    \item Personalaufwand erfassen
    \item Sonstigen betrieblichen Aufwand berücksichtigen
    \item Abschreibungen einbeziehen
    \item Finanzaufwand (Zinsen) berücksichtigen
    \item Steuern berechnen
\end{itemize}

\textbf{Erfolgsrechnung strukturieren}
\begin{itemize}
    \item Bruttogewinn berechnen (Umsatzerlöse - Materialaufwand)
    \item EBITDA ermitteln \\ (Bruttogewinn - Personalaufwand - sonstiger betrieblicher Aufwand)
    \item EBIT berechnen (EBITDA - Abschreibungen)
    \item EBT ermitteln (EBIT +/- Finanzergebnis)
    \item Reingewinn/Nettogewinn berechnen (EBT - Steuern)
\end{itemize}
\end{KR}

\raggedcolumns

\section{Finanzen II: Cashflow und Kennzahlen}

\subsubsection{Geldflussrechnung (Cashflow-Rechnung)}

\begin{definition}{Geldflussrechnung}
    (auch Kapitalflussrechnung oder Cashflow-Rechnung) vermittelt den Anspruchsgruppen ein Bild über die Fähigkeit eines Unternehmens, Zahlungsmittel zu erwirtschaften, und gibt Auskunft über den Zahlungsmittelbedarf eines Unternehmens. Sie ergänzt die Bilanz und Erfolgsrechnung und zeigt die realen Geldflüsse einer Periode.
\end{definition}

\begin{definition}{Bestandteile der Geldflussrechnung}
\begin{itemize}
    \item \textbf{Geldfluss (Cashflow) aus Betriebstätigkeit (Operating Cashflow, OCF)}: Zahlungsmittelzu- und -abflüsse aus der laufenden Geschäftstätigkeit (z.B. Zahlungen von Kunden, Zahlungen an Lieferanten und Mitarbeiter)
    \item \textbf{Geldfluss (Cashflow) aus Investitionstätigkeit (Investing Cashflow, ICF)}: Zahlungsmittelzu- und -abflüsse aus der Investitionstätigkeit (z.B. Kauf/Verkauf von Sachanlagen, Finanzanlagen)
    \item \textbf{Geldfluss (Cashflow) aus Finanzierungstätigkeit (Financing Cashflow, FCF)}: Zahlungsmittelzu- und -abflüsse aus der Finanzierungstätigkeit (z.B. Aufnahme/Rückzahlung von Krediten, Dividendenzahlungen)
\end{itemize}
\end{definition}

\begin{concept}{Cashflow-Schemata} charakterisieren finanzielle Situation Unternehmen:

\textbf{Normalfall (gesunde Unternehmung)}: 
    \begin{itemize}
        \item OCF positiv: Das Unternehmen erwirtschaftet Geld aus operativen Tätigkeit
        \item ICF negativ: Das Unternehmen investiert
        \item FCF negativ: zahlt Schulden zurück oder schüttet Dividenden aus
    \end{itemize}
\textbf{Expandierende Firma (Wachstumsstrategie)}:
    \begin{itemize}
        \item OCF positiv: Das Unternehmen erwirtschaftet Geld aus operativen Tätigkeit
        \item ICF stark negativ: Das Unternehmen tätigt umfangreiche Investitionen
        \item FCF positiv: Das Unternehmen nimmt zusätzliche Finanzierungsmittel auf, da der operative Cashflow für die Investitionen nicht ausreicht
    \end{itemize}
\textbf{Erfolgreiche Firma mit wenig Investitionsmöglichkeiten}:
    \begin{itemize}
        \item OCF stark positiv: erwirtschaftet viel Geld aus operativen Tätigkeit
        \item ICF leicht negativ oder neutral: tätigt nur Ersatzinvestitionen
        \item FCF stark negativ: zahlt hohe Dividenden und/oder kauft eigene Aktien zurück
    \end{itemize}
\textbf{Startup oder Firma mit existenziellen Problemen}:
    \begin{itemize}
        \item OCF negativ: verbrennt Geld in operativen Tätigkeit (Cash Loss)
        \item ICF negativ: Das Unternehmen muss trotzdem investieren
        \item FCF positiv: braucht zusätzliche Finanzierungsmittel, um zu überleben
    \end{itemize}
\end{concept}

\subsubsection{Unternehmensfinanzierung und Finanzkennzahlen}

\begin{definition}{Hauptformen der Unternehmensfinanzierung}\\
    \textbf{Innenfinanzierung}: Finanzierung aus dem Unternehmen selbst
    \begin{itemize}
        \item Selbstfinanzierung durch einbehaltene Gewinne
        \item Finanzierung aus Abschreibungen
        \item Finanzierung durch Umschichtung von Vermögensteilen
    \end{itemize}
\textbf{Aussenfinanzierung}: Finanzierung von ausserhalb des Unternehmens
    \begin{itemize}
        \item Eigenfinanzierung (z.B. Kapitalerhöhung, Aufnahme neuer Gesellschafter)
        \item Fremdfinanzierung (z.B. Bankkredit, Anleihen, Lieferantenkredit)
    \end{itemize}
\end{definition}



\begin{concept}{Rolle des CFO und Finanzziele}
Der Chief Financial Officer (CFO) ist für das Management des Finanzdreiecks verantwortlich:
\begin{itemize}
    \item \textbf{Liquidität}: Sicherstellung der Zahlungsfähigkeit
    \item \textbf{Rentabilität}: Erzielen einer angemessenen Rendite
    \item \textbf{Sicherheit}: Gewährleistung einer stabilen Finanzstruktur
\end{itemize}

Diese drei Bereiche stehen in einem Zielkonflikt zueinander. Der CFO ist dafür verantwortlich, die drei Bereiche in Einklang mit der Unternehmensstrategie zu bringen und so zu gestalten, dass die unternehmerischen Ziele erreicht werden können.
\end{concept}

\begin{definition}{Liquiditätskennzahlen}
     zeigen die Fähigkeit eines Unternehmens, seinen kurzfristigen Zahlungsverpflichtungen nachzukommen:
\begin{itemize}
    \item \textbf{Liquiditätsgrad I (Cash Ratio)}: $\frac{\text{Flüssige Mittel}}{\text{Kurzfristiges Fremdkapital}} \times 100\%$ \\Richtwert: $\geq 20-30\%$
    \item \textbf{Liquiditätsgrad II (Quick Ratio)}: $\frac{\text{Flüssige Mittel + Forderungen}}{\text{Kurzfristiges Fremdkapital}} \times 100\%$ \\ Richtwert: $\geq 100-120\%$
    \item \textbf{Liquiditätsgrad III (Current Ratio)}: $\frac{\text{Umlaufvermögen}}{\text{Kurzfristiges Fremdkapital}} \times 100\%$ \\ Richtwert: $\geq 150-200\%$
\end{itemize}
\end{definition}

\begin{definition}{Sicherheitskennzahlen}
     geben Auskunft über die Kapitalstruktur und die finanzielle Stabilität eines Unternehmens: (Richtwerte abhängig von Branche)
\begin{itemize}
    \item \textbf{Eigenkapitalquote}: $\frac{\text{Eigenkapital}}{\text{Gesamtkapital}} \times 100\%$ (Richtwert: $\geq 30\%$)
    \item \textbf{Fremdkapitalquote}: $\frac{\text{Fremdkapital}}{\text{Gesamtkapital}} \times 100\%$ (Richtwert: $\leq 70\%$)
    \item \textbf{Verschuldungsgrad}: $\frac{\text{Fremdkapital}}{\text{Eigenkapital}} \times 100\%$ (Richtwert: $\leq 200\%$)
\end{itemize}
\end{definition}

\begin{definition}{Rentabilitätskennzahlen} messen die Profitabilität eines Unternehmens:
\begin{itemize}
    \item \textbf{Eigenkapitalrentabilität (Return on Equity, ROE)}: $\frac{\text{Jahresgewinn}}{\text{Eigenkapital}} \times 100\%$
    \begin{itemize}
        \item Richtwert: $> 8-10\%$ (abhängig von Branche und Risikoprämie)
    \end{itemize}
    \item \textbf{Gesamtkapitalrentabilität (Return on Assets, ROA)}: \\ $\frac{\text{Jahresgewinn + Fremdkapitalzinsen}}{\text{Gesamtkapital}} \times 100\%$
    \begin{itemize}
        \item Richtwert: $> 6-8\%$ (abhängig von Branche)
    \end{itemize}
    \item \textbf{Umsatzrentabilität (Return on Sales, ROS)}: $\frac{\text{Jahresgewinn}}{\text{Umsatz}} \times 100\%$
    \begin{itemize}
        \item Richtwert: 2-5\% im Handel, 5-10\% in der Industrie
    \end{itemize}
\end{itemize}
\end{definition}

\begin{concept}{Leverage-Effekt}
     (Hebelwirkung) beschreibt die Auswirkung des Fremdkapitals auf die Eigenkapitalrentabilität:
\begin{itemize}
    \item Solange die Gesamtkapitalrentabilität höher ist als der Fremdkapitalzinssatz, erhöht Fremdkapital die Eigenkapitalrentabilität (positiver Leverage-Effekt).
    \item Wenn die Gesamtkapitalrentabilität niedriger ist als der Fremdkapitalzinssatz, senkt Fremdkapital die Eigenkapitalrentabilität (negativer Leverage-Effekt).
\end{itemize}

Formel: ROE = ROA + (ROA - i) $\times$ Verschuldungsgrad
\begin{itemize}
    \item ROE = Eigenkapitalrentabilität, ROA = Gesamtkapitalrentabilität
    \item i = Fremdkapitalzinssatz
\end{itemize}
\end{concept}

\begin{KR}{Finanzanalyse durchführen}\\
\textbf{Datengrundlage aufbereiten}
\begin{itemize}
    \item Bilanz, Erfolgsrechnung und Geldflussrechnung beschaffen
    \item Daten auf Vollständigkeit und Richtigkeit prüfen
    \item Bilanzpositionen ggf. bereinigen (z.B. stille Reserven auflösen)
    \item Zahlen in ein einheitliches Format bringen
\end{itemize}

\textbf{Kennzahlen berechnen}
\begin{itemize}
    \item Liquiditätskennzahlen ermitteln (Liquiditätsgrade I, II und III)
    \item Sicherheitskennzahlen berechnen (Eigenkapitalquote, Fremdkapitalquote, Verschuldungsgrad)
    \item Rentabilitätskennzahlen bestimmen (ROE, ROA, ROS)
    \item Weitere branchenspezifische Kennzahlen berechnen
\end{itemize}

\textbf{Analyse und Interpretation}
\begin{itemize}
    \item Kennzahlen mit Richtwerten vergleichen
    \item Zeitliche Entwicklung der Kennzahlen analysieren (Trend)
    \item Branchenvergleich durchführen (Benchmarking)
    \item Zusammenhänge zwischen den Kennzahlen erkennen
    \item Stärken und Schwächen identifizieren
\end{itemize}

\textbf{Massnahmen ableiten}
\begin{itemize}
    \item Handlungsbedarf in den Bereichen Liquidität, Sicherheit und Rentabilität
    \item Konkrete Massnahmen zur Verbesserung der Finanzsituation formulieren
    \item Prioritäten setzen und Zeithorizont festlegen
    \item Verantwortlichkeiten für die Umsetzung definieren
\end{itemize}
\end{KR}