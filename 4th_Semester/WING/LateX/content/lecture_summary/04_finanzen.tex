\section{Finanzen I: Bilanz und Erfolgsrechnung}

\subsection{Überblick Rechnungswesen}

\begin{definition}{Rechnungswesen}\\
Das Rechnungswesen eines Unternehmens ist die systematische Erfassung, Überwachung und informatorische Auswertung aller quantifizierbaren Vorgänge eines Unternehmens. Es gliedert sich in:
\begin{itemize}
    \item \textbf{Externes Rechnungswesen (Finanzbuchhaltung)}: Für externe Adressaten wie Investoren, Banken, Staat
    \item \textbf{Internes Rechnungswesen (Betriebsbuchhaltung)}: Für interne Steuerung und Entscheidungsfindung
\end{itemize}
\end{definition}

\begin{concept}{Finanzbuchhaltung (FIBU)}\\
Die Finanzbuchhaltung dient der Dokumentation aller Geschäftsvorfälle eines Unternehmens und bildet die Grundlage für den Jahresabschluss. Sie umfasst:
\begin{itemize}
    \item \textbf{Bilanz}: Überblick über Vermögen und Kapital zu einem bestimmten Stichtag
    \item \textbf{Erfolgsrechnung (GuV)}: Darstellung von Aufwand und Ertrag während einer Periode
    \item \textbf{Geldflussrechnung (Kapitalflussrechnung)}: Darstellung der Veränderung der liquiden Mittel während einer Periode
    \item \textbf{Anhang}: Ergänzende Informationen zum Jahresabschluss
\end{itemize}
Hauptzweck: Gesetzliche Anforderungen erfüllen, Gewinn/Verlust ermitteln, Basis für Besteuerung
\end{concept}

\begin{concept}{Betriebsbuchhaltung (BEBU)}\\
Die Betriebsbuchhaltung (auch Kostenrechnung) dient der internen Steuerung und Kontrolle. Sie umfasst:
\begin{itemize}
    \item \textbf{Kostenartenrechnung}: Welche Kosten in welcher Höhe sind angefallen?
    \item \textbf{Kostenstellenrechnung}: Wo sind Kosten angefallen?
    \item \textbf{Kostenträgerrechnung}: Wofür sind Kosten angefallen?
    \item \textbf{Erfolgsrechnung}: Welche Erträge und Aufwendungen sind entstanden?
\end{itemize}
Hauptzweck: Interne Steuerung, Kalkulation, Preisgestaltung, Entscheidungsgrundlage für Management
\end{concept}

\subsection{Bilanz}

\begin{definition}{Bilanz}\\
Die Bilanz ist eine Gegenüberstellung von Aktiven (Vermögen/Mittelverwendung) und Passiven (Schulden/Mittelherkunft) zum Bilanzstichtag. Sie ist eine Momentaufnahme der Vermögens- und Kapitallage des Unternehmens.

Für die Bilanz gilt stets: Summe Aktiven = Summe Passiven
\end{definition}

\begin{definition}{Aktiven (Vermögen)}\\
Die Aktivseite der Bilanz zeigt, worin das Vermögen des Unternehmens gebunden ist. Sie wird nach dem Liquiditätsprinzip (Verfügbarkeit) gegliedert:
\begin{itemize}
    \item \textbf{Umlaufvermögen}: Vermögenswerte, die kurzfristig (innerhalb eines Jahres) umgeschlagen oder verbraucht werden
    \begin{itemize}
        \item Flüssige Mittel (Kasse, Bank, Post)
        \item Wertschriften (als Liquiditätsreserve)
        \item Forderungen aus Lieferungen und Leistungen (Debitoren)
        \item Vorräte (Waren, Rohstoffe, Fertigprodukte)
        \item Aktive Rechnungsabgrenzungen
    \end{itemize}
    \item \textbf{Anlagevermögen}: Vermögenswerte, die langfristig (über ein Jahr hinaus) im Unternehmen verbleiben
    \begin{itemize}
        \item Finanzanlagen (langfristige Wertschriften, Beteiligungen)
        \item Sachanlagen (Grundstücke, Gebäude, Maschinen, Einrichtungen)
        \item Immaterielle Anlagen (Patente, Lizenzen, Goodwill)
    \end{itemize}
\end{itemize}
\end{definition}

\begin{definition}{Passiven (Kapital)}\\
Die Passivseite der Bilanz zeigt, wie das Vermögen des Unternehmens finanziert ist. Sie wird nach dem Fälligkeitsprinzip (Fristigkeit) gegliedert:
\begin{itemize}
    \item \textbf{Fremdkapital}: Kapital, das dem Unternehmen von Dritten zur Verfügung gestellt wird
    \begin{itemize}
        \item Kurzfristiges Fremdkapital (Fälligkeit bis ein Jahr)
        \begin{itemize}
            \item Verbindlichkeiten aus Lieferungen und Leistungen (Kreditoren)
            \item Kurzfristige Finanzverbindlichkeiten (Bankkredite)
            \item Kundenanzahlungen
            \item Passive Rechnungsabgrenzungen
        \end{itemize}
        \item Langfristiges Fremdkapital (Fälligkeit über ein Jahr)
        \begin{itemize}
            \item Langfristige Finanzverbindlichkeiten (Darlehen, Anleihen)
            \item Hypotheken
            \item Rückstellungen
        \end{itemize}
    \end{itemize}
    \item \textbf{Eigenkapital}: Vom Eigentümer eingebrachtes Kapital und thesaurierte Gewinne
    \begin{itemize}
        \item Grund-/Aktienkapital
        \item Reserven (gesetzliche und freie Reserven)
        \item Gewinnvortrag / Bilanzverlust
        \item Jahresgewinn / Jahresverlust
    \end{itemize}
\end{itemize}

Für das Eigenkapital gilt: Eigenkapital = Summe Aktiven - Fremdkapital
\end{definition}

\begin{concept}{Funktion der Bilanz}\\
Die Bilanz erfüllt verschiedene Funktionen:
\begin{itemize}
    \item \textbf{Dokumentationsfunktion}: Darstellung der finanziellen Lage des Unternehmens
    \item \textbf{Informationsfunktion}: Information der internen und externen Adressaten
    \item \textbf{Rechenschaftsfunktion}: Rechenschaftslegung gegenüber Eigentümern und Gläubigern
    \item \textbf{Grundlage für Erfolgsermittlung}: Basis für die Berechnung des periodischen Erfolgs
    \item \textbf{Grundlage für Finanzanalyse}: Basis für Berechnung von Finanzkennzahlen
\end{itemize}
\end{concept}

\subsection{Erfolgsrechnung}

\begin{definition}{Erfolgsrechnung}\\
Die Erfolgsrechnung (auch Gewinn- und Verlustrechnung, GuV) stellt den während einer Periode angefallenen Aufwand eines Unternehmens den Erträgen gegenüber. Daraus resultiert der Gewinn oder Verlust einer bestimmten Periode.

Grundformel: Ertrag - Aufwand = Gewinn/Verlust
\end{definition}

\begin{definition}{Aufwand und Ertrag}\\
\begin{itemize}
    \item \textbf{Aufwand}: Wertmässiger Verbrauch von Gütern und Dienstleistungen in einer Periode
    \begin{itemize}
        \item Betriebsaufwand (z.B. Materialaufwand, Personalaufwand, Abschreibungen)
        \item Betriebsfremder Aufwand (z.B. Finanzaufwand)
        \item Ausserordentlicher Aufwand
    \end{itemize}
    \item \textbf{Ertrag}: Wertmässige Zunahme durch betriebliche Leistungen in einer Periode
    \begin{itemize}
        \item Betriebsertrag (z.B. Umsatzerlöse, Eigenleistungen)
        \item Betriebsfremder Ertrag (z.B. Finanzertrag)
        \item Ausserordentlicher Ertrag
    \end{itemize}
\end{itemize}
\end{definition}

\begin{definition}{Gewinngrössen}\\
In der Erfolgsrechnung werden verschiedene Gewinngrössen ausgewiesen:
\begin{itemize}
    \item \textbf{Bruttogewinn (Gross Profit)}: Umsatzerlöse - Warenaufwand bzw. Herstellkosten
    \item \textbf{EBITDA} (Earnings Before Interest, Taxes, Depreciation and Amortization): Gewinn vor Zinsen, Steuern, Abschreibungen und Amortisationen
    \item \textbf{EBIT} (Earnings Before Interest and Taxes): Gewinn vor Zinsen und Steuern
    \item \textbf{EBT} (Earnings Before Taxes): Gewinn vor Steuern
    \item \textbf{Reingewinn / Nettogewinn / EAT} (Earnings After Taxes): Gewinn nach Steuern
\end{itemize}
\end{definition}

\begin{KR}{Bilanz erstellen}\\
\paragraph{Vermögenswerte erfassen}
\begin{itemize}
    \item Flüssige Mittel (Kasse, Bank, Post) identifizieren
    \item Forderungen aus Lieferungen und Leistungen ermitteln
    \item Warenvorräte und andere Vorräte bewerten
    \item Sachanlagen (Maschinen, Fahrzeuge, Gebäude) erfassen
    \item Finanzanlagen und immaterielle Anlagen einbeziehen
\end{itemize}

\paragraph{Kapitalstruktur ermitteln}
\begin{itemize}
    \item Kurzfristige Verbindlichkeiten (Kreditoren, kurzfristige Bankschulden) erfassen
    \item Langfristige Verbindlichkeiten (Darlehen, Hypotheken) identifizieren
    \item Eigenkapital (Grund-/Aktienkapital, Reserven) bestimmen
    \item Jahresgewinn/-verlust übertragen
\end{itemize}

\paragraph{Bilanz strukturieren}
\begin{itemize}
    \item Aktivseite nach Liquiditätsprinzip gliedern (Umlauf- vor Anlagevermögen)
    \item Passivseite nach Fälligkeitsprinzip gliedern (kurzfristiges FK, langfristiges FK, EK)
    \item Bilanzsumme berechnen und Bilanzgleichung prüfen (Aktiven = Passiven)
\end{itemize}
\end{KR}

\begin{KR}{Erfolgsrechnung erstellen}\\
\paragraph{Erträge erfassen}
\begin{itemize}
    \item Umsatzerlöse aus Verkäufen von Produkten/Dienstleistungen ermitteln
    \item Andere betriebliche Erträge erfassen
    \item Finanzerträge (Zinsen, Dividenden) berücksichtigen
    \item Ausserordentliche Erträge einbeziehen
\end{itemize}

\paragraph{Aufwendungen ermitteln}
\begin{itemize}
    \item Materialaufwand/Warenaufwand berechnen
    \item Personalaufwand erfassen
    \item Sonstigen betrieblichen Aufwand berücksichtigen
    \item Abschreibungen einbeziehen
    \item Finanzaufwand (Zinsen) berücksichtigen
    \item Steuern berechnen
\end{itemize}

\paragraph{Erfolgsrechnung strukturieren}
\begin{itemize}
    \item Bruttogewinn berechnen (Umsatzerlöse - Materialaufwand)
    \item EBITDA ermitteln (Bruttogewinn - Personalaufwand - sonstiger betrieblicher Aufwand)
    \item EBIT berechnen (EBITDA - Abschreibungen)
    \item EBT ermitteln (EBIT +/- Finanzergebnis)
    \item Reingewinn/Nettogewinn berechnen (EBT - Steuern)
\end{itemize}
\end{KR}

\raggedcolumns

\section{Finanzen II: Cashflow und Kennzahlen}

\subsection{Geldflussrechnung (Cashflow-Rechnung)}

\begin{definition}{Geldflussrechnung}\\
Die Geldflussrechnung (auch Kapitalflussrechnung oder Cashflow-Rechnung) vermittelt den Anspruchsgruppen ein Bild über die Fähigkeit eines Unternehmens, Zahlungsmittel zu erwirtschaften, und gibt Auskunft über den Zahlungsmittelbedarf eines Unternehmens. Sie ergänzt die Bilanz und Erfolgsrechnung und zeigt die realen Geldflüsse einer Periode.
\end{definition}

\begin{definition}{Bestandteile der Geldflussrechnung}\\
Die Geldflussrechnung setzt sich aus folgenden Positionen zusammen:
\begin{itemize}
    \item \textbf{Geldfluss (Cashflow) aus Betriebstätigkeit (Operating Cashflow, OCF)}: Zahlungsmittelzu- und -abflüsse aus der laufenden Geschäftstätigkeit (z.B. Zahlungen von Kunden, Zahlungen an Lieferanten und Mitarbeiter)
    \item \textbf{Geldfluss (Cashflow) aus Investitionstätigkeit (Investing Cashflow, ICF)}: Zahlungsmittelzu- und -abflüsse aus der Investitionstätigkeit (z.B. Kauf/Verkauf von Sachanlagen, Finanzanlagen)
    \item \textbf{Geldfluss (Cashflow) aus Finanzierungstätigkeit (Financing Cashflow, FCF)}: Zahlungsmittelzu- und -abflüsse aus der Finanzierungstätigkeit (z.B. Aufnahme/Rückzahlung von Krediten, Dividendenzahlungen)
\end{itemize}
\end{definition}

\begin{concept}{Cashflow-Schemata}\\
Verschiedene typische Cashflow-Schemata können die finanzielle Situation eines Unternehmens charakterisieren:
\begin{itemize}
    \item \textbf{Normalfall (gesunde Unternehmung)}: 
    \begin{itemize}
        \item OCF positiv: Das Unternehmen erwirtschaftet Geld aus seiner operativen Tätigkeit
        \item ICF negativ: Das Unternehmen investiert
        \item FCF negativ: Das Unternehmen zahlt Schulden zurück oder schüttet Dividenden aus
    \end{itemize}
    \item \textbf{Expandierende Firma (Wachstumsstrategie)}:
    \begin{itemize}
        \item OCF positiv: Das Unternehmen erwirtschaftet Geld aus seiner operativen Tätigkeit
        \item ICF stark negativ: Das Unternehmen tätigt umfangreiche Investitionen
        \item FCF positiv: Das Unternehmen nimmt zusätzliche Finanzierungsmittel auf, da der operative Cashflow für die Investitionen nicht ausreicht
    \end{itemize}
    \item \textbf{Erfolgreiche Firma mit wenig Investitionsmöglichkeiten}:
    \begin{itemize}
        \item OCF stark positiv: Das Unternehmen erwirtschaftet viel Geld aus seiner operativen Tätigkeit
        \item ICF leicht negativ oder neutral: Das Unternehmen tätigt nur Ersatzinvestitionen
        \item FCF stark negativ: Das Unternehmen zahlt hohe Dividenden und/oder kauft eigene Aktien zurück
    \end{itemize}
    \item \textbf{Startup oder Firma mit existenziellen Problemen}:
    \begin{itemize}
        \item OCF negativ: Das Unternehmen verbrennt Geld in seiner operativen Tätigkeit (Cash Loss)
        \item ICF negativ: Das Unternehmen muss trotzdem investieren
        \item FCF positiv: Das Unternehmen braucht zusätzliche Finanzierungsmittel, um zu überleben
    \end{itemize}
\end{itemize}
\end{concept}

\subsection{Unternehmensfinanzierung}

\begin{definition}{Hauptformen der Unternehmensfinanzierung}\\
Es gibt verschiedene Formen der Unternehmensfinanzierung:
\begin{itemize}
    \item \textbf{Innenfinanzierung}: Finanzierung aus dem Unternehmen selbst
    \begin{itemize}
        \item Selbstfinanzierung durch einbehaltene Gewinne
        \item Finanzierung aus Abschreibungen
        \item Finanzierung durch Umschichtung von Vermögensteilen
    \end{itemize}
    \item \textbf{Aussenfinanzierung}: Finanzierung von ausserhalb des Unternehmens
    \begin{itemize}
        \item Eigenfinanzierung (z.B. Kapitalerhöhung, Aufnahme neuer Gesellschafter)
        \item Fremdfinanzierung (z.B. Bankkredit, Anleihen, Lieferantenkredit)
    \end{itemize}
\end{itemize}
\end{definition}

\subsection{Finanzkennzahlen}

\begin{concept}{Rolle des CFO und Finanzziele}\\
Der Chief Financial Officer (CFO) ist für das Management des Finanzdreiecks verantwortlich:
\begin{itemize}
    \item \textbf{Liquidität}: Sicherstellung der Zahlungsfähigkeit
    \item \textbf{Rentabilität}: Erzielen einer angemessenen Rendite
    \item \textbf{Sicherheit}: Gewährleistung einer stabilen Finanzstruktur
\end{itemize}

Diese drei Bereiche stehen in einem Zielkonflikt zueinander. Der CFO ist dafür verantwortlich, die drei Bereiche in Einklang mit der Unternehmensstrategie zu bringen und so zu gestalten, dass die unternehmerischen Ziele erreicht werden können.
\end{concept}

\begin{definition}{Liquiditätskennzahlen}\\
Liquiditätskennzahlen zeigen die Fähigkeit eines Unternehmens, seinen kurzfristigen Zahlungsverpflichtungen nachzukommen:
\begin{itemize}
    \item \textbf{Liquiditätsgrad I (Cash Ratio)}: $\frac{\text{Flüssige Mittel}}{\text{Kurzfristiges Fremdkapital}} \times 100\%$
    \begin{itemize}
        \item Richtwert: $\geq 20-30\%$
    \end{itemize}
    \item \textbf{Liquiditätsgrad II (Quick Ratio)}: $\frac{\text{Flüssige Mittel + Forderungen}}{\text{Kurzfristiges Fremdkapital}} \times 100\%$
    \begin{itemize}
        \item Richtwert: $\geq 100-120\%$
    \end{itemize}
    \item \textbf{Liquiditätsgrad III (Current Ratio)}: $\frac{\text{Umlaufvermögen}}{\text{Kurzfristiges Fremdkapital}} \times 100\%$
    \begin{itemize}
        \item Richtwert: $\geq 150-200\%$
    \end{itemize}
\end{itemize}
\end{definition}

\begin{definition}{Sicherheitskennzahlen}\\
Sicherheitskennzahlen geben Auskunft über die Kapitalstruktur und die finanzielle Stabilität eines Unternehmens:
\begin{itemize}
    \item \textbf{Eigenkapitalquote}: $\frac{\text{Eigenkapital}}{\text{Gesamtkapital}} \times 100\%$
    \begin{itemize}
        \item Richtwert: $\geq 30\%$ (abhängig von Branche)
    \end{itemize}
    \item \textbf{Fremdkapitalquote}: $\frac{\text{Fremdkapital}}{\text{Gesamtkapital}} \times 100\%$
    \begin{itemize}
        \item Richtwert: $\leq 70\%$ (abhängig von Branche)
    \end{itemize}
    \item \textbf{Verschuldungsgrad}: $\frac{\text{Fremdkapital}}{\text{Eigenkapital}} \times 100\%$
    \begin{itemize}
        \item Richtwert: $\leq 200\%$ (abhängig von Branche)
    \end{itemize}
\end{itemize}
\end{definition}

\begin{definition}{Rentabilitätskennzahlen}\\
Rentabilitätskennzahlen messen die Profitabilität eines Unternehmens:
\begin{itemize}
    \item \textbf{Eigenkapitalrentabilität (Return on Equity, ROE)}: $\frac{\text{Jahresgewinn}}{\text{Eigenkapital}} \times 100\%$
    \begin{itemize}
        \item Richtwert: $> 8-10\%$ (abhängig von Branche und Risikoprämie)
    \end{itemize}
    \item \textbf{Gesamtkapitalrentabilität (Return on Assets, ROA)}: $\frac{\text{Jahresgewinn + Fremdkapitalzinsen}}{\text{Gesamtkapital}} \times 100\%$
    \begin{itemize}
        \item Richtwert: $> 6-8\%$ (abhängig von Branche)
    \end{itemize}
    \item \textbf{Umsatzrentabilität (Return on Sales, ROS)}: $\frac{\text{Jahresgewinn}}{\text{Umsatz}} \times 100\%$
    \begin{itemize}
        \item Richtwert: abhängig von Branche, typischerweise 2-5\% im Handel, 5-10\% in der Industrie
    \end{itemize}
\end{itemize}
\end{definition}

\begin{concept}{Leverage-Effekt}\\
Der Leverage-Effekt (Hebelwirkung) beschreibt die Auswirkung des Fremdkapitals auf die Eigenkapitalrentabilität:
\begin{itemize}
    \item Solange die Gesamtkapitalrentabilität höher ist als der Fremdkapitalzinssatz, erhöht Fremdkapital die Eigenkapitalrentabilität (positiver Leverage-Effekt).
    \item Wenn die Gesamtkapitalrentabilität niedriger ist als der Fremdkapitalzinssatz, senkt Fremdkapital die Eigenkapitalrentabilität (negativer Leverage-Effekt).
\end{itemize}

Formel: ROE = ROA + (ROA - i) $\times$ Verschuldungsgrad

Wobei:
\begin{itemize}
    \item ROE = Eigenkapitalrentabilität
    \item ROA = Gesamtkapitalrentabilität
    \item i = Fremdkapitalzinssatz
\end{itemize}
\end{concept}

\begin{KR}{Finanzanalyse durchführen}\\
\paragraph{Datengrundlage aufbereiten}
\begin{itemize}
    \item Bilanz, Erfolgsrechnung und Geldflussrechnung beschaffen
    \item Daten auf Vollständigkeit und Richtigkeit prüfen
    \item Bilanzpositionen ggf. bereinigen (z.B. stille Reserven auflösen)
    \item Zahlen in ein einheitliches Format bringen
\end{itemize}

\paragraph{Kennzahlen berechnen}
\begin{itemize}
    \item Liquiditätskennzahlen ermitteln (Liquiditätsgrade I, II und III)
    \item Sicherheitskennzahlen berechnen (Eigenkapitalquote, Fremdkapitalquote, Verschuldungsgrad)
    \item Rentabilitätskennzahlen bestimmen (ROE, ROA, ROS)
    \item Weitere branchenspezifische Kennzahlen berechnen
\end{itemize}

\paragraph{Analyse und Interpretation}
\begin{itemize}
    \item Kennzahlen mit Richtwerten vergleichen
    \item Zeitliche Entwicklung der Kennzahlen analysieren (Trend)
    \item Branchenvergleich durchführen (Benchmarking)
    \item Zusammenhänge zwischen den Kennzahlen erkennen
    \item Stärken und Schwächen identifizieren
\end{itemize}

\paragraph{Massnahmen ableiten}
\begin{itemize}
    \item Handlungsbedarf in den Bereichen Liquidität, Sicherheit und Rentabilität erkennen
    \item Konkrete Massnahmen zur Verbesserung der Finanzsituation formulieren
    \item Prioritäten setzen und Zeithorizont festlegen
    \item Verantwortlichkeiten für die Umsetzung definieren
\end{itemize}
\end{KR}

\begin{example}
Beispiel zur Berechnung und Interpretation von Finanzkennzahlen:

Ein Unternehmen weist folgende Bilanzpositionen auf (in CHF):
\begin{itemize}
    \item Flüssige Mittel: 100'000
    \item Forderungen: 150'000
    \item Vorräte: 200'000
    \item Kurzfristiges Fremdkapital: 300'000
    \item Eigenkapital: 500'000
    \item Gesamtkapital: 1'100'000
    \item Jahresgewinn: 55'000
    \item Fremdkapitalzinsen: 25'000
    \item Umsatz: 2'000'000
\end{itemize}

\textbf{Liquiditätskennzahlen:}
\begin{itemize}
    \item Liquiditätsgrad I: $\frac{100'000}{300'000} \times 100\% = 33,3\%$ (Richtwert erfüllt)
    \item Liquiditätsgrad II: $\frac{100'000 + 150'000}{300'000} \times 100\% = 83,3\%$ (unter Richtwert)
    \item Liquiditätsgrad III: $\frac{100'000 + 150'000 + 200'000}{300'000} \times 100\% = 150\%$ (Richtwert erfüllt)
\end{itemize}

\textbf{Sicherheitskennzahlen:}
\begin{itemize}
    \item Eigenkapitalquote: $\frac{500'000}{1'100'000} \times 100\% = 45,5\%$ (Richtwert erfüllt)
    \item Fremdkapitalquote: $\frac{600'000}{1'100'000} \times 100\% = 54,5\%$ (Richtwert erfüllt)
    \item Verschuldungsgrad: $\frac{600'000}{500'000} \times 100\% = 120\%$ (Richtwert erfüllt)
\end{itemize}

\textbf{Rentabilitätskennzahlen:}
\begin{itemize}
    \item Eigenkapitalrentabilität: $\frac{55'000}{500'000} \times 100\% = 11\%$ (Richtwert erfüllt)
    \item Gesamtkapitalrentabilität: $\frac{55'000 + 25'000}{1'100'000} \times 100\% = 7,3\%$ (Richtwert erfüllt)
    \item Umsatzrentabilität: $\frac{55'000}{2'000'000} \times 100\% = 2,75\%$ (branchenabhängig)
\end{itemize}

\textbf{Interpretation:} Das Unternehmen weist eine gute Eigenkapitalausstattung auf und erfüllt die meisten Richtwerte. Bei der Quick Ratio (Liquiditätsgrad II) besteht jedoch eine gewisse Liquiditätsanspannung, was auf ein erhöhtes Risiko bei kurzfristigen Zahlungsverpflichtungen hindeuten könnte. Die Rentabilitätskennzahlen sind solide, was auf eine gute Ertragskraft schließen lässt.
\end{example}