\section{Personalmanagement}

\begin{concept}{Hauptaufgaben des Personalmanagements}
\begin{itemize}
    \item \textbf{Personalplanung}: Ermittlung qualitativer/quantitativer Personalbedarf
    \item \textbf{Personalgewinnung}: Beschaffung, Auswahl/Einstellung von Mitarbeitenden
    \item \textbf{Personalentwicklung}: Förderung und Weiterbildung von Mitarbeitenden
    \item \textbf{Personalfreistellung}: Trennung vom Mitarbeitenden (Kündigung, Pensionierung)
\end{itemize}
\end{concept}


\begin{definition}{Kriterien bei der Personalauswahl}
\begin{itemize}
    \item \textbf{Fachliche Eignung}: Ausbildung, Berufserfahrung, Fachkenntnisse, Fertigkeiten
    \item \textbf{Methodische Eignung}: Arbeitsweise, Problemlösungs- und Organisationsfähigkeit
    \item \textbf{Soziale Eignung}: Kommunikationsfähigkeit, Teamfähigkeit, Konfliktfähigkeit
    \item \textbf{Persönliche Eignung}: Motivation, Belastbarkeit, Flexibilität, Zuverlässigkeit
\end{itemize}
\end{definition}



\begin{concept}{Humankapitaltheorie}
\begin{itemize}
    \item \textbf{Allgemeines Humankapital}: Wissen und Fähigkeiten, die in verschiedenen Unternehmen anwendbar sind (erhöht die Attraktivität des Mitarbeiters für andere Arbeitgeber)
    \item \textbf{Spezifisches Humankapital}: Wissen und Fähigkeiten, die nur im eigenen Unternehmen anwendbar sind (macht Mitarbeiter vom Unternehmen abhängig)
\end{itemize}

Das Unternehmen steht vor der Frage, ob es in allgemeines oder spezifisches Humankapital investieren soll:
\begin{itemize}
    \item Investition in allgemeines Humankapital: Erhöht die Attraktivität als Arbeitgeber, birgt aber das Risiko des Mitarbeiterabgangs
    \item Investition in spezifisches Humankapital: Bindet den Mitarbeiter an das Unternehmen, kann aber die Attraktivität als Arbeitgeber mindern
\end{itemize}
\end{concept}

\begin{concept}{Führungsstil-Kontinuum nach Tannenbaum \& Schmidt}
\begin{itemize}
    \item \textbf{Autoritär}: Vorgesetzter entscheidet, setzt durch, notfalls Zwang
    \item \textbf{Patriarchisch}: Vorgesetzter entscheidet, setzt mit Manipulation durch
    \item \textbf{Informierend}: Vorgesetzter entscheidet, setzt mit Überzeugung durch
    \item \textbf{Beratend}: Vorgesetzter informiert, Meinungsäusserung der Betroffenen
    \item \textbf{Konsultativ}: Gruppe entwickelt Vorschläge, Vorgesetzter wählt aus
    \item \textbf{Partizipativ}: Gruppe entscheidet in vereinbartem Rahmen autonom
    \item \textbf{Demokratisch}: Gruppe entscheidet autonom, Vorgesetzter als Koordinator
\end{itemize}

Von links nach rechts nimmt die Willensbildung beim Vorgesetzten ab und die Willensbildung bei den Mitarbeitenden zu.
\end{concept}

\begin{concept}{X-Y-Theorie von McGregor} zwei grundlegende Menschenbilder

\textbf{Theorie X}:
    \begin{itemize}
        \item Menschen haben eine natürliche Abneigung gegen Arbeit
        \item Menschen müssen kontrolliert, geführt und mit Sanktionen bedroht werden
        \item Menschen vermeiden Verantwortung und bevorzugen Anweisungen
        \item Sicherheit ist wichtiger als alle anderen Faktoren
    \end{itemize}
\textbf{Theorie Y}:
    \begin{itemize}
        \item Arbeit ist so natürlich wie Spielen und Ausruhen
        \item Selbstkontrolle ist möglich, wenn Menschen sich den Zielen verpflichtet fühlen
        \item Mitarbeiter akzeptieren und suchen Verantwortung
        \item Kreativität und Einfallsreichtum sind weit verbreitet
    \end{itemize}
Beide Menschenbilder bzw. die darauf aufbauenden Führungsstile und Unternehmenskulturen sind selbstverstärkend.
\end{concept}

\begin{concept}{Zwei-Faktoren-Theorie nach Herzberg} beeinflussen Arbeitszufriedenheit

    \textbf{Hygienefaktoren}: Faktoren, die bei Nichterfüllung zu Unzufriedenheit führen, aber bei Erfüllung keine Zufriedenheit erzeugen
    \begin{itemize}
        \item Unternehmenspolitik und -verwaltung
        \item Führungsstil
        \item Arbeitsbedingungen
        \item Beziehungen zu Vorgesetzten, Kollegen und Untergebenen
        \item Bezahlung und Sicherheit
    \end{itemize}
\textbf{Motivatoren}: Faktoren, die bei Erfüllung zu Zufriedenheit führen, aber bei Nichterfüllung nicht zwingend zu Unzufriedenheit
    \begin{itemize}
        \item Leistungserfolg
        \item Anerkennung
        \item Arbeitsinhalt
        \item Verantwortung
        \item Aufstieg und Entfaltung
    \end{itemize}

Boni (finanzielle Anreize) werden oft als Motivatoren eingesetzt, sind aber nach Herzberg eher den Hygienefaktoren zuzuordnen.
\end{concept}

