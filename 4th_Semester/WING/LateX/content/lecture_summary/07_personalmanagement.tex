\section{Personalmanagement}

\subsection{Grundlagen des Personalmanagements}

\begin{definition}{Personalmanagement}\\
Das Personalmanagement umfasst alle Massnahmen und Entscheidungen, die sich mit der Gewinnung, dem Einsatz, der Führung, der Entwicklung und der Freistellung von Mitarbeitenden befassen. Es ist ein Unterstützungsprozess im St. Galler Management-Modell und trägt wesentlich zur Erreichung der Unternehmensziele bei.
\end{definition}

\begin{concept}{Hauptaufgaben des Personalmanagements}\\
Das Personalmanagement besteht aus vier Hauptaufgabenbereichen:
\begin{itemize}
    \item \textbf{Personalplanung}: Ermittlung des qualitativen und quantitativen Personalbedarfs
    \item \textbf{Personalgewinnung}: Beschaffung, Auswahl und Einstellung von Mitarbeitenden
    \item \textbf{Personalentwicklung}: Förderung und Weiterbildung von Mitarbeitenden
    \item \textbf{Personalfreistellung}: Trennung vom Mitarbeitenden (Kündigung, Pensionierung)
\end{itemize}
\end{concept}

\subsection{Personalplanung}

\begin{definition}{Personalplanung}\\
Die Personalplanung umfasst die systematische Vorausschau auf den künftigen Personalbedarf eines Unternehmens, sowohl in quantitativer als auch in qualitativer Hinsicht. Bei der Personalplanung sind folgende Aspekte zu berücksichtigen:
\begin{itemize}
    \item Welche inhaltlichen, sachlichen Ziele soll der künftige Mitarbeiter erreichen?
    \item Welche Aufgaben sind zu bewältigen?
    \item Welche Qualifikationen sind dafür notwendig?
    \item Welche Mittel stehen zur Verfügung?
    \item Wie ist die Stelle eingegliedert?
    \item Welche persönlichen Merkmale sind für die Erfüllung des Jobs erforderlich?
    \item Welche persönlichen Merkmale sind für die Eingliederung ins Team erwünscht?
\end{itemize}
\end{definition}

\begin{definition}{Stellenbeschreibung}\\
Eine Stellenbeschreibung ist eine verbindliche, in schriftlicher Form abgefasste Beschreibung einer Arbeitsstelle, unter Berücksichtigung von Aufgaben, Verantwortung und Kompetenzen (Befugnisse).
\begin{itemize}
    \item \textbf{Aufgaben}: Was ist zu tun?
    \item \textbf{Verantwortung}: Wofür wird Rechenschaft abgelegt?
    \item \textbf{Kompetenzen}: Welche Entscheidungs- und Weisungsbefugnisse bestehen?
\end{itemize}

Je nach Funktion unterscheidet man:
\begin{itemize}
    \item \textbf{Linienstelle}: Führt Weisungen aus und ist weisungsberechtigt
    \item \textbf{Stabstelle}: Berät und unterstützt, ist aber nicht weisungsberechtigt
\end{itemize}
\end{definition}

\subsection{Personalgewinnung}

\begin{definition}{Personalgewinnung}\\
Die Personalgewinnung umfasst alle Massnahmen zur Beschaffung, Auswahl und Einstellung von Mitarbeitenden. Zu den Aufgaben der Personalgewinnung gehören:
\begin{itemize}
    \item Ermittlung der Anforderungen an die zu besetzende Stelle
    \item Personalsuche und -ansprache
    \item Personalauswahl
    \item Personaleinstellung
    \item Personaleinführung
\end{itemize}
\end{definition}

\begin{definition}{Beschaffungswege}\\
Bei der Personalgewinnung unterscheidet man zwischen internen und externen Beschaffungswegen:
\begin{itemize}
    \item \textbf{Betriebsinterne Personalgewinnung}:
    \begin{itemize}
        \item Überstunden / Mehrarbeit
        \item Verlängerung der Arbeitszeit
        \item Urlaubsverschiebung
        \item Flexible Arbeitszeitmodelle
        \item Qualifizierung
        \item Versetzung
        \item Interne Stellenausschreibung
        \item Personalentwicklung
    \end{itemize}
    \item \textbf{Betriebsexterne Personalgewinnung}:
    \begin{itemize}
        \item Regionale Arbeitsvermittlungszentren
        \item Blindbewerbung
        \item Personaldienstleistung
        \item Stellenanzeige
        \item Personalmessen
        \item Hochschulrecruiting (Praktika, Abschlussarbeiten)
        \item Mitarbeiter werben Mitarbeiter
        \item Social Media (z.B. XING, LinkedIn)
    \end{itemize}
\end{itemize}
\end{definition}

\begin{definition}{Kriterien bei der Personalauswahl}\\
Bei der Personalauswahl werden folgende Kriterien berücksichtigt:
\begin{itemize}
    \item \textbf{Fachliche Eignung}: Ausbildung, Berufserfahrung, Fachkenntnisse, Fertigkeiten
    \item \textbf{Methodische Eignung}: Arbeitsweise, Problemlösungsfähigkeit, Organisationsfähigkeit
    \item \textbf{Soziale Eignung}: Kommunikationsfähigkeit, Teamfähigkeit, Konfliktfähigkeit
    \item \textbf{Persönliche Eignung}: Motivation, Belastbarkeit, Flexibilität, Zuverlässigkeit
\end{itemize}
\end{definition}

\begin{definition}{Methoden zur Beurteilung von Bewerbern}\\
Zur Beurteilung von Bewerbern werden verschiedene Methoden eingesetzt:
\begin{itemize}
    \item \textbf{Analyse der Bewerbungsunterlagen}: Lebenslauf, Zeugnisse, Referenzen
    \item \textbf{Bewerbungsgespräch (Interview)}: Strukturiertes oder unstrukturiertes Gespräch
    \item \textbf{Testverfahren}: Intelligenz-, Persönlichkeits- und Leistungstests
    \item \textbf{Assessment-Center}: Simulation von beruflichen Situationen
    \item \textbf{Probearbeit}: Praktische Arbeitserprobung
\end{itemize}
\end{definition}

\subsection{Personalentwicklung}

\begin{definition}{Personalentwicklung}\\
Die Personalentwicklung umfasst alle Massnahmen zur Förderung und Weiterbildung von Mitarbeitenden mit dem Ziel, ihre fachlichen, methodischen, sozialen und persönlichen Kompetenzen zu erweitern und sie auf aktuelle und zukünftige Anforderungen vorzubereiten.
\end{definition}

\begin{concept}{Kompetenzorientierung}\\
Kompetenzmodelle sichern im Unternehmen eine einheitliche Grundlage der Benennung, des Verständnisses, der Bemessung und des Controllings von Kompetenzen. Sie dienen als Basis für:
\begin{itemize}
    \item Anforderungsprofile
    \item Auswahlverfahren
    \item Kompetenzentwicklungsverfahren
    \item Karriereplanungen
\end{itemize}
\end{concept}

\begin{definition}{Personalbeurteilung}\\
Die Personalbeurteilung dient der systematischen Einschätzung der Leistungen und des Verhaltens von Mitarbeitenden. Ein modernes Instrument ist die 360°-Beurteilung, bei der Rückmeldungen aus verschiedenen Perspektiven eingeholt werden:
\begin{itemize}
    \item Selbstbeurteilung
    \item Vorgesetztenbeurteilung
    \item Kollegenbeurteilung (Peer-Feedback)
    \item Mitarbeiterbeurteilung (Feedback von Untergebenen)
    \item Kundenbeurteilung
\end{itemize}
\end{definition}

\begin{concept}{Humankapitaltheorie}\\
Die Humankapitaltheorie beschreibt den Zielkonflikt zwischen Mitarbeiterentwicklung und Mitarbeiterabgang. Sie unterscheidet:
\begin{itemize}
    \item \textbf{Allgemeines Humankapital}: Wissen und Fähigkeiten, die in verschiedenen Unternehmen anwendbar sind (erhöht die Attraktivität des Mitarbeiters für andere Arbeitgeber)
    \item \textbf{Spezifisches Humankapital}: Wissen und Fähigkeiten, die nur im eigenen Unternehmen anwendbar sind (macht den Mitarbeiter vom Unternehmen abhängig)
\end{itemize}

Das Unternehmen steht vor der Frage, ob es in allgemeines oder spezifisches Humankapital investieren soll:
\begin{itemize}
    \item Investition in allgemeines Humankapital: Erhöht die Attraktivität als Arbeitgeber, birgt aber das Risiko des Mitarbeiterabgangs
    \item Investition in spezifisches Humankapital: Bindet den Mitarbeiter an das Unternehmen, kann aber die Attraktivität als Arbeitgeber mindern
\end{itemize}
\end{concept}

\subsection{Personalhonorierung}

\begin{definition}{Personalhonorierung}\\
Die Personalhonorierung umfasst alle Massnahmen zur materiellen und immateriellen Vergütung der Mitarbeitenden. Sie dient der Motivation, Bindung und Anerkennung der Leistungen der Mitarbeitenden.
\end{definition}

\begin{definition}{Monetäre und nichtmonetäre Anreize}\\
Bei der Motivation von Mitarbeitenden unterscheidet man zwischen:
\begin{itemize}
    \item \textbf{Monetäre (materielle) Anreize}:
    \begin{itemize}
        \item Direkter Geldwert: Gehalt, Boni, Provisionen, Gewinnbeteiligung
        \item Indirekter Geldwert: Firmenwagen, Diensthandy, betriebliche Altersvorsorge
    \end{itemize}
    \item \textbf{Nichtmonetäre (immaterielle) Anreize}:
    \begin{itemize}
        \item Arbeitsinhalt: Interessante Tätigkeiten, Handlungsspielraum, Verantwortung
        \item Soziale Anerkennung: Lob, Status, Titel, Auszeichnungen
        \item Entwicklungsmöglichkeiten: Weiterbildung, Karrierechancen
        \item Arbeitsumfeld: Betriebsklima, Work-Life-Balance, flexible Arbeitszeiten
    \end{itemize}
\end{itemize}
\end{definition}

\begin{concept}{Extrinsische vs. intrinsische Motivation}\\
Bei der Motivation unterscheidet man zwischen:
\begin{itemize}
    \item \textbf{Extrinsische Motivation}: Motivation durch äussere Anreize (z.B. Gehalt, Boni, Beförderung)
    \item \textbf{Intrinsische Motivation}: Motivation aus der Tätigkeit selbst (z.B. Freude an der Arbeit, Erfolgserlebnisse, persönliche Weiterentwicklung)
\end{itemize}

Geld als extrinsischer Motivator ist nicht für alle Mitarbeitenden gleich wichtig. Nach Reinhard Sprenger (Mythos Motivation) sind Menschen durch Geld nicht zu motivieren – es sind ganz andere Motive, die Menschen intrinsisch antreiben.
\end{concept}

\subsection{Personalfreistellung}

\begin{definition}{Personalfreistellung}\\
Die Personalfreistellung umfasst alle Massnahmen zur Trennung vom Mitarbeitenden. Die meisten Massnahmen lassen sich auf eine oder mehrere der folgenden Hauptursachen zurückführen:
\begin{itemize}
    \item Absatz- und Produktionsrückgang als Folge der gesamtwirtschaftlichen Entwicklung
    \item Strukturelle Veränderungen
    \item Saisonal bedingte Beschäftigungsschwankungen
    \item Betriebsstillegungen, Betriebsvernichtung, natürliches Betriebsende
    \item Standortverlegung
    \item Reorganisation
    \item Mechanisierung und Automation
\end{itemize}
\end{definition}

\begin{definition}{Arbeitszeugnisse}\\
Arbeitszeugnisse sind schriftliche Beurteilungen der Leistung und des Verhaltens eines Mitarbeitenden. Sie müssen wahrheitsgemäss, wohlwollend, vollständig und klar formuliert sein.

Besonderheiten bei Arbeitszeugnissen:
\begin{itemize}
    \item Personalfachleute lesen im Arbeitszeugnis nicht nur, was drin steht, sondern auch, was NICHT drin steht
    \item Fehlende positive Aussagen zu wichtigen Aspekten (z.B. Teamverhalten) können als negative Beurteilung interpretiert werden
    \item Bestimmte Formulierungen haben eine "codierte" Bedeutung
\end{itemize}
\end{definition}

\subsection{Führungsstile}

\begin{definition}{Führungsstil}\\
Der Führungsstil beschreibt die Art und Weise, wie eine Führungskraft ihre Mitarbeitenden führt. Es gibt verschiedene Führungsstile, die sich vor allem im Grad der Mitarbeiterbeteiligung an Entscheidungen unterscheiden.
\end{definition}

\begin{concept}{Führungsstil-Kontinuum nach Tannenbaum \& Schmidt}\\
Das Führungsstil-Kontinuum von Tannenbaum und Schmidt beschreibt verschiedene Führungsstile entlang eines Spektrums von autoritär bis demokratisch:
\begin{itemize}
    \item \textbf{Autoritär}: Vorgesetzter entscheidet, setzt durch, notfalls Zwang
    \item \textbf{Patriarchisch}: Vorgesetzter entscheidet, setzt mit Manipulation durch
    \item \textbf{Informierend}: Vorgesetzter entscheidet, setzt mit Überzeugung durch
    \item \textbf{Beratend}: Vorgesetzter informiert, Meinungsäusserung der Betroffenen
    \item \textbf{Konsultativ}: Gruppe entwickelt Vorschläge, Vorgesetzter wählt aus
    \item \textbf{Partizipativ}: Gruppe entscheidet in vereinbartem Rahmen autonom
    \item \textbf{Demokratisch}: Gruppe entscheidet autonom, Vorgesetzter als Koordinator
\end{itemize}

Von links nach rechts nimmt die Willensbildung beim Vorgesetzten ab und die Willensbildung bei den Mitarbeitenden zu.
\end{concept}

\begin{concept}{X-Y-Theorie von McGregor}\\
Douglas McGregor unterscheidet zwei grundlegende Menschenbilder, die Führungskräfte von ihren Mitarbeitenden haben können:
\begin{itemize}
    \item \textbf{Theorie X}:
    \begin{itemize}
        \item Menschen haben eine natürliche Abneigung gegen Arbeit
        \item Menschen müssen kontrolliert, geführt und mit Sanktionen bedroht werden
        \item Menschen vermeiden Verantwortung und bevorzugen Anweisungen
        \item Sicherheit ist wichtiger als alle anderen Faktoren
    \end{itemize}
    \item \textbf{Theorie Y}:
    \begin{itemize}
        \item Arbeit ist so natürlich wie Spielen und Ausruhen
        \item Selbstkontrolle ist möglich, wenn Menschen sich den Zielen verpflichtet fühlen
        \item Mitarbeiter akzeptieren und suchen Verantwortung
        \item Kreativität und Einfallsreichtum sind weit verbreitet
    \end{itemize}
\end{itemize}

Beide Menschenbilder bzw. die darauf aufbauenden Führungsstile und Unternehmenskulturen sind selbstverstärkend.
\end{concept}

\begin{concept}{Zwei-Faktoren-Theorie nach Herzberg}\\
Frederick Herzberg unterscheidet zwei Arten von Faktoren, die die Arbeitszufriedenheit beeinflussen:
\begin{itemize}
    \item \textbf{Hygienefaktoren}: Faktoren, die bei Nichterfüllung zu Unzufriedenheit führen, aber bei Erfüllung keine Zufriedenheit erzeugen
    \begin{itemize}
        \item Unternehmenspolitik und -verwaltung
        \item Führungsstil
        \item Arbeitsbedingungen
        \item Beziehungen zu Vorgesetzten, Kollegen und Untergebenen
        \item Bezahlung und Sicherheit
    \end{itemize}
    \item \textbf{Motivatoren}: Faktoren, die bei Erfüllung zu Zufriedenheit führen, aber bei Nichterfüllung nicht zwingend zu Unzufriedenheit
    \begin{itemize}
        \item Leistungserfolg
        \item Anerkennung
        \item Arbeitsinhalt
        \item Verantwortung
        \item Aufstieg und Entfaltung
    \end{itemize}
\end{itemize}

Boni (finanzielle Anreize) werden oft als Motivatoren eingesetzt, sind aber nach Herzberg eher den Hygienefaktoren zuzuordnen.
\end{concept}

\begin{KR}{Durchführung eines Bewerbungsgesprächs}\\
\paragraph{Vorbereitung des Gesprächs}
\begin{itemize}
    \item Stellenprofil und Anforderungen nochmals überprüfen
    \item Bewerbungsunterlagen gründlich durcharbeiten
    \item Strukturierter Interviewleitfaden erstellen
    \item Gesprächsraum vorbereiten (ruhige Atmosphäre, keine Störungen)
    \item Bei mehreren Interviewern: Rollen und Zuständigkeiten klären
\end{itemize}

\paragraph{Durchführung des Gesprächs}
\begin{itemize}
    \item Begrüssung und Smalltalk zur Auflockerung
    \item Vorstellung der Gesprächsteilnehmer und des Gesprächsablaufs
    \item Vorstellung des Unternehmens und der zu besetzenden Stelle
    \item Fragen zum Lebenslauf und zur Berufserfahrung
    \item Fragen zur fachlichen, methodischen, sozialen und persönlichen Eignung
    \item Fragen zur Motivation und zu den Erwartungen
    \item Raum für Fragen des Bewerbers
    \item Information über das weitere Vorgehen
\end{itemize}

\paragraph{Systematische Beurteilung}
\begin{itemize}
    \item Eindrücke unmittelbar nach dem Gespräch dokumentieren
    \item Beurteilung anhand definierter Kriterien vornehmen
    \item Bei mehreren Interviewern: Eindrücke austauschen und abgleichen
    \item Stärken und Schwächen analysieren
    \item Passung zu Stelle und Unternehmen bewerten
\end{itemize}

\paragraph{Entscheidung und Feedback}
\begin{itemize}
    \item Entscheidung für oder gegen den Bewerber treffen
    \item Bei positiver Entscheidung: Angebot unterbreiten
    \item Bei negativer Entscheidung: Zeitnahe und wertschätzende Absage
    \item Dokumentation des Auswahlverfahrens
\end{itemize}
\end{KR}