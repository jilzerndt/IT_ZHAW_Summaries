\section{Strategie}

\subsection{Definition und Einordnung der Strategie}

\begin{definition}{Strategie}\\
Eine Strategie ist ein langfristig ausgerichtetes, planvolles Anstreben eines Ziels unter Berücksichtigung der verfügbaren Mittel und Ressourcen. Im Unternehmenskontext ist die Strategie Teil der Ordnungsmomente im St. Galler Management-Modell.

Wichtige Merkmale einer Unternehmensstrategie sind:
\begin{itemize}
    \item Weitsicht und langfristige Orientierung
    \item Kontinuität und Konsistenz
    \item Zielgerichtetheit und Nachhaltigkeit
    \item Klare Abgrenzung vom Wettbewerb
\end{itemize}
\end{definition}

\begin{concept}{Handlungsebenen des Managements}\\
Management umfasst verschiedene Handlungsebenen mit unterschiedlichen Herausforderungen und Aufgabenschwerpunkten:
\begin{itemize}
    \item \textbf{Normatives Management}: Beschäftigt sich mit konfligierenden Anliegen und Interessen; Aufbau unternehmerischer Legitimations- und Verständigungspotentiale
    \item \textbf{Strategisches Management}: Bewältigt Komplexität und Ungewissheit der Marktbedingungen; Aufbau nachhaltiger Wettbewerbsvorteile
    \item \textbf{Operatives Management}: Befasst sich mit der Knappheit der Produktionsfaktoren; Gewährleistung effizienter Abläufe und Problemlösungsroutinen
\end{itemize}
\end{concept}

\subsection{Der Strategiefindungsprozess}

\begin{definition}{Strategiefindungsprozess}\\
Der Strategiefindungsprozess besteht aus vier wesentlichen Schritten:
\begin{itemize}
    \item \textbf{Strategische Analyse}: Analyse der Ausgangslage (intern und extern)
    \item \textbf{Strategische Planung}: Strategieformulierung und -auswahl
    \item \textbf{Strategische Implementation}: Umsetzung der Strategie
    \item \textbf{Strategische Kontrolle}: Überprüfung der Strategieumsetzung
\end{itemize}
\end{definition}

\subsection{Strategische Analyse}

\subsubsection{SWOT-Analyse}

\begin{definition}{SWOT-Analyse}\\
Die SWOT-Analyse ist ein Instrument zur Bestandsaufnahme und strategischen Positionierung eines Unternehmens. SWOT steht für:
\begin{itemize}
    \item \textbf{S}trengths (Stärken): Unternehmensspezifische Vorteile, interne Faktoren
    \item \textbf{W}eaknesses (Schwächen): Unternehmensspezifische Nachteile, interne Faktoren
    \item \textbf{O}pportunities (Chancen): Für alle Marktteilnehmer vorhandene positive externe Faktoren
    \item \textbf{T}hreats (Gefahren/Risiken): Für alle Marktteilnehmer vorhandene negative externe Faktoren
\end{itemize}
\end{definition}

\begin{concept}{SWOT-Strategieansätze}\\
Aus der SWOT-Analyse können vier grundlegende Strategieansätze abgeleitet werden:
\begin{itemize}
    \item \textbf{SO-Strategieansatz}: Kompetenzen gezielt ausbauen und stärken (Stärken nutzen, um Chancen zu ergreifen)
    \item \textbf{WO-Strategieansatz}: Kompetenzen gezielt und selektiv aufbauen (Schwächen überwinden, um Chancen zu nutzen)
    \item \textbf{ST-Strategieansatz}: Problemlösepotentiale mit den bestehenden Stärken aufbauen und Gefahren möglichst beherrschen (Stärken nutzen, um Bedrohungen abzuwehren)
    \item \textbf{WT-Strategieansatz}: Gezielte Gefahrenanalyse und selektiv Schwächen ausmerzen (Schwächen minimieren und Bedrohungen ausweichen)
\end{itemize}

\important{Wichtig:} Diese Ansätze sind noch keine eigentlichen Strategien, sondern nur strategische Stossrichtungen.
\end{concept}

\begin{example}
SWOT-Analyse am Beispiel einer Fluggesellschaft (easyJet):

\textbf{Stärken:}
\begin{itemize}
    \item Moderne Flugzeuge mit tiefen Betriebskosten
    \item Eine der ersten Fluggesellschaften mit Online-Buchungsplattform
    \item Einheitliches Angebot (nur Economy Class)
\end{itemize}

\textbf{Schwächen:}
\begin{itemize}
    \item Keine Interkontinentalflüge
    \item Gewisse (teure) Flughäfen gehören nicht zum Streckennetz
\end{itemize}

\textbf{Chancen:}
\begin{itemize}
    \item Wetter (schlechtes Wetter in der Schweiz begünstigt Reisen)
    \item Trend zu Wochenend-Städtereisen
    \item Grössere Flugzeuge
    \item Steigender Wohlstand
\end{itemize}

\textbf{Gefahren:}
\begin{itemize}
    \item Reisebeschränkungen
    \item Verteuerung der Treibstoffkosten
    \item Höhere Flughafentaxen
    \item Verlängerung der Nachtflugsperre
    \item Neue Billig-Airlines
\end{itemize}
\end{example}

\begin{concept}{Kern-Kompetenzen}\\
Damit eine Kompetenz zu einem dauerhaften Wettbewerbsvorteil führt, muss diese folgende Eigenschaften aufweisen:
\begin{itemize}
    \item Wertvoll
    \item Selten
    \item Nicht oder nur schwer imitierbar
    \item Nicht substituierbar
\end{itemize}
\end{concept}

\subsubsection{PESTEL-Analyse}

\begin{definition}{PESTEL-Analyse}\\
Die PESTEL-Analyse untersucht den Einfluss von sechs externen Umweltfaktoren eines Unternehmens:
\begin{itemize}
    \item \textbf{P}olitical (Politische Faktoren)
    \item \textbf{E}conomical (Ökonomische Faktoren)
    \item \textbf{S}ocial (Sozio-Kulturelle Faktoren)
    \item \textbf{T}echnological (Technologische Faktoren)
    \item \textbf{E}nvironmental (Ökologische Faktoren)
    \item \textbf{L}egal (Rechtliche Faktoren)
\end{itemize}

Mit der PESTEL-Analyse können auch die Chancen/Gefahren-Teile der SWOT-Analyse abgedeckt werden.
\end{definition}

\subsubsection{Fünf-Kräfte-Modell (Porter)}

\begin{definition}{Fünf-Kräfte-Modell (Porter)}\\
Das Fünf-Kräfte-Modell (Five Forces) von Michael E. Porter analysiert die Attraktivität und den Wettbewerb innerhalb einer Branche anhand fünf entscheidender Kräfte:
\begin{itemize}
    \item \textbf{Branchenwettbewerb}: Intensität der Rivalität unter bestehenden Wettbewerbern
    \item \textbf{Potenzielle Konkurrenten}: Bedrohung durch neue Anbieter
    \item \textbf{Ersatzprodukte}: Bedrohung durch Ersatzprodukte oder -dienstleistungen
    \item \textbf{Lieferanten}: Verhandlungsmacht der Lieferanten
    \item \textbf{Kunden}: Verhandlungsmacht der Abnehmer
\end{itemize}
\end{definition}

\begin{concept}{Eintrittsbarrieren (nach Porter)}\\
Folgende Faktoren wirken als Eintrittsbarrieren für neue Markteilnehmer:
\begin{itemize}
    \item Economies of Scale (Skaleneffekte)
    \item Produktdifferenzierung
    \item Kapitalbedarf
    \item Umstellungskosten
    \item Zugang zu Vertriebskanälen
    \item Staatlicher Einfluss
\end{itemize}
\end{concept}

\subsection{Strategische Planung}

\subsubsection{Unternehmensleitbild}

\begin{definition}{Unternehmensleitbild}\\
Ein Unternehmensleitbild definiert den Handlungsrahmen eines Unternehmens und umfasst drei wesentliche Elemente:
\begin{itemize}
    \item \textbf{Identität}: Wer sind wir? Welchen generellen Sinn erfüllt unsere Existenz?
    \item \textbf{Ziele}: Welchen wirtschaftlichen Zweck verfolgen wir? Welche Produkte/Dienstleistungen stellen wir her?
    \item \textbf{Verhaltensgrundsätze}: Wie verhalten wir uns gegenüber Anspruchsgruppen? Welche Grundsätze gelten für unser tägliches Handeln?
\end{itemize}
\end{definition}

\begin{concept}{Mission und Vision}\\
Mission und Vision bilden die Kurzform des Leitbilds und geben Orientierung, Motivation und den übergeordneten Sinn des Unternehmens:
\begin{itemize}
    \item \textbf{Vision}: Langfristiges, inspirierendes Zukunftsbild des Unternehmens ("Wohin wollen wir?")
    \item \textbf{Mission}: Grundauftrag, den sich das Unternehmen selbst stellt ("Was tun wir?")
\end{itemize}
\end{concept}

\subsubsection{Branchenwettbewerbsstrategien nach Porter}

\begin{definition}{Branchenwettbewerbsstrategien (Porter)}\\
Porter unterscheidet vier grundlegende Wettbewerbsstrategien, die auf zwei Dimensionen basieren: dem Wettbewerbsfeld (branchenweit oder segmentspezifisch) und dem strategischen Vorteil (Kosten oder Leistung/Differenzierung):
\begin{itemize}
    \item \textbf{Kostenführerschaft}: Branchenweites Angebot eines Standardprodukts zu niedrigsten Kosten
    \item \textbf{Differenzierung}: Branchenweites Angebot eines einzigartigen, vom Kunden als wertvoll wahrgenommenen Produkts
    \item \textbf{Kostenfokus}: Konzentration auf ein Segment mit einem Standardprodukt zu niedrigsten Kosten
    \item \textbf{Differenzierungsfokus}: Konzentration auf ein Segment mit einem einzigartigen Produkt
\end{itemize}

\important{Wichtig:} Eine "Stuck in the Middle"-Positionierung zwischen den Strategien ist gefährlich, da das Unternehmen weder die Vorteile auf der Kostenseite richtig ausschöpfen kann, noch genügend gut positioniert ist, um die Vorteile auf der Differenzierungsseite zu nutzen.
\end{definition}

\subsubsection{Produkt-Markt-Strategien nach Ansoff}

\begin{definition}{Produkt-Markt-Strategien (Ansoff)}\\
Die Produkt-Markt-Matrix nach Ansoff zeigt vier grundlegende Wachstumsstrategien:
\begin{itemize}
    \item \textbf{Marktdurchdringung (Penetration)}: Ausschöpfen des bestehenden Marktes mit gegenwärtigen Produkten
    \item \textbf{Marktentwicklung}: Erschliessung neuer Märkte mit gegenwärtigen Produkten
    \item \textbf{Produktentwicklung}: Entwicklung neuer Produkte für gegenwärtige Märkte
    \item \textbf{Diversifikation}: Entwicklung neuer Produkte für neue Märkte
\end{itemize}
\end{definition}

\begin{concept}{Stossrichtungen und Erfolgsaussichten nach Ansoff}\\
Die vier Produkt-Markt-Strategien lassen sich in drei Stossrichtungen zusammenfassen:
\begin{itemize}
    \item \textbf{Bewahrungsstrategie}: Marktdurchdringung (Erfolgsaussicht ca. 75\%)
    \item \textbf{Entwicklungsstrategie}: Marktentwicklung (Erfolgsaussicht ca. 45\%) oder Produktentwicklung (Erfolgsaussicht ca. 35\%)
    \item \textbf{Ausbruchstrategie}: Diversifikation, entweder durch organisches Wachstum (Erfolgsaussicht ca. 25\%) oder akquisitorisches Wachstum (Erfolgsaussicht ca. 35\%)
\end{itemize}
\end{concept}

\subsection{Praxisbeispiele}

\begin{example2}{Beispiel: Marktdurchdringung}\\
Mögliche Massnahmen zur Marktdurchdringung sind:
\begin{itemize}
    \item Intensivere Nutzung durch bestehende Kunden fördern
    \item Gewinnung von Kunden der Konkurrenz
    \item Gewinnung bisheriger Nicht-Verwender
    \item Preissenkungen oder verstärkte Werbung
\end{itemize}
\end{example2}

\begin{example2}{Beispiel: Marktentwicklung}\\
Mögliche Massnahmen zur Marktentwicklung sind:
\begin{itemize}
    \item Erschliessung neuer Regionen oder Länder
    \item Erschliessung neuer Marktsegmente
    \item Neue Vertriebskanäle nutzen
\end{itemize}
\end{example2}

\begin{example2}{Beispiel: Produktentwicklung}\\
Mögliche Massnahmen zur Produktentwicklung sind:
\begin{itemize}
    \item Produktinnovation: Entwicklung echter Neuheiten
    \item Produktoptimierung: Verbesserung bestehender Produkte
    \item Sortimentserweiterung: Ergänzung des bestehenden Sortiments
\end{itemize}
\end{example2}

\begin{example2}{Beispiel: Diversifikation}\\
Mögliche Formen der Diversifikation sind:
\begin{itemize}
    \item Horizontale Diversifikation: Erweiterung des Produktprogramms um verwandte Produkte
    \item Vertikale Diversifikation: Erweiterung der Wertschöpfungskette (vor- oder nachgelagerte Stufen)
    \item Laterale Diversifikation: Einstieg in völlig neue Geschäftsfelder
\end{itemize}
\end{example2}

\begin{KR}{SWOT-Analyse durchführen}\\
\paragraph{Ausgangslage analysieren}
\begin{itemize}
    \item Sammeln Sie relevante Informationen über das Unternehmen und sein Umfeld
    \item Identifizieren Sie die Kernkompetenzen des Unternehmens
    \item Analysieren Sie den relevanten Markt und die Wettbewerbssituation
\end{itemize}

\paragraph{Stärken und Schwächen identifizieren}
\begin{itemize}
    \item Stärken: Analysieren Sie interne Ressourcen, Fähigkeiten und Vorteile des Unternehmens
    \item Schwächen: Identifizieren Sie interne Defizite, Engpässe und Nachteile des Unternehmens
    \item Wichtig: Stärken und Schwächen beziehen sich ausschliesslich auf unternehmensinterne Faktoren
\end{itemize}

\paragraph{Chancen und Gefahren identifizieren}
\begin{itemize}
    \item Chancen: Analysieren Sie positive externe Entwicklungen und Trends, die das Unternehmen nutzen kann
    \item Gefahren: Identifizieren Sie negative externe Entwicklungen und Risiken, die das Unternehmen bedrohen
    \item Wichtig: Chancen und Gefahren beziehen sich ausschliesslich auf externe, unternehmensunabhängige Faktoren
\end{itemize}

\paragraph{SWOT-Matrix erstellen und Strategien ableiten}
\begin{itemize}
    \item Erstellen Sie eine übersichtliche SWOT-Matrix
    \item Leiten Sie strategische Handlungsoptionen ab (SO-, WO-, ST-, WT-Strategien)
    \item Priorisieren Sie die identifizierten Handlungsoptionen
    \item Entwickeln Sie konkrete Massnahmen zur Umsetzung
\end{itemize}
\end{KR}