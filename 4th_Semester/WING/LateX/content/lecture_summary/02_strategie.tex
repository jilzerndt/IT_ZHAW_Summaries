\section{Strategie}

\begin{definition}{Strategie} 
Langfristig ausgerichtetes, planvolles Anstreben eines Ziels unter Berücksichtigung der verfügbaren Mittel und Ressourcen. Im Unternehmenskontext ist die Strategie Teil der Ordnungsmomente im St. Galler Management-Modell.
\end{definition}

\begin{concept}{Handlungsebenen des Managements}
mit unterschiedlichen Herausforderungen und Aufgabenschwerpunkten:
\begin{itemize}
    \item \textbf{Normatives Management}: Beschäftigt sich mit konfligierenden Anliegen und Interessen; Aufbau unternehmerischer Legitimations- und Verständigungspotentiale
    \item \textbf{Strategisches Management}: Bewältigt Komplexität und Ungewissheit der Marktbedingungen; Aufbau nachhaltiger Wettbewerbsvorteile
    \item \textbf{Operatives Management}: Befasst sich mit der Knappheit der Produktionsfaktoren; Gewährleistung effizienter Abläufe und Problemlösungsroutinen
\end{itemize}
\end{concept}

\begin{minipage}{0.4\linewidth}
\begin{theorem}
{Unternehmensstrategie}

Wichtige Merkmale:
\begin{itemize}
    \item Weitsicht und langfristige Orientierung
    \item Kontinuität und Konsistenz
    \item Zielgerichtetheit und Nachhaltigkeit
    \item Klare Abgrenzung vom Wettbewerb
\end{itemize}
\end{theorem}
\end{minipage}
\begin{minipage}{0.64\linewidth}

\begin{corollary}{Strategiefindungsprozess}
\begin{itemize}
    \item \textbf{Strategische Analyse}: Analyse der Ausgangslage (intern/extern)
    \item \textbf{Strategische Planung}: Strategieformulierung und -auswahl
    \item \textbf{Strategische Implementation}: Umsetzung der Strategie
    \item \textbf{Strategische Kontrolle}: Überprüfung der Strategieumsetzung
\end{itemize}
\end{corollary}
\end{minipage}

\subsubsection{Strategische Analyse}

\paragraph{SWOT-Analyse}

\begin{definition}{SWOT-Analyse}
Instrument zur Bestandsaufnahme und strategischen Positionierung eines Unternehmens
\begin{itemize}
    \item \textbf{S}trengths (Stärken): Unternehmensspezifische Vorteile, interne Faktoren
    \item \textbf{W}eaknesses (Schwächen): Unternehmensspezifische Nachteile, interne Faktoren
    \item \textbf{O}pportunities (Chancen): Für alle Marktteilnehmer vorhandene positive externe Faktoren
    \item \textbf{T}hreats (Gefahren/Risiken): Für alle Marktteilnehmer vorhandene negative externe Faktoren
\end{itemize}
\end{definition}

\begin{concept}{SWOT-Strategieansätze}
\begin{itemize}
    \item \textbf{SO-Strategieansatz}: Kompetenzen gezielt ausbauen und stärken (Stärken nutzen, um Chancen zu ergreifen)
    \item \textbf{WO-Strategieansatz}: Kompetenzen gezielt und selektiv aufbauen (Schwächen überwinden, um Chancen zu nutzen)
    \item \textbf{ST-Strategieansatz}: Problemlösepotentiale mit den bestehenden Stärken aufbauen und Gefahren möglichst beherrschen (Stärken nutzen, um Bedrohungen abzuwehren)
    \item \textbf{WT-Strategieansatz}: Gezielte Gefahrenanalyse und selektiv Schwächen ausmerzen (Schwächen minimieren und Bedrohungen ausweichen)
\end{itemize}
\important{Wichtig:} Diese Ansätze sind noch keine eigentlichen Strategien, sondern nur strategische Stossrichtungen.
\end{concept}

\begin{theorem}{Kern-Kompetenzen}
Damit eine Kompetenz zu einem dauerhaften Wettbewerbsvorteil führt, muss diese folgende Eigenschaften aufweisen:
Wertvoll, Selten, Nicht oder nur schwer imitierbar, Nicht substituierbar
\end{theorem}

\begin{KR}{SWOT-Analyse durchführen}

\textbf{Ausgangslage analysieren}: 
Relevante Infos über Unternehmen und Umfeld sammeln, Kernkompetenzen des Unternehmens identifizieren, relevanten Markt und Wettbewerbssituation analysieren

\textbf{Stärken und Schwächen identifizieren} (! Unternehmensinterne Analyse)
\begin{itemize}
    \item Stärken: Analysieren Sie interne Ressourcen, Fähigkeiten und Vorteile des Unternehmens
    \item Schwächen: Identifizieren Sie interne Defizite, Engpässe und Nachteile des Unternehmens
\end{itemize}

\textbf{Chancen und Gefahren identifizieren} (! externe, unternehmensunabhängige Analyse)
\begin{itemize}
    \item Chancen: Analysieren Sie positive externe Entwicklungen und Trends, die das Unternehmen nutzen kann
    \item Gefahren: Identifizieren Sie negative externe Entwicklungen und Risiken, die das Unternehmen bedrohen
\end{itemize}

\textbf{SWOT-Matrix erstellen und Strategien ableiten}
\begin{itemize}
    \item Leiten Sie strategische Handlungsoptionen ab (SO-, WO-, ST-, WT-Strategien) und priorisiere diese
    \item Entwickeln Sie konkrete Massnahmen zur Umsetzung
\end{itemize}
\end{KR} 

\begin{minipage}{0.38\linewidth}


\begin{definition}{PESTEL-Analyse} untersucht den Einfluss von sechs externen Umweltfaktoren eines Unternehmens:
\begin{itemize}
    \item \textbf{P}olitical (Politische Faktoren)
    \item \textbf{E}conomical (Ökonomische Faktoren)
    \item \textbf{S}ocial (Sozio-Kulturelle Faktoren)
    \item \textbf{T}echnological (Technologische Faktoren)
    \item \textbf{E}nvironmental (Ökologische Faktoren)
    \item \textbf{L}egal (Rechtliche Faktoren)
\end{itemize}
\small
Mit der PESTEL-Analyse können auch die Chancen/Gefahren-Teile der SWOT-Analyse abgedeckt werden.
\end{definition}
\end{minipage}
\begin{minipage}{0.62\linewidth}
\begin{definition}{Fünf-Kräfte-Modell (Porter)} (Five Forces) 
    analysiert Attraktivität und Wettbewerb innerhalb einer Branche:
\begin{itemize}
    \item \textbf{Branchenwettbewerb}: Rivalität unter bestehenden Wettbewerbern
    \item \textbf{Potenzielle Konkurrenten}: Bedrohung durch neue Anbieter
    \item \textbf{Ersatzprodukte}: Bedrohung durch Ersatzprodukte oder -dienstleistungen
    \item \textbf{Lieferanten}: Verhandlungsmacht der Lieferanten
    \item \textbf{Kunden}: Verhandlungsmacht der Abnehmer
\end{itemize}
\end{definition}

\begin{concept}{Eintrittsbarrieren (nach Porter)}
Folgende Faktoren wirken als Eintrittsbarrieren für neue Markteilnehmer:
Economies of Scale (Skaleneffekte), Produktdifferenzierung, Kapitalbedarf, Umstellungskosten, Zugang zu Vertriebskanälen, staatlicher Einfluss
\end{concept}
\end{minipage}

\subsubsection{Strategische Planung}

\paragraph{Unternehmensleitbild}

\begin{definition}{Unternehmensleitbild} definiert den Handlungsrahmen eines Unternehmens 
\begin{itemize}
    \item \textbf{Identität}: Wer sind wir? Welchen generellen Sinn erfüllt unsere Existenz?
    \item \textbf{Ziele}: Welchen wirtschaftlichen Zweck verfolgen wir? Welche Produkte/Dienstleistungen stellen wir her?
    \item \textbf{Verhaltensgrundsätze}: Wie verhalten wir uns gegenüber Anspruchsgruppen? Welche Grundsätze gelten für unser tägliches Handeln?
\end{itemize}
\end{definition}

\begin{concept}{Mission und Vision}
Kurzform des Leitbilds und geben Orientierung, Motivation und den übergeordneten Sinn des Unternehmens:
\begin{itemize}
    \item \textbf{Vision}: Langfristiges, inspirierendes Zukunftsbild des Unternehmens ("Wohin wollen wir?")
    \item \textbf{Mission}: Grundauftrag, den sich das Unternehmen selbst stellt ("Was tun wir?")
\end{itemize}
\end{concept}

\paragraph{Branchenwettbewerbsstrategien nach Porter}

\begin{definition}{Branchenwettbewerbsstrategien (Porter)}
    basieren auf zwei Dimensionen: dem Wettbewerbsfeld (branchenweit oder segmentspezifisch) und dem strategischen Vorteil (Kosten oder Leistung/Differenzierung)
\begin{itemize}
    \item \textbf{Kostenführerschaft}: Branchenweites Angebot eines Standardprodukts zu niedrigsten Kosten
    \item \textbf{Differenzierung}: Branchenweites Angebot eines einzigartigen, vom Kunden als wertvoll wahrgenommenen Produkts
    \item \textbf{Kostenfokus}: Konzentration auf ein Segment mit einem Standardprodukt zu niedrigsten Kosten
    \item \textbf{Differenzierungsfokus}: Konzentration auf ein Segment mit einem einzigartigen Produkt
\end{itemize}
\small
\important{Wichtig:} Eine "Stuck in the Middle"-Positionierung zwischen den Strategien ist gefährlich, da das Unternehmen weder die Vorteile auf der Kostenseite richtig ausschöpfen kann, noch genügend gut positioniert ist, um die Vorteile auf der Differenzierungsseite zu nutzen.
\end{definition}

\paragraph{Produkt-Markt-Strategien nach Ansoff}

\begin{definition}{Produkt-Markt-Strategien (Ansoff)} grundlegende Wachstumsstrategien:
\begin{itemize}
    \item \textbf{Marktdurchdringung (Penetration)}: Ausschöpfen des bestehenden Marktes mit gegenwärtigen Produkten
    \item \textbf{Marktentwicklung}: Erschliessung neuer Märkte mit gegenwärtigen Produkten
    \item \textbf{Produktentwicklung}: Entwicklung neuer Produkte für gegenwärtige Märkte
    \item \textbf{Diversifikation}: Entwicklung neuer Produkte für neue Märkte
\end{itemize}
\end{definition}

\begin{concept}{Stossrichtungen und Erfolgsaussichten nach Ansoff}
\begin{itemize}
    \item \textbf{Bewahrungsstrategie}: Marktdurchdringung (Erfolgsaussicht ca. 75\%)
    \item \textbf{Entwicklungsstrategie}: Marktentwicklung (Erfolgsaussicht ca. 45\%) oder Produktentwicklung (Erfolgsaussicht ca. 35\%)
    \item \textbf{Ausbruchstrategie}: Diversifikation, entweder durch organisches Wachstum (Erfolgsaussicht ca. 25\%) oder akquisitorisches Wachstum (Erfolgsaussicht ca. 35\%)
\end{itemize}
\end{concept}



\subsubsection{Praxisbeispiele}

\small

\mult{2}

\begin{example2}{Beispiel: Marktdurchdringung}
\begin{itemize}
    \item Intensivere Nutzung durch bestehende Kunden fördern
    \item Gewinnung von Kunden der Konkurrenz
    \item Gewinnung bisheriger Nicht-Verwender
    \item Preissenkungen oder verstärkte Werbung
\end{itemize}
\end{example2}



\begin{example2}{Beispiel: Produktentwicklung}
\begin{itemize}
    \item Produktinnovation: Entwicklung echter Neuheiten
    \item Produktoptimierung: Verbesserung bestehender Produkte
    \item Sortimentserweiterung: Ergänzung des bestehenden Sortiments
\end{itemize}
\end{example2}

\begin{example2}{Beispiel: Marktentwicklung}
\begin{itemize}
    \item Erschliessung neuer Regionen oder Länder
    \item Erschliessung neuer Marktsegmente
    \item Neue Vertriebskanäle nutzen
\end{itemize}
\end{example2}

\begin{example2}{Beispiel: Diversifikation}
\begin{itemize}
    \item Horizontale Diversifikation: Erweiterung des Produktprogramms um verwandte Produkte
    \item Vertikale Diversifikation: Erweiterung der Wertschöpfungskette (vor- oder nachgelagerte Stufen)
    \item Laterale Diversifikation: Einstieg in völlig neue Geschäftsfelder
\end{itemize}
\end{example2}

\multend



