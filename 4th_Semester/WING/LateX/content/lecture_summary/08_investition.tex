\section{Investitionsrechnung}

\subsubsection{Grundlagen der Investitionsrechnung}

\begin{definition}{Investition}
    langfristige Bindung finanzieller Mittel in materiellen oder immateriellen Vermögenswerten.
\begin{itemize}
    \item Langfristige Kapitalbindung
    \item Zeitliche Differenz zwischen Auszahlung und Rückflüssen
    \item Unsicherheit bezüglich der zukünftigen Rückflüsse
\end{itemize}
\end{definition}

\begin{definition}{Arten von Investitionen}

    \textbf{Nach Investitionsanlass}:
    \begin{itemize}
        \item \textbf{Neuinvestitionen}: Erstmaliger Erwerb von Anlagegütern
        \item \textbf{Ersatzinvestitionen}: Ersatz vorhandener Anlagegüter
        \item \textbf{Erweiterungsinvestitionen}: Erweiterung bestehender Kapazitäten
    \end{itemize}
\textbf{Nach Investitionsgrund}:
    \begin{itemize}
        \item \textbf{Normative Investitionen}: Aufgrund gesetzlicher Vorschriften oder Standards (z.B. Umweltschutz, Sicherheit)
        \item \textbf{Strategische Investitionen}: Zur Umsetzung der Unternehmensstrategie
        \item \textbf{Operative Investitionen}: Zur Aufrechterhaltung des laufenden Betriebs
    \end{itemize}
\textbf{Nach Investitionsobjekt}:
    \begin{itemize}
        \item \textbf{Sachinvestitionen}: Materielle Vermögenswerte (z.B. Maschinen, Gebäude)
        \item \textbf{Finanzinvestitionen}: Finanzielle Vermögenswerte (z.B. Wertpapiere, Beteiligungen)
        \item \textbf{Immaterielle Investitionen}: Immaterielle Vermögenswerte (z.B. Patente, Lizenzen)
    \end{itemize}
\end{definition}

\begin{concept}{Ertrag/Aufwand vs. Auszahlungen/Einzahlungen}
\begin{itemize}
    \item \textbf{Ertrag}: Wertmässige Zunahme in einer Periode, kann, muss aber keinen Geldzufluss darstellen. Beispiel: Die Wertzunahme einer Wertschrift wird als Ertrag gebucht, ohne dass Geld fliesst.
    \item \textbf{Einzahlung}: Stellt immer einen Geldzufluss dar.
    \item \textbf{Aufwand}: Wertmässiger Verbrauch in einer Periode, kann, muss aber keinen Geldabfluss darstellen. Beispiel: Die Bildung einer Rückstellung wird als Aufwand gebucht, ohne dass Geld fliesst.
    \item \textbf{Auszahlung}: Stellt immer einen Geldabfluss dar.
\end{itemize}

In der Investitionsrechnung werden Ein- und Auszahlungen (Zahlungsströme) betrachtet, nicht Erträge und Aufwendungen.
\end{concept}

\begin{definition}{Zahlungsströme}

    \textbf{Auszahlungen}:
    \begin{itemize}
        \item Erstinvestition / einmalige Zahlung / Anschaffung
        \item Anlaufkosten / Inbetriebnahmekosten
        \item Schulungskosten
        \item Laufende Kosten (Betrieb, Wartung)
    \end{itemize}
\textbf{Einzahlungen}:
    \begin{itemize}
        \item Absatzsteigerung
        \item Einsparungen von Ressourcen
        \item Preiserhöhungen
        \item Erweiterung des Produktspektrums
        \item Liquidationserlös (Restwert am Ende der Nutzungsdauer)
    \end{itemize}
\end{definition}

\subsubsection{Methoden der Investitionsrechnung}

\begin{definition}{Übersicht der Methoden}

    \textbf{Statische Verfahren}: Betrachten nur eine Durchschnittsperiode, berücksichtigen den Zeitwert des Geldes nicht
    \begin{itemize}
        \item Kostenvergleichsmethode
        \item Gewinnvergleichsmethode
        \item Rentabilitätsvergleich (Return on Investment, ROI)
        \item Amortisationsrechnung (Payback-Methode)
    \end{itemize}
\textbf{Dynamische Verfahren}: Betrachten den gesamten Planungszeitraum, berücksichtigen den Zeitwert des Geldes
    \begin{itemize}
        \item Kapitalwertmethode (Net Present Value, NPV)
        \item Annuitätenmethode
        \item Interne-Zinsfuss-Methode (Internal Rate of Return, IRR)
    \end{itemize}

\end{definition}

\subsubsection{Statische Verfahren der Investitionsrechnung}

\begin{definition}{Kostenvergleichsmethode}
     vergleicht die durchschnittlichen jährlichen Kosten verschiedener Investitionsalternativen. Die Alternative mit den niedrigsten Kosten wird ausgewählt.
\begin{itemize}
    \item \textbf{Variable Kosten}: Betriebskosten, Materialkosten
    \item \textbf{Fixe Kosten}: Kalkulatorische Abschreibungen, kalkulatorische Zinsen
\end{itemize}

Formel: Gesamtkosten = Variable Kosten + Fixe Kosten
\end{definition}

\begin{definition}{Gewinnvergleichsmethode}
vergleicht die durchschnittlichen jährlichen Gewinne verschiedener Investitionsalternativen. Die Alternative mit dem höchsten Gewinn wird ausgewählt.

Formel: Gewinn = Erlös - Kosten
\end{definition}

\begin{definition}{Rentabilitätsvergleich (ROI)}
 (Return on Investment, ROI) setzt den durchschnittlichen Jahresgewinn ins Verhältnis zum durchschnittlich eingesetzten Kapital. Die Alternative mit der höchsten Rentabilität wird ausgewählt.

Formel: Rentabilität = $\frac{\text{Gewinn + kalkulatorische Zinsen}}{\text{durchschnittlich eingesetztes Kapital}} \times 100\%$
\end{definition}

\begin{definition}{Amortisationsrechnung (Payback-Methode)}\
 ermittelt die Zeitdauer, die bis zur Rückzahlung des Investitionsbetrages durch die Einzahlungsüberschüsse verstreicht. Die Alternative mit der kürzesten Amortisationszeit wird ausgewählt.
\begin{itemize}
    \item \textbf{Kumulationsrechnung}: Die Einzahlungsüberschüsse werden addiert, bis die Summe dem Investitionsbetrag entspricht.
    \item \textbf{Durchschnittsmethode}: $\text{Amortisationszeit} = \frac{\text{Kapitaleinsatz}}{\text{durchschnittlicher jährlicher Cashflow}}$
\end{itemize}
\end{definition}

\subsubsection{Dynamische Verfahren der Investitionsrechnung}

\begin{concept}{Auf- und Abzinsung}
Geld hat einen Zeitwert: Ein Euro heute ist mehr wert als ein Euro in einem Jahr, da das Geld in der Zwischenzeit angelegt und verzinst werden kann. Die dynamischen Verfahren der Investitionsrechnung berücksichtigen diesen Zeitwert durch Auf- und Abzinsung.

\textbf{Aufzinsung}: Berechnung des zukünftigen Werts eines heutigen Betrags
    \begin{itemize}
        \item Formel: $\text{Zeitwert} = \text{Barwert} \times (1 + i)^n$
        \item Beispiel: 1'000 CHF heute sind bei einem Zinssatz von 5\% in 5 Jahren 1'276,28 CHF wert
    \end{itemize}
     \textbf{Abzinsung}: Berechnung des heutigen Werts eines zukünftigen Betrags
    \begin{itemize}
        \item Formel: $\text{Barwert} = \text{Zeitwert}/(1 + i)^n$
        \item Beispiel: 1'276,28 CHF in 5 Jahren sind bei einem Zinssatz von 5\% heute 1'000 CHF wert
    \end{itemize}

Der Abzinsungsfaktor (AbF) ist $\frac{1}{(1 + i)^n}$, der Rentenbarwertfaktor (RbF) für gleichbleibende Zahlungen über mehrere Perioden ist $\sum_{t=1}^{n} \frac{1}{(1 + i)^t}$.
\end{concept}

\begin{definition}{Kapitalwertmethode (NPV)} (Net Present Value) berechnet den Barwert aller mit einer Investition verbundenen Ein- und Auszahlungen. Der Kapitalwert einer Investition ist die Summe aller abgezinsten Zahlungsströme.

Formel: $NPV = -I_0 + \sum_{t=1}^{n} \frac{CF_t}{(1 + i)^t}$
\begin{itemize}
    \item $I_0$: Anfangsinvestition (zum Zeitpunkt t=0)
    \item $CF_t$: Cashflow in Periode t
    \item $i$: Kalkulationszinssatz
    \item $n$: Nutzungsdauer
\end{itemize}
\begin{itemize}
    \item NPV > 0: Die Investition ist vorteilhaft, da die Rendite höher ist als der Kalkulationszinssatz
    \item NPV = 0: Die Investition erzielt genau die geforderte Mindestverzinsung
    \item NPV < 0: Die Investition ist nicht vorteilhaft, da die Rendite niedriger ist als der Kalkulationszinssatz
\end{itemize}
\end{definition}


\begin{definition}{Interne-Zinsfuss-Methode (IRR)} 
(Internal Rate of Return) ermittelt den Zinssatz, bei dem der Kapitalwert einer Investition gleich Null ist. Der interne Zinsfuss entspricht der Rendite der Investition.

Formel: $0 = -I_0 + \sum_{t=1}^{n} \frac{CF_t}{(1 + IRR)^t}$
\begin{itemize}
    \item IRR > i: Die Investition ist vorteilhaft, da die Rendite höher ist als der Kalkulationszinssatz
    \item IRR = i: Die Investition erzielt genau die geforderte Mindestverzinsung
    \item IRR < i: Die Investition ist nicht vorteilhaft, da die Rendite niedriger ist als der Kalkulationszinssatz
\end{itemize}
\end{definition}

\begin{KR}{Durchführung einer Investitionsrechnung mit der Kapitalwertmethode}\\
\textbf{Datenerhebung}
Anfangsinvestition, Nutzungsdauer, jährliche Ein- und Auszahlungen, Liquidationserlös, Kalkulationszinssatz

\textbf{Zahlungsströme strukturieren}
\begin{itemize}
    \item Anfangsinvestition als Auszahlung zum Zeitpunkt t=0
    \item Jährliche Ein- und Auszahlungen zu einem Nettozahlungsstrom zusammenfassen
    \item Liquidationserlös als Einzahlung in der letzten Periode einplanen
    \item Zahlungsreihe für den gesamten Planungszeitraum erstellen
\end{itemize}

\textbf{Kapitalwert berechnen}
\begin{itemize}
    \item Abzinsungsfaktoren für jede Periode bestimmen: $\frac{1}{(1+i)^t}$
    \item Bei gleichbleibenden jährlichen Zahlungen: Rentenbarwertfaktor verwenden
    \item Alle Zahlungsströme auf den Zeitpunkt t=0 abzinsen
    \item Summe der abgezinsten Werte bilden (inkl. Anfangsinvestition)
\end{itemize}

\textbf{Ergebnis interpretieren}
\begin{itemize}
    \item Kapitalwert > 0: Investition ist vorteilhaft
    \item Kapitalwert = 0: Investition erzielt genau die geforderte Mindestverzinsung
    \item Kapitalwert < 0: Investition ist nicht vorteilhaft
    \item Bei mehreren Alternativen: Alternative mit dem höchsten Kapitalwert wählen
\end{itemize}

\textbf{Sensitivitätsanalyse durchführen}
\begin{itemize}
    \item Einfluss von Änderungen des Kalkulationszinssatzes untersuchen
    \item Auswirkungen von Änderungen der Ein- und Auszahlungen analysieren
    \item Kritische Werte ermitteln, bei denen der Kapitalwert gleich Null wird
    \item Robustheit der Investitionsentscheidung bewerten
\end{itemize}
\end{KR}