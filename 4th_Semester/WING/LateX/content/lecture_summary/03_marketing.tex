\section{Marketing}

\subsection{Einführung ins Marketing}

\begin{definition}{Marketing}\\
Marketing ist ein Prozess im Wirtschafts- und Sozialgefüge, durch den Einzelpersonen und Gruppen ihre Bedürfnisse und Wünsche befriedigen, indem sie Produkte und andere Dinge von Wert erstellen, anbieten und miteinander austauschen (nach Philip Kotler).

Nach Heribert Meffert bedeutet Marketing die Planung, Koordination und Kontrolle aller auf die aktuellen und potenziellen Märkte ausgerichteten Unternehmensaktivitäten mit dem Ziel der dauerhaften Befriedigung der Kundenbedürfnisse zur Verwirklichung der Unternehmensziele.
\end{definition}

\begin{concept}{Marketing im St. Galler Management-Modell}\\
Marketing ist im St. Galler Management-Modell vorwiegend als Geschäftsprozess verortet. Marketing befasst sich mit der Analyse, Planung, Umsetzung und Kontrolle von Marktaktivitäten. Es dient der:
\begin{itemize}
    \item Langfristigen Ausrichtung (strategisches Marketing)
    \item Kurzfristigen Umsetzung (operatives Marketing)
\end{itemize}
\end{concept}

\subsection{Marktanalyse}

\begin{definition}{Marktanalyse}\\
Die Marktanalyse dient der systematischen Erfassung und Auswertung von Informationen über Märkte und bildet die Grundlage für fundierte Marketingentscheidungen. Sie umfasst:
\begin{itemize}
    \item \textbf{Marktforschung}: Qualitative und quantitative Erhebung von Marktdaten
    \item \textbf{Marktsegmentierung}: Aufteilung des Gesamtmarktes in homogene Teilmärkte
    \item \textbf{Positionierung}: Ausrichtung des Angebots, um sich von der Konkurrenz abzugrenzen
\end{itemize}
\end{definition}

\begin{definition}{Absatzmärkte}\\
Je nach Marktteilnehmern unterscheidet man verschiedene Absatzmärkte:
\begin{itemize}
    \item \textbf{B-to-B} (Business-to-Business): Vermarktung von einem Unternehmen zu einem anderen Unternehmen (z.B. Produktionsmaschinen)
    \item \textbf{B-to-C} (Business-to-Customer): Vermarktung von einem Unternehmen zu Privathaushalten (z.B. Konsumgüter)
    \item \textbf{C-to-C} (Customer-to-Customer): Vermarktung von einem Privathaushalt zu einem anderen Privathaushalt (z.B. Privatverkauf über eBay)
\end{itemize}
\end{definition}

\begin{definition}{Marktforschung}\\
Die Marktforschung dient der systematischen Sammlung, Analyse und Interpretation von Daten über Märkte. Man unterscheidet:
\begin{itemize}
    \item \textbf{Quantitative Marktforschung}: Ermittlung von Marktdaten und -grössen
    \item \textbf{Qualitative Marktforschung}: Ermittlung und Verständnis von Kundenbedürfnissen
\end{itemize}

Datenerhebungsmethoden:
\begin{itemize}
    \item \textbf{Desk Research} (Sekundärforschung): Nutzung bereits vorhandener Informationen (wissenschaftliche Publikationen, Unternehmensinformationen, Brancheninformationen, öffentliche Informationen)
    \item \textbf{Field Research} (Primärforschung): Erhebung neuer Informationen durch Befragungen, Beobachtungen, Tests/Experimente oder Panels
\end{itemize}
\end{definition}

\begin{definition}{Marktgrössen}\\
Wichtige Kennzahlen zur Beschreibung von Märkten:
\begin{itemize}
    \item \textbf{Marktpotenzial}: Maximale Aufnahmefähigkeit eines Marktes für ein Produkt bei optimalen Bedingungen
    \item \textbf{Marktvolumen}: Tatsächlich realisierter Absatz aller Anbieter in einem Markt
    \item \textbf{Marktanteil}: Anteil eines Unternehmens am Marktvolumen
    \item \textbf{Marktkapazität}: Nachfragemenge, die ein Markt maximal aufnehmen kann
\end{itemize}
\end{definition}

\begin{definition}{Marktsegmentierung}\\
Die Marktsegmentierung ist die Aufteilung eines Gesamtmarktes in homogene Käufergruppen nach bestimmten Kriterien. Nach Meffert können folgende Segmentierungskriterien verwendet werden:
\begin{itemize}
    \item \textbf{Geographische Kriterien}: Region, Stadt/Land, Klimazone, etc.
    \item \textbf{Demographische Kriterien}: Alter, Geschlecht, Familienstand, Einkommen, etc.
    \item \textbf{Psychographische Kriterien}: Lebensstil, Persönlichkeit, Werte, etc.
    \item \textbf{Verhaltensorientierte Kriterien}: Kaufverhalten, Mediennutzung, Produkt-/Markennutzung, etc.
\end{itemize}
\end{definition}

\begin{definition}{Positionierung}\\
Die Positionierung bezeichnet die Ausrichtung des Angebots auf die Bedürfnisse der Zielgruppe und die Abgrenzung von Wettbewerbern. Eine erfolgreiche Positionierung:
\begin{itemize}
    \item Definiert ein klares Nutzenversprechen
    \item Grenzt sich deutlich vom Wettbewerb ab
    \item Ist langfristig ausgerichtet und konsistent
    \item Orientiert sich an den Bedürfnissen der Zielgruppe
\end{itemize}
\end{definition}

\subsection{Marketing-Mix (4P)}

\begin{definition}{Marketing-Mix (4P)}\\
Der Marketing-Mix beschreibt die Hauptaufgaben im Marketing und umfasst die "4P":
\begin{itemize}
    \item \textbf{Product (Produkt)}: Produktgestaltung, Qualität, Design, Verpackung, Marke
    \item \textbf{Price (Preis)}: Preispolitik, Rabatte, Zahlungsbedingungen
    \item \textbf{Place (Distribution)}: Vertriebswege, Standorte, Logistik
    \item \textbf{Promotion (Kommunikation)}: Werbung, PR, Verkaufsförderung, persönlicher Verkauf
\end{itemize}

Der Marketing-Mix wurde später erweitert zu 7P (zusätzlich: People, Process, Physical Evidence) für Dienstleistungen und 4C (Consumer, Cost, Convenience, Communication) für die Kundenperspektive.
\end{definition}

\subsection{Produktpolitik (Product)}

\begin{definition}{Produktpolitik}\\
Die Produktpolitik umfasst alle Entscheidungen, die sich auf die Gestaltung des Leistungsangebots eines Unternehmens beziehen. Der Wert eines Produktes definiert sich über den Nutzen, den es dem Kunden stiftet:
\begin{itemize}
    \item \textbf{Kernprodukt}: Grundnutzen des Produkts
    \item \textbf{Formales Produkt}: Eigenschaften wie Qualität, Design, Verpackung
    \item \textbf{Erweitertes Produkt}: Zusatzleistungen wie Service, Garantie
    \item \textbf{Generisches Produktkonzept}: Emotionale und symbolische Aspekte
\end{itemize}
\end{definition}

\begin{definition}{Verpackung}\\
Die Verpackung erfüllt mehrere wichtige Funktionen:
\begin{itemize}
    \item \textbf{Schutzfunktion}: Schutz des Produkts vor Beschädigung
    \item \textbf{Informationsfunktion}: Angaben zu Produkt, Anwendung, Inhaltsstoffen
    \item \textbf{Werbefunktion}: Verkaufsförderung am Point of Sale
    \item \textbf{Convenience-Funktion}: Handhabung, Transport, Lagerfähigkeit
    \item \textbf{Ökologische Funktion}: Umweltverträglichkeit, Recyclingfähigkeit
\end{itemize}
\end{definition}

\subsection{Preispolitik (Price)}

\begin{definition}{Preispolitik}\\
Die Preispolitik umfasst alle Entscheidungen zur Bestimmung und Durchsetzung von Preisen für Produkte und Dienstleistungen. Hauptaufgaben der Preispolitik sind:
\begin{itemize}
    \item Festlegung der Preispositionierung im Markt
    \item Bestimmung des Preisniveaus
    \item Ausgestaltung der Preisstruktur
    \item Anpassung an dynamische Wettbewerbssituationen
\end{itemize}
\end{definition}

\begin{definition}{Preisfestsetzung}\\
Zur Preisfestsetzung gibt es drei grundlegende Ansätze:
\begin{itemize}
    \item \textbf{Kostenorientierte Preise}: Bestimmung der Preisuntergrenze basierend auf Kosten
    \item \textbf{Nachfrageorientierte Preise}: Berücksichtigung der Reaktion der Nachfrager auf Preisveränderungen (Preiselastizität)
    \item \textbf{Wettbewerbsorientierte Preise}: Orientierung an den Preisen der Konkurrenz
\end{itemize}
\end{definition}

\begin{concept}{Break-Even-Analyse}\\
Die Break-Even-Analyse ist ein Bewertungsmodell zur Ermittlung der Absatzmenge, die erforderlich ist, um die Gewinnschwelle (Break-Even-Point) zu erreichen. Am Break-Even-Point entsprechen die Umsatzerlöse genau den Gesamtkosten:

Formel: $\text{Break-Even-Menge} = \frac{\text{Fixkosten}}{\text{Deckungsbeitrag pro Stück}}$
\end{concept}

\begin{definition}{Preisdifferenzierung}\\
Verschiedene Formen der Preisdifferenzierung:
\begin{itemize}
    \item \textbf{Zeitliche Preisdifferenzierung}: Tag- und Nachttarif, Saisonpreise
    \item \textbf{Räumliche Preisdifferenzierung}: In- und Auslandspreise
    \item \textbf{Preisdifferenzierung nach Käuferschichten}: Studententarife, Verbilligungen für bestimmte Gruppen
    \item \textbf{Preisdifferenzierung nach Abnahmemenge}: Mengenrabatte, Treueprämien
\end{itemize}
\end{definition}

\begin{definition}{Preisstrategie}\\
Wichtige Preisstrategien sind:
\begin{itemize}
    \item \textbf{Skimming (Abschöpfungsstrategie)}: Hohe Einführungspreise, die im Zeitverlauf gesenkt werden
    \item \textbf{Penetration (Durchdringungsstrategie)}: Niedrige Einführungspreise zur schnellen Marktdurchdringung
    \item \textbf{Psychologische Preissetzung}: z.B. 9,99 EUR statt 10 EUR
    \item \textbf{Freemium}: Basisversion kostenlos, Premium-Funktionen kostenpflichtig
\end{itemize}
\end{definition}

\begin{definition}{Preiselastizität}\\
Die Preiselastizität gibt an, wie die Nachfragemenge nach einem Gut auf Preisänderungen reagiert:
\begin{itemize}
    \item \textbf{Preiselastische Nachfrage}: Nachfrage sinkt überproportional mit steigendem Preis
    \item \textbf{Preisunelastische Nachfrage}: Nachfrage sinkt unterproportional mit steigendem Preis
    \item \textbf{Inverse Nachfrage}: Nachfrage steigt mit steigendem Preis (bei Luxusgütern)
\end{itemize}
\end{definition}

\subsection{Distributionspolitik (Place)}

\begin{definition}{Distributionspolitik}\\
Die Distributionspolitik umfasst alle Aktivitäten, die dazu dienen, Produkte vom Hersteller zum Endkunden zu bringen. Hauptaufgaben der Distribution sind:
\begin{itemize}
    \item Überbrückung von räumlichen Distanzen (Transport)
    \item Überbrückung von zeitlichen Distanzen (Lagerung)
    \item Herstellung von Kontakten zwischen Anbieter und Nachfrager
    \item Anpassung des Angebots an Kundenbedürfnisse
\end{itemize}
\end{definition}

\begin{definition}{Distributionsorgane}\\
Man unterscheidet unternehmens\textbf{interne} und unternehmens\textbf{externe} Distributionsorgane:
\begin{itemize}
    \item \textbf{Unternehmensintern}: Eigene Verkaufsabteilung, eigene Filialen, Verkaufsniederlassungen
    \item \textbf{Unternehmensextern}: Großhandel, Einzelhandel, Handelsvertreter, Online-Marktplätze
\end{itemize}
\end{definition}

\begin{definition}{Absatzwege}\\
Je nach Anzahl der Zwischenstufen unterscheidet man:
\begin{itemize}
    \item \textbf{Direkter Absatz}: Hersteller verkauft direkt an Endkunden (z.B. über eigene Filialen, Internet)
    \item \textbf{Indirekter Absatz}: Hersteller nutzt Absatzmittler (z.B. Großhandel, Einzelhandel)
\end{itemize}

Direkter Absatz eignet sich besonders bei:
\begin{itemize}
    \item Erklärungsbedürftigen Produkten
    \item Serviceleistungen
    \item Hohen Auftragsgrößen
    \item Überschaubarer Kundenanzahl
\end{itemize}

Indirekter Absatz eignet sich besonders bei:
\begin{itemize}
    \item Standardisierte Konsumgüter
    \item Niedrigen Auftragsgrößen
    \item Hoher Kundenanzahl
    \item Breiter geografischer Streuung
\end{itemize}
\end{definition}

\subsection{Kommunikationspolitik (Promotion)}

\begin{definition}{Kommunikationspolitik}\\
Die Kommunikationspolitik umfasst alle Maßnahmen zur Kommunikation zwischen Unternehmen und deren Zielgruppen. Ziel ist es, Aufmerksamkeit zu erregen, Interesse zu wecken, Wünsche zu erzeugen und Handlungen auszulösen (AIDA-Prinzip: Attention, Interest, Desire, Action).
\end{definition}

\begin{definition}{Kommunikationsinstrumente}\\
Wichtige Kommunikationsinstrumente sind:
\begin{itemize}
    \item \textbf{Werbung}: Bezahlte, unpersönliche Präsentation von Ideen, Gütern oder Dienstleistungen
    \item \textbf{Verkaufsförderung (Sales Promotion)}: Kurzfristige Aktionen zur Absatzsteigerung
    \item \textbf{Public Relations (PR)}: Pflege der Beziehungen zur Öffentlichkeit
    \item \textbf{Persönlicher Verkauf}: Direkte Kommunikation mit potenziellen Käufern
    \item \textbf{Direkt-Marketing}: Direkte Ansprache ausgewählter Kunden
    \item \textbf{Sponsoring}: Bereitstellung von Geld, Sachmitteln oder Dienstleistungen für einen Gesponserten
    \item \textbf{Messen und Ausstellungen}: Präsentation von Produkten oder Dienstleistungen
    \item \textbf{Social Media Marketing}: Kommunikation über soziale Netzwerke
    \item \textbf{Influencer Marketing}: Kooperation mit einflussreichen Personen in sozialen Medien
\end{itemize}
\end{definition}

\begin{concept}{AIDA-Formel}\\
Die AIDA-Formel beschreibt die Wirkungsstufen von Werbung:
\begin{itemize}
    \item \textbf{Attention}: Aufmerksamkeit erregen
    \item \textbf{Interest}: Interesse wecken
    \item \textbf{Desire}: Wunsch erzeugen
    \item \textbf{Action}: Handlung auslösen
\end{itemize}
\end{concept}

\begin{definition}{Werbekonzept}\\
Ein Werbekonzept umfasst folgende Elemente:
\begin{itemize}
    \item \textbf{Werbeobjekt}: Wofür soll die Werbung konzipiert sein?
    \item \textbf{Werbesubjekt}: Welche Zielgruppe soll angesprochen werden?
    \item \textbf{Wirkungsziele}: Welche Werbeziele sollen verfolgt werden?
    \item \textbf{Werbebotschaft}: Welche Botschaft soll vermittelt werden?
    \item \textbf{Werbemittel}: Welches Werbemittel soll eingesetzt werden?
    \item \textbf{Werbeperiode}: Wie lange soll geworben werden?
    \item \textbf{Werbebudget}: Wie hoch soll das Werbebudget sein?
\end{itemize}
\end{definition}

\begin{definition}{Werbetechniken}\\
Verschiedene Werbetechniken sind:
\begin{itemize}
    \item \textbf{Lifestyle-Technik}: Betont den Lebensstil, der zum Produkt passt
    \item \textbf{Slice-of-Life-Technik}: Zeigt zufriedene Kunden in Alltagssituationen
    \item \textbf{Dreamworld-Technik}: Nutzt die Träume und Fantasien des Käufers
    \item \textbf{Stimmungs- und Gefühlsbilder}: Schafft eine bestimmte Stimmung
    \item \textbf{Persönlichkeit als Symbolfigur}: Produktpersonifizierung
    \item \textbf{Technische Kompetenz}: Für erklärungsbedürftige Produkte
    \item \textbf{Wissenschaftlicher Nachweis}: Belegt die Wirksamkeit durch Studien
    \item \textbf{Testimonial-Werbung}: Präsentation durch sympathische Persönlichkeiten
\end{itemize}
\end{definition}

\subsection{Harmonischer Marketing-Mix}

\begin{concept}{Harmonischer Marketing-Mix}\\
Ein harmonischer Marketing-Mix zeichnet sich aus durch:
\begin{itemize}
    \item \textbf{Sinnvolle Kombination der Marketinginstrumente}: Die 4P müssen aufeinander abgestimmt sein und sich gegenseitig ergänzen
    \item \textbf{Permanente Marktorientierung}: Kontinuierliche Anpassung an Kundenbedürfnisse und Marktveränderungen
    \item \textbf{Klare Prioritäten}: Fokussierung auf die wirkungsvollsten Instrumente im jeweiligen Kontext
\end{itemize}

Beispiel: Ein Luxusprodukt (Produkt) erfordert ein entsprechend hohes Preisniveau (Preis), exklusive Vertriebskanäle (Distribution) und eine hochwertige Kommunikation (Promotion).
\end{concept}

\begin{KR}{Erstellung eines Marketing-Mix}\\
\paragraph{Marktanalyse durchführen}
\begin{itemize}
    \item Zielgruppenanalyse: Wer sind die potenziellen Kunden?
    \item Wettbewerbsanalyse: Wer sind die Hauptkonkurrenten und wie positionieren sie sich?
    \item Marktpotenzialanalyse: Wie gross ist der relevante Markt?
\end{itemize}

\paragraph{Produktpolitik gestalten}
\begin{itemize}
    \item Produkteigenschaften definieren (Qualität, Funktionalität, Design)
    \item Verpackung und Markenkonzept entwickeln
    \item Service- und Garantieleistungen festlegen
\end{itemize}

\paragraph{Preispolitik festlegen}
\begin{itemize}
    \item Preisstrategie wählen (Skimming oder Penetration)
    \item Preispositionierung im Wettbewerbsumfeld definieren
    \item Rabatt- und Zahlungsbedingungen bestimmen
\end{itemize}

\paragraph{Distributionspolitik gestalten}
\begin{itemize}
    \item Absatzwege festlegen (direkt oder indirekt)
    \item Vertriebspartner auswählen
    \item Logistikkonzept entwickeln
\end{itemize}

\paragraph{Kommunikationspolitik planen}
\begin{itemize}
    \item Kommunikationsziele definieren
    \item Geeignete Kommunikationsinstrumente auswählen
    \item Werbebudget festlegen und verteilen
\end{itemize}

\paragraph{Abstimmung und Integration}
\begin{itemize}
    \item Sicherstellung der Konsistenz zwischen den 4P
    \item Überprüfung der Übereinstimmung mit der Unternehmensstrategie
    \item Kontinuierliche Anpassung an Marktveränderungen
\end{itemize}
\end{KR}

\raggedcolumns

\section{Marketing II - Markenführung und CRM}

\subsection{Markenführung}

\begin{definition}{Marke}\\
Eine Marke ist ein Name, Begriff, Zeichen, Symbol, eine Gestaltungsform oder eine Kombination aus diesen Bestandteilen zum Zwecke der Kennzeichnung der Produkte oder Dienstleistungen eines Anbieters und zu ihrer Differenzierung gegenüber Konkurrenzangeboten. Marken erzeugen Assoziationen und Erwartungen bei den Konsumenten.
\end{definition}

\begin{concept}{Aspekte einer Marke}\\
Eine Marke besteht aus verschiedenen Aspekten:
\begin{itemize}
    \item \textbf{Formale Aspekte}: Name, Logo, Schrifttyp, Farben
    \item \textbf{Inhaltliche Aspekte}: Werte, Versprechen, Geschichte, Vision
    \item \textbf{Wirkungsaspekte}: Image, Wahrnehmung, Assoziationen, Gefühle
\end{itemize}

Die Markenpositionierung entsteht im Kopf des Kunden (Wahrnehmung). Eine starke Marke transportiert ein klares Leistungsversprechen und ein konsistentes Markenbild.
\end{concept}

\begin{definition}{Signalfunktion von Marken}\\
Marken erfüllen eine wichtige Signalfunktion für Kunden bezüglich:
\begin{itemize}
    \item \textbf{Qualität}: Verlässliche Qualitätserwartung
    \item \textbf{Preis}: Preiserwartung und Wertempfinden
    \item \textbf{Funktionalität}: Erwartete Leistungsmerkmale
    \item \textbf{Emotionen}: Gefühlswelt und Erlebnis
\end{itemize}

Dies führt beim Kunden zu Informationseffizienz, reduzierten Risiken und schafft einen ideellen/emotionalen Nutzen.
\end{definition}

\begin{concept}{Markenwert (Brand Equity)}\\
Der Markenwert bezeichnet den finanziellen Wert einer Marke, der über den rein materiellen Wert der Produkte hinausgeht. Er umfasst:
\begin{itemize}
    \item \textbf{Markenbekanntheit}: Grad der Bekanntheit bei der Zielgruppe
    \item \textbf{Markenimage}: Wahrnehmung und Assoziationen mit der Marke
    \item \textbf{Markenloyalität}: Bindung der Kunden an die Marke
    \item \textbf{Markenassets}: Rechtlicher Schutz, Exklusivität, etc.
\end{itemize}

Der Markenwert ist wichtig für:
\begin{itemize}
    \item Unternehmensbewertung bei Übernahmen und Fusionen
    \item Bewertung des immateriellen Vermögens als Teil des Unternehmenswerts
    \item Strategische Entscheidungen in der Markenführung
\end{itemize}
\end{concept}

\begin{definition}{Verknüpfung von Produkten und Marken}\\
Es gibt verschiedene Formen der Verknüpfung von Produkten und Marken:
\begin{itemize}
    \item \textbf{Einzelproduktmarken/Monomarken}: Jedes Produkt hat einen eigenen Markennamen (z.B. Procter \& Gamble mit Ariel, Pampers, etc.)
    \item \textbf{Sortimentsmarke}: Eine Marke für alle Produkte des Unternehmens (z.B. Nivea)
    \item \textbf{Mehrere Sortimentsmarken}: Mehrere Marken für unterschiedliche Produktbereiche (z.B. Volkswagen-Konzern mit VW, Audi, Skoda, etc.)
    \item \textbf{Mehrschichtige Markenverknüpfungen}: Verknüpfung von Firmenname und Markenname (z.B. Nestlé KitKat)
\end{itemize}
\end{definition}

\subsection{Customer Relationship Management (CRM)}

\begin{definition}{Customer Relationship Management}\\
Customer Relationship Management (CRM) bezeichnet die systematische Gestaltung der Kundenbeziehungsprozesse. Der Fokus liegt auf Kundenzufriedenheit und langfristigen Kundenbeziehungen durch:
\begin{itemize}
    \item \textbf{Kundengewinnung}: Akquisition neuer Kunden
    \item \textbf{Kundenbindung}: Erhalt und Ausbau bestehender Kundenbeziehungen
\end{itemize}

Kundenbindung ist deutlich kosteneffizienter als Kundengewinnung: Die Neugewinnung eines Kunden kostet etwa fünfmal so viel wie die Bindung eines bestehenden Kunden.
\end{definition}

\begin{definition}{Aufgaben der Kundenbindung}\\
Die Kundenbindung umfasst verschiedene Aufgaben:
\begin{itemize}
    \item \textbf{Systematische Betreuung}: Regelmäßige Kontaktpflege und Beratung
    \item \textbf{Beschwerdemanagement}: Professioneller Umgang mit Kundenbeschwerden
    \item \textbf{After-Sales-Services}: Kundenbetreuung nach dem Kauf (Wartung, Support)
    \item \textbf{Cross-Selling}: Angebot ergänzender Produkte/Dienstleistungen
    \item \textbf{Up-Selling}: Angebot höherwertiger Produkte/Dienstleistungen
    \item \textbf{Kundenclubs und -karten}: Schaffung exklusiver Kundenvorteile
\end{itemize}
\end{definition}

\begin{definition}{Instrumente des CRM}\\
Wichtige Instrumente des Customer Relationship Managements sind:
\begin{itemize}
    \item \textbf{Kundendatenbank}: Systematische Erfassung und Analyse von Kundendaten
    \item \textbf{Kundensegmentierung}: Einteilung von Kunden nach relevanten Kriterien
    \item \textbf{Kundenlebenszyklus-Management}: Anpassung der Maßnahmen an die Lebenszyklusphasen
    \item \textbf{Beschwerdemanagement}: Systematische Bearbeitung von Kundenreklamationen
    \item \textbf{Kundenzufriedenheits-Messung}: Regelmäßige Erhebung der Kundenzufriedenheit
    \item \textbf{Loyalitätsprogramme}: Belohnungen für treue Kunden (Bonusprogramme, Rabatte)
    \item \textbf{Personalisierte Kommunikation}: Individuell angepasste Kundenansprache
\end{itemize}
\end{definition}

\begin{concept}{Kundenpotenzial ausbauen}\\
Das CRM zielt darauf ab, das Kundenpotenzial systematisch auszubauen:
\begin{itemize}
    \item \textbf{Ausbau der Beziehungsintensität}: Steigerung der Kauffrequenz
    \item \textbf{Ausbau des Share of Wallet}: Erhöhung des Anteils am Kundenbudget
    \item \textbf{Cross-Selling-Potenzial}: Verkauf zusätzlicher Produkte/Dienstleistungen
    \item \textbf{Up-Selling-Potenzial}: Verkauf höherwertiger Produkte/Dienstleistungen
    \item \textbf{Reduktion der Abwanderungsrate}: Senkung der Kundenabwanderung
    \item \textbf{Weiterempfehlungen}: Gewinnung von Neukunden durch Bestandskunden
\end{itemize}
\end{concept}

\begin{concept}{Zusammenhang zwischen CRM und Markenführung}\\
Das Schaffen einer starken Marke, insbesondere das Aufrechterhalten von positiven Emotionen beim Kunden, ist ein starker Kundenbindungsmechanismus. Konsequentes CRM führt zu starken Marken und umgekehrt unterstützen starke Marken die Kundenbindung.
\end{concept}

\begin{KR}{Entwicklung einer CRM-Strategie}\\
\paragraph{Ausgangssituation analysieren}
\begin{itemize}
    \item Bestandskundendaten auswerten
    \item Kundensegmente identifizieren
    \item Kundenlebenszyklus analysieren
    \item Customer Journey erfassen
\end{itemize}

\paragraph{Ziele definieren}
\begin{itemize}
    \item Kundengewinnungsziele festlegen
    \item Kundenbindungsziele definieren
    \item KPIs für CRM-Massnahmen bestimmen
    \item Wirtschaftlichkeitsziele setzen
\end{itemize}

\paragraph{Massnahmen entwickeln}
\begin{itemize}
    \item Massnahmen zur Kundengewinnung konzipieren
    \item Massnahmen zur Kundenbindung planen
    \item Kommunikationskonzept entwickeln
    \item Personalisierungsstrategie festlegen
\end{itemize}

\paragraph{Ressourcen planen}
\begin{itemize}
    \item CRM-System auswählen bzw. anpassen
    \item Personalbedarf ermitteln
    \item Budget festlegen
    \item Prozesse definieren
\end{itemize}

\paragraph{Umsetzung steuern}
\begin{itemize}
    \item Massnahmen implementieren
    \item Mitarbeiter schulen
    \item Controlling-System einrichten
    \item Kontinuierliche Optimierung sicherstellen
\end{itemize}
\end{KR}

