\section{Advanced Scrum \& SAFe}

\subsection{Sprint Review}

\begin{definition}{Der Sprint Review}\\
    Der Sprint Review ist ein kollaboratives Arbeitsmeeting, das am Ende des Sprints stattfindet:
    \begin{itemize}
        \item Keine reine Demo, sondern ein interaktives Meeting
        \item Das Scrum-Team zeigt die Arbeitsergebnisse des Sprints
        \item Das Feedback steht im Zentrum
        \item Das Feedback wird als Input für den nächsten Sprint verwendet
        \item Es wird lauffähige Software gezeigt (keine PowerPoint-Präsentationen)
        \item Zeitbox: 4 Stunden für einen 4-Wochen-Sprint (proportional weniger für kürzere Sprints)
        \item Das gesamte Scrum-Team und Stakeholder nehmen teil
    \end{itemize}
\end{definition}

\begin{concept}{Mechanismen eines Sprint Reviews}\\
    Der Sprint Review folgt einer strukturierten Vorgehensweise:
    \begin{itemize}
        \item Begrüßung und Einführung durch den Scrum Master
        \item Vorstellung des Sprint-Ziels durch den Product Owner
        \item Demonstration der fertigen Funktionalität durch das Entwicklungsteam
        \item Sammlung von Feedback und Diskussion mit allen Teilnehmern
        \item Überprüfung des Product Backlogs und Ausblick auf die nächsten Sprints
        \item Anpassung der Release-Planung basierend auf dem aktuellen Status
    \end{itemize}
\end{concept}

\begin{KR}{Sprint Review Checkliste}\\
    \paragraph{Vorbereitung}
    \begin{itemize}
        \item Sprint-Ziel und User Stories überprüfen
        \item Produktvision, Roadmap und Release-Plan bereithalten
        \item Story Map für den Kontext vorbereiten
        \item Kapazität und tatsächlichen Fortschritt (Story Points) dokumentieren
        \item Demonstrationsfähige Funktionalitäten identifizieren
    \end{itemize}
    
    \paragraph{Durchführung}
    \begin{itemize}
        \item Für jede User Story:
        \begin{itemize}
            \item Priorisierung erläutern
            \item Verständnisfragen klären
            \item Umsetzung demonstrieren
        \end{itemize}
        \item Backlog-Verifikation durchführen
        \item Auf Konflikte, Definition of Done und fehlende Backlog-Einträge achten
        \item Risiken besprechen
    \end{itemize}
    
    \paragraph{Abschluss}
    \begin{itemize}
        \item Commitment für den nächsten Sprint einholen
        \item Tooling und Prozessverbesserungen diskutieren
        \item Nächste Schritte und offene Punkte festhalten
    \end{itemize}
\end{KR}

\subsection{Sprint Retrospektive}

\begin{definition}{Die Sprint Retrospektive}\\
    Die Sprint Retrospektive ist eine Gelegenheit für das Scrum-Team, sich selbst zu inspizieren und zu verbessern:
    \begin{itemize}
        \item Diskussion über den Scrum-Prozess, das Verhalten des Teams und eingesetzte Tools
        \item Erweiterung der "Definition of Done"
        \item Ziel: Umsetzbare Verbesserungen identifizieren, die das Team im nächsten Sprint umsetzen kann
        \item Findet nach dem Sprint Review und vor dem nächsten Sprint Planning statt
        \item Zeitbox: 3 Stunden für einen 4-Wochen-Sprint (proportional weniger für kürzere Sprints)
        \item Das vollständige Scrum-Team nimmt teil
    \end{itemize}
\end{definition}

\begin{concept}{Leitmotiv der Retrospektive}\\
    "Wir gehen davon aus, dass alle Beteiligten den bestmöglichen Einsatz im gegebenen Rahmen (Wissensstand, Ressourcen, Fähigkeiten) geleistet haben, unabhängig davon, was im Rahmen einer Retrospektive entdeckt worden ist."
    
    \begin{itemize}
        \item Schafft eine sichere Umgebung für offene Diskussionen
        \item Fokus auf systemische Probleme statt auf Schuldzuweisungen
        \item Ermöglicht konstruktives Feedback
    \end{itemize}
\end{concept}

\begin{concept}{Typische Fragestellungen in Retrospektiven}\\
    \begin{itemize}
        \item Was lief gut während des Sprints?
        \item Was lief nicht so gut?
        \item Was können wir anders oder besser machen?
        \item Welche konkreten Maßnahmen nehmen wir uns für den nächsten Sprint vor?
        
        \item Alternative Formate:
        \begin{itemize}
            \item Start-Stop-Continue: Was sollten wir beginnen/aufhören/weitermachen?
            \item Glad-Sad-Mad: Was macht uns froh/traurig/wütend?
            \item 4Ls: Liked, Learned, Lacked, Longed For
            \item Segelboot: Wind (Antrieb), Anker (Bremsen), Felsen (Risiken), Sonne (Positives)
        \end{itemize}
    \end{itemize}
\end{concept}

\begin{example}
    Beispiel für ein Start-Stop-Continue Retrospektive-Ergebnis:
    
    \textbf{Start:}
    \begin{itemize}
        \item Tägliches Pair Programming für komplexe Aufgaben
        \item Test-Driven Development konsequent anwenden
        \item Mehr technische Sessions zur Wissensverteilung
    \end{itemize}
    
    \textbf{Stop:}
    \begin{itemize}
        \item Zu viele parallele Aufgaben pro Entwickler
        \item Lange Meetings ohne klare Agenda
        \item Ungetesteten Code integrieren
    \end{itemize}
    
    \textbf{Continue:}
    \begin{itemize}
        \item Daily Standup pünktlich und fokussiert durchführen
        \item Code Reviews vor Integration
        \item Regelmäßige Abstimmung mit Product Owner
    \end{itemize}
\end{example}

\subsection{Definition of Done (DoD)}

\begin{definition}{Definition of Done}\\
    Die "Definition of Done" beschreibt Vollständigkeit im Sinne eines gegenseitig akzeptierten Übereinkommens aller Beteiligten, das konform zu den Governance-Vorgaben der Organisation ist.
    \begin{itemize}
        \item Bezieht sich auf User Stories und Sprints
        \item Verhindert technische Schulden
        \item Schafft klare Erwartungen bezüglich Produktqualität
        \item Wird vom gesamten Team (inklusive Product Owner) definiert
        \item Wird schriftlich festgehalten und kontinuierlich verbessert
    \end{itemize}
\end{definition}

\begin{concept}{Konsequenzen bei fehlender DoD}\\
    Ohne klare Definition of Done drohen:
    \begin{itemize}
        \item Technische Schulden
        \item Nichterledigte Arbeit, die sich auftürmt
        \item Illusionen bezüglich des Projektfortschritts (verfälschte Velocity)
        \item Unvorhersehbares Lieferdatum
        \item Team "over-commitment" in Bezug auf die Arbeit, die während eines Sprints erledigt werden kann
        \item Überraschende (unfertige) Ergebnisse während des Sprint Reviews
    \end{itemize}
\end{concept}

\begin{example}
    Beispiel für eine Definition of Done für eine User Story:
    \begin{enumerate}
        \item Unit-Tests bestehen und Abdeckung entspricht dem Standard (85\% oder mehr)
        \item Ausreichend negative Unit-Tests wurden geschrieben (mehr negative als positive)
        \item Code ist überprüft (oder paarweise programmiert)
        \item Coding-Standards sind erfüllt
        \item CI/CD ist implementiert (automatisierter Build, Einsatz und Test)
        \item Code ist refaktorisiert
        \item UAT-Tests werden bestanden (Testfallanforderungen)
        \item Nicht-funktionale Tests werden bestanden (Skalierbarkeit, Zuverlässigkeit, Sicherheit)
        \item Erforderliche Dokumentation ist fertiggestellt
        \item Performance-Tests zeigen akzeptable Reaktionszeiten
    \end{enumerate}
\end{example}

\begin{definition}{Definition of Ready}\\
    Ergänzend zur Definition of Done definiert die "Definition of Ready", wann eine User Story bereit für die Aufnahme in einen Sprint ist:
    \begin{enumerate}
        \item Die User Story ist definiert
        \item Die Akzeptanzkriterien sind definiert
        \item Die Abhängigkeiten der User Story sind identifiziert
        \item Die Schätzung durch das Team liegt vor
        \item Das Team hat die Artefakte des Nutzererlebnisses akzeptiert
        \item Performance und andere nichtfunktionale Eigenschaften sind definiert, falls notwendig
        \item Die Person, die die User Story abnimmt, ist definiert
        \item Das Team hat eine klare Vorstellung, was gezeigt werden muss
    \end{enumerate}
\end{definition}

\subsection{Scaled Agile Framework (SAFe)}

\begin{definition}{Was ist SAFe?}\\
    Das Scaled Agile Framework (SAFe) ist ein Framework zur Skalierung agiler Methoden auf Unternehmensebene:
    \begin{itemize}
        \item Kombiniert Scrum, Kanban, XP und Lean-Prinzipien
        \item Eignet sich für komplexe, große Organisationen mit vielen Teams
        \item Koordiniert die Arbeit mehrerer agiler Teams
        \item Verbindet Geschäftsstrategie mit Softwareentwicklung
        \item Unterstützt die agile Transformation auf allen Organisationsebenen
    \end{itemize}
\end{definition}

\begin{concept}{Warum SAFe?}\\
    "Immer mehr große Unternehmen und Branchen werden mit Software betrieben und als Online-Dienste angeboten - von der Filmindustrie über die Landwirtschaft bis hin zur nationalen Verteidigung"
    
    Da Software als Produktbestandteil immer wichtiger wird, müssen Organisationen angepasst werden:
    \begin{itemize}
        \item Effizienz und Stabilität einer bewährten Aufbauorganisation kombinieren mit
        \item der Innovationsgeschwindigkeit einer agilen Vorgehensweise
    \end{itemize}
\end{concept}

\begin{concept}{Duales System für agile Organisationen}\\
    SAFe fördert ein duales Betriebssystem, das zwei Netzwerktypen kombiniert:
    \begin{itemize}
        \item \textbf{Wertschöpfungs-Netzwerk:} Agile, kundenorientierte Strukturen mit schnellen Entscheidungswegen
        \item \textbf{Funktionale Hierarchie:} Effiziente, stabile Organisationsstruktur
    \end{itemize}
\end{concept}

\begin{definition}{Die Werte von SAFe}\\
    SAFe basiert auf vier Grundwerten:
    \begin{enumerate}
        \item \textbf{Alignment:} Die Mission wird kommuniziert, indem die Portfoliostrategie und Lösungsvision festgelegt und die Wertschöpfung abgestimmt wird
        
        \item \textbf{Built-in quality:} Es wird eine Umgebung geschaffen, in der integrierte Qualität zum Standard wird
        
        \item \textbf{Transparency:} Förderung der Visualisierung aller relevanten Arbeiten und Schaffung eines Umfelds, in dem "die Fakten immer freundlich sind"
        
        \item \textbf{Program execution:} Führungskräfte nehmen als Business Owner an der PI-Planung und -Ausführung teil
    \end{enumerate}
\end{definition}

\begin{concept}{Die 10 SAFe-Prinzipien}\\
    \begin{enumerate}
        \item Nehmen Sie eine wirtschaftliche Sicht ein
        \item Wenden Sie Systemdenken an
        \item Nehmen Sie Variabilität an; halten Sie Möglichkeiten offen
        \item Machen Sie inkrementelle Fortschritte mit schnellen, integrierten Lernzyklen
        \item Bauen Sie Meilensteine auf objektiven Bewertungen funktionierender Systeme auf
        \item Visualisieren und begrenzen Sie die Produktionszeit (Work-in-Process), reduzieren Sie die Chargengröße und halten Sie die Wartezeiten kurz
        \item Etablieren Sie einen Arbeitsrhythmus und synchronisieren Sie diesen durch bereichsübergreifende Planung
        \item Setzen Sie die intrinsische Motivation der Wissensarbeiter frei
        \item Dezentralisieren Sie die Entscheidungsfindung
        \item Organisieren Sie sich rund um die Wertschöpfung
    \end{enumerate}
\end{concept}

\begin{concept}{Die Ebenen von SAFe}\\
    SAFe definiert verschiedene Konfigurationen mit unterschiedlichen Ebenen:
    \begin{itemize}
        \item \textbf{Team-Ebene:} Agile Teams (Scrum oder Kanban)
        \item \textbf{Programm-Ebene:} Agile Release Trains (ARTs)
        \item \textbf{Large Solution-Ebene:} Koordination mehrerer ARTs
        \item \textbf{Portfolio-Ebene:} Strategische Ausrichtung und Investitionsentscheidungen
        \item \textbf{Enterprise-Ebene:} Unternehmensweite Agilität
    \end{itemize}
\end{concept}

\begin{definition}{Agile Release Train (ART)}\\
    Ein Agile Release Train (ART) ist ein zentrales Element in SAFe:
    \begin{itemize}
        \item Ein langlebiges Team aus agilen Teams
        \item Arbeitet mit einer gemeinsamen Vision und Richtung
        \item Besteht typischerweise aus 5-12 agilen Teams (50-125 Personen)
        \item Entwickelt und liefert Lösungen inkrementell
        \item Arbeitet in synchronisierten Zeitboxen (Program Increments, PI)
        \item Organisiert rund um Wertströme statt funktionaler Abteilungen
    \end{itemize}
\end{definition}

\begin{concept}{SAFe-Rollen}\\
    SAFe definiert zusätzliche Rollen zu den bekannten Scrum-Rollen:
    \begin{itemize}
        \item \textbf{Product Manager:} Verantwortet das übergeordnete Product Backlog (Program Backlog)
        \item \textbf{System Architect:} Zuständig für die Systemgestaltung auf höherer Ebene
        \item \textbf{Release Train Engineer (RTE):} Ähnlich einem Scrum Master auf ART-Ebene, kümmert sich um Zusammenarbeit und Abhängigkeiten zwischen Teams
        \item \textbf{Business Owner:} Behält den Blick auf ROI und Wertmaximierung auf Systemebene
        \item Zusätzlich die bekannten Scrum-Rollen (Product Owner, Scrum Master, Teams)
    \end{itemize}
\end{concept}

\begin{example}
    Ein typisches SAFe-Szenario könnte wie folgt aussehen:
    
    Ein Unternehmen entwickelt eine große Subscription Billing Platform mit mehreren integrierten Komponenten. Die Lösung wird als SAFe-Implementierung organisiert:
    
    \begin{itemize}
        \item 8 Scrum-Teams bilden einen Agile Release Train
        \item Ein Product Manager definiert die übergreifende Produkt-Roadmap
        \item Release Train Engineer koordiniert teamübergreifende Abhängigkeiten
        \item System Architect sorgt für konsistente technische Visionen
        \item PI-Planung alle 10 Wochen mit allen Teams gemeinsam
        \item Synchronisierte 2-Wochen-Sprints für alle Teams
        \item Regelmäßige System Demos zeigen den integrierten Fortschritt
    \end{itemize}
\end{example}