\section{Architecture Patterns}

\subsection{Einführung in Architektur-Muster}

\begin{definition}{Was sind Architektur-Patterns?}\\
    Architektur-Patterns (Architekturmuster) sind bewährte Lösungen für wiederkehrende Architekturprobleme in der Softwareentwicklung:
    \begin{itemize}
        \item Dokumentierte, erprobte Lösungsansätze für typische Probleme
        \item Beschreiben die Struktur, das Verhalten und die Interaktion von Komponenten
        \item Höhere Abstraktionsebene als Design Patterns (GoF)
        \item Befassen sich mit der übergreifenden Systemorganisation
        \item Helfen bei grundlegenden Architekturentscheidungen
    \end{itemize}
\end{definition}

\begin{concept}{Warum Architektur-Patterns wichtig sind}\\
    Architektur-Patterns bieten zahlreiche Vorteile:
    \begin{itemize}
        \item Bewährte Lösungen für komplexe Probleme
        \item Gemeinsame Sprache für Architekturentscheidungen
        \item Reduzieren technische Risiken durch erprobte Ansätze
        \item Verbessern die Wartbarkeit und Erweiterbarkeit von Systemen
        \item Erleichtern die Einarbeitung neuer Teammitglieder
        \item Unterstützen die Dokumentation der Systemarchitektur
    \end{itemize}
\end{concept}

\subsection{CQRS (Command Query Responsibility Segregation)}

\begin{definition}{CQRS - Command Query Responsibility Segregation}\\
    CQRS ist ein Architekturmuster, das Lese- und Schreiboperationen in einem System trennt:
    \begin{itemize}
        \item \textbf{Commands:} Ändern den Systemzustand, liefern kein Ergebnis zurück
        \item \textbf{Queries:} Liefern Ergebnisse, ändern nicht den Systemzustand
        \item Separate Modelle für Lese- und Schreiboperationen
        \item Ermöglicht unterschiedliche Optimierungen für beide Operationstypen
        \item Kann mit Event Sourcing kombiniert werden
    \end{itemize}
\end{definition}

\begin{concept}{Aufbau von CQRS}\\
    Ein CQRS-System besteht aus folgenden Hauptkomponenten:
    \begin{itemize}
        \item \textbf{Command-Seite:} 
        \begin{itemize}
            \item Command-Objekte (repräsentieren Benutzerabsichten)
            \item Command-Handler (verarbeiten Commands)
            \item Write-Model (Domänenmodell, optimiert für Konsistenz)
            \item Write-Datenbank (häufig relationale DB)
        \end{itemize}
        
        \item \textbf{Query-Seite:} 
        \begin{itemize}
            \item Query-Objekte (repräsentieren Leseanfragen)
            \item Query-Handler (verarbeiten Queries)
            \item Read-Model (denormalisierte Views, optimiert für Leseoperationen)
            \item Read-Datenbank (häufig NoSQL oder spezialisierte Leseoptimierung)
        \end{itemize}
        
        \item Optional: Synchronisationsmechanismus (um Änderungen vom Write- zum Read-Model zu propagieren)
    \end{itemize}
\end{concept}

\begin{concept}{Vorteile von CQRS}\\
    \begin{itemize}
        \item Ermöglicht unabhängige Skalierung von Lese- und Schreiboperationen
        \item Optimierung für spezifische Anwendungsfälle (z.B. komplexe Berichte)
        \item Bessere Leistung durch spezialisierte Datenmodelle
        \item Vereinfachte Domänenmodelle durch Trennung der Verantwortlichkeiten
        \item Verbesserte Sicherheit durch getrennte Berechtigungsmodelle möglich
        \item Unterstützung von komplexen Geschäftsregeln auf der Schreibseite
    \end{itemize}
\end{concept}

\begin{concept}{Nachteile von CQRS}\\
    \begin{itemize}
        \item Erhöhte Komplexität durch doppelte Modelle und Synchronisation
        \item Eventual Consistency kann für Benutzer verwirrend sein
        \item Höherer Entwicklungsaufwand, besonders in der Anfangsphase
        \item Größere Lernkurve für Entwickler
        \item Nicht für alle Anwendungen sinnvoll (Overengineering für einfache CRUD-Systeme)
    \end{itemize}
\end{concept}

\begin{examplecode}{CQRS Implementierungsbeispiel}\\
\begin{lstlisting}[language=Java, style=basesmol]
// Command-Seite
public class CreateOrderCommand {
    private final String customerId;
    private final List<OrderItem> items;
    
    // Constructor, getters...
}

public class CreateOrderHandler {
    private final OrderRepository repository;
    
    public void handle(CreateOrderCommand command) {
        // Geschaeftslogik, Validierung
        Order order = new Order(command.getCustomerId(), command.getItems());
        repository.save(order);
        eventBus.publish(new OrderCreatedEvent(order.getId()));
    }
}

// Query-Seite
public class GetOrderQuery {
    private final String orderId;
    // Constructor, getters...
}

public class GetOrderHandler {
    private final OrderReadRepository readRepository;
    
    public OrderDto handle(GetOrderQuery query) {
        return readRepository.findById(query.getOrderId());
    }
}
\end{lstlisting}
\end{examplecode}

\subsection{Event Sourcing}

\begin{definition}{Event Sourcing}\\
    Event Sourcing ist ein Architekturmuster, bei dem der Zustand eines Systems als Sequenz von Ereignissen modelliert wird:
    \begin{itemize}
        \item Der aktuelle Zustand wird durch Anwenden aller historischen Ereignisse rekonstruiert
        \item Ereignisse (Events) sind unveränderlich und werden nur hinzugefügt
        \item Jede Änderung am System wird als Event aufgezeichnet
        \item Bietet vollständige Nachvollziehbarkeit und Audit-Trail
        \item Häufig in Kombination mit CQRS eingesetzt
    \end{itemize}
\end{definition}

\begin{concept}{Kernkonzepte von Event Sourcing}\\
    \begin{itemize}
        \item \textbf{Event:} Aufzeichnung einer Zustandsänderung (z.B. OrderPlaced, ItemAdded)
        \item \textbf{Event Store:} Persistente Speicherung aller Events
        \item \textbf{Aggregate:} Konsistenzgrenzen für Entities (z.B. Order, Customer)
        \item \textbf{Snapshots:} Momentaufnahmen des Zustands zur Performance-Optimierung
        \item \textbf{Event Handlers:} Verarbeiten Events und aktualisieren Read-Modelle
        \item \textbf{Projections:} Erzeugen verschiedene Ansichten der Daten aus Events
    \end{itemize}
\end{concept}

\begin{concept}{Vorteile von Event Sourcing}\\
    \begin{itemize}
        \item Komplette Historie aller Änderungen (Audit-Trail)
        \item Zeitreise-Fähigkeit (Zustand zu jedem historischen Zeitpunkt rekonstruierbar)
        \item Ermöglicht Debugging durch Wiedergabe von Events
        \item Vereinfacht das Testen von Geschäftslogik
        \item Erlaubt nachträgliche Erstellung neuer Projektionen
        \item Verbesserte Skalierbarkeit durch Vermeidung von Update-Konflikten
        \item Robustheit gegen Datenverlust durch append-only Natur
    \end{itemize}
\end{concept}

\begin{concept}{Nachteile von Event Sourcing}\\
    \begin{itemize}
        \item Erhöhte Komplexität, besonders bei der ersten Umsetzung
        \item Erhöhte Anforderungen an die Infrastruktur (Event Store)
        \item Eventual Consistency kann für Benutzer verwirrend sein
        \item Versioning von Events kann herausfordernd sein
        \item Performance-Überlegungen bei großen Event-Streams
        \item Erhöhter initialer Entwicklungsaufwand
    \end{itemize}
\end{concept}

\begin{examplecode}{Event Sourcing Implementierungsbeispiel}\\
\begin{lstlisting}[language=Java, style=basesmol]
// Event
public class OrderPlacedEvent {
    private final String orderId;
    private final String customerId;
    private final LocalDateTime timestamp;
    private final List<OrderItem> items;
    
    // Constructor, getters...
}

// Aggregate
public class Order {
    private String id;
    private String customerId;
    private List<OrderItem> items = new ArrayList<>();
    private OrderStatus status;
    
    // Apply-Methoden fuer Events
    public void apply(OrderPlacedEvent event) {
        this.id = event.getOrderId();
        this.customerId = event.getCustomerId();
        this.items.addAll(event.getItems());
        this.status = OrderStatus.PLACED;
    }
    
    // Command-Handling
    public OrderPlacedEvent placeOrder(String customerId, List<OrderItem> items) {
        // Validierung
        OrderPlacedEvent event = new OrderPlacedEvent(
            UUID.randomUUID().toString(), 
            customerId, 
            LocalDateTime.now(), 
            items
        );
        apply(event);
        return event;
    }
}

// Event Store
public class EventStore {
    private final Map<String, List<Event>> eventStreams = new HashMap<>();
    
    public void saveEvents(String aggregateId, List<Event> events) {
        List<Event> stream = eventStreams.getOrDefault(aggregateId, new ArrayList<>());
        stream.addAll(events);
        eventStreams.put(aggregateId, stream);
    }
    
    public List<Event> getEventsForAggregate(String aggregateId) {
        return eventStreams.getOrDefault(aggregateId, new ArrayList<>());
    }
}
\end{lstlisting}
\end{examplecode}

\subsection{Strangler Pattern / Online-Migration}

\begin{definition}{Strangler Pattern}\\
    Das Strangler Pattern (auch Strangler Fig Pattern) ist ein Ansatz zur schrittweisen Migration eines Legacy-Systems zu einer neuen Architektur:
    \begin{itemize}
        \item Benannt nach der "Würgefeige" (Strangler Fig), die um einen Wirtsbaum wächst und ihn schließlich ersetzt
        \item Neue Funktionalität wird parallel zum alten System entwickelt
        \item Bestehende Funktionen werden schrittweise migriert
        \item Die Umstellung erfolgt inkrementell, nicht als Big-Bang-Ansatz
        \item Das Legacy-System wird nach und nach "erwürgt" (ersetzt)
    \end{itemize}
\end{definition}

\begin{concept}{Implementierung des Strangler Patterns}\\
    Typische Implementierungsstrategie:
    \begin{itemize}
        \item \textbf{Facade:} Einrichten einer Fassade vor dem Legacy-System
        \item \textbf{Interception:} Abfangen von Anfragen und Weiterleitung an das neue oder alte System
        \item \textbf{Koexistenz:} Altes und neues System arbeiten parallel
        \item \textbf{Graduelle Migration:} Feature für Feature wird migriert
        \item \textbf{Auslaufen:} Das alte System wird schrittweise abgeschaltet
    \end{itemize}
\end{concept}

\begin{concept}{Vorteile des Strangler Patterns}\\
    \begin{itemize}
        \item Reduziertes Risiko im Vergleich zu einem kompletten Neubau
        \item Frühes Wertschöpfungspotenzial durch inkrementelle Lieferung
        \item Möglichkeit, früh Feedback zu erhalten und zu reagieren
        \item Kontinuierliche Bereitstellung von Geschäftswert
        \item Lernen während der Migration
        \item Bessere Ressourcenplanung möglich
    \end{itemize}
\end{concept}

\begin{concept}{Herausforderungen des Strangler Patterns}\\
    \begin{itemize}
        \item Komplexere Übergangszustände während der Migration
        \item Höherer Koordinationsaufwand zwischen Systemen
        \item Notwendigkeit für Kompatibilitätsschichten
        \item Daten-Synchronisationsprobleme
        \item Längere Gesamtmigrationsdauer
        \item Risiko unvollständiger Migration ("ewiges" Projekt)
    \end{itemize}
\end{concept}

\begin{KR}{Durchführung einer Strangler-Migration}\\
    \paragraph{Analyse und Planung}
    \begin{itemize}
        \item Legacy-System analysieren und verstehen
        \item Abhängigkeiten und Integrationspunkte identifizieren
        \item Migrationsreihenfolge basierend auf Geschäftswert und technischer Machbarkeit festlegen
        \item Fassadenschicht planen (API Gateway, Reverse Proxy etc.)
    \end{itemize}
    
    \paragraph{Implementierung}
    \begin{itemize}
        \item Fassadenschicht vor dem Legacy-System einrichten
        \item Routingregeln definieren (neu vs. alt)
        \item Mit isolierten, weniger kritischen Funktionen beginnen
        \item Schrittweise weitere Funktionen migrieren
        \item Dual-Write-Mechanismen für Datenmigration einsetzen
    \end{itemize}
    
    \paragraph{Konsolidierung}
    \begin{itemize}
        \item Erfolgskriterien für jede migrierte Funktion definieren
        \item Traffic schrittweise zum neuen System umleiten
        \item Alt-System-Komponenten nach erfolgreicher Migration abschalten
        \item Ressourcen für das Legacy-System reduzieren
        \item Migration dokumentieren und Lessons Learned festhalten
    \end{itemize}
\end{KR}

\subsection{Circuit Breaker / Bulkhead / Retry}

\begin{definition}{Circuit Breaker Pattern}\\
    Das Circuit Breaker Pattern verhindert, dass eine Anwendung versucht, eine Operation auszuführen, die wahrscheinlich fehlschlagen wird:
    \begin{itemize}
        \item Basiert auf der Idee elektrischer Sicherungen
        \item Überwacht Fehler in Aufrufen externer Systeme
        \item Bei zu vielen Fehlern "öffnet" sich der Circuit Breaker und blockiert weitere Aufrufe
        \item Nach einer Wartezeit wird ein "Halboffener" Zustand eingenommen, um probeweise Aufrufe zuzulassen
        \item Bei erfolgreichen Aufrufen schließt sich der Circuit Breaker wieder
    \end{itemize}
\end{definition}

\begin{concept}{Zustände eines Circuit Breakers}\\
    Ein Circuit Breaker hat typischerweise drei Zustände:
    \begin{itemize}
        \item \textbf{Geschlossen (Closed):} Aufrufe werden normal durchgeführt, Fehler werden gezählt
        \item \textbf{Offen (Open):} Aufrufe werden sofort abgewiesen, ohne den externen Service zu kontaktieren
        \item \textbf{Halb-offen (Half-Open):} Einige Testaufrufe werden durchgelassen, um zu prüfen, ob der Dienst wieder verfügbar ist
    \end{itemize}
\end{concept}

\begin{definition}{Bulkhead Pattern}\\
    Das Bulkhead Pattern isoliert Elemente eines Systems, sodass der Ausfall eines Elements nicht zum Ausfall des gesamten Systems führt:
    \begin{itemize}
        \item Benannt nach den wasserdichten Schotten in Schiffen
        \item Ressourcen werden in isolierte Pools aufgeteilt
        \item Jeder Dienst/Komponente erhält eigene Ressourcen (z.B. Threadpools, Verbindungspools)
        \item Verhindert Ressourcenerschöpfung des Gesamtsystems
        \item Begrenzt die Auswirkungen von Fehlern auf einzelne Komponenten
    \end{itemize}
\end{definition}

\begin{definition}{Retry Pattern}\\
    Das Retry Pattern ermöglicht einer Anwendung, vorübergehende Fehler zu behandeln, indem Operationen automatisch wiederholt werden:
    \begin{itemize}
        \item Geeignet für transiente Fehler (z.B. Netzwerkunterbrechungen)
        \item Verschiedene Retry-Strategien möglich:
        \begin{itemize}
            \item Sofortiges Retry
            \item Exponentielles Backoff (zunehmende Wartezeiten)
            \item Jitter (zufällige Variationen der Wartezeiten)
        \end{itemize}
        \item Begrenzte Anzahl von Wiederholungen, um endlose Schleifen zu vermeiden
    \end{itemize}
\end{definition}

\begin{concept}{Kombination der Patterns}\\
    Diese Resilience Patterns werden oft kombiniert:
    \begin{itemize}
        \item Circuit Breaker schützt vor anhaltenden Fehlern
        \item Bulkhead isoliert Auswirkungen von Fehlern
        \item Retry behandelt vorübergehende Fehler
        \item Zusammen bilden sie eine umfassende Resilienzstrategie
        \item Weitere ergänzende Patterns: Timeout, Fallback, Cache
    \end{itemize}
\end{concept}

\begin{examplecode}{Circuit Breaker Implementierung mit Resilience4j}\\
\begin{lstlisting}[language=Java, style=basesmol]
// Circuit Breaker Konfiguration
CircuitBreakerConfig config = CircuitBreakerConfig.custom()
    .failureRateThreshold(50)     // 50% Fehlerrate oeffnet den Circuit Breaker
    .waitDurationInOpenState(Duration.ofMillis(1000))  // 1s im Open-State
    .ringBufferSizeInHalfOpenState(10)  // Anzahl Aufrufe im Half-Open State
    .ringBufferSizeInClosedState(100)   // Anzahl Aufrufe im Closed State
    .build();

// Circuit Breaker erstellen
CircuitBreaker circuitBreaker = CircuitBreaker.of("backendService", config);

// Funktion mit Circuit Breaker absichern
Supplier<String> decoratedSupplier = CircuitBreaker
    .decorateSupplier(circuitBreaker, () -> backendService.doSomething());

// Aufruf der geschuetzten Funktion
try {
    String result = decoratedSupplier.get();
} catch (Exception e) {
    // Handle exception (Circuit open or service error)
}
\end{lstlisting}
\end{examplecode}

\subsection{Vergleich von Architekturstilen}

\begin{concept}{Monolith}\\
    Eine Monolith-Architektur ist eine traditionelle Softwarearchitektur, bei der alle Komponenten in einer einzigen Anwendung integriert sind:
    \begin{itemize}
        \item \textbf{Vorteile:}
        \begin{itemize}
            \item Einfachere Entwicklung und Deployment
            \item Weniger Komplexität bei Transaktionen und Konsistenz
            \item Einfacheres Debugging und Testen
            \item Geringere Latenz bei internen Aufrufen
        \end{itemize}
        
        \item \textbf{Nachteile:}
        \begin{itemize}
            \item Schwierigere Skalierung einzelner Komponenten
            \item Langsamere Entwicklungszyklen bei großen Anwendungen
            \item Höhere Kopplung zwischen Komponenten
            \item Technologieabhängigkeit (Plattform, Sprache)
        \end{itemize}
    \end{itemize}
\end{concept}

\begin{concept}{Modulith}\\
    Ein Modulith ist ein gut strukturierter Monolith mit internen Modulen:
    \begin{itemize}
        \item \textbf{Vorteile:}
        \begin{itemize}
            \item Behält Einfachheit des Monoliths bei
            \item Verbesserte Struktur und Wartbarkeit
            \item Klarere Domänengrenzen innerhalb der Anwendung
            \item Kann später leichter in Microservices aufgeteilt werden
        \end{itemize}
        
        \item \textbf{Nachteile:}
        \begin{itemize}
            \item Weiterhin Einschränkungen bei unabhängiger Skalierbarkeit
            \item Gemeinsame Deployment-Einheit kann weiterhin problematisch sein
            \item Erfordert Disziplin zur Einhaltung der Modulgrenzen
        \end{itemize}
    \end{itemize}
\end{concept}

\begin{concept}{Microservices}\\
    Microservices sind eine Architektur, bei der eine Anwendung als Sammlung kleiner, unabhängiger Dienste entwickelt wird:
    \begin{itemize}
        \item \textbf{Vorteile:}
        \begin{itemize}
            \item Unabhängige Entwicklung und Deployment
            \item Bessere Skalierbarkeit einzelner Komponenten
            \item Technologische Flexibilität
            \item Resilienz durch Isolation von Fehlern
            \item Ermöglicht Conway's Law positiv zu nutzen
        \end{itemize}
        
        \item \textbf{Nachteile:}
        \begin{itemize}
            \item Erhöhte Komplexität der Infrastruktur
            \item Herausforderungen bei verteilten Transaktionen
            \item Potential für erhöhte Latenz
            \item Anspruchsvolleres Monitoring und Debugging
            \item Höhere Betriebskosten
        \end{itemize}
    \end{itemize}
\end{concept}

\begin{concept}{Self-contained Systems (SCS)}\\
    Self-contained Systems sind unabhängige Systeme, die zusammen eine größere Anwendung bilden:
    \begin{itemize}
        \item \textbf{Vorteile:}
        \begin{itemize}
            \item Autonomie einzelner Teams
            \item Gröber als Microservices, daher weniger Overhead
            \item Reduzierte Abhängigkeiten zwischen Teams
            \item Gute Balance zwischen Monolith und Microservices
        \end{itemize}
        
        \item \textbf{Nachteile:}
        \begin{itemize}
            \item Mögliche Duplizierung von Funktionalität
            \item Herausforderungen bei systemübergreifender Konsistenz
            \item Komplexere Benutzeroberflächen-Integration
        \end{itemize}
    \end{itemize}
\end{concept}

\begin{concept}{Serverless}\\
    Serverless ist ein Cloud-Computing-Modell, bei dem der Cloud-Anbieter die Infrastruktur dynamisch verwaltet:
    \begin{itemize}
        \item \textbf{Vorteile:}
        \begin{itemize}
            \item Keine Server-Verwaltung notwendig
            \item Automatische Skalierung
            \item Pay-per-Use-Preismodell
            \item Schnelle Markteinführung
            \item Fokus auf Geschäftslogik statt Infrastruktur
        \end{itemize}
        
        \item \textbf{Nachteile:}
        \begin{itemize}
            \item Cold-Start-Problematik (Latenz)
            \item Vendor Lock-in
            \item Beschränkungen bei Laufzeit und Ressourcen
            \item Schwierigere Fehlersuche und Testing
            \item Komplexeres Monitoring
        \end{itemize}
    \end{itemize}
\end{concept}

\begin{formula}{Wahl der richtigen Architektur}\\
    Folgende Faktoren sollten bei der Wahl der Architektur berücksichtigt werden:
    \begin{itemize}
        \item \textbf{Organisationsstruktur:} Conway's Law beachten
        \item \textbf{Teamgröße:} Kleine Teams - Monolith/Modulith, große Teams - Microservices/SCS
        \item \textbf{Veränderungsrate:} Hohe Änderungsrate begünstigt Microservices
        \item \textbf{Domänenkomplexität:} Komplexe Domänen erfordern klare Grenzen
        \item \textbf{Skalierungsbedarf:} Unterschiedliche Skalierungsanforderungen begünstigen Microservices
        \item \textbf{Konsistenzanforderungen:} Hohe Konsistenz begünstigt Monolith
        \item \textbf{Technologie-Diversität:} Bedarf an verschiedenen Technologien begünstigt Microservices
        \item \textbf{Betriebsreife:} DevOps-Fähigkeiten sind Voraussetzung für Microservices
    \end{itemize}
\end{formula}