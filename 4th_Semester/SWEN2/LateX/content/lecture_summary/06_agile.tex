\section{Software Craft und Pyramid of Agile Competencies}

\subsection{Software Craft(smanship)}

\begin{definition}{Software Craftsmanship}\\
    Software Craftsmanship ist eine Bewegung in der Softwareentwicklung, die sich mit der Arbeitsweise und Wahrnehmung des Berufsstandes der Softwareentwickler beschäftigt:
    \begin{itemize}
        \item Ziel ist es, Softwareentwicklung als eigenständige Profession und als Handwerk (nicht nur als Ingenieursdisziplin) wahrzunehmen
        \item Fokus auf Professionalität, kontinuierliches Lernen und Qualität
        \item Wertschätzung für technische Exzellenz und Clean Code
        \item Reaktion auf die Vernachlässigung technischer Praktiken im agilen Umfeld
    \end{itemize}
\end{definition}

\begin{concept}{Die Entstehung der Software Craftsmanship Bewegung}\\
    \begin{itemize}
        \item 2001: Agiles Manifest fokussiert stark auf den Projektprozess, weniger auf technische Aspekte
        \item 2008: Software Craftsmanship Summit mit Micah Martin
        \item 2009: Entstehung des Software Craftsmanship Manifesto
        \item 2009: Erste Software Craftsmanship Konferenzen in USA und UK
        \item Ab 2010: Gründung lokaler Communities weltweit
        \item Heute: SoCraTes (Software Craftsmanship and Testing) Konferenzen in vielen Ländern
    \end{itemize}
\end{concept}

\begin{definition}{Manifesto for Software Craftsmanship}\\
    Als Ergänzung zum Agilen Manifest betont das Software Craftsmanship Manifesto:
    \begin{itemize}
        \item Nicht nur funktionierende Software, sondern auch \textbf{handwerklich gut gemachte Software}
        \item Nicht nur auf Veränderung reagieren, sondern auch \textbf{kontinuierlich Wert hinzufügen}
        \item Nicht nur Individuen und Interaktionen, sondern auch eine \textbf{Gemeinschaft von Profis}
        \item Nicht nur Zusammenarbeit mit dem Kunden, sondern auch \textbf{produktive Partnerschaften}
    \end{itemize}
\end{definition}

\begin{concept}{Motto der Craftsmanship-Bewegung}\\
    "Wir sind es leid, Mist zu schreiben."
    \begin{itemize}
        \item Wir werden keine Unordnung machen, um einen Zeitplan einzuhalten
        \item Wir werden die dumme alte Lüge vom späteren Aufräumen nicht akzeptieren
        \item Wir werden nicht die Behauptung glauben, dass schnell "not-clean" bedeutet
        \item Wir werden die Option, es falsch zu machen, nicht akzeptieren
        \item Wir werden nicht zulassen, dass man uns zwingt, unprofessionell zu handeln
    \end{itemize}
\end{concept}

\subsection{Üben wie ein Software Crafter}

\begin{concept}{Craftsmanship Prinzipien}\\
    Die vier Kernprinzipien der Software Craftsmanship:
    \begin{itemize}
        \item \textbf{Individuals \& Interactions:} Voneinander lernen
        \item \textbf{Clean Code:} Codequalität und Lesbarkeit priorisieren
        \item \textbf{Lifelong Learning:} Kontinuierliche Weiterbildung
        \item \textbf{Continuous Improvement:} Ständige Verbesserung durch Übung
    \end{itemize}
\end{concept}

\begin{concept}{Möglichkeiten zum Üben}\\
    Software Crafter nutzen verschiedene Formate zum Üben und Verbessern ihrer Fähigkeiten:
    \begin{itemize}
        \item Code Katas
        \item Coding Dojos
        \item Code Retreats
        \item Clean Code Developer
        \item Code Koans
        \item Pair Programming
        \item Mob Programming
    \end{itemize}
\end{concept}

\begin{definition}{Coding Dojo}\\
    "Ein Haufen Coder kommt zusammen, programmiert, lernt und hat Spaß" - Emily Bache
    
    Ein Coding Dojo ist eine sichere Übungsumgebung für Entwickler:
    \begin{itemize}
        \item Keine Manager, keine Deadlines
        \item Fokus auf das Lernen, nicht auf das Ergebnis
        \item Spezielle Übungsformen zur Entwicklung schwieriger Fähigkeiten
        \item Gemeinsames Programmieren und Reflektieren
    \end{itemize}
\end{definition}

\begin{concept}{Dojo-Prinzipien}\\
    \begin{itemize}
        \item Design kann nicht ohne Code besprochen werden, Code kann nicht ohne Tests gezeigt werden
        \item Mit eigenen Erfahrungen kommen
        \item Bereitschaft, neu zu lernen
        \item Entschleunigen
        \item Sich auf die Suche nach einem Meister machen
        \item Sich einem Meister unterwerfen
        \item Einen Untergebenen anleiten
    \end{itemize}
\end{concept}

\begin{definition}{Code Kata}\\
    \begin{itemize}
        \item "Kata" ist ein japanisches Wort für ein detailliert choreografiertes Bewegungsmuster
        \item In der Softwareentwicklung: Eine definierte Programmieraufgabe, die wiederholt wird, um bestimmte Fähigkeiten zu üben
        \item Dave Thomas: "Entwickler sollten immer wieder an kleinen, nicht berufsbezogenen Code-Basen üben, damit sie ihren Beruf wie Musiker beherrschen"
    \end{itemize}
\end{definition}

\begin{concept}{Kata-Typen}\\
    \begin{itemize}
        \item \textbf{Function Katas:} Einfache Algorithmen (FizzBuzz, Roman Numbers)
        \item \textbf{Class Katas:} Objektdesign (Bowling, Stack)
        \item \textbf{Library Katas:} Programmieren gegen Schnittstellen (App Login)
        \item \textbf{Application Katas:} Komplette Anwendungen (Tic Tac Toe)
        \item \textbf{Architecture Katas:} Architekturentwürfe (URL Shortener)
        \item \textbf{Agile Katas:} Übungen für agile Praktiken
    \end{itemize}
\end{concept}

\begin{examplecode}{FizzBuzz Kata}\\
\begin{lstlisting}[language=Java, style=basesmol]
/* FizzBuzz Kata:
 * Schreibe ein Programm, das die Zahlen von 1 bis 100 ausgibt, aber:
 * - fuer Vielfache von 3 gibt es "Fizz" aus
 * - fuer Vielfache von 5 gibt es "Buzz" aus
 * - fuer Vielfache von 3 und 5 gibt es "FizzBuzz" aus
 */
public class FizzBuzz {
    public static void main(String[] args) {
        for (int i = 1; i <= 100; i++) {
            if (i % 3 == 0 && i % 5 == 0) {
                System.out.println("FizzBuzz");
            } else if (i % 3 == 0) {
                System.out.println("Fizz");
            } else if (i % 5 == 0) {
                System.out.println("Buzz");
            } else {
                System.out.println(i);
            }
        }
    }
}
\end{lstlisting}
\end{examplecode}

\begin{concept}{Merkmale einer Code-Kata}\\
    \begin{itemize}
        \item \textbf{Definition:} Ein definierter Lösungsweg einer Code-Übung, der viele Male wiederholt wird
        \item \textbf{Dauer:} Meist kurz (30-60 Minuten), um regelmäßiges Üben zu ermöglichen
        \item \textbf{Fokus:} Nicht die richtige Antwort zu finden, sondern der Lernprozess
        \item \textbf{TDD:} Test-Driven Development wird meist als Standard-Pattern verwendet
        \item \textbf{Wiederholung:} Die gleiche Übung wird mehrfach durchgeführt, um bei jeder Wiederholung kleine Verbesserungen zu erzielen
    \end{itemize}
\end{concept}

\subsection{Pyramid of Agile Competencies}

\begin{definition}{Pyramid of Agile Competencies}\\
    Die Pyramid of Agile Competencies ist ein didaktisches Konzept, das die für agile Softwareentwicklung erforderlichen Kompetenzen auf drei Ebenen darstellt:
    \begin{itemize}
        \item \textbf{Agile Values} (Basis der Pyramide)
        \item \textbf{Collaboration Practices} (mittlere Ebene)
        \item \textbf{Technical Practices} (Spitze der Pyramide)
    \end{itemize}
    Alle drei Ebenen sind für erfolgreiche agile Teams notwendig.
\end{definition}

\begin{concept}{Agile Values}\\
    Die Basis der Pyramide bilden gemeinsame Werte und Einstellungen:
    \begin{itemize}
        \item \textbf{Transparenz und Offenheit:} Werden in der agilen Softwareentwicklung großgeschrieben, um sowohl die Organisation als auch den Kunden über Fortschritte zu informieren und schnelles Feedback zu erhalten
        
        \item \textbf{Organizational Culture:} Es gibt drei Möglichkeiten:
        \begin{itemize}
            \item Agiles Team, agile Organisation und agiles Unternehmen
            \item Agiles Team und agile Organisation, nicht-agiles Unternehmen
            \item Agiles Team, nicht-agile Organisation und nicht-agiles Unternehmen
        \end{itemize}
        
        \item \textbf{Software Craftsmanship:} Mehr als nur Technik - eine Einstellung und Haltung zur Softwareentwicklung
    \end{itemize}
\end{concept}

\begin{concept}{Collaboration Practices}\\
    Die mittlere Ebene umfasst Praktiken zur Zusammenarbeit:
    \begin{itemize}
        \item \textbf{Customer and Requirements:} "Intensive und häufige Kommunikation mit dem Kunden ist von größter Bedeutung"
        
        \item \textbf{Agile Champion:} Eine Person, die Agilität im Team fördert und vorantreibt
        \begin{itemize}
            \item Führt und inspiriert Agilität
            \item Hilft zu definieren, welche Veränderungen notwendig sind
            \item Überzeugt andere, die Veränderung zu unterstützen
            \item Hilft zu zeigen, dass Veränderung stattfindet und gute Ergebnisse liefert
            \item Verhindert "Cowboy-Agile" und Rückfälle zu früheren Ansätzen
        \end{itemize}
        
        \item \textbf{Collaboration and Communication:} Intensive und offene Kommunikation zwischen allen Beteiligten ist ein Schlüsselelement für erfolgreiche agile Projekte
        \begin{itemize}
            \item Kommunikation zwischen Teammitgliedern
            \item Kommunikation zwischen Team und Kunde/Endnutzer
            \item Kommunikation zwischen Team und Management
        \end{itemize}
    \end{itemize}
\end{concept}

\begin{concept}{Technical Practices}\\
    Die Spitze der Pyramide bilden technische Praktiken:
    \begin{itemize}
        \item \textbf{Testing:} Automatisierte Tests auf Unit-Ebene sind gut etabliert und werden als absolutes Muss für eine gute Softwarequalität angesehen
        
        \item \textbf{Continuous Integration:} Wird als absolutes Muss angesehen, um Software mit hoher Frequenz liefern zu können
        
        \item \textbf{Clean Code:} Kontinuierliche Aufmerksamkeit für guten Code von Anfang an wird als immer wichtiger angesehen
    \end{itemize}
\end{concept}

\begin{KR}{Integration der Pyramid of Agile Competencies im Team}\\
    \paragraph{Agile Values etablieren}
    \begin{itemize}
        \item Transparente Kommunikation fördern
        \item Respektvolle Feedback-Kultur aufbauen
        \item Kontinuierliches Lernen als Wert verankern
        \item Gemeinsame Verantwortung für Qualität entwickeln
    \end{itemize}
    
    \paragraph{Collaboration Practices implementieren}
    \begin{itemize}
        \item Regelmäßige Kundenkommunikation strukturieren
        \item Cross-funktionale Zusammenarbeit fördern
        \item Agile Champions identifizieren und unterstützen
        \item Effektive Kommunikationskanäle etablieren
    \end{itemize}
    
    \paragraph{Technical Practices stärken}
    \begin{itemize}
        \item Test-Driven Development als Standard einführen
        \item Continuous Integration und Delivery automatisieren
        \item Code Reviews und Pair Programming etablieren
        \item Regelmäßige Refactoring-Sessions durchführen
        \item Technische Schulden aktiv managen
    \end{itemize}
\end{KR}

\subsection{Agile Software-Entwicklung in der Praxis}

\begin{concept}{Moderne Unternehmen und Software Craft}\\
    Unternehmen, die Software Craft ernst nehmen:
    \begin{itemize}
        \item Bieten Coaching on the Job
        \item Organisieren interne Un-Konferenzen
        \item Richten Coding Dojos oder Code Retreats aus
        \item Unterstützen Lernen mit eigenem Budget (für Bücher, Konferenzen, Schulungen)
        \item Fördern die Software-Craft-Bewegung
        \item Ermöglichen Zeit für technische Verbesserungen und Refactoring
    \end{itemize}
\end{concept}

\begin{concept}{Community of Practice}\\
    Vorteile der Teilnahme an einer Software Craft Community:
    \begin{itemize}
        \item Miteinander und voneinander lernen
        \item Außenperspektive zu eigenen Projekten erhalten
        \item Fragen stellen und Ideen herausfordern lassen
        \item Gleichgesinnte mit ähnlicher Leidenschaft treffen
        \item Berufliche Weiterentwicklung
    \end{itemize}
\end{concept}

\begin{concept}{Clean Code als Grundlage}\\
    Clean Code basiert auf den Ideen von Robert C. Martin und ist ein wesentlicher Bestandteil von Software Craftsmanship:
    \begin{itemize}
        \item Verständlicher, lesbarer Code
        \item Einfache Designprinzipien
        \item Kontinuierliche Verbesserung durch Refactoring
        \item Sinnvolle Benennung von Variablen, Methoden und Klassen
        \item Kleine, fokussierte Funktionen mit einer Verantwortlichkeit
        \item Automatisierte Tests als Dokumentation und Sicherheitsnetz
    \end{itemize}
\end{concept}

\begin{formula}{Integration der drei Ebenen}\\
    Für erfolgreiche agile Teams ist die Balance und Integration aller drei Ebenen der Pyramid of Agile Competencies entscheidend:
    \begin{itemize}
        \item Agile Values ohne technische Praktiken führen zu schlechter Codequalität
        \item Technical Practices ohne Collaboration Practices führen zu Lösungen, die nicht den Kundenbedürfnissen entsprechen
        \item Collaboration Practices ohne Agile Values führen zu Oberflächlichkeit und mangelnder Nachhaltigkeit
        \item Nur durch die Kombination aller drei Ebenen entsteht echte Agilität
    \end{itemize}
\end{formula}