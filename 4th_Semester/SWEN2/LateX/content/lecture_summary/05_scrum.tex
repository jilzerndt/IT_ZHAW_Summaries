\section{Scrum}

\subsection{Einführung in Scrum}

\begin{definition}{Was ist Scrum?}\\
    Scrum ist ein agiler Prozessrahmen für das Projektmanagement, der sich auf die Lieferung des höchstmöglichen Wertes für das Fach konzentriert:
    \begin{itemize}
        \item Regelmäßige und wiederholte Inspektion des aktuellen Standes lauffähiger Software
        \item Das Fach setzt die Prioritäten, Teams organisieren sich selbst
        \item Alle 2-4 Wochen können alle Beteiligten lauffähige Software begutachten
        \item Iterative und inkrementelle Entwicklung in kurzen Zyklen (Sprints)
    \end{itemize}
\end{definition}

\begin{concept}{Herkunft von Scrum}\\
    \begin{itemize}
        \item Der Begriff "Scrum" stammt aus dem Rugby: Ein strukturiertes Gedränge, bei dem die Spieler eng zusammenarbeiten, um den Ball zu kontrollieren
        \item Erstmals 1986 von Hirotaka Takeuchi und Ikujiro Nonaka als Produktentwicklungsansatz im Harvard Business Review beschrieben
        \item 1993: Jeff Sutherland entwickelte die ersten Scrum-Ansätze bei Easel Corporation
        \item 1995: Ken Schwaber formalisierte den Prozess
        \item 1996: Gemeinsame Präsentation von Sutherland und Schwaber bei der OOPSLA-Konferenz
        \item 2001: Scrum wurde als eine der Methoden im Agilen Manifest verankert
    \end{itemize}
\end{concept}

\begin{concept}{Eigenschaften von Scrum}\\
    \begin{itemize}
        \item Selbstorganisierte Teams
        \item Produktfortschritt in einer Serie von "Sprints" im Monats-Rhythmus
        \item Anforderungen werden als Artefakte in einem "Backlog" erfasst
        \item Keine speziellen Engineering-Praktiken vorgegeben (im Gegensatz zu XP)
        \item Empirische Prozesskontrolle: Inspektion, Adaption und Transparenz
    \end{itemize}
\end{concept}

\subsection{Scrum Rollen}

\begin{definition}{Product Owner}\\
    Der Product Owner ist verantwortlich für die Maximierung des Produktwerts:
    \begin{itemize}
        \item Definiert die Features des Produkts
        \item Entscheidet über Datum und Inhalt eines Releases
        \item Ist verantwortlich für die Profitabilität (ROI) des Produkts
        \item Priorisiert Features basierend auf ihrem Marktwert
        \item Passt Features und Prioritäten bei Bedarf für jeden Sprint an
        \item Akzeptiert Arbeitsresultate oder weist sie zurück
    \end{itemize}
\end{definition}

\begin{definition}{Scrum Master}\\
    Der Scrum Master ist ein Dienstleister für das Team und die Organisation:
    \begin{itemize}
        \item Verantwortlich für die Durchsetzung der Scrum-Werte und -Praktiken
        \item Entfernt Hindernisse für das Team
        \item Sorgt dafür, dass das Team vollständig arbeitsfähig und produktiv ist
        \item Ermöglicht enge Zusammenarbeit zwischen allen Beteiligten
        \item Schirmt das Team gegen äußere Einflüsse ab
        \item Fördert Verbesserungen im Entwicklungsprozess
    \end{itemize}
\end{definition}

\begin{definition}{Entwicklungsteam}\\
    Das Entwicklungsteam ist für die Umsetzung der Produktanforderungen verantwortlich:
    \begin{itemize}
        \item Typischerweise 5-9 Personen
        \item Cross-Functional: Programmierer, Tester, Designer etc.
        \item Ideale Auslastung: 100\% für das Projekt
        \item Selbstorganisierend in Bezug auf die Arbeit
        \item Zusammensetzung sollte nur zwischen den Sprints wechseln
    \end{itemize}
\end{definition}

\begin{KR}{Guter Scrum Master werden}\\
    \paragraph{Verantwortlichkeiten verstehen}
    \begin{itemize}
        \item Scrum-Prozess am Laufen halten
        \item Für Machtbalance zwischen Product Owner, Team und Management sorgen
        \item Das Team schützen und moderieren
        \item Bei der Organisation helfen
        \item Team auf Sprint-Ziele fokussieren
    \end{itemize}
    
    \paragraph{Hilfestellung geben}
    \begin{itemize}
        \item Sprint-Ziele erreichen helfen
        \item Mit Product Owner zusammenarbeiten
        \item Hindernisse beseitigen
        \item Transparenz fördern
        \item Team zu einem High-Performance-Team entwickeln
    \end{itemize}
    
    \paragraph{Selbstorganisation fördern}
    \begin{itemize}
        \item Selbstorganisation ermutigen und schützen
        \item Team auf geschäftsorientierte Entwicklung fokussieren
        \item Teambildung unterstützen durch Nutzen individueller Fähigkeiten
        \item Feedback-Kultur fördern
        \item Zur Selbsthilfe befähigen
    \end{itemize}
\end{KR}

\subsection{Scrum Ereignisse (Events)}

\begin{concept}{Sprint}\\
    Ein Sprint ist ein zeitlich begrenzter Entwicklungszyklus:
    \begin{itemize}
        \item Zeitboxen von typischerweise 2-4 Wochen
        \item Konstante Dauer für einen besseren Rhythmus
        \item Das Produkt wird während eines Sprints designed, codiert und getestet
        \item Keine Änderungen am Sprint-Ziel während des Sprints
        \item Die Länge der Sprints hängt davon ab, wie lange Änderungen zurückgehalten werden können
    \end{itemize}
\end{concept}

\begin{definition}{Sprint Planning}\\
    Im Sprint Planning wird festgelegt, was im kommenden Sprint umgesetzt werden soll:
    \begin{itemize}
        \item Das Team wählt Items aus dem Product Backlog aus, die es innerhalb eines Sprints umsetzen kann
        \item Ein Sprint Backlog wird definiert
        \item Tasks werden identifiziert und geschätzt (typischerweise 1-16 Stunden)
        \item Ein Sprint Goal wird formuliert - ein kurzes Statement zur Zielformulierung
        \item Die Planung erfolgt gemeinschaftlich, nicht allein durch den Scrum Master
    \end{itemize}
\end{definition}

\begin{definition}{Daily Scrum}\\
    Das Daily Scrum ist ein tägliches 15-minütiges Meeting des Entwicklungsteams:
    \begin{itemize}
        \item Täglich zur gleichen Zeit, am selben Ort
        \item Stehend durchgeführt (Stand-up)
        \item Jedes Teammitglied beantwortet drei Fragen:
        \begin{enumerate}
            \item Was habe ich gestern getan?
            \item Was werde ich heute tun?
            \item Gibt es Hindernisse?
        \end{enumerate}
        \item Keine Problemlösungen während des Meetings
        \item Alle können teilnehmen, aber nur das Team, der Scrum Master und der Product Owner dürfen sprechen
    \end{itemize}
\end{definition}

\begin{definition}{Sprint Review}\\
    Im Sprint Review wird das Inkrement präsentiert und Feedback eingeholt:
    \begin{itemize}
        \item Das Team präsentiert die Ergebnisse des Sprints als Demo
        \item Informell, keine PowerPoint-Präsentationen
        \item Maximal 2 Stunden Vorbereitungszeit
        \item Alle Stakeholder nehmen teil
        \item Feedback fließt in die Planung des nächsten Sprints ein
        \item Bei einem 4-Wochen-Sprint dauert das Review etwa 4 Stunden
    \end{itemize}
\end{definition}

\begin{definition}{Sprint Retrospektive}\\
    In der Sprint Retrospektive reflektiert das Team über seine Arbeitsweise:
    \begin{itemize}
        \item Periodische Prüfung dessen, was gut läuft und was nicht
        \item Typischerweise 15-30 Minuten nach dem Sprint Review
        \item Alle nehmen teil: Scrum Master, Product Owner, Team
        \item Fokus auf kontinuierliche Verbesserung des Prozesses
        \item Konkrete Verbesserungsmaßnahmen für den nächsten Sprint werden identifiziert
    \end{itemize}
\end{definition}

\begin{example}
    Eine typische Sprint Retrospektive verwendet das "Start-Stop-Continue"-Format:
    \begin{itemize}
        \item \textbf{Start:} Was sollten wir beginnen zu tun?
        \item \textbf{Stop:} Was sollten wir aufhören zu tun?
        \item \textbf{Continue:} Was sollten wir weiterhin tun?
    \end{itemize}
    Das Team sammelt Punkte zu jeder Kategorie und entscheidet gemeinsam, welche Maßnahmen im nächsten Sprint umgesetzt werden sollen.
\end{example}

\subsection{Scrum Artefakte}

\begin{definition}{Product Backlog}\\
    Das Product Backlog ist eine priorisierte Liste aller gewünschten Produktfunktionen:
    \begin{itemize}
        \item Enthält alle Anforderungen und auszuführenden Arbeiten im Projekt
        \item Idealerweise so formuliert, dass der Wertbeitrag für den Benutzer erkennbar ist
        \item Wird vom Product Owner priorisiert
        \item Ist dynamisch und wird kontinuierlich weiterentwickelt
        \item Wird zu Beginn jedes Sprints neu priorisiert
    \end{itemize}
\end{definition}

\begin{definition}{Sprint Backlog}\\
    Das Sprint Backlog enthält die für den aktuellen Sprint ausgewählten Backlog-Items:
    \begin{itemize}
        \item Individuen wählen ihre Arbeit selbst aus (keine Zuweisung)
        \item Die Schätzung des verbleibenden Aufwands wird täglich aktualisiert
        \item Jedes Teammitglied kann das Sprint Backlog anpassen
        \item Bei unklaren Anforderungen werden größere Backlog-Items für die spätere Klärung geschaffen
    \end{itemize}
\end{definition}

\begin{definition}{Inkrement}\\
    Das Inkrement ist die Summe aller Product Backlog-Items, die während eines Sprints fertiggestellt wurden:
    \begin{itemize}
        \item Muss "Done" sein, d.h. den Akzeptanzkriterien entsprechen
        \item Muss eine potentiell auslieferbare Version des Produkts darstellen
        \item Wird im Sprint Review präsentiert
    \end{itemize}
\end{definition}

\begin{concept}{Task Board}\\
    Das Task Board visualisiert den aktuellen Stand aller Aufgaben im Sprint:
    \begin{itemize}
        \item Typische Spalten: "To Do", "In Progress", "Done"
        \item Jeder Task wird als Karte dargestellt
        \item Teammitglieder verschieben ihre Tasks je nach Fortschritt
        \item Bietet Transparenz über den aktuellen Sprint-Status
        \item Zeigt mögliche Blockaden und Engpässe auf
    \end{itemize}
\end{concept}

\begin{concept}{Sprint Burndown Chart}\\
    Das Sprint Burndown Chart zeigt den verbleibenden Aufwand im Sprint:
    \begin{itemize}
        \item X-Achse: Tage im Sprint
        \item Y-Achse: Verbleibender Aufwand (in Stunden oder Story Points)
        \item Ideale Linie zeigt gleichmäßigen Fortschritt
        \item Tatsächliche Linie zeigt realen Fortschritt
        \item Warnsignale: Flache Linie (kein Fortschritt) oder steigende Linie (mehr Arbeit entdeckt)
    \end{itemize}
\end{concept}

\subsection{Definition of Done und Ready}

\begin{definition}{Definition of Done (DoD)}\\
    Die Definition of Done legt fest, wann eine User Story als abgeschlossen gilt:
    \begin{itemize}
        \item Gegenseitig akzeptiertes Übereinkommen aller Beteiligten
        \item Konform mit den Governance-Vorgaben der Organisation
        \item Wird für User Stories und Sprints definiert
        \item Verhindert technische Schulden und Illusionen über den Projektfortschritt
        \item Wird vom gesamten Team inkl. Product Owner definiert
        \item Wird vor jedem Sprint überprüft und bei Bedarf angepasst
    \end{itemize}
\end{definition}

\begin{example}
    Typische Definition of Done für eine User Story:
    \begin{itemize}
        \item Unit-Tests bestehen mit mindestens 85\% Abdeckung
        \item Ausreichend negative Tests wurden geschrieben
        \item Code wurde überprüft (oder als Pair Programming erstellt)
        \item Coding-Standards sind erfüllt
        \item CI/CD ist implementiert
        \item Code wurde refaktoriert
        \item UAT-Tests werden bestanden
        \item Nicht-funktionale Tests werden bestanden
        \item Erforderliche Dokumentation ist fertiggestellt
    \end{itemize}
\end{example}

\begin{definition}{Definition of Ready}\\
    Die Definition of Ready legt fest, wann eine User Story bereit für die Aufnahme in einen Sprint ist:
    \begin{itemize}
        \item Die User Story ist klar definiert
        \item Akzeptanzkriterien sind festgelegt
        \item Abhängigkeiten sind identifiziert
        \item Die Schätzung durch das Team liegt vor
        \item UX-Artefakte wurden akzeptiert
        \item Nicht-funktionale Anforderungen sind definiert
        \item Die abnehmende Person ist benannt
        \item Das Team hat eine klare Vorstellung davon, was zu zeigen ist
    \end{itemize}
\end{definition}

\subsection{Skalierung von Scrum}

\begin{concept}{Scrum of Scrums}\\
    Bei größeren Projekten mit mehreren Scrum-Teams wird Scrum of Scrums eingesetzt:
    \begin{itemize}
        \item Vertreter jedes Teams treffen sich regelmäßig
        \item Koordination und Abstimmung zwischen Teams
        \item Identifikation von teamübergreifenden Abhängigkeiten
        \item Lösung von Konflikten und Blockaden
    \end{itemize}
\end{concept}

\begin{concept}{Scaled Agile Framework (SAFe)}\\
    SAFe ist ein Framework für die Skalierung agiler Methoden auf Unternehmensebene:
    \begin{itemize}
        \item Verschiedene Konfigurationen je nach Unternehmensgröße
        \item Integriert Scrum mit Lean und DevOps
        \item Organisiert Teams in "Agile Release Trains" (ARTs)
        \item Definiert zusätzliche Rollen und Events für die Koordination
        \item Berücksichtigt Portfoliomanagement und Unternehmensstrategie
    \end{itemize}
\end{concept}

\subsection{Scrum-Werte}

\begin{concept}{Die fünf Scrum-Werte}\\
    \begin{itemize}
        \item \textbf{Commitment:} Das Team verpflichtet sich, seine Ziele zu erreichen
        \item \textbf{Fokus:} Konzentration auf die Sprint-Arbeit und die Scrum-Ziele
        \item \textbf{Offenheit:} Transparenz über Arbeit und Herausforderungen
        \item \textbf{Respekt:} Gegenseitiger Respekt für die Fähigkeiten und Unabhängigkeit
        \item \textbf{Mut:} Mut, das Richtige zu tun und schwierige Probleme anzugehen
    \end{itemize}
\end{concept}

\begin{formula}{Scrum-Erfolgsformel}\\
    Für erfolgreiche Scrum-Implementierungen:
    \begin{itemize}
        \item Transparenz: Alle relevanten Aspekte des Prozesses müssen für alle Beteiligten sichtbar sein
        \item Inspektion: Regelmäßige Überprüfung von Artefakten und Fortschritt
        \item Anpassung: Prozess oder Material bei Abweichungen anpassen
        \item Selbstorganisation: Teams entscheiden selbst, wie sie Arbeit erledigen
        \item Timeboxing: Strikte Zeitbegrenzungen für alle Aktivitäten
        \item Inkrementelle Lieferung: Regelmäßige Bereitstellung von Produktinkrementen
    \end{itemize}
\end{formula}