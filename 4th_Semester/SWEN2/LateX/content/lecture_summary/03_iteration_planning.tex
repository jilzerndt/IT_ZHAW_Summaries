\section{Agile Planung und Schätzung}

\subsection{Der Planungsprozess}

\begin{concept}{Der Sinn der Planung}\\
    Schätzung und Planung sind kritisch für den Erfolg eines Softwareprojekts, auch in agilen Methoden:
    \begin{itemize}
        \item Hilft bei Ressourcenallokation: "Wer arbeitet wieviel während welcher Zeit im Projekt?"
        \item Ermöglicht Fortschrittskontrolle: "Ist das Projekt auf dem richtigen Weg?"
        \item Unterstützt bei der Terminplanung: "Wann werden wir fertig sein?"
    \end{itemize}
\end{concept}

\begin{concept}{Missverständnisse über agile Planung}\\
    Häufige Missverständnisse:
    \begin{itemize}
        \item "Agile Teams brauchen keine Planung"
        \item "Wenn wir genug agil sind, brauchen wir keinen Plan, denn wir reagieren ja schnell genug"
    \end{itemize}
    In der Praxis erstellen agile Teams Pläne auf zwei Ebenen:
    \begin{itemize}
        \item Grobe langfristige Planung für die strategische Ausrichtung
        \item Detaillierte kurzfristige Arbeitsplanung für die nächsten Wochen oder Monate
    \end{itemize}
\end{concept}

\begin{concept}{Der Trichter der Unsicherheit}\\
    Projekte beginnen mit hoher Unsicherheit, die im Laufe der Zeit abnimmt. Der "Trichter der Unsicherheit" zeigt, wie sich die Genauigkeit von Schätzungen mit fortschreitendem Projektfortschritt verbessert:
    \begin{itemize}
        \item Zu Projektbeginn: Schätzungen können um +/- 400\% abweichen
        \item Nach detaillierter Anforderungsanalyse: +/- 50\%
        \item Nach Design: +/- 25\%
        \item Während der Implementierung: +/- 10\%
    \end{itemize}
\end{concept}

\begin{concept}{Warum Planen trotz Unsicherheit?}\\
    Gründe für Planung trotz der Herausforderungen:
    \begin{itemize}
        \item Organisationen benötigen Schätzungen für Budget, Marketing, Rollout usw.
        \item Planung unterstützt die Suche nach Wert: "Was sollen wir bauen?"
        \item Ein guter iterativer Planungsprozess:
        \begin{itemize}
            \item Reduziert Risiken
            \item Verringert Unsicherheit
            \item Unterstützt bessere Entscheidungsfindung
            \item Schafft Vertrauen
            \item Verbessert die Kommunikation
        \end{itemize}
    \end{itemize}
\end{concept}

\begin{definition}{Agile Planung}\\
    Merkmale der agilen Planung:
    \begin{itemize}
        \item Fokus liegt auf dem Prozess der Planung, nicht auf dem Plan selbst
        \item "Pläne sind Dokumente – Planung ist eine Aktivität"
        \item Agile Pläne ändern sich oft: Während des Projekts lernen wir ständig Neues
        \item Kundenanforderungen können sich ändern
        \item Die Umgebung kann komplexer sein als erwartet
    \end{itemize}
\end{definition}

\subsection{Agile Herangehensweise}

\begin{concept}{Der agile Ansatz für die Planung}\\
    Zentrale Idee:
    \begin{itemize}
        \item Ein Projekt erzeugt neue Fähigkeiten und neues Wissen in schneller Abfolge
        \item Neue Fähigkeiten werden als Produkt geliefert
        \item Neues Wissen ist die Basis, um das Produkt bestmöglich umzusetzen:
        \begin{itemize}
            \item Wissen über das Produkt: Was soll gebaut werden?
            \item Wissen über das Projekt: Team, Technologie, Risiken etc.
        \end{itemize}
    \end{itemize}
\end{concept}

\begin{concept}{Mehrere Planungsebenen}\\
    Agile Teams planen auf mindestens drei Ebenen:
    \begin{itemize}
        \item \textbf{Release-Ebene:} Umfasst mehrere Iterationen (3-9 Monate)
        \item \textbf{Iterations-Ebene:} Typischerweise 2-4 Wochen
        \item \textbf{Tages-Ebene:} Tägliche Planung und Anpassung
    \end{itemize}
    Dies ermöglicht eine Balance zwischen langfristiger Vision und kurzfristiger Anpassungsfähigkeit.
\end{concept}

\begin{concept}{Condition of Satisfaction}\\
    \begin{itemize}
        \item Jedes Projekt wird mit einer gewissen Menge von Zielen initialisiert
        \item Zusätzlich zu den Features gibt es Ziele bezüglich Zeitplan, Budget und Qualität
        \item Diese Ziele stellen für den Kunden oder Product Owner die "Conditions of Satisfaction" dar
        \item Sie bilden den Rahmen für Release- und Iterationsplanung
    \end{itemize}
\end{concept}

\begin{example}
    Typische "Condition of Satisfaction" für eine User Story "Als Benutzer möchte ich eine Reservierung stornieren können":
    \begin{itemize}
        \item Stornierung bis 24h im Voraus führt zu vollständiger Kostenrückerstattung
        \item Bei weniger als 24h Vorankündigung fällt eine Gebühr an
        \item Ein Stornierungscode wird generiert und per E-Mail zugesandt
        \item Die stornierte Reservierung wird im System als storniert markiert
    \end{itemize}
\end{example}

\subsection{User Stories}

\begin{definition}{User Story}\\
    Eine User Story ist eine knappe und präzise Beschreibung eines Funktionalitätselements, das einem Benutzer der Software einen Nutzen stiftet:
    \begin{itemize}
        \item Format: "Als [Benutzerrolle] möchte ich [Ziel], damit ich [Nutzen]"
        \item Beispiel: "Als Verkäufer möchte ich, dass Verkaufschancen Zusatzinformationen zu Produkten beinhalten, damit ich den Nutzen der Produkte dem Kunden erklären und besser Zusatzprodukte verkaufen kann"
    \end{itemize}
\end{definition}

\begin{concept}{User Story Karten}\\
    User Story Karten haben drei Teile:
    \begin{itemize}
        \item \textbf{Card:} Eine schriftliche Beschreibung für Planungszwecke
        \item \textbf{Conversation:} Weitere Informationen und Abstimmungsdetails
        \item \textbf{Confirmation:} Tests, die sicherstellen, dass die Story vollständig ist
    \end{itemize}
\end{concept}

\begin{concept}{User Rollen}\\
    Warum User Rollen wichtig sind:
    \begin{itemize}
        \item Erweitern den Scope — nicht von einem einzigen User ausgehen
        \item Erlauben die Differenzierung von Nutzern nach:
        \begin{itemize}
            \item Hintergrund
            \item Vertrautheit im Umgang
            \item Nutzungsfrequenz
            \item Verwendetem Zielgerät
            \item Nutzungszweck
        \end{itemize}
    \end{itemize}
\end{concept}

\begin{concept}{Eigenschaften guter User Stories (INVEST)}\\
    Gute User Stories sind:
    \begin{itemize}
        \item \textbf{I}ndependent: Abhängigkeiten vermeiden
        \item \textbf{N}egotiable: Verhandelbar zwischen User und Entwicklung
        \item \textbf{V}aluable: Wertvoll für den Kunden oder die Nutzenden
        \item \textbf{E}stimatable: Für die Umsetzung zentral
        \item \textbf{S}mall: Normalerweise ein Satz
        \item \textbf{T}estable: Durch einen Testfall zu prüfen
    \end{itemize}
\end{concept}

\subsection{Story Points und Velocity}

\begin{definition}{Story Points}\\
    \begin{itemize}
        \item Die Anzahl der Story Points bestimmt die relative Größe einer User Story
        \item Es gibt keine definierte Formel, sondern eine relative Messung
        \item Story Points schätzen den Aufwand, der für die Realisierung eines Features nötig ist
        \item Sie berücksichtigen Komplexität, Umfang und Risiko
    \end{itemize}
\end{definition}

\begin{definition}{Velocity}\\
    \begin{itemize}
        \item Velocity ist die Fortschrittsmessung eines Teams
        \item Sie wird berechnet durch die Summe der während einer Iteration umgesetzten Story Points
        \item Die beste Schätzung ist, dass ein Team pro Iteration eine ähnliche Anzahl Story Points realisieren kann
        \item Sie dient als Basis für Release-Planung
    \end{itemize}
\end{definition}

\begin{KR}{Planning Poker durchführen}\\
    \paragraph{Vorbereitung}
    \begin{itemize}
        \item Jeder Schätzer erhält Kartenset mit gültigen Schätzwerten (z.B. Fibonacci: 1, 2, 3, 5, 8, 13, 21)
        \item Product Owner/Kunde bereitet User Stories vor
    \end{itemize}
    
    \paragraph{Ablauf für jede User Story}
    \begin{itemize}
        \item Product Owner liest die Story vor und erläutert sie
        \item Teammitglieder stellen Fragen zur Klärung
        \item Jeder wählt verdeckt eine Karte für seine Schätzung
        \item Alle decken gleichzeitig ihre Karten auf
        \item Bei Unterschieden diskutieren, besonders Ausreißer erklären ihre Einschätzung
        \item Wiederholen der Schätzung bis zur Übereinstimmung oder Annäherung
    \end{itemize}
    
    \paragraph{Tipps}
    \begin{itemize}
        \item Timeboxing einsetzen (5-10 Minuten pro Story)
        \item Bei anhaltender Uneinigkeit die größere Schätzung wählen oder Story aufteilen
        \item "?" Karte für "ich verstehe die Story nicht"
        \item "$\infty$" Karte für "zu groß für eine Iteration"
    \end{itemize}
\end{KR}

\subsection{Release- und Iterationsplanung}

\begin{concept}{Release-Plan}\\
    Der Release-Plan ist ein Prozess, der einen Plan für mehrere Iterationen erstellt (3-9 Monate):
    \begin{itemize}
        \item Was soll wann durch wen gebaut werden
        \item Vor dem Planungsstart müssen die "Conditions of Satisfaction" bekannt sein
        \item Schätzungen für alle User Stories, die im Vertrag enthalten sind
        \item Festlegung der Iterationslänge (meist 2-4 Wochen)
        \item Schätzung der Velocity
        \item Priorisierung der User Stories durch den Product Owner
    \end{itemize}
\end{concept}

\begin{concept}{User Stories auswählen und Releasedatum festlegen}\\
    Nach Schätzungen und Velocity-Ermittlung:
    \begin{itemize}
        \item \textbf{Bei funktionsgetriebenem Projekt:} Summe der Story Points aller gewünschten Features geteilt durch erwartete Velocity = Anzahl Iterationen bis zum Release
        \item \textbf{Bei datumsgetriebenem Projekt:} Anzahl Iterationen bis zum Wunschdatum multipliziert mit erwarteter Velocity = Maximale Story Points für das Release
    \end{itemize}
\end{concept}

\begin{concept}{Release-Plan aktualisieren}\\
    \begin{itemize}
        \item Nach jeder Iteration sollte der Release-Plan aktualisiert werden
        \item Anpassung basierend auf tatsächlicher Velocity, die von der ursprünglichen Schätzung abweichen kann
        \item Bei Problemen sollte der Product Owner mit dem Kunden über mögliche Scope-Reduktion verhandeln
    \end{itemize}
\end{concept}

\begin{concept}{Iterations-Plan}\\
    \begin{itemize}
        \item Detailliertere Sicht auf die Arbeit während einer Iteration
        \item User Stories werden in konkrete Tasks aufgeteilt
        \item Jeder Task wird in Stunden geschätzt
        \item Team wählt selbst aus, welche Tasks es übernimmt
    \end{itemize}
\end{concept}

\subsection{Priorisierung und Tracking}

\begin{concept}{Priorisierung der User Stories}\\
    Faktoren für die Priorisierung:
    \begin{itemize}
        \item \textbf{Finanzieller Wert:} Wie viel Geld wird verdient oder gespart?
        \item \textbf{Kosten:} Aufwand für Entwicklung und Wartung
        \item \textbf{Neues Wissen:} Lernen über Produkt oder Prozess
        \item \textbf{Risikominderung:} Welche Risiken werden adressiert?
    \end{itemize}
\end{concept}

\begin{concept}{Kano-Modell der Kundenzufriedenheit}\\
    Klassifizierung von Features nach ihrem Einfluss auf die Kundenzufriedenheit:
    \begin{itemize}
        \item \textbf{Must-have Features:} Grundlegende Anforderungen, deren Fehlen zu Unzufriedenheit führt
        \item \textbf{Linear Features:} "Je mehr, desto besser" - Zufriedenheit steigt proportional
        \item \textbf{Exciters/Delighters:} Unerwartete Features, die Begeisterung auslösen
    \end{itemize}
\end{concept}

\begin{concept}{Tracking und Kommunikation}\\
    Überwachung des Release-Plans:
    \begin{itemize}
        \item \textbf{Release Burndown Chart:} Zeigt verbleibende Arbeit über Zeit
        \item \textbf{Release Burndown Bar Chart:} Visualisiert Fortschritt und Änderungen
        \item \textbf{Parking-Lot Chart:} Zeigt, wie viel eines "Themes" bereits realisiert wurde
    \end{itemize}
    
    Überwachung des Iteration-Plans:
    \begin{itemize}
        \item \textbf{Task Board:} Visualisierung des aktuellen Status aller Tasks
        \item \textbf{Iteration Burndown Chart:} Tägliche Fortschrittsmessung
    \end{itemize}
\end{concept}

\begin{example}
    Ein Task Board hat typischerweise folgende Spalten:
    \begin{itemize}
        \item To Do: Noch nicht begonnene Tasks
        \item In Progress: Tasks, an denen gerade gearbeitet wird
        \item Testing/Review: Fertige Tasks, die getestet werden
        \item Done: Vollständig abgeschlossene Tasks
    \end{itemize}
    Jeder Task wird als Karte dargestellt und während der Bearbeitung von links nach rechts bewegt.
\end{example}