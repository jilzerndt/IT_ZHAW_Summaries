\section{Analog-to-Digital Converter (ADC)}

\subsection{ADC Fundamentals}

\begin{concept}{ADC Overview}\\
An Analog-to-Digital Converter (ADC) converts continuous analog signals into discrete digital values:
\begin{itemize}
    \item Transforms analog voltage levels into corresponding digital values
    \item Resolution determined by number of bits (N)
    \item $2^N$ possible digital values (e.g., 12-bit ADC has 4096 levels)
    \item Converts real-world continuous signals (temperature, pressure, etc.) into digital form for processing
\end{itemize}
\end{concept}

\begin{definition}{ADC Terminology}\\
Key terms and concepts:
\begin{itemize}
    \item \textbf{Resolution}: Number of bits (N) in the digital output
    \item \textbf{LSB} (Least Significant Bit): Smallest detectable voltage change
    \begin{itemize}
        \item $LSB = \frac{V_{REF}}{2^N}$
    \end{itemize}
    \item \textbf{FSR} (Full Scale Range): Range between minimum and maximum digital codes
    \begin{itemize}
        \item $FSR = V_{REF} - 1 LSB$
    \end{itemize}
    \item \textbf{Sampling Rate}: Number of conversions per second
    \item \textbf{Conversion Time}: Time from start of sampling to digital output availability
    \item \textbf{Reference Voltage} ($V_{REF}$): Voltage that defines the ADC's full-scale range
\end{itemize}
\end{definition}

\begin{concept}{ADC Operating Principle}\\
\textbf{Flash ADC (Parallel ADC)}:
\begin{itemize}
    \item Uses comparators for each voltage level
    \item Input voltage compared to reference voltages simultaneously
    \item Fast but requires many components (e.g., 255 comparators for 8-bit)
\end{itemize}
\textbf{Successive Approximation Register (SAR) ADC}:
\begin{itemize}
    \item Uses binary search algorithm
    \item Compares input to successive approximations
    \item Takes N steps for N-bit resolution
    \item Good balance of speed, power, and complexity
    \item Most common in microcontrollers
\end{itemize}
\end{concept}

\subsection{ADC Errors and Characteristics}

\begin{definition}{Sampling Theorem}\\
According to the Nyquist-Shannon sampling theorem:
\begin{itemize}
    \item Sampling rate must be at least twice the highest frequency component of the input signal
    \item $f_{sampling} \geq 2 \times f_{max}$
    \item Prevents aliasing (false lower frequencies appearing in sampled signal)
\end{itemize}
\end{definition}

\begin{concept}{ADC Error Types}\\
\textbf{Quantization Error}:
\begin{itemize}
    \item Inherent error due to rounding analog values to discrete digital levels
    \item Range between -0.5 LSB and +0.5 LSB
    \item Cannot be eliminated, but reduced by increasing resolution
\end{itemize}
\textbf{Offset Error}:
\begin{itemize}
    \item Deviation from ideal ADC at zero input
    \item For ideal ADC, first transition occurs at 0.5 LSB
    \item Can be corrected through calibration
\end{itemize}
\textbf{Gain Error}:
\begin{itemize}
    \item Difference in slope between actual and ideal transfer function
    \item Expressed in LSB or as percentage of full-scale range (\%FSR)
    \item Can be corrected through calibration
\end{itemize}
\end{concept}

\begin{formula}{ADC Calculations}\\
\textbf{LSB Voltage}:
\begin{align}
V_{LSB} = \frac{V_{REF}}{2^N}
\end{align}

\textbf{Digital Output Value}:
\begin{align}
D_{out} = \frac{V_{in} \times 2^N}{V_{REF}}
\end{align}

\textbf{Analog Input from Digital Value}:
\begin{align}
V_{in} = \frac{D_{out} \times V_{REF}}{2^N}
\end{align}

\textbf{Percent Full Scale Range}:
\begin{align}
\%FSR = \frac{V_{in}}{V_{REF}} \times 100\%
\end{align}
\end{formula}

\subsection{STM32F4 ADC Features}

\begin{concept}{STM32F4 ADC Architecture}\\
The STM32F4 includes ADC modules with the following features:
\begin{itemize}
    \item Three ADCs (ADC1, ADC2, ADC3)
    \item 12-bit resolution (configurable to 10, 8, or 6 bits)
    \item Up to 24 external channels (16 on each ADC)
    \item Internal channels (temperature sensor, V$_{REFINT}$, V$_{BAT}$)
    \item Multiple operating modes:
    \begin{itemize}
        \item Single conversion vs. continuous conversion
        \item Single channel vs. scan mode (multiple channels)
    \end{itemize}
    \item Maximum sampling rate up to 2.4 MSPS (million samples per second)
    \item DMA capability
    \item Configurable sampling time
    \item Analog watchdog for threshold monitoring
\end{itemize}
\end{concept}

\begin{definition}{ADC Conversion Modes}\\
\textbf{Single vs. Continuous Conversion}:
\begin{itemize}
    \item \textbf{Single Conversion}: Performs one conversion, then stops
    \item \textbf{Continuous Conversion}: Continuously performs conversions without CPU intervention
\end{itemize}
\textbf{Single Channel vs. Scan Mode}:
\begin{itemize}
    \item \textbf{Single Channel}: Converts one channel only
    \item \textbf{Scan Mode}: Converts multiple channels in sequence
\end{itemize}
This results in four possible combinations:
\begin{itemize}
    \item Single channel, single conversion (simplest mode)
    \item Single channel, continuous conversion (monitor one input)
    \item Multi-channel, single conversion (read multiple inputs once)
    \item Multi-channel, continuous conversion (monitor multiple inputs)
\end{itemize}
\end{definition}

\subsection{STM32F4 ADC Configuration}

\begin{definition}{ADC Registers}\\
Key ADC registers on STM32F4:
\begin{itemize}
    \item \textbf{ADC\_SR}: Status register (flags for EOC, overrun, etc.)
    \item \textbf{ADC\_CR1}: Control register 1 (scan mode, resolution, etc.)
    \item \textbf{ADC\_CR2}: Control register 2 (conversion start, data alignment, etc.)
    \item \textbf{ADC\_SMPRx}: Sample time registers
    \item \textbf{ADC\_SQRx}: Regular sequence registers
    \item \textbf{ADC\_DR}: Data register (conversion result)
    \item \textbf{ADC\_CCR}: Common control register (for all ADCs)
\end{itemize}
\end{definition}

\begin{KR}{Basic ADC Configuration}
\paragraph{Step 1: Enable GPIO and ADC clocks}
Enable the clock to the GPIO port and ADC.
\paragraph{Step 2: Configure GPIO pins}
Configure the GPIO pins as analog inputs.
\paragraph{Step 3: Configure ADC parameters}
Set resolution, scan mode, conversion mode, and alignment.
\paragraph{Step 4: Configure channel and sampling time}
Set the channel sequence and sampling time.
\paragraph{Step 5: Enable ADC}
Turn on the ADC.

\begin{lstlisting}[language=C, style=basesmol]
// Configure ADC1 Channel 0 (PA0) for single conversion

// Step 1: Enable GPIO and ADC clocks
RCC->AHB1ENR |= RCC_AHB1ENR_GPIOAEN;  // Enable GPIOA clock
RCC->APB2ENR |= RCC_APB2ENR_ADC1EN;   // Enable ADC1 clock

// Step 2: Configure PA0 as analog input
GPIOA->MODER |= GPIO_MODER_MODER0;  // Analog mode (0b11)

// Step 3: Configure ADC parameters
// ADC Common Control Register
ADC->CCR &= ~ADC_CCR_ADCPRE;  // ADCPRE = 0 (APB2/2, typically 42MHz/2 = 21MHz)

// ADC1 Control Register 1
ADC1->CR1 &= ~ADC_CR1_RES;    // 12-bit resolution (default)
ADC1->CR1 &= ~ADC_CR1_SCAN;   // Disable scan mode (single channel)

// ADC1 Control Register 2
ADC1->CR2 &= ~ADC_CR2_CONT;   // Single conversion mode
ADC1->CR2 &= ~ADC_CR2_ALIGN;  // Right alignment
ADC1->CR2 &= ~ADC_CR2_EXTEN;  // Software trigger

// Step 4: Configure channel and sampling time
// Configure for channel 0
ADC1->SQR1 &= ~ADC_SQR1_L;     // 1 conversion in regular sequence
ADC1->SQR3 &= ~ADC_SQR3_SQ1;   // Clear channel selection
ADC1->SQR3 |= (0 << ADC_SQR3_SQ1_Pos);  // Channel 0 as first conversion

// Set sampling time for channel 0 (e.g., 84 cycles)
ADC1->SMPR2 &= ~ADC_SMPR2_SMP0;  // Clear bits
ADC1->SMPR2 |= (4 << ADC_SMPR2_SMP0_Pos);  // 84 cycles

// Step 5: Enable ADC
ADC1->CR2 |= ADC_CR2_ADON;  // Turn on ADC
\end{lstlisting}
\end{KR}

\begin{KR}{Performing ADC Conversion}
\paragraph{Start conversion}
Trigger the conversion using software or external trigger.
\paragraph{Wait for completion}
Check the EOC flag to determine when conversion is complete.
\paragraph{Read result}
Read the data register to get the conversion result.

\begin{lstlisting}[language=C, style=basesmol]
// Function to perform single ADC conversion
uint16_t ADC_ReadChannel(void) {
    // Start conversion
    ADC1->CR2 |= ADC_CR2_SWSTART;  // Software trigger to start conversion
    
    // Wait for conversion to complete
    while (!(ADC1->SR & ADC_SR_EOC)) { }
    
    // Read and return result
    return ADC1->DR;
}

// Function to convert ADC value to voltage
float ADC_ConvertToVoltage(uint16_t adcValue) {
    // Assuming VREF = 3.3V and 12-bit resolution (4096 levels)
    float voltage = (float)adcValue * 3.3f / 4095.0f;
    return voltage;
}
\end{lstlisting}
\end{KR}

\begin{KR}{Multi-Channel ADC Configuration}
\paragraph{Configure scan mode}
Enable scan mode to convert multiple channels.
\paragraph{Set up channel sequence}
Configure the sequence and number of channels.
\paragraph{Process results}
Read results for each channel in sequence.

\begin{lstlisting}[language=C, style=basesmol]
// Configure ADC for multi-channel scanning (channels 0, 1, and 4)

// Enable scan mode
ADC1->CR1 |= ADC_CR1_SCAN;

// Set number of conversions in sequence (3)
ADC1->SQR1 &= ~ADC_SQR1_L;
ADC1->SQR1 |= (2 << ADC_SQR1_L_Pos);  // L = 2 means 3 conversions

// Set channel sequence
ADC1->SQR3 = 0;  // Clear all
ADC1->SQR3 |= (0 << ADC_SQR3_SQ1_Pos);  // CH0 as 1st conversion
ADC1->SQR3 |= (1 << ADC_SQR3_SQ2_Pos);  // CH1 as 2nd conversion
ADC1->SQR3 |= (4 << ADC_SQR3_SQ3_Pos);  // CH4 as 3rd conversion

// Set sampling times for each channel
ADC1->SMPR2 &= ~(ADC_SMPR2_SMP0 | ADC_SMPR2_SMP1 | ADC_SMPR2_SMP4);
ADC1->SMPR2 |= (4 << ADC_SMPR2_SMP0_Pos);  // 84 cycles for CH0
ADC1->SMPR2 |= (4 << ADC_SMPR2_SMP1_Pos);  // 84 cycles for CH1
ADC1->SMPR2 |= (4 << ADC_SMPR2_SMP4_Pos);  // 84 cycles for CH4
\end{lstlisting}
\end{KR}

\begin{concept}{Analog Watchdog}\\
The Analog Watchdog monitors ADC conversion results against programmable thresholds:
\begin{itemize}
    \item Generates an interrupt if a conversion result is outside the threshold range
    \item Can be configured to monitor a single channel or all channels
    \item Useful for detecting abnormal voltage levels without CPU polling
    \item Programmable high and low thresholds
\end{itemize}
Applications:
\begin{itemize}
    \item Over-voltage/under-voltage detection
    \item Temperature limit monitoring
    \item Battery level monitoring
\end{itemize}
\end{concept}

\begin{KR}{Using DMA with ADC}
\paragraph{Configure DMA}
Set up DMA channel to transfer ADC results to memory.
\paragraph{Enable DMA for ADC}
Configure ADC to use DMA for data transfer.
\paragraph{Start conversion}
Start ADC conversion in continuous mode.

\begin{lstlisting}[language=C, style=basesmol]
// Configure ADC with DMA for continuous multi-channel sampling

// Configure DMA
RCC->AHB1ENR |= RCC_AHB1ENR_DMA2EN;  // Enable DMA2 clock

// Configure DMA2 Stream0 for ADC1
DMA2_Stream0->CR &= ~DMA_SxCR_EN;  // Disable DMA stream
while (DMA2_Stream0->CR & DMA_SxCR_EN) { }  // Wait until disabled

DMA2_Stream0->CR = 0;
DMA2_Stream0->CR |= (0 << DMA_SxCR_CHSEL_Pos);  // Channel 0
DMA2_Stream0->CR |= DMA_SxCR_PL_1;  // Priority high
DMA2_Stream0->CR |= DMA_SxCR_MSIZE_0;  // Memory data size: 16-bit
DMA2_Stream0->CR |= DMA_SxCR_PSIZE_0;  // Peripheral data size: 16-bit
DMA2_Stream0->CR |= DMA_SxCR_MINC;  // Memory increment mode
DMA2_Stream0->CR &= ~DMA_SxCR_PINC;  // Peripheral fixed
DMA2_Stream0->CR |= DMA_SxCR_CIRC;  // Circular mode

// Set addresses
DMA2_Stream0->PAR = (uint32_t)&ADC1->DR;  // Source: ADC1 data register
DMA2_Stream0->M0AR = (uint32_t)adc_values;  // Destination: buffer
DMA2_Stream0->NDTR = 3;  // Number of data items (3 channels)

// Enable DMA stream
DMA2_Stream0->CR |= DMA_SxCR_EN;

// Configure ADC for DMA
ADC1->CR2 |= ADC_CR2_DMA;  // Enable DMA mode
ADC1->CR2 |= ADC_CR2_CONT;  // Continuous conversion mode

// Start ADC conversion
ADC1->CR2 |= ADC_CR2_SWSTART;
\end{lstlisting}
\end{KR}

\begin{example2}{ADC Configuration Exercise}\\
Configure ADC1 to measure an analog voltage on pin PA5, with 12-bit resolution. Convert the result to a voltage between 0-3.3V.
\tcblower
1. Configure PA5 as an analog input
2. Configure ADC1 for 12-bit resolution, single conversion, single channel
3. Set appropriate sampling time
4. Perform conversion and convert result to voltage

\begin{lstlisting}[language=C, style=basesmol]
// Configure PA5 as analog input
RCC->AHB1ENR |= RCC_AHB1ENR_GPIOAEN;  // Enable GPIOA clock
RCC->APB2ENR |= RCC_PB2ENR_ADC1EN;   // Enable ADC1 clock
GPIOA->MODER |= GPIO_MODER_MODER5;    // Set PA5 to analog mode (0b11)

// Configure ADC1
ADC1->CR1 &= ~ADC_CR1_RES;     // 12-bit resolution (default)
ADC1->CR2 &= ~ADC_CR2_CONT;    // Single conversion mode
ADC1->CR2 &= ~ADC_CR2_ALIGN;   // Right alignment

// Configure for channel 5 (PA5)
ADC1->SQR1 &= ~ADC_SQR1_L;      // 1 conversion in sequence
ADC1->SQR3 &= ~ADC_SQR3_SQ1;    // Clear channel selection
ADC1->SQR3 |= (5 << ADC_SQR3_SQ1_Pos);  // Channel 5 as first conversion

// Set sampling time (84 cycles)
ADC1->SMPR2 &= ~ADC_SMPR2_SMP5;
ADC1->SMPR2 |= (4 << ADC_SMPR2_SMP5_Pos);

// Enable ADC
ADC1->CR2 |= ADC_CR2_ADON;

// Function to read ADC and convert to voltage
float ReadVoltage(void) {
    // Start conversion
    ADC1->CR2 |= ADC_CR2_SWSTART;
    
    // Wait for conversion to complete
    while (!(ADC1->SR & ADC_SR_EOC)) { }
    
    // Read ADC value
    uint16_t adcValue = ADC1->DR;
    
    // Convert to voltage (0-3.3V)
    float voltage = (float)adcValue * 3.3f / 4095.0f;
    
    return voltage;
}
\end{lstlisting}
\end{example2}