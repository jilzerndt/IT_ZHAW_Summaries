\section{GPIO (General Purpose I/O)}

\subsection{GPIO Fundamentals}

\begin{concept}{GPIO Overview}\\
General Purpose Input/Output (GPIO) pins allow the microcontroller to interact with the external world.
\begin{itemize}
    \item Pins can be configured as digital inputs or outputs
    \item Most pins support multiple functions (pin sharing) through internal multiplexers
    \item Configuration is done through memory-mapped registers
    \item Each GPIO port typically has 16 pins (e.g., GPIOA, GPIOB, etc.)
\end{itemize}
\end{concept}

\begin{definition}{Pin Sharing}\\
Multiple functions can share a single physical pin:
\begin{itemize}
    \item Digital inputs/outputs (GPIO)
    \item Serial interfaces (UART, SPI, I2C)
    \item Timers/Counters
    \item ADC (Analog-to-Digital Conversion)
    \item Alternate functions
\end{itemize}
Not all functions can be used simultaneously; configuration registers define pin usage.
\end{definition}

\subsection{GPIO Structure}

\begin{concept}{GPIO Hardware Structure}\\
Each GPIO pin contains several hardware components:
\begin{itemize}
    \item Input data register (IDR) - reads the pin state
    \item Output data register (ODR) - sets the output state
    \item Direction control (MODER) - configures pin as input or output
    \item Output type control (OTYPER) - push-pull or open-drain
    \item Pull-up/pull-down resistors (PUPDR)
    \item Speed configuration (OSPEEDR)
    \item Alternate function selection (AFRL/AFRH)
\end{itemize}
\end{concept}

\begin{definition}{GPIO Registers}\\
Each GPIO port has several configuration registers:
\begin{itemize}
    \item \textbf{GPIOx\_MODER}: Mode register (input, output, alternate function, analog)
    \item \textbf{GPIOx\_OTYPER}: Output type register (push-pull or open-drain)
    \item \textbf{GPIOx\_OSPEEDR}: Output speed register (low, medium, high, very high)
    \item \textbf{GPIOx\_PUPDR}: Pull-up/pull-down register
    \item \textbf{GPIOx\_IDR}: Input data register (read-only)
    \item \textbf{GPIOx\_ODR}: Output data register (read/write)
    \item \textbf{GPIOx\_BSRR}: Bit set/reset register (atomic operations)
    \item \textbf{GPIOx\_LCKR}: Configuration lock register
    \item \textbf{GPIOx\_AFRL/H}: Alternate function registers
\end{itemize}
\end{definition}

\columnbreak

\subsection{GPIO Configuration}

\begin{definition}{Direction Configuration (MODER)}\\
The GPIOx\_MODER register configures each pin's direction:
\begin{itemize}
    \item \textbf{00}: Input mode
    \item \textbf{01}: General purpose output mode
    \item \textbf{10}: Alternate function mode
    \item \textbf{11}: Analog mode
\end{itemize}
Each pin uses 2 bits in the register, allowing for 16 pins per port.
\end{definition}

\begin{definition}{Output Type (OTYPER)}\\
The GPIOx\_OTYPER register configures the output driver type:
\begin{itemize}
    \item \textbf{0}: Push-pull (can actively drive high or low)
    \item \textbf{1}: Open-drain (can drive low, relies on external pull-up for high)
\end{itemize}
\end{definition}

\begin{definition}{Pull-up/Pull-down (PUPDR)}\\
The GPIOx\_PUPDR register configures internal resistors:
\begin{itemize}
    \item \textbf{00}: No pull-up, no pull-down
    \item \textbf{01}: Pull-up
    \item \textbf{10}: Pull-down
    \item \textbf{11}: Reserved
\end{itemize}
\end{definition}

\begin{definition}{Speed Configuration (OSPEEDR)}\\
The GPIOx\_OSPEEDR register configures the output slew rate:
\begin{itemize}
    \item \textbf{00}: Low speed
    \item \textbf{01}: Medium speed
    \item \textbf{10}: High speed
    \item \textbf{11}: Very high speed
\end{itemize}
\end{definition}

\subsection{Push-Pull vs. Open-Drain}

\begin{concept}{Push-Pull vs Open-Drain Outputs}
\begin{itemize}
    \item \textbf{Push-Pull:}
    \begin{itemize}
        \item Can actively drive output high (to VDD) or low (to GND)
        \item Faster switching times, can source and sink current
        \item Default output mode for GPIO pins
    \end{itemize}
    \item \textbf{Open-Drain:}
    \begin{itemize}
        \item Can only actively drive output low
        \item Relies on external pull-up resistor to reach high state
        \item Multiple devices can share a line without conflicts (e.g., I2C bus)
        \item Used in "wired-AND" configurations
    \end{itemize}
\end{itemize}
\end{concept}

\subsection{Data Registers and Operations}

\begin{definition}{Input Data Register (IDR)}\\
The GPIOx\_IDR is a read-only register containing the input value of the corresponding I/O port.
\end{definition}

\begin{definition}{Output Data Register (ODR)}\\
The GPIOx\_ODR can be read and written to set the output state of GPIO pins.
\end{definition}

\begin{definition}{Bit Set/Reset Register (BSRR)}\\
The GPIOx\_BSRR allows atomic bit operations:
\begin{itemize}
    \item Bits [15:0]: Set corresponding ODR bit by writing '1'
    \item Bits [31:16]: Reset corresponding ODR bit by writing '1'
\end{itemize}
This ensures atomic access without read-modify-write operations.
\end{definition}

\begin{KR}{Configuring a GPIO Pin}
\paragraph{Step 1: Enable GPIO clock}
Enable the clock to the GPIO port using RCC register.
\paragraph{Step 2: Configure pin direction}
Set the mode register (MODER) to configure as input, output, etc.
\paragraph{Step 3: Configure additional properties}
Configure output type, speed, and pull-up/down as needed.
\paragraph{Step 4: Set initial state (for outputs)}
For output pins, set the initial state in ODR or using BSRR.

\begin{lstlisting}[language=C, style=basesmol] 
// Step 1: Enable GPIOA clock
RCC->AHB1ENR |= (1 << 0);  // Set bit 0 for GPIOA

// Step 2: Configure PA5 as output (bits 10-11 = 01)
GPIOA->MODER &= ~(3 << 10);  // Clear bits 10-11
GPIOA->MODER |= (1 << 10);   // Set as output

// Step 3: Configure as push-pull, high speed, no pull-up/down
GPIOA->OTYPER &= ~(1 << 5);     // Push-pull (clear bit 5)
GPIOA->OSPEEDR |= (3 << 10);    // Very high speed (set bits 10-11)
GPIOA->PUPDR &= ~(3 << 10);     // No pull-up/down (clear bits 10-11)

// Step 4: Set initial state (turn on LED)
GPIOA->ODR |= (1 << 5);         // Set PA5 high
// OR: Atomic set
GPIOA->BSRR = (1 << 5);         // Set PA5 high
\end{lstlisting}
\end{KR}

\begin{KR}{Reading and Writing GPIO}
\paragraph{Reading Input Pins}
Read the current state of GPIO pins using the IDR register.
\paragraph{Writing Output Pins - Using ODR}
Set or clear output pins by modifying the ODR register.
\paragraph{Writing Output Pins - Using BSRR (preferred)}
Set or clear output pins atomically using the BSRR register.

\begin{lstlisting}[language=C, style=basesmol] 
// Reading input from GPIOA pin 0
uint32_t buttonState = (GPIOA->IDR & (1 << 0)) != 0;

// Writing output using ODR (not atomic)
// Set pin high
GPIOA->ODR |= (1 << 5);
// Set pin low
GPIOA->ODR &= ~(1 << 5);

// Writing output using BSRR (atomic)
// Set pin high (bits 0-15)
GPIOA->BSRR = (1 << 5);
// Set pin low (bits 16-31)
GPIOA->BSRR = (1 << (5 + 16));
\end{lstlisting}
\end{KR}

\begin{example2}{GPIO Configuration Exercise}\\
Configure pin PA3 as an output with open-drain configuration, low speed, and pull-up enabled. Then set the pin to low state.
\begin{lstlisting}[language=C, style=basesmol] 
// Enable GPIOA clock
RCC->AHB1ENR |= (1 << 0);  // Bit 0 for GPIOA

// Configure PA3 as output (bits 6-7 = 01)
GPIOA->MODER \&= ~(3 << 6);  // Clear bits 6-7
GPIOA->MODER |= (1 << 6);   // Set bit 6 (output mode)

// Configure as open-drain (bit 3 = 1)
GPIOA->OTYPER |= (1 << 3);  

// Configure as low speed (bits 6-7 = 00)
GPIOA->OSPEEDR \&= ~(3 << 6);  // Clear bits 6-7 (low speed)

// Configure with pull-up (bits 6-7 = 01)
GPIOA->PUPDR \&= ~(3 << 6);    // Clear bits 6-7
GPIOA->PUPDR |= (1 << 6);     // Set bit 6 (pull-up)

// Set pin to low state using BSRR
GPIOA->BSRR = (1 << (3 + 16));  // Set bit 19 (reset PA3)
\end{lstlisting}
\end{example2}

\subsection{Hardware Abstraction Layer (HAL)}

\begin{concept}{HAL for GPIO}\\
The Hardware Abstraction Layer provides a structured way to access GPIO registers:
\begin{itemize}
    \item Based on structs that map to hardware registers
    \item Typedef for register structure (e.g., \texttt{reg\_gpio\_t})
    \item Pointers to each GPIO port (e.g., \texttt{GPIOA}, \texttt{GPIOB})
    \item Base addresses defined in header file
    \item Helper macros for bit manipulation
\end{itemize}
This approach makes code more readable and maintainable than direct register address manipulation.
\end{concept}

\begin{code}{Using HAL for GPIO}
\begin{lstlisting}[language=C, style=basesmol] 
// Using HAL style access
#include "reg_stm32f4xx.h"  // Contains structure definitions

// Configure PA3 as output
GPIOA->MODER &= ~(3 << 6);  // Clear bits 6-7
GPIOA->MODER |= (1 << 6);   // Set bit 6 (output mode)

// Instead of direct register access:
// volatile uint32_t *reg = (volatile uint32_t *)(0x40020000);
// *reg &= ~(3 << 6);
// *reg |= (1 << 6);
\end{lstlisting}
\end{code}
