\section{Memory}

\subsection{Memory Technologies Overview}

\begin{concept}{Semiconductor Memory Classifications}\\
Semiconductor memories can be classified into two main categories:
\begin{itemize}
    \item \textbf{Volatile Memory}: Loses data when power is turned off
    \begin{itemize}
        \item SRAM (Static Random Access Memory)
        \item DRAM (Dynamic Random Access Memory)
    \end{itemize}
    \item \textbf{Non-volatile Memory}: Retains data even without power
    \begin{itemize}
        \item ROM (Read-Only Memory)
        \item PROM (Programmable ROM)
        \item EPROM (Erasable PROM)
        \item EEPROM (Electrically Erasable PROM)
        \item Flash Memory
        \item NV-RAM (Non-Volatile RAM)
    \end{itemize}
\end{itemize}
\end{concept}

\begin{definition}{Memory Organization}\\
Memory devices are organized as arrays of bit cells:
\begin{itemize}
    \item \textbf{Array Size}: $n \times m$ (n words with m data bits each)
    \item \textbf{Address Lines}: $k$ bits can address $2^k$ words
    \item \textbf{Data Lines}: Width of data bus (8, 16, 32 bits, etc.)
    \item \textbf{Control Lines}: Enable read/write operations
\end{itemize}
\end{definition}

\subsection{PROM, EEPROM, and Flash Memory}

\mult{2}

\begin{concept}{PROM (Programmable Read-Only Memory)}
\begin{itemize}
    \item One-time programmable memory
    \item Programming involves physically altering the circuit (e.g., blowing fuses)
    \item Once programmed, contents cannot be changed
    \item Used for permanent storage of code or data
\end{itemize}
\end{concept}

\begin{concept}{EEPROM (Electrically Erasable PROM)}
\begin{itemize}
    \item Uses floating-gate transistors to store data
    \item Can be electrically programmed and erased
    \item Byte-level erase and write operations
    \item Limited write cycles (typically 100,000 to 1,000,000)
    \item Slower and more expensive than SRAM
    \item Used for storing configuration data or parameters
\end{itemize}
\end{concept}

\begin{concept}{Flash Memory}
\begin{itemize}
    \item Based on floating-gate transistor technology (like EEPROM)
    \item Higher density and lower cost per bit than EEPROM
    \item Block-wise erase operations (not byte-level)
    \item Write operations can only change bits from '1' to '0'
    \item Erase operations reset all bits in a block to '1'
    \item Limited write/erase cycles (typically 10,000 to 100,000)
    \item Used for code storage and mass data storage
\end{itemize}
\end{concept}

\begin{definition}{NOR vs. NAND Flash}\\
\textbf{NOR Flash}:
\begin{itemize}
    \item Random access (like RAM)
    \item Execute-in-place capability (XIP)
    \item Fast read access
    \item Slow write and erase operations
    \item Lower density
    \item Used for code storage and execution
\end{itemize}
\textbf{NAND Flash}:
\begin{itemize}
    \item Page-based access (not random)
    \item Cannot execute code directly
    \item Slow random read access
    \item Fast sequential read and write
    \item Higher density
    \item Used for mass storage (SSDs, memory cards)
\end{itemize}
\end{definition}

\multend

\subsection{SRAM (Static RAM)}

\mult{2}

\begin{concept}{SRAM Structure and Characteristics}
\begin{itemize}
    \item Uses flip-flop circuit for each bit (typically 6 transistors)
    \item Maintains data as long as power is supplied
    \item No refresh required (unlike DRAM)
    \item Fast access times (a few nanoseconds)
    \item Low power consumption in standby mode
    \item Higher cost and lower density than DRAM
    \item Used for cache memory and high-speed buffers
\end{itemize}
\end{concept}

\begin{definition}{SRAM Cell}\\
A typical SRAM cell consists of:
\begin{itemize}
    \item Cross-coupled inverters forming a latch to store one bit
    \item Two access transistors to connect the cell to bit lines
    \item Word line to enable/disable access to the cell
    \item High state ('1') and low state ('0') stable as long as power is maintained
\end{itemize}
\end{definition}

\begin{definition}{SRAM Operations}\\
\textbf{Read Operation}:
\begin{itemize}
    \item Word line is activated
    \item Access transistors connect cell to bit lines
    \item Sense amplifiers detect voltage difference on bit lines
    \item Data is read from bit lines
\end{itemize}
\textbf{Write Operation}:
\begin{itemize}
    \item Word line is activated
    \item Access transistors connect cell to bit lines
    \item Write drivers force bit lines to desired values
    \item Cell state changes to match bit line values
\end{itemize}
\end{definition}

\begin{definition}{Asynchronous SRAM Interface}\\
Asynchronous SRAM devices typically have these control signals:
\begin{itemize}
    \item \textbf{CS} (Chip Select): Enables the memory device (active low)
    \item \textbf{OE} (Output Enable): Enables data output during read (active low)
    \item \textbf{WE} (Write Enable): Indicates write operation (active low)
    \item \textbf{Address Lines}: Select memory location
    \item \textbf{Data Lines}: Bidirectional lines for data transfer
\end{itemize}
\end{definition}

\multend

\subsection{SDRAM (Synchronous DRAM)}

\mult{2}

\begin{concept}{SDRAM Structure and Characteristics}
\begin{itemize}
    \item Uses a capacitor and one transistor for each bit
    \item Requires periodic refresh to maintain data (capacitor leakage)
    \item Synchronous interface (clocked)
    \item Row and column addressing (multiplexed address bus)
    \item Higher density and lower cost than SRAM
    \item Higher power consumption due to refresh
    \item Used for main memory in computers and embedded systems
\end{itemize}
\end{concept}

\begin{definition}{SDRAM Operation}\\
Key aspects of SDRAM operation:
\begin{itemize}
    \item \textbf{Refresh}: Periodic read and rewrite of all memory cells
    \item \textbf{Row Activation}: Opening a row copies data to row buffer
    \item \textbf{Column Access}: Selecting specific bytes from row buffer
    \item \textbf{Precharge}: Preparing a bank for next row activation
    \item \textbf{Burst Mode}: Sequential access to multiple columns
\end{itemize}
\end{definition}

\multend

\begin{concept}{SRAM vs. SDRAM Comparison}
\begin{center}
\begin{tabular}{|p{3cm}|p{4cm}|p{4cm}|}
\hline
\textbf{Feature} & \textbf{SRAM} & \textbf{SDRAM} \\
\hline
Cell Structure & 6 transistors (flip-flop) & 1 transistor + 1 capacitor \\
\hline
Refresh & Not required & Required (periodic) \\
\hline
Density & Lower & Higher \\
\hline
Cost per bit & Higher & Lower \\
\hline
Access Time & Faster, uniform & Variable (row hit vs. miss) \\
\hline
Interface & Often asynchronous & Synchronous (clocked) \\
\hline
Power Consumption & Lower static power & Higher due to refresh \\
\hline
Applications & Cache, high-speed buffer & Main memory \\
\hline
\end{tabular}
\end{center}
\end{concept}

\subsection{STM32F4 On-Chip Memory}

\begin{concept}{STM32F4 Memory Architecture}\\
The STM32F429ZI microcontroller includes:
\begin{itemize}
    \item \textbf{Flash Memory}: 2 MB (program storage)
    \begin{itemize}
        \item NOR flash with execute-in-place capability
        \item Divided into sectors of varying sizes (16KB to 128KB)
        \item Organized in two banks for read-while-write operations
    \end{itemize}
    \item \textbf{SRAM}: 256 KB total
    \begin{itemize}
        \item SRAM1: 112 KB
        \item SRAM2: 16 KB
        \item SRAM3: 64 KB
        \item CCM (Core Coupled Memory): 64 KB (accessible only by CPU)
    \end{itemize}
\end{itemize}
\end{concept}

\begin{definition}{STM32F4 Flash Characteristics}
\begin{itemize}
    \item \textbf{Write Operations}: Can only change bits from '1' to '0'
    \item \textbf{Erase Operations}: Resets all bits in a sector to '1'
    \item \textbf{Programming Time}: Around 16µs per double word
    \item \textbf{Erase Time}: 1-2 seconds for a 128KB sector
    \item \textbf{Endurance}: 10,000 erase cycles
    \item \textbf{Access Time}: Higher latency than SRAM (requires wait states)
\end{itemize}
\end{definition}

\subsection{External Memory Interface}

\begin{concept}{Flexible Memory Controller (FMC)}\\
The STM32F4 Flexible Memory Controller (FMC) provides:
\begin{itemize}
    \item Interface between on-chip system bus and external memory devices
    \item Support for different memory types:
    \begin{itemize}
        \item SRAM, NOR Flash, PSRAM
        \item NAND Flash
        \item SDRAM
    \end{itemize}
    \item Configurable bus width (8, 16, or 32 bits)
    \item Programmable timing parameters
    \item Memory banking with up to 6 banks
\end{itemize}
\end{concept}

\begin{definition}{FMC Signals}\\
Key FMC signals for external SRAM/NOR flash:
\begin{itemize}
    \item \textbf{A[25:0]}: Address bus
    \item \textbf{D[31:0]}: Data bus (bidirectional)
    \item \textbf{NE[4:1]}: Chip enable signals (active low)
    \item \textbf{NOE}: Output enable (active low)
    \item \textbf{NWE}: Write enable (active low)
    \item \textbf{NBL[3:0]}: Byte lane enables (active low)
\end{itemize}
\end{definition}

\begin{concept}{External Memory Access}\\
Accessing external memory with different bus widths:
\begin{itemize}
    \item \textbf{32-bit CPU Access to 32-bit Memory}: 1 external bus cycle
    \item \textbf{32-bit CPU Access to 16-bit Memory}: 2 external bus cycles
    \item \textbf{32-bit CPU Access to 8-bit Memory}: 4 external bus cycles
\end{itemize}
\textbf{Write Operations}:
\begin{itemize}
    \item CPU write stored in FMC FIFO buffer
    \item System bus released for other access
    \item FMC completes external write(s)
\end{itemize}
\textbf{Read Operations}:
\begin{itemize}
    \item System bus must wait until all external reads complete
    \item Multiple external cycles for narrow memory widths
\end{itemize}
\end{concept}

\begin{KR}{Connecting Asynchronous SRAM to STM32F4}
\paragraph{Step 1: Configure GPIO pins}
Set the GPIO pins for FMC signals to alternate function mode.
\paragraph{Step 2: Configure FMC timing}
Set appropriate timing parameters (ADDSET, DATAST) based on memory datasheet.
\paragraph{Step 3: Configure FMC bank}
Set the memory type, data width, and other parameters.
\paragraph{Step 4: Enable FMC}
Enable the FMC peripheral.

\begin{lstlisting}[language=C, style=basesmol]
// Configure external SRAM (16-bit) on FMC bank 1

// Step 1: Configure GPIO pins for FMC
// Enable GPIO clocks
RCC->AHB1ENR |= RCC_AHB1ENR_GPIODEN | RCC_AHB1ENR_GPIOEEN |
                RCC_AHB1ENR_GPIOFEN | RCC_AHB1ENR_GPIOGEN;

// Configure GPIO pins (example for some pins)
// Set alternate function mode (0x2)
GPIOD->MODER |= 0x55555555;  // All pins to alternate function
GPIOE->MODER |= 0x55555555;
// Set to AF12 (FMC)
GPIOD->AFR[0] = 0xCCCCCCCC;
GPIOD->AFR[1] = 0xCCCCCCCC;
GPIOE->AFR[0] = 0xCCCCCCCC;
GPIOE->AFR[1] = 0xCCCCCCCC;

// Step 2: Enable FMC clock
RCC->AHB3ENR |= RCC_AHB3ENR_FMCEN;

// Step 3: Configure FMC bank 1 for SRAM
// Set timing for SRAM (example values)
FMC_Bank1->BTCR[0] = 
    FMC_BCR1_MBKEN |    // Memory bank enable
    FMC_BCR1_MTYP_0 |   // Memory type SRAM
    FMC_BCR1_MWID_0 |   // 16-bit data bus
    FMC_BCR1_WREN;      // Write enable

// Set timing (ADDSET=1, DATAST=2)
FMC_Bank1->BTCR[1] = 
    (1 << FMC_BTR1_ADDSET_Pos) |
    (2 << FMC_BTR1_DATAST_Pos);
\end{lstlisting}
\end{KR}

\begin{example2}{Address Space Calculation for External Memory}\\
Calculate the address range for an external 32K x 8-bit SRAM connected to FMC bank 3.
\tcblower
For an external SRAM connected to FMC bank 3:

1. Base address of FMC bank 3 = 0x68000000
2. Memory size = 32K bytes = 32,768 bytes = 0x8000 bytes

Therefore, the address range for this SRAM would be:
- Start address: 0x68000000
- End address: 0x68000000 + 0x8000 - 1 = 0x68007FFF

Address range: 0x68000000 - 0x68007FFF

Note: Due to partial address decoding, this SRAM might also be accessible at other addresses within bank 3. For example, it might also respond to addresses:
0x68008000 - 0x6800FFFF, 0x68010000 - 0x68017FFF, etc.
\end{example2}