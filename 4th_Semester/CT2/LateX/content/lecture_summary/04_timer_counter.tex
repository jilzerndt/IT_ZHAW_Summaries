\section{Timer/Counter}

\subsection{Timer/Counter Basics}

\begin{minipage}{0.54\linewidth}
\begin{concept}{Timer/Counter Fundamentals}\\
A timer/counter is a digital circuit that counts events:
\begin{itemize}
    \item \textbf{Timer}: Counts clock cycles or processor cycles (periodic)
    \item \textbf{Counter}: Counts external events or signals
\end{itemize}
Common applications:
\begin{itemize}
    \item Triggering periodic software tasks (e.g., display refresh)
    \item Sampling inputs (e.g., buttons) at regular intervals
    \item Counting pulses on an input pin
    \item Measuring time between events
    \item Generating pulse sequences on output pins
\end{itemize}
\end{concept}

\begin{definition}{Timer/Counter Structure}\\
Basic components of a timer/counter system:
\begin{itemize}
    \item \textbf{Counter Register}: 16-bit or 32-bit counter that increments/decrements
    \item \textbf{Prescaler}: Divides input clock to extend counting range
    \item \textbf{Auto-Reload Register (ARR)}: Sets upper limit for counting
    \item \textbf{Source Selection}: Internal clocks, input pins, other timers
    \item \textbf{Control Logic}: Configures counting mode and operation
    \item \textbf{Interrupt Flags}: Signal events to the CPU
\end{itemize}
\end{definition}
\end{minipage}
\begin{minipage}{0.42\linewidth}
\begin{concept}{Counter Modes}\\
\textbf{Up-counting mode}:
\begin{itemize}
    \item Counter starts from 0
    \item Increments up to auto-reload value (ARR)
    \item Generates overflow event when reaching ARR
    \item Resets to 0 and continues
\end{itemize}
\textbf{Down-counting mode}:
\begin{itemize}
    \item Counter starts from auto-reload value (ARR)
    \item Decrements down to 0
    \item Generates underflow event when reaching 0
    \item Reloads ARR value and continues
\end{itemize}
\textbf{Center-aligned mode}:
\begin{itemize}
    \item Counter counts from 0 to ARR-1, then back to 1
    \item Generates events at both up and down counting
    \item Useful for symmetric PWM generation
\end{itemize}
\end{concept}
\end{minipage}

\subsection{STM32F4 Timers}

\mult{2}

\begin{concept}{STM32F4 Timer Types}\\
The STM32F4 includes several types of timers:
\begin{itemize}
    \item \textbf{Basic Timers} (TIM6, TIM7):
    \begin{itemize}
        \item 16-bit counter, prescaler, auto-reload
        \item Simplest timers, mainly for time-base generation
    \end{itemize}
    \item \textbf{General-Purpose Timers} (TIM2-TIM5, TIM9-TIM14):
    \begin{itemize}
        \item TIM2, TIM5: 32-bit timers
        \item TIM3, TIM4: 16-bit timers
        \item Multiple capture/compare channels
        \item Input capture, output compare, PWM generation
    \end{itemize}
    \item \textbf{Advanced-Control Timers} (TIM1, TIM8):
    \begin{itemize}
        \item Advanced PWM features
        \item Complementary outputs with dead-time insertion
        \item Break input for motor control safety
    \end{itemize}
\end{itemize}
\end{concept}

\begin{definition}{Timer Clock Sources}\\
STM32F4 timers can use different clock sources:
\begin{itemize}
    \item \textbf{Internal Clock (CK\_INT)}: Default source, derived from system clock
    \item \textbf{External Clock Mode 1}: Timer clock from external pins (TIMx\_CH1, TIMx\_CH2)
    \item \textbf{External Clock Mode 2}: Timer clock from external trigger input (TIMx\_ETR)
    \item \textbf{Internal Trigger Inputs (ITRx)}: Using one timer as prescaler for another
\end{itemize}
\end{definition}



\subsubsection{Timer Configuration}

\begin{definition}{Key Timer Registers}\\
Important registers for timer configuration:
\begin{itemize}
    \item \textbf{TIMx\_CR1} (Control Register 1):
    \begin{itemize}
        \item CEN: Counter Enable
        \item DIR: Direction (0=up, 1=down)
        \item CMS: Center-aligned Mode Selection
    \end{itemize}
    \item \textbf{TIMx\_PSC} (Prescaler):
    \begin{itemize}
        \item Divides clock frequency by a factor between 1 and 65536
        \item Actual division factor is PSC+1
    \end{itemize}
    \item \textbf{TIMx\_ARR} (Auto-reload Register):
    \begin{itemize}
        \item Sets period in up-counting mode
        \item Sets initial value in down-counting mode
    \end{itemize}
    \item \textbf{TIMx\_CNT} (Counter):
    \begin{itemize}
        \item Current counter value
    \end{itemize}
    \item \textbf{TIMx\_SR} (Status Register):
    \begin{itemize}
        \item UIF: Update Interrupt Flag
        \item CCxIF: Capture/Compare Interrupt Flags
    \end{itemize}
    \item \textbf{TIMx\_DIER} (DMA/Interrupt Enable Register):
    \begin{itemize}
        \item UIE: Update Interrupt Enable
        \item CCxIE: Capture/Compare Interrupt Enable
    \end{itemize}
\end{itemize}
\end{definition}

\multend

\begin{KR}{Basic Timer Configuration}
\paragraph{Step 1: Enable timer clock}
Enable the clock to the timer peripheral using RCC register.
\paragraph{Step 2: Configure prescaler}
Set the prescaler value to divide the timer clock.
\paragraph{Step 3: Configure auto-reload value}
Set the period in up-counting mode or initial value in down-counting mode.
\paragraph{Step 4: Configure counting mode}
Select up-counting, down-counting, or center-aligned mode.
\paragraph{Step 5: Configure interrupts (if needed)}
Enable update or capture/compare interrupts.
\paragraph{Step 6: Enable the timer}
Set the CEN bit in the CR1 register.

\begin{lstlisting}[language=C, style=basesmol]
// Configure TIM2 for a 1Hz interrupt (1s period)
// Assuming system clock = 84MHz

// Step 1: Enable TIM2 clock
RCC->APB1ENR |= RCC_APB1ENR_TIM2EN;

// Step 2: Configure prescaler
// PSC = (SystemCoreClock / 2) / 10000 - 1
TIM2->PSC = 8399;  // Produces 10kHz timer clock (84MHz/8400)

// Step 3: Configure auto-reload value
// ARR = (timer clock / desired frequency) - 1
TIM2->ARR = 9999;  // 10kHz/10000 = 1Hz

// Step 4: Configure counting mode (up-counting, default)
TIM2->CR1 &= ~TIM_CR1_DIR;  // Up-counting mode

// Step 5: Enable update interrupt
TIM2->DIER |= TIM_DIER_UIE;

// Step 6: Enable timer
TIM2->CR1 |= TIM_CR1_CEN;

// Configure NVIC to handle TIM2 interrupt
NVIC_EnableIRQ(TIM2_IRQn);
NVIC_SetPriority(TIM2_IRQn, 1);
\end{lstlisting}
\end{KR}

\begin{formula}{Timer Period Calculation}\\
For a given timer frequency, the period is calculated as:
$
T_{timer} = \frac{1}{f_{timer}} 
$

Timer frequency is derived from the system clock:
$
f_{timer} = \frac{f_{system}}{(PSC+1)}
$

For a desired output frequency:
$ARR = \frac{f_{timer}}{f_{desired}} - 1$

For a desired time period:
$
ARR = f_{timer} \times T_{desired} - 1$
\end{formula}

\begin{example2}{Timer Period Calculation}\\
Calculate the prescaler (PSC) and auto-reload (ARR) values to generate a 50ms timer period using a 84MHz clock.
\tcblower
We need to find PSC and ARR values to generate a 50ms (0.05s) period.

Step 1: Choose an appropriate prescaler value.
Let's pick a prescaler to generate a 1MHz timer clock:
PSC = 84MHz / 1MHz - 1 = 84 - 1 = 83

Step 2: Calculate the auto-reload value.
ARR = f\_{timer} × T\_{desired} - 1
ARR = 1MHz × 0.05s - 1
ARR = 50,000 - 1 = 49,999

However, this exceeds the 16-bit range (maximum 65,535).
Let's adjust to use a 10kHz timer clock:
PSC = 84MHz / 10kHz - 1 = 8,400 - 1 = 8,399
ARR = 10kHz × 0.05s - 1 = 500 - 1 = 499

Therefore:
PSC = 8,399
ARR = 499
\end{example2}

\subsection{Input Capture}

\begin{concept}{Input Capture}\\
Input capture is used to measure the timing of external events:
\begin{itemize}
    \item Captures the timer counter value when an event occurs on an input pin
    \item Events can be rising edge, falling edge, or both
    \item Useful for measuring pulse width, period, frequency, or phase difference
    \item Each capture channel has its own register (CCRx)
\end{itemize}
Applications:
\begin{itemize}
    \item Measuring pulse width (e.g., from sensors)
    \item Measuring frequency of input signals
    \item Timing between events
\end{itemize}
\end{concept}

\begin{KR}{Input Capture Configuration}
\paragraph{Step 1: Configure GPIO pin}
Configure the GPIO pin as alternate function for the timer channel.
\paragraph{Step 2: Configure timer base}
Set up the timer's prescaler and period.
\paragraph{Step 3: Configure capture channel}
Configure the channel for input capture and set the edge sensitivity.
\paragraph{Step 4: Enable interrupts}
Enable capture interrupt and NVIC if notification is required.
\paragraph{Step 5: Enable timer}
Start the timer counting.

\begin{lstlisting}[language=C, style=basesmol]
// Configure TIM3 Channel 1 for input capture on rising edge

// Step 1: Configure GPIO pin (PA6 for TIM3_CH1)
RCC->AHB1ENR |= RCC_AHB1ENR_GPIOAEN;
GPIOA->MODER &= ~GPIO_MODER_MODER6;
GPIOA->MODER |= GPIO_MODER_MODER6_1;  // Alternate function
GPIOA->AFR[0] &= ~GPIO_AFRL_AFRL6;
GPIOA->AFR[0] |= 2 << GPIO_AFRL_AFRL6_Pos;  // AF2 for TIM3

// Step 2: Configure timer base
RCC->APB1ENR |= RCC_APB1ENR_TIM3EN;
TIM3->PSC = 83;     // 84MHz/84 = 1MHz timer clock
TIM3->ARR = 65535;  // Maximum period

// Step 3: Configure capture channel
// CC1S[1:0] = 01 (input, IC1 mapped to TI1)
TIM3->CCMR1 &= ~TIM_CCMR1_CC1S;
TIM3->CCMR1 |= TIM_CCMR1_CC1S_0;

// Configure input filter and prescaler if needed
TIM3->CCMR1 &= ~(TIM_CCMR1_IC1F | TIM_CCMR1_IC1PSC);

// Configure edge sensitivity (rising edge)
TIM3->CCER &= ~(TIM_CCER_CC1P | TIM_CCER_CC1NP);

// Enable capture
TIM3->CCER |= TIM_CCER_CC1E;

// Step 4: Enable capture interrupt
TIM3->DIER |= TIM_DIER_CC1IE;
NVIC_EnableIRQ(TIM3_IRQn);

// Step 5: Enable timer
TIM3->CR1 |= TIM_CR1_CEN;
\end{lstlisting}
\end{KR}

\subsection{Pulse Width Modulation (PWM)}

\begin{concept}{PWM Basics}\\
Pulse Width Modulation is a technique to create analog-like signals using digital outputs:
\begin{itemize}
    \item Signal alternates between high and low state
    \item Fixed frequency (period)
    \item Variable duty cycle (ratio of high time to period)
    \item The average voltage is proportional to duty cycle
\end{itemize}
Applications:
\begin{itemize}
    \item LED dimming
    \item Motor speed control
    \item Digital-to-analog conversion
    \item Signal generation
\end{itemize}
\end{concept}

\begin{definition}{PWM Terminology}
\begin{itemize}
    \item \textbf{Period} (T): Time for one complete cycle
    \item \textbf{Frequency} (f): Number of cycles per second (f = 1/T)
    \item \textbf{Duty Cycle} (D): Ratio of high time to period (D = T\_{high}/T)
    \item \textbf{Resolution}: Smallest change in duty cycle possible
\end{itemize}
\end{definition}

\begin{definition}{PWM Alignment Types}
\begin{itemize}
    \item \textbf{Edge-aligned PWM}:
    \begin{itemize}
        \item One edge of the pulse is fixed, the other is modulated
        \item Up-counting mode: Leading edge fixed, trailing edge modulated
        \item Down-counting mode: Trailing edge fixed, leading edge modulated
    \end{itemize}
    \item \textbf{Center-aligned PWM}:
    \begin{itemize}
        \item Center of pulse is fixed, both edges are modulated
        \item Produces less harmonic distortion in motor control applications
        \item Implemented using up/down counting mode
    \end{itemize}
\end{itemize}
\end{definition}

\begin{concept}{PWM Generation Using Output Compare}\\
PWM is implemented using the output compare function:
\begin{itemize}
    \item Counter continuously runs through a defined range (0 to ARR)
    \item Output pin toggles when counter matches the capture/compare value (CCR)
    \item In up-counting PWM mode 1:
    \begin{itemize}
        \item Output active when counter < CCR
        \item Output inactive when counter $\geq$ CCR
    \end{itemize}
    \item Duty cycle is controlled by changing the CCR value
    \begin{itemize}
        \item Duty cycle = CCR / ARR (in up-counting mode)
    \end{itemize}
\end{itemize}
\end{concept}

\begin{KR}{PWM Configuration}
\paragraph{Step 1: Configure GPIO pin}
Configure the GPIO pin as alternate function for the timer channel.
\paragraph{Step 2: Configure timer base}
Set prescaler and auto-reload value to define PWM frequency.
\paragraph{Step 3: Configure PWM mode}
Set output compare mode to PWM and configure polarity.
\paragraph{Step 4: Set initial duty cycle}
Write the initial CCR value.
\paragraph{Step 5: Enable output and timer}
Enable output and start the timer.

\begin{lstlisting}[language=C, style=basesmol]
// Configure TIM3 Channel 1 for PWM output at 1kHz with 50% duty cycle

// Step 1: Configure GPIO pin (PA6 for TIM3_CH1)
RCC->AHB1ENR |= RCC_AHB1ENR_GPIOAEN;
GPIOA->MODER &= ~GPIO_MODER_MODER6;
GPIOA->MODER |= GPIO_MODER_MODER6_1;  // Alternate function
GPIOA->AFR[0] &= ~GPIO_AFRL_AFRL6;
GPIOA->AFR[0] |= 2 << GPIO_AFRL_AFRL6_Pos;  // AF2 for TIM3

// Step 2: Configure timer base
RCC->APB1ENR |= RCC_APB1ENR_TIM3EN;
TIM3->PSC = 83;    // 84MHz/84 = 1MHz timer clock
TIM3->ARR = 999;   // 1MHz/1000 = 1kHz PWM frequency

// Step 3: Configure PWM mode
// CC1S = 00: Channel configured as output
TIM3->CCMR1 &= ~TIM_CCMR1_CC1S;

// OC1M = 110: PWM mode 1 (active when counter < CCR1)
TIM3->CCMR1 &= ~TIM_CCMR1_OC1M;
TIM3->CCMR1 |= TIM_CCMR1_OC1M_1 | TIM_CCMR1_OC1M_2;

// Enable output compare preload
TIM3->CCMR1 |= TIM_CCMR1_OC1PE;

// Set active high polarity
TIM3->CCER &= ~TIM_CCER_CC1P;

// Step 4: Set initial duty cycle (50%)
TIM3->CCR1 = 500;  // 50% of 1000

// Step 5: Enable output and timer
TIM3->CCER |= TIM_CCER_CC1E;  // Enable output
TIM3->CR1 |= TIM_CR1_CEN;     // Enable counter
\end{lstlisting}
\end{KR}

\begin{formula}{PWM Duty Cycle Calculation}\\
\textbf{Up-counting mode}:
\begin{align}
\text{Duty Cycle} = \frac{CCR}{ARR+1} \times 100\%
\end{align}

\textbf{Down-counting mode}:
\begin{align}
\text{Duty Cycle} = \left(1 - \frac{CCR}{ARR+1}\right) \times 100\%
\end{align}

\textbf{To achieve a specific duty cycle}:
\begin{align}
\text{CCR (up-counting)} &= (ARR+1) \times \frac{\text{Duty Cycle}}{100\%} \\
\text{CCR (down-counting)} &= (ARR+1) \times \left(1 - \frac{\text{Duty Cycle}}{100\%}\right)
\end{align}
\end{formula}

\begin{example2}{PWM Configuration Exercise}\\
Configure Timer 4 to generate a 20kHz PWM signal with 75% duty cycle on Channel 2.
\tcblower
Assuming system clock = 84MHz:

1. Calculate timer settings for 20kHz:
   - Let's use prescaler = 3 (divide by 4)
   - Timer clock = 84MHz/4 = 21MHz
   - For 20kHz: ARR = 21MHz/20kHz - 1 = 1049

2. Calculate CCR value for 75% duty cycle:
   - Up-counting mode, PWM mode 1
   - CCR = (ARR+1) * 0.75 = 1050 * 0.75 = 787

3. Configuration code:

\begin{lstlisting}[language=C, style=basesmol]
// Configure GPIO pin (PB7 for TIM4\_CH2)
RCC->AHB1ENR |= RCC_AHB1ENR_GPIOBEN;
GPIOB->MODER &= ~GPIO_MODER_MODER7;
GPIOB->MODER |= GPIO_MODER_MODER7_1;  // Alternate function
GPIOB->AFR[0] &= ~GPIO_AFRL_AFRL7;
GPIOB->AFR[0] |= 2 << GPIO_AFRL_AFRL7_Pos;  // AF2 for TIM4

// Configure timer base
RCC->APB1ENR |= RCC_APB1ENR_TIM4EN;
TIM4->PSC = 3;     // 84MHz/4 = 21MHz
TIM4->ARR = 1049;  // 21MHz/1050 = 20kHz

// Configure PWM mode
TIM4->CCMR1 &= ~TIM_CCMR1_CC2S;  // Channel as output
TIM4->CCMR1 &= ~TIM_CCMR1_OC2M;
TIM4->CCMR1 |= TIM_CCMR1_OC2M_1 | TIM_CCMR1_OC2M_2;  // PWM mode 1
TIM4->CCMR1 |= TIM_CCMR1_OC2PE;  // Enable preload
TIM4->CCER &= ~TIM_CCER_CC2P;    // Active high

// Set duty cycle (75%)
TIM4->CCR2 = 787;

// Enable output and timer
TIM4->CCER |= TIM_CCER_CC2E;
TIM4->CR1 |= TIM_CR1_CEN;
\end{lstlisting}
\end{example2}