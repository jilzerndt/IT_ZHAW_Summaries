\section{Bezugssysteme und Einführung}

\subsection{Einführung in Physik Engines}
\begin{concept}{Physics Engines}\\
    Physics Engines sind Softwarekomponenten, die physikalische Effekte in Computerprogrammen (z.B. Spielen) simulieren. In diesem Modul geht es um:
    \begin{itemize}
        \item Physikalische Modellierung in Unity
        \item Verstehen physikalischer Grundprinzipien für die korrekte Anwendung in Unity
        \item Kopplung von Physiksimulatoren (wie PhysX) mit realistischen Parametern
    \end{itemize}
    Selbst wenn Unity's PhysX viele Funktionen automatisch berechnet, benötigt man ein fundiertes Verständnis der Physik, um realistische Simulationen zu erstellen.
\end{concept}

\begin{remark}
    Die Modellbildung in Unity folgt diesem Ablauf:
    \begin{enumerate}
        \item Wirklichkeit $\rightarrow$ Physikalisches Modell
        \item Physikalisches Modell $\rightarrow$ Mathematisches Modell
        \item Mathematisches Modell $\rightarrow$ Numerisches Modell
        \item Numerisches Modell $\rightarrow$ Modellierte Werte
        \item Diese können dann mit experimentellen Daten verglichen werden
    \end{enumerate}
\end{remark}

\subsection{Bezugssysteme in der Mechanik}
\begin{definition}{Bezugssystem}\\
    Ein Bezugssystem definiert:
    \begin{itemize}
        \item Einen Nullpunkt im Raum
        \item Die Richtungen der Koordinatenachsen (x, y, z)
        \item Eine Zeitmessung
    \end{itemize}
    Dadurch wird die Position eines Körpers eindeutig durch einen Ortsvektor $\vec{r}$ beschrieben.
\end{definition}

\begin{concept}{Vektoren}\\
    Ein Vektor ist eine physikalische Größe mit Betrag und Richtung.
    \begin{itemize}
        \item Darstellung: $\vec{r}$ (mit Pfeil über dem Symbol)
        \item Betrag: $|\vec{r}| = r$ (ohne Pfeil)
        \item In Koordinatendarstellung: $\vec{r} = \begin{pmatrix} r_x \\ r_y \\ r_z \end{pmatrix}$
        \item Einheitsvektor (Betrag = 1): $\vec{e}_r = \frac{\vec{r}}{|\vec{r}|}$
    \end{itemize}
\end{concept}

\begin{formula}{Rechenregeln für Vektoren}\\
    \begin{itemize}
        \item Addition: $\vec{r}_1 + \vec{r}_2 = \begin{pmatrix} r_{x1} \\ r_{y1} \\ r_{z1} \end{pmatrix} + \begin{pmatrix} r_{x2} \\ r_{y2} \\ r_{z2} \end{pmatrix} = \begin{pmatrix} r_{x1} + r_{x2} \\ r_{y1} + r_{y2} \\ r_{z1} + r_{z2} \end{pmatrix}$
        
        \item Skalarprodukt (ergibt einen Skalar): 
        $s = \vec{r}_1 \cdot \vec{r}_2 = |\vec{r}_1| \cdot |\vec{r}_2| \cdot \cos \angle(\vec{r}_1, \vec{r}_2) = r_{x1}r_{x2} + r_{y1}r_{y2} + r_{z1}r_{z2}$
        
        \item Kreuzprodukt (ergibt einen Vektor):
        $\vec{r}_1 \times \vec{r}_2 = \begin{pmatrix} r_{y1}r_{z2} - r_{z1}r_{y2} \\ r_{z1}r_{x2} - r_{x1}r_{z2} \\ r_{x1}r_{y2} - r_{y1}r_{x2} \end{pmatrix}$
    \end{itemize}
\end{formula}

\begin{example2}{Unity Bezugssystem}\\
    Ein Unterschied zwischen Unity und den in der Physik üblichen Bezugssystemen:
    \begin{itemize}
        \item Physik und Mathematik verwenden üblicherweise ein Rechtssystem
        \item Unity verwendet ein Linkssystem
        \item In Unity zeigt die positive y-Achse nach oben (nicht die z-Achse)
    \end{itemize}
    
    Dies hat Auswirkungen auf die Berechnung von Kreuzprodukten und die Interpretation von Rotationen.
\end{example2}

\begin{definition}{SI-Einheiten}\\
    Wichtige SI-Einheiten in der Mechanik:
    \begin{itemize}
        \item Länge in Meter (m)
        \item Masse in Kilogramm (kg)
        \item Zeit in Sekunden (s)
        \item Kraft in Newton (N = kg·m/s²)
    \end{itemize}
    Es ist wichtig, in Unity konsequent SI-Einheiten zu verwenden und bei allen Werten entsprechende Einheiten anzugeben.
\end{definition}

\begin{KR}{Konventionen für Unity-Code}\\
    \paragraph{Deklaration von physikalischen Größen}
    \begin{lstlisting}[language=csh, style=basesmol]
// Korrekte Deklaration physikalischer Groessen in Unity
public float initialVelocity = 6.0f; // in m/s
public float mass = 1.0f; // in kg
public float springConstant = 10.0f; // in N/m
    \end{lstlisting}
    
    \paragraph{Abfrage von Vektoren}
    \begin{lstlisting}[language=csh, style=basesmol]
// Abfrage von Position und Geschwindigkeit
Vector3 position = rigidBody.position; // in m
Vector3 velocity = rigidBody.velocity; // in m/s
    \end{lstlisting}
    
    \paragraph{Berechnung von Kräften}
    \begin{lstlisting}[language=csh, style=basesmol]
// Beispiel: Federkraft berechnen
Vector3 springForce = -springConstant * (position - equilibriumPosition);
rigidBody.AddForce(springForce); // Kraft in N hinzufuegen
    \end{lstlisting}
\end{KR}

\begin{example}
    Für die Umrechnung der Geschwindigkeit von km/h in m/s teilt man durch 3,6:
    $v[m/s] = \frac{v[km/h]}{3,6}$
    
    Beispiel: 72 km/h = 72 / 3,6 = 20 m/s
\end{example}