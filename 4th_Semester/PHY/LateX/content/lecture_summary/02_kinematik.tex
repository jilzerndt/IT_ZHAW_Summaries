\section{Kinematik der Translation}

\subsection{Grundlagen der Kinematik}
\begin{definition}{Kinematik}\\
    Die Kinematik beschreibt die Bewegung eines Körpers, ohne auf deren Ursachen einzugehen. Die Bewegung eines Körpers ist vollständig beschrieben durch:
    \begin{itemize}
        \item seinen Ort (Vektor $\vec{r}$)
        \item seine Geschwindigkeit ($\vec{v}$)
        \item seine Beschleunigung ($\vec{a}$)
    \end{itemize}
    Diese drei Größen hängen durch Ableiten bzw. Integrieren zusammen.
\end{definition}

\begin{formula}{Zusammenhänge zwischen Ort\, Geschwindigkeit und Beschleunigung}\\
    \begin{itemize}
        \item Geschwindigkeit = Ableitung des Ortes nach der Zeit:\\
        $\vec{v} = \frac{d\vec{r}}{dt}$
        \item Beschleunigung = Ableitung der Geschwindigkeit nach der Zeit:\\
        $\vec{a} = \frac{d\vec{v}}{dt}$
        \item Ort = Integral der Geschwindigkeit nach der Zeit:\\
        $\vec{r} = \int \vec{v} \, dt$
        \item Geschwindigkeit = Integral der Beschleunigung nach der Zeit:\\
        $\vec{v} = \int \vec{a} \, dt$
    \end{itemize}
\end{formula}

\subsection{Mittlere Geschwindigkeit und Beschleunigung}
\begin{definition}{Mittlere Geschwindigkeit}\\
    Die mittlere Geschwindigkeit ist die Änderung des Ortes dividiert durch die dafür benötigte Zeit:
    \begin{equation}
        \bar{v}_x = \frac{\Delta r_x}{\Delta t}
    \end{equation}
    Sie stellt den Durchschnittswert über das betrachtete Zeitintervall dar.
\end{definition}

\begin{definition}{Mittlere Beschleunigung}\\
    Die mittlere Beschleunigung ist die Änderung der Geschwindigkeit dividiert durch die dafür benötigte Zeit:
    \begin{equation}
        \bar{a}_x = \frac{\Delta v_x}{\Delta t}
    \end{equation}
\end{definition}

\begin{concept}{Differenzenquotient vs. Differentialquotient}\\
    \begin{itemize}
        \item Der Differenzenquotient (mittlere Geschwindigkeit) ist eine Approximation über ein endliches Zeitintervall: $\frac{\Delta r_x}{\Delta t}$
        \item Der Differentialquotient (Momentangeschwindigkeit) ist der Grenzwert für ein infinitesimal kleines Zeitintervall: $\lim_{\Delta t \to 0} \frac{\Delta r_x}{\Delta t} = \frac{dr_x}{dt}$
        \item In Unity wird mit fixen Zeitschritten $\Delta t = 20$ ms gerechnet, was einer Abtastfrequenz von $f_{sample} = 50$ Hz entspricht
    \end{itemize}
\end{concept}

\subsection{Momentangeschwindigkeit und -beschleunigung}
\begin{definition}{Momentangeschwindigkeit}\\
    Die Momentangeschwindigkeit zur Zeit $t_0$ ist definiert als:
    \begin{equation}
        \vec{v}(t_0) = \lim_{t_1 \to t_0} \frac{\Delta \vec{r}}{t_1 - t_0} = \frac{d\vec{r}}{dt}
    \end{equation}
    Sie entspricht geometrisch der Steigung der Tangente im Punkt $(t_0, r_x(t_0))$.
\end{definition}

\begin{remark}
    Der Betrag der Geschwindigkeit wird oft als Schnelligkeit bezeichnet:
    \begin{equation}
        |\vec{v}| = \sqrt{v_x^2 + v_y^2 + v_z^2}
    \end{equation}
    Bei gleichbleibender Schnelligkeit kann sich dennoch die Richtung der Geschwindigkeit ändern, z.B. bei einer Kreisbewegung.
\end{remark}

\begin{formula}{Fläche unter dem Geschwindigkeits-Zeit-Diagramm}\\
    Bei einer Bewegung mit variablem $v(t)$ berechnet sich die zurückgelegte Strecke als Fläche unter der $v$-$t$-Kurve:
    \begin{equation}
        \Delta x = \int_{t_1}^{t_2} v(t) \, dt
    \end{equation}
\end{formula}

\subsection{Integration und Differentiation}
\begin{formula}{Ableitungsregeln}\\
    \begin{itemize}
        \item Konstante Summanden: $\frac{d}{dt}(C) = 0$
        \item Potenzfunktionen: $\frac{d}{dt}(at^n) = a \cdot n \cdot t^{n-1}$
        \item Exponentialfunktion: $\frac{d}{dx}(e^x) = e^x$
        \item Logarithmus: $\frac{d}{dx}(\ln x) = \frac{1}{x}$
        \item Sinus/Kosinus: $\frac{d}{dx}(\sin x) = \cos x$, $\frac{d}{dx}(\cos x) = -\sin x$
    \end{itemize}
\end{formula}

\begin{formula}{Regeln für zusammengesetzte Funktionen}\\
    \begin{itemize}
        \item Summenregel: $\frac{d}{dt}(f(t) + g(t)) = \frac{df}{dt} + \frac{dg}{dt}$
        \item Produktregel: $\frac{d}{dt}(f(t) \cdot g(t)) = \frac{df}{dt} \cdot g(t) + f(t) \cdot \frac{dg}{dt}$
        \item Kettenregel: $\frac{d}{dt}(f(g(t))) = \frac{df}{dg} \cdot \frac{dg}{dt}$
    \end{itemize}
\end{formula}

\begin{KR}{Berechnung von Bewegungen mit konstanter Beschleunigung}\\
    \paragraph{Gegebene Größen}
    \begin{itemize}
        \item Anfangsposition $r_0$
        \item Anfangsgeschwindigkeit $v_0$
        \item Konstante Beschleunigung $a$
    \end{itemize}
    
    \paragraph{Schritte zur Berechnung}
    \begin{enumerate}
        \item Geschwindigkeit in Abhängigkeit von der Zeit bestimmen:
        \begin{equation}
            v(t) = v_0 + at
        \end{equation}
        
        \item Position in Abhängigkeit von der Zeit bestimmen:
        \begin{equation}
            r(t) = r_0 + v_0t + \frac{1}{2}at^2
        \end{equation}
        
        \item Alternative Formel bei bekannter Strecke (ohne Zeit):
        \begin{equation}
            v^2 = v_0^2 + 2a(r - r_0)
        \end{equation}
    \end{enumerate}
\end{KR}

\begin{examplecode}{Bewegung in Unity implementieren}\\
    \begin{lstlisting}[language=csh, style=basesmol]
// Implementierung von Bewegungen mit konstanter Beschleunigung
void FixedUpdate() {
    // Aktuelle Zeit seit Start
    currentTime += Time.deltaTime;
    
    // Aktuelle Geschwindigkeit nach v = v0 + a*t berechnen
    float currentVelocity = initialVelocity + acceleration * currentTime;
    
    // Bewegung mit aktueller Geschwindigkeit
    Vector3 displacement = new Vector3(currentVelocity, 0, 0) * Time.deltaTime;
    transform.position += displacement;
    
    // Alternative: Direkte Berechnung der Position mit r = r0 + v0*t + 0.5*a*t^2
    // Vector3 newPosition = initialPosition + initialVelocity * currentTime + 
    //                       0.5f * acceleration * currentTime * currentTime;
    // transform.position = newPosition;
}
    \end{lstlisting}
\end{examplecode}

\begin{example2}{Freier Fall}\\
    Ein Körper fällt aus der Höhe $r_0$ mit Anfangsgeschwindigkeit $v_0 = 0$.
    
    \begin{itemize}
        \item Beschleunigung: $a(t) = -g$ (g = 9.81 m/s²)
        \item Geschwindigkeit: $v(t) = -gt$
        \item Position: $r(t) = r_0 - \frac{1}{2}gt^2$
    \end{itemize}
    
    Alternativ: Ein Körper wird mit Anfangsgeschwindigkeit $v_0$ nach oben geworfen:
    \begin{itemize}
        \item Maximale Höhe: $h_{max} = \frac{v_0^2}{2g}$ 
        \item Zeit bis zum höchsten Punkt: $t_{max} = \frac{v_0}{g}$
        \item Gesamtflugzeit: $t_{gesamt} = \frac{2v_0}{g}$
    \end{itemize}
\end{example2}