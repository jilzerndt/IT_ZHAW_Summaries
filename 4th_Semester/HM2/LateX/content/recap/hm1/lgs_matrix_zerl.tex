

\section{Numerische Lösung linearer Gleichungssysteme}

\subsection{Matrix-Zerlegungen}

\begin{definition}{Dreieckszerlegung}
Eine Matrix $A \in \mathbb{R}^{n\times n}$ kann zerlegt werden in:
\vspace{1mm}\\
\begin{minipage}[t]{0.5\textwidth}
    \textbf{Untere Dreiecksmatrix L:}\\
    $l_{ij} = 0$ für $j > i$\\
    Diagonale normiert ($l_{ii}=1$)
\end{minipage}
\hspace{3mm}
\begin{minipage}[t]{0.45\textwidth}
    \textbf{Obere Dreiecksmatrix R:}\\
    $r_{ij} = 0$ für $i > j$\\
    Diagonalelemente $\neq 0$
\end{minipage}
\end{definition}


\subsubsection{LR-Zerlegung}

\begin{theorem}{LR-Zerlegung}\\
Jede reguläre Matrix $A$, für die der Gauss-Algorithmus ohne Zeilenvertauschungen durchführbar ist, lässt sich zerlegen in:
$A = LR$
wobei $L$ eine normierte untere und $R$ eine obere Dreiecksmatrix ist.
\end{theorem}

\begin{KR}{LR-Zerlegung durchführen}
    $(E | A | E) \underbrace{\leadsto}_{Gauss} (P | R | L)$
    \vspace{-3mm}\\
\begin{enumerate}
    \item Zerlegung bestimmen:
    \begin{itemize}
        \item Gauss-Elimination durchführen
        \item Eliminationsfaktoren $-\frac{a_{ji}}{a_{ii}}$ in $L$ speichern
        \item Resultierende obere Dreiecksmatrix ist $R$
    \end{itemize}
    
    \item System lösen:
    \begin{itemize}
        \item Vorwärtseinsetzen: $Ly = b$
        \item Rückwärtseinsetzen: $Rx = y$
    \end{itemize}
    
    \item Bei Pivotisierung:
    \begin{itemize}
        \item Permutationsmatrix $P$ erstellen
        \item $PA = LR$ speichern
        \item $Ly = Pb$ lösen
    \end{itemize}
\end{enumerate}
\end{KR}

\begin{remark}
    $E$ = Einheitsmatrix, $P$ = Permutationsmatrix
\end{remark}


\begin{example2}{LR-Zerlegung}
$\underbrace{\begin{psmallmatrix}
-1 & 1 & 1\\
1 & -3 & -2\\
5 & 1 & 4
\end{psmallmatrix}}_{A}, \underbrace{\begin{psmallmatrix}
0\\
5\\
3
\end{psmallmatrix}}_{b}
\rightarrow \begin{vsmallmatrix}
0 & 0 & 1\\
0 & 1 & 0\\
1 & 0 & 0
\end{vsmallmatrix} \begin{vsmallmatrix}
-1 & 1 & 1\\
1 & -3 & -2\\
5 & 1 & 4
\end{vsmallmatrix}
\begin{vsmallmatrix}
1 & 0 & 0\\
0 & 1 & 0\\
0 & 0 & 1
\end{vsmallmatrix}$

\paragraph{Schritt 1: Erste Spalte}
Max. Element in 1. Spalte: $|a_{31}| = 5$, also Z1 und Z3 tauschen:

\begin{minipage}{0.5\textwidth}
$$\begin{vsmallmatrix}
0 & 0 & 1\\
0 & 1 & 0\\
1 & 0 & 0
\end{vsmallmatrix} \begin{vsmallmatrix}
5 & 1 & 4\\
1 & -3 & -2\\
-1 & 1 & 1
\end{vsmallmatrix}
\begin{vsmallmatrix}
1 & 0 & 0\\
0 & 1 & 0\\
0 & 0 & 1
\end{vsmallmatrix}$$
\end{minipage}
\begin{minipage}{0.45\textwidth}
    \vspace{2mm}
    Eliminationsfaktoren:\\
    $l_{21} = \frac{1}{5}, \quad l_{31} = -\frac{1}{5}$
\end{minipage}


Nach Elimination:
$\begin{vsmallmatrix}
0 & 0 & 1\\
0 & 1 & 0\\
1 & 0 & 0
\end{vsmallmatrix} \begin{vsmallmatrix}
5 & 1 & 4\\
0 & -3.2 & -2.8\\
0 & 1.2 & 1.8
\end{vsmallmatrix}
\begin{vsmallmatrix}
1 & 0 & 0\\
\frac{1}{5} & 1 & 0\\
-\frac{1}{5} & 0 & 1
\end{vsmallmatrix}$

\paragraph{Schritt 2: Zweite Spalte}
Max. Element in 2. Spalte unter Diagonale: $|-3.2| > |1.2|$, \\ keine Vertauschung nötig.
Eliminationsfaktor:
$l_{32} = \frac{1.2}{-3.2} = -\frac{3}{8}$
\vspace{2mm}\\
Nach Elimination:
$\underbrace{\begin{vsmallmatrix}
0 & 0 & 1\\
0 & 1 & 0\\
1 & 0 & 0
\end{vsmallmatrix}}_{P} \underbrace{\begin{vsmallmatrix}
5 & 1 & 4\\
0 & -3.2 & -2.8\\
0 & 0 & 0.75
\end{vsmallmatrix}}_{R}
\underbrace{\begin{vsmallmatrix}
1 & 0 & 0\\
\frac{1}{5} & 1 & 0\\
-\frac{1}{5} & -\frac{3}{8} & 1
\end{vsmallmatrix}}_{L}$

\paragraph{Lösung des Systems}
\begin{enumerate}
    \item $Pb = \begin{psmallmatrix} 3\\ 5\\ 0 \end{psmallmatrix}$
    \item Löse $Ly = Pb$ durch Vorwärtseinsetzen:
    $y = \begin{psmallmatrix} 3\\ 4.4\\ 2.25 \end{psmallmatrix}$
    \item Löse $Rx = y$ durch Rückwärtseinsetzen:
    $x = \begin{psmallmatrix} -1\\ -4\\ 3 \end{psmallmatrix}$
\end{enumerate}

Probe:
$Ax = \begin{psmallmatrix}
-1 & 1 & 1\\
1 & -3 & -2\\
5 & 1 & 4
\end{psmallmatrix} \begin{psmallmatrix} -1\\ -4\\ 3 \end{psmallmatrix} = \begin{psmallmatrix} 0\\ 5\\ 3 \end{psmallmatrix} = b$
\end{example2}

\columnbreak

\subsubsection{QR-Zerlegung}

\begin{concept}{QR-Zerlegung}\\
Eine orthogonale Matrix $Q \in \mathbb{R}^{n\times n}$ erfüllt: $Q^T Q = QQ^T = I_n$
\vspace{1mm}\\
Die QR-Zerlegung einer Matrix $A$ ist: $A = QR$
\vspace{1mm}\\
wobei $Q$ orthogonal und $R$ eine obere Dreiecksmatrix ist.
\end{concept}

\begin{definition}{Householder-Transformation}\\
Eine Householder-Matrix hat die Form:
$H = I_n - 2uu^T$

mit $u \in \mathbb{R}^n$, $\|u\| = 1$. Es gilt:
\begin{itemize}
    \item $H$ ist orthogonal ($H^T = H^{-1}$) und symmetrisch ($H^T = H$)
    \item $H^2 = I_n$
\end{itemize}
\end{definition}

\begin{KR}{QR-Zerlegung mit Householder}
\begin{enumerate}
    \item Initialisierung: $R := A$, $Q := I_n$
    \item Für $i = 1,\ldots,n-1$:
        \begin{itemize}
            \item Bilde Vektor $v_i$ aus i-ter Spalte von $R$ ab Position $i$
            \item $w_i := v_i + \text{sign}(v_{i1})\|v_i\|e_1$
            \item $u_i := w_i/\|w_i\|$
            \item $H_i := I_{n-i+1} - 2u_iu_i^T$
            \item Erweitere $H_i$ zu $Q_i$ durch $I_{i-1}$ links oben
            \item $R := Q_iR$ und $Q := QQ_i^T$
        \end{itemize}
\end{enumerate}
\end{KR}

\begin{example2}[breakable]{QR-Zerlegung mit Householder}
    %TODO: check if this is correct and/or relevant - either correct or replace with better example
$A = \begin{psmallmatrix}
2 & 5 & -1\\
-1 & -4 & 2\\
0 & 2 & 1
\end{psmallmatrix}$

\paragraph{Schritt 1: Erste Spalte}
Erste Spalte $a_1$ und Einheitsvektor $e_1$:
$a_1 = \begin{psmallmatrix} 2\\ -1\\ 0 \end{psmallmatrix}, \quad e_1 = \begin{psmallmatrix} 1\\ 0\\ 0 \end{psmallmatrix}$

Householder-Vektor für erste Spalte:
\vspace{1mm}
\begin{enumerate}
    \item Berechne Norm: $|a_1| = \sqrt{2^2 + (-1)^2 + 0^2} = \sqrt{5}$
    \vspace{1mm}
    \item Bestimme Vorzeichen: $\text{sign}(a_{11}) = \text{sign}(2) = 1$
         \begin{itemize}
              \item Wähle positives Vorzeichen, da erstes Element positiv
              \item Dies maximiert die erste Komponente von $v_1$
              \item Verhindert Auslöschung bei der Subtraktion
         \end{itemize}
         \vspace{1mm}
    \item $v_1 = a_1 + \text{sign}(a_{11})|a_1|e_1 = \begin{psmallmatrix} 2\\ -1\\ 0 \end{psmallmatrix} + \sqrt{5}\begin{psmallmatrix} 1\\ 0\\ 0 \end{psmallmatrix} = \begin{psmallmatrix} 2 + \sqrt{5}\\ -1\\ 0 \end{psmallmatrix}$
    \vspace{1mm}
    \item Normiere $v_1$: $|v_1| = \sqrt{(2 + \sqrt{5})^2 + 1} \Rightarrow
            u_1 = \frac{v_1}{|v_1|} = \begin{psmallmatrix} 0.91\\ -0.41\\ 0 \end{psmallmatrix}$
\end{enumerate}
\vspace{1mm}
Householder-Matrix berechnen:
$H_1 = I - 2u_1u_1^T = \begin{psmallmatrix} 
-0.67 & -0.75 & 0\\
-0.75 & 0.67 & 0\\
0 & 0 & 1
\end{psmallmatrix}$
A nach 1. Transformation:
$A^{(1)} = H_1A = \begin{psmallmatrix}
-\sqrt{5} & -6.71 & 0.45\\
0 & -0.89 & 1.79\\
0 & 2.00 & 1.00
\end{psmallmatrix}$
\paragraph{Schritt 2: Zweite Spalte}
Untermatrix für zweite Transformation:
$A_2 = \begin{psmallmatrix} -0.89 & 1.79\\ 2.00 & 1.00 \end{psmallmatrix}$

Householder-Vektor für zweite Spalte:
\vspace{1mm}
\begin{enumerate}
    \item $|a_2| = \sqrt{(-0.89)^2 + 2^2} = 2.19$
    \vspace{1mm}
    \item $\text{sign}(a_{22}) = \text{sign}(-0.89) = -1$ (da erstes Element negativ)
    \vspace{1mm}
    \item $v_2 = \begin{psmallmatrix} -0.89\\ 2.00 \end{psmallmatrix} - 2.19\begin{psmallmatrix} 1\\ 0 \end{psmallmatrix} = \begin{psmallmatrix} -3.09\\ 2.00 \end{psmallmatrix}$
    \vspace{1mm}
    \item $u_2 = \frac{v_2}{|v_2|} = \begin{psmallmatrix} -0.84\\ 0.54 \end{psmallmatrix}$
\end{enumerate}
\vspace{1mm}
Erweiterte Householder-Matrix: %TODO: nicer formatting!
$Q_2 = \begin{psmallmatrix}
1 & 0 & 0\\
0 & -0.41 & -0.91\\
0 & -0.91 & 0.41
\end{psmallmatrix}$

nach 2. Transformation:
$R = Q_2A^{(1)} = \begin{psmallmatrix}
-\sqrt{5} & -6.71 & 0.45\\
0 & -2.19 & 1.34\\
0 & 0 & -1.79
\end{psmallmatrix}$

\paragraph{Endergebnis}
Die QR-Zerlegung $A = QR$ ist:

$Q = H_1^TQ_2^T = \begin{psmallmatrix}
-0.89 & -0.45 & 0\\
0.45 & -0.89 & 0\\
0 & 0 & 1
\end{psmallmatrix},
R = \begin{psmallmatrix}
-\sqrt{5} & -6.71 & 0.45\\
0 & -2.19 & 1.34\\
0 & 0 & -1.79
\end{psmallmatrix}$

\paragraph{Probe}
\small
\begin{enumerate}
    \item $QR = A$ (bis auf Rundungsfehler)
    \item $Q^TQ = QQ^T = I$ (Orthogonalität)
    \item $R$ ist obere Dreiecksmatrix
\end{enumerate}

\paragraph{Wichtige Beobachtungen}
\small
\begin{itemize}
    \item Vorzeichenwahl bei $v_k$ ist entscheidend für numerische Stabilität
    \item Ein falsches Vorzeichen kann zu Auslöschung führen
    \item Betrag der Diagonalelemente in $R$ = Norm transformierter Spalten
    \item $Q$ ist orthogonal: Spaltenvektoren sind orthonormal
\end{itemize}
\end{example2}




\subsubsection{Iterative Verfahren}

\begin{definition}{Zerlegung der Systemmatrix} $A$ zerlegt in: $A = L + D + R$
    \small
\begin{itemize}
    \item $L$: streng untere Dreiecksmatrix
    \item $D$: Diagonalmatrix
    \item $R$: streng obere Dreiecksmatrix
\end{itemize}
\end{definition}

\begin{concept}{Jacobi-Verfahren}
Gesamtschrittverfahren 
\vspace{1mm}\\
Iteration: $x^{(k+1)} = -D^{-1}(L + R)x^{(k)} + D^{-1}b$
\vspace{1mm}\\
Komponentenweise:
$x_i^{(k+1)} = \frac{1}{a_{ii}}\left(b_i - \sum_{j=1,j\neq i}^n a_{ij}x_j^{(k)}\right)$
\end{concept}

\begin{concept}{Gauss-Seidel-Verfahren}
Einzelschrittverfahren 
\vspace{1mm}\\
Iteration: $x^{(k+1)} = -(D+L)^{-1}Rx^{(k)} + (D+L)^{-1}b$
\vspace{1mm}\\
Komponentenweise:\\
$x_i^{(k+1)} = \frac{1}{a_{ii}}\left(b_i - \sum_{j=1}^{i-1} a_{ij}x_j^{(k+1)} - \sum_{j=i+1}^n a_{ij}x_j^{(k)}\right)$
\end{concept}

\begin{theorem}{Konvergenzkriterien}
Ein iteratives Verfahren konvergiert, wenn:
\begin{enumerate}
    \item Die Matrix $A$ diagonaldominant ist:\\
    $|a_{ii}| > \sum_{j\neq i} |a_{ij}|$ für alle $i$
    \item Der Spektralradius der Iterationsmatrix kleiner 1 ist:\\
    $\rho(B) < 1$ mit $B$ als jeweilige Iterationsmatrix
\end{enumerate}
\end{theorem}

\begin{KR}{Implementierung von Jacobi- und Gauss-Seidel-Verfahren}
    \paragraph{Vorbereitungsphase}
    \begin{itemize}
        \item Matrix zerlegen in $A = L + D + R$
        \item Diagonaldominanz prüfen: $|a_{ii}| > \sum_{j\neq i} |a_{ij}|$ für alle $i$
        \item Sinnvolle Startwerte wählen (z.B. $x^{(0)}=0$ oder $x^{(0)}=b$)
        \item Toleranz $\epsilon$ und max. Iterationszahl $n_{max}$ festlegen
    \end{itemize}

    \paragraph{Verfahren durchführen}
    \begin{itemize}
        \item \textbf{Jacobi}: Komponentenweise parallel berechnen 
        \item \textbf{Gauss-Seidel}: Komponentenweise sequentiell berechnen 
    \end{itemize}

    \paragraph{Konvergenzprüfung/Abbruchkriterien}
    \begin{itemize}
        \item Absolute Änderung: $\|x^{(k+1)} - x^{(k)}\| < \epsilon$
        \item Relatives Residuum: $\frac{\|Ax^{(k)} - b\|}{\|b\|} < \epsilon$
        \item Maximale Iterationszahl: $k < n_{max}$
    \end{itemize}

    \paragraph{A-priori Fehlerabschätzung}
    \begin{itemize}
        \item Spektralradius $\rho$ der Iterationsmatrix bestimmen
        \item Benötigte Iterationen $n$ für Genauigkeit $\epsilon$:\\
        $n \geq \frac{\ln(\epsilon(1-\rho)/\|x^{(1)}-x^{(0)}\|)}{\ln(\rho)}$
    \end{itemize}
\end{KR}







\subsubsection{Fehleranalyse}

\begin{definition}{Matrix- und Vektornormen}\\
Eine Vektornorm $\|\cdot\|$ erfüllt für alle $x,y \in \mathbb{R}^n, \lambda \in \mathbb{R}$:
\begin{itemize}
    \item $\|x\| \geq 0$ und $\|x\| = 0 \Leftrightarrow x = 0$
    \item $\|\lambda x\| = |\lambda| \cdot \|x\|$
    \item $\|x + y\| \leq \|x\| + \|y\|$ (Dreiecksungleichung)
\end{itemize}
\end{definition}

\begin{concept}{Wichtige Normen}

\textbf{1-Norm:}
        $\|x\|_1 = \sum_{i=1}^n |x_i|,
        \|A\|_1 = \max_j \sum_{i=1}^n |a_{ij}|$

\textbf{2-Norm:}
        $\|x\|_2 = \sqrt{\sum_{i=1}^n x_i^2}, 
        \|A\|_2 = \sqrt{\rho(A^TA)}$

$\infty$\textbf{-Norm:}
        $\|x\|_\infty = \max_i |x_i|, 
        \|A\|_\infty = \max_i \sum_{j=1}^n |a_{ij}|$
\end{concept}

\begin{theorem}{Fehlerabschätzung für LGS}\\
Sei $\|\cdot\|$ eine Norm, $A \in \mathbb{R}^{n\times n}$ regulär und $Ax = b$, $A\tilde{x} = \tilde{b}$
\vspace{1mm}\\
\begin{minipage}[t]{0.47\textwidth}
    \textbf{Absoluter Fehler:}
    \vspace{-5mm}\\
    \begin{center}
        $\|x - \tilde{x}\| \leq \|A^{-1}\| \cdot \|b - \tilde{b}\|$
    \end{center}
\end{minipage}
\hspace{2mm}
\begin{minipage}[t]{0.47\textwidth}
    \textbf{Relativer Fehler:}
    \vspace{-5mm}\\
    \begin{center}
        $\frac{\|x - \tilde{x}\|}{\|x\|} \leq \text{cond}(A) \cdot \frac{\|b - \tilde{b}\|}{\|b\|}$
    \end{center}
\end{minipage}
\vspace{1mm}\\
Mit der Konditionszahl $\text{cond}(A) = \|A\| \cdot \|A^{-1}\|$
\end{theorem}

\begin{concept}{Konditionierung}\\
Die Konditionszahl beschreibt die numerische Stabilität eines LGS:
\begin{itemize}
    \item $\text{cond}(A) \approx 1$: gut konditioniert
    \item $\text{cond}(A) \gg 1$: schlecht konditioniert
    \item $\text{cond}(A) \to \infty$: singulär
\end{itemize}
\end{concept}

\begin{example2}{Konditionierung}
$A = \begin{psmallmatrix}
1 & 1\\
1 & 1.01
\end{psmallmatrix}, b = \begin{psmallmatrix}
2\\
2.01
\end{psmallmatrix}$\\
Konditionszahl:
$\text{cond}(A) = \|A\| \cdot \|A^{-1}\| \approx 400$
\paragraph{Fehlerabschätzung}
$\text{Absoluter Fehler: }\|x - \tilde{x}\| \leq 400 \cdot 0.01 = 4$ \\
$\text{Relativer Fehler: }\frac{\|x - \tilde{x}\|}{\|x\|} \leq 400 \cdot \frac{0.01}{2} = 2$
\end{example2}

\begin{KR}{Vergleich Lösungsverfahren}
$A = \begin{psmallmatrix}
1 & 2 & 0\\
2 & 1 & 2\\
0 & 2 & 1
\end{psmallmatrix}, \quad b = \begin{psmallmatrix}
1\\
2\\
3
\end{psmallmatrix}$

\small
\begin{itemize}
    \item Matrix ist symmetrisch und nicht streng diagonaldominant
    \item $\text{cond}_\infty(A) \approx 12.5$
\end{itemize}
\begin{center}
\begin{tabular}{l|ccc}
Verfahren & Iterationen & Residuum & Zeit\\
\hline
LR mit Pivot & 1 & $2.2\cdot10^{-16}$ & 1.0\\
QR & 1 & $2.2\cdot10^{-16}$ & 2.3\\
Jacobi & 12 & $1.0\cdot10^{-6}$ & 1.8\\
Gauss-Seidel & 7 & $1.0\cdot10^{-6}$ & 1.4\\
\end{tabular}
\end{center}
\vspace{-2mm}
\begin{itemize}
    \item Direkte Verfahren erreichen höhere Genauigkeit
    \item Iterative Verfahren brauchen mehrere Schritte
\end{itemize}
\end{KR}

\begin{example2}{Konvergenzverhalten}
$\begin{psmallmatrix}
4 & 1 & 0\\
1 & 4 & 1\\
0 & 1 & 4
\end{psmallmatrix}
\begin{psmallmatrix}
x_1\\
x_2\\
x_3
\end{psmallmatrix} =
\begin{psmallmatrix}
1\\
2\\
3
\end{psmallmatrix}$
\vspace{1mm}\\
\small
Die Matrix ist diagonaldominant:
$|a_{ii}| = 4 > 1 = \sum_{j\neq i} |a_{ij}|$
\vspace{1mm}\\
\begin{tabular}{c|cc|cc}
k & \multicolumn{2}{c|}{Residuum} & \multicolumn{2}{c}{Rel. Fehler}\\
& Jacobi & G-S & Jacobi & G-S\\
\hline
0 & 3.74 & 3.74 & - & -\\
1 & 0.94 & 0.47 & 0.935 & 0.468\\
2 & 0.23 & 0.06 & 0.246 & 0.125\\
3 & 0.06 & 0.01 & 0.065 & 0.017\\
4 & 0.01 & 0.001 & 0.016 & 0.002
\end{tabular}

\paragraph{Beobachtungen:}
\begin{itemize}
    \item Gauss-Seidel konvergiert etwa doppelt so schnell wie Jacobi
    \item Das Residuum fällt linear (geometrische Folge)
    \item Die Konvergenz ist gleichmäßig (keine Oszillationen)
\end{itemize}
\end{example2}

\subsubsection{Vorgehen und Implementation}

\begin{KR}{Systematisches Vorgehen bei LGS}
\begin{enumerate}
    \item Eigenschaften der Matrix analysieren:
    \begin{itemize}
        \item Diagonaldominanz prüfen
        \item Konditionszahl berechnen oder abschätzen
        \item Symmetrie erkennen
    \end{itemize}
    
    \item Verfahren auswählen:
    \begin{itemize}
        \item Direkte Verfahren: für kleinere Systeme
        \item Iterative Verfahren: für große, dünnbesetzte Systeme
        \item Spezialverfahren: für symmetrische/bandförmige Matrizen
    \end{itemize}
    
    \item Implementation planen:
    \begin{itemize}
        \item Pivotisierung bei Gauss berücksichtigen
        \item Speicherbedarf beachten
        \item Abbruchkriterien festlegen
    \end{itemize}
\end{enumerate}
\end{KR}

\begin{KR}{Zeilenvertauschungen verfolgen}
\begin{enumerate}
    \item Initialisiere $P = I_n$
    \item Für jede Vertauschung von Zeile $i$ und $j$:
    \begin{itemize}
        \item Erstelle $P_k$ durch Vertauschen von Zeilen $i,j$ in $I_n$
        \item Aktualisiere $P = P_k \cdot P$
        \item Wende Vertauschung auf Matrix an: $A := P_kA$
    \end{itemize}
    \item Bei der LR-Zerlegung mit Pivotisierung:
    \begin{itemize}
        \item $PA = LR$ 
        \item Löse $Ly = Pb$ und $Rx = y$
    \end{itemize}
\end{enumerate}
\end{KR}














