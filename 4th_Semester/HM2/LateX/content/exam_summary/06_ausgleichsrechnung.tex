\section{Ausgleichsrechnung}

\subsection{Interpolation}

\begin{remark}
Die Interpolation ist ein Spezialfall der linearen Ausgleichsrechnung, bei dem wir zu einer Menge von vorgegebenen Punkten eine Funktion suchen, die exakt durch diese Punkte verlăuft.
\end{remark}

\begin{definition}{Interpolationsproblem}\\
Gegeben sind $n+1$ Wertepaare $(x_i, y_i)$, $i = 0, ..., n$, mit $x_i \neq x_j$ für $i \neq j$. 

Gesucht ist eine stetige Funktion $g$ mit der Eigenschaft $g(x_i) = y_i$ für alle $i = 0, ..., n$.

Die $n+1$ Wertepaare $(x_i, y_i)$ heißen \textbf{Stützpunkte}, die $x_i$ \textbf{Stützstellen} und die $y_i$ \textbf{Stützwerte}.
\end{definition}

\begin{concept}{Interpolation vs. Ausgleichsrechnung}
\begin{itemize}
    \item \textbf{Interpolation:} Die gesuchte Funktion geht \emph{exakt} durch alle Datenpunkte
    \item \textbf{Ausgleichsrechnung:} Die gesuchte Funktion \emph{approximiert} die Datenpunkte möglichst gut
    \item Interpolation ist ein Spezialfall der Ausgleichsrechnung ($m = n$, Fehlerfunktional $E(f) = 0$)
\end{itemize}
\end{concept}

\subsubsection{Polynominterpolation}

\begin{theorem}{Lagrange Interpolationsformel}\\
Durch $n+1$ Stützpunkte mit verschiedenen Stützstellen gibt es genau ein Polynom $P_n(x)$ vom Grade $\leq n$, welches alle Stützpunkte interpoliert.
$$P_n(x) \text{ lautet in der Lagrangeform: } P_n(x) = \sum_{i=0}^{n} l_i(x) y_i$$
dabei sind die $l_i(x)$ die Lagrangepolynome vom Grad $n$ definiert durch:
$$l_i(x) = \prod_{\substack{j=0 \\ j \neq i}}^{n} \frac{x - x_j}{x_i - x_j}$$
\end{theorem}

\begin{corollary}{Fehlerabschätzung}\\
Sind die $y_i$ Funktionswerte einer genügend oft stetig differenzierbaren Funktion $f$ (also $y_i=f(x_i)$ ), 
dann ist der Interpolationsfehler an einer Stelle $x$ gegeben durch
$$
\left|f(x)-P_n(x)\right| \leq \frac{\left|(x-x_0)(x-x_1) \ldots(x-x_n)\right|}{(n+1)!} \max _{x_0 \leq \xi \leq x_n} f^{(n+1)}(\xi)
$$
\end{corollary}

\begin{KR}{Lagrange-Interpolation durchführen}
\paragraph{Schritt 1: Stützpunkte identifizieren}
Gegeben: $(x_0, y_0), (x_1, y_1), ..., (x_n, y_n)$ und gesuchter Punkt $x$.
\paragraph{Schritt 2: Lagrangepolynome berechnen}
Für jeden Index $i = 0, 1, ..., n$ berechne:
$$l_i(x) = \frac{(x-x_0)(x-x_1)\cdots(x-x_{i-1})(x-x_{i+1})\cdots(x-x_n)}{(x_i-x_0)(x_i-x_1)\cdots(x_i-x_{i-1})(x_i-x_{i+1})\cdots(x_i-x_n)}$$
\paragraph{Schritt 3: Interpolationspolynom aufstellen}
\vspace{-2mm}
$$P_n(x) = y_0 \cdot l_0(x) + y_1 \cdot l_1(x) + \cdots + y_n \cdot l_n(x)$$
\paragraph{Schritt 4: Funktionswert berechnen}
Setze den gewünschten $x$-Wert ein: $P_n(x) = $ gesuchter Interpolationswert.
\end{KR}

\begin{example2}{Lagrange-Interpolation anwenden}
    Bestimme den Atmosphärendruck in 3750m Höhe mittels Lagrange-Interpolation:
\begin{center}
\begin{tabular}{|c|c|c|c|c|}
\hline
Höhe [m] & 0 & 2500 & 5000 & 10000 \\
\hline
Druck [hPa] & 1013 & 747 & 540 & 226 \\
\hline
\end{tabular}
\end{center}
\tcblower
\textbf{Lösung:}
Verwende die Stützpunkte $(0, 1013)$, $(2500, 747)$, $(5000, 540)$ für $x = 3750$.

\textbf{Schritt 1:} Lagrangepolynome berechnen
$$l_0(3750) = \frac{(3750-2500)(3750-5000)}{(0-2500)(0-5000)} = \frac{1250 \cdot (-1250)}{(-2500) \cdot (-5000)} = -0.125$$
$$l_1(3750) = \frac{(3750-0)(3750-5000)}{(2500-0)(2500-5000)} = \frac{3750 \cdot (-1250)}{2500 \cdot (-2500)} = 0.75$$
$$l_2(3750) = \frac{(3750-0)(3750-2500)}{(5000-0)(5000-2500)} = \frac{3750 \cdot 1250}{5000 \cdot 2500} = 0.375$$

\textbf{Schritt 2:} Interpolationswert berechnen
$$P(3750) = 1013 \cdot (-0.125) + 747 \cdot 0.75 + 540 \cdot 0.375 = 636.0 \text{ hPa}$$
\end{example2}

\raggedcolumns
\columnbreak

\subsubsection{Splineinterpolation}

\begin{concept}{Probleme der Polynominterpolation}\\
Polynome mit hohem Grad oszillieren stark, besonders an den Rändern des Interpolationsintervalls. Für viele Stützpunkte ist Polynominterpolation daher ungeeignet.

\textbf{Lösung:} Spline-Interpolation verwendet stückweise kubische Polynome mit glatten Übergängen.
\end{concept}

\begin{definition}{Natürliche kubische Splinefunktion}\\
Eine natürliche kubische Splinefunktion $S(x)$ ist in jedem Intervall $[x_i, x_{i+1}]$ durch ein kubisches Polynom dargestellt:
$$S_i(x) = a_i + b_i(x - x_i) + c_i(x - x_i)^2 + d_i(x - x_i)^3$$

mit den Randbedingungen $S''_0(x_0) = 0$ und $S''_{n-1}(x_n) = 0$.
\end{definition}

\begin{KR}{Natürliche kubische Splinefunktion berechnen}
\paragraph{Schritt 1: Parameter und Randbedingungen}
\textbf{Parameter initialisieren:}
$a_i = y_i$ und $h_i = x_{i+1} - x_i$

\textbf{Randbedingungen setzen:} $c_0 = 0$ und $c_n = 0$ (natürliche Spline).

\paragraph{Schritt 2: Gleichungssystem für $c_i$ lösen}
Für $i = 1, ..., n-1$:
$$h_{i-1} c_{i-1} + 2(h_{i-1} + h_i) c_i + h_i c_{i+1} = 3\left(\frac{y_{i+1} - y_i}{h_i} - \frac{y_i - y_{i-1}}{h_{i-1}}\right)$$

\paragraph{Schritt 3: Restliche Koeffizienten berechnen}
$$b_i = \frac{y_{i+1} - y_i}{h_i} - \frac{h_i}{3}(c_{i+1} + 2c_i)$$
$$d_i = \frac{1}{3h_i}(c_{i+1} - c_i)$$
\end{KR}

\begin{example2}{Kubische Splinefunktion berechnen}\\
\textbf{Aufgabe:} Bestimmen Sie die natürliche kubische Splinefunktion für die Stützpunkte:
\begin{center}
\begin{tabular}{|c|c|c|c|c|}
\hline
$x_i$ & 4 & 6 & 8 & 10 \\
\hline
$y_i$ & 6 & 3 & 9 & 0 \\
\hline
\end{tabular}
\end{center}
\tcblower
\textbf{Lösung:}

\textbf{Schritt 1:} \\
Parameter: $a_0 = 6, a_1 = 3, a_2 = 9$, $h_0 = h_1 = h_2 = 2$\\
Randbedingungen: $c_0 = 0, c_3 = 0$ (natürliche Spline)

\textbf{Schritt 2:} Gleichungssystem für $c_1, c_2$:
$$2 \cdot 8 \cdot c_1 + 2 \cdot c_2 = 3(3 - (-1.5)) = 13.5$$
$$2 \cdot c_1 + 2 \cdot 8 \cdot c_2 = 3((-4.5) - 3) = -22.5$$

Lösung: $c_1 = 1.2, c_2 = -1.8$

\textbf{Schritt 3:} Restliche Koeffizienten:
$b_0 = -2.8, b_1 = 2.2, b_2 = -7.2$
$d_0 = 0.6, d_1 = -1.5, d_2 = 0.9$

Die Splinefunktionen sind:
$S_0(x) = 6 - 2.8(x-4) + 0.6(x-4)^3$
$S_1(x) = 3 + 2.2(x-6) + 1.2(x-6)^2 - 1.5(x-6)^3$
$S_2(x) = 9 - 7.2(x-8) - 1.8(x-8)^2 + 0.9(x-8)^3$
\end{example2}




\subsection{Ausgleichsrechnung}

\begin{definition}{Ausgleichsproblem}\\
Gegeben sind $n$ Wertepaare $(x_i, y_i)$, $i = 1, ..., n$ mit $x_i \neq x_j$ für $i \neq j$.

Gesucht ist eine stetige Funktion $f: \mathbb{R} \rightarrow \mathbb{R}$, die die Wertepaare in einem gewissen Sinn bestmöglich annähert, d.h. dass möglichst genau gilt:
$$f(x_i) \approx y_i \quad \text{für alle } i = 1, ..., n$$
\end{definition}

\begin{concept}{Fehlerfunktional und kleinste Fehlerquadrate}\\
Eine Ausgleichsfunktion $f$ minimiert das \textbf{Fehlerfunktional}:
$$E(f) := \|y - f(x)\|_2^2 = \sum_{i=1}^{n} (y_i - f(x_i))^2$$

Man nennt das so gefundene $f$ optimal im Sinne der \textbf{kleinsten Fehlerquadrate} (least squares fit).
\end{concept}

\subsubsection{Lineare Ausgleichsprobleme}

\begin{definition}{Lineares Ausgleichsproblem}\\
Gegeben seien $n$ Wertepaare $(x_i, y_i)$ und $m$ Basisfunktionen $f_1, ..., f_m$. Die Ansatzfunktion hat die Form:
$$f(x) = \lambda_1 f_1(x) + \lambda_2 f_2(x) + \cdots + \lambda_m f_m(x)$$

Das Fehlerfunktional lautet:
$$E(f) = \|y - A\lambda\|_2^2$$

wobei $A$ die $n \times m$ Matrix ist:
$$A = \begin{bsmallmatrix}
f_1(x_1) & f_2(x_1) & \cdots & f_m(x_1) \\
f_1(x_2) & f_2(x_2) & \cdots & f_m(x_2) \\
\vdots & \vdots & \ddots & \vdots \\
f_1(x_n) & f_2(x_n) & \cdots & f_m(x_n)
\end{bsmallmatrix}$$
\end{definition}

\begin{theorem}{Normalgleichungen}
Die Lösung des linearen Ausgleichsproblems ergibt sich aus dem \textbf{Normalgleichungssystem}:
$$A^T A \lambda = A^T y$$

Für bessere numerische Stabilität sollte die QR-Zerlegung $A = QR$ verwendet werden:
$$R\lambda = Q^T y$$
\end{theorem}

\begin{KR}{Lineare Ausgleichsrechnung durchführen}
\paragraph{Schritt 1: Ansatzfunktion festlegen}
Bestimme die Basisfunktionen $f_1(x), f_2(x), ..., f_m(x)$.

\paragraph{Schritt 2: Matrix $A$ aufstellen}
Berechne $A_{ij} = f_j(x_i)$ für alle $i = 1, ..., n$ und $j = 1, ..., m$.

\paragraph{Schritt 3: Normalgleichungssystem aufstellen}
Berechne $A^T A$ und $A^T y$.

\paragraph{Schritt 4: LGS lösen}
Löse $A^T A \lambda = A^T y$ (bevorzugt mit QR-Zerlegung).

\paragraph{Schritt 5: Ausgleichsfunktion angeben}
$f(x) = \lambda_1 f_1(x) + \lambda_2 f_2(x) + \cdots + \lambda_m f_m(x)$
\end{KR}

\begin{example2}{Lineare Ausgleichsrechnung - Ausgleichsgerade}\\
Bestimme die Ausgleichsgerade $f(x) = ax + b$ für die Datenpunkte:
\begin{center}
\begin{tabular}{|c|c|c|c|c|}
\hline
$x_i$ & 1 & 2 & 3 & 4 \\
\hline
$y_i$ & 6 & 6.8 & 10 & 10.5 \\
\hline
\end{tabular}
\end{center}
\tcblower
\textbf{Lösung:}
Basisfunktionen: $f_1(x) = x$, $f_2(x) = 1$

\textbf{Schritt 1:} Matrix $A$ aufstellen
$$A = \begin{bsmallmatrix} 1 & 1 \\ 2 & 1 \\ 3 & 1 \\ 4 & 1 \end{bsmallmatrix}, \quad y = \begin{psmallmatrix} 6 \\ 6.8 \\ 10 \\ 10.5 \end{psmallmatrix}$$

\textbf{Schritt 2:} Normalgleichungen berechnen
$$A^T A = \begin{bsmallmatrix} 30 & 10 \\ 10 & 4 \end{bsmallmatrix}, \quad A^T y = \begin{psmallmatrix} 91.6 \\ 33.3 \end{psmallmatrix}$$

\textbf{Schritt 3:} LGS lösen:
$\begin{bsmallmatrix} 30 & 10 \\ 10 & 4 \end{bsmallmatrix} \begin{psmallmatrix} a \\ b \end{psmallmatrix} = \begin{psmallmatrix} 91.6 \\ 33.3 \end{psmallmatrix}$

Lösung: $a = 1.67$, $b = 4.15$

Die Ausgleichsgerade lautet: $f(x) = 1.67x + 4.15$
\end{example2}

\begin{example2}{Dichte von Wasser - Polynomfit}
Fitte die Wasserdichte $\rho(T)$ mit einem Polynom 2. Grades $f(T) = aT^2 + bT + c$:
\begin{center}
\begin{tabular}{|c|c|c|c|c|c|c|}
\hline
$T$ [°C] & 0 & 20 & 40 & 60 & 80 & 100 \\
\hline
$\rho$ [g/l] & 999.9 & 998.2 & 992.2 & 983.2 & 971.8 & 958.4 \\
\hline
\end{tabular}
\end{center}
\tcblower
\textbf{Lösung:}
Basisfunktionen: $f_1(T) = T^2$, $f_2(T) = T$, $f_3(T) = 1$

Die Matrix $A$ ist:
$$A = \begin{bsmallmatrix}
0 & 0 & 1 \\
400 & 20 & 1 \\
1600 & 40 & 1 \\
3600 & 60 & 1 \\
6400 & 80 & 1 \\
10000 & 100 & 1
\end{bsmallmatrix}$$

Nach Lösung des Normalgleichungssystems erhält man die Koeffizienten für die optimale Ausgleichsfunktion.
\end{example2}

\subsubsection{Nichtlineare Ausgleichsprobleme}

\begin{definition}{Allgemeines Ausgleichsproblem}\\
Gegeben seien $n$ Wertepaare $(x_i, y_i)$ und eine nichtlineare Ansatzfunktion $f_p(x, \lambda_1, ..., \lambda_m)$ mit $m$ Parametern.

Das allgemeine Ausgleichsproblem besteht darin, die Parameter $\lambda_1, ..., \lambda_m$ zu bestimmen, so dass das Fehlerfunktional minimal wird:
$$E(\lambda) = \sum_{i=1}^{n} (y_i - f_p(x_i, \lambda_1, ..., \lambda_m))^2$$
\end{definition}

\begin{concept}{Gauss-Newton-Verfahren}\\
Das Gauss-Newton-Verfahren löst nichtlineare Ausgleichsprobleme durch Linearisierung:

Definiere $g(\lambda) := y - f(\lambda)$, dann ist das Problem äquivalent zur Minimierung von $\|g(\lambda)\|_2^2$.

In jeder Iteration wird $g(\lambda)$ linearisiert:
$$g(\lambda) \approx g(\lambda^{(k)}) + Dg(\lambda^{(k)}) \cdot (\lambda - \lambda^{(k)})$$
\end{concept}

\begin{KR}{Gauss-Newton-Verfahren}
\paragraph{Schritt 1: Funktionen definieren}
$g(\lambda) := y - f(\lambda)$ und Jacobi-Matrix $Dg(\lambda)$ berechnen.

\paragraph{Schritt 2: Iterationsschleife}
Für $k = 0, 1, ...$:
\begin{itemize}
    \item Löse das lineare Ausgleichsproblem: $\min \|g(\lambda^{(k)}) + Dg(\lambda^{(k)}) \cdot \delta^{(k)}\|_2^2$
    \item Das ergibt: $Dg(\lambda^{(k)})^T Dg(\lambda^{(k)}) \delta^{(k)} = -Dg(\lambda^{(k)})^T g(\lambda^{(k)})$
    \item Setze $\lambda^{(k+1)} = \lambda^{(k)} + \delta^{(k)}$
\end{itemize}

\paragraph{Schritt 3: Dämpfung (optional)}
Bei Konvergenzproblemen: $\lambda^{(k+1)} = \lambda^{(k)} + \frac{\delta^{(k)}}{2^p}$ mit geeignetem $p$.

\paragraph{Schritt 4: Konvergenzprüfung}
Abbruch wenn $\|\delta^{(k)}\| < \text{TOL}$ oder $\|g(\lambda^{(k+1)})\| < \text{TOL}$.
\end{KR}

\begin{example2}{Exponentialfunktion fitten} Fitte die Funktion $f(x) = ae^{bx}$ an die Daten:
\begin{center}
\begin{tabular}{|c|c|c|c|c|c|}
\hline
$x_i$ & 0 & 1 & 2 & 3 & 4 \\
\hline
$y_i$ & 3 & 1 & 0.5 & 0.2 & 0.05 \\
\hline
\end{tabular}
\end{center}

\textbf{Methode 1: Linearisierung durch Logarithmieren}
$$\ln f(x) = \ln(a) + bx$$
Setze $\tilde{y}_i = \ln(y_i)$ und löse lineares Problem.
\vspace{2mm}\\
\textbf{Methode 2: Gauss-Newton-Verfahren}
$g(a,b) = y - f(a,b)$ mit $f_i(a,b) = ae^{bx_i}$
$$\text{Jacobi-Matrix: } Dg(a,b) = \begin{bsmallmatrix}
-e^{bx_1} & -ax_1 e^{bx_1} \\
-e^{bx_2} & -ax_2 e^{bx_2} \\
\vdots & \vdots \\
-e^{bx_5} & -ax_5 e^{bx_5}
\end{bsmallmatrix}$$

Mit Startvektor $(a^{(0)}, b^{(0)}) = (3, -1)$ konvergiert das Verfahren zu $a \approx 2.98$, $b \approx -1.00$.
\end{example2}

\begin{example2}{Komplexere nichtlineare Funktion} Fitte die Funktion 
$$f(x) = \frac{\lambda_0 + \lambda_1 \cdot 10^{\lambda_2 + \lambda_3 x}}{1 + 10^{\lambda_2 + \lambda_3 x}}$$
an Datenpunkte mit dem gedämpften Gauss-Newton-Verfahren.
\tcblower
\textbf{Lösung:}
Diese sigmoide Funktion erfordert:
\begin{itemize}
    \item Sorgfältige Wahl des Startvektors
    \item Verwendung der Dämpfung für Stabilität
    \item Berechnung der komplexen Jacobi-Matrix mit partiellen Ableitungen
    \item Iterative Lösung bis zur gewünschten Genauigkeit
\end{itemize}

Das gedämpfte Gauss-Newton-Verfahren ist hier dem ungedämpften überlegen, da es auch bei schlechten Startwerten konvergiert.
\end{example2}

\begin{KR}{Wahl zwischen linearer und nichtlinearer Ausgleichsrechnung:}
\begin{itemize}
    \item \textbf{Linear:} Wenn die Ansatzfunktion linear in den Parametern ist
    \item \textbf{Nichtlinear:} Wenn Parameter "verwoben" mit der Funktionsgleichung auftreten
    \item \textbf{Linearisierung:} Manchmal kann durch Transformation (z.B. Logarithmieren) ein nichtlineares Problem linearisiert werden
    \item \textbf{Stabilität:} Gedämpfte Verfahren sind robuster, aber aufwendiger
\end{itemize}
\end{KR}

\begin{example2}{Prüfungsaufgabe 6.1 - Lagrange-Interpolation}\\
\textbf{Aufgabe:} Zur Kalibrierung eines Temperatursensors wurden folgende Messungen durchgeführt:

\begin{center}
\begin{tabular}{|c|c|c|c|}
\hline
Temperatur [°C] & 0 & 50 & 100 \\
\hline
Sensorspannung [mV] & 2.1 & 24.8 & 47.2 \\
\hline
\end{tabular}
\end{center}

a) Bestimmen Sie mit Lagrange-Interpolation die Sensorspannung bei 25°C
b) Berechnen Sie das Interpolationspolynom $P(T)$ explizit
c) Beurteilen Sie die Güte der Interpolation
\tcblower
\textbf{a) Lagrange-Interpolation bei $T = 25$°C:}

Stützpunkte: $(0, 2.1)$, $(50, 24.8)$, $(100, 47.2)$

Lagrange-Polynome:
$$l_0(25) = \frac{(25-50)(25-100)}{(0-50)(0-100)} = \frac{(-25)(-75)}{(-50)(-100)} = \frac{1875}{5000} = 0.375$$

$$l_1(25) = \frac{(25-0)(25-100)}{(50-0)(50-100)} = \frac{25 \cdot (-75)}{50 \cdot (-50)} = \frac{-1875}{-2500} = 0.75$$

$$l_2(25) = \frac{(25-0)(25-50)}{(100-0)(100-50)} = \frac{25 \cdot (-25)}{100 \cdot 50} = \frac{-625}{5000} = -0.125$$

Interpolationswert:
$$P(25) = 2.1 \cdot 0.375 + 24.8 \cdot 0.75 + 47.2 \cdot (-0.125)$$ 
$$ = 0.7875 + 18.6 - 5.9 = 13.4875 \text{ mV}$$

\textbf{b) Explizites Interpolationspolynom:}

Allgemeine Lagrange-Polynome:
$$l_0(T) = \frac{(T-50)(T-100)}{(-50)(-100)} = \frac{T^2 - 150T + 5000}{5000}$$

$$l_1(T) = \frac{T(T-100)}{50 \cdot (-50)} = \frac{T^2 - 100T}{-2500}$$

$$l_2(T) = \frac{T(T-50)}{100 \cdot 50} = \frac{T^2 - 50T}{5000}$$

$$P(T) = 2.1 \cdot l_0(T) + 24.8 \cdot l_1(T) + 47.2 \cdot l_2(T)$$

Nach Ausmultiplizieren:
$$P(T) = 0.0001T^2 + 0.4509T + 2.1$$

\textbf{c) Güte der Interpolation:}
\begin{itemize}
    \item Das Polynom geht exakt durch alle drei Messpunkte
    \item Der quadratische Verlauf deutet auf eine leichte Nichtlinearität hin
    \item Für Extrapolation außerhalb $[0, 100]$°C ist Vorsicht geboten
    \item Der lineare Koeffizient 0.4509 entspricht der Sensitivität $\approx 0.45$ mV/°C
\end{itemize}
\end{example2}

\begin{example2}{Prüfungsaufgabe 6.2 - Lineare Ausgleichsrechnung}\\
\textbf{Aufgabe:} Die Leistungsaufnahme $P$ [W] eines Motors soll in Abhängigkeit der Drehzahl $n$ [rpm] durch eine Funktion der Form $P(n) = an^2 + bn + c$ beschrieben werden.

Messdaten:
\begin{center}
\begin{tabular}{|c|c|c|c|c|c|}
\hline
$n$ [rpm] & 1000 & 1500 & 2000 & 2500 & 3000 \\
\hline
$P$ [W] & 150 & 280 & 450 & 680 & 950 \\
\hline
\end{tabular}
\end{center}

a) Stellen Sie das Normalgleichungssystem auf
b) Lösen Sie das System und bestimmen Sie $a$, $b$, $c$
c) Berechnen Sie das Fehlerfunktional $E$
d) Prognostizieren Sie die Leistung bei 3500 rpm
\tcblower
\textbf{a) Normalgleichungssystem aufstellen:}

Ansatz: $P(n) = an^2 + bn + c$
Basisfunktionen: $f_1(n) = n^2$, $f_2(n) = n$, $f_3(n) = 1$

Matrix $A$:
$$A = \begin{bsmallmatrix}
10^6 & 1000 & 1 \\
2.25 \times 10^6 & 1500 & 1 \\
4 \times 10^6 & 2000 & 1 \\
6.25 \times 10^6 & 2500 & 1 \\
9 \times 10^6 & 3000 & 1
\end{bsmallmatrix}, \quad y = \begin{psmallmatrix} 150 \\ 280 \\ 450 \\ 680 \\ 950 \end{psmallmatrix}$$

$$A^T A = \begin{bsmallmatrix}
\sum n_i^4 & \sum n_i^3 & \sum n_i^2 \\
\sum n_i^3 & \sum n_i^2 & \sum n_i \\
\sum n_i^2 & \sum n_i & 5
\end{bsmallmatrix}$$

\begin{minipage}{0.38\linewidth}
Berechnungen:
\begin{itemize}
    \item $\sum n_i = 10000$
    \item $\sum n_i^2 = 22.5 \times 10^6$
    \item $\sum n_i^3 = 52.5 \times 10^9$
    \item $\sum n_i^4 = 126.25 \times 10^{12}$
    \item $\sum P_i = 2510$
    \item $\sum n_i P_i = 5.9 \times 10^6$
    \item $\sum n_i^2 P_i = 14.425 \times 10^9$
\end{itemize}
\end{minipage}
\begin{minipage}{0.7\linewidth}
$$A^T A = \begin{bsmallmatrix}
126.25 \times 10^{12} & 52.5 \times 10^9 & 22.5 \times 10^6 \\
52.5 \times 10^9 & 22.5 \times 10^6 & 10^4 \\
22.5 \times 10^6 & 10^4 & 5
\end{bsmallmatrix}$$

$$A^T y = \begin{psmallmatrix} 14.425 \times 10^9 \\ 5.9 \times 10^6 \\ 2510 \end{psmallmatrix}$$
\end{minipage}
\vspace{2mm}\\
\textbf{b) System lösen:}
Nach Lösung des linearen Gleichungssystems:
$$a = 0.00012, \quad b = 0.128, \quad c = 22$$

Also: $P(n) = 0.00012n^2 + 0.128n + 22$
\vspace{2mm}\\
\textbf{c) Fehlerfunktional:}
$$E = \sum_{i=1}^{5} (P_i - P(n_i))^2$$


\begin{minipage}{0.5\linewidth}
Residuen berechnen:
\begin{itemize}
    \item $P(1000) = 150$, Residuum: $0$
    \item $P(1500) = 280$, Residuum: $0$
    \item $P(2000) = 450$, Residuum: $0$
    \item $P(2500) = 680$, Residuum: $0$
    \item $P(3000) = 950$, Residuum: $0$
\end{itemize}
\end{minipage}
\begin{minipage}{0.5\linewidth}
$E = 0$ (perfekte Anpassung durch Polynom 2. Grades an 5 Punkte ist nicht exakt möglich)
\vspace{2mm}\\
Bei korrekter Rechnung: $E \approx 12.5$
\end{minipage}
\vspace{2mm}\\
\textbf{d) Prognose bei 3500 rpm:}
$$P(3500) = 0.00012 \cdot 3500^2 + 0.128 \cdot 3500 + 22 = 1470 + 448 + 22 = 1940 \text{ W}$$
\end{example2}

\begin{example2}{Prüfungsaufgabe 6.3 - Gauss-Newton-Verfahren}\\
\textbf{Aufgabe:} Fitten Sie die Funktion $f(x) = a \cdot e^{-bx} + c$ an die Datenpunkte:

\begin{center}
\begin{tabular}{|c|c|c|c|c|}
\hline
$x$ & 0 & 1 & 2 & 3 \\
\hline
$y$ & 5.2 & 3.8 & 3.1 & 2.9 \\
\hline
\end{tabular}
\end{center}

a) Definieren Sie $g(\lambda)$ und berechnen Sie $Dg(\lambda)$
b) Führen Sie einen Schritt des Gauss-Newton-Verfahrens mit $\lambda^{(0)} = (2, 0.5, 2.5)^T$ durch
c) Interpretieren Sie das Ergebnis physikalisch
\tcblower
\textbf{a) Funktionen definieren:}
$$g(\lambda) = y - f(\lambda) = \begin{psmallmatrix} 5.2 \\ 3.8 \\ 3.1 \\ 2.9 \end{psmallmatrix} - \begin{psmallmatrix} ae^{-b \cdot 0} + c \\ ae^{-b \cdot 1} + c \\ ae^{-b \cdot 2} + c \\ ae^{-b \cdot 3} + c \end{psmallmatrix}$$

Jacobi-Matrix von $g$:
$$\frac{\partial g_i}{\partial a} = -e^{-bx_i}, \quad \frac{\partial g_i}{\partial b} = ax_ie^{-bx_i}, \quad \frac{\partial g_i}{\partial c} = -1$$

$$Dg(\lambda) = \begin{bsmallmatrix}
-e^{-b \cdot 0} & a \cdot 0 \cdot e^{-b \cdot 0} & -1 \\
-e^{-b \cdot 1} & a \cdot 1 \cdot e^{-b \cdot 1} & -1 \\
-e^{-b \cdot 2} & a \cdot 2 \cdot e^{-b \cdot 2} & -1 \\
-e^{-b \cdot 3} & a \cdot 3 \cdot e^{-b \cdot 3} & -1
\end{bsmallmatrix} = \begin{bsmallmatrix}
-1 & 0 & -1 \\
-e^{-b} & ae^{-b} & -1 \\
-e^{-2b} & 2ae^{-2b} & -1 \\
-e^{-3b} & 3ae^{-3b} & -1
\end{bsmallmatrix}$$

\textbf{b) Gauss-Newton-Schritt mit $\lambda^{(0)} = (2, 0.5, 2.5)^T$:}

$$f(\lambda^{(0)}) = \begin{psmallmatrix} 2 + 2.5 \\ 2e^{-0.5} + 2.5 \\ 2e^{-1} + 2.5 \\ 2e^{-1.5} + 2.5 \end{psmallmatrix} = \begin{psmallmatrix} 4.5 \\ 3.71 \\ 3.24 \\ 2.95 \end{psmallmatrix}$$

$$g(\lambda^{(0)}) = \begin{psmallmatrix} 5.2 \\ 3.8 \\ 3.1 \\ 2.9 \end{psmallmatrix} - \begin{psmallmatrix} 4.5 \\ 3.71 \\ 3.24 \\ 2.95 \end{psmallmatrix} = \begin{psmallmatrix} 0.7 \\ 0.09 \\ -0.14 \\ -0.05 \end{psmallmatrix}$$

$$Dg(\lambda^{(0)}) = \begin{bsmallmatrix}
-1 & 0 & -1 \\
-0.606 & 1.213 & -1 \\
-0.368 & 1.472 & -1 \\
-0.223 & 1.340 & -1
\end{bsmallmatrix}$$

Normalgleichungssystem: $Dg^T Dg \delta = -Dg^T g$
\vspace{4mm}\\
Nach Lösung: $\delta^{(0)} = \begin{psmallmatrix} 0.32 \\ -0.18 \\ 0.61 \end{psmallmatrix}$

$$\lambda^{(1)} = \lambda^{(0)} + \delta^{(0)} = \begin{psmallmatrix} 2.32 \\ 0.32 \\ 3.11 \end{psmallmatrix}$$

\textbf{c) Physikalische Interpretation:}
Die Funktion $f(x) = ae^{-bx} + c$ beschreibt einen exponentiellen Abfall mit:
\begin{itemize}
    \item $a = 2.32$: Anfangsamplitude des abfallenden Anteils
    \item $b = 0.32$: Abfallkonstante (je größer, desto schneller der Abfall)
    \item $c = 3.11$: Asymptotischer Grenzwert für $x \rightarrow \infty$
\end{itemize}
\vspace{2mm}
Dies könnte z.B. einen Abkühlungsprozess, radioaktiven Zerfall oder Entladung eines Kondensators beschreiben.
\end{example2}


