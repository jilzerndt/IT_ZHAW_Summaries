\section{Numerische Integration}

\begin{definition}{Numerische Integration (Quadratur)}\\
Für eine Funktion $f: \mathbb{R} \rightarrow \mathbb{R}$ soll das bestimmte Integral
$$I(f) = \int_a^b f(x) dx$$
auf einem Intervall $[a,b]$ numerisch berechnet werden.

Quadraturverfahren haben im Allgemeinen die Form:
$$I(f) = \sum_{i=1}^{n} a_i f(x_i)$$
wobei die $x_i$ die \textbf{Stützstellen} oder \textbf{Knoten} und die $a_i$ die \textbf{Gewichte} der Quadraturformel sind.
\end{definition}

\begin{concept}{Warum numerische Integration?}\\
Im Gegensatz zur Ableitung können Integrale für viele Funktionen nicht analytisch gelöst werden. Beispiele:
\begin{itemize}
    \item $\int e^{-x^2} dx$ (Gaußsche Fehlerfunktion)
    \item $\int \sin(x^2) dx$ (Fresnel-Integral)
    \item $\int \frac{\sin x}{x} dx$ (Integral-Sinus)
\end{itemize}
Numerische Verfahren sind daher essentiell für praktische Anwendungen.
\end{concept}

\subsection{Newton-Cotes Formeln}

\subsubsection{Rechteck- und Trapezregel}

\begin{definition}{Einfache Rechteck- und Trapezregel}\\
Die \textbf{Rechteckregel} (Mittelpunktsregel) und die \textbf{Trapezregel} zur Approximation von $\int_a^b f(x) dx$ sind definiert als:

\textbf{Rechteckregel:} $Rf = f\left(\frac{a+b}{2}\right) \cdot (b-a)$

\textbf{Trapezregel:} $Tf = \frac{f(a) + f(b)}{2} \cdot (b-a)$
\end{definition}

\begin{concept}{Geometrische Interpretation}
\begin{itemize}
    \item \textbf{Rechteckregel:} \\ Approximiert die Fläche durch ein Rechteck mit Höhe $f(\frac{a+b}{2})$
    \item \textbf{Trapezregel:} Approximiert die Fläche durch ein Trapez \\ zwischen $(a,f(a))$ und $(b,f(b))$
\end{itemize}
\end{concept}

\begin{theorem}{Summierte Rechteck- und Trapezregel}\\
Sei $f: [a,b] \rightarrow \mathbb{R}$ stetig, $n \in \mathbb{N}$ die Anzahl Subintervalle mit konstanter Breite $h = \frac{b-a}{n}$ und $x_i = a + ih$ für $i = 0, ..., n$.
\vspace{2mm}\\
\textbf{Summierte Rechteckregel:}
$$Rf(h) = h \cdot \sum_{i=0}^{n-1} f\left(x_i + \frac{h}{2}\right)$$

\textbf{Summierte Trapezregel:}
$$Tf(h) = h \cdot \left(\frac{f(a) + f(b)}{2} + \sum_{i=1}^{n-1} f(x_i)\right)$$
\end{theorem}

\begin{KR}{Summierte Trapezregel anwenden}
\paragraph{Schritt 1: Parameter bestimmen}
Gegeben: Intervall $[a,b]$, Anzahl Subintervalle $n$\\
Berechne: Schrittweite $h = \frac{b-a}{n}$

\paragraph{Schritt 2: Stützstellen berechnen}
$x_i = a + ih$ für $i = 0, 1, ..., n$

\paragraph{Schritt 3: Funktionswerte berechnen}
$f(x_i)$ für alle Stützstellen

\paragraph{Schritt 4: Trapezregel anwenden}
$$Tf(h) = h \cdot \left(\frac{f(a) + f(b)}{2} + \sum_{i=1}^{n-1} f(x_i)\right)$$

\paragraph{Schritt 5: Für nicht-äquidistante Stützstellen}
$$Tf_{neq} = \sum_{i=0}^{n-1} \frac{f(x_i) + f(x_{i+1})}{2} \cdot (x_{i+1} - x_i)$$
\end{KR}

\begin{example2}{Trapezregel berechnen}\\
Berechne $\int_2^4 \frac{1}{x} dx$ mit der summierten Trapezregel für $n = 4$.
\tcblower
\textbf{Schritt 1:} $h = \frac{4-2}{4} = 0.5$

\textbf{Schritt 2:} Stützstellen: $x_0 = 2, x_1 = 2.5, x_2 = 3, x_3 = 3.5, x_4 = 4$

\textbf{Schritt 3:} Funktionswerte:
$$f(2) = 0.5, f(2.5) = 0.4, f(3) = 0.333, f(3.5) = 0.286, f(4) = 0.25$$

\textbf{Schritt 4:} Trapezregel anwenden:
$$Tf(0.5) = 0.5 \cdot \left(\frac{0.5 + 0.25}{2} + 0.4 + 0.333 + 0.286\right) = 0.697$$

\textbf{Vergleich:}\\
 Exakter Wert: $\ln(4) - \ln(2) = \ln(2) \approx 0.693$\\
Absoluter Fehler: $|0.697 - 0.693| = 0.004$
\end{example2}

\raggedcolumns
\columnbreak

\subsubsection{Simpson-Regel}

\begin{definition}{Simpson-Regel}\\
Die Simpson-Regel approximiert $f(x)$ durch ein Polynom 2. Grades an den Stellen $x_1 = a$, $x_2 = \frac{a+b}{2}$ und $x_3 = b$.

\textbf{Einfache Simpson-Regel:}
$$Sf = \frac{b-a}{6} \left(f(a) + 4f\left(\frac{a+b}{2}\right) + f(b)\right)$$

\textbf{Summierte Simpson-Regel:}
$$Sf(h) = \frac{h}{3} \left(\frac{1}{2}f(a) + \sum_{i=1}^{n-1} f(x_i) + 2\sum_{i=1}^{n} f\left(\frac{x_{i-1} + x_i}{2}\right) + \frac{1}{2}f(b)\right)$$
\end{definition}

\begin{concept}{Simpson-Regel als gewichtetes Mittel}\\
Die summierte Simpson-Regel kann als gewichtetes Mittel der summierten Trapez- und Rechteckregel interpretiert werden:
$$Sf(h) = \frac{1}{3}(Tf(h) + 2Rf(h))$$
\end{concept}

\begin{example2}{Bewegung durch Flüssigkeit - Verschiedene Regeln}\\
\textbf{Aufgabe:} Ein Teilchen mit Masse $m = 10$ kg bewegt sich durch eine Flüssigkeit mit Widerstand $R(v) = -v\sqrt{v}$. Für die Verlangsamung von $v_0 = 20$ m/s auf $v = 5$ m/s gilt:
$$t = \int_5^{20} \frac{m}{R(v)} dv = \int_5^{20} \frac{10}{-v\sqrt{v}} dv$$

Berechnen Sie das Integral mit $n = 5$ für:\\
a) Summierte Rechteckregel\\
b) Summierte Trapezregel  \\
c) Summierte Simpson-Regel
\tcblower
\textbf{Parametrisation:} $h = \frac{20-5}{5} = 3$, Stützstellen: $5, 8, 11, 14, 17, 20$

\textbf{a) Rechteckregel:} $Rf(3) = 3 \cdot \sum_{i=0}^{4} f(x_i + 1.5)$
Mittelpunkte: $6.5, 9.5, 12.5, 15.5, 18.5$
$Rf(3) = 3 \cdot (-0.154 - 0.108 - 0.090 - 0.081 - 0.076) = -1.527$

\textbf{b) Trapezregel:} $Tf(3) = 3 \cdot \left(\frac{f(5) + f(20)}{2} + \sum_{i=1}^{4} f(x_i)\right)$
$Tf(3) = 3 \cdot \left(\frac{-0.179 - 0.056}{2} + (-0.125 - 0.096 - 0.082 - 0.072)\right) = -1.477$

\textbf{c) Simpson-Regel:} $Sf(3) = \frac{1}{3}(Tf(3) + 2Rf(3)) = \frac{1}{3}(-1.477 + 2(-1.527)) = -1.510$

\textbf{Exakter Wert:} $\int_5^{20} \frac{-10}{v^{3/2}} dv = \left[\frac{20}{\sqrt{v}}\right]_5^{20} = -1.506$

\textbf{Absolute Fehler:}
\begin{itemize}
    \item Rechteckregel: $|{-1.527} - ({-1.506})| = 0.021$
    \item Trapezregel: $|{-1.477} - ({-1.506})| = 0.029$  
    \item Simpson-Regel: $|{-1.510} - ({-1.506})| = 0.004$
\end{itemize}
\end{example2}

\subsection{Fehlerabschätzung}

\begin{theorem}{Fehlerabschätzung für summierte Quadraturformeln}\\
Für genügend glatte Funktionen gelten folgende Fehlerabschätzungen:

\textbf{Summierte Rechteckregel:}
$$\left|\int_a^b f(x) dx - Rf(h)\right| \leq \frac{h^2}{24}(b-a) \cdot \max_{x \in [a,b]} |f''(x)|$$

\textbf{Summierte Trapezregel:}
$$\left|\int_a^b f(x) dx - Tf(h)\right| \leq \frac{h^2}{12}(b-a) \cdot \max_{x \in [a,b]} |f''(x)|$$

\textbf{Summierte Simpson-Regel:}
$$\left|\int_a^b f(x) dx - Sf(h)\right| \leq \frac{h^4}{2880}(b-a) \cdot \max_{x \in [a,b]} |f^{(4)}(x)|$$
\end{theorem}

\begin{KR}{Schrittweite für gewünschte Genauigkeit bestimmen}
\paragraph{Schritt 1: Gewünschte Genauigkeit festlegen}
Maximaler absoluter Fehler: $\epsilon$

\paragraph{Schritt 2: Höchste Ableitung abschätzen}
Berechne $\max_{x \in [a,b]} |f^{(k)}(x)|$ für entsprechendes $k$.

\paragraph{Schritt 3: Schrittweite berechnen}
\textbf{Für Trapezregel:} $$h \leq \sqrt{\frac{12\epsilon}{(b-a) \max |f''(x)|}}$$

\textbf{Für Simpson-Regel:} $$h \leq \sqrt[4]{\frac{2880\epsilon}{(b-a) \max |f^{(4)}(x)|}}$$

\paragraph{Schritt 4: Anzahl Intervalle bestimmen}
$n = \frac{b-a}{h}$ (aufrunden auf ganze Zahl)
\end{KR}

\begin{example2}{Schrittweite für gewünschte Genauigkeit}\\
\textbf{Aufgabe:} Bestimmen Sie die Schrittweite $h$, um $I = \int_0^{0.5} e^{-x^2} dx$ mit der summierten Trapezregel auf einen absoluten Fehler von maximal $10^{-5}$ genau zu berechnen.
\tcblower
\textbf{Lösung:}

\textbf{Schritt 1:} $\epsilon = 10^{-5}$, $a = 0$, $b = 0.5$

\textbf{Schritt 2:} Zweite Ableitung bestimmen:
$f(x) = e^{-x^2}$
$f'(x) = -2xe^{-x^2}$
$f''(x) = -2e^{-x^2} + 4x^2e^{-x^2} = e^{-x^2}(4x^2 - 2)$

Auf $[0, 0.5]$: $\max |f''(x)| = \max |e^{-x^2}(4x^2 - 2)| = 2$ (bei $x = 0$)

\textbf{Schritt 3:} Schrittweite berechnen:
$$h \leq \sqrt{\frac{12 \cdot 10^{-5}}{0.5 \cdot 2}} = \sqrt{0.00012} \approx 0.011$$

\textbf{Schritt 4:} $n = \frac{0.5}{0.011} \approx 46$ Intervalle
\end{example2}

\subsection{Romberg-Extrapolation}

\begin{concept}{Idee der Romberg-Extrapolation}\\
Die Romberg-Extrapolation verbessert systematisch die Genauigkeit der Trapezregel durch Verwendung mehrerer Schrittweiten und anschließende Extrapolation.

Basis: Trapezregel mit halbierten Schrittweiten $h_j = \frac{b-a}{2^j}$ für $j = 0, 1, 2, ..., m$.
\end{concept}

\begin{theorem}{Romberg-Extrapolation}\\
Für die summierte Trapezregel $Tf(h)$ gilt:

Sei $T_{j0} = Tf\left(\frac{b-a}{2^j}\right)$ für $j = 0, 1, ..., m$. Dann sind durch die Rekursion
$$T_{jk} = \frac{4^k \cdot T_{j+1,k-1} - T_{j,k-1}}{4^k - 1}$$
für $k = 1, 2, ..., m$ und $j = 0, 1, ..., m-k$ Näherungen der Fehlerordnung $2k+2$ gegeben.

Die verwendete Schrittweitenfolge $h_j = \frac{b-a}{2^j}$ heißt \textbf{Romberg-Folge}.
\end{theorem}

\begin{KR}{Romberg-Extrapolation durchführen}
\paragraph{Schritt 1: Trapezwerte für erste Spalte berechnen}
Berechne $T_{j0}$ mit der summierten Trapezregel\\ für $h_j = \frac{b-a}{2^j}$, $j = 0, 1, ..., m$.

\paragraph{Schritt 2: Extrapolationsschema aufstellen}
\begin{center}
\begin{tabular}{|c|c|c|c|}
\hline
$T_{00}$ & & & \\
\hline
$T_{10}$ & $T_{01}$ & & \\
\hline
$T_{20}$ & $T_{11}$ & $T_{02}$ & \\
\hline
$T_{30}$ & $T_{21}$ & $T_{12}$ & $T_{03}$ \\
\hline
\end{tabular}
\end{center}

\paragraph{Schritt 3: Rekursionsformel anwenden}
$$T_{jk} = \frac{4^k \cdot T_{j+1,k-1} - T_{j,k-1}}{4^k - 1}$$

\paragraph{Schritt 4: Genaueste Näherung}
Der Wert rechts unten im Schema ist die genaueste Approximation.
\end{KR}

\begin{example2}{Romberg-Extrapolation anwenden}\\
Berechne $\int_0^{\pi} \cos(x^2) dx$ mit Romberg-Extrapolation für $m = 4$ \\ (d.h. $j = 0, 1, 2, 3, 4$).

\textbf{Schritt 1:} Erste Spalte berechnen
$T_{00} = Tf(\pi)$ mit $h_0 = \pi$ (1 Intervall)
$T_{10} = Tf(\pi/2)$ mit $h_1 = \pi/2$ (2 Intervalle)
$T_{20} = Tf(\pi/4)$ mit $h_2 = \pi/4$ (4 Intervalle)
$T_{30} = Tf(\pi/8)$ mit $h_3 = \pi/8$ (8 Intervalle)
$T_{40} = Tf(\pi/16)$ mit $h_4 = \pi/16$ (16 Intervalle)

\textbf{Beispielrechnung für $T_{00}$:}
$$T_{00} = \pi \cdot \frac{\cos(0) + \cos(\pi^2)}{2} = \frac{\pi}{2}(1 + \cos(\pi^2))$$

\textbf{Schritt 2:} Extrapolationsschema:
\begin{center}
\begin{tabular}{|c|c|c|c|c|}
\hline
$T_{00}$ & & & & \\
\hline
$T_{10}$ & $T_{01}$ & & & \\
\hline
$T_{20}$ & $T_{11}$ & $T_{02}$ & & \\
\hline
$T_{30}$ & $T_{21}$ & $T_{12}$ & $T_{03}$ & \\
\hline
$T_{40}$ & $T_{31}$ & $T_{22}$ & $T_{13}$ & $T_{04}$ \\
\hline
\end{tabular}
\end{center}

Der Wert $T_{04}$ liefert die beste Approximation des Integrals.
\end{example2}

\subsection{Gauss-Formeln}

\begin{concept}{Optimale Stützstellen}\\
Bei Newton-Cotes Formeln sind die Stützstellen äquidistant gewählt. Gauss-Formeln wählen sowohl Stützstellen $x_i$ als auch Gewichte $a_i$ optimal, um die Fehlerordnung zu maximieren.
\end{concept}

\begin{theorem}{Gauss-Formeln} für $n = 1, 2, 3$\\
Die Gauss-Formeln für $\int_a^b f(x) dx \approx \frac{b-a}{2} \sum_{i=1}^{n} a_i f(x_i)$ lauten:

\textbf{$n = 1$:} $G_1f = (b-a) \cdot f\left(\frac{b+a}{2}\right)$

\textbf{$n = 2$:} $G_2f = \frac{b-a}{2}\left[f\left(-\frac{1}{\sqrt{3}} \cdot \frac{b-a}{2} + \frac{b+a}{2}\right) + f\left(\frac{1}{\sqrt{3}} \cdot \frac{b-a}{2} + \frac{b+a}{2}\right)\right]$

\textbf{$n = 3$:} $G_3f = \frac{b-a}{2}\left[\frac{5}{9}f(x_1) + \frac{8}{9}f\left(\frac{b+a}{2}\right) + \frac{5}{9}f(x_3)\right]$

wobei $x_1 = -\sqrt{0.6} \cdot \frac{b-a}{2} + \frac{b+a}{2}$ und $x_3 = \sqrt{0.6} \cdot \frac{b-a}{2} + \frac{b+a}{2}$.
\end{theorem}

\begin{example2}{Erdmasse berechnen}\\
\textbf{Aufgabe:} Berechnen Sie die Masse der Erde mit der nicht-äquidistanten Dichteverteilung:
$$m = \int_0^{6370} \rho(r) \cdot 4\pi r^2 dr$$

\begin{center}
\begin{tabular}{|c|c|c|c|c|c|c|}
\hline
$r$ [km] & 0 & 800 & 1200 & 1400 & 2000 & ... \\
\hline
$\rho$ [kg/m³] & 13000 & 12900 & 12700 & 12000 & 11650 & ... \\
\hline
\end{tabular}
\end{center}
\tcblower
\textbf{Lösung:}

Da die Stützstellen nicht äquidistant sind, verwenden wir die summierte Trapezregel für nicht-äquidistante Daten:
$$\int_0^{6370} \rho(r) \cdot 4\pi r^2 dr \approx \sum_{i=0}^{n-1} \frac{[\rho(r_i) \cdot 4\pi r_i^2] + [\rho(r_{i+1}) \cdot 4\pi r_{i+1}^2]}{2} \cdot (r_{i+1} - r_i)$$

\textbf{Wichtig:} Umrechnung der Einheiten: $r$ in km $\rightarrow$ m, $\rho$ in kg/m³

Ergebnis: $m_{Erde} \approx 5.94 \times 10^{24}$ kg

\textbf{Vergleich mit Literaturwert:} $5.97 \times 10^{24}$ kg
Relativer Fehler: $\approx 0.5\%$
\end{example2}

\begin{KR}{Wahl des Integrationsverfahrens:}
\begin{itemize}
    \item \textbf{Trapezregel:} Einfach, für glatte Funktionen ausreichend
    \item \textbf{Simpson-Regel:} Höhere Genauigkeit, besonders für polynomähnliche Funktionen
    \item \textbf{Romberg-Extrapolation:} Sehr hohe Genauigkeit mit systematischer Verbesserung
    \item \textbf{Gauss-Formeln:} Optimal für begrenzte Anzahl von Funktionsauswertungen
    \item \textbf{Nicht-äquidistante Daten:} Spezielle Trapezregel für tabellarische Daten
\end{itemize}
\end{KR}

\begin{example2}{Prüfungsaufgabe 7.1 - Vergleich der Integrationsregeln}\\
\textbf{Aufgabe:} Berechnen Sie das Integral $I = \int_0^1 e^{x^2} dx$ näherungsweise mit:

a) Summierten Trapezregel mit $n = 4$
b) Summierten Simpson-Regel mit $n = 4$ 
c) Vergleichen Sie mit dem numerischen Referenzwert $I \approx 1.4627$
d) Welche Schrittweite braucht die Trapezregel für einen Fehler $< 10^{-3}$?
\tcblower
\textbf{a) Summierte Trapezregel mit $n = 4$:}
$h = \frac{1-0}{4} = 0.25$

Stützstellen: $x_0 = 0, x_1 = 0.25, x_2 = 0.5, x_3 = 0.75, x_4 = 1$

Funktionswerte:
\begin{itemize}
    \item $f(0) = e^0 = 1$
    \item $f(0.25) = e^{0.0625} = 1.0645$
    \item $f(0.5) = e^{0.25} = 1.2840$
    \item $f(0.75) = e^{0.5625} = 1.7551$
    \item $f(1) = e^1 = 2.7183$
\end{itemize}

$$T_f(0.25) = 0.25 \cdot \left(\frac{1 + 2.7183}{2} + 1.0645 + 1.2840 + 1.7551\right)$$
$$= 0.25 \cdot (1.8592 + 4.1036) = 0.25 \cdot 5.9628 = 1.4907$$

\textbf{b) Summierte Simpson-Regel mit $n = 4$:}
Mittelpunkte: \\ $x_{0.5} = 0.125, x_{1.5} = 0.375, x_{2.5} = 0.625, x_{3.5} = 0.875$

Funktionswerte an Mittelpunkten:
\begin{itemize}
    \item $f(0.125) = e^{0.015625} = 1.0158$
    \item $f(0.375) = e^{0.140625} = 1.1508$
    \item $f(0.625) = e^{0.390625} = 1.4781$
    \item $f(0.875) = e^{0.765625} = 2.1507$
\end{itemize}

$$S_f(0.25) = \frac{1}{3}(T_f(0.25) + 2R_f(0.25))$$

\resizebox{\linewidth}{!}{
$R_f(0.25) = 0.25 \cdot (1.0158 + 1.1508 + 1.4781 + 2.1507) = 0.25 \cdot 5.7954 = 1.4489$}

$$S_f(0.25) = \frac{1}{3}(1.4907 + 2 \cdot 1.4489) = \frac{1}{3} \cdot 4.3885 = 1.4628$$

\textbf{c) Fehlervergleich:}
\begin{itemize}
    \item Trapezregel: $|1.4907 - 1.4627| = 0.0280$
    \item Simpson-Regel: $|1.4628 - 1.4627| = 0.0001$
    \item Die Simpson-Regel ist deutlich genauer (Fehlerordnung 4 vs. 2)
\end{itemize}
\vspace{2mm}
\textbf{d) Schrittweite für Trapezregel mit Fehler $< 10^{-3}$:}

Fehlerabschätzung: $|I - T_f(h)| \leq \frac{h^2}{12}(b-a) \max_{x \in [a,b]} |f''(x)|$
\vspace{2mm}\\
Für $f(x) = e^{x^2}$:
$$f'(x) = 2xe^{x^2}, \quad
f''(x) = 2e^{x^2} + 4x^2e^{x^2} = e^{x^2}(2 + 4x^2)$$

Auf $[0,1]$: $$\max |f''(x)| = f''(1) = e^1(2 + 4) = 6e \approx 16.31$$

$\frac{h^2}{12} \cdot 1 \cdot 16.31 \leq 10^{-3}$
$h^2 \leq \frac{0.012}{16.31} \approx 7.36 \times 10^{-4}$
$h \leq 0.0271$
\vspace{2mm}\\
Also: $n = \frac{1}{0.0271} \approx 37$ Intervalle erforderlich.
\end{example2}

\begin{example2}{Prüfungsaufgabe 7.2 - Romberg-Extrapolation}\\
\textbf{Aufgabe:} Berechnen Sie $I = \int_0^{\pi/2} \sin(x) dx$ mit Romberg-Extrapolation für $m = 3$ (d.h. $j = 0, 1, 2, 3$).

a) Berechnen Sie die erste Spalte $T_{j0}$ für $j = 0, 1, 2, 3$
b) Führen Sie die Romberg-Extrapolation durch
c) Vergleichen Sie mit dem exakten Wert $I = 1$
\tcblower
\textbf{a) Erste Spalte berechnen:}

$h_j = \frac{\pi/2 - 0}{2^j} = \frac{\pi/2}{2^j}$

\textbf{$j = 0$: $h_0 = \pi/2$, $n_0 = 1$}
$$T_{00} = \frac{\pi/2}{2} \cdot (\sin(0) + \sin(\pi/2)) = \frac{\pi}{4} \cdot (0 + 1) = \frac{\pi}{4} = 0.7854$$

\textbf{$j = 1$: $h_1 = \pi/4$, $n_1 = 2$}
$$T_{10} = \frac{\pi/4}{2} \cdot (\sin(0) + \sin(\pi/2)) + \frac{\pi}{4} \cdot \sin(\pi/4)$$
$$= \frac{\pi}{8} \cdot (0 + 1) + \frac{\pi}{4} \cdot \frac{\sqrt{2}}{2} = \frac{\pi}{8} + \frac{\pi\sqrt{2}}{8} = \frac{\pi(1 + \sqrt{2})}{8} = 0.9480$$

\textbf{$j = 2$: $h_2 = \pi/8$, $n_2 = 4$}
Stützstellen: $0, \pi/8, \pi/4, 3\pi/8, \pi/2$
$$T_{20} = \frac{\pi/8}{2} \cdot (\sin(0) + \sin(\pi/2)) + \frac{\pi}{8} \cdot (\sin(\pi/8) + \sin(\pi/4) + \sin(3\pi/8))$$

Mit $\sin(\pi/8) = \sqrt{\frac{2 - \sqrt{2}}{4}} \approx 0.3827$ und $\sin(3\pi/8) \approx 0.9239$:
$$T_{20} = \frac{\pi}{16} + \frac{\pi}{8} \cdot (0.3827 + 0.7071 + 0.9239) = 0.9871$$

\textbf{$j = 3$: $h_3 = \pi/16$, $n_3 = 8$}
$T_{30} = 0.9968$ (detaillierte Rechnung analog)
\vspace{2mm}\\
\textbf{b) Romberg-Extrapolation:}

$T_{01} = \frac{4T_{10} - T_{00}}{4 - 1} = \frac{4 \cdot 0.9480 - 0.7854}{3} = \frac{3.0366}{3} = 1.0122$

$T_{11} = \frac{4T_{20} - T_{10}}{4 - 1} = \frac{4 \cdot 0.9871 - 0.9480}{3} = \frac{2.9964}{3} = 0.9988$

$T_{21} = \frac{4T_{30} - T_{20}}{4 - 1} = \frac{4 \cdot 0.9968 - 0.9871}{3} = \frac{3.0001}{3} = 1.0000$

$T_{02} = \frac{16T_{11} - T_{01}}{16 - 1} = \frac{16 \cdot 0.9988 - 1.0122}{15} = \frac{14.9686}{15} = 0.9979$

$T_{12} = \frac{16T_{21} - T_{11}}{16 - 1} = \frac{16 \cdot 1.0000 - 0.9988}{15} = \frac{15.0012}{15} = 1.0001$

$T_{03} = \frac{64T_{12} - T_{02}}{64 - 1} = \frac{64 \cdot 1.0001 - 0.9979}{63} = 1.0000$

\textbf{Romberg-Schema:}
\begin{center}
\begin{tabular}{|c|c|c|c|}
\hline
0.7854 & & & \\
\hline
0.9480 & 1.0122 & & \\
\hline
0.9871 & 0.9988 & 0.9979 & \\
\hline
0.9968 & 1.0000 & 1.0001 & 1.0000 \\
\hline
\end{tabular}
\end{center}

\textbf{c) Vergleich mit exaktem Wert:}
Der extrapolierte Wert $T_{03} = 1.0000$ stimmt mit dem exakten Wert $I = 1$ überein (bis auf Rundungsfehler). Die Romberg-Extrapolation konvergiert sehr schnell für glatte Funktionen wie $\sin(x)$.
\end{example2}

\begin{example2}{Prüfungsaufgabe 7.3 - Anwendung in der Physik}\\
\textbf{Aufgabe:} Die Geschwindigkeit eines fallenden Objekts mit Luftwiderstand ist gegeben durch:
$v(t) = \sqrt{\frac{mg}{k}} \tanh\left(\sqrt{\frac{gk}{m}} \cdot t\right)$

mit $m = 80$ kg, $g = 9.81$ m/s², $k = 0.25$ kg/m.

a) Berechnen Sie die in den ersten 10 Sekunden zurückgelegte Strecke $s = \int_0^{10} v(t) dt$
b) Verwenden Sie die Simpson-Regel mit $n = 10$
c) Interpretieren Sie das Ergebnis physikalisch
\tcblower
\textbf{a) Parameter und Funktion:}\\
$\sqrt{\frac{mg}{k}} = \sqrt{\frac{80 \cdot 9.81}{0.25}} = \sqrt{3139.2} = 56.0 \text{ m/s}$\\
$\sqrt{\frac{gk}{m}} = \sqrt{\frac{9.81 \cdot 0.25}{80}} = \sqrt{0.0306} = 0.175 \text{ s}^{-1}$
\vspace{2mm}\\
$v(t) = 56.0 \cdot \tanh(0.175 \cdot t)$
\vspace{2mm}\\
\textbf{b) Simpson-Regel mit $n = 10$:}
$h = \frac{10 - 0}{10} = 1$ s

Funktionswerte an Stützstellen:
\begin{itemize}
    \item $v(0) = 56.0 \cdot \tanh(0) = 0$
    \item $v(1) = 56.0 \cdot \tanh(0.175) = 56.0 \cdot 0.173 = 9.69$
    \item $v(2) = 56.0 \cdot \tanh(0.35) = 56.0 \cdot 0.336 = 18.82$
    \item $v(3) = 56.0 \cdot \tanh(0.525) = 56.0 \cdot 0.481 = 26.94$
    \item $v(4) = 56.0 \cdot \tanh(0.7) = 56.0 \cdot 0.604 = 33.82$
    \item $v(5) = 56.0 \cdot \tanh(0.875) = 56.0 \cdot 0.704 = 39.42$
    \item $v(6) = 56.0 \cdot \tanh(1.05) = 56.0 \cdot 0.785 = 43.96$
    \item $v(7) = 56.0 \cdot \tanh(1.225) = 56.0 \cdot 0.847 = 47.43$
    \item $v(8) = 56.0 \cdot \tanh(1.4) = 56.0 \cdot 0.896 = 50.18$
    \item $v(9) = 56.0 \cdot \tanh(1.575) = 56.0 \cdot 0.933 = 52.25$
    \item $v(10) = 56.0 \cdot \tanh(1.75) = 56.0 \cdot 0.959 = 53.70$
\end{itemize}

Mittelpunkte:
\begin{itemize}
    \item $v(0.5) = 56.0 \cdot \tanh(0.0875) = 4.89$
    \item $v(1.5) = 56.0 \cdot \tanh(0.2625) = 14.53$
    \item $v(2.5) = 56.0 \cdot \tanh(0.4375) = 22.99$
    \item $v(3.5) = 56.0 \cdot \tanh(0.6125) = 30.52$
    \item $v(4.5) = 56.0 \cdot \tanh(0.7875) = 36.75$
    \item $v(5.5) = 56.0 \cdot \tanh(0.9625) = 41.87$
    \item $v(6.5) = 56.0 \cdot \tanh(1.1375) = 45.97$
    \item $v(7.5) = 56.0 \cdot \tanh(1.3125) = 48.84$
    \item $v(8.5) = 56.0 \cdot \tanh(1.4875) = 51.16$
    \item $v(9.5) = 56.0 \cdot \tanh(1.6625) = 52.99$
\end{itemize}

Trapezregel:
$$T = 1 \cdot \left(\frac{0 + 53.70}{2} + 9.69 + 18.82 + ... + 52.25\right) = 356.2 \text{ m}$$

Rechteckregel:
$$R = 1 \cdot (4.89 + 14.53 + 22.99 + ... + 52.99) = 350.5 \text{ m}$$

Simpson-Regel:
$$S = \frac{1}{3}(T + 2R) = \frac{1}{3}(356.2 + 2 \cdot 350.5) = \frac{1057.2}{3} = 352.4 \text{ m}$$

\textbf{c) Physikalische Interpretation:}
\begin{itemize}
    \item Das Objekt legt in 10 Sekunden etwa 352 m zurück
    \item Die Endgeschwindigkeit nähert sich asymptotisch $v_{max} = 56$ m/s
    \item Der $\tanh$-Verlauf zeigt die typische Charakteristik: anfangs beschleunigt das Objekt stark, dann verlangsamt sich die Beschleunigung, bis die Endgeschwindigkeit erreicht ist
    \item Ohne Luftwiderstand wäre $s = \frac{1}{2}gt^2 = 490$ m (deutlich mehr)
\end{itemize}
\end{example2}

