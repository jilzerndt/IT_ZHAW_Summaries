\section{Numerische Integration}

\begin{definition}{Numerische Integration (Quadratur)}\\
Für  $f: \mathbb{R} \rightarrow \mathbb{R}$ soll das bestimmte Integral
$I(f) = \int_a^b f(x) dx$

auf einem Intervall $[a,b]$ numerisch berechnet werden.
\vspace{-2mm}\\
$$\text{Quadraturverfahren allgemeine Form: }I(f) = \sum_{i=1}^{n} a_i f(x_i)$$
\vspace{-2mm}
wobei $x_i$ = \textbf{Stützstellen} oder \textbf{Knoten} und $a_i$ = \textbf{Gewichte}
\end{definition}

\subsubsection{Newton-Cotes Formeln}

\begin{definition}{Einfache Rechteck- und Trapezregel}\\
Die \textbf{Rechteckregel} (Mittelpunktsregel) und die \textbf{Trapezregel} \\ zur Approximation von $\int_a^b f(x) dx$ sind definiert als:
$$\text{\textbf{Rechteckregel:} } Rf = f(\frac{a+b}{2}) \cdot (b-a)$$
$$\text{\textbf{Trapezregel:} } Tf = \frac{f(a) + f(b)}{2} \cdot (b-a)$$
\end{definition}

\begin{concept}{Geometrische Interpretation}
\begin{itemize}
    \item \textbf{Rechteckregel:} \\Approximiert die Fläche durch ein Rechteck mit Höhe $f(\frac{a+b}{2})$
    \item \textbf{Trapezregel:} Approximiert die Fläche durch ein Trapez \\ zwischen $(a,f(a))$ und $(b,f(b))$
\end{itemize}
\end{concept}

\begin{theorem}{Summierte Rechteck- und Trapezregel} $f: [a,b] \rightarrow \mathbb{R}$ stetig
\begin{itemize}
    \item $n \in \mathbb{N}$ = Anzahl Subintervalle
    \item Schrittweite $h = \frac{b-a}{n}$
    \item Stützstellen $x_i = a + ih$ für $i = 0, 1, ..., n$
\end{itemize}
$$\text{\textbf{Summierte Rechteckregel: }} Rf(h) = h \cdot \sum_{i=0}^{n-1} f(x_i + \frac{h}{2})$$
$$\text{\textbf{Summierte Trapezregel:} } Tf(h) = h \cdot (\frac{f(a) + f(b)}{2} + \sum_{i=1}^{n-1} f(x_i))$$
\end{theorem}

\begin{corollary}{Trapezregel für nicht-äquidistante Stützstellen}
$$Tf_{neq} = \sum_{i=0}^{n-1} \frac{f(x_i) + f(x_{i+1})}{2} \cdot (x_{i+1} - x_i)$$
\end{corollary}

\raggedcolumns

\begin{definition}{Simpson-Regel}
approximiert $f(x)$ durch Polynom 2. Grades an den Stellen $x_1 = a$, $x_2 = \frac{a+b}{2}$ und $x_3 = b$.
\vspace{-2mm}\\
$$\text{\textbf{Einfache Simpson-Regel:} } Sf = \frac{b-a}{6} (f(a) + 4f(\frac{a+b}{2}) + f(b))$$
\textbf{Summierte Simpson-Regel:}
\vspace{-2mm}\\
$$Sf(h) = \frac{h}{3} (\frac{1}{2}f(a) + \sum_{i=1}^{n-1} f(x_i) + 2\sum_{i=1}^{n} f(\frac{x_{i-1} + x_i}{2}) + \frac{1}{2}f(b))$$
\end{definition}

\begin{concept}{Simpson-Regel als gewichtetes Mittel}\\
Die summierte Simpson-Regel kann als gewichtetes Mittel der summierten Trapez- und Rechteckregel interpretiert werden:
$$Sf(h) = \frac{1}{3}(Tf(h) + 2Rf(h))$$
\end{concept}

\begin{theorem}{Fehlerabschätzung} für \textbf{summierte} Quadraturformeln\\
Für genügend glatte Funktionen gelten folgende Fehlerabschätzungen:
\vspace{1mm}\\
\resizebox{\linewidth}{!}{
$\text{\textbf{Rechteckregel:} } \left|\int_a^b f(x) dx - Rf(h)\right| \leq \frac{h^2}{24}(b-a) \cdot \max_{x \in [a,b]} |f''(x)|$}
\vspace{1mm}\\
\resizebox{\linewidth}{!}{
$\text{\textbf{Trapezregel:} } \left|\int_a^b f(x) dx - Tf(h)\right| \leq \frac{h^2}{12}(b-a) \cdot \max_{x \in [a,b]} |f''(x)|$}
\vspace{1mm}\\
\resizebox{\linewidth}{!}{
$\text{\textbf{Simpson-Regel:} } \left|\int_a^b f(x) dx - Sf(h)\right| \leq \frac{h^4}{2880}(b-a) \cdot \max_{x \in [a,b]} |f^{(4)}(x)|$}
\end{theorem}

\begin{KR}{Schrittweite für gewünschte Genauigkeit bestimmen}

Maximaler absoluter Fehler: $\epsilon$
\vspace{1mm}\\
\textbf{Höchste Ableitung abschätzen}:

Berechne $\max_{x \in [a,b]} |f^{(k)}(x)|$ für entsprechendes $k$.
\vspace{1mm}\\
\textbf{Schrittweite berechnen}:
\vspace{-3mm}\\
$$\text{Für Trapezregel: } h \leq \sqrt{\frac{12\epsilon}{(b-a) \max |f''(x)|}}$$
\vspace{-2mm}
$$\text{Für Simpson-Regel: } h \leq \sqrt[4]{\frac{2880\epsilon}{(b-a) \max |f^{(4)}(x)|}}$$
\textbf{Anzahl Intervalle bestimmen}:
$n = \frac{b-a}{h}$ (aufrunden auf ganze Zahl)
\end{KR}

\begin{example2}{Anwendung Newton-Cotes Formeln}\\
\textbf{Aufgabe:} Ein Teilchen mit Masse $m = 10$ kg bewegt sich durch eine Flüssigkeit mit Widerstand $R(v) = -v\sqrt{v}$. Für die Verlangsamung von $v_0 = 20$ m/s auf $v = 5$ m/s gilt:
$$t = \int_5^{20} \frac{m}{R(v)} dv = \int_5^{20} \frac{10}{-v\sqrt{v}} dv$$

Berechnen Sie das Integral mit $n = 5$
\tcblower

\textbf{Parametrisation:} $h = \frac{20-5}{5} = 3$, Stützstellen: $5, 8, 11, 14, 17, 20$
\vspace{1mm}\\
\textbf{Rechteckregel:} $$Rf(3) = 3 \cdot \sum_{i=0}^{4} f(x_i + 1.5)$$

Mittelpunkte: $6.5, 9.5, 12.5, 15.5, 18.5$

$Rf(3) = 3 \cdot (-0.154 - 0.108 - 0.090 - 0.081 - 0.076) = -1.527$
\vspace{2mm}\\
\textbf{Trapezregel:} $$Tf(3) = 3 \cdot (\frac{f(5) + f(20)}{2} + \sum_{i=1}^{4} f(x_i))$$

$Tf(3) = 3 \cdot (\frac{-0.179 - 0.056}{2} + (-0.125 - 0.096 - 0.082 - 0.072)) = -1.477$
\vspace{1mm}\\
\textbf{Simpson-Regel:} $$Sf(3) = \frac{1}{3}(Tf(3) + 2Rf(3)) = \frac{1}{3}(-1.477 + 2(-1.527)) = -1.510$$

\textbf{Exakter Wert:} $\int_5^{20} \frac{-10}{v^{3/2}} dv = \left[\frac{20}{\sqrt{v}}\right]_5^{20} = -1.506$
\vspace{2mm}\\
\textbf{Absolute Fehler:}
\begin{itemize}
    \item Rechteckregel: $|{-1.527} - ({-1.506})| = 0.021$
    \item Trapezregel: $|{-1.477} - ({-1.506})| = 0.029$  
    \item Simpson-Regel: $|{-1.510} - ({-1.506})| = 0.004$
\end{itemize}
\end{example2}


\begin{example2}{Schrittweite für gewünschte Genauigkeit}\\
\textbf{Aufgabe:} Bestimmen Sie die Schrittweite $h$, um $I = \int_0^{0.5} e^{-x^2} dx$ mit der summierten Trapezregel auf einen absoluten Fehler von maximal $10^{-5}$ genau zu berechnen.

\textbf{Parameter:} $\epsilon = 10^{-5}$, $a = 0$, $b = 0.5$

\textbf{Zweite Ableitung bestimmen:} für
$f(x) = e^{-x^2}$

$f'(x) = -2xe^{-x^2}$

$f''(x) = -2e^{-x^2} + 4x^2e^{-x^2} = e^{-x^2}(4x^2 - 2)$

$$\text{Auf } [0, 0.5]\text{: } \max |f''(x)| = \max |e^{-x^2}(4x^2 - 2)| = 2 \text{ (bei $x = 0$)}$$

\textbf{Schrittweite berechnen:}
$$h \leq \sqrt{\frac{12 \cdot 10^{-5}}{0.5 \cdot 2}} = \sqrt{0.00012} \approx 0.011$$

\textbf{Anzahl Intervalle:} $n = \frac{0.5}{0.011} \approx 46$ Intervalle
\end{example2}

\raggedcolumns
\columnbreak

\subsection{Romberg-Extrapolation}

\begin{concept}{Idee der Romberg-Extrapolation}\\
Die Romberg-Extrapolation verbessert systematisch die Genauigkeit der Trapezregel durch Verwendung mehrerer Schrittweiten und anschließende Extrapolation.

Basis: Trapezregel mit halbierten Schrittweiten $h_j = \frac{b-a}{2^j}$ für $j = 0, 1, 2, ..., m$.
\end{concept}

\begin{theorem}{Romberg-Extrapolation}\\
Für die summierte Trapezregel $Tf(h)$ gilt:

Sei $T_{j0} = Tf(\frac{b-a}{2^j})$ für $j = 0, 1, ..., m$. Dann sind durch die Rekursion
$$T_{jk} = \frac{4^k \cdot T_{j+1,k-1} - T_{j,k-1}}{4^k - 1}$$
für $k = 1, 2, ..., m$ und $j = 0, 1, ..., m-k$ Näherungen der Fehlerordnung $2k+2$ gegeben.

Die verwendete Schrittweitenfolge $h_j = \frac{b-a}{2^j}$ heißt \textbf{Romberg-Folge}.
\end{theorem}

\begin{KR}{Romberg-Extrapolation durchführen}
\paragraph{Schritt 1: Trapezwerte für erste Spalte berechnen}
Berechne $T_{j0}$ mit der summierten Trapezregel\\ für $h_j = \frac{b-a}{2^j}$, $j = 0, 1, ..., m$.

\paragraph{Schritt 2: Extrapolationsschema aufstellen}
\begin{center}
\begin{tabular}{|c|c|c|c|}
\hline
$T_{00}$ & & & \\
\hline
$T_{10}$ & $T_{01}$ & & \\
\hline
$T_{20}$ & $T_{11}$ & $T_{02}$ & \\
\hline
$T_{30}$ & $T_{21}$ & $T_{12}$ & $T_{03}$ \\
\hline
\end{tabular}
\end{center}

\paragraph{Schritt 3: Rekursionsformel anwenden}
$$T_{jk} = \frac{4^k \cdot T_{j+1,k-1} - T_{j,k-1}}{4^k - 1}$$

\paragraph{Schritt 4: Genaueste Näherung}
Der Wert rechts unten im Schema ist die genaueste Approximation.
\end{KR}

\begin{example2}{Romberg-Extrapolation anwenden}\\
Berechne $\int_0^{\pi} \cos(x^2) dx$ mit Romberg-Extrapolation für $m = 4$ \\ (d.h. $j = 0, 1, 2, 3, 4$).

\textbf{Schritt 1:} Erste Spalte berechnen
$T_{00} = Tf(\pi)$ mit $h_0 = \pi$ (1 Intervall)
$T_{10} = Tf(\pi/2)$ mit $h_1 = \pi/2$ (2 Intervalle)
$T_{20} = Tf(\pi/4)$ mit $h_2 = \pi/4$ (4 Intervalle)
$T_{30} = Tf(\pi/8)$ mit $h_3 = \pi/8$ (8 Intervalle)
$T_{40} = Tf(\pi/16)$ mit $h_4 = \pi/16$ (16 Intervalle)

\textbf{Beispielrechnung für $T_{00}$:}
$$T_{00} = \pi \cdot \frac{\cos(0) + \cos(\pi^2)}{2} = \frac{\pi}{2}(1 + \cos(\pi^2))$$

\textbf{Schritt 2:} Extrapolationsschema:
\begin{center}
\begin{tabular}{|c|c|c|c|c|}
\hline
$T_{00}$ & & & & \\
\hline
$T_{10}$ & $T_{01}$ & & & \\
\hline
$T_{20}$ & $T_{11}$ & $T_{02}$ & & \\
\hline
$T_{30}$ & $T_{21}$ & $T_{12}$ & $T_{03}$ & \\
\hline
$T_{40}$ & $T_{31}$ & $T_{22}$ & $T_{13}$ & $T_{04}$ \\
\hline
\end{tabular}
\end{center}

Der Wert $T_{04}$ liefert die beste Approximation des Integrals.
\end{example2}

\subsection{Gauss-Formeln}

\begin{concept}{Optimale Stützstellen}\\
Bei Newton-Cotes Formeln sind die Stützstellen äquidistant gewählt. Gauss-Formeln wählen sowohl Stützstellen $x_i$ als auch Gewichte $a_i$ optimal, um die Fehlerordnung zu maximieren.
\end{concept}

\begin{theorem}{Gauss-Formeln} für $n = 1, 2, 3$\\
Die Gauss-Formeln für $\int_a^b f(x) dx \approx \frac{b-a}{2} \sum_{i=1}^{n} a_i f(x_i)$ lauten:

\textbf{$n = 1$:} $G_1f = (b-a) \cdot f(\frac{b+a}{2})$

\textbf{$n = 2$:} $G_2f = \frac{b-a}{2}\left[f(-\frac{1}{\sqrt{3}} \cdot \frac{b-a}{2} + \frac{b+a}{2}) + f(\frac{1}{\sqrt{3}} \cdot \frac{b-a}{2} + \frac{b+a}{2})\right]$

\textbf{$n = 3$:} $G_3f = \frac{b-a}{2}\left[\frac{5}{9}f(x_1) + \frac{8}{9}f(\frac{b+a}{2}) + \frac{5}{9}f(x_3)\right]$

wobei $x_1 = -\sqrt{0.6} \cdot \frac{b-a}{2} + \frac{b+a}{2}$ und $x_3 = \sqrt{0.6} \cdot \frac{b-a}{2} + \frac{b+a}{2}$.
\end{theorem}


\begin{KR}{Gauss-Formeln anwenden}
\paragraph{Schritt 1: Passende Gauss-Formel wählen}
Bestimme die Anzahl der gewünschten Stützstellen $n$ basierend auf:
\begin{itemize}
    \item Verfügbaren Funktionsauswertungen
    \item Gewünschter Genauigkeit
    \item Glätte der Funktion
\end{itemize}

\paragraph{Schritt 2: Stützstellen berechnen}
Für das Intervall $[a,b]$ transformiere die Standard-Stützstellen:
\begin{itemize}
    \item \textbf{$n=1$:} $x_1 = \frac{a+b}{2}$ (Mittelpunkt)
    \item \textbf{$n=2$:} $x_1 = -\frac{1}{\sqrt{3}} \cdot \frac{b-a}{2} + \frac{a+b}{2}$, $x_2 = \frac{1}{\sqrt{3}} \cdot \frac{b-a}{2} + \frac{a+b}{2}$
    \item \textbf{$n=3$:} $x_1 = -\sqrt{0.6} \cdot \frac{b-a}{2} + \frac{a+b}{2}$, $x_2 = \frac{a+b}{2}$, $x_3 = \sqrt{0.6} \cdot \frac{b-a}{2} + \frac{a+b}{2}$
\end{itemize}

\paragraph{Schritt 3: Funktionswerte berechnen}
Evaluiere $f(x_i)$ für alle Stützstellen $x_i$.

\paragraph{Schritt 4: Gauss-Formel anwenden}
\begin{itemize}
    \item \textbf{$n=1$:} $G_1f = (b-a) \cdot f(x_1)$
    \item \textbf{$n=2$:} $G_2f = \frac{b-a}{2}[f(x_1) + f(x_2)]$
    \item \textbf{$n=3$:} $G_3f = \frac{b-a}{2}[\frac{5}{9}f(x_1) + \frac{8}{9}f(x_2) + \frac{5}{9}f(x_3)]$
\end{itemize}

\paragraph{Fazit}
Gauss-Formeln erreichen höchste Genauigkeit bei minimaler Anzahl von Funktionsauswertungen. Besonders effizient für glatte Funktionen und wenn Funktionsauswertungen teuer sind.
\end{KR}

\begin{example2}{Erdmasse berechnen} 
mit der nicht-äquidistanten Dichteverteilung:
\vspace{-3mm}\\
$$m = \int_0^{6370} \rho(r) \cdot 4\pi r^2 dr$$

\begin{tabular}{|c|c|c|c|c|c|c|}
\hline
$r$ [km] & 0 & 800 & 1200 & 1400 & 2000 & ... \\
\hline
$\rho$ [kg/m³] & 13000 & 12900 & 12700 & 12000 & 11650 & ... \\
\hline
\end{tabular}
\vspace{1mm}\\
Stützstellen nicht äquidistant \\ $\rightarrow$ speziell summierte Trapezregel für nicht-äquidistante Daten:
$$\int_0^{6370} \rho(r) \cdot 4\pi r^2 dr \approx \sum_{i=0}^{n-1} \frac{[\rho(r_i) \cdot 4\pi r_i^2] + [\rho(r_{i+1}) \cdot 4\pi r_{i+1}^2]}{2} \cdot (r_{i+1} - r_i)$$

\textbf{Wichtig:} Umrechnung der Einheiten: $r$ in km $\rightarrow$ m, $\rho$ in kg/m³

Ergebnis: $m_{Erde} \approx 5.94 \times 10^{24}$ kg

\textbf{Vergleich mit Literaturwert:} $5.97 \times 10^{24}$ kg
Relativer Fehler: $\approx 0.5\%$
\end{example2}


\begin{example2}{Gauss-Formeln: Vergleich der Effizienz}\\
\textbf{Aufgabe:} Berechne $I = \int_0^{0.5} e^{-x^2} dx$ mit Gauss-Formeln $G_1$, $G_2$, $G_3$. Vergleiche Genauigkeit mit der Simpson-Regel bei gleichem Aufwand

\textbf{Parameter:} $a = 0$, $b = 0.5$, $f(x) = e^{-x^2}$

\textbf{Exakter Wert:} $I = 0.46115$ (Referenzwert)
\vspace{2mm}\\
\textbf{Gauss-Formel $G_1$ (1 Stützstelle):}

$x_1 = \frac{0+0.5}{2} = 0.25$

$f(0.25) = e^{-0.25^2} = e^{-0.0625} = 0.93941$

$G_1f = 0.5 \cdot 0.93941 = 0.46971$

\textbf{Fehler:} $|0.46971 - 0.46115| = 0.00856$
\vspace{1mm}\\
\textbf{Gauss-Formel $G_2$ (2 Stützstellen):}

$x_1 = -\frac{1}{\sqrt{3}} \cdot \frac{0.25}{2} + 0.25 = 0.25 - \frac{0.125}{\sqrt{3}} = 0.1778$

$x_2 = \frac{1}{\sqrt{3}} \cdot \frac{0.25}{2} + 0.25 = 0.25 + \frac{0.125}{\sqrt{3}} = 0.3222$

$f(0.1778) = e^{-0.1778^2} = 0.9688$

$f(0.3222) = e^{-0.3222^2} = 0.9009$

$G_2f = \frac{0.5}{2}[0.9688 + 0.9009] = 0.4674$

\textbf{Fehler:} $|0.4674 - 0.46115| = 0.00625$
\vspace{1mm}\\
\textbf{Gauss-Formel $G_3$ (3 Stützstellen):}

$x_1 = -\sqrt{0.6} \cdot 0.125 + 0.25 = 0.1532$

$x_2 = 0.25$ (Mittelpunkt)

$x_3 = \sqrt{0.6} \cdot 0.125 + 0.25 = 0.3468$

$f(0.1532) = 0.9766$, $f(0.25) = 0.9394$, $f(0.3468) = 0.8909$

$G_3f = \frac{0.5}{2}[\frac{5}{9} \cdot 0.9766 + \frac{8}{9} \cdot 0.9394 + \frac{5}{9} \cdot 0.8909] = 0.4614$

\textbf{Fehler:} $|0.4614 - 0.46115| = 0.00025$
\vspace{1mm}\\
\textbf{Vergleich mit Simpson-Regel (3 Funktionsauswertungen):}

Simpson mit $n=2$: $x = 0, 0.25, 0.5$

$Sf = \frac{0.25}{3}[f(0) + 4f(0.25) + f(0.5)] = \frac{0.25}{3}[1 + 4(0.9394) + 0.7788] = 0.4621$

\textbf{Fehler:} $|0.4621 - 0.46115| = 0.00095$
\vspace{1mm}\\
\textbf{Fazit:} Gauss-Formel $G_3$ ist deutlich genauer als die Simpson-Regel bei gleicher Anzahl Funktionsauswertungen:
\begin{itemize}
    \item $G_3$: Fehler $= 0.00025$ (3 Auswertungen)
    \item Simpson: Fehler $= 0.00095$ (3 Auswertungen)
    \item $G_3$ ist ca. 4× genauer!
\end{itemize}
\end{example2}

\begin{KR}{Wahl des Integrationsverfahrens:}
\begin{itemize}
    \item \textbf{Trapezregel:} Einfach, für glatte Funktionen ausreichend
    \item \textbf{Simpson-Regel:} Höhere Genauigkeit (polynomähnliche Funktionen!)
    \item \textbf{Romberg-Extrapolation:} Sehr hohe Genauigkeit\\ (systematische Verbesserung)
    \item \textbf{Gauss-Formeln:} Für begrenzte Anzahl von Funktionsauswertungen
    \item \textbf{Nicht-äquidistante Daten:} Spezielle Trapezregel
\end{itemize}
\end{KR}

\begin{concept}{Tipps zur Interpretation der Ergebnisse}
\begin{itemize}
    \item Überprüfen Sie die physikalische Plausibilität der Ergebnisse
    \item Vergleiche numerische Ergebnisse mit analytischen Lösungen
    \item Achte auf Fehlerabschätzungen und wähle Schrittweite entsprechend
\end{itemize}
Häufige Interpretationen:
\begin{itemize}
    \item Physikalische Größen wie Geschwindigkeit, Beschleunigung, Energie
    \item Geometrische Größen wie Fläche, Volumen
    \item Statistische Größen wie Mittelwert, Varianz
    \item Numerische Stabilität und Konvergenz der Verfahren
\end{itemize}    
\end{concept}


\begin{example2}{Prüfungsaufgabe 7.3 - Anwendung in der Physik}\\
\textbf{Aufgabe:} Die Geschwindigkeit eines fallenden Objekts mit Luftwiderstand ist gegeben durch:
$v(t) = \sqrt{\frac{mg}{k}} \tanh\left(\sqrt{\frac{gk}{m}} \cdot t\right)$

mit $m = 80$ kg, $g = 9.81$ m/s², $k = 0.25$ kg/m.

\begin{itemize}
    \item Berechnen Sie die in den ersten 10 Sekunden zurückgelegte Strecke $s = \int_0^{10} v(t) dt$
    \item Verwenden Sie die Simpson-Regel mit $n = 10$ Intervallen
    \item Interpretieren Sie das Ergebnis physikalisch
\end{itemize}

\textbf{Parameter und Funktion:}
\vspace{1mm}\\
$\sqrt{\frac{mg}{k}} = \sqrt{\frac{80 \cdot 9.81}{0.25}} = \sqrt{3139.2} = 56.0 \text{ m/s}$\\
$\sqrt{\frac{gk}{m}} = \sqrt{\frac{9.81 \cdot 0.25}{80}} = \sqrt{0.0306} = 0.175 \text{ s}^{-1}$
\vspace{2mm}\\
$v(t) = 56.0 \cdot \tanh(0.175 \cdot t)$
\vspace{2mm}\\
\textbf{Simpson-Regel mit $n = 10$:}
$h = \frac{10 - 0}{10} = 1$ s

Funktionswerte an Stützstellen:
\begin{itemize}
    \item $v(0) = 56.0 \cdot \tanh(0) = 0$
    \item $v(1) = 56.0 \cdot \tanh(0.175) = 56.0 \cdot 0.173 = 9.69$
    \item $v(2) = 56.0 \cdot \tanh(0.35) = 56.0 \cdot 0.336 = 18.82$
    \item $v(3) = 56.0 \cdot \tanh(0.525) = 56.0 \cdot 0.481 = 26.94$
    \item $v(4) = 56.0 \cdot \tanh(0.7) = 56.0 \cdot 0.604 = 33.82$
    \item $v(5) = 56.0 \cdot \tanh(0.875) = 56.0 \cdot 0.704 = 39.42$
    \item $v(6) = 56.0 \cdot \tanh(1.05) = 56.0 \cdot 0.785 = 43.96$
    \item $v(7) = 56.0 \cdot \tanh(1.225) = 56.0 \cdot 0.847 = 47.43$
    \item $v(8) = 56.0 \cdot \tanh(1.4) = 56.0 \cdot 0.896 = 50.18$
    \item $v(9) = 56.0 \cdot \tanh(1.575) = 56.0 \cdot 0.933 = 52.25$
    \item $v(10) = 56.0 \cdot \tanh(1.75) = 56.0 \cdot 0.959 = 53.70$
\end{itemize}

Mittelpunkte:
\begin{itemize}
    \item $v(0.5) = 56.0 \cdot \tanh(0.0875) = 4.89$
    \item $v(1.5) = 56.0 \cdot \tanh(0.2625) = 14.53$
    \item $v(2.5) = 56.0 \cdot \tanh(0.4375) = 22.99$
    \item $v(3.5) = 56.0 \cdot \tanh(0.6125) = 30.52$
    \item $v(4.5) = 56.0 \cdot \tanh(0.7875) = 36.75$
    \item $v(5.5) = 56.0 \cdot \tanh(0.9625) = 41.87$
    \item $v(6.5) = 56.0 \cdot \tanh(1.1375) = 45.97$
    \item $v(7.5) = 56.0 \cdot \tanh(1.3125) = 48.84$
    \item $v(8.5) = 56.0 \cdot \tanh(1.4875) = 51.16$
    \item $v(9.5) = 56.0 \cdot \tanh(1.6625) = 52.99$
\end{itemize}

Trapezregel:
$$T = 1 \cdot \left(\frac{0 + 53.70}{2} + 9.69 + 18.82 + ... + 52.25\right) = 356.2 \text{ m}$$

Rechteckregel:
$$R = 1 \cdot (4.89 + 14.53 + 22.99 + ... + 52.99) = 350.5 \text{ m}$$

Simpson-Regel:
$$S = \frac{1}{3}(T + 2R) = \frac{1}{3}(356.2 + 2 \cdot 350.5) = \frac{1057.2}{3} = 352.4 \text{ m}$$

\textbf{Physikalische Interpretation:}
\begin{itemize}
    \item Das Objekt legt in 10 Sekunden etwa 352 m zurück
    \item Die Endgeschwindigkeit nähert sich asymptotisch $v_{max} = 56$ m/s
    \item Der $\tanh$-Verlauf zeigt die typische Charakteristik: anfangs beschleunigt das Objekt stark, dann verlangsamt sich die Beschleunigung, bis die Endgeschwindigkeit erreicht ist
    \item Ohne Luftwiderstand wäre $s = \frac{1}{2}gt^2 = 490$ m (deutlich mehr)
\end{itemize}
\end{example2}
