
\subsubsection{Approximations- und Rundungsfehler}

\begin{definition}{Fehlerarten}
Sei $\tilde{x}$ eine Näherung des exakten Wertes $x$:
\vspace{1mm}\\
\begin{minipage}[t]{0.45\textwidth}
    \textbf{Absoluter Fehler:} 
    \begin{center} $\left|\tilde{x}-x\right|$ \end{center}
\end{minipage}
\hspace{3mm}
\begin{minipage}[t]{0.5\textwidth}
    \textbf{Relativer Fehler:} 
    \begin{center} $\left|\frac{\tilde{x}-x}{x}\right| \text{ bzw. } \frac{|\tilde{x}-x|}{|x|} \text{ für } x \neq 0$ \end{center}
\end{minipage}
\end{definition}

\begin{concept}{Konditionierung}
    Die Konditionszahl $K$ beschreibt die relative Fehlervergrösserung bei Funktionsauswertungen:
    \vspace{1mm}\\
\begin{minipage}{0.3\textwidth}
    \vspace{-2mm}
    $$K := \frac{|f'(x)| \cdot |x|}{|f(x)|}$$
\end{minipage}
\hspace{2mm}
\begin{minipage}{0.6\textwidth}
\begin{itemize}
    \item $K \leq 1$: gut konditioniert
    \item $K > 1$: schlecht konditioniert
    \item $K \gg 1$: sehr schlecht konditioniert
\end{itemize}
\end{minipage}
\end{concept}


\begin{theorem}{Fehlerfortpflanzung}
Für $f$ (differenzierbar) gilt näherungsweise:
\vspace{1mm}\\
\begin{minipage}[t]{0.47\textwidth}
    \textbf{Absoluter Fehler:}  
    \vspace{-2mm}\\
    $$|f(\tilde{x})-f(x)| \approx |f'(x)| \cdot |\tilde{x}-x|$$
\end{minipage}
\hspace{3mm}
\begin{minipage}[t]{0.43\textwidth}
    \textbf{Relativer Fehler:}  
    \vspace{-2mm}\\
    $$\frac{|f(\tilde{x})-f(x)|}{|f(x)|} \approx K \cdot \frac{|\tilde{x}-x|}{|x|}$$
\end{minipage}
\end{theorem}



\raggedcolumns




\begin{KR}{Fehlerabschätzung für Nullstellen}\\
So schätzen Sie den Fehler einer Näherungslösung ab:
\begin{enumerate}
    \item Sei $x_n$ der aktuelle Näherungswert
    \item Wähle Toleranz $\epsilon > 0$
    \item Prüfe Vorzeichenwechsel: $f(x_n-\epsilon) \cdot f(x_n+\epsilon) < 0$
    \item Falls ja: Nullstelle liegt in $(x_n-\epsilon, x_n+\epsilon)$
    \item Damit gilt: $|x_n-\xi| < \epsilon$
\end{enumerate}
\end{KR}




















