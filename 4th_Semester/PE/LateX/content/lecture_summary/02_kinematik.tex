\section{Kinematik der Translation}

\subsection{Grundlagen der Kinematik}
\begin{definition}{Kinematik}\\
    Die Kinematik beschreibt die Bewegung eines Körpers, ohne auf deren Ursachen einzugehen. Die Bewegung eines Körpers ist vollständig beschrieben durch:
    \begin{itemize}
        \item seinen Ort (Vektor $\vec{r}$)
        \item seine Geschwindigkeit ($\vec{v}$)
        \item seine Beschleunigung ($\vec{a}$)
    \end{itemize}
    Diese drei Größen hängen durch Ableiten bzw. Integrieren zusammen.
\end{definition}

\begin{formula}{Zusammenhänge zwischen Ort\, Geschwindigkeit und Beschleunigung}\\
    \begin{itemize}
        \item Geschwindigkeit = Ableitung des Ortes nach der Zeit:\\
        $\vec{v} = \frac{d\vec{r}}{dt}$
        \item Beschleunigung = Ableitung der Geschwindigkeit nach der Zeit:\\
        $\vec{a} = \frac{d\vec{v}}{dt}$
        \item Ort = Integral der Geschwindigkeit nach der Zeit:\\
        $\vec{r} = \int \vec{v} \, dt$
        \item Geschwindigkeit = Integral der Beschleunigung nach der Zeit:\\
        $\vec{v} = \int \vec{a} \, dt$
    \end{itemize}
\end{formula}

\subsection{Mittlere Geschwindigkeit und Beschleunigung}
\begin{definition}{Mittlere Geschwindigkeit}\\
    Die mittlere Geschwindigkeit ist die Änderung des Ortes dividiert durch die dafür benötigte Zeit:
    \begin{equation}
        \bar{v}_x = \frac{\Delta r_x}{\Delta t}
    \end{equation}
    Sie stellt den Durchschnittswert über das betrachtete Zeitintervall dar.
\end{definition}

\begin{definition}{Mittlere Beschleunigung}\\
    Die mittlere Beschleunigung ist die Änderung der Geschwindigkeit dividiert durch die dafür benötigte Zeit:
    \begin{equation}
        \bar{a}_x = \frac{\Delta v_x}{\Delta t}
    \end{equation}
\end{definition}

\begin{concept}{Differenzenquotient vs. Differentialquotient}\\
    \begin{itemize}
        \item Der Differenzenquotient (mittlere Geschwindigkeit) ist eine Approximation über ein endliches Zeitintervall: $\frac{\Delta r_x}{\Delta t}$
        \item Der Differentialquotient (Momentangeschwindigkeit) ist der Grenzwert für ein infinitesimal kleines Zeitintervall: $\lim_{\Delta t \to 0} \frac{\Delta r_x}{\Delta t} = \frac{dr_x}{dt}$
        \item In Unity wird mit fixen Zeitschritten $\Delta t = 20$ ms gerechnet, was einer Abtastfrequenz von $f_{sample} = 50$ Hz entspricht
    \end{itemize}
\end{concept}

\subsection{Momentangeschwindigkeit und -beschleunigung}
\begin{definition}{Momentangeschwindigkeit}\\
    Die Momentangeschwindigkeit zur Zeit $t_0$ ist definiert als:
    \begin{equation}
        \vec{v}(t_0) = \lim_{t_1 \to t_0} \frac{\Delta \vec{r}}{t_1 - t_0} = \frac{d\vec{r}}{dt}
    \end{equation}
    Sie entspricht geometrisch der Steigung der Tangente im Punkt $(t_0, r_x(t_0))$.
\end{definition}

\begin{remark}
    Der Betrag der Geschwindigkeit wird oft als Schnelligkeit bezeichnet:
    \begin{equation}
        |\vec{v}| = \sqrt{v_x^2 + v_y^2 + v_z^2}
    \end{equation}
    Bei gleichbleibender Schnelligkeit kann sich dennoch die Richtung der Geschwindigkeit ändern, z.B. bei einer Kreisbewegung.
\end{remark}

\begin{formula}{Fläche unter dem Geschwindigkeits-Zeit-Diagramm}\\
    Bei einer Bewegung mit variablem $v(t)$ berechnet sich die zurückgelegte Strecke als Fläche unter der $v$-$t$-Kurve:
    \begin{equation}
        \Delta x = \int_{t_1}^{t_2} v(t) \, dt
    \end{equation}
\end{formula}

\subsection{Integration und Differentiation}
\begin{formula}{Ableitungsregeln}\\
    \begin{itemize}
        \item Konstante Summanden: $\frac{d}{dt}(C) = 0$
        \item Potenzfunktionen: $\frac{d}{dt}(at^n) = a \cdot n \cdot t^{n-1}$
        \item Exponentialfunktion: $\frac{d}{dx}(e^x) = e^x$
        \item Logarithmus: $\frac{d}{dx}(\ln x) = \frac{1}{x}$
        \item Sinus/Kosinus: $\frac{d}{dx}(\sin x) = \cos x$, $\frac{d}{dx}(\cos x) = -\sin x$
    \end{itemize}
\end{formula}

\begin{formula}{Regeln für zusammengesetzte Funktionen}\\
    \begin{itemize}
        \item Summenregel: $\frac{d}{dt}(f(t) + g(t)) = \frac{df}{dt} + \frac{dg}{dt}$
        \item Produktregel: $\frac{d}{dt}(f(t) \cdot g(t)) = \frac{df}{dt} \cdot g(t) + f(t) \cdot \frac{dg}{dt}$
        \item Kettenregel: $\frac{d}{dt}(f(g(t))) = \frac{df}{dg} \cdot \frac{dg}{dt}$
    \end{itemize}
\end{formula}

\begin{KR}{Berechnung von Bewegungen mit konstanter Beschleunigung}\\
    \paragraph{Gegebene Größen}
    \begin{itemize}
        \item Anfangsposition $r_0$
        \item Anfangsgeschwindigkeit $v_0$
        \item Konstante Beschleunigung $a$
    \end{itemize}
    
    \paragraph{Schritte zur Berechnung}
    \begin{enumerate}
        \item Geschwindigkeit in Abhängigkeit von der Zeit bestimmen:
        \begin{equation}
            v(t) = v_0 + at
        \end{equation}
        
        \item Position in Abhängigkeit von der Zeit bestimmen:
        \begin{equation}
            r(t) = r_0 + v_0t + \frac{1}{2}at^2
        \end{equation}
        
        \item Alternative Formel bei bekannter Strecke (ohne Zeit):
        \begin{equation}
            v^2 = v_0^2 + 2a(r - r_0)
        \end{equation}
    \end{enumerate}
\end{KR}

\begin{examplecode}{Bewegung in Unity implementieren}\\
    \begin{lstlisting}[language=csh, style=basesmol]
// Implementierung von Bewegungen mit konstanter Beschleunigung
void FixedUpdate() {
    // Aktuelle Zeit seit Start
    currentTime += Time.deltaTime;
    
    // Aktuelle Geschwindigkeit nach v = v0 + a*t berechnen
    float currentVelocity = initialVelocity + acceleration * currentTime;
    
    // Bewegung mit aktueller Geschwindigkeit
    Vector3 displacement = new Vector3(currentVelocity, 0, 0) * Time.deltaTime;
    transform.position += displacement;
    
    // Alternative: Direkte Berechnung der Position mit r = r0 + v0*t + 0.5*a*t^2
    // Vector3 newPosition = initialPosition + initialVelocity * currentTime + 
    //                       0.5f * acceleration * currentTime * currentTime;
    // transform.position = newPosition;
}
    \end{lstlisting}
\end{examplecode}

\begin{example2}{Freier Fall}\\
    Ein Körper fällt aus der Höhe $r_0$ mit Anfangsgeschwindigkeit $v_0 = 0$.
    
    \begin{itemize}
        \item Beschleunigung: $a(t) = -g$ (g = 9.81 m/s²)
        \item Geschwindigkeit: $v(t) = -gt$
        \item Position: $r(t) = r_0 - \frac{1}{2}gt^2$
    \end{itemize}
    
    Alternativ: Ein Körper wird mit Anfangsgeschwindigkeit $v_0$ nach oben geworfen:
    \begin{itemize}
        \item Maximale Höhe: $h_{max} = \frac{v_0^2}{2g}$ 
        \item Zeit bis zum höchsten Punkt: $t_{max} = \frac{v_0}{g}$
        \item Gesamtflugzeit: $t_{gesamt} = \frac{2v_0}{g}$
    \end{itemize}
\end{example2}

\section{Kinematics}

\subsection{Basic Concepts}

\begin{definition}{Kinematics}\\
    The branch of mechanics that describes motion without considering the forces that cause it. Focuses on position, velocity, and acceleration as functions of time.
\end{definition}

\begin{concept}{Reference Frames}\\
    All motion is relative to a chosen coordinate system. Unity uses a left-handed coordinate system where:
    \begin{itemize}
        \item X-axis: right (positive) / left (negative)
        \item Y-axis: up (positive) / down (negative) 
        \item Z-axis: forward (positive) / backward (negative)
    \end{itemize}
\end{concept}

\subsection{Position and Displacement}

\begin{definition}{Position Vector}\\
    Vector $\vec{r}(t)$ that describes the location of an object at time $t$ relative to the origin of the coordinate system.
    $$\vec{r}(t) = x(t)\hat{i} + y(t)\hat{j} + z(t)\hat{k}$$
\end{definition}

\begin{definition}{Displacement}\\
    Change in position vector over a time interval:
    $$\Delta\vec{r} = \vec{r}(t_2) - \vec{r}(t_1)$$
\end{definition}

\begin{code}{Unity Position Implementation}\\
\begin{lstlisting}[language=C, style=basesmol]
// Get current position of a GameObject
Vector3 currentPosition = transform.position;

// Calculate displacement between two positions
Vector3 displacement = finalPosition - initialPosition;

// Update position over time
transform.position += velocity * Time.deltaTime;
\end{lstlisting}
\end{code}

\subsection{Velocity}

\begin{definition}{Average Velocity}\\
    $$\vec{v}_{avg} = \frac{\Delta\vec{r}}{\Delta t} = \frac{\vec{r}(t_2) - \vec{r}(t_1)}{t_2 - t_1}$$
\end{definition}

\begin{definition}{Instantaneous Velocity}\\
    $$\vec{v}(t) = \lim_{\Delta t \to 0} \frac{\Delta\vec{r}}{\Delta t} = \frac{d\vec{r}}{dt}$$
\end{definition}

\begin{concept}{Speed vs. Velocity}\\
    \begin{itemize}
        \item Speed: magnitude of velocity vector (scalar)
        \item Velocity: vector quantity with both magnitude and direction
        \item $|\vec{v}| = \sqrt{v_x^2 + v_y^2 + v_z^2}$
    \end{itemize}
\end{concept}

\subsection{Acceleration}

\begin{definition}{Average Acceleration}\\
    $$\vec{a}_{avg} = \frac{\Delta\vec{v}}{\Delta t} = \frac{\vec{v}(t_2) - \vec{v}(t_1)}{t_2 - t_1}$$
\end{definition}

\begin{definition}{Instantaneous Acceleration}\\
    $$\vec{a}(t) = \lim_{\Delta t \to 0} \frac{\Delta\vec{v}}{\Delta t} = \frac{d\vec{v}}{dt} = \frac{d^2\vec{r}}{dt^2}$$
\end{definition}

\subsection{Motion Equations}

\begin{formula}{Kinematic Equations for Constant Acceleration}\\
    For motion with constant acceleration $\vec{a}$:
    
    \paragraph{Position as function of time:}
    $$\vec{r}(t) = \vec{r}_0 + \vec{v}_0 t + \frac{1}{2}\vec{a}t^2$$
    
    \paragraph{Velocity as function of time:}
    $$\vec{v}(t) = \vec{v}_0 + \vec{a}t$$
    
    \paragraph{Velocity-position relation:}
    $$\vec{v}^2 = \vec{v}_0^2 + 2\vec{a} \cdot (\vec{r} - \vec{r}_0)$$
\end{formula}

\begin{code}{Unity Kinematic Implementation}\\
\begin{lstlisting}[language=C, style=basesmol]
public class KinematicMotion : MonoBehaviour 
{
    public Vector3 initialVelocity = Vector3.zero;
    public Vector3 acceleration = Vector3.zero;
    
    private Vector3 initialPosition;
    private float startTime;
    
    void Start() 
    {
        initialPosition = transform.position;
        startTime = Time.time;
    }
    
    void Update() 
    {
        float t = Time.time - startTime;
        
        // Calculate new position using kinematic equation
        Vector3 newPosition = initialPosition + 
                             initialVelocity * t + 
                             0.5f * acceleration * t * t;
        
        transform.position = newPosition;
    }
}
\end{lstlisting}
\end{code}

\subsection{Special Cases}

\begin{definition}{Uniform Motion}\\
    Motion with constant velocity ($\vec{a} = 0$):
    $$\vec{r}(t) = \vec{r}_0 + \vec{v}t$$
\end{definition}

\begin{definition}{Free Fall}\\
    Motion under gravity alone ($\vec{a} = -g\hat{j}$ in Unity coordinates):
    \begin{itemize}
        \item $g = 9.81 \, m/s^2$ (Earth's gravitational acceleration)
        \item Unity default gravity: $-9.81 \, m/s^2$ in Y-direction
    \end{itemize}
\end{definition}

\begin{KR}{Solving Kinematic Problems}\\
    \paragraph{Step 1: Identify known variables}
    \begin{itemize}
        \item Initial position $\vec{r}_0$
        \item Initial velocity $\vec{v}_0$
        \item Acceleration $\vec{a}$
        \item Time $t$ or final position/velocity
    \end{itemize}
    
    \paragraph{Step 2: Choose appropriate equation}
    \begin{itemize}
        \item Use $\vec{v}(t) = \vec{v}_0 + \vec{a}t$ when time is known
        \item Use $\vec{r}(t) = \vec{r}_0 + \vec{v}_0 t + \frac{1}{2}\vec{a}t^2$ for position
        \item Use $\vec{v}^2 = \vec{v}_0^2 + 2\vec{a} \cdot \Delta\vec{r}$ when time is unknown
    \end{itemize}
    
    \paragraph{Step 3: Solve component-wise}
    \begin{itemize}
        \item Break vectors into x, y, z components
        \item Solve each component independently
        \item Combine results into final vector
    \end{itemize}
\end{KR}

\begin{example2}{Projectile Motion Problem}\\
    A ball is thrown from height $h = 10m$ with initial velocity $\vec{v}_0 = (5, 8, 0) \, m/s$. Calculate time to hit ground and horizontal distance traveled.
    \tcblower
    \textbf{Given:} $\vec{r}_0 = (0, 10, 0)$, $\vec{v}_0 = (5, 8, 0)$, $\vec{a} = (0, -9.81, 0)$
    
    \textbf{Y-component (vertical):}
    $y(t) = 10 + 8t - 4.905t^2$
    
    When ball hits ground: $y(t) = 0$
    $10 + 8t - 4.905t^2 = 0$
    $t = 2.24s$ (using quadratic formula)
    
    \textbf{X-component (horizontal):}
    $x(t) = 5t = 5 \times 2.24 = 11.2m$
\end{example2}