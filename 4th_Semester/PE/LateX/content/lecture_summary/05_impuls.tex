\section{Impuls und Stoßgesetze}

\subsection{Impuls}
\begin{definition}{Impuls}\\
    Der Impuls $\vec{p}$ eines Körpers ist das Produkt aus seiner Masse und seiner Geschwindigkeit:
    \begin{equation}
        \vec{p} = m \cdot \vec{v}
    \end{equation}
    
    Der Impuls ist eine vektorielle Größe mit der Einheit kg·m/s.
\end{definition}

\begin{concept}{Impulserhaltung}\\
    Das Prinzip der Impulserhaltung besagt, dass in einem abgeschlossenen System der Gesamtimpuls konstant bleibt, wenn keine äußeren Kräfte wirken:
    \begin{equation}
        \sum \vec{p}_{\text{vorher}} = \sum \vec{p}_{\text{nachher}}
    \end{equation}
    
    In Komponenten ausgedrückt für zwei Körper:
    \begin{align}
        m_1 \cdot v_{1,\text{vorher}} + m_2 \cdot v_{2,\text{vorher}} = m_1 \cdot v_{1,\text{nachher}} + m_2 \cdot v_{2,\text{nachher}}
    \end{align}
    
    Die Impulserhaltung gilt auch bei dissipativen Vorgängen wie inelastischen Stößen, bei denen Energie verloren geht.
\end{concept}

\begin{formula}{Kraftstoß}\\
    Der Kraftstoß $\vec{I}$ ist das Zeitintegral der Kraft:
    \begin{equation}
        \vec{I} = \int_{t_1}^{t_2} \vec{F}(t) \, dt
    \end{equation}
    
    Der Kraftstoß ist gleich der Impulsänderung:
    \begin{equation}
        \vec{I} = \Delta \vec{p} = \vec{p}_{\text{nachher}} - \vec{p}_{\text{vorher}}
    \end{equation}
    
    Für eine konstante Kraft während der Zeitspanne $\Delta t$ gilt:
    \begin{equation}
        \vec{I} = \vec{F} \cdot \Delta t
    \end{equation}
\end{formula}

\subsection{Stöße}
\begin{definition}{Elastischer Stoß}\\
    Bei einem elastischen Stoß bleiben sowohl der Gesamtimpuls als auch die Gesamtenergie erhalten. Es gilt:
    \begin{align}
        m_1 \cdot v_{1,\text{vorher}} + m_2 \cdot v_{2,\text{vorher}} &= m_1 \cdot v_{1,\text{nachher}} + m_2 \cdot v_{2,\text{nachher}} \\
        \frac{1}{2}m_1 \cdot v_{1,\text{vorher}}^2 + \frac{1}{2}m_2 \cdot v_{2,\text{vorher}}^2 &= \frac{1}{2}m_1 \cdot v_{1,\text{nachher}}^2 + \frac{1}{2}m_2 \cdot v_{2,\text{nachher}}^2
    \end{align}
\end{definition}

\begin{definition}{Inelastischer Stoß}\\
    Bei einem inelastischen Stoß bleibt nur der Gesamtimpuls erhalten, während die mechanische Energie teilweise in andere Energieformen (z.B. Wärme) umgewandelt wird.
    
    Im Extremfall des vollständig inelastischen Stoßes "kleben" die Körper nach dem Stoß zusammen und bewegen sich mit einer gemeinsamen Geschwindigkeit $v_{\text{nachher}}$:
    \begin{equation}
        v_{\text{nachher}} = \frac{m_1 \cdot v_{1,\text{vorher}} + m_2 \cdot v_{2,\text{vorher}}}{m_1 + m_2}
    \end{equation}
\end{definition}

\begin{KR}{Berechnung von Stößen in 1D}\\
    \paragraph{Elastischer Stoß}
    Gegeben: Massen $m_1$ und $m_2$, Anfangsgeschwindigkeiten $v_1$ und $v_2$
    \begin{enumerate}
        \item Schwerpunktgeschwindigkeit berechnen:
        \begin{equation}
            v_{Spt} = \frac{m_1 \cdot v_1 + m_2 \cdot v_2}{m_1 + m_2}
        \end{equation}
        
        \item Geschwindigkeiten nach dem Stoß berechnen (Spiegelung an der Schwerpunktgeschwindigkeit):
        \begin{align}
            u_1 &= v_{Spt} - (v_1 - v_{Spt}) = 2v_{Spt} - v_1 \\
            u_2 &= v_{Spt} - (v_2 - v_{Spt}) = 2v_{Spt} - v_2
        \end{align}
        
        Oder mit den Formeln:
        \begin{align}
            u_1 &= \frac{m_1 - m_2}{m_1 + m_2}v_1 + \frac{2m_2}{m_1 + m_2}v_2 \\
            u_2 &= \frac{2m_1}{m_1 + m_2}v_1 + \frac{m_2 - m_1}{m_1 + m_2}v_2
        \end{align}
    \end{enumerate}
    
    \paragraph{Vollständig inelastischer Stoß}
    Gegeben: Massen $m_1$ und $m_2$, Anfangsgeschwindigkeiten $v_1$ und $v_2$
    \begin{enumerate}
        \item Gemeinsame Geschwindigkeit nach dem Stoß berechnen:
        \begin{equation}
            u_1 = u_2 = \frac{m_1 \cdot v_1 + m_2 \cdot v_2}{m_1 + m_2}
        \end{equation}
    \end{enumerate}
\end{KR}

\begin{example2}{Elastischer Stoß mit ungleichen Massen}\\
    Wenn eine leichte Kugel auf eine ruhende schwere Kugel trifft ($m_1 < m_2$ und $v_2 = 0$):
    \begin{align}
        u_1 &= \frac{m_1 - m_2}{m_1 + m_2}v_1 \\
        u_2 &= \frac{2m_1}{m_1 + m_2}v_1
    \end{align}
    
    Für $m_1 \ll m_2$ (z.B. Tennisball gegen Wand) gilt näherungsweise:
    \begin{align}
        u_1 &\approx -v_1 \quad \text{(Richtungsumkehr)} \\
        u_2 &\approx 0 \quad \text{(schwerer Körper bleibt fast in Ruhe)}
    \end{align}
    
    Umgekehrt, wenn eine schwere Kugel auf eine leichte, ruhende Kugel trifft ($m_1 > m_2$ und $v_2 = 0$):
    \begin{align}
        u_1 &\approx v_1 \quad \text{(fast unverändert)} \\
        u_2 &\approx 2v_1 \quad \text{(doppelte Geschwindigkeit des auftreffenden Körpers)}
    \end{align}
\end{example2}

\begin{concept}{Vergleich: Impuls- und Energieerhaltung}\\
    \begin{itemize}
        \item Impulserhaltung gilt immer, wenn keine äußeren Kräfte wirken (Grundprinzip der Mechanik)
        \item Energieerhaltung gilt nur für konservative Vorgänge ohne Energieverluste
        \item Bei elastischen Stößen gelten beide Erhaltungssätze
        \item Bei inelastischen Stößen gilt nur die Impulserhaltung, die mechanische Energie nimmt ab
        \item Die Impulserhaltung liefert Vektorgleichungen (Richtungen wichtig)
        \item Die Energieerhaltung liefert eine Skalargleichung (nur Beträge wichtig)
    \end{itemize}
\end{concept}

\subsection{Anwendungen der Impulserhaltung}
\begin{example2}{Ballistisches Pendel}\\
    Ein ballistisches Pendel ist ein Gerät zur Messung der Geschwindigkeit von Projektilen. Es besteht aus einem Pendelkörper, in den das Projektil einschlägt und stecken bleibt (vollständig inelastischer Stoß).
    
    Gegeben:
    \begin{itemize}
        \item Masse des Projektils $m_K$
        \item Masse des Pendels $m_P$
        \item Auslenkung des Pendels $x$
        \item Pendellänge $L$
    \end{itemize}
    
    Die Geschwindigkeit des Projektils lässt sich berechnen:
    \begin{equation}
        v_K = \frac{m_P + m_K}{m_K} \cdot \sqrt{2g \cdot (L - \sqrt{L^2 - x^2})}
    \end{equation}
    
    Dabei wird ein zweistufiger Prozess betrachtet:
    \begin{enumerate}
        \item Inelastischer Stoß: Impulserhaltung, Energieverlust
        \item Pendelschwingung: Energieerhaltung (kinetische zu potentieller Energie)
    \end{enumerate}
\end{example2}

\begin{example2}{Raketenantrieb}\\
    Der Raketenantrieb basiert auf dem Rückstoßprinzip und der Impulserhaltung. Die Rakete stößt Treibstoff mit hoher Geschwindigkeit aus und erhält dadurch einen Impuls in die entgegengesetzte Richtung.
    
    Die Raketengleichung (Ziolkowski-Gleichung) beschreibt die maximale Geschwindigkeit einer Rakete:
    \begin{equation}
        v_{\text{max}} = v_{\text{rel}} \cdot \ln\frac{m_0}{m_{\text{leer}}}
    \end{equation}
    
    Dabei ist:
    \begin{itemize}
        \item $v_{\text{rel}}$: Austrittsgeschwindigkeit des Treibstoffs relativ zur Rakete
        \item $m_0$: Anfangsmasse (Rakete + Treibstoff)
        \item $m_{\text{leer}}$: Leermasse der Rakete (ohne Treibstoff)
    \end{itemize}
\end{example2}

\begin{examplecode}{Stoßsimulation in Unity}\\
    \begin{lstlisting}[language=csh, style=basesmol]
// Elastischen Stoss zwischen zwei Objekten simulieren
void OnCollisionEnter(Collision collision) {
    Rigidbody otherRigidbody = collision.rigidbody;
    
    // Geschwindigkeiten vor dem Stoss
    Vector3 v1 = rigidbody.velocity;
    Vector3 v2 = otherRigidbody.velocity;
    
    // Massen
    float m1 = rigidbody.mass;
    float m2 = otherRigidbody.mass;
    
    // Normale des Stosses (Verbindungslinie der Mittelpunkte)
    Vector3 normal = (otherRigidbody.position - rigidbody.position).normalized;
    
    // Projektion der Geschwindigkeiten auf die Stossnormale
    float v1n = Vector3.Dot(v1, normal);
    float v2n = Vector3.Dot(v2, normal);
    
    // Nur berechnen, wenn Objekte sich annaehern
    if (v1n - v2n > 0) {
        // Geschwindigkeiten nach dem Stoss entlang der Normalen
        float u1n = ((m1 - m2) * v1n + 2 * m2 * v2n) / (m1 + m2);
        float u2n = ((m2 - m1) * v2n + 2 * m1 * v1n) / (m1 + m2);
        
        // Geschwindigkeitsaenderung nur entlang der Normalen
        Vector3 v1Tangential = v1 - v1n * normal;
        Vector3 v2Tangential = v2 - v2n * normal;
        
        // Neue Geschwindigkeiten
        Vector3 newV1 = v1Tangential + u1n * normal;
        Vector3 newV2 = v2Tangential + u2n * normal;
        
        // Geschwindigkeiten setzen
        rigidbody.velocity = newV1;
        otherRigidbody.velocity = newV2;
    }
}
    \end{lstlisting}
\end{examplecode}