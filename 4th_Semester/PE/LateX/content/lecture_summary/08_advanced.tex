\section{Erweiterte Themen und Unity-Implementierung}

\subsection{Unity Physics System}

\begin{concept}{Unity Physics Pipeline}
    Unity's Physiksimulation läuft in diskreten Schritten:
    \begin{itemize}
        \item \texttt{FixedUpdate()}: Wird in festen Intervallen aufgerufen (Standard 50Hz)
        \item Physikberechnungen zwischen FixedUpdate-Aufrufen
        \item \texttt{Update()}: Wird einmal pro Frame aufgerufen (variable Rate)
        \item \textbf{Regel}: FixedUpdate für physikbezogenen Code verwenden
    \end{itemize}
    
    \textbf{Zeitsteuerung:}
    \begin{itemize}
        \item \texttt{Time.fixedDeltaTime}: Zeitschritt für Physik (Standard: 0.02s = 50Hz)
        \item \texttt{Time.deltaTime}: Zeit seit letztem Frame (variabel)
        \item \texttt{Time.timeScale}: Globaler Zeitfaktor (für Slow-Motion etc.)
    \end{itemize}
\end{concept}

\begin{definition}{Rigidbody-Komponenten-Eigenschaften}
    \textbf{Grundlegende Eigenschaften:}
    \begin{itemize}
        \item \textbf{Mass}: Objektmasse in kg
        \item \textbf{Drag}: Linearer Dämpfungskoeffizient (Luftwiderstand)
        \item \textbf{Angular Drag}: Rotationsdämpfungskoeffizient
        \item \textbf{Use Gravity}: Reagiert auf globale Gravitation
        \item \textbf{Is Kinematic}: Physik-gesteuert vs. Skript-gesteuert
    \end{itemize}
    
    \textbf{Erweiterte Eigenschaften:}
    \begin{itemize}
        \item \textbf{Interpolate}: Glättung für visuelle Darstellung
        \item \textbf{Collision Detection}: Continuous, Discrete, etc.
        \item \textbf{Constraints}: Einschränkung von Position/Rotation
        \item \textbf{Center of Mass}: Schwerpunkt-Position
    \end{itemize}
\end{definition}

\begin{code}{Unity Physics-Konfiguration}
\begin{lstlisting}[language=C, style=basesmol]
public class PhysicsConfiguration : MonoBehaviour 
{
    void Start() 
    {
        // Globale Physik-Einstellungen
        Physics.gravity = new Vector3(0, -9.81f, 0);
        Time.fixedDeltaTime = 0.02f; // 50 Hz Physik-Update
        
        // Solver-Einstellungen fuer bessere Stabilitaet
        Physics.defaultSolverIterations = 6;
        Physics.defaultSolverVelocityIterations = 1;
        
        // Rigidbody-Konfiguration
        Rigidbody rb = GetComponent<Rigidbody>();
        rb.mass = 1.0f;
        rb.drag = 0.1f;                    // Lineare Daempfung
        rb.angularDrag = 0.05f;            // Winkel-Daempfung
        rb.useGravity = true;
        rb.isKinematic = false;
        
        // Kollisionserkennung-Modus
        rb.collisionDetectionMode = CollisionDetectionMode.Continuous;
        
        // Interpolation fuer glatte Bewegung
        rb.interpolation = RigidbodyInterpolation.Interpolate;
        
        // Schwerpunkt manuell setzen (falls noetig)
        rb.centerOfMass = new Vector3(0, -0.5f, 0);
    }
}
\end{lstlisting}
\end{code}

\subsection{Kraftanwendungsmethoden}

\begin{definition}{Unity Kraftanwendung}
    \textbf{Kraftmethoden:}
    \begin{itemize}
        \item \textbf{AddForce()}: Kontinuierliche Kraftanwendung
        \item \textbf{AddForceAtPosition()}: Kraft an spezifischem Weltpunkt
        \item \textbf{AddTorque()}: Rotationskraft-Anwendung
        \item \textbf{AddExplosionForce()}: Radiale Kraft von Explosionszentrum
        \item \textbf{AddRelativeForce()}: Kraft im lokalen Koordinatensystem
    \end{itemize}
    
    \textbf{Force-Modi:}
    \begin{itemize}
        \item \textbf{Force}: Kontinuierliche Kraft (berücksichtigt Masse und Zeit)
        \item \textbf{Acceleration}: Beschleunigung (ignoriert Masse)
        \item \textbf{Impulse}: Impulsive Kraft (berücksichtigt Masse)
        \item \textbf{VelocityChange}: Geschwindigkeitsänderung (ignoriert Masse)
    \end{itemize}
\end{definition}

\begin{code}{Umfassender Kraft-Controller}
\begin{lstlisting}[language=C, style=basesmol]
public class AdvancedForceController : MonoBehaviour 
{
    private Rigidbody rb;
    
    [Header("Kraftparameter")]
    public float thrustForce = 100f;
    public float torqueStrength = 50f;
    public float explosionForce = 500f;
    public float explosionRadius = 10f;
    public float maxVelocity = 20f;
    
    [Header("Steuerung")]
    public KeyCode thrustKey = KeyCode.W;
    public KeyCode leftKey = KeyCode.A;
    public KeyCode rightKey = KeyCode.D;
    public KeyCode explosionKey = KeyCode.Space;
    
    void Start() 
    {
        rb = GetComponent<Rigidbody>();
        
        // Sicherstellen, dass Rigidbody korrekt konfiguriert ist
        rb.useGravity = true;
        rb.drag = 0.1f;
        rb.angularDrag = 0.1f;
    }
    
    void FixedUpdate() 
    {
        HandleMovement();
        HandleRotation();
        HandleSpecialForces();
        
        // Geschwindigkeitsbegrenzung
        LimitVelocity();
    }
    
    void HandleMovement() 
    {
        // Schubkraft in Vorwaertsrichtung
        if (Input.GetKey(thrustKey)) 
        {
            Vector3 thrust = transform.forward * thrustForce;
            rb.AddForce(thrust, ForceMode.Force);
            
            // Visuelle/Audio-Effekte
            CreateThrustEffects();
        }
        
        // Seitenschub
        if (Input.GetKey(leftKey)) 
        {
            rb.AddForce(-transform.right * thrustForce * 0.5f, ForceMode.Force);
        }
        if (Input.GetKey(rightKey)) 
        {
            rb.AddForce(transform.right * thrustForce * 0.5f, ForceMode.Force);
        }
    }
    
    void HandleRotation() 
    {
        // Gier-Steuerung (Yaw)
        if (Input.GetKey(KeyCode.Q)) 
        {
            rb.AddTorque(-transform.up * torqueStrength, ForceMode.Force);
        }
        if (Input.GetKey(KeyCode.E)) 
        {
            rb.AddTorque(transform.up * torqueStrength, ForceMode.Force);
        }
        
        // Nick-Steuerung (Pitch)
        if (Input.GetKey(KeyCode.R)) 
        {
            rb.AddTorque(transform.right * torqueStrength * 0.5f, ForceMode.Force);
        }
        if (Input.GetKey(KeyCode.F)) 
        {
            rb.AddTorque(-transform.right * torqueStrength * 0.5f, ForceMode.Force);
        }
    }
    
    void HandleSpecialForces() 
    {
        // Explosionskraft
        if (Input.GetKeyDown(explosionKey)) 
        {
            Vector3 explosionCenter = transform.position - Vector3.up * 2f;
            rb.AddExplosionForce(explosionForce, explosionCenter, explosionRadius, 3f);
            
            // Alle anderen Rigidbodies in der Naehe beeinflussen
            ApplyExplosionToNearbyObjects(explosionCenter);
        }
        
        // Kraft an spezifischer Position (erzeugt Kraft und Drehmoment)
        if (Input.GetKey(KeyCode.T)) 
        {
            Vector3 forcePosition = transform.position + transform.right;
            Vector3 force = transform.up * thrustForce;
            rb.AddForceAtPosition(force, forcePosition, ForceMode.Force);
        }
    }
    
    void LimitVelocity() 
    {
        // Geschwindigkeitsbegrenzung
        if (rb.velocity.magnitude > maxVelocity) 
        {
            rb.velocity = rb.velocity.normalized * maxVelocity;
        }
    }
    
    void CreateThrustEffects() 
    {
        // Partikeleffekte, Sounds, etc.
        // Hier wuerden normalerweise visuelle Effekte erstellt
    }
    
    void ApplyExplosionToNearbyObjects(Vector3 center) 
    {
        Collider[] nearbyObjects = Physics.OverlapSphere(center, explosionRadius);
        
        foreach (Collider col in nearbyObjects) 
        {
            Rigidbody nearbyRb = col.GetComponent<Rigidbody>();
            if (nearbyRb != null && nearbyRb != rb) 
            {
                nearbyRb.AddExplosionForce(explosionForce, center, explosionRadius);
            }
        }
    }
}
\end{lstlisting}
\end{code}

\subsection{Kollisionserkennung und -reaktion}

\begin{definition}{Unity Kollisions-Events}
    \textbf{Kollisions-Events (für Collider):}
    \begin{itemize}
        \item \textbf{OnCollisionEnter()}: Beim Start der Kollision
        \item \textbf{OnCollisionStay()}: Während anhaltender Kollision
        \item \textbf{OnCollisionExit()}: Beim Ende der Kollision
    \end{itemize}
    
    \textbf{Trigger-Events (für Trigger):}
    \begin{itemize}
        \item \textbf{OnTriggerEnter()}: Objekt betritt Trigger-Zone
        \item \textbf{OnTriggerStay()}: Objekt bleibt in Trigger-Zone
        \item \textbf{OnTriggerExit()}: Objekt verlässt Trigger-Zone
    \end{itemize}
    
    \textbf{Kollisions-Informationen:}
    \begin{itemize}
        \item \textbf{Collision.contacts}: Kontaktpunkte
        \item \textbf{Collision.impulse}: Impuls der Kollision
        \item \textbf{Collision.relativeVelocity}: Relative Geschwindigkeit
        \item \textbf{ContactPoint.normal}: Oberflächennormale
    \end{itemize}
\end{definition}

\begin{code}{Erweiterte Kollisionsbehandlung}
\begin{lstlisting}[language=C, style=basesmol]
public class CollisionAnalyzer : MonoBehaviour 
{
    [Header("Kollisions-Einstellungen")]
    public float restitution = 0.8f;           // Restitutionskoeffizient
    public float minVelocityForSound = 2f;     // Min. Geschw. fuer Sound
    public float minVelocityForSparks = 5f;    // Min. Geschw. fuer Funken
    public AudioClip collisionSound;
    public ParticleSystem sparkEffect;
    
    [Header("Material-Eigenschaften")]
    public float metalDensity = 7800f;         // kg/m^3
    public float woodDensity = 600f;
    public float plasticDensity = 1200f;
    
    private Rigidbody rb;
    private AudioSource audioSource;
    
    void Start() 
    {
        rb = GetComponent<Rigidbody>();
        audioSource = GetComponent<AudioSource>();
    }
    
    void OnCollisionEnter(Collision collision) 
    {
        // Kollisionsdetails berechnen
        Vector3 relativeVelocity = rb.velocity - 
                                  (collision.rigidbody?.velocity ?? Vector3.zero);
        float collisionSpeed = relativeVelocity.magnitude;
        float collisionEnergy = 0.5f * rb.mass * collisionSpeed * collisionSpeed;
        
        // Sound basierend auf Kollisionsintensitaet
        if (collisionSpeed > minVelocityForSound && audioSource && collisionSound) 
        {
            float volume = Mathf.Clamp01(collisionSpeed / 10f);
            float pitch = 0.8f + collisionSpeed / 20f;
            audioSource.pitch = pitch;
            audioSource.PlayOneShot(collisionSound, volume);
        }
        
        // Partikeleffekte fuer starke Kollisionen
        if (collisionSpeed > minVelocityForSparks && sparkEffect) 
        {
            Vector3 contactPoint = collision.contacts[0].point;
            CreateSparkEffect(contactPoint, collision.contacts[0].normal);
        }
        
        // Materialspezifische Reaktionen
        HandleMaterialInteraction(collision, collisionSpeed);
        
        // Benutzerdefinierte Kollisionsreaktion
        ApplyCustomCollisionResponse(collision, relativeVelocity);
        
        // Kollisionsdaten fuer Analyse protokollieren
        LogCollisionData(collision, collisionSpeed, collisionEnergy);
    }
    
    void HandleMaterialInteraction(Collision collision, float speed) 
    {
        // Materialeigenschaften basierend auf Tags
        string otherTag = collision.gameObject.tag;
        
        switch (otherTag) 
        {
            case "Metal":
                // Metallische Kollision - hoher Restitutionskoeffizient
                restitution = 0.9f;
                break;
            case "Wood":
                // Holz - mittlerer Restitutionskoeffizient
                restitution = 0.6f;
                break;
            case "Rubber":
                // Gummi - sehr hoher Restitutionskoeffizient
                restitution = 0.95f;
                break;
            default:
                restitution = 0.8f;
                break;
        }
    }
    
    void ApplyCustomCollisionResponse(Collision collision, Vector3 relativeVelocity) 
    {
        // Fuer spezielle Kollisionstypen
        if (collision.gameObject.CompareTag("Bouncy")) 
        {
            ApplyBouncyCollision(collision, relativeVelocity);
        }
        else if (collision.gameObject.CompareTag("Sticky")) 
        {
            ApplyStickyCollision(collision);
        }
        else if (collision.gameObject.CompareTag("Breakable")) 
        {
            CheckForBreaking(collision, relativeVelocity.magnitude);
        }
    }
    
    void ApplyBouncyCollision(Collision collision, Vector3 relativeVelocity) 
    {
        // Kollisionsnormale ermitteln
        Vector3 normal = collision.contacts[0].normal;
        
        // Reflektierte Geschwindigkeit berechnen
        Vector3 reflectedVelocity = Vector3.Reflect(relativeVelocity, normal);
        
        // Restitution anwenden
        rb.velocity = reflectedVelocity * restitution;
    }
    
    void ApplyStickyCollision(Collision collision) 
    {
        // Objekte "kleben" zusammen
        if (collision.rigidbody != null) 
        {
            // Joint erstellen fuer klebende Verbindung
            FixedJoint joint = gameObject.AddComponent<FixedJoint>();
            joint.connectedBody = collision.rigidbody;
            joint.breakForce = 1000f;  // Kraft zum Trennen
        }
    }
    
    void CheckForBreaking(Collision collision, float impactSpeed) 
    {
        // Zerstoerung bei starkem Aufprall
        float breakingThreshold = 10f;
        
        if (impactSpeed > breakingThreshold) 
        {
            BreakObject(collision.gameObject);
        }
    }
    
    void BreakObject(GameObject obj) 
    {
        // Objekt in Fragmente zerteilen
        // Hier wuerde normalerweise ein Fragmentierungs-System aufgerufen
        Debug.Log($"Object {obj.name} shattered due to high impact!");
        
        // Partikeleffekt fuer Zerstoerung
        // CreateDestructionEffect(obj.transform.position);
        
        // Objekt deaktivieren oder zerstoeren
        obj.SetActive(false);
    }
    
    void CreateSparkEffect(Vector3 position, Vector3 normal) 
    {
        if (sparkEffect != null) 
        {
            ParticleSystem sparks = Instantiate(sparkEffect, position, 
                                               Quaternion.LookRotation(normal));
            sparks.Play();
            
            // Automatisches Loeschen nach Abschluss
            Destroy(sparks.gameObject, sparks.main.duration + 2f);
        }
    }
    
    void LogCollisionData(Collision collision, float speed, float energy) 
    {
        Debug.Log($"Kollision mit {collision.gameObject.name}: " +
                  $"Geschwindigkeit: {speed:F2} m/s, " +
                  $"Energie: {energy:F2} J, " +
                  $"Kontaktpunkte: {collision.contactCount}, " +
                  $"Impuls: {collision.impulse.magnitude:F2} N*s");
    }
}
\end{lstlisting}
\end{code}

\subsection{Benutzerdefinierte Physik-Implementierung}

\begin{concept}{Numerische Integrationsmethoden}
    Unity verwendet implizite Euler-Integration, aber benutzerdefinierte Implementierungen können verwenden:
    \begin{itemize}
        \item \textbf{Expliziter Euler}: Einfach, aber potentiell instabil
        \item \textbf{Verlet-Integration}: Bessere Energieerhaltung
        \item \textbf{Runge-Kutta}: Höhere Genauigkeit für komplexe Systeme
        \item \textbf{Leapfrog}: Gut für Orbitalprobleme
    \end{itemize}
    
    \textbf{Vergleich der Methoden:}
    \begin{itemize}
        \item \textbf{Stabilität}: Verlet > Runge-Kutta > Euler
        \item \textbf{Genauigkeit}: Runge-Kutta > Verlet > Euler
        \item \textbf{Performance}: Euler > Verlet > Runge-Kutta
        \item \textbf{Energieerhaltung}: Verlet > Runge-Kutta > Euler
    \end{itemize}
\end{concept}

\begin{code}{Benutzerdefinierter Physik-Integrator}
\begin{lstlisting}[language=C, style=basesmol]
public class CustomPhysicsObject : MonoBehaviour 
{
    [Header("Physik-Eigenschaften")]
    public float mass = 1f;
    public Vector3 velocity = Vector3.zero;
    public Vector3 acceleration = Vector3.zero;
    public bool useGravity = true;
    public float damping = 0.1f;
    
    [Header("Integrations-Einstellungen")]
    public enum IntegrationMethod { Euler, Verlet, RungeKutta, Leapfrog }
    public IntegrationMethod method = IntegrationMethod.Verlet;
    public bool showTrajectory = true;
    public int trajectoryPoints = 50;
    
    [Header("Kraefte")]
    public Vector3 constantForce = Vector3.zero;
    public float springConstant = 50f;
    public Vector3 springCenter = Vector3.zero;
    
    private Vector3 previousPosition;
    private Vector3 currentPosition;
    private List<Vector3> trajectory = new List<Vector3>();
    private LineRenderer trajectoryRenderer;
    
    void Start() 
    {
        currentPosition = transform.position;
        previousPosition = currentPosition;
        
        SetupTrajectoryRenderer();
        
        // Initiale Geschwindigkeit setzen
        if (velocity == Vector3.zero) 
        {
            velocity = new Vector3(5f, 0f, 0f); // Standardgeschwindigkeit
        }
    }
    
    void SetupTrajectoryRenderer() 
    {
        if (showTrajectory) 
        {
            trajectoryRenderer = gameObject.AddComponent<LineRenderer>();
            trajectoryRenderer.material = new Material(Shader.Find("Sprites/Default"));
            trajectoryRenderer.color = Color.yellow;
            trajectoryRenderer.startWidth = 0.05f;
            trajectoryRenderer.endWidth = 0.05f;
            trajectoryRenderer.positionCount = trajectoryPoints;
        }
    }
    
    void FixedUpdate() 
    {
        // Kraefte berechnen
        Vector3 totalForce = CalculateForces();
        acceleration = totalForce / mass;
        
        // Integration basierend auf gewaehlter Methode
        switch (method) 
        {
            case IntegrationMethod.Euler:
                EulerIntegration();
                break;
            case IntegrationMethod.Verlet:
                VerletIntegration();
                break;
            case IntegrationMethod.RungeKutta:
                RungeKuttaIntegration();
                break;
            case IntegrationMethod.Leapfrog:
                LeapfrogIntegration();
                break;
        }
        
        // Transform aktualisieren
        transform.position = currentPosition;
        
        // Trajektorie aktualisieren
        UpdateTrajectory();
        
        // Energieanalyse
        AnalyzeEnergy();
    }
    
    Vector3 CalculateForces() 
    {
        Vector3 totalForce = Vector3.zero;
        
        // Gravitation
        if (useGravity) 
        {
            totalForce += mass * Physics.gravity;
        }
        
        // Konstante Kraft
        totalForce += constantForce;
        
        // Federkraft zum Zentrum
        Vector3 displacement = currentPosition - springCenter;
        totalForce += -springConstant * displacement;
        
        // Daempfung (geschwindigkeitsproportional)
        totalForce += -damping * velocity;
        
        // Luftwiderstand (geschwindigkeitsquadratisch)
        float airResistance = 0.01f;
        totalForce += -airResistance * velocity.magnitude * velocity;
        
        return totalForce;
    }
    
    void EulerIntegration() 
    {
        float dt = Time.fixedDeltaTime;
        
        velocity += acceleration * dt;
        currentPosition += velocity * dt;
    }
    
    void VerletIntegration() 
    {
        float dt = Time.fixedDeltaTime;
        
        Vector3 newPosition = 2f * currentPosition - previousPosition + 
                             acceleration * dt * dt;
        
        velocity = (newPosition - previousPosition) / (2f * dt);
        
        previousPosition = currentPosition;
        currentPosition = newPosition;
    }
    
    void RungeKuttaIntegration() 
    {
        float dt = Time.fixedDeltaTime;
        
        // RK4-Implementierung
        Vector3 k1v = acceleration;
        Vector3 k1x = velocity;
        
        Vector3 k2v = CalculateAcceleration(currentPosition + k1x * dt * 0.5f, 
                                           velocity + k1v * dt * 0.5f);
        Vector3 k2x = velocity + k1v * dt * 0.5f;
        
        Vector3 k3v = CalculateAcceleration(currentPosition + k2x * dt * 0.5f, 
                                           velocity + k2v * dt * 0.5f);
        Vector3 k3x = velocity + k2v * dt * 0.5f;
        
        Vector3 k4v = CalculateAcceleration(currentPosition + k3x * dt, 
                                           velocity + k3v * dt);
        Vector3 k4x = velocity + k3v * dt;
        
        velocity += (k1v + 2f * k2v + 2f * k3v + k4v) * dt / 6f;
        currentPosition += (k1x + 2f * k2x + 2f * k3x + k4x) * dt / 6f;
    }
    
    void LeapfrogIntegration() 
    {
        float dt = Time.fixedDeltaTime;
        
        // Leapfrog fuer bessere Energieerhaltung
        velocity += acceleration * dt;
        currentPosition += velocity * dt;
    }
    
    Vector3 CalculateAcceleration(Vector3 pos, Vector3 vel) 
    {
        Vector3 force = Vector3.zero;
        
        if (useGravity) force += mass * Physics.gravity;
        force += constantForce;
        force += -springConstant * (pos - springCenter);
        force += -damping * vel;
        
        return force / mass;
    }
    
    void UpdateTrajectory() 
    {
        if (!showTrajectory) return;
        
        trajectory.Add(currentPosition);
        if (trajectory.Count > trajectoryPoints) 
        {
            trajectory.RemoveAt(0);
        }
        
        if (trajectoryRenderer != null) 
        {
            trajectoryRenderer.positionCount = trajectory.Count;
            trajectoryRenderer.SetPositions(trajectory.ToArray());
        }
    }
    
    void AnalyzeEnergy() 
    {
        // Kinetische Energie
        float kineticEnergy = 0.5f * mass * velocity.sqrMagnitude;
        
        // Potentielle Energie (Gravitation)
        float gravitationalPE = mass * Mathf.Abs(Physics.gravity.y) * 
                               (currentPosition.y - springCenter.y);
        
        // Elastische potentielle Energie
        Vector3 displacement = currentPosition - springCenter;
        float elasticPE = 0.5f * springConstant * displacement.sqrMagnitude;
        
        // Gesamtenergie
        float totalEnergy = kineticEnergy + gravitationalPE + elasticPE;
        
        // Debug-Ausgabe alle 50 Frames
        if (Time.fixedTime % 1f < Time.fixedDeltaTime) 
        {
            Debug.Log($"Energie - Kinetisch: {kineticEnergy:F2}, " +
                      $"Potentiell: {gravitationalPE + elasticPE:F2}, " +
                      $"Gesamt: {totalEnergy:F2}");
        }
    }
}
\end{lstlisting}
\end{code}

\subsection{Performance-Optimierung}

\begin{concept}{Physik-Performance-Tipps}
    \textbf{Kollisionserkennung optimieren:}
    \begin{itemize}
        \item Geeignete Kollisionserkennungs-Modi verwenden
        \item Compound Collider für komplexe Geometrien
        \item Mesh Collider nur wenn nötig (sehr teuer)
        \item Layer-basierte Kollisionsmatrix definieren
    \end{itemize}
    
    \textbf{Rigidbody-Management:}
    \begin{itemize}
        \item Anzahl aktiver Rigidbodies minimieren
        \item Ruhende Objekte als Kinematic setzen
        \item Object Pooling für Projektile verwenden
        \item Unnötige Constraints vermeiden
    \end{itemize}
    
    \textbf{Optimierungen:}
    \begin{itemize}
        \item Fixed Timestep angemessen anpassen
        \item Solver-Iterationen reduzieren wenn möglich
        \item Continuous Collision Detection sparsam einsetzen
        \item Physics Raycasts in Threads auslagern
    \end{itemize}
\end{concept}

\begin{code}{Physik-Objekt-Pool}
\begin{lstlisting}[language=C, style=basesmol]
public class PhysicsObjectPool : MonoBehaviour 
{
    [Header("Pool-Einstellungen")]
    public GameObject prefab;
    public int poolSize = 100;
    public int maxActiveObjects = 50;
    public float autoReturnTime = 10f;
    
    [Header("Performance")]
    public bool enableStatistics = true;
    public float cleanupInterval = 5f;
    
    private Queue<GameObject> pool = new Queue<GameObject>();
    private List<PooledObject> activeObjects = new List<PooledObject>();
    private Transform poolParent;
    
    private class PooledObject 
    {
        public GameObject gameObject;
        public float spawnTime;
        public Rigidbody rigidbody;
        
        public PooledObject(GameObject obj) 
        {
            gameObject = obj;
            spawnTime = Time.time;
            rigidbody = obj.GetComponent<Rigidbody>();
        }
    }
    
    void Start() 
    {
        // Pool-Container erstellen
        poolParent = new GameObject("ObjectPool").transform;
        poolParent.SetParent(transform);
        
        // Pool vorab fuellen
        for (int i = 0; i < poolSize; i++) 
        {
            GameObject obj = Instantiate(prefab, poolParent);
            obj.SetActive(false);
            pool.Enqueue(obj);
        }
        
        // Regelmaessige Bereinigung starten
        InvokeRepeating(nameof(CleanupObjects), cleanupInterval, cleanupInterval);
    }
    
    public GameObject GetObject(Vector3 position, Vector3 velocity, Vector3 angularVelocity = default) 
    {
        // Begrenzung aktiver Objekte
        if (activeObjects.Count >= maxActiveObjects) 
        {
            // Aeltestes Objekt zurueckgeben
            ReturnOldestObject();
        }
        
        GameObject obj;
        
        if (pool.Count > 0) 
        {
            obj = pool.Dequeue();
        } 
        else 
        {
            // Pool erschoepft, neues Objekt erstellen
            obj = Instantiate(prefab, poolParent);
            Debug.LogWarning("Pool exhausted, creating new object");
        }
        
        // Objekt initialisieren
        obj.transform.position = position;
        obj.transform.rotation = Random.rotation;
        obj.SetActive(true);
        
        Rigidbody rb = obj.GetComponent<Rigidbody>();
        if (rb != null) 
        {
            rb.velocity = velocity;
            rb.angularVelocity = angularVelocity;
            
            // Physik-Zustand zuruecksetzen
            rb.Sleep();
            rb.WakeUp();
        }
        
        // Zur aktiven Liste hinzufuegen
        PooledObject pooledObj = new PooledObject(obj);
        activeObjects.Add(pooledObj);
        
        return obj;
    }
    
    public void ReturnObject(GameObject obj) 
    {
        if (obj == null) return;
        
        // Aus aktiver Liste entfernen
        for (int i = activeObjects.Count - 1; i >= 0; i--) 
        {
            if (activeObjects[i].gameObject == obj) 
            {
                activeObjects.RemoveAt(i);
                break;
            }
        }
        
        // Physik-Zustand zuruecksetzen
        Rigidbody rb = obj.GetComponent<Rigidbody>();
        if (rb != null) 
        {
            rb.velocity = Vector3.zero;
            rb.angularVelocity = Vector3.zero;
            rb.Sleep();
        }
        
        // Objekt deaktivieren und zurueck in Pool
        obj.SetActive(false);
        obj.transform.SetParent(poolParent);
        pool.Enqueue(obj);
    }
    
    void ReturnOldestObject() 
    {
        if (activeObjects.Count > 0) 
        {
            PooledObject oldest = activeObjects[0];
            ReturnObject(oldest.gameObject);
        }
    }
    
    void CleanupObjects() 
    {
        // Objekte zurueckgeben, die zu lange aktiv waren
        for (int i = activeObjects.Count - 1; i >= 0; i--) 
        {
            PooledObject obj = activeObjects[i];
            
            if (Time.time - obj.spawnTime > autoReturnTime) 
            {
                ReturnObject(obj.gameObject);
            }
            // Objekte zurueckgeben, die ausserhalb der Welt gefallen sind
            else if (obj.gameObject.transform.position.y < -50f) 
            {
                ReturnObject(obj.gameObject);
            }
            // Objekte zurueckgeben, die zu langsam sind
            else if (obj.rigidbody != null && obj.rigidbody.velocity.magnitude < 0.1f && 
                     Time.time - obj.spawnTime > 2f) 
            {
                ReturnObject(obj.gameObject);
            }
        }
        
        // Statistiken ausgeben
        if (enableStatistics) 
        {
            Debug.Log($"Pool-Statistiken: Aktiv: {activeObjects.Count}, " +
                      $"Pool: {pool.Count}, Total: {poolSize}");
        }
    }
    
    // Oeffentliche API fuer externe Nutzung
    public void ReturnAllObjects() 
    {
        while (activeObjects.Count > 0) 
        {
            ReturnObject(activeObjects[0].gameObject);
        }
    }
    
    public int GetActiveCount() => activeObjects.Count;
    public int GetPoolCount() => pool.Count;
    
    void OnDrawGizmosSelected() 
    {
        // Visualisierung der aktiven Objekte
        Gizmos.color = Color.green;
        foreach (var obj in activeObjects) 
        {
            if (obj.gameObject != null) 
            {
                Gizmos.DrawWireSphere(obj.gameObject.transform.position, 0.5f);
            }
        }
    }
}
\end{lstlisting}
\end{code}

\subsection{Prüfungsvorbereitung}

\begin{KR}{Unity Physics Prüfungsstrategie}
    \paragraph{Kernphysik-Konzepte beherrschen}
    \begin{itemize}
        \item Newton'sche Gesetze und ihre Anwendungen
        \item Energie- und Impulserhaltung verstehen
        \item Rotationsdynamik-Berechnungen üben
        \item Wichtige Formeln und ihre Herleitungen kennen
        \item Vektormathematik und Koordinatensysteme
    \end{itemize}
    
    \paragraph{Unity-Implementierungs-Wissen}
    \begin{itemize}
        \item Rigidbody-Komponenten-Eigenschaften und -Methoden
        \item Kraftanwendungstechniken und Force-Modi
        \item Kollisionserkennung und -reaktion
        \item Koordinatensystem-Transformationen
        \item Physics Material und Reibungsmodelle
    \end{itemize}
    
    \paragraph{Problemlösungsansatz}
    \begin{itemize}
        \item Beteiligte physikalische Prinzipien identifizieren
        \item Koordinatensystem und Freikörperdiagramme aufstellen
        \item Erhaltungsgesetze anwenden wo angemessen
        \item In Unity-Implementierungskonzepte übersetzen
        \item Code-Snippets für häufige Szenarien memorieren
    \end{itemize}
    
    \paragraph{Häufige Prüfungsthemen}
    \begin{itemize}
        \item Projektilbewegung mit Unity-Vektoren
        \item Kollisionsanalyse und Impulserhaltung
        \item Rotationsbewegung und Drehmoment-Berechnungen
        \item Energieerhaltung in mechanischen Systemen
        \item Custom Physics Implementation
        \item Performance-Optimierung in Unity Physics
    \end{itemize}
\end{KR}

\begin{example2}{Umfassendes Physik-Problem}
    Eine Kugel (Masse 1 kg, Radius 0.1 m) rollt eine 30°-Schräge hinunter, kollidiert dann elastisch mit einer ruhenden Kugel gleicher Masse. Berechne die Endgeschwindigkeiten und implementiere die Simulation in Unity.
    \tcblower
    \textbf{Teil 1: Rollen auf der Schräge}
    
    Für Vollkugel: $I = \frac{2}{5}mr^2$, Beschleunigung: $a = \frac{5g\sin\theta}{7}$
    
    $a = \frac{5 \times 9.81 \times \sin(30°)}{7} = 3.51 \, m/s^2$
    
    Nach Rollstrecke $d = 2m$: $v = \sqrt{2ad} = \sqrt{2 \times 3.51 \times 2} = 3.74 \, m/s$
    
    \textbf{Teil 2: Elastische Kollision}
    
    Für gleiche Massen in 1D elastischer Kollision:
    $v_1' = 0$, $v_2' = v_1 = 3.74 \, m/s$
    
    \textbf{Unity-Implementierung:}
\begin{lstlisting}[language=C, style=basesmol]
public class RollingBallSimulation : MonoBehaviour 
{
    [Header("Physik-Parameter")]
    public float inclineAngle = 30f;
    public float ballMass = 1f;
    public float ballRadius = 0.1f;
    public float rollDistance = 2f;
    
    [Header("Simulation")]
    public bool autoStart = true;
    public float simulationSpeed = 1f;
    
    private GameObject ball1, ball2;
    private Rigidbody rb1, rb2;
    
    void Start() 
    {
        if (autoStart) 
        {
            StartSimulation();
        }
    }
    
    public void StartSimulation() 
    {
        // Theoretische Berechnungen
        CalculateTheoreticalValues();
        
        // Physik-Simulation aufsetzen
        SetupSimulation();
    }
    
    void CalculateTheoreticalValues() 
    {
        float angleRad = inclineAngle * Mathf.Deg2Rad;
        float rollingAccel = (5f * Physics.gravity.magnitude * 
                             Mathf.Sin(angleRad)) / 7f;
        
        float finalVelocity = Mathf.Sqrt(2f * rollingAccel * rollDistance);
        
        Debug.Log($"Theoretische Werte:");
        Debug.Log($"Rollbeschleunigung: {rollingAccel:F2} m/s^2");
        Debug.Log($"Endgeschwindigkeit vor Kollision: {finalVelocity:F2} m/s");
        Debug.Log($"Nach elastischer Kollision: Ball1 = 0 m/s, Ball2 = {finalVelocity:F2} m/s");
    }
    
    void SetupSimulation() 
    {
        // Schraege erstellen
        SetupIncline();
        
        // Kugeln erstellen
        SetupBalls();
        
        // Physik-Materialien konfigurieren
        SetupPhysicsMaterials();
    }
    
    void SetupIncline() 
    {
        GameObject incline = GameObject.CreatePrimitive(PrimitiveType.Cube);
        incline.name = "Incline";
        incline.transform.localScale = new Vector3(rollDistance * 2f, 0.1f, 1f);
        incline.transform.position = new Vector3(-rollDistance / 2f, -0.5f, 0);
        incline.transform.rotation = Quaternion.Euler(0, 0, -inclineAngle);
        
        // Material fuer Rollen ohne Rutschen
        PhysicMaterial rollMaterial = new PhysicMaterial("Rolling");
        rollMaterial.staticFriction = 1f;
        rollMaterial.dynamicFriction = 1f;
        rollMaterial.frictionCombine = PhysicMaterialCombine.Maximum;
        
        incline.GetComponent<Collider>().material = rollMaterial;
    }
    
    void SetupBalls() 
    {
        // Rollende Kugel (Ball 1)
        ball1 = GameObject.CreatePrimitive(PrimitiveType.Sphere);
        ball1.name = "RollingBall";
        ball1.transform.position = new Vector3(-rollDistance - 1f, 2f, 0);
        ball1.transform.localScale = Vector3.one * ballRadius * 2f;
        
        rb1 = ball1.AddComponent<Rigidbody>();
        rb1.mass = ballMass;
        rb1.drag = 0f;
        rb1.angularDrag = 0f;
        
        // Ruhende Zielkugel (Ball 2)
        ball2 = GameObject.CreatePrimitive(PrimitiveType.Sphere);
        ball2.name = "TargetBall";
        ball2.transform.position = new Vector3(rollDistance + 1f, ballRadius, 0);
        ball2.transform.localScale = Vector3.one * ballRadius * 2f;
        
        rb2 = ball2.AddComponent<Rigidbody>();
        rb2.mass = ballMass;
        rb2.useGravity = true;
        
        // Kollisions-Handler hinzufuegen
        ball1.AddComponent<ElasticCollisionHandler>();
        ball2.AddComponent<ElasticCollisionHandler>();
        
        // Visualisierung
        ball1.GetComponent<Renderer>().material.color = Color.red;
        ball2.GetComponent<Renderer>().material.color = Color.blue;
    }
    
    void SetupPhysicsMaterials() 
    {
        // Elastisches Material fuer Kollisionen
        PhysicMaterial elasticMaterial = new PhysicMaterial("Elastic");
        elasticMaterial.bounciness = 1f;
        elasticMaterial.frictionCombine = PhysicMaterialCombine.Minimum;
        elasticMaterial.bounceCombine = PhysicMaterialCombine.Maximum;
        
        ball1.GetComponent<Collider>().material = elasticMaterial;
        ball2.GetComponent<Collider>().material = elasticMaterial;
        
        // Globale Physik-Einstellungen
        Time.timeScale = simulationSpeed;
        Physics.bounceThreshold = 0f; // Minimaler Bounce-Threshold
    }
    
    void Update() 
    {
        // Geschwindigkeiten anzeigen
        if (rb1 != null && rb2 != null) 
        {
            if (Input.GetKeyDown(KeyCode.Space)) 
            {
                Debug.Log($"Ball1 Geschwindigkeit: {rb1.velocity.magnitude:F2} m/s");
                Debug.Log($"Ball2 Geschwindigkeit: {rb2.velocity.magnitude:F2} m/s");
            }
        }
        
        // Simulation zuruecksetzen
        if (Input.GetKeyDown(KeyCode.R)) 
        {
            ResetSimulation();
        }
    }
    
    void ResetSimulation() 
    {
        if (ball1 != null) DestroyImmediate(ball1);
        if (ball2 != null) DestroyImmediate(ball2);
        
        SetupBalls();
    }
}

public class ElasticCollisionHandler : MonoBehaviour 
{
    private bool hasCollided = false;
    
    void OnCollisionEnter(Collision collision) 
    {
        if (hasCollided) return;
        
        Rigidbody otherRb = collision.rigidbody;
        if (otherRb == null) return;
        
        Rigidbody thisRb = GetComponent<Rigidbody>();
        
        // Nur zwischen Kugeln, nicht mit Schraege
        if (collision.gameObject.name.Contains("Ball")) 
        {
            // Kollisionsdaten protokollieren
            float v1_before = thisRb.velocity.magnitude;
            float v2_before = otherRb.velocity.magnitude;
            
            Debug.Log($"Kollision erkannt!");
            Debug.Log($"Geschwindigkeiten vorher: {v1_before:F2}, {v2_before:F2}");
            
            // Fuer gleiche Massen in elastischer Kollision
            Vector3 tempVelocity = thisRb.velocity;
            thisRb.velocity = otherRb.velocity;
            otherRb.velocity = tempVelocity;
            
            // Nach Kollision protokollieren
            StartCoroutine(LogPostCollisionVelocities(thisRb, otherRb));
            
            hasCollided = true;
        }
    }
    
    System.Collections.IEnumerator LogPostCollisionVelocities(Rigidbody rb1, Rigidbody rb2) 
    {
        yield return new WaitForFixedUpdate();
        
        Debug.Log($"Geschwindigkeiten nachher: {rb1.velocity.magnitude:F2}, {rb2.velocity.magnitude:F2}");
        
        // Energieerhaltung pruefen
        float kineticBefore = 0.5f * rb1.mass * Mathf.Pow(3.74f, 2);
        float kineticAfter = 0.5f * rb1.mass * rb1.velocity.sqrMagnitude + 
                           0.5f * rb2.mass * rb2.velocity.sqrMagnitude;
        
        Debug.Log($"Kinetische Energie vorher: {kineticBefore:F2} J");
        Debug.Log($"Kinetische Energie nachher: {kineticAfter:F2} J");
        Debug.Log($"Energieerhaltung: {Mathf.Abs(kineticBefore - kineticAfter) < 0.1f}");
    }
}
\end{lstlisting}
\end{example2}

\begin{concept}{Wichtige Erkenntnisse}
    \textbf{Theoretische Grundlagen:}
    \begin{itemize}
        \item Physikalische Gesetze gelten universell
        \item Erhaltungsgesetze sind zentral für Problemlösungen
        \item Numerische Integration beeinflusst Genauigkeit
        \item Reibung und Dämpfung sind realitätsnah wichtig
    \end{itemize}
    
    \textbf{Unity-spezifische Aspekte:}
    \begin{itemize}
        \item Fixed Timestep ist entscheidend für Stabilität
        \item Physics Materials beeinflussen Verhalten stark
        \item Performance-Optimierung ist für komplexe Szenen notwendig
        \item Debugging-Tools helfen bei der Problemanalyse
    \end{itemize}
\end{concept}