\section{Dynamik der Translation}

\subsection{Grundlagen der Dynamik}
\begin{definition}{Dynamik}\\
    Die Dynamik beschäftigt sich mit den Ursachen von Bewegungen, also den Kräften, die auf einen Körper wirken. Sie baut auf der Kinematik auf und erweitert diese um die Betrachtung der wirkenden Kräfte.
\end{definition}

\begin{concept}{Newton'sche Axiome}\\
    Isaac Newton formulierte die drei grundlegenden Gesetze der Bewegung:
    \begin{enumerate}
        \item Trägheitsgesetz: Ein Körper bleibt im Zustand der Ruhe oder der gleichförmigen geradlinigen Bewegung, solange keine Kraft auf ihn wirkt.
        \item Bewegungsgesetz: Die Änderung der Bewegung ist proportional zur einwirkenden Kraft und erfolgt in Richtung der Kraft.
        \item Wechselwirkungsgesetz: Übt ein Körper auf einen anderen eine Kraft aus (actio), so wirkt eine gleich große, entgegengesetzte Kraft zurück (reactio).
    \end{enumerate}
    Ein viertes Prinzip ist das Superpositionsprinzip: Kräfte addieren sich vektoriell.
\end{concept}

\subsection{Erstes Newton'sches Gesetz (Trägheitsgesetz)}
\begin{formula}{Trägheitsgesetz}\\
    Das erste Newton'sche Gesetz lässt sich mathematisch ausdrücken als:
    \begin{equation}
        \vec{F} = 0 \Rightarrow \vec{v} = \text{const.}
    \end{equation}
    
    Dies bedeutet auch:
    \begin{equation}
        \vec{F} = 0 \Rightarrow \vec{a} = \frac{d\vec{v}}{dt} = \frac{d}{dt}(\text{const.}) = 0
    \end{equation}
    
    Das Gesetz gilt nur in sogenannten Inertialsystemen: Bezugssystemen ohne Beschleunigung oder Rotation.
\end{formula}

\begin{example2}{Trägheitsgesetz auf der Luftkissenbahn}\\
    Ein Gleiter auf einer Luftkissenbahn bewegt sich ohne Reibung mit konstanter Geschwindigkeit weiter, wenn keine Kraft auf ihn wirkt. Dies ist eine direkte Demonstration des Trägheitsgesetzes.
    
    Das gleiche Prinzip gilt für einen Stein, der von der Spitze eines fahrenden Schiffes fällt - er behält seine horizontale Geschwindigkeit bei und landet am Fuß des Mastes, nicht dahinter oder davor.
\end{example2}

\subsection{Zweites Newton'sches Gesetz (Bewegungsgesetz)}
\begin{definition}{Impuls}\\
    Der Impuls $\vec{p}$ eines Körpers ist das Produkt aus seiner Masse und seiner Geschwindigkeit:
    \begin{equation}
        \vec{p} = m \cdot \vec{v}
    \end{equation}
    
    Der Impuls ist eine vektorielle Größe mit der Einheit kg·m/s.
\end{definition}

\begin{formula}{Bewegungsgesetz}\\
    Das zweite Newton'sche Gesetz in seiner allgemeinen Form:
    \begin{equation}
        \vec{F} = \frac{d\vec{p}}{dt}
    \end{equation}
    
    Für Körper mit konstanter Masse vereinfacht sich dies zu:
    \begin{equation}
        \vec{F} = m \cdot \vec{a}
    \end{equation}
    
    In integraler Form:
    \begin{equation}
        \vec{p} = \vec{p}_0 + \int_0^t \vec{F}(t) \, dt
    \end{equation}
    wobei $\int \vec{F} \, dt$ als Kraftstoß bezeichnet wird.
\end{formula}

\begin{KR}{Berechnung von Kräften und Beschleunigungen}\\
    \paragraph{Berechnung der Beschleunigung aus Kräften}
    \begin{itemize}
        \item Alle auf den Körper wirkenden Kräfte identifizieren
        \item Kräfte vektoriell addieren zur resultierenden Kraft $\vec{F}_{res}$
        \item Beschleunigung berechnen: $\vec{a} = \frac{\vec{F}_{res}}{m}$
    \end{itemize}
    
    \paragraph{Beispiel: Fahrstuhl}
    \begin{itemize}
        \item Fahrstuhl in Ruhe: $F_{\text{Seil}} - F_G = 0 \Rightarrow F_{\text{Seil}} = m \cdot g$
        \item Fahrstuhl beschleunigt nach oben: $F_{\text{Seil}} - F_G = m \cdot a \Rightarrow F_{\text{Seil}} = m \cdot (g + a)$
        \item Fahrstuhl beschleunigt nach unten: $F_{\text{Seil}} - F_G = -m \cdot a \Rightarrow F_{\text{Seil}} = m \cdot (g - a)$
    \end{itemize}
\end{KR}

\subsection{Drittes Newton'sches Gesetz (Wechselwirkungsgesetz)}
\begin{formula}{Wechselwirkungsgesetz}\\
    Das dritte Newton'sche Gesetz lautet mathematisch:
    \begin{equation}
        \vec{F}_{12} = -\vec{F}_{21}
    \end{equation}
    wobei $\vec{F}_{12}$ die Kraft bezeichnet, die Körper 1 auf Körper 2 ausübt, und $\vec{F}_{21}$ die Kraft, die Körper 2 auf Körper 1 ausübt.
\end{formula}

\begin{concept}{Merkmale von Kräftepaaren}\\
    Kräftepaare des Wechselwirkungsgesetzes haben folgende Eigenschaften:
    \begin{enumerate}
        \item Gleicher Betrag, entgegengesetzte Richtung
        \item Greifen an verschiedenen Körpern an
        \item Haben die gleiche physikalische Ursache
    \end{enumerate}
\end{concept}

\begin{example2}{Anwendungen des Wechselwirkungsgesetzes}\\
    \begin{itemize}
        \item Gravitationskraft: Die Erde zieht einen Menschen an, aber der Mensch zieht auch die Erde an (mit der gleichen Kraft)
        \item Bremsen eines Autos: Das Auto stößt die Straße nach vorn, die Straße stößt das Auto nach hinten
        \item Zwei verbundene Boote: Wenn eine Person in einem Boot das andere Boot an einer Leine zu sich zieht, bewegen sich beide Boote aufeinander zu
    \end{itemize}
\end{example2}

\subsection{Superpositionsprinzip}
\begin{formula}{Superpositionsprinzip}\\
    Das Superpositionsprinzip besagt, dass sich mehrere Kräfte $\vec{F}_1, \vec{F}_2, ..., \vec{F}_n$, die auf einen Punkt wirken, vektoriell zu einer resultierenden Kraft $\vec{F}$ addieren:
    \begin{equation}
        \vec{F} = \sum_{i=1}^{n} \vec{F}_i
    \end{equation}
    
    Für jede Komponente gilt entsprechend:
    \begin{equation}
        F_x = \sum_{i=1}^{n} F_{xi} \quad F_y = \sum_{i=1}^{n} F_{yi} \quad F_z = \sum_{i=1}^{n} F_{zi}
    \end{equation}
\end{formula}

\begin{KR}{Kräfte "freischneiden"}\\
    \paragraph{Schritte zum Freischneiden}
    \begin{enumerate}
        \item Zeichne den zu untersuchenden Körper isoliert
        \item Zeichne alle auf den Körper wirkenden Kräfte als Vektoren ein
        \item Bestimme die resultierende Kraft: $\vec{F}_{res} = \sum \vec{F}_i$
        \item Berechne die Beschleunigung: $\vec{a} = \frac{\vec{F}_{res}}{m}$
    \end{enumerate}
    
    \paragraph{Beispiel: Kiste auf schiefer Ebene}
    \begin{itemize}
        \item Gewichtskraft: $\vec{F}_G = m \cdot \vec{g}$ (nach unten)
        \item Normalkraft: $\vec{F}_N$ (senkrecht zur Ebene)
        \item Reibungskraft: $\vec{F}_R = \mu \cdot \vec{F}_N$ (parallel zur Ebene, entgegen der Bewegungsrichtung)
        \item Hangabtriebskraft: $F_{Hang} = m \cdot g \cdot \sin\alpha$ (parallel zur Ebene nach unten)
    \end{itemize}
\end{KR}

\begin{examplecode}{Kräfte in Unity implementieren}\\
    \begin{lstlisting}[language=csh, style=basesmol]
// Anwenden einer Kraft in Unity
void FixedUpdate() {
    // Gravitationskraft berechnen
    Vector3 gravityForce = new Vector3(0, -9.81f * mass, 0);
    
    // Normalkraft berechnet Unity automatisch bei Kollisionen
    
    // Hangabtriebskraft bei schiefer Ebene mit Winkel alpha
    float alpha = 30f * Mathf.Deg2Rad; // 30 Grad in Radian
    float hangForce = mass * 9.81f * Mathf.Sin(alpha);
    Vector3 hangForceVector = new Vector3(hangForce, 0, 0);
    
    // Reibungskraft
    float frictionCoeff = 0.3f;
    Vector3 frictionForce = -frictionCoeff * mass * 9.81f * Mathf.Cos(alpha) * rigidBody.velocity.normalized;
    
    // Alle Kraefte anwenden
    rigidBody.AddForce(gravityForce + hangForceVector + frictionForce);
}
    \end{lstlisting}
\end{examplecode}