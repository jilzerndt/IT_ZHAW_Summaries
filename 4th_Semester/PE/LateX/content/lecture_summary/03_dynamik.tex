\section{Dynamik}

\begin{definition}{Dynamik}\\
    Die Dynamik untersucht die Ursachen von Bewegungen - die wirkenden Kräfte. Grundlage sind die Newton'schen Gesetze.
\end{definition>

\begin{concept}{Newton'sche Axiome}\\
    Isaac Newton formulierte die drei grundlegenden Gesetze der Bewegung:
    \begin{enumerate}
        \item Trägheitsgesetz: Ein Körper bleibt im Zustand der Ruhe oder der gleichförmigen geradlinigen Bewegung, solange keine Kraft auf ihn wirkt.
        \item Bewegungsgesetz: Die Änderung der Bewegung ist proportional zur einwirkenden Kraft und erfolgt in Richtung der Kraft.
        \item Wechselwirkungsgesetz: Übt ein Körper auf einen anderen eine Kraft aus (actio), so wirkt eine gleich große, entgegengesetzte Kraft zurück (reactio).
    \end{enumerate}
    Ein viertes Prinzip ist das Superpositionsprinzip: Kräfte addieren sich vektoriell.
\end{concept}

\subsection{Newton'sche Gesetze}

\begin{formula}{Trägheitsgesetz}\\
    Das erste Newton'sche Gesetz:
    \begin{equation}
        \vec{F} = 0 \Rightarrow \vec{v} = \text{const.}
    \end{equation}
    
    Dies bedeutet auch: $\vec{F} = 0 \Rightarrow \vec{a} = 0$
    
    Das Gesetz gilt nur in Inertialsystemen (Bezugssysteme ohne Beschleunigung oder Rotation).
\end{formula}

\begin{definition}{Impuls}\\
    Der Impuls $\vec{p}$ eines Körpers ist das Produkt aus seiner Masse und seiner Geschwindigkeit:
    \begin{equation}
        \vec{p} = m \cdot \vec{v}
    \end{equation}
    
    Der Impuls ist eine vektorielle Größe mit der Einheit kg·m/s.
\end{definition}

\begin{formula}{Bewegungsgesetz}\\
    Das zweite Newton'sche Gesetz in seiner allgemeinen Form:
    \begin{equation}
        \vec{F} = \frac{d\vec{p}}{dt}
    \end{equation}
    
    Für Körper mit konstanter Masse:
    \begin{equation}
        \vec{F} = m \cdot \vec{a}
    \end{equation}
    
    In integraler Form:
    \begin{equation}
        \vec{p} = \vec{p}_0 + \int_0^t \vec{F}(t) \, dt
    \end{equation}
    wobei $\int \vec{F} \, dt$ als Kraftstoß bezeichnet wird.
\end{formula}

\begin{formula}{Wechselwirkungsgesetz}\\
    Das dritte Newton'sche Gesetz:
    \begin{equation}
        \vec{F}_{12} = -\vec{F}_{21}
    \end{equation}
    
    Kräftepaare haben folgende Eigenschaften:
    \begin{enumerate}
        \item Gleicher Betrag, entgegengesetzte Richtung
        \item Greifen an verschiedenen Körpern an
        \item Haben die gleiche physikalische Ursache
    \end{enumerate}
\end{formula}

\begin{formula}{Superpositionsprinzip}\\
    Mehrere Kräfte addieren sich vektoriell:
    \begin{equation}
        \vec{F}_{res} = \sum_{i=1}^{n} \vec{F}_i
    \end{equation}
    
    Für jede Komponente:
    \begin{equation}
        F_x = \sum_{i=1}^{n} F_{xi} \quad F_y = \sum_{i=1}^{n} F_{yi} \quad F_z = \sum_{i=1}^{n} F_{zi}
    \end{equation}
\end{formula}

\subsection{Wichtige Krafttypen}

\begin{formula}{Grundlegende Kräfte}\\
    \paragraph{Gravitationskraft (nahe Erdoberfläche):} $\vec{F}_g = m\vec{g}$ mit $g = 9.81 \, m/s^2$
    
    \paragraph{Federkraft (Hooke'sches Gesetz):} $\vec{F}_s = -k\Delta\vec{l}$
    
    \paragraph{Reibungskraft:} 
    \begin{itemize}
        \item Haftreibung: $f_s \leq \mu_s N$
        \item Gleitreibung: $f_k = \mu_k N$
    \end{itemize}
    
    \paragraph{Dämpfungskraft:} $\vec{F}_d = -b\vec{v}$ (geschwindigkeitsproportional)
\end{formula}

\subsection{Harmonischer Oszillator}

\begin{definition}{Harmonische Schwingung}\\
    Bei einer Federkraft entsteht eine harmonische Schwingung mit der Bewegungsgleichung:
    $$m\frac{d^2x}{dt^2} = -kx$$
    
    Lösung: $x(t) = A\cos(\omega t + \phi)$
    
    wobei $\omega = \sqrt{\frac{k}{m}}$ die Kreisfrequenz ist.
\end{definition}

\begin{formula}{Eigenschaften des harmonischen Oszillators}\\
    \paragraph{Kreisfrequenz:} $\omega = \sqrt{\frac{k}{m}}$
    
    \paragraph{Schwingungsdauer:} $T = \frac{2\pi}{\omega} = 2\pi\sqrt{\frac{m}{k}}$
    
    \paragraph{Frequenz:} $f = \frac{1}{T} = \frac{\omega}{2\pi}$
    
    \paragraph{Gesamtenergie:} $E = \frac{1}{2}kA^2$ (konstant)
\end{formula}

\subsection{Unity Implementation}

\begin{concept}{Unity Force Modes}\\
    Unity bietet verschiedene Modi für Kraftanwendung über \texttt{Rigidbody.AddForce()}:
    \begin{itemize}
        \item \texttt{Force}: Kontinuierliche Kraft mit Masse (Standard)
        \item \texttt{Acceleration}: Kontinuierliche Kraft ohne Masse
        \item \texttt{Impulse}: Impulsive Kraft mit Masse
        \item \texttt{VelocityChange}: Impulsive Kraft ohne Masse
    \end{itemize}
\end{concept}

\begin{code}{Kräfte in Unity anwenden}\\
\begin{lstlisting}[language=C, style=basesmol]
public class ForceController : MonoBehaviour 
{
    private Rigidbody rb;
    public float springConstant = 100f;
    public float damping = 10f;
    
    void Start() 
    {
        rb = GetComponent<Rigidbody>();
    }
    
    void FixedUpdate() 
    {
        // Gravitation (automatisch von Unity angewandt)
        // Vector3 gravity = Physics.gravity * rb.mass;
        
        // Federkraft zum Ursprung
        Vector3 springForce = -springConstant * transform.position;
        
        // Dämpfungskraft
        Vector3 dampingForce = -damping * rb.velocity;
        
        // Gesamtkraft anwenden
        rb.AddForce(springForce + dampingForce);
    }
}
\end{lstlisting}
\end{code}

\begin{code}{Harmonischer Oszillator in Unity}\\
\begin{lstlisting}[language=C, style=basesmol]
public class HarmonicOscillator : MonoBehaviour 
{
    public float springConstant = 50f;
    public float amplitude = 2f;
    public float phase = 0f;
    
    private float mass;
    private float omega;
    private Vector3 equilibrium;
    private float startTime;
    
    void Start() 
    {
        mass = GetComponent<Rigidbody>().mass;
        omega = Mathf.Sqrt(springConstant / mass);
        equilibrium = transform.position;
        startTime = Time.time;
    }
    
    void Update() 
    {
        float t = Time.time - startTime;
        float x = amplitude * Mathf.Cos(omega * t + phase);
        
        transform.position = equilibrium + Vector3.right * x;
    }
}
\end{lstlisting}
\end{code}

\subsection{Drehmoment}

\begin{definition}{Drehmoment}\\
    Rotatorisches Äquivalent zur Kraft:
    $$\vec{\tau} = \vec{r} \times \vec{F}$$
    
    Für Rotation um feste Achse: $\tau = rF\sin\theta$
\end{definition}

\begin{code}{Drehmoment in Unity}\\
\begin{lstlisting}[language=C, style=basesmol]
public class TorqueExample : MonoBehaviour 
{
    private Rigidbody rb;
    public float torqueStrength = 100f;
    
    void Start() 
    {
        rb = GetComponent<Rigidbody>();
    }
    
    void Update() 
    {
        if (Input.GetKey(KeyCode.Q)) 
        {
            // Drehmoment um Y-Achse anwenden
            rb.AddTorque(Vector3.up * torqueStrength);
        }
        
        // Rotationsfeder zur aufrechten Position
        Vector3 restoreTorque = -torqueStrength * transform.eulerAngles.x 
                               * Vector3.right;
        rb.AddTorque(restoreTorque);
    }
}
\end{lstlisting>
\end{code>

\begin{KR}{Problemlösungsstrategie für Kräfte}\\
    \paragraph{Schritt 1: System identifizieren}
    \begin{itemize}
        \item Untersuchungsobjekt(e) definieren
        \item Koordinatensystem wählen
        \item Zeitintervall bestimmen
    \end{itemize}
    
    \paragraph{Schritt 2: Freikörperdiagramm zeichnen}
    \begin{itemize}
        \item Alle auf das Objekt wirkenden Kräfte zeigen
        \item Kraftvektoren mit Symbolen beschriften
        \item Keine Kräfte einbeziehen, die das Objekt auf andere ausübt
    \end{itemize}
    
    \paragraph{Schritt 3: Newton'sches Bewegungsgesetz anwenden}
    \begin{itemize}
        \item $\sum \vec{F} = m\vec{a}$ in Komponentenform schreiben
        \item $\sum F_x = ma_x$, $\sum F_y = ma_y$, $\sum F_z = ma_z$
        \item Unbekannte Größen lösen
    \end{itemize}
    
    \paragraph{Schritt 4: In Unity implementieren}
    \begin{itemize}
        \item \texttt{Rigidbody.AddForce()} für jede Kraft verwenden
        \item Passenden Force-Modus berücksichtigen
        \item \texttt{FixedUpdate()} für Physikberechnungen nutzen
    \end{itemize}
\end{KR}

\begin{example2}{Klotz auf schiefer Ebene}\\
    Ein Klotz der Masse $m = 2kg$ gleitet eine reibungsfreie schiefe Ebene mit Winkel $\theta = 30°$ hinunter. Berechne die Beschleunigung und implementiere in Unity.
    \tcblower
    \textbf{Freikörperdiagramm:} Gewichtskraft $mg$ nach unten, Normalkraft $N$ senkrecht zur Oberfläche.
    
    \textbf{Komponentenanalyse:}
    \begin{itemize}
        \item Entlang der Ebene: $mg\sin\theta = ma$
        \item Senkrecht dazu: $N - mg\cos\theta = 0$
    \end{itemize}
    
    \textbf{Lösung:} $a = g\sin\theta = 9.81 \times \sin(30°) = 4.905 \, m/s^2$
    
    \textbf{Unity Implementation:}
\begin{lstlisting}[language=C, style=basesmol]
Vector3 inclineForce = mass * Physics.gravity.magnitude * 
                      Mathf.Sin(30f * Mathf.Deg2Rad) * 
                      inclineDirection;
rb.AddForce(inclineForce);
\end{lstlisting}
\end{example2}