\section{Dynamik der Translation}

\subsection{Grundlagen der Dynamik}
\begin{definition}{Dynamik}\\
    Die Dynamik beschäftigt sich mit den Ursachen von Bewegungen, also den Kräften, die auf einen Körper wirken. Sie baut auf der Kinematik auf und erweitert diese um die Betrachtung der wirkenden Kräfte.
\end{definition}

\begin{concept}{Newton'sche Axiome}\\
    Isaac Newton formulierte die drei grundlegenden Gesetze der Bewegung:
    \begin{enumerate}
        \item Trägheitsgesetz: Ein Körper bleibt im Zustand der Ruhe oder der gleichförmigen geradlinigen Bewegung, solange keine Kraft auf ihn wirkt.
        \item Bewegungsgesetz: Die Änderung der Bewegung ist proportional zur einwirkenden Kraft und erfolgt in Richtung der Kraft.
        \item Wechselwirkungsgesetz: Übt ein Körper auf einen anderen eine Kraft aus (actio), so wirkt eine gleich große, entgegengesetzte Kraft zurück (reactio).
    \end{enumerate}
    Ein viertes Prinzip ist das Superpositionsprinzip: Kräfte addieren sich vektoriell.
\end{concept}

\subsection{Erstes Newton'sches Gesetz (Trägheitsgesetz)}
\begin{formula}{Trägheitsgesetz}\\
    Das erste Newton'sche Gesetz lässt sich mathematisch ausdrücken als:
    \begin{equation}
        \vec{F} = 0 \Rightarrow \vec{v} = \text{const.}
    \end{equation}
    
    Dies bedeutet auch:
    \begin{equation}
        \vec{F} = 0 \Rightarrow \vec{a} = \frac{d\vec{v}}{dt} = \frac{d}{dt}(\text{const.}) = 0
    \end{equation}
    
    Das Gesetz gilt nur in sogenannten Inertialsystemen: Bezugssystemen ohne Beschleunigung oder Rotation.
\end{formula}

\begin{example2}{Trägheitsgesetz auf der Luftkissenbahn}\\
    Ein Gleiter auf einer Luftkissenbahn bewegt sich ohne Reibung mit konstanter Geschwindigkeit weiter, wenn keine Kraft auf ihn wirkt. Dies ist eine direkte Demonstration des Trägheitsgesetzes.
    
    Das gleiche Prinzip gilt für einen Stein, der von der Spitze eines fahrenden Schiffes fällt - er behält seine horizontale Geschwindigkeit bei und landet am Fuß des Mastes, nicht dahinter oder davor.
\end{example2}

\subsection{Zweites Newton'sches Gesetz (Bewegungsgesetz)}
\begin{definition}{Impuls}\\
    Der Impuls $\vec{p}$ eines Körpers ist das Produkt aus seiner Masse und seiner Geschwindigkeit:
    \begin{equation}
        \vec{p} = m \cdot \vec{v}
    \end{equation}
    
    Der Impuls ist eine vektorielle Größe mit der Einheit kg·m/s.
\end{definition}

\begin{formula}{Bewegungsgesetz}\\
    Das zweite Newton'sche Gesetz in seiner allgemeinen Form:
    \begin{equation}
        \vec{F} = \frac{d\vec{p}}{dt}
    \end{equation}
    
    Für Körper mit konstanter Masse vereinfacht sich dies zu:
    \begin{equation}
        \vec{F} = m \cdot \vec{a}
    \end{equation}
    
    In integraler Form:
    \begin{equation}
        \vec{p} = \vec{p}_0 + \int_0^t \vec{F}(t) \, dt
    \end{equation}
    wobei $\int \vec{F} \, dt$ als Kraftstoß bezeichnet wird.
\end{formula}

\begin{KR}{Berechnung von Kräften und Beschleunigungen}\\
    \paragraph{Berechnung der Beschleunigung aus Kräften}
    \begin{itemize}
        \item Alle auf den Körper wirkenden Kräfte identifizieren
        \item Kräfte vektoriell addieren zur resultierenden Kraft $\vec{F}_{res}$
        \item Beschleunigung berechnen: $\vec{a} = \frac{\vec{F}_{res}}{m}$
    \end{itemize}
    
    \paragraph{Beispiel: Fahrstuhl}
    \begin{itemize}
        \item Fahrstuhl in Ruhe: $F_{\text{Seil}} - F_G = 0 \Rightarrow F_{\text{Seil}} = m \cdot g$
        \item Fahrstuhl beschleunigt nach oben: $F_{\text{Seil}} - F_G = m \cdot a \Rightarrow F_{\text{Seil}} = m \cdot (g + a)$
        \item Fahrstuhl beschleunigt nach unten: $F_{\text{Seil}} - F_G = -m \cdot a \Rightarrow F_{\text{Seil}} = m \cdot (g - a)$
    \end{itemize}
\end{KR}

\subsection{Drittes Newton'sches Gesetz (Wechselwirkungsgesetz)}
\begin{formula}{Wechselwirkungsgesetz}\\
    Das dritte Newton'sche Gesetz lautet mathematisch:
    \begin{equation}
        \vec{F}_{12} = -\vec{F}_{21}
    \end{equation}
    wobei $\vec{F}_{12}$ die Kraft bezeichnet, die Körper 1 auf Körper 2 ausübt, und $\vec{F}_{21}$ die Kraft, die Körper 2 auf Körper 1 ausübt.
\end{formula}

\begin{concept}{Merkmale von Kräftepaaren}\\
    Kräftepaare des Wechselwirkungsgesetzes haben folgende Eigenschaften:
    \begin{enumerate}
        \item Gleicher Betrag, entgegengesetzte Richtung
        \item Greifen an verschiedenen Körpern an
        \item Haben die gleiche physikalische Ursache
    \end{enumerate}
\end{concept}

\begin{example2}{Anwendungen des Wechselwirkungsgesetzes}\\
    \begin{itemize}
        \item Gravitationskraft: Die Erde zieht einen Menschen an, aber der Mensch zieht auch die Erde an (mit der gleichen Kraft)
        \item Bremsen eines Autos: Das Auto stößt die Straße nach vorn, die Straße stößt das Auto nach hinten
        \item Zwei verbundene Boote: Wenn eine Person in einem Boot das andere Boot an einer Leine zu sich zieht, bewegen sich beide Boote aufeinander zu
    \end{itemize}
\end{example2}

\subsection{Superpositionsprinzip}
\begin{formula}{Superpositionsprinzip}\\
    Das Superpositionsprinzip besagt, dass sich mehrere Kräfte $\vec{F}_1, \vec{F}_2, ..., \vec{F}_n$, die auf einen Punkt wirken, vektoriell zu einer resultierenden Kraft $\vec{F}$ addieren:
    \begin{equation}
        \vec{F} = \sum_{i=1}^{n} \vec{F}_i
    \end{equation}
    
    Für jede Komponente gilt entsprechend:
    \begin{equation}
        F_x = \sum_{i=1}^{n} F_{xi} \quad F_y = \sum_{i=1}^{n} F_{yi} \quad F_z = \sum_{i=1}^{n} F_{zi}
    \end{equation}
\end{formula}

\begin{KR}{Kräfte "freischneiden"}\\
    \paragraph{Schritte zum Freischneiden}
    \begin{enumerate}
        \item Zeichne den zu untersuchenden Körper isoliert
        \item Zeichne alle auf den Körper wirkenden Kräfte als Vektoren ein
        \item Bestimme die resultierende Kraft: $\vec{F}_{res} = \sum \vec{F}_i$
        \item Berechne die Beschleunigung: $\vec{a} = \frac{\vec{F}_{res}}{m}$
    \end{enumerate}
    
    \paragraph{Beispiel: Kiste auf schiefer Ebene}
    \begin{itemize}
        \item Gewichtskraft: $\vec{F}_G = m \cdot \vec{g}$ (nach unten)
        \item Normalkraft: $\vec{F}_N$ (senkrecht zur Ebene)
        \item Reibungskraft: $\vec{F}_R = \mu \cdot \vec{F}_N$ (parallel zur Ebene, entgegen der Bewegungsrichtung)
        \item Hangabtriebskraft: $F_{Hang} = m \cdot g \cdot \sin\alpha$ (parallel zur Ebene nach unten)
    \end{itemize}
\end{KR}

\begin{examplecode}{Kräfte in Unity implementieren}\\
    \begin{lstlisting}[language=csh, style=basesmol]
// Anwenden einer Kraft in Unity
void FixedUpdate() {
    // Gravitationskraft berechnen
    Vector3 gravityForce = new Vector3(0, -9.81f * mass, 0);
    
    // Normalkraft berechnet Unity automatisch bei Kollisionen
    
    // Hangabtriebskraft bei schiefer Ebene mit Winkel alpha
    float alpha = 30f * Mathf.Deg2Rad; // 30 Grad in Radian
    float hangForce = mass * 9.81f * Mathf.Sin(alpha);
    Vector3 hangForceVector = new Vector3(hangForce, 0, 0);
    
    // Reibungskraft
    float frictionCoeff = 0.3f;
    Vector3 frictionForce = -frictionCoeff * mass * 9.81f * Mathf.Cos(alpha) * rigidBody.velocity.normalized;
    
    // Alle Kraefte anwenden
    rigidBody.AddForce(gravityForce + hangForceVector + frictionForce);
}
    \end{lstlisting}
\end{examplecode}

\section{Dynamics and Forces}

\subsection{Newton's Laws of Motion}

\begin{concept}{Newton's First Law (Law of Inertia)}\\
    An object at rest stays at rest, and an object in motion stays in motion at constant velocity, unless acted upon by a net external force.
    $$\sum \vec{F} = 0 \Rightarrow \vec{v} = \text{constant}$$
\end{concept}

\begin{concept}{Newton's Second Law}\\
    The acceleration of an object is directly proportional to the net force acting on it and inversely proportional to its mass.
    $$\sum \vec{F} = m\vec{a}$$
\end{concept}

\begin{concept}{Newton's Third Law (Action-Reaction)}\\
    For every action, there is an equal and opposite reaction.
    $$\vec{F}_{AB} = -\vec{F}_{BA}$$
\end{concept}

\subsection{Common Forces}

\begin{definition}{Gravitational Force}\\
    Force between two masses:
    $$\vec{F}_g = -\frac{Gm_1m_2}{r^2}\hat{r}$$
    
    Near Earth's surface:
    $$\vec{F}_g = m\vec{g} = -mg\hat{j}$$
    where $g = 9.81 \, m/s^2$
\end{definition}

\begin{definition}{Spring Force (Hooke's Law)}\\
    Force exerted by a spring:
    $$\vec{F}_s = -k\Delta\vec{l}$$
    where $k$ is the spring constant and $\Delta\vec{l}$ is displacement from equilibrium.
\end{definition}

\begin{definition}{Friction Force}\\
    \textbf{Static friction:} $f_s \leq \mu_s N$
    
    \textbf{Kinetic friction:} $f_k = \mu_k N$
    
    where $\mu_s, \mu_k$ are friction coefficients and $N$ is normal force.
\end{definition}

\begin{definition}{Damping Force}\\
    Force opposing motion, proportional to velocity:
    $$\vec{F}_d = -k_{damper} \vec{v}$$
    Used in Unity for realistic motion with air resistance or viscous damping.
\end{definition}

\subsection{Unity Force Implementation}

\begin{code}{Applying Forces in Unity}\\
\begin{lstlisting}[language=C, style=basesmol]
public class ForceController : MonoBehaviour 
{
    private Rigidbody rb;
    public float springConstant = 100f;
    public float damping = 10f;
    
    void Start() 
    {
        rb = GetComponent<Rigidbody>();
    }
    
    void FixedUpdate() 
    {
        // Gravity (automatically applied by Unity)
        // Vector3 gravity = Physics.gravity * rb.mass;
        
        // Spring force to origin
        Vector3 springForce = -springConstant * transform.position;
        
        // Damping force
        Vector3 dampingForce = -damping * rb.velocity;
        
        // Apply total force
        rb.AddForce(springForce + dampingForce);
    }
}
\end{lstlisting}
\end{code}

\begin{concept}{Unity Force Modes}\\
    Unity provides different ways to apply forces through \texttt{Rigidbody.AddForce()}:
    \begin{itemize}
        \item \texttt{Force}: Continuous force using mass (default)
        \item \texttt{Acceleration}: Continuous force ignoring mass
        \item \texttt{Impulse}: Instantaneous force using mass
        \item \texttt{VelocityChange}: Instantaneous force ignoring mass
    \end{itemize}
\end{concept}

\subsection{Harmonic Oscillator}

\begin{definition}{Simple Harmonic Motion}\\
    Motion under spring force with equation:
    $$m\frac{d^2x}{dt^2} = -kx$$
    
    Solution: $x(t) = A\cos(\omega t + \phi)$
    
    where $\omega = \sqrt{\frac{k}{m}}$ is the angular frequency.
\end{definition}

\begin{formula}{Harmonic Oscillator Properties}\\
    \paragraph{Angular frequency:} $\omega = \sqrt{\frac{k}{m}}$
    
    \paragraph{Period:} $T = \frac{2\pi}{\omega} = 2\pi\sqrt{\frac{m}{k}}$
    
    \paragraph{Frequency:} $f = \frac{1}{T} = \frac{\omega}{2\pi}$
    
    \paragraph{Total energy:} $E = \frac{1}{2}kA^2$ (constant)
\end{formula}

\begin{code}{Harmonic Oscillator in Unity}\\
\begin{lstlisting}[language=C, style=basesmol]
public class HarmonicOscillator : MonoBehaviour 
{
    public float springConstant = 50f;
    public float amplitude = 2f;
    public float phase = 0f;
    
    private float mass;
    private float omega;
    private Vector3 equilibrium;
    private float startTime;
    
    void Start() 
    {
        mass = GetComponent<Rigidbody>().mass;
        omega = Mathf.Sqrt(springConstant / mass);
        equilibrium = transform.position;
        startTime = Time.time;
    }
    
    void Update() 
    {
        float t = Time.time - startTime;
        float x = amplitude * Mathf.Cos(omega * t + phase);
        
        transform.position = equilibrium + Vector3.right * x;
    }
}
\end{lstlisting}
\end{code}

\subsection{Rotational Forces}

\begin{definition}{Torque (Moment)}\\
    Rotational equivalent of force:
    $$\vec{\tau} = \vec{r} \times \vec{F}$$
    
    For rotation about fixed axis:
    $$\tau = rF\sin\theta$$
\end{definition}

\begin{definition}{Angular Spring (Torsional Spring)}\\
    Analogous to linear spring for rotational motion:
    $$\tau = -D\Delta\phi$$
    
    where $D$ is the angular spring constant and $\Delta\phi$ is angular displacement.
\end{definition}

\begin{code}{Applying Torque in Unity}\\
\begin{lstlisting}[language=C, style=basesmol]
public class TorqueExample : MonoBehaviour 
{
    private Rigidbody rb;
    public float torqueStrength = 100f;
    
    void Start() 
    {
        rb = GetComponent<Rigidbody>();
    }
    
    void Update() 
    {
        if (Input.GetKey(KeyCode.Q)) 
        {
            // Apply torque around Y-axis
            rb.AddTorque(Vector3.up * torqueStrength);
        }
        
        // Rotational spring to upright position
        Vector3 restoreToruqe = -torqueStrength * transform.eulerAngles.x 
                               * Vector3.right;
        rb.AddTorque(restoreToruqe);
    }
}
\end{lstlisting}
\end{code}

\subsection{Problem Solving Strategy}

\begin{KR}{Force Analysis Method}\\
    \paragraph{Step 1: Identify the system}
    \begin{itemize}
        \item Define the object(s) of interest
        \item Choose coordinate system
        \item Identify time interval of interest
    \end{itemize}
    
    \paragraph{Step 2: Draw free-body diagram}
    \begin{itemize}
        \item Show all forces acting on the object
        \item Label force vectors with symbols
        \item Do not include forces the object exerts on other objects
    \end{itemize}
    
    \paragraph{Step 3: Apply Newton's Second Law}
    \begin{itemize}
        \item Write $\sum \vec{F} = m\vec{a}$ in component form
        \item $\sum F_x = ma_x$, $\sum F_y = ma_y$, $\sum F_z = ma_z$
        \item Solve for unknown quantities
    \end{itemize}
    
    \paragraph{Step 4: Implement in Unity}
    \begin{itemize}
        \item Use \texttt{Rigidbody.AddForce()} for each force
        \item Consider appropriate force mode
        \item Use \texttt{FixedUpdate()} for physics calculations
    \end{itemize}
\end{KR}

\begin{example2}{Block on Inclined Plane}\\
    A block of mass $m = 2kg$ slides down a frictionless incline of angle $\theta = 30°$. Find acceleration and implement in Unity.
    \tcblower
    \textbf{Free-body diagram:} Weight $mg$ downward, normal force $N$ perpendicular to surface.
    
    \textbf{Component analysis:}
    \begin{itemize}
        \item Along incline: $mg\sin\theta = ma$
        \item Perpendicular: $N - mg\cos\theta = 0$
    \end{itemize}
    
    \textbf{Solution:} $a = g\sin\theta = 9.81 \times \sin(30°) = 4.905 \, m/s^2$
    
    \textbf{Unity implementation:}
\begin{lstlisting}[language=C, style=basesmol]
Vector3 inclineForce = mass * Physics.gravity.magnitude * 
                      Mathf.Sin(30f * Mathf.Deg2Rad) * 
                      inclineDirection;
rb.AddForce(inclineForce);
\end{lstlisting}
\end{example2}