\section{Kräfte}

\subsection{Arten von Kräften}
\begin{concept}{Grundlegende Wechselwirkungen}\\
    In der Physik gibt es vier fundamentale Wechselwirkungen:
    \begin{enumerate}
        \item Gravitation (Anziehung zwischen Massen)
        \item Elektromagnetische Kraft (Kräfte zwischen elektrischen Ladungen)
        \item Starke Kernkraft (hält den Atomkern zusammen)
        \item Schwache Kernkraft (verantwortlich für radioaktiven Beta-Zerfall)
    \end{enumerate}
    Für die makroskopische Mechanik sind hauptsächlich die Gravitation und die elektromagnetischen Kräfte relevant. Die starke und schwache Kernkraft wirken nur auf atomarer Ebene.
\end{concept}

\begin{definition}{Kraft}\\
    Eine Kraft ist ein Einfluss, der den Bewegungszustand eines Körpers ändert. Im SI-System wird sie in Newton (N) gemessen:
    \begin{equation}
        1 \, \text{N} = 1 \, \frac{\text{kg} \cdot \text{m}}{\text{s}^2}
    \end{equation}
    
    Kräfte ändern den Impuls eines Körpers gemäß der Formel:
    \begin{equation}
        \vec{F} = \frac{d\vec{p}}{dt}
    \end{equation}
\end{definition}

\subsection{Gravitationskraft}
\begin{formula}{Gravitationsgesetz}\\
    Das Newton'sche Gravitationsgesetz beschreibt die Anziehungskraft zwischen zwei Massen $m_1$ und $m_2$, die im Abstand $R$ voneinander entfernt sind:
    \begin{equation}
        F_G = G \cdot \frac{m_1 \cdot m_2}{R^2}
    \end{equation}
    
    Dabei ist $G$ die Gravitationskonstante mit dem Wert:
    \begin{equation}
        G = 6.67 \cdot 10^{-11} \, \frac{\text{m}^3}{\text{kg} \cdot \text{s}^2}
    \end{equation}
    
    Die Kraft wirkt entlang der Verbindungslinie der beiden Massen und ist eine Anziehungskraft.
\end{formula}

\begin{example2}{Erdbeschleunigung}\\
    Die Gewichtskraft $F_G$ eines Körpers der Masse $m$ auf der Erdoberfläche ist:
    \begin{equation}
        F_G = m \cdot g
    \end{equation}
    
    Die Erdbeschleunigung $g$ lässt sich aus dem Gravitationsgesetz herleiten:
    \begin{equation}
        g = G \cdot \frac{M_{\text{Erde}}}{R_{\text{Erde}}^2} \approx 9.81 \, \frac{\text{m}}{\text{s}^2}
    \end{equation}
    
    Mit zunehmender Höhe $h$ nimmt die Fallbeschleunigung ab:
    \begin{equation}
        g(h) = G \cdot \frac{M_{\text{Erde}}}{(R_{\text{Erde}} + h)^2}
    \end{equation}
\end{example2}

\subsection{Federkraft}
\begin{definition}{Federkraft (Hooke'sches Gesetz)}\\
    Für eine lineare Feder gilt das Hooke'sche Gesetz:
    \begin{equation}
        \vec{F} = -k \cdot \vec{x}
    \end{equation}
    
    Dabei ist:
    \begin{itemize}
        \item $k$: die Federkonstante in N/m
        \item $\vec{x}$: die Auslenkung der Feder aus ihrer Ruhelage
        \item Das negative Vorzeichen bedeutet, dass die Kraft der Auslenkung entgegengerichtet ist
    \end{itemize}
\end{definition}

\begin{formula}{Spannenergie einer Feder}\\
    Die in einer gespannten Feder gespeicherte potentielle Energie beträgt:
    \begin{equation}
        E_{\text{spann}} = \frac{1}{2} \cdot k \cdot x^2
    \end{equation}
    
    Diese Energie kann in kinetische Energie umgewandelt werden, wenn die Feder entspannt wird.
\end{formula}

\begin{concept}{Harmonischer Oszillator}\\
    Ein harmonischer Oszillator ist ein System, bei dem die rücktreibende Kraft proportional zur Auslenkung ist. Beispiele sind eine Masse an einer Feder oder ein Pendel bei kleinen Auslenkungen.
    
    Die Bewegungsgleichung eines harmonischen Oszillators lautet:
    \begin{equation}
        \frac{d^2 x}{dt^2} = -\frac{k}{m} \cdot x
    \end{equation}
    
    Die Lösung ist eine harmonische Schwingung mit der Kreisfrequenz $\omega$:
    \begin{equation}
        x(t) = x_0 \cdot \cos(\omega t)
    \end{equation}
    
    wobei $\omega = \sqrt{\frac{k}{m}}$ und die Schwingungsdauer $T = \frac{2\pi}{\omega}$ beträgt.
\end{concept}

\begin{examplecode}{Simulation eines harmonischen Oszillators in Unity}\\
    \begin{lstlisting}[language=csh, style=basesmol]
// Harmonischer Oszillator
void FixedUpdate() {
    // Federkraft berechnen (F = -k*x)
    float springForce = -springConstant * rigidbody.position.x;
    
    // Kraft anwenden
    rigidbody.AddForce(new Vector3(springForce, 0f, 0f));
    
    // Zeit aktualisieren
    currentTimeStep += Time.deltaTime;
}
    \end{lstlisting}
\end{examplecode}

\subsection{Reibungskräfte}
\begin{definition}{Arten der Reibung}\\
    Man unterscheidet zwischen:
    \begin{itemize}
        \item Äußerer Reibung (Kontaktflächen von sich berührenden Festkörpern):
        \begin{itemize}
            \item Haftreibung
            \item Gleitreibung
            \item Rollreibung, Wälzreibung, Bohrreibung, Seilreibung
        \end{itemize}
        \item Innerer Reibung (zwischen benachbarten Teilchen bei Verformungen innerhalb von Festkörpern, Flüssigkeiten und Gasen)
    \end{itemize}
\end{definition}

\begin{formula}{Trockene Reibung}\\
    Für trockene Reibung (Coulomb-Reibung) gilt:
    \begin{equation}
        \vec{F}_R = \mu \cdot \vec{F}_N
    \end{equation}
    
    Dabei ist:
    \begin{itemize}
        \item $\mu$: Reibungskoeffizient (dimensionslos)
        \item $\vec{F}_N$: Normalkraft
        \item Die Richtung der Reibungskraft ist stets entgegengesetzt zur Bewegungsrichtung bzw. zur Zugkraft
    \end{itemize}
    
    Man unterscheidet zwischen:
    \begin{itemize}
        \item Haftreibung: $\vec{F}_{\text{Haft}} \leq \mu_{\text{Haft}} \cdot \vec{F}_N$ (gleicht äußere Kraft bis zu einem Maximalwert aus)
        \item Gleitreibung: $\vec{F}_{\text{Gleit}} = \mu_{\text{Gleit}} \cdot \vec{F}_N$ (konstant bei gegebener Normalkraft)
        \item Dabei gilt in der Regel: $\mu_{\text{Haft}} > \mu_{\text{Gleit}}$
    \end{itemize}
\end{formula}

\begin{formula}{Viskose Reibung}\\
    Für die Reibung in Flüssigkeiten und Gasen gelten je nach Strömungsregime:
    
    \begin{itemize}
        \item Laminare Strömung (Stokes'sche Reibung für eine Kugel):
        \begin{equation}
            \vec{F}_R = -6 \cdot \pi \cdot \eta \cdot r \cdot v \cdot \vec{e}_v
        \end{equation}
        wobei $\eta$ die Viskosität des Mediums ist
        
        \item Turbulente Strömung:
        \begin{equation}
            \vec{F}_R = -\frac{1}{2} \cdot \rho \cdot A \cdot c_w \cdot \vec{v}^2 \cdot \vec{e}_v
        \end{equation}
        wobei $\rho$ die Dichte des Mediums, $A$ die Stirnfläche und $c_w$ der Widerstandsbeiwert ist
    \end{itemize}
\end{formula}

\begin{KR}{Reibung in Unity implementieren}\\
    \paragraph{Trockene Reibung}
    \begin{lstlisting}[language=csh, style=basesmol]
// Trockene Reibung implementieren
void FixedUpdate() {
    // Normale Kraft berechnen (Gewichtskraft bei horizontaler Flaeche)
    float normalForce = rigidbody.mass * 9.81f;
    
    // Reibungskraft berechnen
    float frictionForce = frictionCoefficient * normalForce;
    
    // Richtung der Geschwindigkeit bestimmen
    Vector3 velocityDirection = rigidbody.velocity.normalized;
    
    // Reibungskraft nur anwenden, wenn Objekt sich bewegt
    if (rigidbody.velocity.magnitude > 0.01f) {
        // Reibungskraft entgegen der Bewegungsrichtung
        Vector3 frictionVector = -velocityDirection * frictionForce;
        rigidbody.AddForce(frictionVector);
    }
}
    \end{lstlisting}
    
    \paragraph{Luftwiderstand (turbulente Reibung)}
    \begin{lstlisting}[language=csh, style=basesmol]
// Luftwiderstand implementieren
void FixedUpdate() {
    // Parameter fuer Luftwiderstand
    float airDensity = 1.2f;      // kg/m^3
    float dragCoefficient = 0.5f; // dimensionslos
    float frontalArea = 1.0f;     // m^2
    
    // Aktuelle Geschwindigkeit und Betrag
    Vector3 velocity = rigidbody.velocity;
    float velocityMagnitude = velocity.magnitude;
    
    // Luftwiderstandskraft berechnen
    float dragForceMagnitude = 0.5f * airDensity * frontalArea * 
                             dragCoefficient * velocityMagnitude * velocityMagnitude;
    
    // Richtung entgegen der Bewegung
    Vector3 dragForce = -velocity.normalized * dragForceMagnitude;
    
    // Kraft anwenden
    rigidbody.AddForce(dragForce);
}
    \end{lstlisting}
\end{KR}

\subsection{Trägheitskräfte}
\begin{concept}{Beschleunigte Bezugssysteme}\\
    In beschleunigten Bezugssystemen treten Trägheitskräfte (auch Scheinkräfte genannt) auf, um die Newton'schen Gesetze weiterhin anwenden zu können.
    
    Beispiele:
    \begin{itemize}
        \item Translatorische Trägheitskraft bei linearer Beschleunigung:
        \begin{equation}
            \vec{F}_{\text{Trägheit}} = -m \cdot \vec{a}_{\text{System}}
        \end{equation}
        
        \item Zentrifugalkraft bei Rotation:
        \begin{equation}
            \vec{F}_{\text{Zentrifugal}} = m \cdot \omega^2 \cdot r \cdot \vec{e}_r
        \end{equation}
        
        \item Corioliskraft bei Bewegung in einem rotierenden System:
        \begin{equation}
            \vec{F}_{\text{Coriolis}} = 2 \cdot m \cdot \vec{v} \times \vec{\omega}
        \end{equation}
    \end{itemize}
\end{concept}

\begin{example2}{Trägheitskräfte im Alltag}\\
    \begin{itemize}
        \item Im bremsenden Zug fühlt man sich nach vorne gedrückt
        \item In einer Kurve wird man nach außen gedrückt (Zentrifugalkraft)
        \item Die Corioliskraft beeinflusst Winde und Meeresströmungen auf der Erde (auf der Nordhalbkugel werden sie nach rechts, auf der Südhalbkugel nach links abgelenkt)
    \end{itemize}
\end{example2}

\begin{remark}
    Trägheitskräfte sind keine "echten" Kräfte im Sinne von Wechselwirkungen zwischen Körpern, sondern entstehen durch die Wahl des Bezugssystems. In einem Inertialsystem existieren sie nicht.
\end{remark}