\section{Arbeit und Energie}

\subsection{Arbeit}

\begin{definition}{Arbeit}
    Die physikalische Arbeit $W$ ist das Skalarprodukt aus Kraft und Weg:
    \begin{equation}
        W = \int_{\vec{r}_0}^{\vec{r}_1} \vec{F} \cdot d\vec{r}
    \end{equation}
    
    Einheit: Joule (J) = $1 \, \text{N} \cdot \text{m} = 1 \, \frac{\text{kg} \cdot \text{m}^2}{\text{s}^2}$
    
    Für konstante Kraft: $W = F \cdot \Delta x \cdot \cos(\alpha)$
\end{definition}

\begin{concept}{Graphische Darstellung}
    Im Kraft-Weg-Diagramm entspricht die Arbeit der Fläche unter der Kurve:
    \begin{equation}
        W = \int_{x_0}^{x_1} F_x(x) \, dx
    \end{equation}
    
    \textbf{Beispiele:}
    \begin{itemize}
        \item Konstante Kraft: $W = F \cdot \Delta x$ (Rechteck)
        \item Linear ansteigende Kraft: $W = \frac{1}{2} F_{max} \cdot \Delta x$ (Dreieck)
    \end{itemize}
\end{concept}

\begin{formula}{Arten physikalischer Arbeit}
    \paragraph{Hubarbeit:} $W = m \cdot g \cdot h$
    
    \paragraph{Beschleunigungsarbeit:} $W = \frac{1}{2} m (v_1^2 - v_0^2)$
    
    \paragraph{Deformationsarbeit (Feder):} $W = \frac{1}{2} k (x_1^2 - x_0^2)$
    
    \paragraph{Reibungsarbeit:} $W = \mu \cdot F_N \cdot \Delta x$
\end{formula}

\subsection{Energie}

\begin{definition}{Energie}
    Energie ist die Fähigkeit eines Systems, Arbeit zu verrichten.
    \begin{equation}
        W = \Delta E = E_{nachher} - E_{vorher}
    \end{equation}
    
    \textbf{Unterschied zu Arbeit:}
    \begin{itemize}
        \item Arbeit: Prozessgröße (beschreibt einen Vorgang)
        \item Energie: Zustandsgröße (charakterisiert einen Zustand)
    \end{itemize}
\end{definition}

\begin{formula}{Energieformen in der Mechanik}
    \paragraph{Kinetische Energie:} $E_{kin} = \frac{1}{2} m v^2$
    
    \paragraph{Potentielle Energie im Schwerefeld:} $E_{pot} = m g h$
    
    \paragraph{Spannenergie einer Feder:} $E_{spann} = \frac{1}{2} k x^2$
    
    \paragraph{Rotationsenergie:} $E_{rot} = \frac{1}{2} J \omega^2$
\end{formula}

\begin{concept}{Energieerhaltung}
    In einem abgeschlossenen System ohne nicht-konservative Kräfte bleibt die Gesamtenergie konstant:
    \begin{equation}
        E_{ges} = E_{kin} + E_{pot} + E_{spann} + E_{rot} = \text{const.}
    \end{equation}
    
    \textbf{Konservative vs. nicht-konservative Kräfte:}
    \begin{itemize}
        \item \textbf{Konservativ}: Arbeit ist wegunabhängig (Gravitation, Federkraft)
        \item \textbf{Nicht-konservativ}: Arbeit hängt vom Weg ab (Reibung, Luftwiderstand)
    \end{itemize}
\end{concept}

\begin{example2}{Energiewandlungen beim Pendel}
    Pendel der Masse $m$ und Länge $l$ zeigt Umwandlung zwischen potentieller und kinetischer Energie:
    
    \textbf{Am höchsten Punkt:}
    \begin{equation}
        E_{pot,max} = m g l (1 - \cos\alpha)
    \end{equation}
    
    \textbf{Am tiefsten Punkt:}
    \begin{equation}
        E_{kin,max} = \frac{1}{2} m v_{max}^2 = m g l (1 - \cos\alpha)
    \end{equation}
    
    Gesamtenergie bleibt konstant: $E_{ges} = E_{pot} + E_{kin}$
\end{example2}

\subsection{Leistung und Wirkungsgrad}

\begin{definition}{Leistung}
    Die Leistung $P$ ist die pro Zeiteinheit verrichtete Arbeit:
    \begin{equation}
        P = \frac{dW}{dt} = \vec{F} \cdot \vec{v}
    \end{equation}
    
    Einheit: Watt (W) = $1 \, \frac{\text{J}}{\text{s}} = 1 \, \frac{\text{kg} \cdot \text{m}^2}{\text{s}^3}$
    
    Für Rotationsbewegung: $P = \tau \omega$
\end{definition}

\begin{definition}{Wirkungsgrad}
    Verhältnis von Nutzenergie zu zugeführter Energie:
    \begin{equation}
        \eta = \frac{E_{Nutz}}{E_{zugeführt}} = \frac{P_{Nutz}}{P_{zugeführt}}
    \end{equation}
    
    Dimensionslose Zahl zwischen 0 und 1 (oder 0-100\%).
\end{definition}

\subsection{Impuls und Drehimpuls}

\begin{definition}{Linearer Impuls}
    \begin{equation}
        \vec{p} = m\vec{v}
    \end{equation}
    
    Newton'sches Gesetz: $\vec{F} = \frac{d\vec{p}}{dt}$
\end{definition}

\begin{definition}{Kraftstoß}
    Änderung des Impulses durch Kraft über Zeit:
    \begin{equation}
        \vec{J} = \int_{t_1}^{t_2} \vec{F} \, dt = \Delta\vec{p} = m\vec{v}_f - m\vec{v}_i
    \end{equation}
    
    Für konstante Kraft: $\vec{J} = \vec{F}\Delta t$
\end{definition}

\begin{concept}{Impulserhaltung}
    Ohne äußere Kräfte bleibt der Gesamtimpuls konstant:
    \begin{equation}
        \sum \vec{p}_i = \sum \vec{p}_f
    \end{equation}
    
    Für Zwei-Körper-System:
    \begin{equation}
        m_1\vec{v}_{1i} + m_2\vec{v}_{2i} = m_1\vec{v}_{1f} + m_2\vec{v}_{2f}
    \end{equation}
\end{concept}

\begin{definition}{Drehimpuls}
    \textbf{Für Punktteilchen:}
    \begin{equation}
        \vec{L} = \vec{r} \times \vec{p} = m\vec{r} \times \vec{v}
    \end{equation}
    
    \textbf{Für starren Körper:}
    \begin{equation}
        \vec{L} = J\vec{\omega}
    \end{equation}
\end{definition}

\begin{concept}{Drehimpulserhaltung}
    Ohne äußere Drehmomente:
    \begin{equation}
        \frac{d\vec{L}}{dt} = \vec{\tau}_{ext} = 0 \Rightarrow \vec{L} = \text{const.}
    \end{equation}
    
    \textbf{Anwendungen:} Eiskunstläufer, Planetenbewegung, Kreisel
\end{concept}

\subsection{Kollisionen}

\begin{definition}{Kollisionstypen}
    \textbf{Elastisch:} Impuls und kinetische Energie erhalten
    
    \textbf{Inelastisch:} Nur Impuls erhalten
    
    \textbf{Vollständig inelastisch:} Objekte kleben nach Kollision zusammen
\end{definition}

\begin{formula}{Elastische Kollision (1D)}
    Für zwei Objekte mit Massen $m_1, m_2$ und Anfangsgeschwindigkeiten $v_{1i}, v_{2i}$:
    
    \begin{align}
        v_{1f} &= \frac{(m_1 - m_2)v_{1i} + 2m_2v_{2i}}{m_1 + m_2} \\
        v_{2f} &= \frac{(m_2 - m_1)v_{2i} + 2m_1v_{1i}}{m_1 + m_2}
    \end{align}
\end{formula}

\subsection{Unity Implementation}

\begin{code}{Energieberechnung in Unity}
\begin{lstlisting}[language=C, style=basesmol]
public class EnergyMonitor : MonoBehaviour 
{
    private Rigidbody rb;
    public float springConstant = 100f;
    public Vector3 equilibriumPosition = Vector3.zero;
    
    void Start() 
    {
        rb = GetComponent<Rigidbody>();
    }
    
    void Update() 
    {
        // Kinetische Energie berechnen
        float kineticEnergy = 0.5f * rb.mass * rb.velocity.sqrMagnitude;
        
        // Potentielle Energie (Gravitation)
        float gravitationalPE = rb.mass * Mathf.Abs(Physics.gravity.y) * 
                               transform.position.y;
        
        // Elastische potentielle Energie
        Vector3 displacement = transform.position - equilibriumPosition;
        float elasticPE = 0.5f * springConstant * displacement.sqrMagnitude;
        
        // Gesamtenergie
        float totalEnergy = kineticEnergy + gravitationalPE + elasticPE;
        
        Debug.Log($"KE: {kineticEnergy:F2}, PE: {gravitationalPE + elasticPE:F2}, " +
                  $"Total: {totalEnergy:F2}");
    }
}
\end{lstlisting}
\end{code}

\begin{code}{Kollisionsbehandlung in Unity}
\begin{lstlisting}[language=C, style=basesmol]
public class CollisionHandler : MonoBehaviour 
{
    private Rigidbody rb;
    
    void Start() 
    {
        rb = GetComponent<Rigidbody>();
    }
    
    void OnCollisionEnter(Collision collision) 
    {
        Rigidbody otherRb = collision.rigidbody;
        if (otherRb == null) return;
        
        // Massen und Geschwindigkeiten vor Kollision
        float m1 = rb.mass;
        float m2 = otherRb.mass;
        Vector3 v1i = rb.velocity;
        Vector3 v2i = otherRb.velocity;
        
        // Geschwindigkeiten nach elastischer Kollision
        Vector3 v1f = ((m1 - m2) * v1i + 2 * m2 * v2i) / (m1 + m2);
        Vector3 v2f = ((m2 - m1) * v2i + 2 * m1 * v1i) / (m1 + m2);
        
        // Neue Geschwindigkeiten anwenden
        rb.velocity = v1f;
        otherRb.velocity = v2f;
    }
}
\end{lstlisting}
\end{code}

\begin{code}{Drehimpuls in Unity}
\begin{lstlisting}[language=C, style=basesmol]
// Drehimpuls eines Koerpers bezueglich eines Pivot-Punkts berechnen
Vector3 AngularMomentum(Rigidbody rb, Vector3 pivotPoint) 
{
    // Vektor vom Pivot zum Schwerpunkt
    Vector3 rPQ = rb.transform.position - pivotPoint;
    
    // Linearer Impuls des Koerpers
    Vector3 p = rb.velocity * rb.mass;
    
    // Drehimpuls berechnen
    Vector3 angularMomentum = Vector3.Cross(rPQ, p);
    return angularMomentum;
}
\end{lstlisting}
\end{code}

\begin{KR}{Energie-Impuls-Problemlösung}
    \paragraph{Schritt 1: System und Randbedingungen identifizieren}
    \begin{itemize}
        \item Systemgrenzen definieren
        \item Konservative und nicht-konservative Kräfte identifizieren
        \item Erhaltungsgesetze prüfen
    \end{itemize}
    
    \paragraph{Schritt 2: Passenden Erhaltungssatz wählen}
    \begin{itemize}
        \item Energieerhaltung für Höhen-, Federprobleme
        \item Impulserhaltung für Kollisionsprobleme
        \item Beide für komplexe mehrstufige Probleme
    \end{itemize}
    
    \paragraph{Schritt 3: Gleichungen aufstellen}
    \begin{itemize}
        \item Anfangs- und Endzustände beschreiben
        \item Erhaltungsprinzipien anwenden
        \item Zusätzliche Randbedingungen einbeziehen
    \end{itemize}
    
    \paragraph{Schritt 4: Lösen und verifizieren}
    \begin{itemize}
        \item Algebraisch lösen vor Zahleneinsetzung
        \item Einheiten und physikalische Plausibilität prüfen
        \item Erhaltungsgesetze verifizieren
    \end{itemize}
\end{KR}

\begin{example2}{Kollisionsanalyse}
    Zwei Autos: Auto A (1000 kg) mit 20 m/s, Auto B (1500 kg) mit -15 m/s. Nach Kollision bewegt sich Auto A mit 5 m/s. Berechne Auto B's Endgeschwindigkeit und Energieverlust.
    \tcblower
    \textbf{Gegeben:} $m_A = 1000kg$, $m_B = 1500kg$, $v_{Ai} = 20m/s$, $v_{Bi} = -15m/s$, $v_{Af} = 5m/s$
    
    \textbf{Impulserhaltung:}
    $1000(20) + 1500(-15) = 1000(5) + 1500v_{Bf}$
    
    $20000 - 22500 = 5000 + 1500v_{Bf}$
    
    $v_{Bf} = -5m/s$
    
    \textbf{Energieanalyse:}
    $KE_i = \frac{1}{2}(1000)(20^2) + \frac{1}{2}(1500)(15^2) = 368750J$
    
    $KE_f = \frac{1}{2}(1000)(5^2) + \frac{1}{2}(1500)(5^2) = 31250J$
    
    $\Delta E = 368750 - 31250 = 337500J$ verloren
\end{example2}

\begin{example2}{Projektilbewegung mit Energie}
    Ein Projektil wird mit Winkel $\theta = 45°$ und Anfangsgeschwindigkeit $v_0 = 20m/s$ abgeschossen. Maximale Höhe mit Energieerhaltung berechnen.
    \tcblower
    \textbf{Energiemethode:}
    Start: $E_i = \frac{1}{2}mv_0^2$ (Boden als Referenz)
    
    Max. Höhe: $E_f = mgh_{max} + \frac{1}{2}mv_x^2$
    
    Da $v_x = v_0\cos\theta$ konstant bleibt:
    
    $\frac{1}{2}mv_0^2 = mgh_{max} + \frac{1}{2}m(v_0\cos\theta)^2$
    
    $h_{max} = \frac{v_0^2\sin^2\theta}{2g} = \frac{(20)^2\sin^2(45°)}{2(9.81)} = 10.2m$
\end{example2}