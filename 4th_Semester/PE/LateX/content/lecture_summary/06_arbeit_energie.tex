\section{Arbeit und Energie}

\subsection{Arbeit}
\begin{definition}{Arbeit}\\
    Die physikalische Arbeit $W$ ist definiert als das Skalarprodukt aus Kraft und Weg:
    \begin{equation}
        W = \int_{\vec{r}_0}^{\vec{r}_1} \vec{F} \cdot d\vec{r}
    \end{equation}
    
    Die Einheit der Arbeit ist Joule (J): $1 \, \text{J} = 1 \, \text{N} \cdot \text{m} = 1 \, \frac{\text{kg} \cdot \text{m}^2}{\text{s}^2}$
    
    Für den Spezialfall einer konstanten Kraft in Richtung des Weges vereinfacht sich die Formel zu:
    \begin{equation}
        W = F \cdot \Delta x
    \end{equation}
\end{definition}

\begin{formula}{Arbeit mit Winkel zwischen Kraft und Weg}\\
    Wenn die Kraft einen Winkel $\alpha$ mit der Wegrichtung einschließt, gilt:
    \begin{equation}
        W = F \cdot \cos(\alpha) \cdot \Delta x
    \end{equation}
    
    Nur die Komponente der Kraft in Richtung des Weges verrichtet Arbeit.
\end{formula}

\begin{concept}{Graphische Darstellung der Arbeit}\\
    In einem Weg-Kraft-Diagramm entspricht die Arbeit der Fläche unter der Kraft-Kurve:
    \begin{equation}
        W = \int_{x_0}^{x_1} F_x(x) \, dx
    \end{equation}
    
    Beispiele:
    \begin{itemize}
        \item Konstante Kraft: $W = F \cdot \Delta x$ (Rechteck)
        \item Linear ansteigende Kraft: $W = \frac{1}{2} \cdot F_{max} \cdot \Delta x$ (Dreieck)
    \end{itemize}
\end{concept}

\subsection{Formen physikalischer Arbeit}
\begin{concept}{Arten von Arbeit}\\
    Verschiedene Arten physikalischer Arbeit:
    \begin{itemize}
        \item Hubarbeit: $W = m \cdot g \cdot h$
        \item Beschleunigungsarbeit: $W = \frac{1}{2} \cdot m \cdot (v_1^2 - v_0^2)$
        \item Deformationsarbeit (Federspannung): $W = \frac{1}{2} \cdot k \cdot (x_1^2 - x_0^2)$
        \item Reibungsarbeit: $W = \mu \cdot F_N \cdot \Delta x$
    \end{itemize}
\end{concept}

\begin{KR}{Berechnung der Arbeit in Praxisbeispielen}\\
    \paragraph{Hubarbeit}
    \begin{itemize}
        \item Gegeben: Masse $m$, Höhendifferenz $h$
        \item Berechnung: $W = m \cdot g \cdot h$
        \item Beispiel: Ein 20 kg schwerer Körper wird 5 m angehoben:
        $W = 20 \text{ kg} \cdot 9.81 \frac{\text{m}}{\text{s}^2} \cdot 5 \text{ m} = 981 \text{ J}$
    \end{itemize}
    
    \paragraph{Federarbeit}
    \begin{itemize}
        \item Gegeben: Federkonstante $k$, Auslenkungen $x_0$ und $x_1$
        \item Berechnung: $W = \frac{1}{2} \cdot k \cdot (x_1^2 - x_0^2)$
        \item Beispiel: Eine Feder mit $k = 100 \text{ N/m}$ wird von 0 auf 0.2 m gedehnt:
        $W = \frac{1}{2} \cdot 100 \frac{\text{N}}{\text{m}} \cdot (0.2^2 - 0^2) \text{ m}^2 = 2 \text{ J}$
    \end{itemize}
    
    \paragraph{Reibungsarbeit}
    \begin{itemize}
        \item Gegeben: Reibungskoeffizient $\mu$, Normalkraft $F_N$, Strecke $\Delta x$
        \item Berechnung: $W = \mu \cdot F_N \cdot \Delta x$
        \item Beispiel: Ein 5 kg schwerer Körper wird 10 m weit gezogen mit $\mu = 0.2$:
        $W = 0.2 \cdot (5 \text{ kg} \cdot 9.81 \frac{\text{m}}{\text{s}^2}) \cdot 10 \text{ m} = 98.1 \text{ J}$
    \end{itemize}
\end{KR}

\subsection{Energie}
\begin{definition}{Energie}\\
    Energie ist die Fähigkeit eines Systems, Arbeit zu verrichten. Die an einem Körper verrichtete Arbeit ist gleich der Änderung seiner Energie:
    \begin{equation}
        W = \Delta E = E_{nachher} - E_{vorher}
    \end{equation}
    
    Die Einheit der Energie ist ebenfalls Joule (J).
    
    Während Arbeit einen Prozess beschreibt (Prozessgröße), charakterisiert Energie einen Zustand (Zustandsgröße).
\end{definition}

\begin{concept}{Energieformen}\\
    Die wichtigsten Energieformen in der Mechanik:
    \begin{itemize}
        \item Kinetische Energie (Bewegungsenergie): $E_{kin} = \frac{1}{2} \cdot m \cdot v^2$
        \item Potentielle Energie im Schwerefeld: $E_{pot} = m \cdot g \cdot h$
        \item Spannenergie einer Feder: $E_{spann} = \frac{1}{2} \cdot k \cdot x^2$
        \item Rotationsenergie: $E_{rot} = \frac{1}{2} \cdot J \cdot \omega^2$
    \end{itemize}
\end{concept}

\begin{formula}{Energieerhaltung}\\
    Das Prinzip der Energieerhaltung besagt, dass in einem abgeschlossenen System ohne Einwirkung nicht-konservativer Kräfte die Gesamtenergie konstant bleibt:
    \begin{equation}
        E_{ges} = E_{kin} + E_{pot} + E_{spann} + ... = \text{const.}
    \end{equation}
    
    Bei konservativen Kräften bleibt die mechanische Energie erhalten. Bei dissipativen Kräften (wie Reibung) wird mechanische Energie in andere Energieformen (meist Wärme) umgewandelt.
\end{formula}

\begin{example2}{Energiewandlungen beim Pendel}\\
    Ein Pendel der Masse $m$ und Länge $l$ zeigt deutlich die Umwandlung zwischen potentieller und kinetischer Energie:
    
    \begin{itemize}
        \item Am höchsten Punkt: maximale potentielle Energie, keine kinetische Energie
        \begin{equation}
            E_{pot,max} = m \cdot g \cdot h = m \cdot g \cdot l \cdot (1 - \cos\alpha)
        \end{equation}
        
        \item Am tiefsten Punkt: maximale kinetische Energie, minimale potentielle Energie
        \begin{equation}
            E_{kin,max} = \frac{1}{2} \cdot m \cdot v_{max}^2 = m \cdot g \cdot l \cdot (1 - \cos\alpha)
        \end{equation}
        
        \item Die Gesamtenergie $E_{ges} = E_{pot} + E_{kin}$ bleibt konstant (bei Vernachlässigung der Reibung)
    \end{itemize}
\end{example2}

\begin{KR}{Anwendung der Energieerhaltung zur Problemlösung}\\
    \paragraph{Allgemeines Vorgehen}
    \begin{enumerate}
        \item Identifiziere die relevanten Energieformen im Anfangs- und Endzustand
        \item Stelle die Energieerhaltungsgleichung auf (bei konservativen Systemen)
        \item Berücksichtige eventuelle Energieverluste durch Reibung
        \item Löse die Gleichung nach der gesuchten Größe auf
    \end{enumerate}
    
    \paragraph{Beispiel: Ball wird von Höhe $h_1$ fallen gelassen und springt auf Höhe $h_2$}
    \begin{itemize}
        \item Anfangszustand: $E_1 = m \cdot g \cdot h_1$
        \item Endzustand: $E_2 = m \cdot g \cdot h_2$
        \item Bei teilelastischem Stoß gilt: $h_2 = e^2 \cdot h_1$, wobei $e$ der Restitutionskoeffizient ist
        \item Der Energieverlust beträgt: $\Delta E = m \cdot g \cdot (h_1 - h_2) = m \cdot g \cdot h_1 \cdot (1 - e^2)$
    \end{itemize}
\end{KR}

\subsection{Leistung und Wirkungsgrad}
\begin{definition}{Leistung}\\
    Die Leistung $P$ ist die pro Zeiteinheit verrichtete Arbeit:
    \begin{equation}
        P = \frac{dW}{dt}
    \end{equation}
    
    Die Einheit der Leistung ist Watt (W): $1 \, \text{W} = 1 \, \frac{\text{J}}{\text{s}} = 1 \, \frac{\text{kg} \cdot \text{m}^2}{\text{s}^3}$
    
    Für konstante Leistung gilt:
    \begin{equation}
        W = P \cdot t
    \end{equation}
\end{definition}

\begin{definition}{Wirkungsgrad}\\
    Der Wirkungsgrad $\eta$ ist das Verhältnis von Nutzenergie zu zugeführter Energie:
    \begin{equation}
        \eta = \frac{E_{Nutz}}{E_{zugeführt}}
    \end{equation}
    
    Der Wirkungsgrad ist eine dimensionslose Zahl zwischen 0 und 1 (oder in Prozent ausgedrückt: 0-100\%).
\end{definition}

\begin{example2}{Leistung und Wirkungsgrad einer Pumpe}\\
    Eine Pumpe hebt Wasser aus einer Tiefe von 5 m auf eine Höhe von 15 m. Sie liefert 0.2 m³ Wasser pro Minute und benötigt eine elektrische Leistung von 700 W.
    
    \begin{itemize}
        \item Nützliche Leistung: 
        \begin{align}
            P_{Nutz} &= \frac{\Delta E_{pot}}{\Delta t} = \frac{m \cdot g \cdot \Delta h}{\Delta t} \\
            &= \rho \cdot V \cdot g \cdot \Delta h / \Delta t \\
            &= 1000 \frac{\text{kg}}{\text{m}^3} \cdot 0.2 \frac{\text{m}^3}{60 \text{ s}} \cdot 9.81 \frac{\text{m}}{\text{s}^2} \cdot 20 \text{ m} \\
            &\approx 654 \text{ W}
        \end{align}
        
        \item Wirkungsgrad: 
        \begin{equation}
            \eta = \frac{P_{Nutz}}{P_{el}} = \frac{654 \text{ W}}{700 \text{ W}} \approx 0.93 = 93\%
        \end{equation}
    \end{itemize}
\end{example2}

\begin{examplecode}{Energie und Leistung in Unity}\\
    \begin{lstlisting}[language=csh, style=basesmol]
// Berechnung und Ueberwachung der Energien in Unity
void CalculateEnergies() {
    // Kinetische Energie
    float kineticEnergy = 0.5f * rigidbody.mass * 
                         rigidbody.velocity.sqrMagnitude;
    
    // Potentielle Energie (Hoehe relativ zum Boden)
    float potentialEnergy = rigidbody.mass * 9.81f * 
                           (transform.position.y - groundLevel);
    
    // Spannenergie (falls zutreffend)
    float springEnergy = 0.5f * springConstant * 
                        (transform.position.x - equilibriumPosition)^2;
    
    // Gesamtenergie
    float totalEnergy = kineticEnergy + potentialEnergy + springEnergy;
    
    // Ueberwachen der Energieerhaltung
    if (Mathf.Abs(totalEnergy - initialTotalEnergy) > tolerance) {
        Debug.LogWarning("Energy not conserved! Check for numerical errors or non-conservative forces.");
    }
    
    // Momentane Leistung berechnen (Energieaenderung pro Zeit)
    float power = (totalEnergy - lastTotalEnergy) / Time.deltaTime;
    lastTotalEnergy = totalEnergy;
}
    \end{lstlisting}
\end{examplecode}