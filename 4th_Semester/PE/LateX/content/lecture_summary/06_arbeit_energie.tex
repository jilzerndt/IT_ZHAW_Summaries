\section{Arbeit und Energie}

\subsection{Arbeit}
\begin{definition}{Arbeit}\\
    Die physikalische Arbeit $W$ ist definiert als das Skalarprodukt aus Kraft und Weg:
    \begin{equation}
        W = \int_{\vec{r}_0}^{\vec{r}_1} \vec{F} \cdot d\vec{r}
    \end{equation}
    
    Die Einheit der Arbeit ist Joule (J): $1 \, \text{J} = 1 \, \text{N} \cdot \text{m} = 1 \, \frac{\text{kg} \cdot \text{m}^2}{\text{s}^2}$
    
    Für den Spezialfall einer konstanten Kraft in Richtung des Weges vereinfacht sich die Formel zu:
    \begin{equation}
        W = F \cdot \Delta x
    \end{equation}
\end{definition}

\begin{formula}{Arbeit mit Winkel zwischen Kraft und Weg}\\
    Wenn die Kraft einen Winkel $\alpha$ mit der Wegrichtung einschließt, gilt:
    \begin{equation}
        W = F \cdot \cos(\alpha) \cdot \Delta x
    \end{equation}
    
    Nur die Komponente der Kraft in Richtung des Weges verrichtet Arbeit.
\end{formula}

\begin{concept}{Graphische Darstellung der Arbeit}\\
    In einem Weg-Kraft-Diagramm entspricht die Arbeit der Fläche unter der Kraft-Kurve:
    \begin{equation}
        W = \int_{x_0}^{x_1} F_x(x) \, dx
    \end{equation}
    
    Beispiele:
    \begin{itemize}
        \item Konstante Kraft: $W = F \cdot \Delta x$ (Rechteck)
        \item Linear ansteigende Kraft: $W = \frac{1}{2} \cdot F_{max} \cdot \Delta x$ (Dreieck)
    \end{itemize}
\end{concept}

\subsection{Formen physikalischer Arbeit}
\begin{concept}{Arten von Arbeit}\\
    Verschiedene Arten physikalischer Arbeit:
    \begin{itemize}
        \item Hubarbeit: $W = m \cdot g \cdot h$
        \item Beschleunigungsarbeit: $W = \frac{1}{2} \cdot m \cdot (v_1^2 - v_0^2)$
        \item Deformationsarbeit (Federspannung): $W = \frac{1}{2} \cdot k \cdot (x_1^2 - x_0^2)$
        \item Reibungsarbeit: $W = \mu \cdot F_N \cdot \Delta x$
    \end{itemize}
\end{concept}

\begin{KR}{Berechnung der Arbeit in Praxisbeispielen}\\
    \paragraph{Hubarbeit}
    \begin{itemize}
        \item Gegeben: Masse $m$, Höhendifferenz $h$
        \item Berechnung: $W = m \cdot g \cdot h$
        \item Beispiel: Ein 20 kg schwerer Körper wird 5 m angehoben:
        $W = 20 \text{ kg} \cdot 9.81 \frac{\text{m}}{\text{s}^2} \cdot 5 \text{ m} = 981 \text{ J}$
    \end{itemize}
    
    \paragraph{Federarbeit}
    \begin{itemize}
        \item Gegeben: Federkonstante $k$, Auslenkungen $x_0$ und $x_1$
        \item Berechnung: $W = \frac{1}{2} \cdot k \cdot (x_1^2 - x_0^2)$
        \item Beispiel: Eine Feder mit $k = 100 \text{ N/m}$ wird von 0 auf 0.2 m gedehnt:
        $W = \frac{1}{2} \cdot 100 \frac{\text{N}}{\text{m}} \cdot (0.2^2 - 0^2) \text{ m}^2 = 2 \text{ J}$
    \end{itemize}
    
    \paragraph{Reibungsarbeit}
    \begin{itemize}
        \item Gegeben: Reibungskoeffizient $\mu$, Normalkraft $F_N$, Strecke $\Delta x$
        \item Berechnung: $W = \mu \cdot F_N \cdot \Delta x$
        \item Beispiel: Ein 5 kg schwerer Körper wird 10 m weit gezogen mit $\mu = 0.2$:
        $W = 0.2 \cdot (5 \text{ kg} \cdot 9.81 \frac{\text{m}}{\text{s}^2}) \cdot 10 \text{ m} = 98.1 \text{ J}$
    \end{itemize}
\end{KR}

\subsection{Energie}
\begin{definition}{Energie}\\
    Energie ist die Fähigkeit eines Systems, Arbeit zu verrichten. Die an einem Körper verrichtete Arbeit ist gleich der Änderung seiner Energie:
    \begin{equation}
        W = \Delta E = E_{nachher} - E_{vorher}
    \end{equation}
    
    Die Einheit der Energie ist ebenfalls Joule (J).
    
    Während Arbeit einen Prozess beschreibt (Prozessgröße), charakterisiert Energie einen Zustand (Zustandsgröße).
\end{definition}

\begin{concept}{Energieformen}\\
    Die wichtigsten Energieformen in der Mechanik:
    \begin{itemize}
        \item Kinetische Energie (Bewegungsenergie): $E_{kin} = \frac{1}{2} \cdot m \cdot v^2$
        \item Potentielle Energie im Schwerefeld: $E_{pot} = m \cdot g \cdot h$
        \item Spannenergie einer Feder: $E_{spann} = \frac{1}{2} \cdot k \cdot x^2$
        \item Rotationsenergie: $E_{rot} = \frac{1}{2} \cdot J \cdot \omega^2$
    \end{itemize}
\end{concept}

\begin{formula}{Energieerhaltung}\\
    Das Prinzip der Energieerhaltung besagt, dass in einem abgeschlossenen System ohne Einwirkung nicht-konservativer Kräfte die Gesamtenergie konstant bleibt:
    \begin{equation}
        E_{ges} = E_{kin} + E_{pot} + E_{spann} + ... = \text{const.}
    \end{equation}
    
    Bei konservativen Kräften bleibt die mechanische Energie erhalten. Bei dissipativen Kräften (wie Reibung) wird mechanische Energie in andere Energieformen (meist Wärme) umgewandelt.
\end{formula}

\begin{example2}{Energiewandlungen beim Pendel}\\
    Ein Pendel der Masse $m$ und Länge $l$ zeigt deutlich die Umwandlung zwischen potentieller und kinetischer Energie:
    
    \begin{itemize}
        \item Am höchsten Punkt: maximale potentielle Energie, keine kinetische Energie
        \begin{equation}
            E_{pot,max} = m \cdot g \cdot h = m \cdot g \cdot l \cdot (1 - \cos\alpha)
        \end{equation}
        
        \item Am tiefsten Punkt: maximale kinetische Energie, minimale potentielle Energie
        \begin{equation}
            E_{kin,max} = \frac{1}{2} \cdot m \cdot v_{max}^2 = m \cdot g \cdot l \cdot (1 - \cos\alpha)
        \end{equation}
        
        \item Die Gesamtenergie $E_{ges} = E_{pot} + E_{kin}$ bleibt konstant (bei Vernachlässigung der Reibung)
    \end{itemize}
\end{example2}

\begin{KR}{Anwendung der Energieerhaltung zur Problemlösung}\\
    \paragraph{Allgemeines Vorgehen}
    \begin{enumerate}
        \item Identifiziere die relevanten Energieformen im Anfangs- und Endzustand
        \item Stelle die Energieerhaltungsgleichung auf (bei konservativen Systemen)
        \item Berücksichtige eventuelle Energieverluste durch Reibung
        \item Löse die Gleichung nach der gesuchten Größe auf
    \end{enumerate}
    
    \paragraph{Beispiel: Ball wird von Höhe $h_1$ fallen gelassen und springt auf Höhe $h_2$}
    \begin{itemize}
        \item Anfangszustand: $E_1 = m \cdot g \cdot h_1$
        \item Endzustand: $E_2 = m \cdot g \cdot h_2$
        \item Bei teilelastischem Stoß gilt: $h_2 = e^2 \cdot h_1$, wobei $e$ der Restitutionskoeffizient ist
        \item Der Energieverlust beträgt: $\Delta E = m \cdot g \cdot (h_1 - h_2) = m \cdot g \cdot h_1 \cdot (1 - e^2)$
    \end{itemize}
\end{KR}

\subsection{Leistung und Wirkungsgrad}
\begin{definition}{Leistung}\\
    Die Leistung $P$ ist die pro Zeiteinheit verrichtete Arbeit:
    \begin{equation}
        P = \frac{dW}{dt}
    \end{equation}
    
    Die Einheit der Leistung ist Watt (W): $1 \, \text{W} = 1 \, \frac{\text{J}}{\text{s}} = 1 \, \frac{\text{kg} \cdot \text{m}^2}{\text{s}^3}$
    
    Für konstante Leistung gilt:
    \begin{equation}
        W = P \cdot t
    \end{equation}
\end{definition}

\begin{definition}{Wirkungsgrad}\\
    Der Wirkungsgrad $\eta$ ist das Verhältnis von Nutzenergie zu zugeführter Energie:
    \begin{equation}
        \eta = \frac{E_{Nutz}}{E_{zugeführt}}
    \end{equation}
    
    Der Wirkungsgrad ist eine dimensionslose Zahl zwischen 0 und 1 (oder in Prozent ausgedrückt: 0-100\%).
\end{definition}

\begin{example2}{Leistung und Wirkungsgrad einer Pumpe}\\
    Eine Pumpe hebt Wasser aus einer Tiefe von 5 m auf eine Höhe von 15 m. Sie liefert 0.2 m³ Wasser pro Minute und benötigt eine elektrische Leistung von 700 W.
    
    \begin{itemize}
        \item Nützliche Leistung: 
        \begin{align}
            P_{Nutz} &= \frac{\Delta E_{pot}}{\Delta t} = \frac{m \cdot g \cdot \Delta h}{\Delta t} \\
            &= \rho \cdot V \cdot g \cdot \Delta h / \Delta t \\
            &= 1000 \frac{\text{kg}}{\text{m}^3} \cdot 0.2 \frac{\text{m}^3}{60 \text{ s}} \cdot 9.81 \frac{\text{m}}{\text{s}^2} \cdot 20 \text{ m} \\
            &\approx 654 \text{ W}
        \end{align}
        
        \item Wirkungsgrad: 
        \begin{equation}
            \eta = \frac{P_{Nutz}}{P_{el}} = \frac{654 \text{ W}}{700 \text{ W}} \approx 0.93 = 93\%
        \end{equation}
    \end{itemize}
\end{example2}

\begin{examplecode}{Energie und Leistung in Unity}\\
    \begin{lstlisting}[language=csh, style=basesmol]
// Berechnung und Ueberwachung der Energien in Unity
void CalculateEnergies() {
    // Kinetische Energie
    float kineticEnergy = 0.5f * rigidbody.mass * 
                         rigidbody.velocity.sqrMagnitude;
    
    // Potentielle Energie (Hoehe relativ zum Boden)
    float potentialEnergy = rigidbody.mass * 9.81f * 
                           (transform.position.y - groundLevel);
    
    // Spannenergie (falls zutreffend)
    float springEnergy = 0.5f * springConstant * 
                        (transform.position.x - equilibriumPosition)^2;
    
    // Gesamtenergie
    float totalEnergy = kineticEnergy + potentialEnergy + springEnergy;
    
    // Ueberwachen der Energieerhaltung
    if (Mathf.Abs(totalEnergy - initialTotalEnergy) > tolerance) {
        Debug.LogWarning("Energy not conserved! Check for numerical errors or non-conservative forces.");
    }
    
    // Momentane Leistung berechnen (Energieaenderung pro Zeit)
    float power = (totalEnergy - lastTotalEnergy) / Time.deltaTime;
    lastTotalEnergy = totalEnergy;
}
    \end{lstlisting}
\end{examplecode}

\section{Energy and Momentum}

\subsection{Work and Energy}

\begin{definition}{Work}\\
    Work done by a force $\vec{F}$ over displacement $\vec{s}$:
    $$W = \vec{F} \cdot \vec{s} = Fs\cos\theta$$
    
    For variable force:
    $$W = \int \vec{F} \cdot d\vec{s}$$
\end{definition}

\begin{definition}{Kinetic Energy}\\
    Energy of motion:
    $$K = \frac{1}{2}mv^2$$
    
    For rotational motion:
    $$K_{rot} = \frac{1}{2}I\omega^2$$
    
    where $I$ is moment of inertia and $\omega$ is angular velocity.
\end{definition}

\begin{definition}{Potential Energy}\\
    Energy stored in position or configuration:
    
    \textbf{Gravitational:} $U_g = mgh$
    
    \textbf{Elastic (spring):} $U_s = \frac{1}{2}kx^2$
    
    \textbf{General:} $U(\vec{r}) = -\int \vec{F} \cdot d\vec{r}$
\end{definition}

\begin{concept}{Work-Energy Theorem}\\
    The work done on an object equals its change in kinetic energy:
    $$W_{net} = \Delta K = K_f - K_i$$
\end{concept}

\subsection{Conservation of Energy}

\begin{concept}{Mechanical Energy Conservation}\\
    In absence of non-conservative forces:
    $$E = K + U = \text{constant}$$
    
    For conservative systems:
    $$\frac{1}{2}mv_1^2 + U_1 = \frac{1}{2}mv_2^2 + U_2$$
\end{concept}

\begin{definition}{Conservative vs. Non-Conservative Forces}
    \begin{itemize}
        \item \textbf{Conservative:} Work is path-independent (gravity, spring force)
        \item \textbf{Non-conservative:} Work depends on path (friction, air resistance)
    \end{itemize}
\end{definition}

\begin{code}{Energy Calculation in Unity}\\
\begin{lstlisting}[language=C, style=basesmol]
public class EnergyMonitor : MonoBehaviour 
{
    private Rigidbody rb;
    public float springConstant = 100f;
    public Vector3 equilibriumPosition = Vector3.zero;
    
    void Start() 
    {
        rb = GetComponent<Rigidbody>();
    }
    
    void Update() 
    {
        // Calculate kinetic energy
        float kineticEnergy = 0.5f * rb.mass * rb.velocity.sqrMagnitude;
        
        // Calculate gravitational potential energy
        float gravitationalPE = rb.mass * Mathf.Abs(Physics.gravity.y) * 
                               transform.position.y;
        
        // Calculate elastic potential energy
        Vector3 displacement = transform.position - equilibriumPosition;
        float elasticPE = 0.5f * springConstant * displacement.sqrMagnitude;
        
        // Total mechanical energy
        float totalEnergy = kineticEnergy + gravitationalPE + elasticPE;
        
        Debug.Log($"KE: {kineticEnergy:F2}, PE: {gravitationalPE + elasticPE:F2}, " +
                  $"Total: {totalEnergy:F2}");
    }
}
\end{lstlisting}
\end{code}

\subsection{Power}

\begin{definition}{Power}\\
    Rate of doing work:
    $P = \frac{dW}{dt} = \vec{F} \cdot \vec{v}$
    
    For rotational motion:
    $P = \tau \omega$
\end{definition}

\subsection{Momentum and Impulse}

\begin{definition}{Linear Momentum}\\
    $\vec{p} = m\vec{v}$
    
    Newton's Second Law in terms of momentum:
    $\vec{F} = \frac{d\vec{p}}{dt}$
\end{definition}

\begin{definition}{Impulse}\\
    Change in momentum due to force over time:
    $\vec{J} = \int_{t_1}^{t_2} \vec{F} \, dt = \Delta\vec{p} = m\vec{v}_f - m\vec{v}_i$
    
    For constant force: $\vec{J} = \vec{F}\Delta t$
\end{definition}

\begin{concept}{Conservation of Momentum}\\
    In absence of external forces, total momentum is conserved:
    $\sum \vec{p}_i = \sum \vec{p}_f$
    
    For two-body system:
    $m_1\vec{v}_{1i} + m_2\vec{v}_{2i} = m_1\vec{v}_{1f} + m_2\vec{v}_{2f}$
\end{concept}

\subsection{Collisions}

\begin{definition}{Types of Collisions}
    \begin{itemize}
        \item \textbf{Elastic:} Both momentum and kinetic energy conserved
        \item \textbf{Inelastic:} Only momentum conserved
        \item \textbf{Perfectly inelastic:} Objects stick together after collision
    \end{itemize}
\end{definition}

\begin{formula}{Elastic Collision (1D)}\\
    For two objects with masses $m_1, m_2$ and initial velocities $v_{1i}, v_{2i}$:
    
    \paragraph{Final velocities:}
    $v_{1f} = \frac{(m_1 - m_2)v_{1i} + 2m_2v_{2i}}{m_1 + m_2}$
    
    $v_{2f} = \frac{(m_2 - m_1)v_{2i} + 2m_1v_{1i}}{m_1 + m_2}$
\end{formula}

\begin{code}{Collision Implementation in Unity}\\
\begin{lstlisting}[language=C, style=basesmol]
public class CollisionHandler : MonoBehaviour 
{
    private Rigidbody rb;
    
    void Start() 
    {
        rb = GetComponent<Rigidbody>();
    }
    
    void OnCollisionEnter(Collision collision) 
    {
        Rigidbody otherRb = collision.rigidbody;
        if (otherRb == null) return;
        
        // Get masses and velocities before collision
        float m1 = rb.mass;
        float m2 = otherRb.mass;
        Vector3 v1i = rb.velocity;
        Vector3 v2i = otherRb.velocity;
        
        // Calculate velocities after elastic collision
        Vector3 v1f = ((m1 - m2) * v1i + 2 * m2 * v2i) / (m1 + m2);
        Vector3 v2f = ((m2 - m1) * v2i + 2 * m1 * v1i) / (m1 + m2);
        
        // Apply new velocities
        rb.velocity = v1f;
        otherRb.velocity = v2f;
    }
}
\end{lstlisting}
\end{code}

\subsection{Angular Momentum}

\begin{definition}{Angular Momentum}\\
    For point particle:
    $\vec{L} = \vec{r} \times \vec{p} = \vec{r} \times m\vec{v}$
    
    For rigid body:
    $\vec{L} = I\vec{\omega}$
    
    where $I$ is moment of inertia and $\vec{\omega}$ is angular velocity.
\end{definition}

\begin{concept}{Conservation of Angular Momentum}\\
    In absence of external torques:
    $\frac{d\vec{L}}{dt} = \vec{\tau}_{ext} = 0 \Rightarrow \vec{L} = \text{constant}$
\end{concept}

\begin{code}{Angular Momentum in Unity}\\
\begin{lstlisting}[language=C, style=basesmol]
// Calculate angular momentum of a body at point Q with respect to pivot P
Vector3 AngularMomentum(Rigidbody rb, Vector3 pivotPoint) 
{
    // Vector from pivot point to center of mass
    Vector3 rPQ = rb.transform.position - pivotPoint;
    
    // Linear momentum of the body
    Vector3 p = rb.velocity * rb.mass;
    
    // Calculate and return angular momentum
    Vector3 angularMomentum = Vector3.Cross(rPQ, p);
    return angularMomentum;
}
\end{lstlisting}
\end{code}

\begin{KR}{Energy-Momentum Problem Solving}\\
    \paragraph{Step 1: Identify the system and constraints}
    \begin{itemize}
        \item Define system boundaries
        \item Identify conservative and non-conservative forces
        \item Check for momentum/energy conservation conditions
    \end{itemize}
    
    \paragraph{Step 2: Choose appropriate conservation law}
    \begin{itemize}
        \item Use energy conservation for problems involving heights, springs
        \item Use momentum conservation for collision problems
        \item Use both for complex multi-stage problems
    \end{itemize}
    
    \paragraph{Step 3: Set up equations}
    \begin{itemize}
        \item Write initial and final energy/momentum expressions
        \item Apply conservation principles
        \item Include any additional constraints
    \end{itemize}
    
    \paragraph{Step 4: Solve and verify}
    \begin{itemize}
        \item Solve algebraically before substituting numbers
        \item Check units and physical reasonableness
        \item Verify conservation laws are satisfied
    \end{itemize}
\end{KR}

\begin{example2}{Collision Analysis Problem}\\
    Two cars approach each other: Car A (1000 kg) at 20 m/s, Car B (1500 kg) at -15 m/s. After collision, Car A moves at 5 m/s. Find Car B's final velocity and energy lost.
    \tcblower
    \textbf{Given:} $m_A = 1000kg$, $m_B = 1500kg$, $v_{Ai} = 20m/s$, $v_{Bi} = -15m/s$, $v_{Af} = 5m/s$
    
    \textbf{Momentum conservation:}
    $m_Av_{Ai} + m_Bv_{Bi} = m_Av_{Af} + m_Bv_{Bf}$
    
    $1000(20) + 1500(-15) = 1000(5) + 1500v_{Bf}$
    
    $20000 - 22500 = 5000 + 1500v_{Bf}$
    
    $v_{Bf} = -5m/s$
    
    \textbf{Energy analysis:}
    $KE_i = \frac{1}{2}(1000)(20^2) + \frac{1}{2}(1500)(15^2) = 368750J$
    
    $KE_f = \frac{1}{2}(1000)(5^2) + \frac{1}{2}(1500)(5^2) = 31250J$
    
    $\Delta E = 368750 - 31250 = 337500J$ lost
\end{example2}

\begin{example2}{Projectile Motion with Energy}\\
    A projectile is launched at angle $\theta = 45°$ with initial speed $v_0 = 20m/s$. Find maximum height using energy conservation.
    \tcblower
    \textbf{Energy approach:}
    At launch: $E_i = \frac{1}{2}mv_0^2$ (taking ground as reference)
    
    At max height: $E_f = mgh_{max} + \frac{1}{2}mv_x^2$
    
    Since $v_x = v_0\cos\theta$ remains constant:
    
    $\frac{1}{2}mv_0^2 = mgh_{max} + \frac{1}{2}m(v_0\cos\theta)^2$
    
    $\frac{1}{2}v_0^2 = gh_{max} + \frac{1}{2}v_0^2\cos^2\theta$
    
    $h_{max} = \frac{v_0^2\sin^2\theta}{2g} = \frac{(20)^2\sin^2(45°)}{2(9.81)} = 10.2m$
\end{example2}