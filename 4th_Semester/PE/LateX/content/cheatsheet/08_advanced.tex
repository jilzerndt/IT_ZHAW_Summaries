\section{Erweiterte Themen und Unity-Implementierung}

\subsection{Unity Physics System}

\begin{concept}{Unity Physics Pipeline}
    Unity's Physiksimulation läuft in diskreten Schritten:
    \begin{itemize}
        \item \texttt{FixedUpdate()}: Wird in festen Intervallen aufgerufen (Standard 50Hz)
        \item Physikberechnungen zwischen FixedUpdate-Aufrufen
        \item \texttt{Update()}: Wird einmal pro Frame aufgerufen (variable Rate)
        \item \textbf{Regel}: FixedUpdate für physikbezogenen Code verwenden
    \end{itemize}
    
    \textbf{Zeitsteuerung:}
    \begin{itemize}
        \item \texttt{Time.fixedDeltaTime}: Zeitschritt für Physik (Standard: 0.02s = 50Hz)
        \item \texttt{Time.deltaTime}: Zeit seit letztem Frame (variabel)
        \item \texttt{Time.timeScale}: Globaler Zeitfaktor (für Slow-Motion etc.)
    \end{itemize}
\end{concept}

\mult{2}

\begin{definition}{Rigidbody-Komponenten-Eigenschaften}

    \textbf{Grundlegende Eigenschaften:}
    \begin{itemize}
        \item \textbf{Mass}: Objektmasse in kg
        \item \textbf{Drag}: Linearer Dämpfungskoeffizient\\ (Luftwiderstand)
        \item \textbf{Angular Drag}: Rotationsdämpfungskoeffizient
        \item \textbf{Use Gravity}: Reagiert auf globale Gravitation
        \item \textbf{Is Kinematic}: \\ Physik-gesteuert vs. Skript-gesteuert
    \end{itemize}
    
    \textbf{Erweiterte Eigenschaften:}
    \begin{itemize}
        \item \textbf{Interpolate}: Glättung für visuelle Darstellung
        \item \textbf{Collision Detection}: Continuous, Discrete, etc.
        \item \textbf{Constraints}: Einschränkung Position/Rotation
        \item \textbf{Center of Mass}: Schwerpunkt-Position
    \end{itemize}
\end{definition}

\begin{concept}{Unity Physics Setup}

    \textbf{Wichtige Konfigurationen:}
    \begin{itemize}
        \item Physics.gravity für globale Gravitation
        \item Time.fixedDeltaTime für Physik-Zeitschritt
        \item Rigidbody: mass, drag, angularDrag
        \item CollisionDetectionMode.Continuous für schnelle Objekte
        \item RigidbodyInterpolation für glatte Bewegung
    \end{itemize}
\end{concept}


\begin{theorem}{Wichtige Erkenntnisse}

    \textbf{Theoretische Grundlagen:}
    \begin{itemize}
        \item Physikalische Gesetze gelten universell
        \item Erhaltungsgesetze zentral für Problemlösungen
        \item Numerische Integration beeinflusst Genauigkeit
        \item Reibung und Dämpfung sind realitätsnah wichtig
    \end{itemize}
    
    \textbf{Unity-spezifische Aspekte:}
    \begin{itemize}
        \item Fixed Timestep ist entscheidend für Stabilität
        \item Physics Materials beeinflussen Verhalten stark
        \item Performance-Optimierung ist für komplexe Szenen notwendig
        \item Debugging-Tools helfen bei der Problemanalyse
    \end{itemize}
\end{theorem}

\begin{example2}{Kugel-Kollisions-Simulation}

    \textbf{Problem:} Rollende Kugel auf Schräge kollidiert elastisch
    
    \textbf{Physik-Berechnung:}
    \begin{itemize}
        \item Rollbeschleunigung: $a = \frac{5g\sin\theta}{7} = 3.51 \, m/s^2$
        \item Endgeschwindigkeit: $v = \sqrt{2ad} = 3.74 \, m/s$
        \item Elastische Kollision: Geschwindigkeiten tauschen
    \end{itemize}
    
    \textbf{Unity-Konzept:}
    \begin{itemize}
        \item Schräge mit PhysicMaterial (hohe Reibung)
        \item Zwei Kugeln mit elastischem Material
        \item OnCollisionEnter() für Geschwindigkeitstausch
        \item Debug-Ausgabe für Energieerhaltung
    \end{itemize}
\end{example2}

\multend

\subsubsection{Unity Kraft-Modi}

\mult{2}

\begin{definition}{Unity Kraftanwendung}

    \textbf{Kraftmethoden:}
    \begin{itemize}
        \item \textbf{AddForce()}: Standard-Kraftanwendung
        \item \textbf{AddForceAtPosition()}: Kraft + Drehmoment
        \item \textbf{AddTorque()}: Nur Rotation
        \item \textbf{AddExplosionForce()}: Radiale Explosion
    \end{itemize}
    
    \textbf{Force-Modi:}
    \begin{itemize}
        \item \textbf{Force}: $F = ma$ (mit Masse/Zeit)
        \item \textbf{Acceleration}: $a$ direkt (ohne Masse)
        \item \textbf{Impulse}: $J = F\Delta t$ (mit Masse)
        \item \textbf{VelocityChange}: $\Delta v$ direkt (ohne Masse)
    \end{itemize}
\end{definition}

\begin{concept}{Erweiterte Kraftsteuerung}

    \textbf{Controller-Konzept:}
    \begin{itemize}
        \item Input-basierte Kraftanwendung
        \item Bewegung: transform.forward * thrustForce
        \item Rotation: AddTorque mit transform.up/right
        \item Explosion: AddExplosionForce für Umgebung
        \item Geschwindigkeitsbegrenzung mit velocity.normalized
    \end{itemize}
\end{concept}

\multend

\subsubsection{Unity Kollisionen}

\mult{2}

\begin{definition}{Unity Kollisions-Events}

    \textbf{Events:}
    \begin{itemize}
        \item \textbf{Collision}: OnCollisionEnter/Stay/Exit
        \item \textbf{Trigger}: OnTriggerEnter/Stay/Exit
    \end{itemize}
    
    \textbf{Informationen:}
    \begin{itemize}
        \item \textbf{contacts}: Kontaktpunkte
        \item \textbf{impulse}: Stoßimpuls
        \item \textbf{relativeVelocity}: Relativgeschwindigkeit
        \item \textbf{normal}: Oberflächennormale
    \end{itemize}
\end{definition}

\begin{concept}{Erweiterte Kollisions-Analyse}
    \textbf{Reaktions-System:}
    \begin{itemize}
        \item Kollisionsgeschwindigkeit aus relativeVelocity
        \item Material-spezifische Restitution (Tags)
        \item Audio/Partikel basierend auf Intensität
        \item Spezielle Reaktionen: Bouncy, Sticky, Breakable
        \item Vector3.Reflect() für elastische Reflektion
    \end{itemize}
\end{concept}

\multend

\subsubsection{Custom Physics}

\begin{concept}{Numerische Integrationsmethoden}

    Unity verwendet implizite Euler-Integration, aber benutzerdefinierte Implementierungen können verwenden:
    \begin{itemize}
        \item \textbf{Expliziter Euler}: Einfach, aber potentiell instabil
        \item \textbf{Verlet-Integration}: Bessere Energieerhaltung
        \item \textbf{Runge-Kutta}: Höhere Genauigkeit für komplexe Systeme
        \item \textbf{Leapfrog}: Gut für Orbitalprobleme
    \end{itemize}
    
    \textbf{Vergleich der Methoden:}
    \begin{itemize}
        \item \textbf{Stabilität}: Verlet > Runge-Kutta > Euler
        \item \textbf{Genauigkeit}: Runge-Kutta > Verlet > Euler
        \item \textbf{Performance}: Euler > Verlet > Runge-Kutta
        \item \textbf{Energieerhaltung}: Verlet > Runge-Kutta > Euler
    \end{itemize}
\end{concept}

\begin{concept}{Custom Physics Integration}

    \textbf{Integrationsmethoden:}
    \begin{itemize}
        \item \textbf{Euler}: $v += a \Delta t$, $x += v \Delta t$
        \item \textbf{Verlet}: $x_{new} = 2x - x_{old} + a \Delta t^2$
        \item \textbf{Runge-Kutta}: Höhere Genauigkeit, 4 Evaluationen
        \item Kraftberechnung: Gravitation + Feder + Dämpfung
        \item LineRenderer für Trajektorie
    \end{itemize}
\end{concept}

\subsubsection{Performance-Tipps}

\mult{2}

\begin{concept}{Physik-Performance-Tipps}

    \textbf{Kollisions-Optimierung:}
    \begin{itemize}
        \item Passende CollisionDetectionMode wählen
        \item Primitive Collider bevorzugen
        \item Layer-Matrix für Kollisions-Filtering
    \end{itemize}
    
    \textbf{Rigidbody-Optimierung:}
    \begin{itemize}
        \item Kinematic für statische Objekte
        \item Object Pooling für viele Objekte
        \item Fixed Timestep anpassen
    \end{itemize}
\end{concept}

\begin{concept}{Object Pooling}

    \textbf{Performance-Optimierung:}
    \begin{itemize}
        \item Queue<GameObject> für Pool-Verwaltung
        \item GetObject() / ReturnObject() API
        \item Automatisches Cleanup nach Zeit/Position
        \item Rigidbody-Zustand zurücksetzen (Sleep/WakeUp)
        \item Prefab-basierte Instanziierung
    \end{itemize}
\end{concept}

\multend

\raggedcolumns
\columnbreak

\subsection{Prüfungstipps}

\begin{KR}{Unity Physics Prüfungsstrategie}
    \paragraph{Kernphysik-Konzepte beherrschen}
    \begin{itemize}
        \item Newton'sche Gesetze und ihre Anwendungen
        \item Energie- und Impulserhaltung verstehen
        \item Rotationsdynamik-Berechnungen üben
        \item Wichtige Formeln und ihre Herleitungen kennen
        \item Vektormathematik und Koordinatensysteme
    \end{itemize}
    
    \paragraph{Unity-Implementierungs-Wissen}
    \begin{itemize}
        \item Rigidbody-Komponenten-Eigenschaften und -Methoden
        \item Kraftanwendungstechniken und Force-Modi
        \item Kollisionserkennung und -reaktion
        \item Koordinatensystem-Transformationen
        \item Physics Material und Reibungsmodelle
    \end{itemize}
    
    \paragraph{Problemlösungsansatz}
    \begin{itemize}
        \item Beteiligte physikalische Prinzipien identifizieren
        \item Koordinatensystem und Freikörperdiagramme aufstellen
        \item Erhaltungsgesetze anwenden wo angemessen
        \item In Unity-Implementierungskonzepte übersetzen
        \item Code-Snippets für häufige Szenarien memorieren
    \end{itemize}
    
    \paragraph{Häufige Prüfungsthemen}
    \begin{itemize}
        \item Projektilbewegung mit Unity-Vektoren
        \item Kollisionsanalyse und Impulserhaltung
        \item Rotationsbewegung und Drehmoment-Berechnungen
        \item Energieerhaltung in mechanischen Systemen
        \item Custom Physics Implementation
        \item Performance-Optimierung in Unity Physics
    \end{itemize}
\end{KR}
