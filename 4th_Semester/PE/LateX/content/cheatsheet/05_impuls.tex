\section{Impuls und Stoßgesetze}

\subsection{Impuls}

\begin{definition}{Impuls}
    Der Impuls $\vec{p}$ eines Körpers ist das Produkt aus seiner Masse und seiner Geschwindigkeit:
    \begin{equation}
        \vec{p} = m \cdot \vec{v}
    \end{equation}
    
    Einheit: kg·m/s (vektorielle Größe)
\end{definition}

\begin{concept}{Impulserhaltung}
    In einem abgeschlossenen System ohne äußere Kräfte bleibt der Gesamtimpuls konstant:
    \begin{equation}
        \sum \vec{p}_{\text{vorher}} = \sum \vec{p}_{\text{nachher}}
    \end{equation}
    
    Für zwei Körper:
    \begin{equation}
        m_1 \vec{v}_{1,\text{vorher}} + m_2 \vec{v}_{2,\text{vorher}} = m_1 \vec{v}_{1,\text{nachher}} + m_2 \vec{v}_{2,\text{nachher}}
    \end{equation}
    
    \textbf{Wichtig:} Impulserhaltung gilt auch bei dissipativen Vorgängen (z.B. inelastische Stöße).
\end{concept}

\begin{formula}{Kraftstoß}
    Der Kraftstoß $\vec{I}$ ist das Zeitintegral der Kraft:
    \begin{equation}
        \vec{I} = \int_{t_1}^{t_2} \vec{F}(t) \, dt = \Delta \vec{p}
    \end{equation}
    
    Für konstante Kraft:
    \begin{equation}
        \vec{I} = \vec{F} \cdot \Delta t = \vec{p}_{\text{nachher}} - \vec{p}_{\text{vorher}}
    \end{equation}
\end{formula}

\subsection{Stöße}

\begin{definition}{Elastischer Stoß}
    Sowohl Gesamtimpuls als auch Gesamtenergie bleiben erhalten:
    
    \textbf{Impulserhaltung:}
    \begin{equation}
        m_1 v_{1,\text{vorher}} + m_2 v_{2,\text{vorher}} = m_1 v_{1,\text{nachher}} + m_2 v_{2,\text{nachher}}
    \end{equation}
    
    \textbf{Energieerhaltung:}
    \begin{equation}
        \frac{1}{2}m_1 v_{1,\text{vorher}}^2 + \frac{1}{2}m_2 v_{2,\text{vorher}}^2 = \frac{1}{2}m_1 v_{1,\text{nachher}}^2 + \frac{1}{2}m_2 v_{2,\text{nachher}}^2
    \end{equation}
\end{definition}

\begin{definition}{Inelastischer Stoß}
    Nur der Gesamtimpuls bleibt erhalten, mechanische Energie wird teilweise in andere Formen umgewandelt.
    
    \textbf{Vollständig inelastischer Stoß:} Körper "kleben" nach dem Stoß zusammen.
    \begin{equation}
        v_{\text{nachher}} = \frac{m_1 v_{1,\text{vorher}} + m_2 v_{2,\text{vorher}}}{m_1 + m_2}
    \end{equation}
\end{definition}

\begin{KR}{Berechnung von Stößen in 1D}
    \paragraph{Elastischer Stoß}
    \textbf{Methode 1: Schwerpunktgeschwindigkeit}
    \begin{enumerate}
        \item Schwerpunktgeschwindigkeit:
        \begin{equation}
            v_{SP} = \frac{m_1 v_1 + m_2 v_2}{m_1 + m_2}
        \end{equation}
        
        \item Geschwindigkeiten nach dem Stoß (Spiegelung):
        \begin{align}
            u_1 &= 2v_{SP} - v_1 \\
            u_2 &= 2v_{SP} - v_2
        \end{align}
    \end{enumerate}
    
    \textbf{Methode 2: Direkte Formeln}
    \begin{align}
        u_1 &= \frac{(m_1 - m_2)v_1 + 2m_2 v_2}{m_1 + m_2} \\
        u_2 &= \frac{2m_1 v_1 + (m_2 - m_1)v_2}{m_1 + m_2}
    \end{align}
    
    \paragraph{Vollständig inelastischer Stoß}
    \begin{equation}
        u_1 = u_2 = \frac{m_1 v_1 + m_2 v_2}{m_1 + m_2}
    \end{equation}
\end{KR}

\begin{example2}{Elastischer Stoß mit ungleichen Massen}
    \textbf{Leichte Kugel trifft schwere ruhende Kugel} ($m_1 \ll m_2$, $v_2 = 0$):
    \begin{align}
        u_1 &\approx -v_1 \quad \text{(Richtungsumkehr)} \\
        u_2 &\approx 0 \quad \text{(schwerer Körper bleibt fast in Ruhe)}
    \end{align}
    
    \textbf{Schwere Kugel trifft leichte ruhende Kugel} ($m_1 \gg m_2$, $v_2 = 0$):
    \begin{align}
        u_1 &\approx v_1 \quad \text{(fast unverändert)} \\
        u_2 &\approx 2v_1 \quad \text{(doppelte Geschwindigkeit)}
    \end{align}
\end{example2}

\begin{concept}{Impuls- vs. Energieerhaltung}
    \begin{itemize}
        \item \textbf{Impulserhaltung}: Gilt immer ohne äußere Kräfte (Grundprinzip)
        \item \textbf{Energieerhaltung}: Nur für konservative Vorgänge
        \item \textbf{Elastische Stöße}: Beide Erhaltungssätze
        \item \textbf{Inelastische Stöße}: Nur Impulserhaltung
        \item \textbf{Impuls}: Vektorgleichungen (Richtungen wichtig)
        \item \textbf{Energie}: Skalargleichung (nur Beträge)
    \end{itemize}
\end{concept}

\subsection{Anwendungen}

\begin{example2}{Ballistisches Pendel}
    Gerät zur Messung von Projektilgeschwindigkeiten durch vollständig inelastischen Stoß.
    
    \textbf{Gegeben:} Projektilmasse $m_K$, Pendelmasse $m_P$, Auslenkung $x$, Pendellänge $L$
    
    \textbf{Projektilgeschwindigkeit:}
    \begin{equation}
        v_K = \frac{m_P + m_K}{m_K} \cdot \sqrt{2g \cdot (L - \sqrt{L^2 - x^2})}
    \end{equation}
    
    \textbf{Zweistufiger Prozess:}
    \begin{enumerate}
        \item Inelastischer Stoß: Impulserhaltung
        \item Pendelschwingung: Energieerhaltung
    \end{enumerate}
\end{example2}

\begin{example2}{Raketenantrieb}
    Basiert auf Rückstoßprinzip und Impulserhaltung.
    
    \textbf{Ziolkowski-Gleichung (maximale Geschwindigkeit):}
    \begin{equation}
        v_{\text{max}} = v_{\text{rel}} \cdot \ln\frac{m_0}{m_{\text{leer}}}
    \end{equation}
    
    \begin{itemize}
        \item $v_{\text{rel}}$: Austrittsgeschwindigkeit des Treibstoffs
        \item $m_0$: Anfangsmasse (Rakete + Treibstoff)
        \item $m_{\text{leer}}$: Leermasse der Rakete
    \end{itemize}
\end{example2}

\subsection{Unity Implementation}

\begin{concept}{Unity Stoß-Simulation}
    \textbf{Elastischer Stoß:}
    \begin{itemize}
        \item OnCollisionEnter() für Kollisionserkennung
        \item Stoßnormale aus Positionsdifferenz
        \item Geschwindigkeiten in normal/tangential zerlegen
        \item Elastische Stoßformeln anwenden
        \item Neue Geschwindigkeiten setzen
    \end{itemize}
\end{concept}

\begin{concept}{Wichtige Erkenntnisse}
    \begin{itemize}
        \item Impulserhaltung ist universell gültig
        \item Kollisionsberechnung erfolgt komponentenweise
        \item Tangentiale Geschwindigkeiten bleiben bei Stößen unverändert
        \item Energieverluste nur bei inelastischen Stößen
    \end{itemize}
\end{concept}