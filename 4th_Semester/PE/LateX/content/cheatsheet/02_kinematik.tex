\section{Kinematik}

\begin{definition}{Kinematik}
    Die Kinematik beschreibt die Bewegung ohne Betrachtung der Ursachen. Eine Bewegung wird vollständig charakterisiert durch:
    \begin{itemize}
        \item Ort: $\vec{r}(t)$
        \item Geschwindigkeit: $\vec{v}(t) = \frac{d\vec{r}}{dt}$
        \item Beschleunigung: $\vec{a}(t) = \frac{d\vec{v}}{dt} = \frac{d^2\vec{r}}{dt^2}$
    \end{itemize}
\end{definition}

\begin{formula}{Zusammenhänge zwischen Ort\, Geschwindigkeit und Beschleunigung}
    \begin{itemize}
        \item Geschwindigkeit = Ableitung des Ortes nach der Zeit: $\vec{v} = \frac{d\vec{r}}{dt}$
        \item Beschleunigung = Ableitung der Geschwindigkeit nach der Zeit: $\vec{a} = \frac{d\vec{v}}{dt}$
        \item Ort = Integral der Geschwindigkeit nach der Zeit: $\vec{r} = \int \vec{v} \, dt$
        \item Geschwindigkeit = Integral der Beschleunigung nach der Zeit: $\vec{v} = \int \vec{a} \, dt$
    \end{itemize}
\end{formula}

\subsection{Geschwindigkeit und Beschleunigung}

\mult{2}

\begin{definition}{Mittlere vs. Momentangeschwindigkeit}\\
    \textbf{Mittlere Geschwindigkeit:}
    $$
        \bar{v}_x = \frac{\Delta r_x}{\Delta t} = \frac{r_x(t_2) - r_x(t_1)}{t_2 - t_1}
    $$
    
    \textbf{Momentangeschwindigkeit:}
    $$
        v_x(t) = \lim_{\Delta t \to 0} \frac{\Delta r_x}{\Delta t} = \frac{dr_x}{dt}
    $$
    
    
\end{definition}

\begin{definition}{Mittlere vs. Momentanbeschleunigung}\\
    \textbf{Mittlere Beschleunigung:}
    $$
        \bar{a}_x = \frac{\Delta v_x}{\Delta t} = \frac{v_x(t_2) - v_x(t_1)}{t_2 - t_1}
    $$
    
    \textbf{Momentanbeschleunigung:}
    $$
        a_x(t) = \lim_{\Delta t \to 0} \frac{\Delta v_x}{\Delta t} = \frac{dv_x}{dt} = \frac{d^2r_x}{dt^2}
    $$
\end{definition}

\multend

\begin{concept}{Unterschied Geschwindigkeit und Schnelligkeit:}
    \begin{itemize}
        \item Geschwindigkeit: Vektorielle Größe mit Richtung
        \item Schnelligkeit: Betrag der Geschwindigkeit (Skalar)
        \item $|\vec{v}| = \sqrt{v_x^2 + v_y^2 + v_z^2}$
    \end{itemize}
\end{concept}

\begin{theorem}{Differenzenquotient vs. Differentialquotient}
    \begin{itemize}
        \item Differenzenquotient (mittlere Geschwindigkeit): Approximation über ein endliches Zeitintervall: $\frac{\Delta r_x}{\Delta t}$
        \item Differentialquotient (Momentangeschwindigkeit): \\Grenzwert für ein infinitesimal kleines Zeitintervall: $\lim_{\Delta t \to 0} \frac{\Delta r_x}{\Delta t} = \frac{dr_x}{dt}$
        \item In Unity wird mit fixen Zeitschritten $\Delta t = 20$ ms gerechnet\\ $\rightarrow$ entspricht einer Abtastfrequenz von $f_{sample} = 50$ Hz 
    \end{itemize}
\end{theorem}

\subsubsection{Momentangeschwindigkeit und -beschleunigung}
\begin{definition}{Momentangeschwindigkeit}
    Die Momentangeschwindigkeit zur Zeit $t_0$ ist definiert als:
    $$
        \vec{v}(t_0) = \lim_{t_1 \to t_0} \frac{\Delta \vec{r}}{t_1 - t_0} = \frac{d\vec{r}}{dt}
    $$
    Sie entspricht geometrisch der Steigung der Tangente im Punkt $(t_0, r_x(t_0))$.
\end{definition}

\begin{remark}
    Der Betrag der Geschwindigkeit wird oft als Schnelligkeit bezeichnet:
    $$
        |\vec{v}| = \sqrt{v_x^2 + v_y^2 + v_z^2}
    $$
    Bei gleichbleibender Schnelligkeit kann sich dennoch die Richtung der Geschwindigkeit ändern, z.B. bei einer Kreisbewegung.
\end{remark}

\begin{formula}{Fläche unter dem Geschwindigkeits-Zeit-Diagramm}
    Bei einer Bewegung mit variablem $v(t)$ berechnet sich die zurückgelegte Strecke als Fläche unter der $v$-$t$-Kurve:
    $$
        \Delta x = \int_{t_1}^{t_2} v(t) \, dt
    $$
\end{formula}

\subsection{Integration und Differentiation}

\mult{2}

\begin{formula}{Ableitungsregeln}
    \begin{itemize}
        \item Konstante Summanden: $\frac{d}{dt}(C) = 0$
        \item Potenzfunktionen: $\frac{d}{dt}(at^n) = a \cdot n \cdot t^{n-1}$
        \item Exponentialfunktion: $\frac{d}{dx}(e^x) = e^x$
        \item Logarithmus: $\frac{d}{dx}(\ln x) = \frac{1}{x}$
    \end{itemize}
\end{formula}

\begin{formula}{Regeln für zusammengesetzte Funktionen}
    \begin{itemize}
        \item Summenregel: $\frac{d}{dt}(f(t) + g(t)) = \frac{df}{dt} + \frac{dg}{dt}$
        \item Produktregel: $\frac{d}{dt}(f(t) \cdot g(t)) = \frac{df}{dt} \cdot g(t) + f(t) \cdot \frac{dg}{dt}$
        \item Kettenregel: $\frac{d}{dt}(f(g(t))) = \frac{df}{dg} \cdot \frac{dg}{dt}$
    \end{itemize}
\end{formula}

\multend
\begin{formula}{Sinus/Kosinus}
     $\frac{d}{dx}(\sin x) = \cos x$, $\frac{d}{dx}(\cos x) = -\sin x$
\end{formula}

\begin{KR}{Berechnung von Bewegungen mit konstanter Beschleunigung}
    \paragraph{Gegebene Größen}
    Anfangsposition $r_0$, Anfangsgeschwindigkeit $v_0$, konstante Beschleunigung $a$
    
    \paragraph{Schritte zur Berechnung}
    \begin{enumerate}
        \item Geschwindigkeit in Abhängigkeit von der Zeit bestimmen:
        $
            v(t) = v_0 + at
        $
        
        \item Position in Abhängigkeit von der Zeit bestimmen:
        $
            r(t) = r_0 + v_0t + \frac{1}{2}at^2
        $
        
        \item Alternative Formel bei bekannter Strecke (ohne Zeit):
        $
            v^2 = v_0^2 + 2a(r - r_0)
        $
    \end{enumerate}
\end{KR}




\begin{example} \textbf{Freier Fall}

    \begin{minipage}{0.5\linewidth}
    Ein Körper fällt aus der Höhe $r_0$\\ mit Anfangsgeschwindigkeit $v_0 = 0$.
    
    \begin{itemize}
        \item Beschleunigung: $a(t) = -g$ (g = 9.81 m/s²)
        \item Geschwindigkeit: $v(t) = -gt$
        \item Position: $r(t) = r_0 - \frac{1}{2}gt^2$
    \end{itemize}
    \end{minipage}
    \begin{minipage}{0.5\linewidth}    
    Alternativ: Ein Körper wird mit Anfangsgeschwindigkeit $v_0$ nach oben geworfen:
    \begin{itemize}
        \item Maximale Höhe: $h_{max} = \frac{v_0^2}{2g}$ 
        \item Zeit bis zum höchsten Punkt: $t_{max} = \frac{v_0}{g}$
        \item Gesamtflugzeit: $t_{gesamt} = \frac{2v_0}{g}$
    \end{itemize}
    \end{minipage}
\end{example}


\subsubsection{Unity-Implementation}

\mult{2}

\begin{concept}{Unity Position/Geschwindigkeit}\\
    \textbf{Grundoperationen:}
    \begin{itemize}
        \item transform.position für Ortsvektor
        \item Verschiebung als Vektordifferenz
        \item Position mit velocity * Time.deltaTime aktualisieren
    \end{itemize}
\end{concept}

\begin{concept}{Kinematik-Script Struktur}\\
    \textbf{Wichtige Komponenten:}
    \begin{itemize}
        \item Anfangswerte (Position, Geschwindigkeit) speichern
        \item Zeitvariable t = Time.time - startTime
        \item Kinematische Gleichung in Update() anwenden
        \item transform.position direkt setzen
    \end{itemize}
\end{concept}

\begin{concept}{Spezialfälle der Bewegung}\\
    \textbf{Gleichförmige Bewegung} ($\vec{a} = 0$):
    $$\vec{r}(t) = \vec{r}_0 + \vec{v}t$$
    
    \textbf{Freier Fall in Unity} ($\vec{a} = -g\hat{j}$):
    \begin{itemize}
        \item $g = 9.81 \, m/s^2$ (Erdbeschleunigung)
        \item Unity Standard-Gravitation: $-9.81 \, m/s^2$ in Y-Richtung
    \end{itemize}
\end{concept}

\begin{concept}{Unity Kinematik}\\
    \textbf{Implementierung konstanter Beschleunigung:}
    \begin{itemize}
        \item Geschwindigkeit: $v = v_0 + at$ mit Time.deltaTime
        \item Position direkt: $r = r_0 + v_0t + \frac{1}{2}at^2$
        \item FixedUpdate() für Physikberechnungen verwenden
    \end{itemize}
\end{concept}

\begin{KR}{Kinematische Probleme lösen}
    \paragraph{Schritt 1: Bekannte Variablen identifizieren}
    \begin{itemize}
        \item Anfangsposition $\vec{r}_0$
        \item Anfangsgeschwindigkeit $\vec{v}_0$
        \item Beschleunigung $\vec{a}$
        \item Zeit $t$ oder End-Position/Geschwindigkeit
    \end{itemize}
    
    \paragraph{Schritt 2: Passende Gleichung wählen}
    \begin{itemize}
        \item $\vec{v}(t) = \vec{v}_0 + \vec{a}t$ wenn Zeit bekannt ist
        \item $\vec{r}(t) = \vec{r}_0 + \vec{v}_0 t + \frac{1}{2}\vec{a}t^2$ für Position
        \item $\vec{v}^2 = \vec{v}_0^2 + 2\vec{a} \cdot \Delta\vec{r}$ wenn Zeit unbekannt ist
    \end{itemize}
    
    \paragraph{Schritt 3: Komponentenweise lösen}
    \begin{itemize}
        \item Vektoren in x-, y-, z-Komponenten aufteilen
        \item Jede Komponente unabhängig lösen
        \item Ergebnisse zum finalen Vektor kombinieren
    \end{itemize}
\end{KR}

\multend

\begin{example} \textbf{Wurfbewegung}:
    Ein Ball wird aus Höhe $h = 10m$ mit Anfangsgeschwindigkeit $\vec{v}_0 = (5, 8, 0) \, m/s$ geworfen. Berechne Zeit bis zum Aufprall und horizontale Distanz.
    
    \textbf{Gegeben:} $\vec{r}_0 = (0, 10, 0)$, $\vec{v}_0 = (5, 8, 0)$, $\vec{a} = (0, -9.81, 0)$
    
    \textbf{Y-Komponente (vertikal):}
    $y(t) = 10 + 8t - 4.905t^2$
    
    Ball trifft Boden bei $y(t) = 0$:
    $10 + 8t - 4.905t^2 = 0$
    $t = 2.24s$ (mit quadratischer Formel)
    
    \textbf{X-Komponente (horizontal):}
    $x(t) = 5t = 5 \times 2.24 = 11.2m$
\end{example}

