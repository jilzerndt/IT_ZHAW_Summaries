\section{Kräfte}

\subsubsection{Grundlegende Wechselwirkungen}

\begin{concept}{Vier fundamentale Kräfte}
    In der Physik gibt es vier fundamentale Wechselwirkungen:
    \begin{enumerate}
        \item \textbf{Gravitation}: Anziehung zwischen Massen
        \item \textbf{Elektromagnetische Kraft}: Kräfte zwischen elektrischen Ladungen
        \item \textbf{Starke Kernkraft}: hält den Atomkern zusammen
        \item \textbf{Schwache Kernkraft}: verantwortlich für radioaktiven Beta-Zerfall
    \end{enumerate}
    
    Für die makroskopische Mechanik sind hauptsächlich Gravitation und elektromagnetische Kräfte relevant.
\end{concept}

\mult{2}

\begin{definition}{Kraft}
    Eine Kraft ist ein Einfluss, der den Bewegungszustand eines Körpers ändert.
    $$
        \vec{F} = \frac{d\vec{p}}{dt}
    $$
    
    Einheit: Newton (N) = $1 \, \frac{\text{kg} \cdot \text{m}}{\text{s}^2}$
\end{definition}


\subsubsection{Gravitationskraft}

\begin{formula}{Newton'sches Gravitationsgesetz}\\
    Anziehungskraft zwischen zwei Massen $m_1$ und $m_2$ im Abstand $R$:
    $$
        F_G = G \cdot \frac{m_1 \cdot m_2}{R^2}
    $$
    
    Gravitationskonstante: $G = 6.67 \cdot 10^{-11} \, \frac{\text{m}^3}{\text{kg} \cdot \text{s}^2}$
\end{formula}

\begin{example2}{Erdbeschleunigung}
    Gewichtskraft eines Körpers der Masse $m$ auf der Erdoberfläche:
    $$
        F_G = m \cdot g
    $$
    
    Erdbeschleunigung:
    $
        g = G \cdot \frac{M_{\text{Erde}}}{R_{\text{Erde}}^2} \approx 9.81 \, \frac{\text{m}}{\text{s}^2}
    $
    \vspace{1mm}\\
    
    Mit zunehmender Höhe $h$:
    $$
        g(h) = G \cdot \frac{M_{\text{Erde}}}{(R_{\text{Erde}} + h)^2}
    $$
\end{example2}

\subsubsection{Federkraft}

\begin{definition}{Hooke'sches Gesetz}
    Für eine lineare Feder:
    $$
        \vec{F} = -k \cdot \vec{x}
    $$
    
    \begin{itemize}
        \item $k$: Federkonstante (N/m)
        \item $\vec{x}$: Auslenkung aus der Ruhelage
        \item Negatives Vorzeichen: Kraft wirkt der Auslenkung entgegen
    \end{itemize}
\end{definition}

\begin{formula}{Spannenergie einer Feder}\\
    Potentielle Energie einer gespannten Feder:
    $$
        E_{\text{spann}} = \frac{1}{2} \cdot k \cdot x^2
    $$
\end{formula}

\begin{concept}{Harmonischer Oszillator}
    System mit rücktreibender Kraft proportional zur Auslenkung.
    
    \textbf{Bewegungsgleichung:}
    $$
        \frac{d^2 x}{dt^2} = -\frac{k}{m} \cdot x
    $$
    
    \textbf{Lösung:}
    $$
        x(t) = x_0 \cdot \cos(\omega t) \quad \text{mit } \omega = \sqrt{\frac{k}{m}}
    $$
    
    \textbf{Schwingungsdauer:} $T = \frac{2\pi}{\omega}$
\end{concept}

\multend

\subsubsection{Reibungskräfte}

\begin{definition}{Arten der Reibung}\\
    \textbf{Äußere Reibung} (Kontaktflächen von Festkörpern): Haftreibung, Gleitreibung, Rollreibung, Wälzreibung, Bohrreibung, Seilreibung

    \textbf{Innere Reibung}: zwischen benachbarten Teilchen bei Verformungen
\end{definition}

\mult{2}

\begin{formula}{Trockene Reibung (Coulomb-Reibung)}
    $$
        \vec{F}_R = \mu \cdot \vec{F}_N
    $$
    
    \begin{itemize}
        \item $\mu$: Reibungskoeffizient (dimensionslos)
        \item $\vec{F}_N$: Normalkraft
        \item Richtung: entgegengesetzt zur Bewegungsrichtung
    \end{itemize}
    
    \textbf{Unterscheidung:}
    \begin{itemize}
        \item Haftreibung: $\vec{F}_{\text{Haft}} \leq \mu_{\text{Haft}} \cdot \vec{F}_N$
        \item Gleitreibung: $\vec{F}_{\text{Gleit}} = \mu_{\text{Gleit}} \cdot \vec{F}_N$
        \item Regel: $\mu_{\text{Haft}} > \mu_{\text{Gleit}}$
    \end{itemize}
\end{formula}

\begin{formula}{Viskose Reibung}
    \textbf{Laminare Strömung \\ (Stokes'sche Reibung für Kugel):}
    $$
        \vec{F}_R = -6 \cdot \pi \cdot \eta \cdot r \cdot v \cdot \vec{e}_v
    $$
    \small ($\eta$: Viskosität des Mediums)
    \vspace{2mm}\\
    \normalsize
    \textbf{Turbulente Strömung:}
    $$
        \vec{F}_R = -\frac{1}{2} \cdot \rho \cdot A \cdot c_w \cdot \vec{v}^2 \cdot \vec{e}_v
    $$
    \small
    ($\rho$: Dichte, $A$: Stirnfläche, $c_w$: Widerstandsbeiwert)
\end{formula}

\multend

\subsection{Unity Implementation}

\mult{2}

\begin{concept}{Unity Reibung}
    \textbf{Implementierung:}
    \begin{itemize}
        \item Normalkraft: $F_N = mg$ (horizontal)
        \item Reibung: $F_R = \mu F_N$ entgegen Bewegung
        \item Geschwindigkeitscheck für Stillstand
        \item AddForce() in entgegengesetzte Richtung
    \end{itemize}
\end{concept}

\begin{concept}{Unity Luftwiderstand}
    \textbf{Turbulente Strömung:}
    \begin{itemize}
        \item $F_R = \frac{1}{2}\rho A c_w v^2$ entgegen Bewegung
        \item Quadratische Abhängigkeit von Geschwindigkeit
        \item velocity.normalized für Richtung
        \item Material-spezifische Parameter
    \end{itemize}
\end{concept}

\multend

\subsection{Trägheitskräfte}

\begin{concept}{Beschleunigte Bezugssysteme}
    In beschleunigten Bezugssystemen treten Trägheitskräfte (Scheinkräfte) auf:
    
    \textbf{Translatorische Trägheitskraft:}
    $
        \vec{F}_{\text{Trägheit}} = -m \cdot \vec{a}_{\text{System}}
    $

    \textbf{Zentrifugalkraft bei Rotation:}
    $
        \vec{F}_{\text{Zentrifugal}} = m \cdot \omega^2 \cdot r \cdot \vec{e}_r
    $
    
    \textbf{Corioliskraft bei Bewegung in rotierendem System:}
    $
        \vec{F}_{\text{Coriolis}} = 2 \cdot m \cdot \vec{v} \times \vec{\omega}
    $
\end{concept}

\begin{example}{Trägheitskräfte im Alltag}
    \begin{itemize}
        \item Im bremsenden Zug: nach vorne gedrückt fühlen
        \item In der Kurve: nach außen gedrückt (Zentrifugalkraft)
        \item Corioliskraft: Ablenkung von Winden und Meeresströmungen
        \begin{itemize}
            \item Nordhalbkugel: Ablenkung nach rechts
            \item Südhalbkugel: Ablenkung nach links
        \end{itemize}
    \end{itemize}
\end{example}

\begin{remark}
    Trägheitskräfte sind keine "echten" Kräfte im Sinne von Wechselwirkungen zwischen Körpern, sondern entstehen durch die Wahl des Bezugssystems. In einem Inertialsystem existieren sie nicht.
\end{remark}