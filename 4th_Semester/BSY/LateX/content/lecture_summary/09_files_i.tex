\section{File Systems I}

\subsection{Introduction to File Systems}

\begin{concept}{File System Requirements}\\
    File systems address three essential requirements for long-term information storage:
    \begin{itemize}
        \item It must be possible to store very large amounts of information
        \item Information must survive termination of processes using it
        \item Multiple processes must be able to access information concurrently
    \end{itemize}
    
    File systems organize, store, access, and manage persistent data by:
    \begin{itemize}
        \item Supporting multiple storage media
        \item Providing fast data location and protection mechanisms
        \item Enabling efficient read and write operations on fixed-size blocks
    \end{itemize}
\end{concept}

\begin{definition}{File System Architecture}\\
    File systems consist of two main organizational levels:
    \begin{itemize}
        \item \textbf{Logical organization} (user perspective):
            \begin{itemize}
                \item Hierarchical directory structure
                \item File abstractions with names and attributes
                \item Access control and permissions
            \end{itemize}
        \item \textbf{Physical organization} (OS perspective):
            \begin{itemize}
                \item Block allocation and management
                \item Data structure layout on storage media
                \item Device driver interfaces
            \end{itemize}
    \end{itemize}
\end{definition}

\subsection{Files - User Perspective}

\begin{definition}{File Concept and Structure}\\
    A file is an abstraction with multiple perspectives:
    \begin{itemize}
        \item \textbf{Logical perspective}: Abstract object with a name and access methods for storing data
        \item \textbf{Physical perspective}: Set of equivalent records or bytes on persistent media
        \item Files are created and accessed through processes
        \item Files provide persistent storage beyond process lifetime
    \end{itemize}
    
    File naming conventions:
    \begin{itemize}
        \item File names are strings with system-specific rules
        \item Many systems support multi-part names separated by periods
        \item File extension typically follows the last period
        \item Extensions often indicate file type or intended application
    \end{itemize}
\end{definition}

\begin{definition}{File Types}\\
    Operating systems recognize several types of files:
    \begin{itemize}
        \item \textbf{Ordinary/Regular Files}: User files containing programs and data
        \item \textbf{Directory Files}: System files containing filesystem structure information
        \item \textbf{Character Special Files}: Represent character-oriented I/O devices
        \item \textbf{Block Special Files}: Represent block-oriented I/O devices
        \item \textbf{Link Files}: References to other files in the filesystem
        \item \textbf{Socket Files}: Communication endpoints for network or IPC
        \item \textbf{Named Pipe Files}: Communication channels between processes
    \end{itemize}
\end{definition}

\begin{definition}{File Access Methods}\\
    Files can be accessed in different ways:
    \begin{itemize}
        \item \textbf{Sequential Access}: Data read/written in order from beginning to end
            \begin{itemize}
                \item Common for streaming data, logs, backups
                \item Simple to implement and efficient for linear processing
            \end{itemize}
        \item \textbf{Random Access}: Data can be read/written at any position
            \begin{itemize}
                \item Essential for databases, binary files, memory-mapped files
                \item Requires address calculation and seek operations
            \end{itemize}
    \end{itemize}
\end{definition}

\begin{definition}{File Attributes and Permissions}\\
    Files have various attributes and access rights:
    \begin{itemize}
        \item \textbf{Basic attributes}:
            \begin{itemize}
                \item Name: Human-readable identifier
                \item Size: Number of bytes in the file
                \item Timestamps: Creation, modification, access times
                \item Owner: User and group ownership
                \item Type: Regular, directory, device, etc.
            \end{itemize}
        \item \textbf{Unix/Linux permissions}:
            \begin{itemize}
                \item Read (r): Ability to read file contents
                \item Write (w): Ability to modify file contents
                \item Execute (x): Ability to execute file or search directory
                \item Applied to three categories: owner, group, others
            \end{itemize}
        \item \textbf{Links}: Number of hard links pointing to the file
    \end{itemize}
\end{definition}

\begin{definition}{File Operations}\\
    Common file operations supported by operating systems:
    \begin{itemize}
        \item \textbf{Creation and deletion}: Create new files, delete existing files
        \item \textbf{Opening and closing}: Establish/terminate access to files
        \item \textbf{Reading and writing}: Transfer data to/from files
        \item \textbf{Seeking}: Change current position in file for random access
        \item \textbf{Attribute manipulation}: Get and set file metadata
        \item \textbf{Renaming}: Change file names
        \item \textbf{Linking}: Create references to existing files
    \end{itemize}
\end{definition}

\begin{example}
    Basic file operations in C:
    
\begin{lstlisting}[language=C, style=basesmol]
#include <stdio.h>
#include <fcntl.h>
#include <unistd.h>

int main() {
    int fd;
    char buffer[100];
    ssize_t bytes_read;
    
    // Open file for reading
    fd = open("example.txt", O_RDONLY);
    if (fd == -1) {
        perror("Error opening file");
        return 1;
    }
    
    // Read from file
    bytes_read = read(fd, buffer, sizeof(buffer) - 1);
    if (bytes_read > 0) {
        buffer[bytes_read] = '\0';
        printf("Read: %s\n", buffer);
    }
    
    // Close file
    close(fd);
    
    return 0;
}
\end{lstlisting}
\end{example}

\subsection{Directory Organization}

\begin{definition}{File Organization on Storage}\\
    Files are organized on storage devices through:
    \begin{itemize}
        \item \textbf{Volumes}: Logical storage units (can span multiple physical devices)
        \item \textbf{Partitions}: Subdivisions of physical storage devices
        \item \textbf{Directories}: Hierarchical organization of files and subdirectories
    \end{itemize}
    
    Two main organizational approaches:
    \begin{itemize}
        \item \textbf{Physical volume with partitions}: Single device divided into sections
        \item \textbf{Logical volume with multiple disks}: Multiple devices combined into one logical unit
    \end{itemize}
\end{definition}

\begin{definition}{Directory Systems}\\
    Directory systems range from simple to complex:
    \begin{itemize}
        \item \textbf{Flat file system}:
            \begin{itemize}
                \item Only one root directory containing all files
                \item Used in embedded devices, simple systems
                \item No subdirectories or hierarchical organization
            \end{itemize}
        \item \textbf{Hierarchical file system}:
            \begin{itemize}
                \item Root directory and subdirectories
                \item Allows grouping of related files
                \item Enables multiple files with the same name in different directories
                \item Supports different user areas and system organization
            \end{itemize}
    \end{itemize}
\end{definition}

\begin{definition}{Directory Navigation}\\
    Hierarchical systems use various directory concepts:
    \begin{itemize}
        \item \textbf{Root directory}: Top-level directory (/ in Unix/Linux)
        \item \textbf{Home directory}: User's personal directory (\textasciitilde{} in Unix/Linux)
        \item \textbf{Working directory}: Current directory (. in Unix/Linux)
        \item \textbf{Parent directory}: Directory containing current directory (.. in Unix/Linux)
        \item \textbf{Absolute path}: Complete path from root (e.g., /usr/local/bin/program)
        \item \textbf{Relative path}: Path from current directory (e.g., ../docs/readme.txt)
    \end{itemize}
\end{definition}

\begin{definition}{Directory Operations}\\
    Common directory operations:
    \begin{itemize}
        \item Create and delete directories
        \item Open and close directories for reading
        \item Read directory entries
        \item Change current working directory
        \item Rename directories
        \item Create and remove links between directories and files
    \end{itemize}
\end{definition}

\begin{definition}{Shared Files and Linking}\\
    File systems support file sharing through linking:
    \begin{itemize}
        \item \textbf{Hard links}:
            \begin{itemize}
                \item Directory entries that point to the same inode
                \item Multiple names for the same file
                \item Cannot span filesystem boundaries
                \item Cannot link to directories (usually)
                \item File persists until all hard links are removed
            \end{itemize}
        \item \textbf{Symbolic (soft) links}:
            \begin{itemize}
                \item Special files containing paths to other files
                \item Can span filesystem boundaries
                \item Can link to directories
                \item Become invalid if target is moved or deleted
                \item More flexible but with additional indirection overhead
            \end{itemize}
    \end{itemize}
\end{definition}

\begin{KR}{Working with Files and Directories in Linux}\\
    \paragraph{Basic file operations}
    \begin{itemize}
        \item Create empty file: \texttt{touch filename}
        \item Copy files: \texttt{cp source destination}
        \item Move/rename files: \texttt{mv oldname newname}
        \item Delete files: \texttt{rm filename}
        \item View file contents: \texttt{cat filename}, \texttt{less filename}
        \item Edit files: \texttt{nano filename}, \texttt{vim filename}
    \end{itemize}
    
    \paragraph{Directory operations}
    \begin{itemize}
        \item Show current directory: \texttt{pwd}
        \item Change directory: \texttt{cd /path/to/directory}
        \item List directory contents: \texttt{ls -la}
        \item Create directory: \texttt{mkdir dirname}
        \item Remove empty directory: \texttt{rmdir dirname}
        \item Remove directory and contents: \texttt{rm -rf dirname}
        \item Display directory tree: \texttt{tree dirname}
    \end{itemize}
    
    \paragraph{File permissions and ownership}
    \begin{itemize}
        \item View permissions: \texttt{ls -l filename}
        \item Change permissions: \texttt{chmod 755 filename}
        \item Change ownership: \texttt{chown user:group filename}
        \item Change group: \texttt{chgrp group filename}
    \end{itemize}
    
    \paragraph{Linking}
    \begin{itemize}
        \item Create hard link: \texttt{ln source linkname}
        \item Create symbolic link: \texttt{ln -s source linkname}
        \item Find hard links: \texttt{find /path -samefile filename}
        \item Show link details: \texttt{ls -l linkname}
    \end{itemize}
\end{KR}

\subsection{File System Implementation}

\begin{definition}{File System Layout}\\
    File systems are typically organized on storage devices with the following layout:
    \begin{itemize}
        \item \textbf{Boot block}: Contains boot code for loading the operating system
        \item \textbf{Superblock}: Contains metadata about the filesystem
            \begin{itemize}
                \item Filesystem type and version
                \item Block size and total number of blocks
                \item Number of inodes and free space information
                \item Pointers to key data structures
            \end{itemize}
        \item \textbf{Inode table}: Contains file metadata and allocation information
        \item \textbf{Data blocks}: Contain actual file data and directory contents
    \end{itemize}
\end{definition}

\subsection{File Allocation Methods}

\begin{definition}{Contiguous Allocation}\\
    Files are stored in contiguous blocks on the storage device:
    \begin{itemize}
        \item \textbf{Advantages}:
            \begin{itemize}
                \item Simple to implement
                \item Excellent read performance (no seeking)
                \item Minimal metadata required (start block + length)
            \end{itemize}
        \item \textbf{Disadvantages}:
            \begin{itemize}
                \item External fragmentation over time
                \item Difficult to grow files (may require moving entire file)
                \item Need to know maximum file size in advance
            \end{itemize}
        \item \textbf{Use cases}: Read-only media, multimedia files
    \end{itemize}
\end{definition}

\begin{definition}{Linked-List Allocation}\\
    Files stored as linked lists of disk blocks:
    \begin{itemize}
        \item Each block contains a pointer to the next block
        \item Directory entry contains pointer to first block
        \item \textbf{Advantages}:
            \begin{itemize}
                \item No external fragmentation
                \item Files can grow dynamically
                \item Only need to store address of first block
            \end{itemize}
        \item \textbf{Disadvantages}:
            \begin{itemize}
                \item Very slow random access (must traverse links)
                \item Storage overhead for pointers
                \item Vulnerability to pointer corruption
            \end{itemize}
    \end{itemize}
\end{definition}

\begin{definition}{Indexed Allocation (Inodes)}\\
    Files use index structures to track data blocks:
    \begin{itemize}
        \item Each file has an index node (inode) containing metadata and block addresses
        \item Inode structure typically includes:
            \begin{itemize}
                \item File attributes (size, permissions, timestamps)
                \item Direct pointers to data blocks
                \item Indirect pointers for large files
                \item Double and triple indirect pointers for very large files
            \end{itemize}
        \item \textbf{Advantages}:
            \begin{itemize}
                \item Fast random access
                \item No external fragmentation
                \item Efficient for small and large files
            \end{itemize}
        \item \textbf{Disadvantages}:
            \begin{itemize}
                \item Additional storage overhead for inodes
                \item Complex management of indirect blocks
            \end{itemize}
    \end{itemize}
\end{definition}

\subsection{Directory Implementation}

\begin{definition}{Directory Storage}\\
    Directories map file names to file location information:
    \begin{itemize}
        \item Directory entries contain:
            \begin{itemize}
                \item File name
                \item Inode number (in Unix-like systems)
                \item File type indicator
                \item Optional cached attributes
            \end{itemize}
        \item Two main approaches for storing file attributes:
            \begin{itemize}
                \item In directory entry (simpler, but larger directories)
                \item In separate inode structure (more efficient, used by Unix/Linux)
            \end{itemize}
    \end{itemize}
\end{definition}

\begin{definition}{File Name Length Handling}\\
    Different approaches to variable-length file names:
    \begin{itemize}
        \item \textbf{Fixed-length names}: Simple but wastes space or limits name length
        \item \textbf{Variable-length inline}: Names stored directly in directory entries
        \item \textbf{Variable-length with pointers}: Directory entries contain pointers to names stored elsewhere
        \item \textbf{Length prefix}: Each name preceded by its length
    \end{itemize}
\end{definition}

\subsection{Free Space Management}

\begin{definition}{Free Block Tracking}\\
    File systems must track available storage space:
    \begin{itemize}
        \item \textbf{Linked list of free blocks}:
            \begin{itemize}
                \item Free blocks linked together
                \item Simple but requires disk access to find free space
            \end{itemize}
        \item \textbf{Bitmap}:
            \begin{itemize}
                \item One bit per block (0 = free, 1 = used)
                \item Fast to scan and modify
                \item Compact representation
            \end{itemize}
        \item \textbf{Grouping}: Free blocks grouped together with one block containing addresses of other free blocks
        \item \textbf{Counting}: Store starting address and count of consecutive free blocks
    \end{itemize}
\end{definition}

\begin{definition}{Block Size Considerations}\\
    Choosing the right block size involves trade-offs:
    \begin{itemize}
        \item \textbf{Small blocks}:
            \begin{itemize}
                \item Less internal fragmentation
                \item More blocks per file (more metadata overhead)
                \item Slower access due to more operations
            \end{itemize}
        \item \textbf{Large blocks}:
            \begin{itemize}
                \item Higher data transfer rates
                \item More internal fragmentation for small files
                \item Fewer blocks per file (less metadata overhead)
            \end{itemize}
        \item Optimal block size depends on file size distribution and usage patterns
    \end{itemize}
\end{definition}

\subsection{File System Reliability}

\begin{definition}{File System Consistency}\\
    File systems can become inconsistent due to system crashes:
    \begin{itemize}
        \item \textbf{Consistency checking}: Tools like \texttt{fsck} examine filesystem structures
            \begin{itemize}
                \item Check block allocation consistency
                \item Verify directory structure integrity
                \item Cross-reference inodes and directory entries
                \item Identify and fix common inconsistencies
            \end{itemize}
        \item Common problems detected:
            \begin{itemize}
                \item Missing blocks (allocated but not referenced)
                \item Duplicate blocks (referenced by multiple files)
                \item Free blocks marked as allocated
                \item Directory inconsistencies
            \end{itemize}
    \end{itemize}
\end{definition}

\begin{definition}{Journaling File Systems}\\
    Journaling improves filesystem reliability:
    \begin{itemize}
        \item \textbf{Goal}: Maintain filesystem consistency in the face of system failures
        \item \textbf{Concept}: Keep a log (journal) of intended changes before making them
        \item \textbf{Process}:
            \begin{itemize}
                \item Write planned changes to journal
                \item Mark journal entry as committed
                \item Apply changes to filesystem
                \item Mark journal entry as completed
            \end{itemize}
        \item \textbf{Recovery}: After crash, replay uncommitted journal entries
        \item \textbf{Types}:
            \begin{itemize}
                \item Metadata journaling: Only filesystem metadata is journaled
                \item Full journaling: Both metadata and data are journaled
            \end{itemize}
    \end{itemize}
\end{definition}

\begin{example2}{Understanding Inodes and Links}\\
    Exploring inodes and linking in Linux:
    
\begin{lstlisting}[language=bash, style=basesmol]
# Create a test file
echo "Hello, World!" > testfile.txt

# Check inode number
ls -i testfile.txt

# Create a hard link
ln testfile.txt hardlink.txt

# Check that both files have the same inode
ls -i testfile.txt hardlink.txt

# Check link count
ls -l testfile.txt

# Create a symbolic link
ln -s testfile.txt symlink.txt

# Check inode numbers (symlink has different inode)
ls -i testfile.txt hardlink.txt symlink.txt

# Check what symlink points to
ls -l symlink.txt

# Delete the original file
rm testfile.txt

# Hard link still works
cat hardlink.txt

# Symbolic link is broken
cat symlink.txt
# Output: cat: symlink.txt: No such file or directory

# Clean up
rm hardlink.txt symlink.txt
\end{lstlisting}

    This demonstrates:
    \begin{itemize}
        \item Hard links share the same inode as the original file
        \item Symbolic links have their own inode and store a path
        \item Hard links remain valid even if the original name is removed
        \item Symbolic links break if the target is removed
    \end{itemize}
\end{example2}

\begin{example2}{File System Analysis}\\
    Examining filesystem structure and usage:
    
\begin{lstlisting}[language=bash, style=basesmol]
# Show filesystem usage
df -h

# Show inode usage
df -i

# Display filesystem type
lsblk -f

# Check filesystem properties
tune2fs -l /dev/sda1 | head -20

# Find large files
find /home -type f -size +100M 2>/dev/null | head -10

# Check filesystem fragmentation (ext2/3/4)
e2fsck -fn /dev/sda1

# Monitor filesystem activity
iotop

# Check for filesystem errors in logs
journalctl -u systemd-fsck* --no-pager
\end{lstlisting}
\end{example2}