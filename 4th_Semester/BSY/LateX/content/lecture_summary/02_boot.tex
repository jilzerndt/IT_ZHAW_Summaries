\section{Booting an OS}

\subsection{Boot Process Overview}

\begin{definition}{Boot Process Stages}\\
    The typical boot process follows these common steps:
    \begin{itemize}
        \item Hardware initialization (BIOS/UEFI)
        \item Bootloader execution
        \item Kernel loading and initialization
        \item System initialization (services, environment)
        \item User interface presentation
    \end{itemize}
\end{definition}

\subsection{BIOS and Hardware Initialization}

\begin{definition}{Basic Input Output System (BIOS)}\\
    BIOS is the first program loaded on boot:
    \begin{itemize}
        \item Stored in ROM on the motherboard
        \item Contains low-level I/O software for interfacing with devices
        \item Performs Power-On Self-Test (POST)
        \item Discovers devices by scanning PCI buses
        \item Initializes hardware devices
        \item Selects a boot device from the list in CMOS
        \item Reads the first sector from the boot device (Master Boot Record)
        \item Loads MBR into memory and executes it
    \end{itemize}
\end{definition}

\subsection{Bootloader}

\begin{definition}{Bootloader Function}\\
    The bootloader (loaded from MBR):
    \begin{itemize}
        \item Accesses the boot partition containing the OS
        \item Loads the OS kernel into memory
        \item Sets up registers (Program Counter, Processor Status Word)
        \item Transfers control to the OS by jumping to the first instruction
    \end{itemize}
\end{definition}

\begin{definition}{GRUB (GRand Unified Bootloader)}\\
    GRUB is a common bootloader for Linux systems:
    \begin{itemize}
        \item Requires filesystem access to read the OS
        \item Contains device drivers and filesystem modules
        \item Provides a menu to select the OS to boot
        \item Loads the selected kernel and optional initial RAM disk (initrd)
        \item Passes control to the kernel
    \end{itemize}
\end{definition}

\subsection{OS Initialization}

\begin{definition}{Kernel Initialization}\\
    The OS kernel initialization process:
    \begin{itemize}
        \item Queries hardware information 
        \item Loads and initializes device drivers
        \item Initializes internal management structures (e.g., process table)
        \item Sets up virtual memory and memory management
        \item Creates system services
        \item Launches the first user process (init)
    \end{itemize}
    
    Linux kernel initialization phases:
    \begin{itemize}
        \item Architecture-specific assembly code
            \begin{itemize}
                \item Sets up OS memory map
                \item Identifies CPU type
                \item Calculates total RAM
                \item Disables interrupts
                \item Enables MMU and caches
            \end{itemize}
        \item C main() procedure
            \begin{itemize}
                \item Initializes process tables, interrupt/system-call tables
                \item Sets up virtual memory and page cache
                \item Configures resource control
                \item Loads drivers and initializes OS services
            \end{itemize}
    \end{itemize}
\end{definition}

\subsection{BIOS vs UEFI}

\begin{definition}{BIOS Limitations}\\
    Traditional BIOS has several limitations:
    \begin{itemize}
        \item Operates in 16-bit mode
        \item Relies on MBR (Master Boot Record) from 1983
        \item Limited partition number and size (2 TB max)
        \item Not designed for extendability
        \item Vulnerable to rootkit and bootkit attacks
    \end{itemize}
\end{definition}

\begin{definition}{Unified Extensible Firmware Interface (UEFI)}\\
    UEFI is the modern replacement for BIOS:
    \begin{itemize}
        \item Replaces MBR with GPT (GUID Partitioning Table)
        \item Supports arbitrary number of partitions
        \item Addresses disk space up to 2\textsuperscript{64} bytes
        \item Uses unique UUIDs for partitions to avoid collisions
        \item Features modular design for extendability
        \item Includes "Secure Boot" to restrict which binaries can be executed
        \item Uses cryptographic signatures and X.509 certificates for verification
    \end{itemize}
\end{definition}

\begin{definition}{UEFI Functioning}\\
    UEFI uses an architecture-independent virtual machine:
    \begin{itemize}
        \item Executes special binary files compiled for UEFI (*.efi)
        \item These files can be device drivers, bootloaders, or extensions
        \item Files are stored in the EFI System Partition (ESP)
        \item ESP uses FAT filesystem and can be reused in multi-boot systems
        \item EFI Boot Manager configures which EFI binary to execute
    \end{itemize}
\end{definition}

\begin{example}
    Examining UEFI boot configuration:
    \begin{lstlisting}[language=bash, style=basesmol]
    # Display current boot entries
    sudo efibootmgr -v
    
    # List contents of EFI system partition
    sudo ls -la /boot/efi/EFI/
    
    # View disk partition table format (MBR or GPT)
    sudo fdisk -l
    \end{lstlisting}
\end{example}

\subsection{Initial RAM Disk (initrd/initramfs)}

\begin{definition}{Initial RAM Disk}\\
    The initial RAM disk addresses the "chicken-and-egg" problem:
    \begin{itemize}
        \item OS needs drivers to access hard drive and its filesystem
        \item These drivers are stored on the hard drive itself
        \item Solution: initrd/initramfs provides temporary root filesystem
        \item Contains kernel modules and basic device files
        \item Bootloader uncompresses both kernel and initrd into RAM
        \item Kernel mounts initrd as the initial filesystem
        \item Kernel uses tools found in initrd to find and mount the real filesystem
    \end{itemize}
\end{definition}

\begin{example}
    Examining GRUB configuration for kernel and initrd:
    \begin{lstlisting}[language=bash, style=basesmol]
    # View GRUB configuration
    cat /boot/grub/grub.cfg
    
    # Examine a specific menu entry showing kernel and initrd paths
    grep -A 10 "menuentry 'Ubuntu'" /boot/grub/grub.cfg
    \end{lstlisting}
\end{example}

\subsection{System Services Initialization}

\begin{definition}{System V Init}\\
    Traditional System V initialization:
    \begin{itemize}
        \item Init process runs initialization shell scripts from /etc/rc\# directories
        \item Uses predefined runlevels (0-6) to determine system state
        \item Each runlevel has a set of services defined in scripts
        \item Dependencies among services are coded in the scripts themselves
        \item Results in complex initialization process
    \end{itemize}
    
    System V runlevels:
    \begin{itemize}
        \item 0: Halt - Shuts down the system
        \item 1: Single-user mode - Administrative tasks
        \item 2: Multi-user mode without networking
        \item 3: Multi-user mode with networking
        \item 4: Not used/user-definable
        \item 5: Multi-user mode with GUI
        \item 6: Reboot - Restarts the system
    \end{itemize}
\end{definition}

\begin{definition}{systemd}\\
    Modern initialization system (systemd):
    \begin{itemize}
        \item Replacement for SysV Init
        \item Provides coordinated and parallel service startup
        \item Features on-demand activation and runtime management
        \item Uses dependency-based service control logic
        \item Takes a holistic management approach for the entire system
    \end{itemize}
\end{definition}

\subsection{systemd}

\begin{definition}{systemd Units}\\
    systemd organizes system components as "units":
    \begin{itemize}
        \item Units encapsulate system objects (services, mounts, devices, etc.)
        \item Units have states (active, inactive, activating, deactivating, failed)
        \item Units can depend on other units
        \item Most units are configured in unit configuration files
        \item Some units can be created automatically or programmatically
    \end{itemize}
    
    Common unit types:
    \begin{itemize}
        \item .service - A system service
        \item .target - A group of systemd units (similar to runlevels)
        \item .mount - A filesystem mount point
        \item .device - A device file recognized by the kernel
        \item .socket - An inter-process communication socket
        \item .timer - A systemd timer
    \end{itemize}
\end{definition}

\begin{definition}{systemd Dependencies}\\
    systemd supports various dependency types:
    \begin{itemize}
        \item \textbf{Requires=}: Units that must be started when this unit is started
        \item \textbf{Wants=}: Units that should be started (but not required) when this unit is started
        \item \textbf{Conflicts=}: Units that must be stopped when this unit is started
        \item \textbf{After=}: This unit should be started after the listed units
        \item \textbf{Before=}: This unit should be started before the listed units
    \end{itemize}
\end{definition}

\begin{code}{systemd Service Unit File}\\
    A systemd service unit file consists of three sections:
    
\begin{lstlisting}[language=bash, style=basesmol]
[Unit]
Description=Example Service
Documentation=https://example.com/docs
After=network.target
Wants=network-online.target
Requires=example-dependency.service

[Service]
Type=simple
ExecStart=/usr/bin/example-service
Restart=on-failure
User=exampleuser

[Install]
WantedBy=multi-user.target
\end{lstlisting}

    The sections have specific purposes:
    \begin{itemize}
        \item [Unit]: Metadata and dependencies
        \item [Service]: Execution configuration for services
        \item [Install]: Configuration for enabling the unit
    \end{itemize}
\end{code}

\begin{KR}{Working with systemd}\\
    \paragraph{Viewing unit status}
    \begin{itemize}
        \item \texttt{systemctl status -all} - Show status of all units
        \item \texttt{systemctl status SERVICE} - Show status of specific service
        \item \texttt{systemctl list-units -t service} - List only service units
        \item \texttt{systemctl list-unit-files --type service} - List service unit files
        \item \texttt{systemctl cat SERVICE} - View the content of a unit file
    \end{itemize}
    
    \paragraph{Managing units}
    \begin{itemize}
        \item \texttt{systemctl start SERVICE} - Start a service
        \item \texttt{systemctl stop SERVICE} - Stop a service
        \item \texttt{systemctl restart SERVICE} - Restart a service
        \item \texttt{systemctl enable SERVICE} - Enable service autostart
        \item \texttt{systemctl disable SERVICE} - Disable service autostart
    \end{itemize}
    
    \paragraph{Working with targets}
    \begin{itemize}
        \item \texttt{systemctl list-units -t target} - List available targets
        \item \texttt{systemctl get-default} - Show default target
        \item \texttt{systemctl set-default TARGET} - Set default target
    \end{itemize}
\end{KR}

\begin{example2}{Creating a Simple systemd Service}\\
    Creating a custom service to write to a file when started:
    
\begin{lstlisting}[language=bash, style=basesmol]
# Create a service unit file
sudo nano /etc/systemd/system/myservice.service

# Add the following content
[Unit]
Description=My Simple Service
After=network.target

[Service]
Type=simple
ExecStart=/bin/bash -c 'echo "Service started at $(date)" > /home/ubuntu/service_started'
Restart=on-failure

[Install]
WantedBy=multi-user.target

# Reload systemd to recognize the new service
sudo systemctl daemon-reload

# Start the service
sudo systemctl start myservice

# Verify it worked
cat /home/ubuntu/service_started

# Enable auto-start on boot
sudo systemctl enable myservice
\end{lstlisting}
\end{example2}

\begin{definition}{systemd User Services}\\
    systemd supports user instance services:
    \begin{itemize}
        \item Services managed by individual users without requiring root privileges
        \item User service units are stored in:
            \begin{itemize}
                \item /usr/lib/systemd/user/: Units provided by packages
                \item \textasciitilde/.local/share/systemd/user/: User-installed package units
                \item /etc/systemd/user/: System-wide user units
                \item \textasciitilde/.config/systemd/user/: User's own units
            \end{itemize}
        \item By default, user services start on login and stop on logout
        \item Lingering allows user services to start at boot without login
    \end{itemize}
\end{definition}