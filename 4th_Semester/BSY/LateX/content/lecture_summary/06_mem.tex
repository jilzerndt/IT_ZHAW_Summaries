\section{Memory Management}

\subsection{Memory Management Basics}

\begin{definition}{Memory Management Introduction}\\
    Memory is a critical system resource managed by the operating system:
    \begin{itemize}
        \item The Memory Manager controls main memory allocation and usage
        \item Secondary memory can be used as a buffer zone (swap)
    \end{itemize}
    
    Memory is organized in a hierarchy:
    \begin{itemize}
        \item Fast cache in 1-3 layers (L1, L2, L3)
        \item Main memory (RAM) - slower but larger
        \item Secondary memory (hard disks, SSDs) - for programs and files
        \item Tertiary memory (backup storage, tapes)
    \end{itemize}
\end{definition}

\begin{definition}{Memory Management Tasks}\\
    Operating systems handle several memory management tasks:
    \begin{itemize}
        \item Determining how much memory processes require
        \item Deciding where in memory processes are located (position of residency)
        \item Managing how long processes remain in memory (length of residency)
        \item Subdividing memory for co-existence of multiple processes
        \item Handling memory fragmentation
    \end{itemize}
\end{definition}

\subsection{Memory Allocation Approaches}

\begin{definition}{Memory Division and Fragmentation}\\
    Two basic approaches to memory division:
    \begin{itemize}
        \item \textbf{Static memory division}:
            \begin{itemize}
                \item Memory divided into fixed-size segments
                \item Problem: Internal fragmentation (wasted space within allocated segments)
            \end{itemize}
        \item \textbf{Dynamic memory division}:
            \begin{itemize}
                \item Memory divided according to process needs
                \item Problem: External fragmentation (free space becomes fragmented)
                \item Solution: Compaction (expensive operation)
            \end{itemize}
    \end{itemize}
\end{definition}

\begin{definition}{Buddy Algorithm}\\
    The Buddy Algorithm addresses fragmentation:
    \begin{itemize}
        \item Memory divided into blocks of power-of-2 sizes
        \item When a request arrives, the system:
            \begin{itemize}
                \item Finds the smallest block that fits the request
                \item If no suitable block exists, splits a larger block into two "buddies"
                \item Allocates one buddy and keeps the other free
            \end{itemize}
        \item When a block is freed, the system:
            \begin{itemize}
                \item Checks if its buddy is also free
                \item If both are free, merges them into a larger block
                \item Continues merging recursively if possible
            \end{itemize}
        \item Simpler to implement than other dynamic allocation schemes
        \item Still experiences some internal fragmentation
    \end{itemize}
\end{definition}

\begin{definition}{Free Space Management}\\
    Finding free space during allocation requires efficient algorithms:
    \begin{itemize}
        \item \textbf{Bitmap approach}:
            \begin{itemize}
                \item Space-efficient representation
                \item One bit per allocation unit
                \item Fast free-space finding
            \end{itemize}
        \item \textbf{Linked list approach}:
            \begin{itemize}
                \item List of free blocks
                \item Supports various placement algorithms (first/next/best fit)
            \end{itemize}
    \end{itemize}
\end{definition}

\begin{definition}{Swapping}\\
    When main memory becomes scarce, swapping is used:
    \begin{itemize}
        \item Secondary memory serves as temporary storage for processes
        \item When processes are suspended or exit, their memory is freed
        \item When processes restart, they are reloaded into memory
        \item Allows more processes to run concurrently than physical memory would permit
    \end{itemize}
\end{definition}

\subsection{Virtual Memory}

\begin{definition}{Virtual Memory Concept}\\
    Virtual memory leverages program behavior characteristics:
    \begin{itemize}
        \item Programs exhibit spatial locality (tend to use a limited area of code at any time)
        \item Entire processes don't need to be fully resident in memory
        \item Non-required code/data can be swapped out when not immediately needed
        \item Enables more processes to run concurrently
        \item Must be transparent to programmer/process
    \end{itemize}
\end{definition}

\begin{definition}{Paging}\\
    Paging is the mechanism that enables virtual memory:
    \begin{itemize}
        \item Process memory is divided into fixed-size pages
        \item Physical memory is divided into frames of the same size
        \item Pages are loaded into frames as needed
        \item Memory Management Unit (MMU) manages mapping between pages and frames
        \item Typically uses "on-demand paging" (lazy loading)
            \begin{itemize}
                \item Only loads pages when they are accessed
                \item Sometimes prefetches additional pages based on locality
            \end{itemize}
        \item Process then has a set of resident pages in memory
            \begin{itemize}
                \item Resident set: All process pages in memory
                \item Working set: Pages currently being used
            \end{itemize}
    \end{itemize}
\end{definition}

\begin{definition}{Address Translation}\\
    Virtual memory requires translation between logical and physical addresses:
    \begin{itemize}
        \item Logical address consists of page number and page offset
        \item Page number used to look up frame number in page table
        \item Physical address formed by combining frame number with page offset
        \item Translation process:
            \begin{itemize}
                \item Extract page number and offset from logical address
                \item Use page number to index page table
                \item Retrieve frame number from page table
                \item Combine frame number with offset to form physical address
            \end{itemize}
    \end{itemize}
\end{definition}

\begin{definition}{Page Table}\\
    Page tables maintain the mapping between pages and frames:
    \begin{itemize}
        \item Each entry contains a frame number
        \item Entries also include status bits:
            \begin{itemize}
                \item Valid bit: Indicates if page holds valid data
                \item Present bit: Indicates if page is in memory
                \item Modified bit (dirty): Indicates if page has been written to
                \item Referenced bit: Indicates recent usage
                \item Protection bits: Control read/write/execute permissions
            \end{itemize}
        \item Page tables are stored in main memory
        \item Can be very large for large address spaces
    \end{itemize}
\end{definition}

\begin{definition}{Translation Lookaside Buffer (TLB)}\\
    The TLB addresses the performance overhead of page table lookups:
    \begin{itemize}
        \item Cache for recently accessed page table entries
        \item Small (typically ~64 entries)
        \item Uses content-addressable memory (CAM) for fast lookups
        \item Memory access process with TLB:
            \begin{itemize}
                \item Check TLB for page number
                \item If found (TLB hit), use frame number directly
                \item If not found (TLB miss), search page table
                \item If not in page table, trigger page fault
                \item Add entry to TLB for future accesses
            \end{itemize}
    \end{itemize}
\end{definition}

\begin{definition}{Page Replacement}\\
    When memory is full and a new page is needed, page replacement occurs:
    \begin{itemize}
        \item System must decide which resident page to evict
        \item Can use global strategy (any process's pages) or local strategy (only faulting process's pages)
        \item Common page replacement algorithms:
            \begin{itemize}
                \item Optimal: Replace page used furthest in future (theoretical only)
                \item Least Recently Used (LRU): Replace page unused for longest time
                \item First-In-First-Out (FIFO): Replace oldest page
            \end{itemize}
    \end{itemize}
\end{definition}

\subsection{Linux Memory Management}

\begin{definition}{Linux Page Table Organization}\\
    Linux uses a hierarchical page table structure:
    \begin{itemize}
        \item Multi-level page directory to reduce size and improve lookup speed
        \item Typically 4-level structure:
            \begin{itemize}
                \item Page Global Directory (PGD)
                \item Page Upper Directory (PUD)
                \item Page Middle Directory (PMD)
                \item Page Table Entry (PTE)
            \end{itemize}
        \item Allows efficient handling of sparse address spaces
        \item Only allocates page tables for used parts of address space
    \end{itemize}
\end{definition}

\begin{definition}{Huge Pages}\\
    Linux supports huge pages to improve performance:
    \begin{itemize}
        \item Standard page size is 4KB
        \item Huge pages can be 2MB or 1GB (architecture-dependent)
        \item Advantages:
            \begin{itemize}
                \item Reduces TLB pressure (fewer entries needed to cover same memory)
                \item Improves performance for memory-intensive applications
                \item More efficient for large memory allocations
            \end{itemize}
        \item Uses higher-level page table entries (PUD/PMD)
        \item Requires explicit configuration
    \end{itemize}
\end{definition}

\begin{definition}{Memory Zones}\\
    Linux divides physical memory into zones to handle hardware limitations:
    \begin{itemize}
        \item \textbf{ZONE\_DMA}: Memory addressable by DMA controllers (typically below 16MB)
        \item \textbf{ZONE\_NORMAL}: Regularly mapped memory in kernel space
        \item \textbf{ZONE\_HIGHMEM}: Memory beyond what the kernel can directly address
        \item Zones are managed separately to accommodate different hardware constraints
        \item Each zone has its own free lists and allocation policies
    \end{itemize}
\end{definition}

\begin{definition}{Buddy Allocator}\\
    Linux uses the buddy system for frame allocation:
    \begin{itemize}
        \item Fundamental allocation unit is the page frame
        \item Maintains lists of free blocks of various sizes (powers of 2)
        \item When a process requests memory:
            \begin{itemize}
                \item System finds the smallest block size that fits the request
                \item If necessary, splits larger blocks into "buddies"
                \item Allocates memory from appropriate free list
            \end{itemize}
        \item When memory is freed:
            \begin{itemize}
                \item System checks if buddy is also free
                \item If so, merges buddies to form larger block
                \item Continues merging recursively if possible
            \end{itemize}
        \item Maintains free lists up to MAX\_ORDER-1 (typically 10, so up to 512 contiguous pages)
    \end{itemize}
\end{definition}

\begin{definition}{Slab Allocator}\\
    Linux uses the slab allocator for kernel objects:
    \begin{itemize}
        \item Kernel often requires small allocations for data structures
        \item Pages (4KB) are too large for many kernel objects
        \item Slab allocator:
            \begin{itemize}
                \item Gets pages from buddy allocator
                \item Divides them into smaller objects of specific types
                \item Maintains caches of frequently used object types
                \item Reuses recently freed objects (helps prevent fragmentation)
                \item Preserves object state between uses
            \end{itemize}
        \item Improves memory utilization and allocation speed
        \item Minimizes internal fragmentation
    \end{itemize}
\end{definition}

\begin{definition}{Memory Compaction}\\
    Linux performs memory compaction to address fragmentation:
    \begin{itemize}
        \item Problem: Buddy allocator may not find large contiguous blocks
        \item Solution: kcompactd daemon performs compaction
        \item Process:
            \begin{itemize}
                \item Balances memory zones by swapping out non-working-set pages
                \item Moves movable pages toward the top of the memory zone
                \item Leaves bottom of memory free for new allocations
                \item Creates larger contiguous free blocks
            \end{itemize}
        \item Performed on-demand or periodically
        \item Enables allocation of huge pages and other large memory blocks
    \end{itemize}
\end{definition}

\begin{definition}{Shared Libraries}\\
    Linux uses shared libraries to reduce memory usage:
    \begin{itemize}
        \item Multiple processes can use the same library code
        \item Libraries compiled with -fPIC (Position Independent Code)
        \item Dynamically linked with processes using ld.so
        \item Read-only code pages memory-mapped into processes
        \item Benefits:
            \begin{itemize}
                \item Reduces memory footprint
                \item Shared code/text pages only loaded once
                \item Only data pages need to be process-specific
            \end{itemize}
        \item Challenge: Version compatibility ("DLL hell")
    \end{itemize}
\end{definition}

\begin{definition}{Page Reclamation}\\
    Linux uses page reclamation to recover memory:
    \begin{itemize}
        \item Working set: Pages actively in use by processes
        \item Resident set: All pages in memory
        \item A page is in the working set if:
            \begin{itemize}
                \item Accessed via process address space
                \item Accessed via system call
                \item Accessed via device driver
            \end{itemize}
        \item Linux identifies non-working set pages using a bitmap
        \item Pages marked idle can be reclaimed when memory is needed
    \end{itemize}
\end{definition}

\begin{definition}{Least Recently Used in Linux}\\
    Linux implements a two-stage LRU algorithm:
    \begin{itemize}
        \item Maintains two lists of page frames:
            \begin{itemize}
                \item Active list: Recently accessed pages
                \item Inactive list: Less recently accessed pages
            \end{itemize}
        \item Pages move between lists based on access patterns
        \item Inactive pages are candidates for reclamation
        \item Recently accessed inactive pages can be promoted to active list
        \item Linux uses a global strategy for page reclamation
        \item Recent development: Multi-Generational LRU (MGLRU)
            \begin{itemize}
                \item Assigns generation numbers to page frames based on recent access
                \item Older generations are reclaimed first
                \item Improves performance and responsiveness
            \end{itemize}
    \end{itemize}
\end{definition}

\begin{definition}{Out-of-Memory (OOM) Killer}\\
    Linux has an OOM Killer to handle critical memory shortages:
    \begin{itemize}
        \item Activates when system is critically low on memory
        \item Linux tends to be optimistic in memory allocation
            \begin{itemize}
                \item Processes typically request more memory than needed
                \item System may over-allocate (more than physical memory)
            \end{itemize}
        \item OOM Killer selects processes to terminate based on heuristics
            \begin{itemize}
                \item Considers memory usage, runtime, nice value, etc.
                \item Each process has an oom\_score in /proc/\$PID/oom\_score
                \item Can be adjusted through oom\_score\_adj
            \end{itemize}
        \item Prioritizes system stability over individual process survival
    \end{itemize}
\end{definition}

\begin{KR}{Memory Management Analysis in Linux}\\
    \paragraph{Basic memory information}
    \begin{itemize}
        \item Display system memory usage: \texttt{free -h}
        \item View memory details: \texttt{cat /proc/meminfo}
        \item Check process memory usage: \texttt{ps -eo pid,ppid,cmd,vsz,rss}
        \item Interactive memory monitor: \texttt{top} or \texttt{htop}
    \end{itemize}
    
    \paragraph{Process memory analysis}
    \begin{itemize}
        \item Check process memory maps: \texttt{cat /proc/\$PID/maps}
        \item View process memory status: \texttt{cat /proc/\$PID/status}
        \item Analyze memory usage in detail: \texttt{pmap \$PID}
        \item Track memory over time: \texttt{smem}
    \end{itemize}
    
    \paragraph{Memory page information}
    \begin{itemize}
        \item Get page size: \texttt{getconf PAGE\_SIZE}
        \item Check huge pages: \texttt{cat /proc/meminfo | grep Huge}
        \item View page stats: \texttt{cat /proc/pagetypeinfo}
        \item Check page faults: \texttt{ps -o min\_flt,maj\_flt \$PID}
    \end{itemize}
    
    \paragraph{Memory limits and control}
    \begin{itemize}
        \item Set memory limits: \texttt{ulimit -m [size]}
        \item Control cgroup memory: \texttt{echo [value] > /sys/fs/cgroup/memory/[group]/memory.limit\_in\_bytes}
        \item Check swappiness: \texttt{cat /proc/sys/vm/swappiness}
        \item Adjust swappiness: \texttt{sysctl vm.swappiness=[value]}
    \end{itemize}
\end{KR}

\begin{example2}{Analyzing Process Memory}\\
    Check memory usage details for a process:
    
\begin{lstlisting}[language=bash, style=basesmol]
# Get PID of a process
PID=$(pidof firefox)

# Check virtual and resident memory size
ps -o pid,comm,vsz,rss -p $PID

# Analyze memory map segments
cat /proc/$PID/maps | head -10

# View detailed memory status
grep -E 'VmSize|VmRSS|VmData|VmStk|VmExe' /proc/$PID/status

# Map process address space in detail
pmap -x $PID | head -20

# Check page faults
ps -o min_flt,maj_flt -p $PID
\end{lstlisting}

    \tcblower
    
    \textbf{Explanation of memory terms:}
    \begin{itemize}
        \item \textbf{VSZ (Virtual Size)}: Total virtual memory allocated to process
        \item \textbf{RSS (Resident Set Size)}: Actual physical memory used
        \item \textbf{VmData}: Size of data segment
        \item \textbf{VmStk}: Size of stack
        \item \textbf{VmExe}: Size of text segment
        \item \textbf{min\_flt}: Minor page faults (page in memory but not in process's page table)
        \item \textbf{maj\_flt}: Major page faults (page had to be loaded from disk)
    \end{itemize}
\end{example2}

\begin{example2}{Working with Page Size and Memory Allocation}\\
    Analyzing page size and memory allocation:
    
\begin{lstlisting}[language=C, style=basesmol]
#include <stdio.h>
#include <stdlib.h>
#include <unistd.h>
#include <sys/mman.h>

int main() {
    // Get process ID
    pid_t pid = getpid();
    printf("Process ID: %d\n", pid);
    
    // Get page size
    int page_size = getpagesize();
    printf("Page size: %d bytes\n", page_size);
    
    // Allocate memory for 10 pages
    size_t size = 10 * page_size;
    char *buffer = malloc(size);
    printf("Allocated %zu bytes (%zu pages)\n", 
           size, size / page_size);
    
    // Check page faults before access
    printf("Check page faults with: ps -o min_flt,maj_flt %d\n", pid);
    printf("Press Enter to continue...\n");
    getchar();
    
    // Access the memory (causes page faults)
    for (int i = 0; i < size; i += page_size) {
        buffer[i] = 1;  // Touch one byte per page
    }
    
    // Check page faults after access
    printf("Check page faults again with: ps -o min_flt,maj_flt %d\n", pid);
    printf("Press Enter to continue...\n");
    getchar();
    
    // Allocate page-aligned memory
    void *aligned_buf = NULL;
    int result = posix_memalign(&aligned_buf, page_size, size);
    if (result == 0) {
        printf("Allocated %zu bytes aligned on page boundary\n", size);
    }
    
    // Free memory
    free(buffer);
    free(aligned_buf);
    
    return 0;
}
\end{lstlisting}

    This example demonstrates:
    \begin{itemize}
        \item Getting the system page size
        \item Allocating memory
        \item Observing page faults due to lazy allocation
        \item Creating page-aligned memory allocations
    \end{itemize}
\end{example2}