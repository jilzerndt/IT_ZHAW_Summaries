\section{Networking}

\subsection{Networking Requirements on Operating Systems}

\begin{concept}{Operating System Networking Requirements}\\
    Modern operating systems must support comprehensive networking capabilities:
    \begin{itemize}
        \item \textbf{Local interprocess communication}: Stream or datagram communication between processes
        \item \textbf{External network connectivity}: Connect machines to networks for distributed applications
        \item \textbf{Security and protection}: Protect against malicious traffic and network-related attacks
        \item \textbf{Traffic routing}: Support and route traffic between different network interfaces
        \item \textbf{Virtualization support}: Network namespaces and virtual interfaces for cloud computing
        \item \textbf{Distributed applications}: Support for mail, web services, remote access, printing, etc.
    \end{itemize}
\end{concept}

\subsection{Network Interface Hardware}

\mult{2}

\begin{definition}{Network Interface Card (NIC) Architecture}\\
    Network interface cards connect systems to networks via the system bus:
    \begin{itemize}
        \item \textbf{Physical Layer Device (PHY)}:
            \begin{itemize}
                \item Maintains electrical characteristics defined by the protocol
                \item Establishes communication with counterpart devices
                \item Decodes, descrambles, and deserializes incoming signals
                \item Encodes, scrambles, and serializes outgoing data
                \item Delivers bit stream to the Media Access Controller
            \end{itemize}
        \item \textbf{Media Access Controller (MAC)}:
            \begin{itemize}
                \item Handles Ethernet Layer-2 protocols and traffic
                \item Configures PHY via MDIO interface
                \item Processes frame integrity (FCS, buffer management)
                \item Places data into main memory buffers
                \item Activates NIC driver via interrupts
            \end{itemize}
    \end{itemize}
\end{definition}



\begin{concept}{DMA and Data Transfer}\\
    Modern NICs use Direct Memory Access for efficient data transfer:
    \begin{itemize}
        \item MAC-memory data transfer performed by DMA
        \item Supports multiple concurrent DMA operations
        \item Each operation described by buffer descriptors
        \item Often uses Scatter-Gather mechanism for efficiency
        \item Prevents buffer underflow and overflow conditions
        \item Reduces CPU overhead for bulk data transfers
    \end{itemize}
\end{concept}

\begin{theorem}{NIC Device Drivers}\\
    NIC drivers operate in kernel space and manage hardware interaction:
    \begin{itemize}
        \item \textbf{Receive (Rx) side}:
            \begin{itemize}
                \item Notified of incoming frames via interrupts
                \item Checks frame status and discards erroneous frames
                \item Formats frames for software stack consumption
            \end{itemize}
        \item \textbf{Transmit (Tx) side}:
            \begin{itemize}
                \item Initializes MAC DMA channels for transmission
                \item Notifies MAC when frames need transmission
                \item Monitors transmission status
            \end{itemize}
        \item Requires host-side buffering system for efficient operation
    \end{itemize}
\end{theorem}

\multend

\subsection{Network Stack Architecture}

\mult{2}

\begin{concept}{Zero-Copy Stack Design}\\
    Traditional layered networking creates inefficiencies:
    \begin{itemize}
        \item \textbf{Naive approach}: Frame copied between each layer
        \item \textbf{Problems}: Multiple memory copies, CPU intensive operations
        \item \textbf{Zero-copy solution}:
            \begin{itemize}
                \item All frames pooled in buffers at Layer 2
                \item Only pointers to frame metadata passed between layers
                \item Frame payload remains in original location
                \item Significantly reduces memory operations and CPU usage
            \end{itemize}
    \end{itemize}
\end{concept}

\begin{definition}{Linux sk\_buff Structure}\\
    Linux implements zero-copy networking using sk\_buff (SKB):
    \begin{itemize}
        \item Fundamental structure containing frame processing variables
        \item Enables communication between network layers
        \item SKB cloned during layer transfers (metadata copied, payload referenced)
        \item NIC maintains ring buffer of SKB pointers in kernel memory
        \item Scatter-Gather DMA transfers data directly to SKB structures
        \item Provides efficient frame processing with minimal copying
    \end{itemize}
\end{definition}

\multend

\subsubsection{Layer 2 - Data Link Layer}

\mult{2}

\begin{definition}{Layer 2 Processing}\\
    The data link layer handles frame-level operations:
    \begin{itemize}
        \item Differentiates between ARP and other frame types
        \item Checks Ethernet frame characteristics
        \item Validates frame integrity and format
        \item Passes frames to appropriate upper layer protocols
        \item Implements bridging between network segments
        \item Manages local network addressing (MAC addresses)
    \end{itemize}
\end{definition}

\begin{concept}{Layer 2 Bridging}\\
    Bridging connects multiple network segments:
    \begin{itemize}
        \item \textbf{Network bridging}: Connects two or more LAN segments
        \item \textbf{Interface bonding}: Combines multiple NICs for performance or redundancy
            \begin{itemize}
                \item Load balancing across multiple interfaces
                \item Failover protection
                \item Increased aggregate bandwidth
            \end{itemize}
        \item \textbf{Virtual bridges}: Software-based bridging for virtual networks
        \item Integration with Linux netfilter system for filtering and processing
    \end{itemize}
\end{concept}

\multend

\subsubsection{Layer 3 - Network Layer}

\mult{2}

\begin{definition}{Layer 3 Network Processing}
    The network layer provides internetworking capabilities:
    \begin{itemize}
        \item \textbf{Addressing}: IP addressing beyond LAN boundaries
        \item \textbf{Routing}: Forwards packets to destinations beyond local subnets
        \item \textbf{Fragmentation}: Handles packet size limitations
        \item \textbf{Quality of Service}: Manages traffic priorities and bandwidth
        \item \textbf{Security}: Firewall functionality for traffic filtering
    \end{itemize}
\end{definition}

\begin{concept}{IP Packet Processing}\\
    Linux IP layer processing involves several steps:
    \begin{itemize}
        \item Network driver calls IP handler (ip\_rcv)
        \item Perform sanity checks on IP header (length, checksum)
        \item Routing subsystem determines packet destination
        \item Local delivery or forwarding decision
        \item Integration with netfilter hooks for security and processing
        \item Connection tracking for stateful operations
    \end{itemize}
\end{concept}

\begin{theorem}{Linux Netfilter System}\\
    Netfilter provides packet filtering and manipulation:
    \begin{itemize}
        \item \textbf{Functions}:
            \begin{itemize}
                \item Packet selection and filtering
                \item Network Address Translation (NAT)
                \item Packet mangling and modification
                \item Connection tracking and state management
                \item Statistics collection and logging
            \end{itemize}
        \item \textbf{Implementation}: Hook system with five points in routing path
        \item \textbf{Hook return values}:
            \begin{itemize}
                \item NF\_ACCEPT: Packet continues traversal
                \item NF\_DROP: Discard packet silently
                \item NF\_STOLEN: Packet extracted for special processing
                \item NF\_QUEUE: Queue packet for userspace processing
                \item NF\_REPEAT: Call hook function again
            \end{itemize}
        \item \textbf{Frontends}: iptables, nftables for user configuration
    \end{itemize}
\end{theorem}



\multend



\raggedcolumns
\columnbreak

\subsection{Layer 4 - Transport Layer}

\begin{definition}{Transport Layer Protocols}
    Layer 4 provides end-to-end communication services:
    \vspace{2mm}\\

    \begin{minipage}{0.5\linewidth}
   \textbf{UDP (User Datagram Protocol)}:
            \begin{itemize}
                \item Thin wrapper around IP frames
                \item Provides unreliable, connectionless transmission
                \item Minimal overhead for simple request-response applications
                \item Used for DNS, DHCP, streaming media
            \end{itemize}
    \end{minipage}
    \begin{minipage}{0.5\linewidth}
        \textbf{TCP (Transmission Control Protocol)}:
            \begin{itemize}
                \item Reliable, connection-oriented transport
                \item Manages connections, flow control, congestion control
                \item Assembles data streams from individual segments
                \item Provides error detection and recovery
                \item Used for HTTP, FTP, SSH, email
            \end{itemize}
    \end{minipage}

\end{definition}

\begin{code}{Basic TCP Socket Programming}
    Example of TCP socket server and client:
    
\begin{lstlisting}[language=C, style=basesmol]
// TCP Server
#include <sys/socket.h>
#include <netinet/in.h>
#include <arpa/inet.h>

int server_fd, client_fd;
struct sockaddr_in address;
int addrlen = sizeof(address);

// Create socket
server_fd = socket(AF_INET, SOCK_STREAM, 0);

// Setup address
address.sin_family = AF_INET;
address.sin_addr.s_addr = INADDR_ANY;
address.sin_port = htons(8080);

// Bind socket to address
bind(server_fd, (struct sockaddr *)&address, sizeof(address));

// Listen for connections
listen(server_fd, 3);

// Accept connection
client_fd = accept(server_fd, (struct sockaddr *)&address, 
                   (socklen_t*)&addrlen);

// Read/write data
char buffer[1024] = {0};
read(client_fd, buffer, 1024);
send(client_fd, "Hello from server", 17, 0);

// TCP Client
int sock = 0;
struct sockaddr_in serv_addr;

// Create socket
sock = socket(AF_INET, SOCK_STREAM, 0);

// Setup server address
serv_addr.sin_family = AF_INET;
serv_addr.sin_port = htons(8080);
inet_pton(AF_INET, "127.0.0.1", &serv_addr.sin_addr);

// Connect to server
connect(sock, (struct sockaddr *)&serv_addr, sizeof(serv_addr));

// Send/receive data
send(sock, "Hello from client", 17, 0);
read(sock, buffer, 1024);
\end{lstlisting}
\end{code}


\subsubsection{Layer 7 - Application Interface}



\begin{definition}{Berkeley Sockets}
    Sockets provide the user-facing API for network programming:
    \begin{itemize}
        \item Connect port numbers and network addresses with protocols and processes
        \item \textbf{Socket types}:
            \begin{itemize}
                \item SOCK\_STREAM: Reliable, ordered, connection-based (TCP)
                \item SOCK\_DGRAM: Unreliable, unordered, connectionless (UDP)
                \item SOCK\_RAW: Direct access to IP layer
            \end{itemize}
        \item \textbf{Address families}:
            \begin{itemize}
                \item AF\_INET: IPv4 Internet protocols
                \item AF\_INET6: IPv6 Internet protocols
                \item AF\_UNIX: Local inter-process communication
                \item AF\_NETLINK: Kernel-userspace communication
            \end{itemize}
        \item Sockets have states and support standard file operations
    \end{itemize}
\end{definition}

\begin{concept}{Special Socket Types}
    Linux supports various specialized socket types:
    \begin{itemize}
        \item \textbf{Unix Domain Sockets}:
            \begin{itemize}
                \item Inter-process communication using socket interface
                \item Operates over the filesystem namespace
                \item Cannot traverse host boundaries
                \item Higher performance than network sockets for local communication
            \end{itemize}
        \item \textbf{Loopback Interface}:
            \begin{itemize}
                \item Host network interface with itself
                \item Can be accessed like any network interface
                \item Used for local services and testing
            \end{itemize}
        \item \textbf{Netlink Sockets}:
            \begin{itemize}
                \item Kernel-userspace IPC mechanism
                \item Used for configuration and monitoring
                \item Families include NETLINK\_ROUTE, NETLINK\_AUDIT, etc.
                \item Standard message format with adaptable headers
            \end{itemize}
    \end{itemize}
\end{concept}


\raggedcolumns
\columnbreak


\subsection{Network Configuration Tools}

\begin{definition}{Linux Network Configuration}
    Linux provides various tools for network management:
    \begin{itemize}
        \item \textbf{iproute2 suite} (modern, preferred):
            \begin{itemize}
                \item \texttt{ip}: Network interface and routing configuration
                \item \texttt{tc}: Traffic control and QoS management
                \item \texttt{ss}: Socket statistics and monitoring
                \item \texttt{bridge}: Bridge configuration and management
            \end{itemize}
        \item \textbf{net-tools suite} (legacy, deprecated):
            \begin{itemize}
                \item \texttt{ifconfig}: Interface configuration
                \item \texttt{route}: Routing table manipulation
                \item \texttt{netstat}: Network statistics
                \item \texttt{arp}: ARP table management
            \end{itemize}
        \item Both suites use different underlying mechanisms (netlink vs ioctl)
    \end{itemize}
\end{definition}

\begin{KR}{Network Configuration and Troubleshooting}
    \paragraph{Interface management}
    \begin{itemize}
        \item Show interfaces: \texttt{ip link show}
        \item Bring interface up: \texttt{sudo ip link set eth0 up}
        \item Assign IP address: \texttt{sudo ip addr add 192.168.1.100/24 dev eth0}
        \item Show IP addresses: \texttt{ip addr show}
        \item Remove IP address: \texttt{sudo ip addr del 192.168.1.100/24 dev eth0}
    \end{itemize}
    
    \paragraph{Routing management}
    \begin{itemize}
        \item Show routing table: \texttt{ip route show}
        \item Add default route: \texttt{sudo ip route add default via 192.168.1.1}
        \item Add specific route: \texttt{sudo ip route add 10.0.0.0/8 via 192.168.1.1}
        \item Delete route: \texttt{sudo ip route del 10.0.0.0/8}
        \item Monitor route changes: \texttt{ip monitor route}
    \end{itemize}
    
    \paragraph{Network diagnostics}
    \begin{itemize}
        \item Test connectivity: \texttt{ping 8.8.8.8}
        \item Trace route: \texttt{traceroute google.com}
        \item DNS lookup: \texttt{dig google.com}
        \item Show listening sockets: \texttt{ss -tuln}
        \item Show network statistics: \texttt{ss -s}
        \item Monitor network traffic: \texttt{tcpdump -i eth0}
    \end{itemize}
    
    \paragraph{Firewall configuration}
    \begin{itemize}
        \item List iptables rules: \texttt{sudo iptables -L}
        \item Allow SSH: \texttt{sudo iptables -A INPUT -p tcp --dport 22 -j ACCEPT}
        \item Block specific IP: \texttt{sudo iptables -A INPUT -s 192.168.1.100 -j DROP}
        \item Save rules: \texttt{sudo iptables-save > /etc/iptables/rules.v4}
        \item Modern alternative: \texttt{sudo nft list ruleset}
    \end{itemize}
\end{KR}

\begin{example2}{Network Interface Configuration}
    Complete network interface setup:
    
\begin{lstlisting}[language=bash, style=basesmol]
# Show current network configuration
ip addr show

# Configure a static IP address
sudo ip addr add 192.168.1.100/24 dev eth0
sudo ip link set eth0 up

# Add default gateway
sudo ip route add default via 192.168.1.1

# Configure DNS (temporary)
echo "nameserver 8.8.8.8" | sudo tee /etc/resolv.conf

# Test connectivity
ping -c 3 8.8.8.8
ping -c 3 google.com

# Show routing table
ip route show

# Show ARP table
ip neigh show

# Monitor network activity
ss -tuln | grep :80

# Configure persistent networking (systemd-networkd)
cat << EOF | sudo tee /etc/systemd/network/10-eth0.network
[Match]
Name=eth0

[Network]
Address=192.168.1.100/24
Gateway=192.168.1.1
DNS=8.8.8.8
DNS=8.8.4.4
EOF

# Enable systemd-networkd
sudo systemctl enable systemd-networkd
sudo systemctl restart systemd-networkd
\end{lstlisting}
\end{example2}

\begin{example2}{Socket Programming Example}\\
    Simple client-server application:
    
\begin{lstlisting}[language=C, style=basesmol]
// Simple UDP Echo Server
#include <stdio.h>
#include <stdlib.h>
#include <string.h>
#include <sys/socket.h>
#include <netinet/in.h>

int main() {
    int sockfd;
    struct sockaddr_in servaddr, cliaddr;
    char buffer[1024];
    socklen_t len = sizeof(cliaddr);
    
    // Create UDP socket
    sockfd = socket(AF_INET, SOCK_DGRAM, 0);
    if (sockfd < 0) {
        perror("socket creation failed");
        exit(EXIT_FAILURE);
    }
    
    // Setup server address
    memset(&servaddr, 0, sizeof(servaddr));
    servaddr.sin_family = AF_INET;
    servaddr.sin_addr.s_addr = INADDR_ANY;
    servaddr.sin_port = htons(8080);
    
    // Bind socket
    if (bind(sockfd, (const struct sockaddr *)&servaddr, 
             sizeof(servaddr)) < 0) {
        perror("bind failed");
        exit(EXIT_FAILURE);
    }
    
    printf("UDP Echo Server listening on port 8080...\n");
    
    // Echo loop
    while (1) {
        int n = recvfrom(sockfd, buffer, 1024, 0, 
                         (struct sockaddr *)&cliaddr, &len);
        buffer[n] = '\0';
        printf("Received: %s\n", buffer);
        
        sendto(sockfd, buffer, strlen(buffer), 0, 
               (const struct sockaddr *)&cliaddr, len);
    }
    
    return 0;
}

// Compile: gcc -o udp_server udp_server.c
// Test: echo "Hello" | nc -u localhost 8080
\end{lstlisting}
\end{example2}

\begin{example2}{Network Namespace Isolation}\\
    Creating isolated network environments:
    
\begin{lstlisting}[language=bash, style=basesmol]
# Create a new network namespace
sudo ip netns add isolated

# List network namespaces
ip netns list

# Show interfaces in the new namespace (should be empty except lo)
sudo ip netns exec isolated ip link show

# Create a virtual ethernet pair
sudo ip link add veth0 type veth peer name veth1

# Move one end to the isolated namespace
sudo ip link set veth1 netns isolated

# Configure the host-side interface
sudo ip addr add 10.0.0.1/24 dev veth0
sudo ip link set veth0 up

# Configure the namespace-side interface
sudo ip netns exec isolated ip addr add 10.0.0.2/24 dev veth1
sudo ip netns exec isolated ip link set veth1 up
sudo ip netns exec isolated ip link set lo up

# Test connectivity
ping -c 3 10.0.0.2

# Run commands in the isolated namespace
sudo ip netns exec isolated ping -c 3 10.0.0.1

# Show routing in the namespace
sudo ip netns exec isolated ip route show

# Add default route in namespace (through host)
sudo ip netns exec isolated ip route add default via 10.0.0.1

# Enable IP forwarding on host for internet access
echo 1 | sudo tee /proc/sys/net/ipv4/ip_forward

# Add NAT rule for namespace traffic
sudo iptables -t nat -A POSTROUTING -s 10.0.0.0/24 -j MASQUERADE

# Test internet connectivity from namespace
sudo ip netns exec isolated ping -c 3 8.8.8.8

# Clean up
sudo ip netns delete isolated
sudo ip link delete veth0
\end{lstlisting}
\end{example2}



\begin{KR}{Network Service Management}\\
    \paragraph{Service discovery and management}
    \begin{itemize}
        \item Check listening services: \texttt{sudo ss -tulpn}
        \item Find process using port: \texttt{sudo lsof -i :80}
        \item Check service status: \texttt{systemctl status nginx}
        \item Control network services: \texttt{sudo systemctl start/stop/restart nginx}
    \end{itemize}
    
    \paragraph{Network security}
    \begin{itemize}
        \item Check open ports: \texttt{nmap localhost}
        \item Monitor failed connections: \texttt{sudo journalctl -f | grep ssh}
        \item Configure fail2ban: \texttt{sudo systemctl enable fail2ban}
        \item Check SELinux network policies: \texttt{getsebool -a | grep httpd}
    \end{itemize}
    
    \paragraph{Performance optimization}
    \begin{itemize}
        \item Tune network buffers: \texttt{sysctl -w net.core.rmem\_max=16777216}
        \item Enable TCP window scaling: \texttt{sysctl -w net.ipv4.tcp\_window\_scaling=1}
        \item Configure congestion control: \texttt{sysctl -w net.ipv4.tcp\_congestion\_control=bbr}
        \item Monitor network performance: \texttt{iperf3 -c server\_ip}
    \end{itemize}
\end{KR}

\begin{definition}{Extended Berkeley Packet Filter (eBPF)}\\
    Modern Linux networking includes advanced packet filtering:
    \begin{itemize}
        \item \textbf{eBPF}: Extended packet filtering beyond traditional BPF
        \item \textbf{Features}:
            \begin{itemize}
                \item Programmable packet processing in kernel space
                \item JIT compilation for performance
                \item Safe execution environment with verification
                \item Integration with various kernel subsystems
            \end{itemize}
        \item \textbf{Applications}:
            \begin{itemize}
                \item High-performance networking
                \item Container networking
                \item Network security and monitoring
                \item Traffic analysis and manipulation
            \end{itemize}
        \item Represents the future of Linux packet filtering and network programming
    \end{itemize}
\end{definition}


\begin{example2}{Advanced Network Monitoring}\\
    Comprehensive network monitoring and analysis:
    
\begin{lstlisting}[language=bash, style=basesmol]
# Monitor real-time network connections
watch -n 1 'ss -tuln'

# Show detailed socket information
ss -tulpn | grep :80

# Monitor network traffic with detailed statistics
iftop -i eth0

# Capture and analyze packets
sudo tcpdump -i eth0 -w capture.pcap

# Analyze captured packets
tcpdump -r capture.pcap -n | head -20

# Monitor bandwidth usage per process
sudo nethogs eth0

# Show network interface statistics
cat /proc/net/dev

# Monitor netlink messages
ip monitor all

# Check connection tracking
sudo cat /proc/net/nf_conntrack | head -10

# Show detailed network stack statistics
ss --info --processes --numeric

# Monitor DNS queries
sudo tcpdump -i eth0 port 53

# Show kernel network parameters
sysctl -a | grep net.ipv4 | head -10

# Monitor network errors
netstat -i
\end{lstlisting}
\end{example2}
