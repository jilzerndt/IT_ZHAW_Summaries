\section{Advanced Storage Management}

\subsection{Advanced Storage Management}

\begin{concept}{Beyond Basic Partitioning}
    Modern storage management goes beyond simple partitioning:
    \begin{itemize}
        \item Traditional approach: Fixed partitions with static sizing
        \item Advanced approach: Dynamic, flexible storage allocation
        \item Enables efficient use of storage resources
        \item Supports features like snapshots, thin provisioning, and live resizing
        \item Provides better fault tolerance and performance optimization
    \end{itemize}
\end{concept}

\subsection{Non-persistent Storage: tmpfs}

\begin{definition}{tmpfs (Temporary File System)}
    tmpfs is a file system that stores files in volatile memory:
    \begin{itemize}
        \item Files stored in RAM (not on disk)
        \item Extremely fast read and write operations based on memcpy
        \item Data is lost on system crash or shutdown (non-persistent)
        \item Used by many distributions for temporary directories (/tmp, /run)
        \item Can be mounted with size limits to prevent memory exhaustion
        \item Ideal for fast access to non-critical, temporary data
    \end{itemize}
\end{definition}

\begin{example}
    Working with tmpfs:
    
\begin{lstlisting}[language=bash, style=basesmol]
# Create a directory for tmpfs mount
mkdir -p /mnt/tmp

# Mount tmpfs with 2GB size limit
sudo mount -t tmpfs -o size=2G tmpfs /mnt/tmp

# Check filesystem usage
df -h /mnt/tmp

# Test performance
dd if=/dev/zero of=/mnt/tmp/speedtest bs=10M count=100

# Compare with disk-based filesystem
dd if=/dev/zero of=/tmp/speedtest bs=10M count=100

# Unmount when done
sudo umount /mnt/tmp
\end{lstlisting}
\end{example}

\subsection{Logical Volume Management (LVM)}

\begin{definition}{Logical Volume Management}
    LVM provides a method of allocating space on mass-storage devices that is more flexible than conventional partitioning:
    \begin{itemize}
        \item \textbf{Physical Volumes (PV)}: Physical disks or partitions providing block storage
        \item \textbf{Volume Groups (VG)}: Collections of physical volumes that form a storage pool
        \item \textbf{Logical Volumes (LV)}: Virtual partitions created from volume group space
        \item Logical volumes can span multiple physical devices
        \item Enables dynamic resizing without unmounting filesystems
        \item Supports advanced features like snapshots and thin provisioning
    \end{itemize}
\end{definition}

\begin{concept}{LVM Features}
    LVM provides several advanced storage management features:
    \begin{itemize}
        \item \textbf{Free space management}: Allocate space from any available physical volume
        \item \textbf{Online space management}: Resize logical volumes while mounted
        \item \textbf{Snapshots}: Create point-in-time copies using copy-on-write
        \item \textbf{Balancing}: Distribute data evenly across physical volumes
        \item \textbf{Thin provisioning}: Allocate space on demand rather than upfront
        \item \textbf{Software RAID}: Combine multiple disks for redundancy or performance
    \end{itemize}
\end{concept}

\begin{KR}{Setting up LVM}
    \paragraph{Initialize physical volumes}
    \begin{itemize}
        \item Prepare disks: \texttt{sudo pvcreate /dev/sdb /dev/sdc}
        \item Display PVs: \texttt{sudo pvdisplay}
        \item Scan for PVs: \texttt{sudo pvscan}
    \end{itemize}
    
    \paragraph{Create volume group}
    \begin{itemize}
        \item Create VG: \texttt{sudo vgcreate myVG /dev/sdb /dev/sdc}
        \item Display VGs: \texttt{sudo vgdisplay}
        \item Extend VG: \texttt{sudo vgextend myVG /dev/sdd}
    \end{itemize}
    
    \paragraph{Create logical volumes}
    \begin{itemize}
        \item Create LV: \texttt{sudo lvcreate -L 10G -n myLV myVG}
        \item Create LV using percentage: \texttt{sudo lvcreate -l 50\%FREE -n myLV2 myVG}
        \item Display LVs: \texttt{sudo lvdisplay}
    \end{itemize}
    
    \paragraph{Resize logical volumes}
    \begin{itemize}
        \item Extend LV: \texttt{sudo lvextend --resizefs -L +5G /dev/myVG/myLV}
        \item Reduce LV: \texttt{sudo lvreduce --resizefs -L -2G /dev/myVG/myLV}
        \item Resize to specific size: \texttt{sudo lvresize --resizefs -L 20G /dev/myVG/myLV}
    \end{itemize}
    
    \paragraph{Snapshot management}
    \begin{itemize}
        \item Create snapshot: \texttt{sudo lvcreate --size 1G --snapshot --name myLV-snap /dev/myVG/myLV}
        \item Mount snapshot: \texttt{sudo mount /dev/myVG/myLV-snap /mnt/snapshot}
        \item Remove snapshot: \texttt{sudo lvremove /dev/myVG/myLV-snap}
    \end{itemize}
\end{KR}

\subsection{Copy-on-Write and Snapshots}

\begin{concept}{Copy-on-Write (COW)}
    COW is a resource management technique for efficient duplication:
    \begin{itemize}
        \item \textbf{Goal}: Efficiently duplicate or copy modifiable resources
        \item \textbf{Principle}: If a resource is shared but not modified, no copy is needed
        \item \textbf{Implementation}: Copy is made only upon modification (write operation)
        \item \textbf{Benefits}:
            \begin{itemize}
                \item Saves storage space for identical data
                \item Reduces initial duplication time
                \item Minimizes I/O operations
            \end{itemize}
        \item \textbf{Applications}: Virtual machines, snapshots, process forking
    \end{itemize}
\end{concept}

\begin{definition}{Snapshots}
    Snapshots provide point-in-time copies of data:
    \begin{itemize}
        \item \textbf{Purpose}: Allow full backup without disabling write access
        \item \textbf{Implementation}: Fix the state of a filesystem at a specific time using COW
        \item \textbf{Process}:
            \begin{itemize}
                \item Create snapshot at time T
                \item Original data continues to be accessible
                \item New writes create copies, preserving original blocks
                \item Snapshot contains the data as it existed at time T
            \end{itemize}
        \item \textbf{Use cases}: Backups, testing, rollback scenarios
    \end{itemize}
\end{definition}

\raggedcolumns
\columnbreak

\subsection{BtrFS (B-tree File System)}

\begin{definition}{BtrFS Overview}
    BtrFS is a modern copy-on-write filesystem for Linux:
    \begin{itemize}
        \item Open source general-purpose COW filesystem (developed since 2007)
        \item Highly scalable (max volume size: 16 EB, max file size: 16 EB)
        \item Consolidates advanced features from multiple filesystems
        \item Includes copy-on-write, snapshots, logical volume management
        \item Built-in data and metadata integrity checking with checksums
        \item Integrated RAID support and compression capabilities
        \item Foundation for distributed storage systems like Ceph
    \end{itemize}
\end{definition}

\begin{definition}{BtrFS Design Goals}
    BtrFS aims to address limitations of traditional filesystems:
    \begin{itemize}
        \item Work well for diverse workloads and hardware configurations
        \item Maintain performance as the filesystem ages (avoid fragmentation issues)
        \item Scale from small devices (smartphones) to enterprise systems
        \item Provide scalability in multiple dimensions (disk space, memory, CPUs)
        \item Ensure data integrity through checksums and metadata duplication
        \item Support diverse storage devices (HDDs, SSDs, mixed arrays)
        \item Deliver performance comparable to or better than ext4 and XFS
    \end{itemize}
\end{definition}

\begin{concept}{BtrFS Key Concepts}
    BtrFS introduces several important concepts:
    \begin{itemize}
        \item \textbf{COW filesystem}: Uses copy-on-write for all updates instead of in-place modifications
        \item \textbf{Subvolumes}: Independent file trees that can be mounted separately
        \item \textbf{Extents}: Mappings from logical file areas to contiguous physical areas
        \item \textbf{Physical chunks}: Division of each device into manageable segments
        \item \textbf{Logical chunks}: Groupings of physical chunks for RAID and allocation
        \item \textbf{B-tree forest}: Multiple B-trees manage different aspects of the filesystem
    \end{itemize}
\end{concept}

\begin{theorem}{BtrFS Architecture}
    BtrFS is constructed from a forest of COW-friendly B-trees:
    \begin{itemize}
        \item \textbf{Superblock}: Fixed location anchor point
        \item \textbf{Tree of tree roots}: Manages the forest of B-trees
        \item \textbf{Filesystem trees}: Manage directories and files within subvolumes
        \item \textbf{Extent tree}: Manages allocation of data blocks
        \item \textbf{Chunk tree}: Maps logical chunks to physical chunks
        \item \textbf{Device tree}: Manages multiple storage devices
        \item \textbf{Checksum tree}: Stores checksums for data integrity
    \end{itemize}
\end{theorem}

\begin{corollary}{BtrFS Advanced Features}
    BtrFS provides numerous advanced features:
    \begin{itemize}
        \item \textbf{Efficient small file storage}: Store file data inline within inodes
        \item \textbf{Data integrity}: Checksums for both data and metadata
        \item \textbf{Compression}: ZLIB (strong) or LZO (fast) compression at extent level
        \item \textbf{Multi-device support}: RAID 0, 1, 10 via logical chunks
        \item \textbf{Dynamic device management}: Add, remove, and rebalance devices online
        \item \textbf{Writable snapshots}: Memory-efficient clones as "first-class citizens"
        \item \textbf{Online operations}: Resize, defragmentation, extent relocation
    \end{itemize}
\end{corollary}

\begin{example2}{BtrFS Setup and Management}
    Setting up and using BtrFS features:
    
\begin{lstlisting}[language=bash, style=basesmol]
# Create BtrFS filesystem on a single device
sudo mkfs.btrfs /dev/sdb1
# Mount the filesystem
sudo mount /dev/sdb1 /mnt/btrfs
# Create subvolumes
sudo btrfs subvolume create /mnt/btrfs/home
sudo btrfs subvolume create /mnt/btrfs/var
# List subvolumes
sudo btrfs subvolume list /mnt/btrfs
# Create a snapshot of a subvolume
sudo btrfs subvolume snapshot /mnt/btrfs/home /mnt/btrfs/home-snapshot
# Add another device to the filesystem
sudo btrfs device add /dev/sdc1 /mnt/btrfs
# Balance data across devices
sudo btrfs balance start -draid1 -mraid1 /mnt/btrfs
# Check filesystem status
sudo btrfs filesystem show /mnt/btrfs
sudo btrfs filesystem usage /mnt/btrfs
# Enable compression
sudo mount -o remount,compress=zlib /mnt/btrfs
# Defragment with compression
sudo btrfs filesystem defragment -czlib /mnt/btrfs/home
# Check for errors
sudo btrfs scrub start /mnt/btrfs
sudo btrfs scrub status /mnt/btrfs
\end{lstlisting}
\end{example2}



\subsection{RAID (Redundant Array of Independent Disks)}

\begin{concept}{RAID Concept}
    RAID combines multiple physical drives for performance and/or reliability:
    \begin{itemize}
        \item \textbf{Goal}: Build performant and reliable mass storage systems
        \item \textbf{Methods}: Use multiple disks, parity information, and data striping
        \item \textbf{Benefits}: Improved performance, fault tolerance, or both
        \item \textbf{Implementation}: Hardware RAID (dedicated controllers) or software RAID (OS-based)
    \end{itemize}
\end{concept}

\begin{theorem}{Common RAID Levels}
    Different RAID levels provide various combinations of performance and redundancy:
    \begin{itemize}
        \item \textbf{RAID 0 (Striping)}:
            \begin{itemize}
                \item Data striped across multiple drives
                \item Improved performance, no redundancy
                \item Total capacity equals sum of all drives
                \item Failure of any drive results in total data loss
            \end{itemize}
        \item \textbf{RAID 1 (Mirroring)}:
            \begin{itemize}
                \item Data duplicated on multiple drives
                \item High redundancy, no performance improvement for writes
                \item Total capacity equals smallest drive
                \item Can survive failure of all but one drive
            \end{itemize}
        \item \textbf{RAID 5 (Striping with parity)}:
            \begin{itemize}
                \item Data and parity information striped across drives
                \item Good balance of performance, storage efficiency, and redundancy
                \item Can survive failure of one drive
                \item Requires minimum of three drives
            \end{itemize}
        \item \textbf{RAID 10 (Stripe of mirrors)}:
            \begin{itemize}
                \item Combines RAID 0 and RAID 1
                \item High performance and redundancy
                \item Requires minimum of four drives
                \item Can survive multiple drive failures if not in same mirror set
            \end{itemize}
    \end{itemize}
\end{theorem}

\raggedcolumns
\columnbreak

\subsection{Union File Systems}

\begin{concept}{Union File Systems Concept}
    Union file systems overlay multiple file systems to create a unified view:
    \begin{itemize}
        \item \textbf{Purpose}: Combine read-only and read-write filesystems
        \item \textbf{Original motivation}: Enable writing on read-only media (LiveCDs, DVDs)
        \item \textbf{Modern applications}: Container technologies, software distribution
        \item \textbf{Key implementations}: AuFS (Another Union File System), OverlayFS
    \end{itemize}
\end{concept}

\begin{definition}{OverlayFS}
    OverlayFS is the mainline Linux union filesystem:
    \begin{itemize}
        \item \textbf{Components}:
            \begin{itemize}
                \item \textbf{lowerdir}: Read-only base layer(s)
                \item \textbf{upperdir}: Read-write layer for modifications
                \item \textbf{workdir}: Working directory for atomic operations
                \item \textbf{merged}: Combined view of all layers
            \end{itemize}
        \item \textbf{Operation}:
            \begin{itemize}
                \item All modifications go to the upper layer
                \item Lower layers remain unchanged
                \item Files in upper layer "shadow" files in lower layers
                \item Deletions marked with "whiteout" files
            \end{itemize}
        \item \textbf{Use cases}: Docker containers, software packaging, live systems
    \end{itemize}
\end{definition}

\begin{example2}{OverlayFS Setup}
    Creating and using an overlay filesystem:
    
\begin{lstlisting}[language=bash, style=basesmol]
# Create directory structure
mkdir -p /tmp/overlay/{lower,upper,work,merged}
# Populate lower directory (read-only base)
echo "Original file from lower" > /tmp/overlay/lower/file1.txt
echo "Lower layer file" > /tmp/overlay/lower/file2.txt
# Create the overlay mount
sudo mount -t overlay overlay \ -o lowerdir=/tmp/overlay/lower,upperdir=/tmp/overlay/upper,workdir=/tmp/overlay/work \ /tmp/overlay/merged

# View merged contents
ls -la /tmp/overlay/merged/
# Modify a file (goes to upper layer)
echo "Modified content" > /tmp/overlay/merged/file1.txt
# Create a new file (goes to upper layer)
echo "New file in upper" > /tmp/overlay/merged/file3.txt
# Check what's in each layer
echo "=== Lower layer ==="
ls -la /tmp/overlay/lower/
echo "=== Upper layer ==="
ls -la /tmp/overlay/upper/
echo "=== Merged view ==="
ls -la /tmp/overlay/merged/
# Delete a file from lower layer (creates whiteout)
rm /tmp/overlay/merged/file2.txt
# Check for whiteout file
ls -la /tmp/overlay/upper/
# Unmount overlay
sudo umount /tmp/overlay/merged
\end{lstlisting}
\end{example2}

\begin{KR}{Working with Advanced Filesystems}
    \paragraph{Partition and format management}
    \begin{itemize}
        \item Create GPT partition table: \texttt{sudo parted /dev/sdb mklabel gpt}
        \item Create partition: \texttt{sudo parted /dev/sdb mkpart primary 0\% 100\%}
        \item Format with different filesystems:
            \begin{itemize}
                \item ext4: \texttt{sudo mkfs.ext4 /dev/sdb1}
                \item XFS: \texttt{sudo mkfs.xfs /dev/sdb1}
                \item BtrFS: \texttt{sudo mkfs.btrfs /dev/sdb1}
            \end{itemize}
    \end{itemize}
    
    \paragraph{Filesystem mounting and options}
    \begin{itemize}
        \item Mount with compression: \texttt{sudo mount -o compress=zlib /dev/sdb1 /mnt}
        \item Mount with specific options: \texttt{sudo mount -o noatime,nodiratime /dev/sdb1 /mnt}
        \item Check mount options: \texttt{mount | grep /mnt}
        \item Add to fstab: \texttt{echo '/dev/sdb1 /mnt btrfs defaults,compress=zlib 0 2' | sudo tee -a /etc/fstab}
    \end{itemize}
    
    \paragraph{Performance testing}
    \begin{itemize}
        \item Test sequential write: \texttt{dd if=/dev/zero of=/mnt/testfile bs=1M count=1000}
        \item Test random I/O: \texttt{fio --name=random-rw --ioengine=posix --rw=randrw --bs=4k --numjobs=1 --size=1g --runtime=60 --time\_based --filename=/mnt/testfile}
        \item Monitor I/O: \texttt{iotop -a}
        \item Check compression ratio: \texttt{btrfs filesystem usage /mnt}
    \end{itemize}
    
    \paragraph{Filesystem maintenance}
    \begin{itemize}
        \item Check filesystem: \texttt{sudo fsck /dev/sdb1}
        \item Defragment (ext4): \texttt{sudo e4defrag /mnt}
        \item Defragment (BtrFS): \texttt{sudo btrfs filesystem defragment /mnt}
        \item Balance (BtrFS): \texttt{sudo btrfs balance start /mnt}
        \item Scrub (BtrFS): \texttt{sudo btrfs scrub start /mnt}
    \end{itemize}
\end{KR}

\begin{example2}{Comparing Filesystem Performance}
    Testing different filesystems for performance characteristics:
    
\begin{lstlisting}[language=bash, style=basesmol]
# Create test partitions
sudo parted /dev/sdb mklabel gpt
sudo parted /dev/sdb mkpart primary 0% 33%
sudo parted /dev/sdb mkpart primary 33% 66%
sudo parted /dev/sdb mkpart primary 66% 100%
# Format with different filesystems
sudo mkfs.ext4 /dev/sdb1
sudo mkfs.xfs /dev/sdb2
sudo mkfs.btrfs -f /dev/sdb3
# Create mount points
sudo mkdir -p /mnt/{ext4,xfs,btrfs}
# Mount filesystems
sudo mount /dev/sdb1 /mnt/ext4
sudo mount /dev/sdb2 /mnt/xfs
sudo mount -o compress=zlib /dev/sdb3 /mnt/btrfs
# Test file creation performance
for fs in ext4 xfs btrfs; do
    echo "Testing $fs file creation..."
    time sh -c "for i in {1..1000}; do touch /mnt/$fs/file_$i; done"
done
# Test large file write performance
for fs in ext4 xfs btrfs; do
    echo "Testing $fs large file write..."
    time dd if=/dev/zero of=/mnt/$fs/bigfile bs=1M count=500 2>/dev/null
done
# Test compression ratio (BtrFS only)
echo "=== BtrFS compression analysis ==="
sudo btrfs filesystem usage /mnt/btrfs
# Clean up
sudo umount /mnt/{ext4,xfs,btrfs}
\end{lstlisting}
\end{example2}

\begin{example2}{Container Storage with OverlayFS}
    Demonstrating container-like storage layering:
    
\begin{lstlisting}[language=bash, style=basesmol]
# Create base image layer
mkdir -p /tmp/container/{base,app,runtime,merged,work}

# Base layer (OS files)
echo "#!/bin/bash" > /tmp/container/base/entrypoint.sh
echo "echo 'Hello from base layer'" >> /tmp/container/base/entrypoint.sh
chmod +x /tmp/container/base/entrypoint.sh

# Application layer
echo "config_file=app.conf" > /tmp/container/app/app.config
echo "#!/bin/bash" > /tmp/container/app/start-app.sh
echo "echo 'Starting application...'" >> /tmp/container/app/start-app.sh
chmod +x /tmp/container/app/start-app.sh

# Create overlay combining base and app layers
sudo mount -t overlay overlay \ -o lowerdir=/tmp/container/app:/tmp/container/base,upperdir=/tmp/container/runtime,workdir=/tmp/container/work \/tmp/container/merged

# View combined filesystem
ls -la /tmp/container/merged/

# Make runtime changes (goes to runtime layer)
echo "Runtime log entry" > /tmp/container/merged/runtime.log
echo "user_data=changed" > /tmp/container/merged/app.config

# Show layering
echo "=== Base layer ==="
ls -la /tmp/container/base/

echo "=== App layer ==="
ls -la /tmp/container/app/

echo "=== Runtime layer ==="
ls -la /tmp/container/runtime/

echo "=== Merged view ==="
ls -la /tmp/container/merged/

# Unmount
sudo umount /tmp/container/merged
\end{lstlisting}
\end{example2}