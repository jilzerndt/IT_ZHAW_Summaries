\section{Input / Output - Part II}

\subsection{Building and Using a Custom Linux Kernel}

\begin{concept}{Custom Kernel Motivation}\\
    There are various reasons to build a custom kernel:
    \begin{itemize}
        \item Create a minimalist kernel (disable unused features, load needed ones as modules)
        \item Add custom OS features for specific requirements
        \item Support special hardware that may not be in the standard kernel
        \item Optimize for specific workloads or hardware platforms
        \item Learn about kernel internals and development processes
    \end{itemize}
\end{concept}

\begin{definition}{Linux Kernel Development Process}\\
    Linux follows a time-based release process:
    \begin{itemize}
        \item New major kernel releases every 2-3 months (e.g., 5.11, 5.12)
        \item Development cycle phases:
            \begin{itemize}
                \item Merge window (first two weeks): New features merged into mainline
                \item Release candidates (weekly): Bug fixes only, no new features
                \item Final release: After several release candidates
            \end{itemize}
        \item Long-term support (LTS) kernels receive updates for extended periods
        \item Community development model with maintainers for various subsystems
    \end{itemize}
\end{definition}

\begin{definition}{Choosing a Kernel Version}\\
    When building a custom kernel, version selection matters:
    \begin{itemize}
        \item Mainline: Latest development version (might be unstable)
        \item Stable: Recent release with proven stability
        \item Long-term: Longer support period (good for production systems)
        \item Distribution-specific: Modified by Linux distributions
    \end{itemize}
    
    Version numbers indicate:
    \begin{itemize}
        \item First number: Major version (rarely changes)
        \item Second number: Minor version (major features)
        \item Third number: Patch level (bug fixes and minor improvements)
    \end{itemize}
\end{definition}

\begin{KR}{Building a Custom Linux Kernel}\\
    \paragraph{Getting the source code}
    \begin{itemize}
        \item Download from kernel.org: \texttt{wget https://cdn.kernel.org/pub/linux/kernel/v5.x/linux-5.xx.tar.xz}
        \item Extract: \texttt{tar xf linux-5.xx.tar.xz}
        \item Change directory: \texttt{cd linux-5.xx}
    \end{itemize}
    
    \paragraph{Installing build dependencies}
    \begin{itemize}
        \item On Debian/Ubuntu: \texttt{sudo apt-get install build-essential gcc bc bison flex libssl-dev libncurses5-dev libelf-dev}
    \end{itemize}
    
    \paragraph{Configuring the kernel}
    \begin{itemize}
        \item Start from existing config (recommended):
            \begin{itemize}
                \item Current running kernel: \texttt{cp /boot/config-\$(uname -r) ./.config}
                \item Distribution default: \texttt{make defconfig}
            \end{itemize}
        \item Update config for new options: \texttt{make oldconfig}
        \item Interactive configuration tools:
            \begin{itemize}
                \item Text-based menu: \texttt{make menuconfig}
                \item GUI-based: \texttt{make xconfig} or \texttt{make gconfig}
            \end{itemize}
        \item Important configuration options:
            \begin{itemize}
                \item Custom version name: \texttt{CONFIG\_LOCALVERSION}
                \item Module support: \texttt{CONFIG\_MODULES}
                \item CPU and architecture options
                \item Device drivers
                \item File systems
            \end{itemize}
    \end{itemize}
    
    \paragraph{Building the kernel}
    \begin{itemize}
        \item Compile: \texttt{make -j\$(nproc)}
        \item Build modules: \texttt{make modules}
        \item Create Debian/Ubuntu packages: \texttt{make -j\$(nproc) deb-pkg}
        \item For RPM-based systems: \texttt{make -j\$(nproc) rpm-pkg}
    \end{itemize}
    
    \paragraph{Installing the kernel}
    \begin{itemize}
        \item Debian packages: \texttt{sudo dpkg -i ../linux-*.deb}
        \item RPM packages: \texttt{sudo rpm -ivh ../linux-*.rpm}
        \item Manual installation:
            \begin{itemize}
                \item Install modules: \texttt{sudo make modules\_install}
                \item Install kernel: \texttt{sudo make install}
            \end{itemize}
        \item Update bootloader: \texttt{sudo update-grub} (GRUB)
    \end{itemize}
    
    \paragraph{Booting the new kernel}
    \begin{itemize}
        \item Reboot: \texttt{sudo reboot}
        \item Select the new kernel at the bootloader menu
        \item Verify running kernel: \texttt{uname -a}
    \end{itemize}
\end{KR}

\subsection{Linux Kernel Modules}

\begin{definition}{Kernel Modules Concept}\\
    Kernel modules extend kernel functionality without rebuilding:
    \begin{itemize}
        \item Loadable code that can be added to or removed from a running kernel
        \item Allow dynamic extension of kernel capabilities
        \item Provide device drivers, filesystem drivers, system calls, etc.
        \item Reduce the size of the base kernel
        \item Enable support for hardware that is hot-pluggable
        \item Support for special features used in specific applications
    \end{itemize}
\end{definition}

\begin{definition}{Module Structure}\\
    A kernel module follows a specific structure:
    \begin{itemize}
        \item Must include necessary kernel headers
        \item Initialization function (\texttt{init\_module()} or custom named)
            \begin{itemize}
                \item Called when the module is loaded
                \item Sets up resources, registers with kernel subsystems
                \item Returns success (0) or error code
            \end{itemize}
        \item Cleanup function (\texttt{cleanup\_module()} or custom named)
            \begin{itemize}
                \item Called when the module is unloaded
                \item Releases resources, unregisters from kernel subsystems
            \end{itemize}
        \item Uses \texttt{MODULE\_LICENSE()} macro to specify license
        \item Can include other macros for author, description, etc.
    \end{itemize}
\end{definition}

\begin{code}{Hello World Kernel Module}\\
    Minimal kernel module example:
    
\begin{lstlisting}[language=C, style=basesmol]
#include <linux/module.h>    /* Needed by all modules */
#include <linux/kernel.h>    /* Needed for KERN_INFO */
#include <linux/init.h>      /* Needed for macros */

MODULE_LICENSE("GPL");
MODULE_AUTHOR("Your Name");
MODULE_DESCRIPTION("A simple Hello World module");

static int __init hello_init(void)
{
    printk(KERN_INFO "Hello, World!\n");
    return 0;    /* Success */
}

static void __exit hello_exit(void)
{
    printk(KERN_INFO "Goodbye, World!\n");
}

module_init(hello_init);
module_exit(hello_exit);
\end{lstlisting}

    Key components:
    \begin{itemize}
        \item \texttt{module\_init} and \texttt{module\_exit} macros register functions
        \item \texttt{printk} is the kernel's version of printf
        \item \texttt{KERN\_INFO} sets the message priority level
        \item \texttt{MODULE\_LICENSE} declares the license (important for symbol exports)
    \end{itemize}
\end{code}

\begin{definition}{Module Building Process}\\
    Building a kernel module requires:
    \begin{itemize}
        \item Pre-built kernel with module support
        \item Module source code
        \item Makefile defining the build process
        \item Build tools and headers
    \end{itemize}
    
    The build process:
    \begin{itemize}
        \item Kbuild system builds \texttt{<module\_name>.o} from source
        \item Links to create \texttt{<module\_name>.ko} (kernel object)
        \item Kernel modules must be built against the same kernel version they will run on
        \item Distribution-specific kernel headers needed for module compatibility
    \end{itemize}
\end{definition}

\begin{code}{Module Makefile}\\
    Example Makefile for a kernel module:
    
\begin{lstlisting}[style=basesmol]
# Define the module name
obj-m := hello.o

# For standalone module build
all:
	make -C /lib/modules/$(shell uname -r)/build M=$(PWD) modules

# Clean up
clean:
	make -C /lib/modules/$(shell uname -r)/build M=$(PWD) clean
\end{lstlisting}

    For multiple source files:
    
\begin{lstlisting}[style=basesmol]
# Module with multiple source files
hello-y := hello_main.o hello_func.o
obj-m := hello.o

# For standalone module build
all:
	make -C /lib/modules/$(shell uname -r)/build M=$(PWD) modules

# Clean up
clean:
	make -C /lib/modules/$(shell uname -r)/build M=$(PWD) clean
\end{lstlisting}
\end{code}

\begin{KR}{Working with Kernel Modules}\\
    \paragraph{Building a module}
    \begin{itemize}
        \item Create module source file
        \item Create Makefile
        \item Build module: \texttt{make}
        \item Result: \texttt{<module\_name>.ko} file
    \end{itemize}
    
    \paragraph{Loading and unloading modules}
    \begin{itemize}
        \item List loaded modules: \texttt{lsmod}
        \item Insert module: \texttt{sudo insmod <module\_name>.ko}
        \item Remove module: \texttt{sudo rmmod <module\_name>}
        \item Load module with dependencies: \texttt{sudo modprobe <module\_name>}
        \item Unload module and dependencies: \texttt{sudo modprobe -r <module\_name>}
        \item View kernel messages: \texttt{dmesg}
    \end{itemize}
    
    \paragraph{Module information}
    \begin{itemize}
        \item Show module info: \texttt{modinfo <module\_name>.ko}
        \item Check if module is loaded: \texttt{lsmod | grep <module\_name>}
        \item Display module parameters: \texttt{systool -vm <module\_name>}
        \item View module details in sysfs: \texttt{ls -l /sys/module/<module\_name>/}
    \end{itemize}
    
    \paragraph{Module autoloading}
    \begin{itemize}
        \item Install module: \texttt{sudo make modules\_install}
        \item Update module dependencies: \texttt{sudo depmod -a}
        \item Configure autoloading: \texttt{echo "<module\_name>" | sudo tee /etc/modules-load.d/<module\_name>.conf}
    \end{itemize}
\end{KR}

\subsection{Character Device Drivers}

\begin{definition}{Character Device Drivers}\\
    Character device drivers are a common type of Linux device driver:
    \begin{itemize}
        \item Handle devices that transfer data as a stream of bytes
        \item Support operations like read, write, open, release, ioctl
        \item Examples: serial ports, keyboards, mice, sensors
        \item Appear as files in /dev with major and minor numbers
        \item Major number identifies the driver
        \item Minor number identifies the specific device
    \end{itemize}
\end{definition}

\begin{code}{Basic Character Device Driver}\\
    Structure of a simple character device driver:
    
\begin{lstlisting}[language=C, style=basesmol]
#include <linux/kernel.h>
#include <linux/module.h>
#include <linux/fs.h>
#include <linux/uaccess.h>

MODULE_LICENSE("GPL");

/* Prototypes */
static int device_open(struct inode *, struct file *);
static int device_release(struct inode *, struct file *);
static ssize_t device_read(struct file *, char *, size_t, loff_t *);
static ssize_t device_write(struct file *, const char *, size_t, loff_t *);

#define DEVICE_NAME "chardev"
#define BUF_LEN 80

/* Global variables */
static int Major;                /* Major number assigned */
static int Device_Open = 0;      /* Is device open? */
static char msg[BUF_LEN];        /* Message for the device */
static char *msg_Ptr;

static struct file_operations fops = {
    .read = device_read,
    .write = device_write,
    .open = device_open,
    .release = device_release
};

/* Initialization function */
int init_module(void)
{
    Major = register_chrdev(0, DEVICE_NAME, &fops);
    if (Major < 0) {
        printk(KERN_ALERT "Failed to register with %d\n", Major);
        return Major;
    }

    printk(KERN_INFO "Major number assigned: %d\n", Major);
    printk(KERN_INFO "Create device with: 'mknod /dev/%s c %d 0'\n", 
           DEVICE_NAME, Major);
    return 0;
}

/* Cleanup function */
void cleanup_module(void)
{
    unregister_chrdev(Major, DEVICE_NAME);
}

/* Device methods */
static int device_open(struct inode *inode, struct file *file)
{
    static int counter = 0;
    
    if (Device_Open)
        return -EBUSY;
        
    Device_Open++;
    sprintf(msg, "Called device_open %d times\n", counter++);
    msg_Ptr = msg;
    try_module_get(THIS_MODULE);
    
    return 0;
}

static int device_release(struct inode *inode, struct file *file)
{
    Device_Open--;
    module_put(THIS_MODULE);
    return 0;
}

static ssize_t device_read(struct file *filp, char *buffer, 
                           size_t length, loff_t *offset)
{
    int bytes_read = 0;
    
    /* End of message */
    if (*msg_Ptr == 0)
        return 0;
        
    /* Transfer data to user space */
    while (length && *msg_Ptr) {
        put_user(*(msg_Ptr++), buffer++);
        length--;
        bytes_read++;
    }
    
    return bytes_read;
}

static ssize_t device_write(struct file *filp, const char *buff, 
                            size_t len, loff_t *off)
{
    printk(KERN_ALERT "Operation not supported\n");
    return -EINVAL;
}
\end{lstlisting}

    Key components:
    \begin{itemize}
        \item \texttt{file\_operations} structure defines callbacks for file operations
        \item \texttt{register\_chrdev} allocates a major number
        \item Device methods handle open, read, write, and close operations
        \item \texttt{put\_user} safely copies data from kernel to user space
        \item Reference counting with \texttt{try\_module\_get} and \texttt{module\_put}
    \end{itemize}
\end{code}

\begin{example2}{Building\, Installing\, and Using a Character Device Module}\\
    Complete workflow for a character device driver:
    
\begin{lstlisting}[language=bash, style=basesmol]
# 1. Create source file (chardev.c) and Makefile as shown above

# 2. Build the module
make

# 3. Load the module
sudo insmod chardev.ko

# 4. Check kernel messages for major number
dmesg | tail

# 5. Create device node (assuming major number is 250)
sudo mknod /dev/chardev c 250 0
sudo chmod 666 /dev/chardev

# 6. Test reading from the device
cat /dev/chardev

# 7. Check module information in sysfs
ls -l /sys/module/chardev/
cat /sys/module/chardev/parameters/* 2>/dev/null

# 8. Unload the module when done
sudo rmmod chardev

# 9. Clean up the device node
sudo rm /dev/chardev
\end{lstlisting}

    This demonstrates:
    \begin{itemize}
        \item Building and loading a custom character device driver
        \item Creating a device node with appropriate permissions
        \item Interacting with the device through standard file operations
        \item Examining module information through sysfs
        \item Proper cleanup when the module is no longer needed
    \end{itemize}
\end{example2}

\begin{KR}{Linux Kernel Module Debugging}\\
    \paragraph{Using printk}
    \begin{itemize}
        \item Add debug messages: \texttt{printk(KERN\_DEBUG "Debug: \%d\\n", value);}
        \item Set console log level: \texttt{echo 7 > /proc/sys/kernel/printk}
        \item View kernel messages: \texttt{dmesg}
        \item Follow kernel messages: \texttt{dmesg -w}
        \item Clear buffer: \texttt{dmesg -c}
    \end{itemize}
    
    \paragraph{Debug filesystem}
    \begin{itemize}
        \item Mount debugfs: \texttt{mount -t debugfs none /sys/kernel/debug}
        \item Create debug files in your module:
            \begin{itemize}
                \item Include \texttt{<linux/debugfs.h>}
                \item Create directory: \texttt{debugfs\_create\_dir()}
                \item Create files: \texttt{debugfs\_create\_file()}
            \end{itemize}
        \item Access from user space: \texttt{cat /sys/kernel/debug/mymodule/myfile}
    \end{itemize}
    
    \paragraph{Module parameters}
    \begin{itemize}
        \item Define parameters: \texttt{module\_param(name, type, permissions);}
        \item Load with parameters: \texttt{insmod mymodule.ko debug=1}
        \item Change at runtime: \texttt{echo 1 > /sys/module/mymodule/parameters/debug}
    \end{itemize}
    
    \paragraph{Kernel/system crashes}
    \begin{itemize}
        \item Configure kdump: Install and configure \texttt{kdump-tools}
        \item Analyze crash dumps with \texttt{crash} utility
        \item Check system logs after reboot: \texttt{journalctl -b -1}
    \end{itemize}
\end{KR}

\begin{example2}{I/O Performance Testing with Custom I/O Scheduler}\\
    Testing I/O performance with different schedulers:
    
\begin{lstlisting}[language=bash, style=basesmol]
# 1. Check available I/O schedulers
cat /sys/block/sda/queue/scheduler

# 2. Test performance with default scheduler
echo 3 > /proc/sys/vm/drop_caches  # Clear cache
dd if=/dev/zero of=testfile bs=1M count=1000 oflag=direct
rm testfile

# 3. Change to a different scheduler (e.g., BFQ)
echo bfq > /sys/block/sda/queue/scheduler

# 4. Test performance with new scheduler
echo 3 > /proc/sys/vm/drop_caches  # Clear cache
dd if=/dev/zero of=testfile bs=1M count=1000 oflag=direct
rm testfile

# 5. Test with different I/O priorities
# Start a background write process
dd if=/dev/zero of=bg_file bs=1M count=2000 oflag=direct &
BG_PID=$!

# Run a foreground process with higher priority
ionice -c2 -n0 dd if=/dev/zero of=fg_file bs=1M count=500 oflag=direct

# Clean up
kill $BG_PID
rm bg_file fg_file

# 6. Return to the default scheduler
echo deadline > /sys/block/sda/queue/scheduler
\end{lstlisting}
\end{example2}