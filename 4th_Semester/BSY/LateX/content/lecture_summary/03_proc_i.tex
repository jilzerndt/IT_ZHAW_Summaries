\section{Processes and Threads}

\subsection{Processes}

\subsubsection{Process Model}

\begin{definition}{Programs vs. Processes}
    A program is fundamentally different from a process:
    \vspace{1mm}\\
    \begin{minipage}{0.5\linewidth}
    \textbf{Program}: A compiled executable (binary)
            \begin{itemize}
                \item Set of CPU instructions and related data
                \item Targets a specific platform (OS + hardware)
                \item Static entity stored on disk
            \end{itemize}
        \end{minipage}
    \begin{minipage}{0.5\linewidth}
        \textbf{Process}: An active instance of a program
            \begin{itemize}
                \item Has a program, input, output, and state
                \item Dynamic entity in memory
                \item Multiple instances of the same program can run as separate processes
            \end{itemize}
    \end{minipage}
\end{definition}

\mult{2}

\begin{definition}{Process Characteristics}\\
    Processes have several important characteristics:
    \begin{itemize}
        \item Can run sequentially or in parallel (multi-tasking)
        \item Selected for execution by the OS scheduler
        \item Associated with an owner \\ (defines access privileges)
        \item Run within an environment \\ (environment variables)
        \item Can run in foreground (interactive) \\ or background (non-interactive)
        \item Can be user processes or system processes
    \end{itemize}
\end{definition}

\begin{example2}{Viewing process information in Linux}
\begin{lstlisting}[language=bash, style=basesmol]
# List all processes with details
ps aux
# Interactive process viewer
top
# Show environment variables
env
# Run a process in the background
wc /dev/zero &
# List background jobs
jobs
# Bring a background job to foreground
fg %job_number
\end{lstlisting}
\end{example2}

\multend

\subsubsection{Process Creation}

\mult{2}

\begin{definition}{Process Creation by the OS}\\
    When creating a process, the OS performs several operations:
    \begin{itemize}
        \item Loads the executable into RAM
        \item Sets up the memory map
        \item Updates scheduler and process table entries
        \item Sets program counter and stack pointer
        \item Switches from system mode to user mode
    \end{itemize}
\end{definition}

\begin{definition}{Memory Layout}\\
    Process memory is typically divided into several segments:
    \begin{itemize}
        \item \textbf{Text}: Contains CPU instructions and constant data
        \item \textbf{Data}: Global and static variables
            \begin{itemize}
                \item Initialized data segment: Pre-initialized variables
                \item Uninitialized data segment (BSS): Zero-initialized variables
            \end{itemize}
        \item \textbf{Stack}: Local variables and function call information (LIFO structure)
        \item \textbf{Heap}: Dynamically allocated memory (controlled by the programmer)
    \end{itemize}
\end{definition}

\begin{definition}{Process Creation Events}\\
    Processes can be created in several ways:
    \begin{itemize}
        \item System boot (initial processes)
        \item User request (launching an application)
        \item Process creating another process (fork/exec)
        \item Scheduled creation (cron jobs)
        \item System request (responding to interrupts)
    \end{itemize}
\end{definition}

\begin{definition}{Parent-Child Process Relationship}\\
    When a process creates another process:
    \begin{itemize}
        \item The creating process becomes the parent
        \item The new process becomes the child
        \item Child's memory map is initially a copy of the parent's memory map
        \item Two memory handling approaches:
            \begin{itemize}
                \item Distinct address space: Separate memory regions for parent and child
                \item Copy-on-write: Memory shared until a change is made by either process
            \end{itemize}
        \item Forms a process hierarchy (starting with PID 1)
    \end{itemize}
\end{definition}

\begin{example2}{Linux Process Creation}\\
    Linux terminal command execution involves process creation:
\begin{lstlisting}[language=bash, style=basesmol]
# When you run a command in the terminal:
ls -la

# The shell:
# 1. Forks itself (duplicate the process)
# 2. Child process executes the 'ls' binary
# 3. Parent waits for child to complete
# 4. Shell continues after child terminates
\end{lstlisting}

    This process can be visualized with \texttt{pstree} showing the parent-child relationships.
\end{example2}

\multend

\subsubsection{Process Termination}

\mult{2}

\begin{definition}{Process Termination}\\
    A process can terminate in several ways:
    \begin{itemize}
        \item Voluntary normal exit (job completed, return from main())
        \item Voluntary error exit (required resource unavailable)
        \item Involuntary error exit (segmentation fault, division by zero)
        \item Termination by another process (kill signal)
    \end{itemize}
\end{definition}

\begin{example}
    Process termination in Linux:
    \begin{lstlisting}[language=bash, style=basesmol]
    # Start a process
    wc /dev/zero &
    
    # Get its process ID
    pidof wc
    
    # Terminate the process
    kill -9 <PID>
    \end{lstlisting}
\end{example}

\multend

\subsubsection{Process States}

\mult{2}

\begin{definition}{Process States}\\
    Each process has a life-cycle with specific states:
    \begin{itemize}
        \item \textbf{Running}: The process is currently executing on the CPU
        \item \textbf{Ready}: The process is ready to run but waiting for CPU allocation
        \item \textbf{Blocked}: The process is unable to run (waiting for an event or resource)
            \begin{itemize}
                \item Dependencies not met for running
                \item Waiting for external resource
                \item Waiting for I/O completion
                \item Sleeping
                \item Under job control or debugger
            \end{itemize}
    \end{itemize}
\end{definition}



\begin{concept}{User Mode vs. System Mode}\\
    CPU supports two execution modes:
    \begin{itemize}
        \item \textbf{User Mode}: Limited privileges
            \begin{itemize}
                \item Application logic execution
                \item Application data manipulation
                \item Restricted access to hardware and system resources
            \end{itemize}
        \item \textbf{System Mode} (Kernel Mode): Full privileges
            \begin{itemize}
                \item System management operations
                \item Hardware access
                \item Interrupt handling
                \item Device management
            \end{itemize}
    \end{itemize}
\end{concept}

\begin{definition}{Process State Changes}\\
    Processes change state for various reasons:
    \begin{itemize}
        \item \textbf{Timer expiration}: CPU allocation time ended
        \item \textbf{Interrupt}: Hardware/resource calls for service
        \item \textbf{Page fault}: Data not in memory, requires disk access
        \item \textbf{System call}: Explicit OS service request
    \end{itemize}
    
    State changes involve a switch between user mode and system (kernel) mode.
\end{definition}

\begin{concept}{Context Switch vs. Mode Switch}\\
    Two types of switches occur during system operation:
    \begin{itemize}
        \item \textbf{Mode Switch}: Transition between user mode and kernel mode
            \begin{itemize}
                \item Occurs during system calls, interrupts, exceptions
                \item Same process continues execution in different mode
                \item No scheduler involvement
            \end{itemize}
        \item \textbf{Context Switch}: Changing from one process to another
            \begin{itemize}
                \item Involves saving the state of the current process
                \item Loading the state of another process
                \item Typically involves mode switches (user $\rightarrow$ kernel $\rightarrow$ user)
                \item Requires scheduler involvement
            \end{itemize}
    \end{itemize}
\end{concept}

\multend

\raggedcolumns
\columnbreak

\subsubsection{Process Management}

\begin{definition}{Process Control Block}\\
    The OS maintains a Process Control Block (PCB) for each process:
    \begin{itemize}
        \item Process identification (PID, UID, GID)
        \item Process state information
        \item Program counter and CPU registers
        \item CPU scheduling information (priority)
        \item Memory management information
        \item I/O status information
        \item Accounting information
    \end{itemize}
    
    In Linux, PCBs are implemented as \texttt{task\_struct} entries in the process table.
\end{definition}

\begin{KR}{Process Management in Linux}
    \paragraph{Viewing process information}
    \begin{itemize}
        \item \texttt{ps aux} - List all processes
        \item \texttt{ps -ef} - List processes in full format
        \item \texttt{ps -eLF} - List processes and threads
        \item \texttt{top} - Interactive process viewer
        \item \texttt{top -H} - Show threads in top
        \item \texttt{pstree} - Display process tree
    \end{itemize}
    
    \paragraph{Creating processes}
    \begin{itemize}
        \item Write a C program using \texttt{fork()} to create a child process
        \item Use \texttt{getpid()} to retrieve the process ID
        \item Use \texttt{sleep()} to pause execution
    \end{itemize}
    
    \paragraph{Identifying zombie processes}
    \begin{itemize}
        \item Create a parent process that doesn't wait for its child
        \item Child terminates while parent continues running
        \item Check process state with \texttt{ps} (state "Z" indicates zombie)
    \end{itemize}
\end{KR}

\begin{example2}{Creating Processes in C}\\
    Simple process creation with fork():
    
\begin{lstlisting}[language=C, style=basesmol]
#include <stdio.h>
#include <stdlib.h>
#include <unistd.h>

int main() {
    pid_t pid = fork();
    
    if (pid < 0) {
        // Error
        fprintf(stderr, "Fork failed\n");
        return 1;
    } else if (pid == 0) {
        // Child process
        printf("Child process: PID = %d\n", getpid());
        sleep(5);
    } else {
        // Parent process
        printf("Parent process: PID = %d, Child PID = %d\n", getpid(), pid);
        sleep(10);
    }
    
    return 0;
}
\end{lstlisting}
\end{example2}

\subsubsection{Fork Process Trees}

\begin{KR}{Fork() Process Tree Analysis}
    \paragraph{Understanding fork() behavior}
    \begin{itemize}
        \item fork() creates an exact copy of the calling process
        \item Returns child PID to parent, 0 to child
        \item fork() > 0 is true only in parent process
        \item Each fork() doubles the number of processes
    \end{itemize}
    
    \paragraph{Analysis method}
    \begin{itemize}
        \item Draw a process tree starting with the original process
        \item For each fork(), branch into parent and child
        \item Track which path each process takes through if/else statements
        \item Count total processes and specific execution paths
    \end{itemize}

    \paragraph{Trace execution paths}
    \begin{itemize}
        \item Parent: fork() returns child PID (> 0)
        \item Child: fork() returns 0
        \item Follow if-else logic for each process
    \end{itemize}
    
    \paragraph{Counting strategy}
    \begin{itemize}
        \item Start with 1 process (the original)
        \item Each successful fork() adds 1 new process
        \item Count processes that reach specific code sections
        \item Be careful with nested forks and conditional execution
    \end{itemize}

    \paragraph{Count results}
    \begin{itemize}
        \item Total processes = original + all children created
        \item Count specific outputs by tracing paths to printf statements
    \end{itemize}
\end{KR}

\begin{example2}{Process Creation Analysis}
    Analyze the following code:
    
\begin{lstlisting}[language=C, style=basesmol]
if (fork() > 0)
    fork();
else {
    fork();
    if (fork() > 0)
        printf("Hello World\n");
}
\end{lstlisting}

    \tcblower

    $$
    \begin{array}{lll}
        f--- & f---| & \\
        \lfloor     & \llcorner ---| & \\
        ---- & f---- & f--- Hello World \\
             & \lfloor      & \llcorner ---| \\
             & ----- & f--- Hello World \\
             &       & \llcorner ---|\\
    \end{array} 
    $$   
    
    \textbf{Process tree and analysis:}
    \begin{itemize}
        \item Original process P forks $\rightarrow$ creates child C1
        \item Parent P (fork() > 0): executes second fork() $\rightarrow$ creates child C2
        \item Child C1 (fork() == 0): executes fork() $\rightarrow$ creates child C3
        \item Child C1: executes second fork() $\rightarrow$ creates child C4, then prints
        \item Child C3: executes second fork() $\rightarrow$ creates child C5, then prints
    \end{itemize}
    
    \textbf{Results:}
    \begin{itemize}
        \item Additional processes created: 5 (C1, C2, C3, C4, C5)
        \item "Hello World" printed: 2 times (by C1 and C3)
    \end{itemize}
\end{example2}

\subsection{Threads}

\mult{2}

\begin{definition}{Threads}\\
    Threads are execution entities within a process:
    \begin{itemize}
        \item Created and owned by a process
        \item Share the address space and all data of the owning process
        \item Allow multiple executions to take place within the same process environment
        \item Enable a process to continue even when some operations would block
        \item Cooperate toward the objective of the owning process
    \end{itemize}
    
    Each thread has:
    \begin{itemize}
        \item Program counter (tracks next instruction)
        \item Registers (hold working variables)
        \item Stack (with frames for procedure calls)
    \end{itemize}
\end{definition}

\begin{concept}{Thread Implementation Approaches}\\
    Threads can be implemented in different ways:
    \begin{itemize}
        \item \textbf{User Space Threading (M:1)}:
            \begin{itemize}
                \item All threads in user space appear as a single process to the OS
                \item Thread functionality provided by a library
                \item Scheduling handled by the process (no mode switch required)
                \item Based on cooperation (threads voluntarily yield CPU)
                \item Disadvantages: Single thread can monopolize CPU time, blocked threads block the entire process, no SMP advantage
            \end{itemize}
        \item \textbf{Kernel-supported Threading (1:1)}:
            \begin{itemize}
                \item One kernel structure per user thread
                \item Kernel schedules all threads
                \item Advantages: Better handling of blocking I/O, full SMP exploitation
                \item Disadvantages: Higher overhead, slower creation/removal, mode switching for scheduling
            \end{itemize}
    \end{itemize}
\end{concept}



\begin{definition}{Kernel Threads}\\
    Kernel threads are special processes that:
    \begin{itemize}
        \item Run exclusively in system (kernel) mode
        \item Have a PID and state like any process
        \item Are listed in the process table but flagged as "kernel thread"
        \item Are scheduled/dispatched like regular processes
        \item Uninterruptible (!! voluntarily yield CPU)
        \item Listen to kernel signals
    \end{itemize}
    
    Kernel threads are organized around working queues and thread pools:
    \begin{itemize}
        \item Working queues ordered by priority
        \item Mapped onto a pool of reusable kernel threads
        \item Number of threads dynamically managed based on workload
        \item Managed by the \texttt{kthread} daemon
    \end{itemize}
\end{definition}

\begin{concept}{Thread Implementation in Linux}\\
    Linux doesn't have a dedicated concept of threads:
    \begin{itemize}
        \item All threads are standard processes (tasks)
        \item A thread is a process that shares resources with other processes
        \item Shared resources can include: address space, file descriptors, sockets, signal handlers, etc.
        \item Implemented using system calls: \texttt{fork()} and \texttt{clone()}
        \item Two implementation frameworks:
            \begin{itemize}
                \item LinuxThreads (legacy)
                \item NPTL (Native POSIX Thread Library, current standard)
            \end{itemize}
    \end{itemize}
\end{concept}

\begin{formula}{Thread Management in Linux}
    \paragraph{Creating threads}
    \begin{itemize}
        \item Use POSIX threads library (pthread)
        \item Include \texttt{<pthread.h>}
        \item Create threads with \texttt{pthread\_create()}
        \item Join threads with \texttt{pthread\_join()}
    \end{itemize}
\end{formula}

\multend

\begin{KR}{POSIX Thread Programming}
    \paragraph{Essential POSIX thread functions}
    \begin{itemize}
        \item pthread\_create(): Create new thread
        \item pthread\_join(): Wait for thread completion
        \item pthread\_exit(): Terminate calling thread
        \item pthread\_detach(): Detach thread (no join needed)
    \end{itemize}
    
    \paragraph{Programming pattern}
    \begin{itemize}
        \item Include pthread.h header
        \item Define thread function with void* signature
        \item Create threads in loop if multiple needed
        \item Always join threads to avoid zombies
        \item Compile with -lpthread flag
    \end{itemize}
\end{KR}

\begin{example2}{Creating Threads in C}\\
    Simple thread creation with POSIX threads:
    
\begin{lstlisting}[language=C, style=basesmol]
#include <stdio.h>
#include <stdlib.h>
#include <pthread.h>
#include <unistd.h>

void* thread_function(void* arg) {
    int thread_id = *(int*)arg;
    printf("Thread %d: running\n", thread_id);
    sleep(3);
    printf("Thread %d: exiting\n", thread_id);
    return NULL;
}

int main() {
    pthread_t threads[2];
    int thread_args[2];
    
    for (int i = 0; i < 2; i++) {
        thread_args[i] = i;
        pthread_create(&threads[i], NULL, thread_function, &thread_args[i]);
        printf("Main: created thread %d\n", i);
    }
    
    // Wait for threads to finish
    for (int i = 0; i < 2; i++) {
        pthread_join(threads[i], NULL);
    }
    
    printf("Main: all threads completed\n");
    return 0;
}
\end{lstlisting}
\end{example2}



\begin{example2}{Thread Programming with POSIX}
    Write a program that creates two threads, both printing "Hello World":
    
    \tcblower
    
\begin{lstlisting}[language=C, style=basesmol]
#include <stdio.h>
#include <pthread.h>

void *hello_world(void *arg) {
    printf("Hello World\n");
    return NULL;
}

int main() {
    pthread_t thread[2];
    
    // Create two threads
    for (int i = 0; i < 2; i++) {
        pthread_create(&thread[i], NULL, hello_world, NULL);
    }
    
    // Wait for both threads to complete
    for (int i = 0; i < 2; i++) {
        pthread_join(thread[i], NULL);
    }
    
    return 0;
}
\end{lstlisting}

    Note: Threads created with pthread\_create() start immediately.
\end{example2}



\subsubsection{Processes vs. Threads}

\begin{theorem}{Linux Process vs. Thread Creation}\\
    Linux offers different system calls for process and thread creation:
    \begin{itemize}
        \item \texttt{fork()}: Creates a child process by duplicating the parent
            \begin{itemize}
                \item Child and parent run in separate memory spaces
            \end{itemize}
        \item \texttt{clone()}: Provides precise control over what execution context is shared
            \begin{itemize}
                \item Allows sharing of address space, file descriptors, signal handlers, etc.
                \item Used to implement threads in Linux
            \end{itemize}
    \end{itemize}
\end{theorem}

\begin{example2}{Process vs. Thread Differences}
    Explain the difference between processes and threads:
    
    \tcblower
    
    \textbf{Solution:}
    The main difference is that processes have their own memory space, while threads (belonging to the same process) share the process memory space.
    
    \begin{minipage}{0.5\linewidth}
    \textbf{Processes:}
    \begin{itemize}
        \item Independent memory spaces
        \item Higher creation/switching overhead
        \item Better isolation and fault tolerance
        \item Communication via IPC mechanisms
    \end{itemize}
    \end{minipage}%
    \begin{minipage}{0.5\linewidth}    
    \textbf{Threads:}
    \begin{itemize}
        \item Shared memory space within process
        \item Lower creation/switching overhead
        \item Faster communication via shared memory
        \item Risk of interference between threads
    \end{itemize}
    \end{minipage}
\end{example2}

\begin{KR}{Analyzing Process vs. Thread Questions}
    \paragraph{Key comparison points}
    \begin{itemize}
        \item Memory organization: separate vs. shared address space
        \item Creation overhead: high vs. low
        \item Communication methods: IPC vs. shared memory
        \item Isolation level: strong vs. weak
        \item Context switching cost: expensive vs. cheap
    \end{itemize}
\end{KR}

\important{ADD EXERCISE 1D SEP07 USER VS KERNEL-LEVEL-THREADS}

