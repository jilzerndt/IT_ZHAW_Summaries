\section{Linux}

\subsection{Terminal Basics}

\mult{2}

\begin{definition}{Terminal}\\
    A terminal is a text-based user interface program that allows users to:
    \begin{itemize}
        \item Accepts text input via a prompt
        \item Interprets text as commands
        \item Executes operations based on user input
        \item Returns output to the user
    \end{itemize}
    \vspace{2mm}
    Terminal commands can be:
    \begin{itemize}
        \item Binary programs loaded from disk (e.g., \texttt{mkdir})
        \item Built-in functions of the terminal (e.g., \texttt{cd})
        \item Operations like regular expressions for complex command preparation
    \end{itemize}
\end{definition}

\begin{KR}{Essential Terminal Commands}
    \paragraph{System information}
    \begin{itemize}
        \item \texttt{whoami} - Display current user
        \item \texttt{uname -a} - Display kernel information
        \item \texttt{lsmod} - List loaded kernel modules
        \item \texttt{dmesg} - Display kernel messages
    \end{itemize}
    
    \paragraph{Command output manipulation}
    \begin{itemize}
        \item \texttt{command > file.txt} - Redirect output to a file
        \item \texttt{command | grep pattern} \\ - Filter output through grep
        \item \texttt{clear} - Clear terminal screen
    \end{itemize}
    
    \paragraph{System control}
    \begin{itemize}
        \item \texttt{env} - Display environment variables
        \item \texttt{exit} - Exit the terminal
        \item \texttt{shutdown} - Shutdown the system
    \end{itemize}
\end{KR}

\multend

\mult{2}

\subsection{User Management}

\begin{definition}{User Management in Linux} Tools:
    \begin{itemize}
        \item \texttt{users} - List users currently logged in
        \item \texttt{who} - Show who is logged in
        \item \texttt{id} - Display user and group IDs
        \item \texttt{passwd} - Change user password
        \item \texttt{usermod} - Modify user account
        \item \texttt{groupmod} - Modify a group
    \end{itemize}
    
    Administrative access:
    \begin{itemize}
        \item \texttt{root} - Account with full system privileges
        \item \texttt{sudo} - Execute command as root user
        \item \texttt{su} - Switch user
        \item \texttt{su -} - Switch user and load their environment
    \end{itemize}
\end{definition}

\subsection{Process Management}

\begin{definition}{Process Management Commands}\\
    Commands for monitoring and controlling processes:
    \begin{itemize}
        \item \texttt{ps} - Display current processes
        \item \texttt{top} - Interactive process viewer
        \item \texttt{pstree} - Display process tree
        \item \texttt{pidof} - Find process ID of a program
        \item \texttt{kill} - Send signal to process
        \item \texttt{kill -9} - Force terminate process
        \item \texttt{killall} - Kill processes by name
    \end{itemize}
\end{definition}

\multend

\subsection{Files, Directories, and Filesystems}

\mult{2}

\begin{definition}{File System Operations}\\
    Basic file and directory operations:
    \begin{itemize}
        \item \texttt{pwd} - Print working directory
        \item \texttt{cd} - Change directory
        \item \texttt{ls -al} - List all files with details
        \item \texttt{touch} - Create empty file or update timestamp
        \item \texttt{mkdir} - Create directory
        \item \texttt{tree} - Display directory structure as a tree
        \item \texttt{cp} - Copy files/directories
        \item \texttt{mv} - Move/rename files/directories
        \item \texttt{rm} - Remove files/directories
    \end{itemize}
    
    File permissions and ownership:
    \begin{itemize}
        \item \texttt{chown} - Change file owner
        \item \texttt{chmod} - Change file permissions
    \end{itemize}
    
    Disk and filesystem management:
    \begin{itemize}
        \item \texttt{fdisk -l} - List disk partitions
        \item \texttt{mount} - Mount a filesystem
    \end{itemize}
\end{definition}

\begin{KR}{Working with Files in Linux}
    \paragraph{Creating and viewing files}
    \begin{itemize}
        \item \texttt{touch filename} - Create empty file
        \item \texttt{cat filename} - Display entire file contents
        \item \texttt{less filename} - View file with pagination
        \item \texttt{tail filename} - Display last 10 lines of file
        \item \texttt{tail -f filename} - Continuously monitor file for changes
    \end{itemize}
    
    \paragraph{Text editing with vim}
    \begin{itemize}
        \item \texttt{vim filename} - Open file in vim
        \item Press \texttt{i} for insert mode
        \item Press \texttt{Esc} to exit insert mode
        \item Type \texttt{:w} to save
        \item Type \texttt{:q} to quit
        \item Type \texttt{:wq} to save and quit
        \item Type \texttt{:q!} to quit without saving
    \end{itemize}
\end{KR}

\multend

\subsection{System Information and Configuration}

\mult{2}

\begin{definition}{System Information Commands}\\
    Commands for system information/configuration:
    \begin{itemize}
        \item \texttt{hwinfo} - Hardware information
        \item \texttt{lshw} - List hardware
        \item \texttt{/proc} - Virtual filesystem for kernel information
        \item \texttt{/etc} - Configuration files directory
        \item \texttt{sysctl} - Read/modify kernel parameters
        \item \texttt{systemctl} - Control the systemd system and service manager
    \end{itemize}
    
    Network-related commands:
    \begin{itemize}
        \item \texttt{ifconfig} - Configure network interfaces
        \item \texttt{ip} - Show/manipulate routing, devices, policy routing
        \item \texttt{dig} - DNS lookup utility
        \item \texttt{hostname} - Show or set system hostname
    \end{itemize}
    
    Package management:
    \begin{itemize}
        \item \texttt{apt-get} - Package handling utility
    \end{itemize}
\end{definition}

\begin{example2}
    {Working with the Linux terminal}
    
\begin{lstlisting}[language=bash, style=basesmol]
# Check current user
whoami

# View hardware information
lshw -short

# Monitor system messages
dmesg | grep usb

# Check disk usage
df -h

# Find a process ID
pidof firefox

# Install a package
sudo apt-get install htop
\end{lstlisting}
\end{example2}

\begin{KR}{OpenStack Lab Environment Setup}
    \paragraph{Initial setup}
    \begin{itemize}
        \item Connect to ZHAW VPN if not at ZHAW facilities
        \item Navigate to OpenStack Horizon dashboard: \texttt{https://ned.cloudlab.zhaw.ch}
        \item Log in with provided credentials
        \item Change password if using default credentials
    \end{itemize}
    
    \paragraph{Creating SSH key pair}
    \begin{itemize}
        \item Go to Compute $\rightarrow$ Key Pairs $\rightarrow$ Create Key Pair
        \item Download the private key file (*.pem)
        \item Set appropriate permissions: \\ \texttt{chmod 600 key.pem}
    \end{itemize}
    
    \paragraph{Creating a VM}
    \begin{itemize}
        \item Go to Compute $\rightarrow$ Instances $\rightarrow$ Launch Instance
        \item Select Ubuntu image
        \item Attach to 'internal' network
        \item Set SSH key and security groups
        \item Launch VM and associate a Floating IP
    \end{itemize}
    
    \paragraph{Connecting to VM}
    \begin{itemize}
        \item SSH to the VM: \\ \texttt{ssh -i key.pem ubuntu@IP\_ADDRESS}
    \end{itemize}
\end{KR}

\begin{example2}{Linux Lab Basic Tasks}
    Creating/managing files:
\begin{lstlisting}[language=bash, style=basesmol]
# Create a test file
touch delta.txt
# Add content to the file
echo "Hello, this is a test" > delta.txt
# View the file content
cat delta.txt
# Stop and restart the VM from OpenStack dashboard
# Then check if the file persists
cat delta.txt
\end{lstlisting}
\end{example2}

\multend