\section{Input / Output}

\begin{concept}{I/O Challenges}
    I/O management presents unique challenges for operating systems:
    \begin{itemize}
        \item Huge variety of I/O devices (keyboards, mice, drives, sensors, etc.)
        \item Diverse interfaces (ATA, SATA, USB, PCI, etc.)
        \item Wide range of speeds and capacities
        \item Different data transfer characteristics
        \item Need for a unified approach to device interaction
    \end{itemize}
    
    The operating system must abstract this heterogeneity by providing consistent interfaces for applications to access diverse devices.
\end{concept}

\begin{definition}{Device Categories}
    From a user perspective, devices fall into two main categories:

    \begin{minipage}{0.5\linewidth}
        \textbf{Block Devices}:
            \begin{itemize}
                \item Operate on fixed-size blocks of data
                \item Support random access to any block
                \item Example: Hard drives, SSDs
                \item Block operations are independent from each other
                \item Block size defined when formatting (logical)
                \item Sector size is the physical unit on the device
            \end{itemize}
    \end{minipage}
    \begin{minipage}{0.5\linewidth}
        \textbf{Character Devices}:
            \begin{itemize}
                \item Operate on streams of characters
                \item Generally sequential access
                \item Example: Keyboards, mice, printers
                \item Characters are interpretations of bit patterns according to specifications (ASCII, Unicode)
                \item Character devices ultimately operate on bit-level too
            \end{itemize}
    \end{minipage}
\end{definition}

\subsection{I/O Hardware Architecture}

\begin{definition}{I/O Hardware Components}\\
    Key components in I/O hardware architecture:
    \begin{itemize}
        \item \textbf{I/O Controller}: Electronic interface to the device
            \begin{itemize}
                \item Typically one per device category (SCSI, IDE, USB, Ethernet)
                \item Contains registers that control device operation
                \item Maintains buffers for read/write operations
                \item Communicates with CPU and main memory
            \end{itemize}
        \item \textbf{I/O Ports}: Addresses that point to controller registers
        \item \textbf{Buffers}: Memory areas for data transfer
        \item \textbf{Bus System}: Connects CPU, memory, and I/O devices
    \end{itemize}
\end{definition}

\begin{definition}{I/O Address Space}\\
    X86 architecture has two approaches for I/O address allocation:
    \begin{itemize}
        \item \textbf{Port Mapped I/O (PMIO)}:
            \begin{itemize}
                \item Peripheral I/O registers assigned to a dedicated address range
                \item Distinct from system memory
                \item Uses dedicated I/O instructions (IN and OUT)
                \item Limited to 64K ports (16-bit addressing)
                \item Common for older peripherals (e.g., ISA cards)
                \item Listed in \texttt{/proc/ioports}
            \end{itemize}
        \item \textbf{Memory Mapped I/O (MMIO)}:
            \begin{itemize}
                \item Peripheral registers mapped into main memory address space
                \item Uses regular memory instructions (e.g., MOV)
                \item Default in modern systems (e.g., PCIe, PCI)
                \item Listed in \texttt{/proc/iomem}
            \end{itemize}
    \end{itemize}
\end{definition}

\begin{definition}{Direct Memory Access (DMA)}\\
    DMA improves I/O performance:
    \begin{itemize}
        \item Allows devices to transfer data directly to/from memory
        \item Bypasses CPU for bulk data transfer
        \item CPU sets up transfer parameters:
            \begin{itemize}
                \item Memory address
                \item Count of bytes to transfer
                \item Direction (read/write)
                \item Device-specific parameters
            \end{itemize}
        \item DMA controller handles the actual transfer
        \item Interrupts CPU when transfer completes
        \item Reduces CPU overhead for data-intensive I/O operations
    \end{itemize}
\end{definition}

\subsection{I/O Principles}

\mult{2}

\begin{definition}{I/O Access Methods}\\
    \textbf{Programmed I/O}:
            \begin{itemize}
                \item CPU does all the work
                \item CPU executes instructions to transfer data
                \item CPU repeatedly checks device status \\ (polling/busy-waiting)
                \item Synchronous operation \\ (CPU waits for I/O completion)
                \item Simple but inefficient for slow devices
            \end{itemize}
    \textbf{Interrupt-driven I/O}:
            \begin{itemize}
                \item Devices signal CPU when they need attention
                \item CPU init. operation then continues other work
                \item Device interrupts when operation completes
                \item Asynchronous operation
                \item More efficient, especially for slow devices
            \end{itemize}
\end{definition}

\begin{definition}{Interrupt Handling}\\
    Interrupts are fundamental to efficient I/O:
    \begin{itemize}
        \item \textbf{Interrupt}: Event that defers the normal flow of CPU execution
        \item When an interrupt occurs:
            \begin{itemize}
                \item Current execution is suspended
                \item CPU state is saved
                \item Special routine (interrupt handler) is executed
                \item After completion, previous execution resumes
            \end{itemize}
        \item \textbf{Interrupt types}:
            \begin{itemize}
                \item \textbf{Synchronous}: Generated by executing an \\ instruction (e.g., divide by zero)
                \item \textbf{Asynchronous}: Generated by external events \\ (e.g., I/O completion)
            \end{itemize}
        \item \textbf{Interrupt classifications}:
            \begin{itemize}
                \item \textbf{Maskable}: \\ Can be ignored by setting interrupt mask
                \item \textbf{Non-maskable}: \\ Cannot be ignored (critical events)
            \end{itemize}
    \end{itemize}
\end{definition}

\begin{definition}{Interrupt Flow}\\
    Hardware interrupt flow:
    \begin{itemize}
        \item Device raises interrupt on its IRQ line
        \item Programmable Interrupt Controller (PIC) converts IRQ to vector number
        \item PIC signals CPU via INTR pin
        \item CPU acknowledges interrupt
        \item CPU executes the appropriate interrupt handler
    \end{itemize}
        In Linux, interrupt handling occurs in three phases:
            \begin{itemize}
                \item Critical: Minimal processing, acknowledge interrupt
                \item Immediate: Essential processing, can't be deferred
                \item Deferred: Non-critical processing, scheduled for later
            \end{itemize}
\end{definition}

\begin{definition}{I/O Access Patterns}\\
    Different access patterns affect I/O performance:\\
        \textbf{Exclusive vs. Shared Access}:
            \begin{itemize}
                \item Exclusive: Device dedicated to one process
                \item Shared: Multiple processes access same device
                \item Shared access requires scheduling (e.g., disk I/O scheduling)
            \end{itemize}
        \textbf{Sequential vs. Random Access}:
            \begin{itemize}
                \item Sequential: Data accessed in order (e.g., tape)
                \item Random: Data accessed in any order (e.g., disk)
            \end{itemize}
        \textbf{Blocking vs. Non-Blocking}:
            \begin{itemize}
                \item Blocking: Process waits until I/O completes
                \item Non-Blocking: Process continues, checks completion later
            \end{itemize}
        \textbf{Synchronous vs. Asynchronous}:
            \begin{itemize}
                \item Synchronous: Process execution synchronized with I/O completion
                \item Asynchronous: Process continues, notified of I/O completion
            \end{itemize}
\end{definition}

\begin{definition}{Buffered vs. Direct I/O}\\
    Buffering improves I/O performance:
    \begin{itemize}
        \item \textbf{Buffered I/O}:
            \begin{itemize}
                \item Data passes through intermediate buffer
                \item Decouples data access from data generation
                \item Handles different speeds between devices
                \item Enables rate control
                \item Allows data manipulation before final transfer
                \item Supports data verification
            \end{itemize}
        \item \textbf{Direct I/O}:
            \begin{itemize}
                \item Data transferred directly to/from device
                \item Bypasses system caches
                \item Reduces memory usage and CPU overhead
                \item Useful for applications with their own caching
                \item May be slower for some workloads
            \end{itemize}
    \end{itemize}
\end{definition}

\begin{definition}{Error Handling}\\
    I/O systems must handle errors effectively:
    \begin{itemize}
        \item Errors can occur at various levels:
            \begin{itemize}
                \item Bit-level, byte-level, packet-level, block-level
                \item Hardware vs. software detection
                \item User space vs. kernel space handling
            \end{itemize}
        \item Error handling approaches:
            \begin{itemize}
                \item Error Detection: Identify errors (e.g., checksums, parity)
                \item Error Correction: Fix errors without retransmission
                \item Error Recovery: Return to consistent state after error
            \end{itemize}
        \item Trade-offs between performance and reliability
    \end{itemize}
\end{definition}

\multend

\subsection{Linux I/O Subsystem}

\mult{2}

\begin{definition}{Linux Device Model}\\
    The Linux Device Model maintains the state and structure of the system:
    \begin{itemize}
        \item Maintains information about devices, drivers, buses, etc.
        \item Key entities:
            \begin{itemize}
                \item \textbf{Device}: Physical device attached to a bus
                \item \textbf{Driver}: Software entity that operates devices
                \item \textbf{Bus}: Device to which other devices can be attached
                \item \textbf{Class}: Type of device with similar behavior
                \item \textbf{Subsystem}: View on the system structure
            \end{itemize}
        \item Represented in user space via sysfs (mounted at /sys)
    \end{itemize}
\end{definition}

\begin{concept}{Device Access in Linux}\\
    Linux provides a unified interface for device access:
    \begin{itemize}
        \item Applications use standard file operations:
            \begin{itemize}
                \item \texttt{open()}: Open device
                \item \texttt{read()}: Read from device
                \item \texttt{write()}: Write to device
                \item \texttt{ioctl()}: Device-specific operations
                \item \texttt{close()}: Close device
            \end{itemize}
        \item Kernel translates these operations to device-specific commands
        \item Virtual File System (VFS) provides abstraction layer
        \item Device drivers implement the specific operations
    \end{itemize}
\end{concept}

\multend

\mult{2}

\begin{definition}{sysfs}\\
    virtual filesystem that exposes the Linux Device Model:
    \begin{itemize}
        \item Located at /sys
        \item Key directories:
            \begin{itemize}
                \item \texttt{/sys/block}: Block devices
                \item \texttt{/sys/bus}: Bus types
                \item \texttt{/sys/class}: Device classes
                \item \texttt{/sys/devices}: Hierarchical device structure
                \item \texttt{/sys/firmware}: Firmware information
                \item \texttt{/sys/fs}: Filesystem information
                \item \texttt{/sys/kernel}: Kernel status
                \item \texttt{/sys/module}: Loaded modules
                \item \texttt{/sys/power}: Power management
            \end{itemize}
        \item Contains attributes in files for configuration and status
    \end{itemize}
\end{definition}

\begin{definition}{udev}\\
    userspace device manager for Linux:
    \begin{itemize}
        \item Part of systemd
        \item systemd-udevd listens to kernel events
        \item Executes rules based on device information from sysfs
        \item Creates or removes device nodes in /dev
        \item Provides consistent device naming
        \item Enables automatic device setup
        \item Supports user-defined rules
    \end{itemize}
\end{definition}

\multend

\begin{definition}{/dev Directory}
    The /dev directory contains special files representing devices:
    \begin{itemize}
        \item Each file represents an I/O device
        \item Allows standard file operations (open, read, write, close) on devices
        \item When accessed, kernel routes operations to appropriate device drivers
        \item Types of device files:
            \begin{itemize}
                \item Character device files: For character devices
                \item Block device files: For block devices
            \end{itemize}
        \item Naming conventions:
            \begin{itemize}
                \item \texttt{/dev/sdX}: SCSI/SATA disk devices
                \item \texttt{/dev/ttyX}: Terminal devices
                \item \texttt{/dev/nullX}: Special device files
            \end{itemize}
    \end{itemize}
\end{definition}





\begin{KR}{Working with I/O in Linux}
    \paragraph{Exploring device information}
    \begin{itemize}
        \item List I/O ports: \texttt{cat /proc/ioports}
        \item List I/O memory: \texttt{cat /proc/iomem}
        \item View interrupts: \texttt{cat /proc/interrupts}
        \item List block devices: \texttt{lsblk}
        \item Show hardware: \texttt{lshw}
        \item Display PCI devices: \texttt{lspci}
        \item List USB devices: \texttt{lsusb}
    \end{itemize}
    
    \paragraph{Working with devices}
    \begin{itemize}
        \item Check device information: \texttt{udevadm info --query=all --name=/dev/sda}
        \item Monitor device events: \texttt{udevadm monitor}
        \item Show device properties: \texttt{udevadm info --attribute-walk --name=/dev/sda}
        \item Check I/O performance: \texttt{iostat -x}
        \item Monitor I/O activity: \texttt{iotop}
    \end{itemize}
    
    \paragraph{Device file operations}
    \begin{itemize}
        \item Create device file: \texttt{mknod /dev/example c 1 3}
        \item Read from device: \texttt{cat /dev/input/mouse0 | hexdump}
        \item Write to device: \texttt{echo "test" > /dev/tty1}
        \item Control device: \texttt{ioctl} system call in C programs
    \end{itemize}
\end{KR}

\begin{example2}{I/O Performance Testing}
    Testing disk write performance with different options:
    
\begin{lstlisting}[language=bash, style=basesmol]
# Basic write test (with caching)
dd if=/dev/zero of=speedtest bs=10M count=100
rm speedtest
# Write test with synchronous I/O (forces data to disk)
dd if=/dev/zero of=speedtest bs=10M count=100 conv=fdatasync
rm speedtest
# Write test with direct I/O (bypasses the page cache)
dd if=/dev/zero of=speedtest bs=10M count=100 oflag=direct
rm speedtest
# Monitor disk I/O activity during test
iostat -x 1
# Check disk utilization statistics
iostat -p sda

# Explanation:
# - Standard dd shows high performance but may not be on disk yet
# - fdatasync ensures data is physically written to disk
# - direct bypasses the OS cache, showing raw device performance
\end{lstlisting}
\end{example2}

\begin{example2}{Device Information Analysis}
    Exploring device information in Linux:
    
\begin{lstlisting}[language=bash, style=basesmol]
# List block devices with details
lsblk -f
# Get detailed information about a specific device
udevadm info --query=all --name=/dev/sda1
# Examine sysfs entries for the device
ls -l /sys/block/sda/sda1/
# Check device attributes
cat /sys/block/sda/queue/scheduler
cat /sys/block/sda/queue/read_ahead_kb
# Get PCI information for a disk controller
lspci | grep -i sata
# Check all properties of a device
udevadm info --attribute-walk --name=/dev/sda1 | less
# Monitor udev events when plugging in a USB device
udevadm monitor --property
# (Now plug in a USB device to see events)
\end{lstlisting}
\end{example2}