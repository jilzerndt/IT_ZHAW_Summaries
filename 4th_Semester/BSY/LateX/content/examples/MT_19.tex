\section{Midterm 2019} 

This chapter contains typical exam questions with detailed solutions and systematic approaches (Kochrezepte) for solving similar problems.

\subsection{Cache Performance Analysis}

\begin{KR}{Cache Hit Rate Calculation}
    \paragraph{Problem identification}
    \begin{itemize}
        \item Given: Array access pattern, processor architecture, cache block size
        \item Find: Cache hit rate during sequential array access
    \end{itemize}
    
    \paragraph{Solution approach}
    \begin{itemize}
        \item Calculate data type size based on processor architecture
        \item Determine how many array elements fit in one cache line
        \item First access to cache line = miss, subsequent accesses = hits
        \item Hit rate = (accesses per line - 1) / accesses per line
    \end{itemize}
    
    \paragraph{Key formulas}
    \begin{itemize}
        \item Elements per cache line = Cache line size / Data type size
        \item Hit rate = (Elements per line - 1) / Elements per line
        \item For 32-bit processor: int = 4 bytes
        \item For 64-bit processor: int = 4 bytes, long = 8 bytes
    \end{itemize}
\end{KR}

\begin{example2}{Cache Hit Rate Calculation}
    Given program code accessing an integer array:
    
\begin{lstlisting}[language=C, style=basesmol]
#define N (10*1000*1000)
int arVL[N];
for (int i = 0; i < N; i++) {
    sum += arVL[i];
}
\end{lstlisting}

    Calculate hit rate on a 32-bit processor with 64-byte cache lines.
    
    \tcblower
    
    \textbf{Solution:}
    \begin{itemize}
        \item 32-bit processor: \texttt{int} = 4 bytes
        \item Cache line size: 64 bytes
        \item Elements per cache line: 64 / 4 = 16 integers
        \item First access per line: cache miss
        \item Next 15 accesses per line: cache hits
        \item Hit rate: $h_c = \frac{15}{16} = 0.9375 = 93.75\%$
    \end{itemize}
\end{example2}

\subsection{Memory Access Time Calculation}

\begin{KR}{Average Memory Access Time}
    \paragraph{Problem components}
    \begin{itemize}
        \item Multi-level cache hierarchy
        \item Hit rates for each level
        \item Access times for each level
        \item Main memory access time
    \end{itemize}
    
    \paragraph{Calculation method}
    \begin{itemize}
        \item Start from L1 cache (always accessed first)
        \item For each miss, calculate probability of accessing next level
        \item Multiply access time by probability of reaching that level
        \item Sum all weighted access times
    \end{itemize}
    
    \paragraph{Formula}
    \begin{itemize}
        \item $T_{avg} = T_{L1} + (1-h_1) \times T_{L2} + (1-h_1)(1-h_2) \times T_{L3} + \ldots$
        \item Convert clock cycles to time: Time = Cycles $\times$ Clock period
        \item Clock period = 1 / Frequency
    \end{itemize}
\end{KR}

\begin{example2}{Memory Access Time Calculation}
    2 GHz processor (clock cycle = 0.5 ns) with three-level cache:
    
    \begin{tabular}{|l|l|}
        \hline
        Cache Level 1 & 4 clock cycles \\
        Cache Level 2 & 10 clock cycles \\
        Cache Level 3 & 40 clock cycles \\
        Main Memory & 60 ns \\
        \hline
    \end{tabular}
    
    Hit rate on each level: 90\%
    
    \tcblower
    
    \textbf{Solution:}
    \begin{itemize}
        \item L1 access: $4 \times 0.5 = 2$ ns (always accessed)
        \item L2 access: $10 \times 0.5 = 5$ ns (10\% probability)
        \item L3 access: $40 \times 0.5 = 20$ ns (1\% probability)  
        \item Main memory: 60 ns (0.1\% probability)
        \item $T_{avg} = 2 + 0.1 \times 5 + 0.01 \times 20 + 0.001 \times 60$
        \item $T_{avg} = 2 + 0.5 + 0.2 + 0.06 = 2.76$ ns
    \end{itemize}
\end{example2}

\subsection{Makefile Analysis}

\begin{KR}{Makefile Dependency Analysis}
    \paragraph{Understanding make behavior}
    \begin{itemize}
        \item Make compares timestamps of targets and dependencies
        \item Target is rebuilt if it's older than any dependency
        \item Missing targets are always built
        \item Implicit rules (.c.o:) handle automatic compilation
    \end{itemize}
    
    \paragraph{Analysis steps}
    \begin{itemize}
        \item Identify target dependencies from makefile
        \item Check file timestamps in directory listing
        \item Determine which object files need recompilation
        \item Follow linking rules to determine final executable name
    \end{itemize}
    
    \paragraph{Common patterns}
    \begin{itemize}
        \item .c.o: rule compiles .c files to .o files automatically
        \item \$@ refers to the target name
        \item If .c file is newer than .o file, recompilation occurs
        \item Missing .o files are always created
    \end{itemize}
\end{KR}

\begin{example2}{Makefile Analysis}
    Given Makefile:
    
\begin{lstlisting}[style=basesmol]
CFL = -g
CMP = gcc $(CFL)
app: main1.o mythread.o scheduler.o queues.o mylist.o
	$(CMP) main1.o mythread.o scheduler.o queues.o mylist.o -o $@.e
.c.o:
	$(CMP) -c $<
\end{lstlisting}

    Directory listing shows:
    \begin{itemize}
        \item mylist.c: Feb 24 09:31 (newer)
        \item mylist.o: Feb 24 09:21 (older)  
        \item queues.c: Feb 24 09:25 (newer)
        \item queues.o: Feb 24 09:21 (older)
        \item All other .o files are newer than corresponding .c files
    \end{itemize}
    
    \tcblower
    
    \textbf{Solution:}
    \begin{itemize}
        \item Files to be compiled: mylist.c, queues.c (because .c files are newer than .o files)
        \item Executable name: app.e (from \$@.e where \$@ = app)
    \end{itemize}
\end{example2}

\subsection{Process Creation with fork()}

\begin{KR}{Fork() Process Tree Analysis}
    \paragraph{Understanding fork() behavior}
    \begin{itemize}
        \item fork() creates an exact copy of the calling process
        \item Returns child PID to parent, 0 to child
        \item fork() > 0 is true only in parent process
        \item Each fork() doubles the number of processes
    \end{itemize}
    
    \paragraph{Analysis method}
    \begin{itemize}
        \item Draw a process tree starting with the original process
        \item For each fork(), branch into parent and child
        \item Track which path each process takes through if/else statements
        \item Count total processes and specific execution paths
    \end{itemize}
    
    \paragraph{Counting strategy}
    \begin{itemize}
        \item Start with 1 process (the original)
        \item Each successful fork() adds 1 new process
        \item Count processes that reach specific code sections
        \item Be careful with nested forks and conditional execution
    \end{itemize}
\end{KR}

\begin{example2}{Fork() Process Creation}
    \important{ADDED GRAPHIC HERE!!}
    Analyze this code snippet:
    
\begin{lstlisting}[language=C, style=basesmol]
if (fork() > 0)
    fork();
else {
    fork();
    if (fork() > 0)
        printf("Hello World\n");
}
\end{lstlisting}

    Assume all fork() calls succeed.
    
    \tcblower
    
    \textbf{Solution:}
    $$
    \begin{array}{lll}
        f--- & f---| & \\
        \lfloor     & \llcorner ---| & \\
        ---- & f---- & f--- Hello World \\
             & \lfloor      & \llcorner ---| \\
             & ----- & f--- Hello World \\
             &       & \llcorner ---|\\
    \end{array} 
    $$   
    
    Process tree analysis:
    \begin{itemize}
        \item Original process P1 forks $\rightarrow$ creates P2
        \item P1 (parent): fork() > 0 is true, executes fork() $\rightarrow$ creates P3
        \item P2 (child): fork() > 0 is false, goes to else block
        \item P2 executes fork() $\rightarrow$ creates P4
        \item P2 executes second fork() $\rightarrow$ creates P5 
        \item P4 executes second fork() $\rightarrow$ creates P6
        \item P2 (parent of P5): fork() > 0 is true, prints "Hello World"
        \item P4 (parent of P6): fork() > 0 is true, prints "Hello World"
    \end{itemize}
    
    \textbf{Results:}
    \begin{itemize}
        \item Additional processes created: 5 (P2, P3, P4, P5, P6)
        \item "Hello World" printed: 2 times (by P2 and P4)
    \end{itemize}
\end{example2}

\subsection{Real-Time Scheduling}

\begin{KR}{Earliest Deadline First Scheduling}
    \paragraph{EDF Algorithm principles}
    \begin{itemize}
        \item Always schedule the task with the earliest deadline
        \item Deadlines = arrival time + period for periodic tasks
        \item Preemption occurs when a task with earlier deadline arrives
        \item Re-scheduling happens at specified intervals or task completion
    \end{itemize}
    
    \paragraph{Scheduling steps}
    \begin{itemize}
        \item Calculate all task deadlines for the scheduling period
        \item At each scheduling point, identify available tasks
        \item Select task with earliest deadline
        \item Handle ties with specified priority rules (e.g., shorter period)
        \item Mark execution periods in timeline diagram
    \end{itemize}
    
    \paragraph{Timeline construction}
    \begin{itemize}
        \item Create timeline with scheduling intervals marked
        \item For each interval, determine which task should run
        \item Show task execution as filled blocks
        \item Verify all deadlines are met
    \end{itemize}
\end{KR}

\begin{example2}{EDF Real-Time Scheduling}
    \important{ADD GRAPHIC HERE!!}
    Three periodic tasks with EDF scheduling:
    
    \begin{tabular}{|c|c|c|}
        \hline
        Task & Period & Execution Time \\
        \hline
        1 & 4ms & 1ms \\
        2 & 8ms & 4ms \\
        3 & 12ms & 3ms \\
        \hline
    \end{tabular}
    
    Re-scheduling every 4ms or when task suspends.
    For equal deadlines, shorter period has higher priority.
    
    \tcblower
    
    \textbf{Solution approach:}
    
    Timeline analysis (0-12ms):
    \begin{itemize}
        \item t=0: T1 (deadline 4), T2 (deadline 8), T3 (deadline 12)
        \item T1 has earliest deadline $\rightarrow$ runs 0-1ms
        \item t=1: T2 (deadline 8), T3 (deadline 12)  
        \item T2 has earliest deadline $\rightarrow$ runs 1-4ms (suspended after 3ms)
        \item t=4: T1 arrives again (deadline 8), T2 continues (1ms left), T3 (deadline 12)
        \item Multiple tasks with deadline 8: T1 has shorter period $\rightarrow$ higher priority
        \item Continue this analysis through the full timeline
    \end{itemize}
\end{example2}

\raggedcolumns
\columnbreak

\subsection{Multi-Level Scheduling}

\begin{KR}{Multi-Level Round Robin Scheduling}
    \paragraph{Multi-level principles}
    \begin{itemize}
        \item Separate queues for different priority levels
        \item Higher priority queues are served first
        \item Round robin within each priority level
        \item Lower priority tasks only run when higher queues are empty
    \end{itemize}
    
    \paragraph{Scheduling algorithm}
    \begin{itemize}
        \item Group processes by priority level
        \item Always check highest priority queue first
        \item Use round robin with specified time quantum within each queue
        \item Preemption occurs when higher priority process arrives
        \item Track remaining execution time for each process
    \end{itemize}
    
    \paragraph{Timeline construction}
    \begin{itemize}
        \item Mark process arrivals on timeline
        \item At each time unit, determine which process should run
        \item Apply round robin within the highest non-empty priority queue
        \item Show context switches and queue changes
    \end{itemize}
\end{KR}

\begin{example2}{Multi-Level Scheduling}
    \important{ADD GRAPHIC HERE!!}
    Five processes with priorities and times:
    
    \begin{tabular}{|c|c|c|c|}
        \hline
        Process & Priority & Arrival & Execution \\
        \hline
        P1 & 0 & 0 & 4 \\
        P2 & 1 & 1 & 3 \\
        P3 & 1 & 2 & 2 \\
        P4 & 0 & 5 & 3 \\
        P5 & 0 & 7 & 2 \\
        \hline
    \end{tabular}
    
    Multi-level scheduling (not feedback), no blocking, RR with q=1.
    Higher number = higher priority.
    
    \tcblower
    
    \textbf{Solution timeline (0-14):}
    \begin{itemize}
        \item t=0-1: P1 runs (only process, priority 0)
        \item t=1-2: P2 runs (arrives with priority 1, preempts P1)
        \item t=2-3: P3 runs (priority 1, round robin with P2)
        \item t=3-4: P2 runs (continues in priority 1 queue)
        \item t=4-5: P3 runs (finishes remaining 1 time unit)
        \item t=5-6: P2 runs (finishes last time unit)
        \item t=6-7: P1 runs (priority 1 queue empty, resume P1)
        \item Continue pattern with P4, P5 arrivals...
    \end{itemize}
\end{example2}

\raggedcolumns
\columnbreak

\subsection{Operating System Concepts}

\begin{KR}{OS Concept Questions}
    \paragraph{System call identification}
    \begin{itemize}
        \item System calls required for kernel mode operations
        \item I/O operations (screen output, file access) need system calls
        \item Time-related operations (sleep) need system calls
        \item Pure computation (arithmetic) doesn't need system calls
        \item Memory operations within process space don't need system calls
    \end{itemize}
    
    \paragraph{Thread and process concepts}
    \begin{itemize}
        \item pthread\_join() waits for thread completion
        \item Prevents main process from terminating before threads finish
        \item Software interrupts provide controlled kernel mode entry
        \item PCB (Process Control Block) stores OS process information
    \end{itemize}
    
    \paragraph{Process states}
    \begin{itemize}
        \item Zombie: process finished but parent hasn't waited
        \item Daemon: background process without controlling terminal
        \item Orphan: process whose parent has terminated
    \end{itemize}
\end{KR}

\begin{example2}{Operating System Multiple Choice}
    Typical OS concept questions:
    
    \textbf{Which activities require system calls?}
    \begin{itemize}
        \item \textcolor{frog}{$\surd$} Display variable value on screen (I/O operation)
        \item \textcolor{red}{\textbf{X}} Compare two variables (CPU operation)
        \item \textcolor{frog}{$\surd$} Sleep for 1 second (timer operation)
        \item \textcolor{red}{\textbf{X}} Increment integer variable (memory operation)
        \item \textcolor{red}{\textbf{X}} Increment float variable (CPU operation)
    \end{itemize}
    
    \textbf{What is pthread\_join() used for?}
    \begin{itemize}
        \item \textcolor{red}{\textbf{X}} Terminates all user-level threads
        \item \textcolor{red}{\textbf{X}} Can be replaced by sleep(0)
        \item \textcolor{frog}{$\surd$} Prevents process termination before threads finish
        \item \textcolor{red}{\textbf{X}} Blocks CPU until all threads terminate
    \end{itemize}
    
    \textbf{Why use software interrupts for system calls?}
    \begin{itemize}
        \item \textcolor{red}{\textbf{X}} They are faster than procedure calls
        \item \textcolor{frog}{$\surd$} Procedure calls cannot switch to system mode
        \item \textcolor{frog}{$\surd$} Applications don't know system call routine addresses
    \end{itemize}
\end{example2}

\begin{KR}{Exam Preparation Strategy}
    \paragraph{Time management}
    \begin{itemize}
        \item Read all questions first to identify easy wins
        \item Allocate time based on point values
        \item Don't spend too much time on single difficult problems
        \item Reserve time for checking answers
    \end{itemize}
    
    \paragraph{Problem-solving approach}
    \begin{itemize}
        \item Identify the problem type and appropriate "Kochrezept"
        \item Write down given values and what needs to be found
        \item Draw diagrams for scheduling and process problems
        \item Show calculation steps clearly
        \item Double-check units and reasonableness of answers
    \end{itemize}
    
    \paragraph{Common mistakes to avoid}
    \begin{itemize}
        \item Forgetting to convert time units (ns, ms, clock cycles)
        \item Not considering all processes in fork() analysis
        \item Missing preemption in scheduling problems
        \item Confusing parent/child behavior in process creation
        \item Not reading multiple choice questions carefully
    \end{itemize}
\end{KR}