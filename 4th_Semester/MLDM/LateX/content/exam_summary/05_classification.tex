\section{Classification with Logistic Regression}

\subsection{Binary Classification}

\begin{definition}{Logistic Regression}\\
Logistic regression is a supervised learning algorithm for binary classification. It models the probability that an input belongs to the positive class using the logistic function:
\[\hat{y} = h_\theta(x) = g(\theta^T x) = \frac{1}{1 + e^{-\theta^T x}}\]
where $g(z) = \frac{1}{1 + e^{-z}}$ is the sigmoid (logistic) function that maps any real number to the range $(0,1)$.
\end{definition}

\begin{concept}{Decision Boundary}\\
The decision boundary is the line (or hyperplane in higher dimensions) that separates the regions where the model predicts different classes. For logistic regression:
\begin{itemize}
    \item Predict class 1 if $h_\theta(x) \geq 0.5$ 
    \item Predict class 0 if $h_\theta(x) < 0.5$
    \item This boundary occurs when $\theta^T x = 0$
\end{itemize}
\end{concept}

\begin{definition}{Cost Function for Logistic Regression}\\
The cost function for logistic regression is the Log Loss (also called Cross-Entropy Loss):
\[J(\theta) = \frac{1}{M}\sum_{m=1}^{M}[-y^{(m)}\log(h_\theta(x^{(m)})) - (1-y^{(m)})\log(1-h_\theta(x^{(m)}))]\]
This function penalizes wrong predictions more heavily as they get more confident.
\end{definition}

\begin{KR}{Training a Logistic Regression Model}\\
\paragraph{Initialize parameters}
Start with random values for parameters $\theta$

\paragraph{Gradient descent}
Apply gradient descent to minimize the cost function:
\[\theta_j = \theta_j - \alpha \frac{1}{M}\sum_{m=1}^{M}(h_\theta(x^{(m)}) - y^{(m)})x^{(m)}_j\]

\paragraph{Make predictions}
For new data point $x$:
\begin{itemize}
    \item Calculate probability $h_\theta(x) = \frac{1}{1 + e^{-\theta^T x}}$
    \item Predict class 1 if $h_\theta(x) \geq 0.5$, otherwise class 0
\end{itemize}

\paragraph{Evaluate model}
Use metrics like accuracy, precision, recall, or F1-score to evaluate model performance on a test set.
\end{KR}

\begin{example}{Logistic Regression Application}
Suppose we want to predict whether a student will pass an exam based on hours studied and previous GPA.
\begin{itemize}
    \item Features: Hours studied ($x_1$), Previous GPA ($x_2$)
    \item Output: Pass (1) or Fail (0)
    \item After training, we get: $\theta_0 = -6$, $\theta_1 = 0.05$, $\theta_2 = 1.5$
    \item Decision boundary: $-6 + 0.05 \times \text{hours} + 1.5 \times \text{GPA} = 0$
    \item For a student who studied 80 hours with GPA 3.2:
    \begin{itemize}
        \item $z = -6 + 0.05 \times 80 + 1.5 \times 3.2 = -6 + 4 + 4.8 = 2.8$
        \item $h_\theta(x) = \frac{1}{1 + e^{-2.8}} \approx 0.94$
        \item Prediction: Pass (94\% confidence)
    \end{itemize}
\end{itemize}
\end{example}

\subsection{Differences from Linear Regression}

\begin{concept}{Contrasting Logistic and Linear Regression}\\
While both logistic and linear regression find a linear relationship between features and output, they differ in key ways:
\begin{itemize}
    \item \textbf{Output range}: Linear regression predicts any real number, while logistic regression outputs probabilities between 0 and 1
    \item \textbf{Interpretation}: Linear regression predicts a quantity, while logistic regression predicts a probability
    \item \textbf{Cost function}: Linear regression uses squared error, while logistic regression uses log loss
    \item \textbf{Solution method}: Linear regression has a closed-form solution (normal equation), while logistic regression typically requires iterative optimization
\end{itemize}
\end{concept}

\begin{concept}{When to Use Logistic vs. Linear Regression}\\
\begin{itemize}
    \item Use logistic regression when:
    \begin{itemize}
        \item The target variable is categorical (especially binary)
        \item You need probabilistic outputs
        \item You want a decision boundary for classification
    \end{itemize}
    \item Use linear regression when:
    \begin{itemize}
        \item The target variable is continuous
        \item You need to predict a quantity rather than a category
        \item A linear relationship is appropriate for the data
    \end{itemize}
\end{itemize}
\end{concept}

\subsection{Regularization in Logistic Regression}

\begin{definition}{Regularized Logistic Regression}\\
Regularization can be applied to logistic regression to prevent overfitting:
\[J(\theta) = \frac{1}{M}\sum_{m=1}^{M}[-y^{(m)}\log(h_\theta(x^{(m)})) - (1-y^{(m)})\log(1-h_\theta(x^{(m)}))] + \frac{\lambda}{2M}\sum_{j=1}^{n}\theta_j^2\]
The regularization term penalizes large parameter values, encouraging a simpler model.
\end{definition}

\begin{KR}{Implementing Regularized Logistic Regression}\\
\paragraph{Modify the cost function}
Include the regularization term in the cost function:
\[J(\theta) = \frac{1}{M}\sum_{m=1}^{M}[-y^{(m)}\log(h_\theta(x^{(m)})) - (1-y^{(m)})\log(1-h_\theta(x^{(m)}))] + \frac{\lambda}{2M}\sum_{j=1}^{n}\theta_j^2\]

\paragraph{Update gradient descent}
Modify the update rule for regularized parameters ($j \geq 1$):
\[\theta_j = \theta_j(1-\alpha\frac{\lambda}{M}) - \alpha\frac{1}{M}\sum_{m=1}^{M}(h_\theta(x^{(m)}) - y^{(m)})x^{(m)}_j\]
Note: The intercept term $\theta_0$ is typically not regularized.

\paragraph{Select regularization parameter}
Use cross-validation to choose an appropriate value for $\lambda$.
\end{KR}

\subsection{Multi-class Classification}

\begin{definition}{One-vs-rest Classification}\\
One-vs-rest (also called one-vs-all) is an approach to extend binary classification algorithms like logistic regression to multi-class problems:
\begin{itemize}
    \item Train $K$ separate binary classifiers, one for each class
    \item The $k$-th classifier distinguishes class $k$ from all other classes
    \item For prediction, apply all classifiers and select the class with highest confidence
\end{itemize}
\end{definition}

\begin{KR}{Implementing One-vs-rest Classification}\\
\paragraph{Prepare the data}
For each class $k$ in $\{1, 2, ..., K\}$, create binary labels:
\begin{itemize}
    \item $y^{(m)}_k = 1$ if $y^{(m)} = k$ (sample $m$ belongs to class $k$)
    \item $y^{(m)}_k = 0$ otherwise (sample $m$ belongs to any other class)
\end{itemize}

\paragraph{Train multiple classifiers}
For each class $k$:
\begin{itemize}
    \item Train a binary logistic regression classifier to predict $y^{(m)}_k$
    \item This gives parameters $\theta^{(k)}$ for class $k$
\end{itemize}

\paragraph{Make predictions}
For a new data point $x$:
\begin{itemize}
    \item Calculate $h_{\theta^{(k)}}(x)$ for each $k$
    \item Predict the class with highest probability: $\hat{y} = \text{argmax}_k h_{\theta^{(k)}}(x)$
\end{itemize}
\end{KR}

\begin{example2}{One-vs-rest Classification}\\
Consider classifying iris flowers into three classes: Setosa, Versicolor, and Virginica.
\begin{itemize}
    \item Features: Petal length and width
    \item Three binary classifiers:
    \begin{itemize}
        \item Classifier 1: Setosa (1) vs. rest (0)
        \item Classifier 2: Versicolor (1) vs. rest (0)
        \item Classifier 3: Virginica (1) vs. rest (0)
    \end{itemize}
\end{itemize}
\tcblower
For a new flower with petal length 4.5 cm and width 1.3 cm:
\begin{itemize}
    \item Classifier 1 probability: 0.02 (Setosa)
    \item Classifier 2 probability: 0.87 (Versicolor)
    \item Classifier 3 probability: 0.15 (Virginica)
\end{itemize}
Since Classifier 2 gives the highest probability, we predict this flower is a Versicolor.
\end{example2}

\subsection{Evaluating Classification Models}

\begin{definition}{Classification Metrics}\\
Common metrics for evaluating binary classification models:
\begin{itemize}
    \item \textbf{Accuracy}: Proportion of correct predictions
    \item \textbf{Precision}: Proportion of positive predictions that are correct
    \item \textbf{Recall}: Proportion of actual positives that are correctly identified
    \item \textbf{F1-score}: Harmonic mean of precision and recall
    \item \textbf{ROC curve}: Plot of true positive rate vs. false positive rate at different thresholds
    \item \textbf{AUC}: Area under the ROC curve
\end{itemize}
\end{definition}

\begin{KR}{Choosing the Right Evaluation Metric}\\
\paragraph{Consider class balance}
\begin{itemize}
    \item For balanced classes: Accuracy is often sufficient
    \item For imbalanced classes: Consider precision, recall, F1-score
\end{itemize}

\paragraph{Consider business context}
\begin{itemize}
    \item False positives more costly: Focus on precision
    \item False negatives more costly: Focus on recall
    \item Need to balance both: Use F1-score
\end{itemize}

\paragraph{Consider probability calibration}
\begin{itemize}
    \item If probabilities are important (not just class predictions): Use log loss or AUC
    \item For comparing different types of models: AUC is often useful
\end{itemize}

\paragraph{Use threshold-independent metrics}
If the classification threshold might change:
\begin{itemize}
    \item Use ROC curve to visualize performance across thresholds
    \item Use AUC to get a single performance number
\end{itemize}
\end{KR}

\begin{example}{Confusion Matrix Analysis}
Consider a medical test for a disease with 1000 test cases:
\begin{itemize}
    \item True Positives (TP): 150 (correctly identified disease cases)
    \item False Positives (FP): 50 (incorrectly flagged as disease)
    \item True Negatives (TN): 700 (correctly identified healthy cases)
    \item False Negatives (FN): 100 (missed disease cases)
\end{itemize}

Evaluation metrics:
\begin{itemize}
    \item Accuracy = $\frac{TP+TN}{TP+FP+TN+FN} = \frac{150+700}{1000} = 0.85$ (85\%)
    \item Precision = $\frac{TP}{TP+FP} = \frac{150}{150+50} = 0.75$ (75\%)
    \item Recall = $\frac{TP}{TP+FN} = \frac{150}{150+100} = 0.60$ (60\%)
    \item F1-score = $\frac{2 \times Precision \times Recall}{Precision + Recall} = \frac{2 \times 0.75 \times 0.60}{0.75 + 0.60} = 0.67$
\end{itemize}
\end{example}