\section{Naive Bayes}

\begin{definition}{Naive Bayes}\\
Naive Bayes classifiers are a family of simple "probabilistic classifiers" based on applying Bayes' theorem with strong (naive) independence assumptions between the features.
\end{definition}

\subsection{Bayes Theorem}

\begin{formula}{Bayes Theorem}
\begin{itemize}
    \item $X$ is a data tuple
    \item $H$ is some hypothesis, such as $X$ belongs to a specified class $C$
    \item $P(H|X)$ is the posteriori probability of $H$ conditioned on $X$
    \item $P(C_i|X)$ is the probability that $X$ belongs to $C_i$, given that we know the attribute description of $X$
\end{itemize}

$$P(H|X) = \frac{P(X|H) \cdot P(H)}{P(X)}$$

$$P(C_i|X) = \frac{P(X|C_i) \cdot P(C_i)}{P(X)}$$
\end{formula}

\subsection{Advantages and Disadvantages}

\begin{concept}{Advantages}
\begin{itemize}
    \item Fast to train and classify
    \item Performance is similar to decision trees and neural networks
    \item Easy to implement
    \item Handles numeric and categorical data
    \item Useful for very large data sets
\end{itemize}
\end{concept}

\begin{concept}{Disadvantages}
\begin{itemize}
    \item Assumes class conditional independence, therefore, loss of accuracy
    \item Model is difficult to interpret
\end{itemize}
\end{concept}

\subsection{Naive Bayes Example}

\begin{example2}{Computer Purchase Prediction}\\
$C_1 = buys_{computer} = yes$, $C_2 = buys_{computer} = no$

$X = (age = youth; income = med; student = yes; creditrating = fair)$

\textbf{Step 1: Compute $P(C_i)$}
\begin{itemize}
    \item $P(C_1) = 0.643$
    \item $P(C_2) = 0.357$
\end{itemize}

\textbf{Step 2: Compute $P(X|C_i)$}
\begin{itemize}
    \item $P(X|C_1) = P(age = youth|C_1) \cdot P(income = med|C_1) \cdots = 0.044$
    \item $P(X|C_2) = P(age = youth|C_2) \cdot P(income = med|C_2) \cdots = 0.019$
\end{itemize}

\textbf{Step 3: Compute $P(X|C_i) \cdot P(C_i)$}
\begin{itemize}
    \item $P(X|C_1) \cdot P(C_1) = 0.044 \cdot 0.643 = 0.028$
    \item $P(X|C_2) \cdot P(C_2) = 0.019 \cdot 0.357 = 0.007$
\end{itemize}
\end{example2}

% NOTE: Add table showing the training data with RID, age, income, student, credit_rating, buys_computer columns