\section{Authorization}

\mult{2}

\begin{definition}{Authorization}
determining whether an authenticated entity is permitted to perform a requested action:
\begin{itemize}
    \item Follows successful authentication
    \item Controls access to resources and operations
    \item Implements security policies
    \item Enforces principle of least privilege
\end{itemize}
\end{definition}

\begin{concept}{Fundamental Building Blocks}
\begin{itemize}
    \item \textbf{Access Control Model} - Conceptual framework defining how access decisions are made
    \item \textbf{Security Policy} - Rules defining who can access what resources
    \item \textbf{Security Mechanism} - Technical implementation enforcing the security policy
\end{itemize}
\end{concept}

\begin{theorem}{Security Policy Drivers}
\begin{itemize}
    \item \textbf{Business Drivers} - Efficiency, usability, operational requirements
    \item \textbf{Security Drivers} - Risk management, threat mitigation, data protection
    \item \textbf{Regulatory Drivers} - Legal compliance, industry standards
\end{itemize}
\end{theorem}



\subsubsection{Discretionary Access Control (DAC)}


\begin{definition}{Discretionary Access Control}
    resource owner controls access rights:
\begin{itemize}
    \item Resource owners determine who can access their resources
    \item Owners can delegate access control to others
    \item Access rights can be transferred between users
    \item Most common model in general-purpose operating systems
\end{itemize}
\end{definition}

\begin{concept}{DAC Implementation Methods}

    \textbf{Access Control Lists (ACLs)}:
    \begin{itemize}
        \item Lists of permissions attached to objects
        \item Each entry specifies subject and allowed operations
        \item Efficiently answers "who can access this object?"
    \end{itemize}

    \textbf{Capabilities}:
    \begin{itemize}
        \item Unforgeable tokens owned by subjects
        \item Each token grants specific access rights
        \item Efficiently answers "what can this subject access?"
    \end{itemize}

\end{concept}

\begin{theorem}{DAC Advantages and Disadvantages}

    \textbf{Advantages}:
    Flexible and intuitive for users, 
    Simple to understand and administer,
    Enables fine-grained access control

    \textbf{Disadvantages}:
    Vulnerable to Trojan horse attacks,
    Cannot enforce system-wide security policies,
    May lead to uncontrolled information flow,
    Prone to privilege creep over time

\end{theorem}

\multend

\mult{2}


\begin{concept}{Major Access Control Models}

\begin{itemize}
    \item \textbf{Discretionary Access Control (DAC)} - Access decisions made by resource owners
    \item \textbf{Mandatory Access Control (MAC)} - Access decisions enforced by system policy
    \item \textbf{Role-Based Access Control (RBAC)} - Access based on user roles within an organization
\end{itemize}
Additional models include:
\begin{itemize}
    \item \textbf{Attribute-Based Access Control (ABAC)} - Access based on attributes of users, resources, and environment
    \item \textbf{Rule-Based Access Control} - Access based on predefined rules
    \item \textbf{Relationship-Based Access Control (ReBAC)} - Access based on relationships between entities
\end{itemize}
\end{concept}

\begin{definition}{Standard UNIX Permissions}
UNIX-style file permissions implement a simple form of DAC:
\begin{itemize}
    \item Each file/directory has an owner and group
    \item Permissions divided into three categories:
    \begin{itemize}
        \item Owner (user) permissions
        \item Group permissions
        \item Others (world) permissions
    \end{itemize}
    \item Each category can have read (r), write (w), and execute (x) permissions
    \item Represented as 9 bits, often shown in octal notation (e.g., 755)
\end{itemize}
\end{definition}



\begin{KR}{UNIX Permission Interpretation}

Permission bits have different meanings for files and directories:

\textbf{For Files}:
    \begin{itemize}
        \item Read (r) - View file contents
        \item Write (w) - Modify file contents
        \item Execute (x) - Run file as a program
    \end{itemize}
\textbf{For Directories}:
    \begin{itemize}
        \item Read (r) - List directory contents
        \item Write (w) - Create, delete, or rename files within directory
        \item Execute (x) - Access files and subdirectories (traverse)
    \end{itemize}
\end{KR}


\begin{concept}{Special Permission Bits}\\
UNIX systems have additional permission bits for special purposes:
\begin{itemize}
    \item \textbf{Setuid bit (4000)} - When set on executable files, the program runs with the permissions of the file owner
    \item \textbf{Setgid bit (2000)}:
    \begin{itemize}
        \item On executable files: Program runs with the permissions of the file's group
        \item On directories: New files inherit the directory's group
    \end{itemize}
    \item \textbf{Sticky bit (1000)} - On directories, files can only be deleted or renamed by their owner (commonly used on /tmp)
\end{itemize}
\end{concept}



\begin{definition}{POSIX ACLs}
POSIX Access Control Lists extend standard UNIX permissions:
\begin{itemize}
    \item Allow specifying permissions for multiple users and groups
    \item Maintain backward compatibility with standard permissions
    \item Include:
    \begin{itemize}
        \item Minimal ACLs - Equivalent to standard permissions
        \item Extended ACLs - Additional users/groups with specific permissions
    \end{itemize}
\end{itemize}
\end{definition}

\begin{concept}{ACL Components}
\begin{itemize}
    \item \textbf{Owner entry} - Permissions for the file owner (user::rwx)
    \item \textbf{Named user entries} - Permissions for specific users (user:alice:rwx)
    \item \textbf{Group owner entry} - Permissions for the file's group (group::rwx)
    \item \textbf{Named group entries} - Permissions for specific groups (group:developers:rwx)
    \item \textbf{Other entry} - Permissions for everyone else (other::rwx)
    \item \textbf{Mask entry} - Maximum permissions for named users/groups and group owner
\end{itemize}
\end{concept}


\subsubsection{Mandatory Access Control (MAC)}

\begin{definition}{Mandatory Access Control}
  system-wide policy determines access rights:
\begin{itemize}
    \item Access decisions enforced by the system, not resource owners
    \item Based on security labels assigned to subjects and objects
    \item Security policy defined by system administrators
    \item Users cannot override or modify access controls
\end{itemize}
\end{definition}

\begin{concept}{MAC Implementations}
\begin{itemize}
    \item \textbf{SELinux (Security-Enhanced Linux)}:
    \begin{itemize}
        \item Developed by NSA
        \item Label-based access control
        \item Fine-grained policy control
        \item Used in Red Hat/Fedora distributions
    \end{itemize}
    \item \textbf{AppArmor}:
    \begin{itemize}
        \item Path-based access control
        \item Simpler configuration than SELinux
        \item Used in Ubuntu/SUSE distributions
    \end{itemize}
    \item \textbf{Windows Mandatory Integrity Control}:
    \begin{itemize}
        \item Introduced in Windows Vista
        \item Based on integrity levels (Low, Medium, High, System)
        \item Prevents lower integrity processes from modifying higher integrity resources
    \end{itemize}
\end{itemize}
\end{concept}

\multend

\begin{theorem}{MAC vs. DAC Comparison}

    \begin{minipage}{0.45\linewidth}
\textbf{Control}:
    \begin{itemize}
        \item DAC - Resource owners control access
        \item MAC - System policy controls access
    \end{itemize}
\textbf{Policy Management}:
    \begin{itemize}
        \item DAC - Distributed management
        \item MAC - Centralized management
    \end{itemize}
\end{minipage}
    \begin{minipage}{0.55\linewidth}
\textbf{Security Assurance}:
    \begin{itemize}
        \item DAC - Lower, vulnerable to user errors
        \item MAC - Higher, enforces system-wide policy
    \end{itemize}
\textbf{Complexity}:
    \begin{itemize}
        \item DAC - Simpler to implement and understand
        \item MAC - More complex, requires specialized knowledge
    \end{itemize}
    \end{minipage}

\end{theorem}

\raggedcolumns
\columnbreak



\mult{2}

\subsubsection{Role-Based Access Control (RBAC)}

\begin{definition}{Role-Based Access Control}
based on user roles within an organization:
\begin{itemize}
    \item Users are assigned to roles based on job responsibilities
    \item Permissions are assigned to roles, not directly to users
    \item Users acquire permissions through role membership
    \item Standardized in INCITS 359-2004
\end{itemize}
\end{definition}

\begin{concept}{RBAC Components}
\begin{itemize}
    \item \textbf{Users} - Entities that need access to resources
    \item \textbf{Roles} - Collections of permissions representing job functions
    \item \textbf{Permissions} - Approved operations on protected resources
    \item \textbf{Sessions} - Mappings between users and their activated roles
\end{itemize}
\end{concept}

\begin{theorem}{RBAC Security Principles}
\begin{itemize}
    \item \textbf{Principle of Least Privilege} - Users have only the permissions needed for their job
    \item \textbf{Separation of Duties} - Critical operations require multiple users
    \item \textbf{Data Abstraction} - Permissions defined at higher levels than individual objects
\end{itemize}
\end{theorem}

\subsubsection{Attribute-Based Access Control (ABAC)}

\begin{definition}{Attribute-Based Access Control}\\
ABAC is a flexible access control model based on attributes:
\begin{itemize}
    \item Access decisions based on attributes of:
    \begin{itemize}
        \item Subjects (users, applications)
        \item Resources (data, services)
        \item Actions (read, write, execute)
        \item Environment (time, location, security level)
    \end{itemize}
    \item Uses policies that combine attributes with logical rules
    \item More dynamic than traditional RBAC
    \item Can implement complex, context-aware policies
\end{itemize}
\end{definition}

\begin{concept}{ABAC vs. RBAC}\\
ABAC extends beyond traditional RBAC:

\textbf{Flexibility}:
    \begin{itemize}
        \item RBAC - Predetermined access based on roles
        \item ABAC - Dynamic access based on multiple attributes
    \end{itemize}

    \textbf{Granularity}:
    \begin{itemize}
        \item RBAC - Role-level permissions
        \item ABAC - Fine-grained, attribute-based decisions
    \end{itemize}

     \textbf{Context Awareness}:
    \begin{itemize}
        \item RBAC - Limited environmental context
        \item ABAC - Rich environmental attributes (time, location, etc.)
    \end{itemize}
\textbf{Policy Expression}:
    \begin{itemize}
        \item RBAC - Role-permission assignments
        \item ABAC - Complex boolean logic combining attributes
    \end{itemize}

\end{concept}


\multend


\subsubsection{Zero Trust Security Model}

\begin{definition}{Zero Trust} security model that assumes no implicit trust for any entity:
\begin{itemize}
    \item Eliminates the concept of trusted internal networks
    \item Requires continuous verification for all access requests
    \item Applies least privilege access controls
    \item Inspects and logs all traffic
    \item Focuses on protecting resources rather than network segments
\end{itemize}
\end{definition}

\begin{concept}{Zero Trust Principles}
\begin{itemize}
    \item \textbf{Verify explicitly} - Always authenticate and authorize based on all available data points
    \item \textbf{Use least privilege access} - Limit user access with Just-In-Time and Just-Enough-Access
    \item \textbf{Assume breach} - Minimize blast radius and segment access
    \item \textbf{Identity-based security} - Identity becomes the primary security perimeter
    \item \textbf{Continuous monitoring} - Collect and analyze data continuously
\end{itemize}
\end{concept}


% Duplicate content removed - comprehensive authorization coverage is provided above