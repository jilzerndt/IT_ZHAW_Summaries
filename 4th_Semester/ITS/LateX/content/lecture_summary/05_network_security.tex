\section{Network Security}

\mult{2}

\begin{definition}{Network Segmentation}
dividing a computer network into smaller, isolated segments:
\begin{itemize}
    \item Increases operational performance by containing traffic
    \item Limits damage from cyber attacks to specific segments
    \item Protects vulnerable devices by isolating them
    \item Reduces the scope of compliance requirements
    \item Protects against insider threats
\end{itemize}
\end{definition}

\begin{concept}{Typical Network Segments}

    \textbf{DMZ (Demilitarized Zone)} - Hosts public-facing services

    \textbf{Server Network} - Hosts internal servers and applications

    \textbf{Client Network} - Hosts employee workstations

    \textbf{Guest Network} - Hosts visitor devices with limited access

    \textbf{IoT Network} - Hosts Internet of Things devices

    \textbf{Industrial Network} - Hosts industrial control systems

    \textbf{Admin Network} - Hosts administrative systems and tools

\end{concept}

\begin{theorem}{Zero Trust Challenges}
\begin{itemize}
    \item Single points of failure in policy enforcement
    \item Potential for accidental or malicious misconfigurations
    \item Vulnerability to denial-of-service attacks
    \item Difficulties with encrypted traffic visibility
    \item Lack of protection against credential theft
\end{itemize}
\end{theorem}


\begin{definition}{Microsegmentation}
\begin{itemize}
    \item Segments as small as individual workloads or applications
    \item Implemented through network virtualization or host-based firewalls
    \item Limits lateral movement within a traditional network segment
    \item Reduces the attack surface and blast radius
    \item Balances security with management complexity
\end{itemize}
\end{definition}

\begin{concept}{Zero Trust Architecture}
assumes no implicit trust within or outside the network:

\textbf{Core Principles}:
    \begin{itemize}
        \item Verify explicitly - Always authenticate and authorize
        \item Use least privilege access - Provide minimal necessary access
        \item Assume breach - Operate as if attackers are already present
    \end{itemize}
    \textbf{Components}:
    \begin{itemize}
        \item Policy Enforcement Point (PEP) - Enforces access decisions
        \item Policy Decision Point (PDP) - Makes access decisions
        \item Policy Administration Point (PAP) - Manages policies
    \end{itemize}
\end{concept}



\multend

\subsubsection{Firewalls}

\mult{2}

\begin{definition}{Packet Filtering Firewalls}
control traffic flow between network segments:
\begin{itemize}
    \item Examine packet headers to make filtering decisions
    \item Apply rules based on source/destination addresses, ports, and protocols
    \item Operate at the network and transport layers of the OSI model
    \item Form the foundation of network security architecture
\end{itemize}
\end{definition}

\begin{concept}{Firewall Rule Management}
\begin{itemize}
    \item \textbf{New Request} - Submission, approval, design, testing, deployment, validation
    \item \textbf{Operation} - Monitoring, audits, reports
    \item \textbf{Recertification} - Regular review of existing rules
    \item \textbf{Decommissioning} - Removal of unnecessary rules
\end{itemize}
Without proper management processes, firewall rule sets tend to grow uncontrolled and become difficult to understand.
\end{concept}






\begin{definition}{Next Generation Firewalls (NGFW)}
additional features:

\textbf{Deep Packet Inspection} Examines packet contents

\textbf{Application Awareness} IDs/controls traffic by application

\textbf{Intrusion Prevention} Detects and blocks network attacks

\textbf{Antivirus/Anti-malware} Scans traffic for malicious content

\textbf{Sandboxing} exec suspicious files in isolated environments

\textbf{Threat Intelligence} inc. external information about threats

\end{definition}



\begin{definition}{Host-Based Firewalls}
additional layer of protection:
\begin{itemize}
    \item Operate directly on individual devices
    \item Protect against threats that bypass network firewalls
    \item Filter traffic based on local policies
    \item Examples include Windows Defender Firewall, Linux nftables, and macOS firewall
\end{itemize}
\end{definition}

\begin{concept}{Endpoint Detection and Response (EDR)}
combine several capabilities to secure endpoints:

\textbf{Monitoring}:
    Anomaly detection, vulnerability scanning, integrity checks

    \textbf{Protection}:
    Host-based firewall, antivirus, app control

    \textbf{Investigation and Response}:
    device isolation, user session termination, change rollback, evidence collection
\end{concept}


\begin{definition}{Web Application Firewalls (WAF)}
protect web applications from attacks:
\begin{itemize}
    \item Examine HTTP/HTTPS traffic at the application layer
    \item Protect against OWASP Top 10 vulnerabilities:
    Cross-Site Scripting (XSS), SQL Injection, Cross-Site Request Forgery (CSRF)
    \item Require TLS termination to inspect encrypted traffic
    \item Deploy in front of web servers
\end{itemize}
\end{definition}

\begin{theorem}{Firewall Benefits and Limitations}

    \textbf{Benefits}:
    Block unwanted traffic before it enters protected networks, Centralize access control, Hide internal network structure

\textbf{Limitations}:
Primarily perimeter protection (assumes threats are external), Cannot prevent internal threats, Basic packet filtering cannot detect application-layer attacks,
Cannot protect against allowed protocols misuse

\end{theorem}


\begin{definition}{Secure Web Gateway (SWG)}
protect users from web-based threats:
\begin{itemize}
    \item URL filtering to block malicious sites
    \item Data Loss Prevention (DLP) to prevent sensitive data exfiltration
    \item TLS inspection to examine encrypted traffic
    \item User and application control
\end{itemize}
\end{definition}

\begin{concept}{Cloud Access Security Broker (CASB)}
provide security controls for cloud services:

\textbf{Shadow IT discovery} ID unauthorized cloud service usage

\textbf{Cloud usage control} Set access rights to cloud services

\textbf{Data leakage prevention} ctrl data sharing in cloud services

\textbf{Anomaly detection} Alert on unusual behavior patterns

\textbf{Implementation methods}:
    \begin{itemize}
        \item API scanning - Directly interfaces with cloud providers
        \item Forward proxy - Controls outbound access to cloud services
        \item Reverse proxy - Intermediates between users and cloud services
    \end{itemize}
\end{concept}




\begin{definition}{Network Detection and Response (NDR)}
monitor network traffic to detect and respond to threats:
\begin{itemize}
    \item Establish baseline network behavior
    \item Detect anomalies that may indicate attacks
    \item Analyze potential incidents to identify true positives
    \item Automatically respond to confirmed threats
\end{itemize}
\end{definition}

\begin{concept}{Security Information and Event Management (SIEM)}
 collect and analyze security events:
\begin{itemize}
    \item Aggregate logs from multiple sources
    \item Correlate events to identify attack patterns
    \item Provide a centralized dashboard for security monitoring
    \item Generate reports for compliance and security analysis
\end{itemize}
\end{concept}

\begin{concept}{Security Orchestration\, Automation and Response (SOAR)}
extend SIEM capabilities:
\begin{itemize}
    \item Automate security response with playbooks
    \item Integrate with multiple security tools
    \item Orchestrate complex security workflows
    \item Enhance incident response with automation
\end{itemize}
\end{concept}

\multend

\subsubsection{Linux Firewall with nftables}

\begin{definition}{nftables Framework}
nftables is a packet classification and filtering framework in Linux:
\begin{itemize}
    \item Replaces the legacy iptables framework
    \item Part of the Linux kernel since version 2.4
    \item Provides packet filtering, network address translation, and packet mangling
    \item Configured using the nft command-line tool
\end{itemize}
\end{definition}

\begin{concept}{nftables Architecture}
\begin{itemize}
    \item \textbf{Tables} - Group chains for a specific packet type (address family)
    \item \textbf{Chains} - Group rules and attach to hooks in the network stack
    \item \textbf{Rules} - Define matching criteria and actions for packets
    \item \textbf{Expressions} - Match packet properties (addresses, ports, etc.)
    \item \textbf{Actions} - Determine what happens to matching packets (accept, drop, reject, etc.)
\end{itemize}
\end{concept}

\begin{code}{Basic nftables Commands}
\begin{lstlisting}[language=bash, style=basesmolll]
# Create a table
nft add table inet filter
# Create a chain in the table
nft add chain inet filter input { type filter hook input priority 0 \; policy drop \; }
# Add a rule to the chain
nft add rule inet filter input tcp dport 22 accept
# List the ruleset
nft list ruleset
# Delete a rule (using its handle)
nft delete rule inet filter input handle 4
# Flush a table (delete all chains and rules)
nft flush table inet filter
\end{lstlisting}
\end{code}


\mult{2}

\begin{definition}{Port Scanning}
discover available services on a network:
\begin{itemize}
    \item Identifies open ports on target systems
    \item Helps determine running services and potential vulnerabilities
    \item Used by both attackers for reconnaissance and administrators for security testing
    \item Most commonly performed using tools like nmap
\end{itemize}
\end{definition}

\begin{examplecode}{Basic Port Scanning with nmap}
\begin{lstlisting}[language=bash, style=basesmolll]
# Scan a single host for common ports
nmap 192.168.1.1
# Scan specific ports
nmap -p 22,80,443 192.168.1.1
# Scan an entire subnet
nmap 192.168.1.0/24
\end{lstlisting}
\end{examplecode}


% Basic firewall concepts integrated into comprehensive network security section above



\begin{definition}{netfilter/nftables}
    netfilter is a mechanism that allows to access the packets in the network stack to analyze, modify, extract, and delete them.
    \begin{itemize}
        \item Hooks that are called at different points during packet processing
    \end{itemize}
    
    nftables is a packet classification and mangling framework that runs on rulesets that are applied to the packets
    \begin{itemize}
        \item \textbf{Table} Container for specific type of package
        \item \textbf{Chain} Container with rules for a specific hook
        \item \textbf{Hook} Called at different points of processing
        \item \textbf{Priority} Lowest priority first until rule is accepted
        \item \textbf{Policy} Default behaviour
        \item \textbf{Rule} Classification + Action
        \item \textbf{Classification} What packets does a rule apply to
        \item \textbf{Action} What to do with the packet
    \end{itemize}
\end{definition}





\begin{examplecode}{nftables Rules}
\begin{lstlisting}[language=bash, style=basesmolll]
# Drop all packets going to IPv4 address 8.8.8.8
ip daddr 8.8.8.8 drop
# Accept all packets coming on interface eth2
iifname eth2 accept
# Accept all IPv6 packets carrying TCP
ip6 nexthdr tcp accept
\end{lstlisting}
\end{examplecode}

\multend