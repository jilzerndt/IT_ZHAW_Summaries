\section{Transport Layer Security (TLS)}

\mult{2}

\begin{definition}{Transport Layer Security (TLS)}
TLS is a cryptographic protocol designed to provide secure communication over a computer network. It provides:
\begin{itemize}
    \item \textbf{Confidentiality} - Protection against eavesdropping
    \item \textbf{Integrity} - Detection of message tampering
    \item \textbf{Authentication} - Verification of communicating parties
    \item \textbf{Protection against replay attacks} - Prevention of message reuse
\end{itemize}
TLS is the successor to Secure Sockets Layer (SSL).
\end{definition}


\begin{concept}{TLS Building Blocks}
TLS 1.3 relies on several cryptographic building blocks:
\begin{itemize}
    \item Strong block ciphers (AES, ChaCha20)
    \item Authenticated encryption modes (GCM, CCM, Poly1305)
    \item Diffie-Hellman key exchange (including elliptic curve variants)
    \item Public key authentication with certificates
    \item Cryptographic hash functions for key derivation (HKDF)
\end{itemize}
\end{concept}

\begin{concept}{TLS Protocol Layers}
TLS consists of multiple protocol layers:
\begin{itemize}
    \item \textbf{TLS Record Protocol} - The base layer that defines packet format
    \item \textbf{Handshake Protocol} - Negotiates cryptographic parameters and authenticates parties
    \item \textbf{Alert Protocol} - Communicates errors and warnings
    \item \textbf{Application Data Protocol} - Transmits encrypted application data
    \item \textbf{Change Cipher Spec Protocol} - (Deprecated in TLS 1.3, retained for backward compatibility)
\end{itemize}
\end{concept}

\begin{definition}{TLS Protocol Phases}
TLS operates in three distinct phases:
\begin{itemize}
    \item \textbf{Handshake} - Establishes cryptographic parameters and authenticates parties
    \item \textbf{Data Exchange} - Transfers protected application data
    \item \textbf{Connection Teardown} - Securely terminates the connection
\end{itemize}
\end{definition}

\multend



\subsubsection{TLS Protocol Stack}

\mult{2}




\begin{concept}{TLS 1.3 Handshake Goals}
The TLS 1.3 handshake accomplishes several goals:
\begin{itemize}
    \item Negotiate cryptographic algorithms
    \item Perform Diffie-Hellman key exchange
    \item Generate handshake keys for encryption of subsequent messages
    \item Authenticate the server (and optionally the client)
    \item Verify message integrity during the handshake
    \item Generate keys for data encryption
\end{itemize}
\end{concept}

\begin{KR}{TLS 1.3 Handshake Process}

\textbf{Initial Exchange}
\begin{itemize}
    \item Client sends ClientHello message with:
    \begin{itemize}
        \item Supported TLS versions
        \item Supported cipher suites
        \item Supported key exchange groups
        \item Client's key shares for preferred groups
    \end{itemize}
    \item Server responds with ServerHello message:
    \begin{itemize}
        \item Selected TLS version
        \item Selected cipher suite
        \item Server's key share for selected group
    \end{itemize}
\end{itemize}

\textbf{Key Derivation}
\begin{itemize}
    \item Both parties compute shared secret using Diffie-Hellman
    \item Handshake keys are derived using HKDF with the shared secret
    \item All subsequent handshake messages are encrypted
\end{itemize}

\textbf{Server Authentication} Server sends...
\begin{itemize}
    \item Certificate message with certificate chain
    \item CertificateVerify with signature over handshake transcript
    \item Finished message with MAC over handshake transcript
\end{itemize}

\textbf{Client Verification} Client...
\begin{itemize}
    \item validates server certificate
    \item verifies server's signature in CertificateVerify
    \item verifies server's Finished message
    \item sends its own Finished message
\end{itemize}

\textbf{Application Data}
\begin{itemize}
    \item Both parties derive application data keys
    \item Encrypted application data can be exchanged
\end{itemize}
\end{KR}




\begin{concept}{TLS Record Protection} protects app data
\begin{itemize}
    \item Data is fragmented into manageable chunks
    \item Each fragment is encrypted and authenticated using AEAD ciphers
    \item A sequence number is included in the authentication to prevent replay attacks
    \item TLS records include headers that are not encrypted but are authenticated
\end{itemize}
\end{concept}


\begin{definition}{TLS Session Resumption}
Session resumption allows clients to reconnect to servers more efficiently:
\begin{itemize}
    \item Avoids the computational cost of public key operations
    \item Reduces the number of round-trips required
    \item In TLS 1.3, uses pre-shared keys (PSKs) derived from previous connections
    \item Server provides NewSessionTicket messages containing ticket data
    \item Client can use this ticket in future connections
\end{itemize}
\end{definition}

\begin{concept}{TLS Connection Termination}
\begin{itemize}
    \item Uses close\_notify alerts to signal completion of data transmission
    \item Prevents truncation attacks where an attacker prematurely terminates a connection
    \item Both parties should send close\_notify alerts before closing the underlying TCP connection
\end{itemize}
\end{concept}




\begin{definition}{Datagram Transport Layer Security (DTLS)}\\
DTLS is an adaptation of TLS for datagram transport protocols like UDP:
\begin{itemize}
    \item Based on TLS with minimal modifications for unreliable transport
    \item Uses explicit sequence numbers to handle packet reordering
    \item Implements message loss detection and retransmission for handshake messages
    \item Provides optional replay detection for application data
    \item Current version is DTLS 1.3 (RFC 9147)
\end{itemize}
\end{definition}

\multend




% IPsec content moved to Virtual Private Networks chapter for better organization