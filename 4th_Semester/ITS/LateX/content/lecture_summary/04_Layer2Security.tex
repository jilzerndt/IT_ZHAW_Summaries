\section{Layer 2 Security}


\begin{minipage}{0.4\linewidth}

\begin{concept}{Security at Different OSI Layers}
\begin{itemize}
    \item \textbf{Physical Layer (Layer 1)} - Quantum cryptography, physical isolation
    \item \textbf{Data Link Layer (Layer 2)} - IEEE 802.1X, WLAN security (WEP, WPA, WPA2, WPA3), MACsec
    \item \textbf{Network Layer (Layer 3)} - IPsec
    \item \textbf{Transport Layer (Layer 4)} - TLS, QUIC
    \item \textbf{Application Layer (Layer 7)} - PGP, S/MIME, Signal, WhatsApp
\end{itemize}
\end{concept}
\end{minipage}
\begin{minipage}{0.6\linewidth}

\begin{theorem}{Layer Selection Tradeoffs}

    \textbf{Higher Layer Security}:
    \begin{itemize}
        \item Easier to deploy (often included in applications)
        \item Typically provides end-to-end protection
        \item Less generally applicable (often optimized for specific applications)
    \end{itemize}

    \textbf{Lower Layer Security}:
    \begin{itemize}
        \item More difficult to deploy (may require kernel, router, or switch modifications)
        \item Often provides only hop-to-hop protection at layer 2
        \item More generally applicable (protects all protocols above it)
    \end{itemize}

\end{theorem}
\end{minipage}

\begin{definition}{Extensible Authentication Protocol (EAP)}
framework for authentication that supports multiple authentication methods:
\begin{itemize}
    \item Provides a flexible authentication mechanism
    \item Designed to be transport-independent (can run over various protocols)
    \item Allows network access control without knowledge of each individual user
    \item Supports numerous authentication methods via different EAP types
\end{itemize}
\end{definition}

\begin{concept}{EAP Authentication Methods}
EAP supports numerous authentication methods with varying security properties:
\begin{itemize}
    \item \textbf{EAP-MD5} - Simple challenge-response mechanism (insecure, no mutual authentication)
    \item \important{EAP-TLS} - Uses TLS protocol with client certificates (strong security)
    \item \textbf{EAP-TTLS} - Tunneled TLS without requiring client certificates
    \item \textbf{PEAP} - Microsoft's Protected EAP, similar to EAP-TTLS
    \item \textbf{EAP-FAST} - Cisco's Flexible Authentication via Secure Tunneling
\end{itemize}
\end{concept}

\begin{code}{EAP Packet Format}
EAP packets have a simple structure:
\begin{lstlisting}[language=C, style=basesmolll]
struct eap_packet {
    uint8_t code;       // Request (1), Response (2), Success (3), Failure (4)
    uint8_t id;         // Used to match requests and responses
    uint16_t length;    // Total length of the EAP packet
    uint8_t type;       // When code=1,2: authentication method type
    uint8_t data[];     // Type-specific data
};
\end{lstlisting}
\end{code}


\mult{2}
\subsubsection{IEEE 802.1X Port-Based Access Control}



\begin{definition}{IEEE 802.1X}
port-based network access control:
\begin{itemize}
    \item Controls access to a network by authenticating devices before allowing network access
    \item Prevents unauthorized access through publicly accessible network ports
    \item Uses EAP for authentication
    \item Provides centralized VLAN assignment based on authentication
\end{itemize}
\end{definition}

\begin{concept}{802.1X Components}
\begin{itemize}
    \item \textbf{Supplicant} - The client device requesting network access
    \item \textbf{Authenticator} - The network device (switch, access point) controlling port access
    \item \textbf{Authentication Server} - Typically a RADIUS server that validates user credentials
\end{itemize}
\end{concept}

\begin{theorem}{Security Limitations of 802.1X}
While 802.1X provides significant security benefits, it has limitations:
\begin{itemize}
    \item Man-in-the-middle attacks are possible when an attacker inserts themselves between a legitimate device and the switch
    \item An attacker could monitor traffic by creating a bridge between the legitimate device and the switch
    \item Once initial authentication is complete, the ongoing session is not continuously verified
    \item These limitations are addressed in 802.1X-2010 with the use of MACsec
\end{itemize}
\end{theorem}


\begin{definition}{MACsec}
IEEE 802.1AE (Media Access Control Security)
\begin{itemize}
    \item Data confidentiality, integrity, and authenticity on layer 2 (Ethernet)
    \item Protection for all data including DHCP, ARP, and higher layer protocols
    \item Security for both physical and virtual links between devices
    \item Line-rate encryption in hardware implementations
\end{itemize}
\end{definition}

\begin{concept}{MACsec Operation}
MACsec operates by:
\begin{itemize}
    \item Encrypting the payload of Ethernet frames between adjacent devices
    \item Adding a SecTAG to identify the keys and provide packet numbering
    \item Integrity Check Value (ICV) based on GCM-AES
    \item Providing hop-by-hop protection where each device on the path decrypts and re-encrypts the data
\end{itemize}
\end{concept}

\begin{concept}{WPA/WPA2 Operation}
\begin{itemize}
    \item \textbf{Authentication Phase}:
    \begin{itemize}
        \item WPA-Personal: Uses pre-shared key (PSK)
        \item WPA-Enterprise: Uses 802.1X and EAP with a RADIUS server
    \end{itemize}
    \item \textbf{Key Exchange Phase}:
    \begin{itemize}
        \item Establishes pairwise transient keys for unicast traffic
        \item Establishes group keys for broadcast/multicast traffic
        \item Implements key rotation to prevent attacks
    \end{itemize}
\end{itemize}
\end{concept}


\subsubsection{WLAN Security}

\begin{definition}{WLAN Security Challenges}
\begin{itemize}
    \item No physical cable requirement for packet sniffing
    \item Signals extend beyond physical boundaries
    \item Difficult to control who can attempt to connect
    \item Need for strong encryption and authentication
\end{itemize}
\end{definition}

\begin{definition}{Wired Equivalent Privacy (WEP)}
OG security standard for 802.11 networks:
\begin{itemize}
    \item Shared key between all clients and the access point
    \item Uses RC4 stream cipher with 40 or 104-bit keys
    \item Uses a 24-bit Initialization Vector (IV)
    \item Includes a CRC-32 checksum for integrity checking
\end{itemize}
\end{definition}

\begin{theorem}{WEP Security Flaws}
\begin{itemize}
    \item Small IV space (24 bits) leads to inevitable IV reuse
    \item Weak key scheduling algorithm in RC4
    \item CRC-32 is not cryptographically secure for integrity protection
    \item No protection against bit-flipping attacks
    \item No replay protection
    \item No per-user/per-session keys
\end{itemize}
\end{theorem}


\begin{definition}{Wi-Fi Protected Access (WPA)}
interim solution while waiting for 802.11i (WPA2):
\begin{itemize}
    \item Introduced the Temporal Key Integrity Protocol (TKIP)
    \item Provided per-packet key mixing
    \item Added message integrity code (Michael)
    \item Implemented a frame counter to prevent replay attacks
    \item Available in personal (PSK) and enterprise (802.1X) modes
\end{itemize}
\end{definition}

\begin{definition}{WPA2 (IEEE 802.11i)}
complete implementation of the IEEE 802.11i standard:
\begin{itemize}
    \item Introduced Counter Mode with CBC-MAC Protocol (CCMP)
    \item Based on AES encryption (replacing RC4)
    \item Provides stronger authentication, integrity, and confidentiality
    \item Retained TKIP only for backward compatibility
    \item Available in personal (PSK) and enterprise (802.1X) modes
\end{itemize}
\end{definition}



\begin{definition}{WPA3}
latest Wi-Fi security standard:
\begin{itemize}
    \item Introduces Simultaneous Authentication of Equals (SAE) to replace WPA2-Personal PSK
    \item Provides forward secrecy
    \item Protects against offline dictionary attacks
    \item Includes enhanced protection for enterprise networks
    \item Improves encryption for public networks with Opportunistic Wireless Encryption (OWE)
\end{itemize}
\end{definition}

\multend