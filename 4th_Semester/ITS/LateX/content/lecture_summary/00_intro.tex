


\section{Overview of IT Security}


\begin{definition}{Key IT Security Goals}
Fundamental CIA triad:

\begin{minipage}{0.6\linewidth}
\begin{itemize}
    \item \textbf{Confidentiality} - Ensuring data is only accessible to authorized users
    \item \textbf{Integrity} - Ensuring data is not modified in an unauthorized way
    \item \textbf{Availability} - Ensuring systems and data are accessible when needed
\end{itemize}
\end{minipage}
\begin{minipage}{0.4\linewidth}
\includegraphics[width=\linewidth]{its_goals.png}
\end{minipage}
\end{definition}



\raggedcolumns

\mult{2}

\begin{concept}{Security Control Frameworks}\\
Security frameworks provide structured approaches to implementing security controls:
\begin{itemize}
    \item \textbf{CIS Controls} - Prioritized set of actions to protect organizations
    \item Controls are typically organized in implementation groups based on difficulty and impact
    \item Focus on preventing the most common attack vectors first
\end{itemize}
\end{concept}

\begin{definition}{Types of Security Measures}
Security measures can be categorized based on their focus:
\begin{itemize}
    \item \textbf{Preventive} - Block threats before they occur (firewalls, access controls)
    \item \textbf{Detective} - Identify when a breach has occurred (IDS, audit logs)
    \item \textbf{Corrective} - Mitigate damage after an incident (backups, incident response)
\end{itemize}
\end{definition}

\multend


\begin{theorem}{Business Continuity Management and Disaster Recovery}
for maintaining availability:
\begin{itemize}
    \item \textbf{Recovery Plan} - Detailed procedures for recovering from incidents
    \item \textbf{Recovery Tests} - Regular testing of recovery procedures
    \item \textbf{Redundancy} - Duplicate systems, power supplies, and network connections
    \item \textbf{Offline backups} - Protection against ransomware and other threats
\end{itemize}
\end{theorem}


\raggedcolumns




