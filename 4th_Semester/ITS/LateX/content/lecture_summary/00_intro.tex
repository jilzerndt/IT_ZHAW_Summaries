


\section{Overview of IT Security}

\mult{2}

\begin{definition}{Key IT Security Goals}
Information security is based on three fundamental principles, commonly known as the CIA triad:
\begin{itemize}
    \item \textbf{Confidentiality} - Ensuring data is only accessible to authorized users
    \item \textbf{Integrity} - Ensuring data is not modified in an unauthorized way
    \item \textbf{Availability} - Ensuring systems and data are accessible when needed
\end{itemize}
\includegraphics[width=0.6\linewidth]{its_goals.png}
\end{definition}

\begin{concept}{Business IT Risks}
\begin{itemize}
    \item Data loss
    \item System outage
    \item Espionage
    \item Sabotage
    \item Reputation loss
    \item Misuse of computing resources
    \item Violation of regulations
    \item Fraud
    \item Brand misuse
    \item Ransom demands
\end{itemize}
These risks can have significant financial, operational, and reputational impacts.
\end{concept}

\multend

\raggedcolumns

\subsubsection{Security Frameworks and Controls}

\mult{2}

\begin{concept}{Security Control Frameworks}
Security frameworks provide structured approaches to implementing security controls:
\begin{itemize}
    \item \textbf{CIS Controls} - Prioritized set of actions to protect organizations
    \item Controls are typically organized in implementation groups based on difficulty and impact
    \item Focus on preventing the most common attack vectors first
\end{itemize}
\end{concept}

\begin{definition}{Types of Security Measures}
Security measures can be categorized based on their focus:
\begin{itemize}
    \item \textbf{Preventive} - Block threats before they occur (firewalls, access controls)
    \item \textbf{Detective} - Identify when a breach has occurred (IDS, audit logs)
    \item \textbf{Corrective} - Mitigate damage after an incident (backups, incident response)
\end{itemize}
\end{definition}

\multend

\subsubsection{Disaster Recovery}



\begin{concept}{Business Continuity Management}
Disaster recovery and business continuity planning are essential for maintaining availability:
\begin{itemize}
    \item \textbf{Recovery Plan} - Detailed procedures for recovering from incidents
    \item \textbf{Recovery Tests} - Regular testing of recovery procedures
    \item \textbf{Redundancy} - Duplicate systems, power supplies, and network connections
    \item \textbf{Offline backups} - Protection against ransomware and other threats
\end{itemize}
\end{concept}

\begin{KR}{Disaster Recovery Planning}
\paragraph{Initial Assessment}
\begin{itemize}
    \item Identify critical systems and data
    \item Determine acceptable recovery time objectives (RTO)
    \item Determine acceptable recovery point objectives (RPO)
\end{itemize}

\paragraph{Plan Development}
\begin{itemize}
    \item Document recovery procedures
    \item Assign roles and responsibilities
    \item Include contact details for all relevant parties
    \item Develop technical instructions for restoration
\end{itemize}

\paragraph{Testing}
\begin{itemize}
    \item Conduct regular theoretical dry runs
    \item Perform practical tests (e.g., server shutdown, data restoration)
    \item Update procedures based on test results
\end{itemize}

\paragraph{Regular Review}
\begin{itemize}
    \item Review and update plans regularly
    \item Consider changes in infrastructure, personnel, and threats
\end{itemize}
\end{KR}

\begin{example}
A medium-sized company implements a disaster recovery plan for their customer database. They define an RTO of 4 hours and an RPO of 15 minutes, meaning they need to restore service within 4 hours with no more than 15 minutes of data loss. To achieve this, they implement a combination of hourly differential backups with continuous transaction log shipping to a standby site. Regular recovery tests are scheduled quarterly to ensure the plan remains effective.
\end{example}

\mult{2}


\begin{concept}{Problems - Overview}\\ and their impact on data availability
\paragraph{Physisch}
\textbf{unabsichtlich}
\begin{itemize}
    \item Naturkatastrophen
    \item Feuer
    \item Ausfall
    \item Kaffee auf Server
\end{itemize}

\textbf{absichtlich/bösartig}
\begin{itemize}
    \item Feuer
    \item Vandalismus
    \item Garantie läuft aus -> absichtlich langsamer
    \item Social Engineering
\end{itemize}

\paragraph{Virtuell}
\textbf{unabsichtlich}
\begin{itemize}
    \item Bitflip
    \item Config Fehler
    \item Bugs im SW
    \item Phishing klicken
\end{itemize}

\textbf{absichtlich/bösartig}
\begin{itemize}
    \item DDoS
    \item Malware
    \item Ransomware
    \item Phishing senden
    \item Trojaner
\end{itemize}
\end{concept}

\begin{theorem}{Countermeasures - Overview}

\textbf{Disaster Recovery}
    \begin{itemize}
        \item Offline backup solutions
        \item Restoring from images
    \end{itemize}

\textbf{Access Control}
    \begin{itemize}
        \item Restricted Access Rights
        \item Multi-Factor Authentication
        \item Firewalls
        \item Traffic Management Solutions
    \end{itemize}

\textbf{Physical Protection}
    \begin{itemize}
        \item Physical Access Control (locks, fences, etc.)
        \item Fire Protection (extinguishers, alarms, etc.)
        \item Monitoring (CCTV, Guards etc.)
    \end{itemize}

\textbf{Training Processes}
    \begin{itemize}
        \item Employee Training
        \item Four eyes principle
        \item Automation of routine processes
        \item Monitoring
        \item Preventive maintenance
    \end{itemize}

\textbf{Redundancy}
    \begin{itemize}
        \item Uninterruptable Power Supplies
        \item High Availability setups
        \item Load Balancing
        \item Redundant data center
        \item Redundant network connections
    \end{itemize}
\end{theorem}

\multend

\mult{2}

\begin{formula}{Recovery Plan and Test}

    \includegraphics[width=0.7\linewidth]{recovery_plan_test.png}

    \textbf{Recovery Plan} - description of what to do if something goes wrong
    \begin{itemize}
        \item Roles and responsibilities
        \item Processes
        \item Contact details
        \item Technical instructions
    \end{itemize}

    \textbf{Recovery Test} - testing the recovery plan
    \begin{itemize}
        \item Theoretical dry run
        \item Practical tests
        \begin{itemize}
            \item turn off a server or DC
            \item restore data from backup
        \end{itemize}
    \end{itemize}
\end{formula}


\multend

\raggedcolumns




