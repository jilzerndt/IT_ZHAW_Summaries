\section{User Authentication}

\mult{2}

\begin{definition}{Access Control Framework}

    \textbf{Identification} Claiming an identity (username)

    \textbf{Authentication} Proving the claimed identity (password)

    \textbf{Authorization} Determining what the authenticated identity may do

     \textbf{Accounting/Auditing} - Recording what actions were performed

\end{definition}

\begin{concept}{Authentication Fundamentals}
    verifying that a claimed identity is genuine:
\begin{itemize}
    \item Establishes trust in the identity claim
    \item Serves as the foundation for access control
    \item Must be completed before authorization
    \item Should be resistant to impersonation attacks
\end{itemize}
\end{concept}
\multend


\mult{2}

\begin{definition}{Authentication Factors}

    \textbf{Something you know} Knowledge factors (passwords, PINs)

    \textbf{Something you have} Possession factors (physical tokens)

    \textbf{Something you are} Inherence factors (biometrics)

    \textbf{Somewhere you are} - Location factors (GPS)

     \textbf{Something you do} - Behavior factors 

\end{definition}


\begin{concept}{Password Security Challenges}

     \textbf{Human memory limitations} - Difficult to remember strong, unique passwords

     \textbf{Password reuse} - Users often use the same password across multiple sites

     \textbf{Password sharing} - Accounts accessed by multiple people

     \textbf{Password theft} - Phishing, keylogging, and data breaches

     \textbf{Brute force attacks} - Systematic guessing of passwords

\end{concept}


\subsubsection{Secure Password Storage}

\begin{definition}{Password Hashing}

    one-way transformation to securely store passwords:

    Converts passwords into fixed-length strings that cannot be reversed

    Allows verification without storing the actual password

     Must use cryptographic hash functions designed for security

     Should be combined with salting and key stretching

\end{definition}

\begin{concept}{Salting Passwords}

    adding random data to passwords before hashing:

    Prevents use of precomputed tables (rainbow tables) for cracking

    Forces attackers to crack each password individually

    Salt should be unique per user and sufficiently long (64-128 bits)

    Salt is stored alongside the hash in the password database

\end{concept}

\begin{concept}{Key Stretching}
    increases the computational work needed to compute password hashes:
\begin{itemize}
    \item Applies the hash function repeatedly (thousands or millions of times)
    \item Increases the time needed for brute force attacks
    \item Can be adjusted as hardware becomes faster
    \item Implemented in specialized password hashing functions
\end{itemize}
\end{concept}

\begin{theorem}{Password Hashing Functions}
     password storage:

     \textbf{bcrypt} - Based on Blowfish cipher, salt and cost factor

     \textbf{PBKDF2} - Password-Based Key Derivation Function, configurable iterations

     \textbf{Argon2} - Winner of the Password Hashing Competition, memory-hard function

     \textbf{scrypt} - Memory-hard function designed to be resistant to hardware acceleration

These functions are preferable to general-purpose cryptographic hash functions for password storage.
\end{theorem}

\paragraph{Authentication Protocols}

\begin{concept}{Direct vs. Indirect Authentication}
    
    \textbf{Direct Authentication}:
    \begin{itemize}
        \item User authenticates directly to the service
        \item Service has all information needed to authenticate users
        \item Credentials are configured on each service
        \item Example: Local login on a server
    \end{itemize}

    \textbf{Indirect Authentication}:
    \begin{itemize}
        \item Authentication delegated to a central authority
        \item Service trusts authentication decisions of authority
        \item Credentials are managed centrally
        \item Examples: RADIUS, Kerberos, SAML, OpenID Connect
    \end{itemize}
\end{concept}

\begin{definition}{Multi-Factor Authentication (MFA)}

    Requires factors from different categories (know, have, are)

    Significantly increases security compared to single-factor authentication

    Mitigates risks of compromised passwords

    Widely recommended for sensitive systems and accounts

\end{definition}

\begin{concept}{Common MFA Methods}

     \textbf{One-Time Passwords (OTP)}:
    \begin{itemize}
        \item Time-based (TOTP) - Generated from a shared secret and current time
        \item HMAC-based (HOTP) - Generated from a shared secret and counter
        \item SMS-based - Codes sent via text message (less secure)
    \end{itemize}

    \textbf{Push Notifications} - Authent. req. sent to mobile apps

     \textbf{Hardware Security Keys} devices FIDO/U2F standards

     \textbf{Biometric Verification} Fingerprint, face, voice recognition

\end{concept}

\begin{theorem}{MFA Attack Vectors}
Despite its strength, MFA can still be vulnerable to certain attacks:
\begin{itemize}
    \item \textbf{Real-time phishing} - Intercepting and forwarding authentication in real-time
    \item \textbf{SIM swapping} - Taking control of a victim's phone number for SMS-based MFA
    \item \textbf{MFA fatigue} - Bombarding users with authentication requests until they approve
    \item \textbf{Malware on endpoint devices} - Capturing authentication data locally
    \item \textbf{Social engineering} - Tricking users into bypassing MFA protections
\end{itemize}
\end{theorem}

\begin{concept}{MFA Fatigue Countermeasures}

     \textbf{Number matching} - Displaying a number on login screen that must be entered in authenticator

     \textbf{Geographic context} - location info of login attempt

     \textbf{Device context} - device info of  login attempt

     \textbf{Request limiting} - Restricting nr of auth. requests

     \textbf{Suspicious activity detection} - Identifying unusual patterns

\end{concept}

\multend








\subsubsection{Kerberos}

\mult{2}

\begin{concept}{Kerberos Components}

    \textbf{Principals} - Users, services, or systems that participate 

    \textbf{Key Distribution Center (KDC)} - Central authentication server consisting of:
    \begin{itemize}
        \item Authentication Service (AS) - Verifies user identities
        \item Ticket-Granting Service (TGS) - Issues service tickets
    \end{itemize}

    \textbf{Realm} - Administrative domain of a KDC

    \textbf{Tickets} - Cryptographically protected credentials:
    \begin{itemize}
        \item Ticket-Granting Ticket (TGT) - Used to request service 
        \item Service Ticket - Used to access specific services
    \end{itemize}

\end{concept}





\begin{theorem}{Kerberos Security Properties}

    \textbf{No password transmission} - Passwords never sent over the network

    \textbf{Limited ticket lifetime} - Tickets expire after a defined period

    \textbf{Mutual authentication} - Both client and server can verify each other

    \textbf{Replay protection} - Timestamps prevent reuse of authentication messages

    \textbf{Key distribution} - Session keys established for secure communication

\end{theorem}

\begin{concept}{Cross-Realm Authentication}
Kerberos supports authentication across different administrative domains:
\begin{itemize}
    \item Enables federated identity across organizations
    \item Requires trust relationship between realms
    \item Uses a hierarchical trust model or direct cross-realm keys
    \item Allows users from one realm to access services in another
\end{itemize}
\end{concept}


\begin{KR}{Kerberos Authentication Process}
\paragraph{Initial Authentication}
\begin{itemize}
    \item User logs in and sends username to Authentication Service
    \item AS checks if user exists in database
    \item AS sends back data encrypted with user's key (derived from password)
    \item Client decrypts this using the user's password
    \item Client extracts Ticket-Granting Ticket (TGT) and session key
\end{itemize}

\paragraph{Service Ticket Acquisition}
\begin{itemize}
    \item When user needs to access a service, client contacts Ticket-Granting Service
    \item Client sends TGT and an authenticator encrypted with session key
    \item TGS verifies TGT and authenticator
    \item TGS issues service ticket and new session key for the client-service communication
\end{itemize}

\paragraph{Service Access}
\begin{itemize}
    \item Client presents service ticket to the service
    \item Client also sends a new authenticator encrypted with client-service session key
    \item Service decrypts ticket and authenticator
    \item Service verifies authenticator contents (timestamp within allowed skew)
    \item Service grants access if verification succeeds
\end{itemize}
\end{KR}

\multend

\subsubsection{Federated Authentication}

\mult{2}

\begin{definition}{Federated Authentication}
    enables identity verification across organization boundaries:
\begin{itemize}
    \item Allows users to authenticate with their home organization
    \item Enables access to resources in partner organizations
    \item Eliminates need for multiple accounts and credentials
    \item Maintains security boundaries between organizations
    \item Examples include academic federation (SWITCHaai) and enterprise federations
\end{itemize}
\end{definition}

\begin{concept}{Shibboleth}
    system for federated identity management:
\begin{itemize}
    \item Based on SAML
    \item Separates authentication from authorization
    \item Preserves privacy by limiting attribute sharing
    \item Widely used in academic and research communities
    \item Components include:
    \begin{itemize}
        \item Identity Provider (IdP) - Authenticates users
        \item Service Provider (SP) - Protects resources
        \item Discovery Service - Helps users find their home organization
    \end{itemize}
\end{itemize}
\end{concept}

\begin{KR}{Shibboleth Authentication Flow}
\textbf{Resource Access Request}
User attempts to access a protected resource, Service Provider determines resource requires authentication,
SP redirects user to Discovery Service 

\textbf{Identity Provider Selection and Authentication}
User selects their home organization from the Discovery Service, 
User is redirected to their Identity Provider,
User authenticates with their home organization credentials

\textbf{Assertion Generation and Validation}
IdP generates a SAML assertion containing authentication statement,
IdP may include additional user attributes (depending on configuration),
Assertion is cryptographically signed and optionally encrypted,
User's browser posts the assertion to the Service Provider

\textbf{Resource Access}
SP validates the assertion's signature,
SP extracts user identity and attributes from the assertion,
SP makes an authorization decision based on the attributes,
SP grants access to the requested resource if authorized
\end{KR}
\multend


\subsubsection{Passwordless Authentication}

\mult{2}

\begin{definition}{Passwordless Authentication}
\begin{itemize}
    \item Based on public key cryptography and digital certificates
    \item Typically combines possession and inherence factors
    \item Eliminates password-related vulnerabilities
    \item Improves user experience by removing password entry
\end{itemize}
\end{definition}

\begin{definition}{Authentication Protocols} messages and procedures for user authentication:
\begin{itemize}
    \item Standardize authentication across different systems
    \item Can be direct (authenticating directly to service) or indirect (using a third party)
    \item Often establish session keys for secure communication
    \item May support various authentication methods
\end{itemize}
\end{definition}


\begin{concept}{FIDO2/WebAuthn}
\begin{itemize}
    \item Combines WebAuthn (web authentication standard) and CTAP (client-to-authenticator protocol)
    \item Uses public key cryptography for authentication
    \item Supports various authenticator types:
    \begin{itemize}
        \item Platform authenticators (built into devices)
        \item Roaming authenticators (external security keys)
    \end{itemize}
    \item Protects against phishing through origin binding
    \item Private keys never leave the authenticator
\end{itemize}
\end{concept}


\multend




\subsubsection{Single Sign-On (SSO)}

\mult{2}


\begin{definition}{Single Sign-On (SSO)}
 enables users to authenticate once and access multiple services:
\begin{itemize}
    \item Reduces the number of authentication events
    \item Centralizes credential management
    \item Improves user experience
    \item Enhances security by reducing password fatigue
    \item Can implement consistent security policies across services
\end{itemize}
\end{definition}









\begin{definition}{OAuth 2.0}
     authorization framework (not authentication):

     Enables third-party applications to access resources without sharing credentials

      Uses tokens to grant limited access to resources

      Defines various authorization flows for different scenarios

      Standardized in RFC 6749

\end{definition}

\begin{concept}{OAuth 2.0 Roles}

    \textbf{Resource Owner} Entity that grants access (user)

    \textbf{Client} Application requesting access to resources

    \textbf{Resource Server} Server hosting the protected resources

    \textbf{Authorization Server} Server issuing access tokens

\end{concept}

\begin{definition}{OpenID Connect (OIDC)}
    built on top of OAuth 2.0:
\begin{itemize}
    \item Adds authentication functionality to OAuth 2.0
    \item Provides user profile information
    \item Uses JSON Web Tokens (JWT) for identity tokens
    \item Enables "social login" and federated identity
\end{itemize}
\end{definition}

\begin{concept}{OpenID Connect Terminology}

    \textbf{Relying Party (RP)} Application that wants to verify user identity

    \textbf{OpenID Provider (OP)} Service that authenticates users

    \textbf{ID Token} JWT containing user identity information

    \textbf{UserInfo Endpoint} API for retrieving additional user information

    \textbf{Claims} Pieces of information about the user

\end{concept}


\begin{concept}{SSO Implementation Approaches}

    \textbf{Kerberos} - Primarily for on-premises environments (e.g., Active Directory)

    \textbf{SAML} - Web-based SSO between different organizations

    \textbf{OpenID Connect} - Modern protocol for web and mobile applications

    \textbf{OAuth 2.0} - Authorization framework often used with OpenID Connect

\end{concept}


\begin{definition}{Security Assertion Markup Language (SAML)}
 XML-based framework for exchanging authentication and authorization data:
\begin{itemize}
    \item Primarily used for web-based SSO
    \item Enables cross-domain identity and access management
    \item Separates identity provider from service provider
    \item Current version is SAML 2.0
\end{itemize}
\end{definition}

\begin{concept}{SAML Components}
\begin{itemize}
    \item \textbf{Assertions} - XML documents containing authentication statements
    \item \textbf{Protocols} - Rules for requesting and receiving assertions
    \item \textbf{Bindings} - Methods for transporting assertions over different protocols
    \item \textbf{Profiles} - Combinations of assertions, protocols, and bindings for specific use cases
\end{itemize}
\end{concept}

\begin{concept}{SAML vs. OpenID Connect}

    \textbf{Format} - SAML uses XML, OIDC uses JSON

    \textbf{Platform support} - SAML for web only, OIDC for web, mobile, and APIs

    \textbf{Complexity} - SAML more complex, OIDC simpler to implement

    \textbf{Age} - SAML older and more established, OIDC newer

    \textbf{Token size} - SAML assertions larger than OIDC ID tokens

\end{concept}
\multend

% Duplicate content removed - comprehensive authentication coverage is provided above